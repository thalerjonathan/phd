\section*{2016 October 1st}
\subsection*{Aim of my PhD}
After reflecting on the whole thing a bit I've started to rewrite my research-proposal to cover the topic from the direction of Gintis models \cite{Gintis2006} and \cite{Botta20114025}.

\section*{2016 October 3rd}

\subsection*{A paper: "The dynamics of general equilibrium" by Herbert Gintis \cite{gintis_dynamics_2007}}
This paper is a more complicated and comprehensive model of \cite{Gintis2006} which can be considered to be a kind of complete economy with firms producing goods, agents being employed by the firms,... According to Gintis:

\begin{quote}
The article presents the first general, highly decentralized, agent-based model of the dynamics of general equilibrium.
\end{quote}

\begin{quote}
Moreover, for the first time, we know something substantive about the dynamic properties of the Walrasian system. \textbf{They are nothing like tatonnement.}
\end{quote}

Gintis also shows that when using public prices called out by a central authority leads to highly chaotic prices with extreme volatility. He claims that

\begin{quote}
Economic theory assumes public prices without justification. Public prices do not generally exist and equilibrium public prices cannot even be calculated in an economy of any appreciable size.
\end{quote}

This is a fundamental critique of the equilibrium theory of modern economics and inherently of many models built upon this notion of Walrasian tatonnement using public prices. \\
I have the feeling that I have now arrived at the right direction, the right paper, the right model. This is what I was looking for - and thank god, it has already been done by highly competent men (e.g. Gintis) and I can start following that road now.

\subsection*{An interesting article of Gintis on Traditional Economics and new approaches to it}
In his article, to be found at \url{http://evonomics.com/new-economics-with-tradtional-economics/}, Gintis argues for a new way of economics, where Behaviour \& Evolutionary Sciences combine with Traditional Economics to form a new understanding of economics which tries to understand the dynamics of economic processes e.g. equilibrium. \\

A remarkable quote is the following:

\begin{quote}
At least since the end of World War II economists have analyzed market failures in the terms described above. The recommended solution to market failure has always been the same: government intervention to replace markets or to regulate markets in the public interest. \textbf{But why would anyone seriously believe that the government would actually do what is in the best interest of efficiency and justice?} This belief, many years ago when I was a graduate student, was simply a precondition for admission to the inner sanctum of professional economics. I, on the other hand, found it about as reasonable as belief in reincarnation or flying saucers.
\end{quote}

