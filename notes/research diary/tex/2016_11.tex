\section*{2016 November 3rd}
The entries become less frequent with less volume each. I think this is because I am settling more and more in, calming down, sinking into research, not wanting to solve \textit{everything now}, but to work continuous and focus on specific things.

\subsection*{Wildfire with Akka}
I implemented the Wildfire with Wind-Simulation of AnyLogic about 1 week ago (just finished it before the supervision meeting) using Akka \& Scala. I was able to implement it in about 1 day in total. Scala is a nice language and the Akka-Actors are also very convenient to work with: it all feels like an updated and modern version of Erlang which I was experimenting with at the beginning of this year (and which taught a few basics of functional programming). This is definitely gonna be a path to follow on how ABM/S can be functionally done using Scala \& Akka. In the end it all boils down to 1. the agent-model and 2. how the agent-model is implemented in the according language. Of course both points influence each other: functional languages will come up with a different agent-model (e.g. hybrid like yampa) than object-oriented ones (e.g. actors).

\subsection*{Wildfire with Haskell \& Yampa}
Things are progressing slowly but steadily with Haskell \& Yampa where I try to implement the Wildfire simulation already done with Scala \& Akka. I decided NOT to use a GUI-framework but to use OpenGL with GLFW to do the rendering - it turned out to work very easy and convenient (using OpenGL 3.0, so can submit vertices and their properties directly without the need for VBOs and shaders as in OpenGL 4.0). \\
Yampa was but another difficulty as it is a completely new paradigm to learn: advanced functional programming, Arrows (and their syntax) and reactive programming which needs to think a system in a different way.

\section*{2016 November 4th}
Meeting with Alexander Possajenikov. In the meeting the following points came up

\begin{itemize}
\item Reproducability is a BIG issue and not easy - researches have to trust each other.
\item Economic models are often qualitative ones: specify the quality of the results instead of quantiative concrete results - trying to give general results.
\item Book: Computable Economics by Vellupilai
\item I should look at a few agent-based market-models instead of the whole ACE because there is no established consencus what ACE is and how it should be approached. I should maybe look at a reduces subset of models (markets) and look into them, Alex said that there are about 5-6 models.
\end{itemize}

We agreed to meet again in January after I have looked into a few market-models. 