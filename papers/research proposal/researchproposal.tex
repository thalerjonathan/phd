%%%%%%%%%%%%%%%%%%%%%%%%%%%%%%%%%%%%%%%%%
% University/School Laboratory Report
% LaTeX Template
% Version 3.1 (25/3/14)
%
% This template has been downloaded from:
% http://www.LaTeXTemplates.com
%
% Original author:
% Linux and Unix Users Group at Virginia Tech Wiki 
% (https://vtluug.org/wiki/Example_LaTeX_chem_lab_report)
%
% License:
% CC BY-NC-SA 3.0 (http://creativecommons.org/licenses/by-nc-sa/3.0/)
%
%%%%%%%%%%%%%%%%%%%%%%%%%%%%%%%%%%%%%%%%%

%----------------------------------------------------------------------------------------
%	PACKAGES AND DOCUMENT CONFIGURATIONS
%----------------------------------------------------------------------------------------

\documentclass{article}

\usepackage{graphicx} % Required for the inclusion of images
\usepackage{natbib} % Required to change bibliography style to APA
\usepackage{amsmath} % Required for some math elements 

\setlength\parindent{0pt} % Removes all indentation from paragraphs

%\usepackage{times} % Uncomment to use the Times New Roman font

%----------------------------------------------------------------------------------------
%	DOCUMENT INFORMATION
%----------------------------------------------------------------------------------------

\title{Research Proposal \\ A dependently typed functional framework for the research of equilibrium-dynamics in agent-based bilateral trading and bartering.} % Title

\author{Jonathan \textsc{Thaler}} % Author name

\date{\today} % Date for the report

\begin{document}

\maketitle % Insert the title, author and date

\begin{center}
\begin{tabular}{l r}
Institution: University of Nottingham \\
Department: School of Computer Science \\
Project: PhD Programme 2016-2019 \\
Supervisor: Dr. Peer-Olaf Siebers \\
Co-Supersvisor: Dr. Thorsten Altenkirch 
\end{tabular}
\end{center}

% If you wish to include an abstract, uncomment the lines below
\begin{abstract}
Agent-Based Modeling and Simulation (ABM/S) is still a young discipline and the dominant approach to it is object oriented computation. This thesis goes into the opposite direction and asks how ABM/S can be mapped to and implemented using pure functional computation and what one gains from doing so. To the best knowledge of the author, so far no proper treatment of ABM/S in pure functional computation exists but only a few papers which only scratch the surface. The author argues that approaching ABM/S from a pure functional direction  offers a wealth of new powerful tools and methods. The most obvious one is that when using pure functional computation (equational) resoning about the correctness and about total and partial correctness of the simulation becomes possible. The ultimate benefit is that Agda becomes applicable which is - according to the Curry-Howard isomorphism - both a pure functional programming language and a proof assistant for intuitionistic logic allowing both to compute the dynamics of the simulation and to look at meta-level properties of the simulation - termination, convergence, equilibria, domain specific properties - by constructing proofs utilizing computer aided verification. \\
The application will be in the field of agent-based computational economics where the main focus will be on simulating the dynamics of equilibrium in bilateral trading. This is a quite new and ongoing research in economics and very strongly driven by simulative approaches and still a very open question among economists how prices reach equilibrium over time.

\end{abstract}

%----------------------------------------------------------------------------------------
%	SECTION 1
%----------------------------------------------------------------------------------------

\section{Ideas, Influences \& Applications}
In the paper \cite{Gintis2006} Herbert Gintis proposed a model of decentralized barter between agents in which he could show that the prices approach an equilibrium. His results made quite a fuzz in the field of economics because so far how prices in bilateral trading approach the walrasian equilibrium was (and is still) an open question as Walrasian Equilibrium definition just states the static properties when equilibrium is reached but says nothing about how it is reached. \\
Cezar Ionescu et al. tried to reimplement the findings of Gintis but were unable to correctly reproduce the results due to a misunderstanding and misinterpretation of the model in \cite{Gintis2006} which was given in natural language with very few formalisms. \\
Ionescu et al. went then on to build upon these findings, that specifications of such models should be much more formal and proposed a functional framework in their paper \cite{Botta20114025}. This framework allows to specify a model of agent-based exchange like the one of Gintis in \cite{Gintis2006}, where Ionescu et al. gave in an companion-paper a case study by "implementing" Gintis model in their functional framework. The according paper is "N. Botta, A. Mandel, M. Hofmann, S. Schupp, Mathematical specifications for agent-based models of exchange: a case study" which can be found on the internet but was ultimately rejected in the review process and is thus not published. \\
Ionescu et al. state in \cite{Botta20114025} that a real implementation of their functional framework would need a dependently typed programming language and is by far out of the scope of this paper, leaving it for further research \\
Also in this framework time is not modelled explicitly which is a further extension point.

\bigskip

The paper \cite{gintis_dynamics_2007} by Gintis goes one step further than \cite{Gintis2006} and presents a more complicated and comprehensive model which can be considered to be a kind of complete economy with firms producing goods, agents being employed by the firms. This is too complicated to be implemented but is considered for further research.

\bigskip

TODO: evolutionary model of agents: imitation, copying the behaviour of successful ones / mutating prices. maybe could go further into evolutionary strategies

\bigskip

TODO: BUT not only theoretical approach. Also include empirical and practical knowledge of market microstructure: how markets behave and how agents behave in these markets. The following books will be of importance: \cite{LehalleLaruelle2013}

\cite{baker_market_2013} Part II: Chapter 8-12

\cite{aldridge_high_frequency_2009}

\bigskip

TODO: Include HFT

In the paper \cite{Budish2015}, Budish et al. proposed a new kind of auction on the stock market, which they termed "Frequent Batch Auctions" (FBA) as a remedy against the High-Frequency arms race. 

%----------------------------------------------------------------------------------------
%	SECTION 2
%----------------------------------------------------------------------------------------

\section{Goals and Outcome}
The goal is to develop an implementation of a functional framework for specifying and simulating bilateral exchange \& barter between modelled agents. It should be possible to translate a concrete specification directly without (important) loss of expressiveness to the simulation framework and be executed as a simulation-process to reveal its dynamics over time when actual computation is carried out over the formal language.

\bigskip

An important emphasis will be put on the verification of correctness, whether the implementation matches the formal specification or not, of both the simulation framework and translated auction-type programs.
Research will also go into proofs of termination / convergence-possibility: proving whether a specified auction can ever terminate / converge, ignoring the time-space constraint and the quality of the convergence. Another focus of research will focus on formal testing of an auction specification / implementation for deadlock-free as such auction-processes can get stuck when specifications exhibit configurations which can result in deadlock-situations between traders which are still willing to trade but cannot because of circular dependencies.
Of special interest is also the visualization and the support to investigate the dynamics of such simulation-processes. Part of the research will be how to describe and visualize dynamics of such an auction-process in a meaningful and comprehensive way so to detect patterns in the dynamics and to provide a new way at looking at such dynamics. With the use of a pure functional approach it may be possible to develop a new way of looking at such dynamics which can then also be visualized in a new way.

\bigskip

Note: it is not in the interest of the researcher to proof the match of the outcome of the various auction types with established equilibrium-theorems (which conditions must be satisfied that the dynamic process reaches the static equilibrium?) because this would touch too deep on economics and complex equilibrium theory which is not the focus of this PhD. The focus is much more on the research of a specification tool and then on the dynamics of such an auction- process (e.g. market \& agent activities over time) in the context of the field of Agent-Based Modelling/Simulation.

%----------------------------------------------------------------------------------------
%	SECTION 3
%----------------------------------------------------------------------------------------

\section{Impact}
The resulting simulation framework allows computational economists to specify problems in this field in a precise and concise formal language thus they can refrain from having to share and explain all painstaking details in natural language. The simulation framework allows them to run formal specifications of trading-processes and investigate the dynamics of them. Also they will be able to proof if their specification is able to converge / terminate or not and that it is deadlock-free or not.


%----------------------------------------------------------------------------------------
%	SECTION 4
%----------------------------------------------------------------------------------------

\section{Approach}

\begin{enumerate}
\item Designing and implementing the software for the functional framework of the paper \cite{Botta20114025} in Haskell \& Agda
\item Apply it to the model of Gintis as presented in \cite{Gintis2006} and specified in the paper "Mathematical specifications for agent-based models of exchange: a case study" by N. Botta, A. Mandel, M. Hofmann, S. Schupp.
\item Extend framework to the whole model as presented in \cite{gintis_dynamics_2007}
\item Extend the framework to model time explicitly to allow the simulation of continuous time.
\item Implement FBA as proposed by Budish et al. in the paper \cite{Budish2015} and look at the results.
\end{enumerate}

\subsection{1st year}

\begin{itemize}
\item Understand Equilibrium Models
\item Understand Market Micro-structure
\item Do literature research on dynamics of equilibrium 
\item Learn dependent-type programming in Agda
\item Learn monadic programming in Haskell
\item Fully understand the framework in \cite{Botta20114025}
\item Design software of the framework: lay out modules, define interfaces, define functions, specify agents
\item Implement the framework
\item Publish first paper about the implementation of the framework
\end{itemize}

\subsection{2nd year}
Explicitly model continuous time
Implement continuous batch auction
Publish paper with results

\subsection{3rd year}
?

%----------------------------------------------------------------------------------------
%	SECTION 5
%----------------------------------------------------------------------------------------

\section{Research Questions}
\begin{itemize}
\item How can the functional framework of Ionescu et al. be implemented in a dependently typed programming language?
	\begin{itemize}
	\item How can agents be represented?
	\item How can the framework be implemented to be extensible?
	\item How can the framework be implemented to be open to other models of exchange?
	\end{itemize}
\item How can continuous time be explicitly modelled in the functional framework and be incorporated in the implementation?
\item In which way does the use of a frequent batch auction instead of a continuous double-auction influence the dynamics of equilibrium?
\end{itemize}

%----------------------------------------------------------------------------------------
%	BIBLIOGRAPHY
%----------------------------------------------------------------------------------------

\bibliographystyle{apalike}

\bibliography{./bib/researchproposal.bib}

%----------------------------------------------------------------------------------------

\end{document}