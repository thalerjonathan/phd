\section*{2016 September 2nd}
\subsection*{Black Swan}
I've been reading the book "Black Swan" (TODO: cite) since about one week and I'm thrilled by it. \\ 
He does a fundamental critic on economics as a "science" which he claims it is not because it can't predict due to the extreme complexity and uncertainty (and randomness) of the world. The question is then whether computational economics and simulation are also a dead end or not? \\
On position 4289 he writes that "If you hear a "prominent" economist using the word equilibrium, or normal distribution, do not argue with him; just ignore him, or try to put a rat down his shirt.". I thought right away from the beginning that the author would make fun of equilibria (and normal distribution) because the world is just not in an equilibrium and not normal distributed and will never be. What does this imply to my PhD studies? It should encourage me to go away from equilibrium research towards simulation of dynamics and to look force a different approach to computational economics. Fact is: I don't want to do something unrelated to reality and purely academic - but is it really within my reach to make an impact? It is not (one never can plan to make an impact on the world, as shown in the Black Swan Book) but I can select my focus accordingly to increase the possibility of an impact. Thus I feel that I should focus on alternative approaches to computational economics which abandon equilibrium theories altogether and look only on the dynamics instead of equilibria - a thing which I already mentioned in the entry of the 24th of August 2016.

\bigskip

The author also attacks linear regression and r square: I also never understood what one gains from it as it is only a linear fit and can blowup terribly when compared to the real-world. I remember an economic workshop in which I participated where all presenters gave statistics how well their model fits linear regression. I felt it was just dead numbers, giving you absolution and a stamp on your model which says: \textit{accepted by the (conservative) economics community} - but it all seemed so far away from reality.

\bigskip

The author also praises a few authors - I noted 4 from which I should definitely read a bit of.

\begin{itemize}
\item Karl Popper
\item Henri Poincaré
\item J.M. Keynes
\item Friedrich Hayek
\end{itemize}

Hayek is a major proponent of the Austrian School of Economics, which I should definitely look at because it seems to be in opposition to the neo-classical (cambridge school) which emphasises mathematical models of equilibrium where the Austrian School focuses on TODO?

\bigskip

Also I am aware that I know very very little of economics to be in the position to be a competent critic of it. 

\section*{2016 September 5th}
On the ferry from Bergen to Hirtshals on my way home. \\

\subsection*{Black Swan finished}
I've finished the Black-Swan book 2 days ago. The quintessence of it is basically an attack on the established economic theories and the people behind it. It all boils down that to a massive critique of the Gaussian Normal Distribution. Although it has its applications in the distribution of \"normal\" quantities like height, weight, age of persons it fails completely in case of predicting probabilities of unbounded quantities like they are common in economics where those parameters can totally explode. Also the author says that history is totally unpredictable due to it being an extremely complex process thus also implies that prediction for economics is impossible.  What is interesting is that the author does never go explicitly in discussing equilibria and their theories but only mentions them twice from where it becomes imminent that he thinks that equilibria are bullshit as well as gaussian distribution - this is what I've already expected at the beginning of the book. \\
This book has thus quite an implication for my Ph.D. as I have the feeling this author is right I cannot ignore what he says and continue as if nothing happened. Thus I have to question the use of the gaussian distribution and more specific the use and the pursue of equilibria in my applications. But still I do not know enough - but I just stumbled across a book of Mandelbrot (TODO: cite) in which he goes into analyzing markets and price-formation from a fractal point of view (which he invented). \bigskip
But what I think the author of the Black Swan misses or never mentioned is that we all want too much: newer cars, newer phones, newer computers. We want it every year and we want it cheap. In fact the problem is greed, we must return to a more simplistic life-style and not live above what is ours.

\subsection*{Mandelbrot book started}
This book is not about predicting prices or future economic trends but to explain and better understand the overall structure of price formation and volatility. Also a critique of the gaussian distribution and a huge proponent of power laws (see my Masterthesis). \bigskip
On position 160 (kindle) Mandelbrot claims that biologists do research on the healthy body, physicists collide particles, meteorologists look into hurricanes but economists seem to be curious incurious. \\
Thus my implication: economists should rely much much more on simulation and simulate their models using computer power (Mandelbrot says in the introduction that \"financial economics as a discipline is where chemistry was in the 16th century", which pretty says all. Also the author of black swan says that there rarely happens a feedback from the real-world to the models). This could be a more specific direction of my research (as already mentioned in other diary notes): looking into the dynamics of markets but with non-gaussian, fractal models. The result will not be a predictive price model but a system in which dynamics which resemble those on markets can be analysed and better understood to be better prepared in case of crashes. Mandelbrot himself sums it best up: \"basic research into the dynamics of pricing and volatility in a global marketplace gets short shrift." - this is what I want to do in my simulations.

\subsection*{A 1st semester Roadmap}
I still need to read and learn much more about economics, but how much exactly and which kind of direction? I think it is enough to cover the basics and then go deep into the subjects of equilibrium theory, market micro-structure, the fractal models of markets and agent-based computational economics. Thus the very basic approach in the 1st semester is doing literature research and do prototyping of the new formal agent-based simulation methods. During literature research the goal is to find a paper to start with, which allows to apply the methods developed and then to do further research based upon that paper in which I can incorporate my ideas of dynamics instead of equilibria and fractal/creative randomness/complexity from simplicity instead of predictability, Gaussian distribution and complexity from complex models.

\begin{itemize}
\item Literature Research in Economics
	\begin{itemize}
	\item Basic overview by reading the "Understanding Capitalism" Book (TODO: cite)
	\item Basic understanding in equilibrium theory by studying specific chapters of microeconomics book (TODO: cite)
	\item Basic understanding of market micro-structure by reading specific chapters of the Market Micro-structure book (TODO: cite)
	\item Find and read papers in the fields of 
		\begin{itemize}
		\item Fractal Market Models inspired by Mandelbrot
		\item agent-based computational economics
		\item Combinations of both.
		\end{itemize}		
	\end{itemize}
\item Method development and experiments in Computer Science
	\begin{itemize}
	\item Get better understanding of Agda
	\item Get deeper into Haskell and monadic programming
	\item Get basic understanding of Category Theory
	\item Get introduction into Computer Aided Formal Verification
	\end{itemize}
\end{itemize}

\subsection*{TODOs Update}
Note that these TODOs should be done till the 26th of September because I expect to meet my supervisors then - maybe there is time until the 1st of October but better earlier than too late (Research Proposal!) \\ 
I think I will abandon the whole Ethereum stuff as I will need all the time for the preparations and the Ph.D. itself - and because I also want a little bit of free-time without comptuers.

\begin{itemize}
\item Read Mandelbrot Book (TODO: cite)
\item Read Gode \& Sunders paper again (TODO: cite).
\item Read Everything what you wanted to know about continuous double-auction paper again (TODO: cite).
\item Read my Masterthesis.
\item Read the following chapters of "Handbook of Market-Design" (TODO: cite) in the given order: 2,3,12,10,13,16,17
\item Rework my Reseach-Proposal and reformulate it to cover the above idea. 
This will then be the basics of the initial meeting with my supervisors. Also include the thoughts of the entry to the diary of 2016 August 24th. And include essence of entry to diary of 2016 August 30th.
\item Get basic understanding in Agda 
	\begin{itemize}
	\item Work through "Dependent Types at Work"
	\item Look at Thorsten Altenkirchs Lecture "Computer Aided Formal Verification
	\item Read online Lectures about Dependent Types: TODO
	\end{itemize}
\item Haskell Learning: Implement the core of my Masterthesis in Haskell as I know it at the moment and trying to incorporate Monads: the core is the replicated auction-mechanism where every Agent knows all others.
\end{itemize}

\section*{2016 September 6th}
Currently on the Wewelsburg on a stop on my way home. Of course I can't stop thinking about a potential application of my formal methods in agent-based computational economics. \bigskip

My hypothesis is that a finite number of interacting agents following very simple rules in placing bids and asks lead to price dynamics that resemble power law distributions invented by Mandelbrot - price formation is unpredictable but the dynamics follow a rule and emerge out of the interaction of the agents which form a market. I would need to build a model of the agents and interactions among them and then implement the simulation. I should also look at continuous double auction and continuous batch auction and compare their results. A big question is what the role of equilibrium (theory) in this approach is.

\bigskip

I want to abandon the idea of a simulation which will hunt for equilibrium as I became convinced by the books of black swan and Mandelbrot that this is infeasible arcane magic, something I already thought right away from the beginning of my masterthesis and which I also discussed with Prof. Vollbrecht. I MUST NOT blindly follow a scientific road in which I don't believe and I don't believe in equilibrium in economics as the world is totally obviously NOT going towards equilibrium. \\
Of course the process of reaching it must not be confused with equilibrium itself and no process has been given so far thus meaning we could be on the way to reaching it and will at an unknown time in the future because we don't know the process - but that lies no in the field of belief and I don't believe thus abandoning the search for it. Also the conditions of the system in which equilibrium may unfold are always a changing: traders enter and leave at various points thus shuffling all new as one knows regarding system dynamic. \medskip

But i have to be honest: I find the concept of equilibrium beautiful and it is still fascinating to look at the dynamics of how equilibria form in a process: how long does it take, how efficient is it, is it even reachable, ... maybe i can nonetheless incorporate it in SOME way in the new approach: see if and what Mandelbrot says about them and incorporate that then.

\bigskip

So the overall direction is clear and will be one of two ways (or a combination of both, as both topics are extremely interesting):

\begin{enumerate}
\item Research equilibrium processes: look at how and under which conditions equilibrium is established, what properties the agents have, how long it takes, if it is feasible,...
\item Research price dynamics and trading dynamcis: ignore equilibrium and only look at the dynamics of trading processes where tools developed of Mandelbrot should be used instead of neo-classical economics (gaussian distributions, equilibrium theory)
\end{enumerate}

\bigskip

I've written an email to Martin Summer asking him a few questions about the Black Swand and Mandelbrot Books regarding Gaussian distribution and equilibrium theory - I am curious what he will answer. I will give a summary of the questions and his answers when he answered.

\bigskip

I just found "The future of agent-based modelling", Santa Fe Bulletin Winter 2006. It seems that what I wanted to do already exists in the form of a software package called Santa Fe Artificial Stock Market SF-ASM. According to this Bulletin this software package challenged the idea that financial markets are in equilibrium. The paper also says that the SF-ASM is currently not under development - maybe I can take a look at it. Also this paper talks about the zero-intelligence traders invented by Gode \& Sunder and how useful they are, as already shown in the classical paper by them - maybe a hint to go start again with this paper? 

\bigskip

I also did a bit of paper research and found a few interesting papers I should read before reworking my research-proposal. 
\begin{itemize}
\item Agent-based Computational Economics. A Short Introduction. MATTEO G. RICHIARDI (TODO: cite)
\item Agent-based modeling and General Equilibrium, Lastis symposium, ETHZ, September 11 2012, Antoine Mandel (its a presentation) 
\item MANDELBROT AND THE STABLE PARETIAN HYPOTHESIS, Eugene Fama (TODO: cite)
\item THE VARIATION OF CERTAIN SPECULATIVE PRICES, BENOIT MANDELBROT (TODO: cite)
\item Out-of-Equilibrium Economics and Agent-Based Modeling, Brian Arthur (TODO: cite)
\item The emergence of a price system from decentralized bilateral exchange, Herbert Gintis (TODO: cite)
\end{itemize}

The paper "The emergence of a price system from decentralized bilateral exchange" (TODO: cite) was mentioned in the paper about formal verification of economics (TODO: cite) as a paper which made lot of fuss but then an error was found - a perfect example for Computer Aided Formal Verification, maybe this is a nice paper to start from?


\subsection*{Starting Papers}
I introduce hereby a new section which lists all papers which could act as a starting point for the application of the formal methods and agent-based simulation. So far we have the following papers:

\begin{enumerate}
\item Allocative efficiency of Markets with Zero-Intelligence Traders, Gode  \& sunder (TODO: cite)
\item The emergence of a price system from decentralized bilateral exchange, Herbert Gintis (TODO: cite)
\end{enumerate}


\subsection*{Updated TODOs}
Added SF-ASM entry and for reworking my research-proposal i should read through all research-diary entries and incorporate the essence of them. Added entry to read the enumerated papers.

\begin{itemize}
\item Read Mandelbrot Book (TODO: cite).
\item Read the papers enumerated in entry of 2016 September 6th.
\item Look into SF-ASM.
\item Read Gode \& Sunders paper again (TODO: cite).
\item Read Everything what you wanted to know about continuous double-auction paper again (TODO: cite).
\item Read my Masterthesis again.
\item Read the following chapters of "Handbook of Market-Design" (TODO: cite) in the given order: 2,3,12,10,13,16,17
\item Rework my Reseach-Proposal and reformulate it to cover the above idea. 
This will then be the basics of the initial meeting with my supervisors. Also include the thoughts of the entry to the diary of 2016 August 24th. And include essence of entry to diary of 2016 August 30th. Read through ALL the research-diary entries and incorporate the essence of them.
\item Get basic understanding in Agda 
	\begin{itemize}
	\item Work through "Dependent Types at Work"
	\item Look at Thorsten Altenkirchs Lecture "Computer Aided Formal Verification
	\item Read online Lectures about Dependent Types: TODO
	\end{itemize}
\item Haskell Learning: Implement the core of my Masterthesis in Haskell as I know it at the moment and trying to incorporate Monads: the core is the replicated auction-mechanism where every Agent knows all others.
\end{itemize}

\section*{2016 September 7th}

\subsection*{SF-ASM and equilibrium}
I couldn't stop thinking about that the SF-ASM challenged the idea that financial markets are in equilibrium. Why did research stop there? Is equilibrium a too holy grail to put it down even if it is misguiding? Is there new research going on in this direction? Maybe I can start from this point?

\subsection*{An idea for a model \& simulation}
After thinking about the whole thing while driving home from Germany I came up with the simple idea that the real problem may be the one that every trader just wants to make a profit instead of really owning a share of a company. Also I asked myself electronic trading has changed how the markets behave, whether they exhibit now different dynamics in volatility as shown by Mandelbrot. When boiled down to the very basics of how profits are made then there should be no difference: one buys at low and sells at high. Thus the idea is the following:

\begin{quote}
A considerable amount of the traders don't do long term trading, they are not interested in owning shares for a long period (say months, years) but are only interested in the quick profits when selling at a higher price than bought and also all trading algorithms (should) work the same. My hypothesis is that this quick trading creates those unpredictable dynamics and fat tail variance of the prices (does this also imply that markets are never in equilibrium?). Also because trading algorithms work basically the same way the same dynamics should be dominant also despite the massive use of electronic and automated trading. \\
To validate the hypothesis the goal is to develop a suitable model and simulate it. The approach which comes to my mind is to model zero intelligence agents in continuous double auctions with time explicitly modelled. The agents are divide into long- and short-term traders where short-term traders jump on price-changes very quickly where long-term traders don't. Of interest are then the price-dynamics and equilibrium under different parametrisation: different distribution of short- and long-term traders (also changing over time), "quickness" of short-term traders,...
\end{quote}

The approach would then be to 1. develop model, 2. transform the model to a formal representation (pi calcumus/category theory/actor model), 3. implement the simulation in Haskell \& Agda and 4. do computer aided formal verification in Agda.

\section*{2016 September 18th}
Currently in Glastonbury (lots of strange people). Here a sum up of the last days in which I was too busy to write down anything. \\

Martin has replied in email. He didn't read the Black Swan book but did note it as it caused some stir in the economics field but he does not like the style of it as he says it fishes for sensation - I partly agree but Taleb seems to bring in real evidence that economics models are just wrong (something Mandelbrot confirms in his book). Martin did send me a few interesting papers as I asked him for ideas:

\begin{itemize}
\item Hot and Cold Seasons in the Housing Market - L. Rachel Ngai Silvana Tenreyro
\item GETTING AT SYSTEMIC RISK VIA AN AGENT-BASED MODEL OF THE HOUSING MARKET - John Geanakoplos et al
\item LEVERAGE CAUSES FAT TAILS AND CLUSTERED VOLATILITY - Stefan Thurner, J. Doyne Farmer and John Geanakopolos
\item Contagion in Financial Networks - Paul Glasserman and H. Peyton Young*
\item When Bitcoin Grows Up - John Lanchester
\item HIGHER-DIMENSIONAL MODELS OF NETWORKS - DAVID I. SPIVAK
\item Housing and Macroeconomics - Monika Piazzesi, Martin Schneider
\end{itemize}

I will read through these papers as soon as possible and I think if not the Mandelbrot Book Idea (see below) will do, then something out of those papers will be interesting enough.

\bigskip

My former Professor Mr. Vollbrecht did provide me with an interesting hint, that I should look at the Journal of Artificial Societies and Social Simulation as they seem to have an "alternative" way of looking into things and have a very broad interest. It is to be found at \url{http://jasss.soc.surrey.ac.uk/JASSS.html}

\bigskip

I continued reading the Mandelbrot Book and got a few good quotations.

\begin{quote}
The fastest way to simplify things is to spot the symmetries, or invariances - the fundamental properties that do not change from one object under study to another. (Position 1957)
\end{quote}

\begin{quote}
A lesson arises from this: never hurry and never publish any result based on a single tool. (Position 2830)
\end{quote}

\begin{quote}
Many a grand theory has died under the onslaught of real data. (Position 2840)
\end{quote}

He also makes it very clear that economics is making wrong assumptions, which are:

\begin{enumerate}
\item Homo economicus is rational and self-interested - wrong, see the bubbles and bursts of the 90s.
\item Price variations follow the bell curve - wrong, see widely accepted by Mandelbrot and many others since the 1960s
\item Price variations are i.i.d. - evidence for short-term dependence has already been mounting and also evidence of long-term dependence.
\end{enumerate}

Because the book is from 2004 I have a few questions regarding the up-to-date state of it:

\begin{itemize}
\item Is there research going on at the moment investigating fractal price changes today (opposed to changes mandelbrot looked at in the 60s)?
\item How are fractal prices showing up today in stock markets and do they differ? is there a change detectable due to electronic trading? if no then why? what has stayed the same (buy low and sell high?)?
\item Are the price changes also present on a millisecond scale?
\end{itemize}

\bigskip

Mandelbrot talks in his book about a model which he developed an which was tested by his students Laurent Calvet and Adlai Fisher in their doctoral thesis and proofed to be true. Thus my dream would an agent based simulation which is able to mimic the price dynamics of the model as an endogenous property of agent behaviour. The key would be to use use multifractal time to create busy and smooth market activity (see mandelbrot diagram starting from pos 3099).
\\ The question is: is it not too complicated? Is it possible? Hasn't it already been done? \\
After a quick googling I found out that the most recent model Calvet and Fisher have developed is called Markov switching multifractal \url{https://en.wikipedia.org/wiki/Markov_switching_multifractal} 

\section*{2016 September 19th}

Finally finished with the Mandelbrot Book and got a few more quotes

\begin{quote}
Economics has no intrinsic time scales. in fractal analysis time is flexible - expanding and contracting.
\end{quote}

Maybe this is interesting for modelling agent actions: many in dense times, few in stretched times.

\begin{quote}
When examining price charts we should guard against jumping to conclusions that the invisible hand of Adam smith is somehow guiding them.
\end{quote}

\begin{quote}
Forecasting process may be perilous but you can estimate the odds of future volatility
\end{quote}

\begin{quote}
If there is one message I would wish to survive this book, it is this: finance must abandon its bad habits and adopt a scientific method (pos 3648)
\end{quote}

Also Mandelbrot last chapter talks about a trader who looks at a heterogeneous market consisting of traders with different time scales coming together at various points in transactions which creates the multi-fractal behaviour of the market. This was also something i had in mind but of course not so articulated.

\bigskip

So the book was VERY interesting, and very fruitful and inspiring and giving a good hint at some workings in economics, finance and trading. Mandelbrot himself said that we are still very very far from REALLY understanding how markets work and that lot of research needs to go into that and that finance needs to change. \\
So what I take out of this book is the critic of the orthodox models of economics and finance and the inspiration that markets follow a multifractal nature. As already said in the last entry it would be interesting to have an agent-based simulation which generates prices which follow such a multi-fractal volatiliy - the question is how, if it is possible/or not too complicated (which I fear it is), if it hasn't been already done and what model to follow (e.g. Markov switching multifractal).

\bigskip

Thomas Schwarz (a former Student- and Working-colleague) told me about Zotero, which is an open-source reference management software to manage bibilographic data and related search materials (wiki). I definitely should manage all my papers in this one for a better search and find and management.

\subsection*{Updated TODOs}
Finished Mandelbrot, combined all paper-entries into one, added Zotero entry.

\begin{itemize}
\item Read interesting papers in my folder - need to enumerate them in a following entry.
\item Put ALL printed papers of my folders into Zotero. 
\item Read my Masterthesis again.
\item Read the following chapters of "Handbook of Market-Design" (TODO: cite) in the given order: 2,3,12,10,13,16,17
\item Rework my Reseach-Proposal and reformulate it to cover the above idea. 
This will then be the basics of the initial meeting with my supervisors. Read through ALL the research-diary entries and incorporate the essence of them and make a list of the research-ideas found so far in my diary.
\item Get basic understanding in Agda 
	\begin{itemize}
	\item Work through "Dependent Types at Work"
	\item Look at Thorsten Altenkirchs Lecture "Computer Aided Formal Verification
	\item Read online Lectures about Dependent Types: TODO
	\end{itemize}
\item Haskell Learning: Implement the core of my Masterthesis in Haskell as I know it at the moment and trying to incorporate Monads: the core is the replicated auction-mechanism where every Agent knows all others.
\end{itemize}

\section*{2016 September 21st}

\subsection*{A paper: Contagion in Financial Networks (TODO: Cite)}
It is more or less an overview of current research and models in this field and reviews it and emphasises weaknesses of the models and proposes extensions. It gives also an overview of Open Problems where the two I would be interested in are 1. How the links between institution are formed (dynamic network formation) and 2. how to deal with network opacity:

\begin{quote}
The conventional view is that institutions establish links with one another as a way of diversifying risk and facilitating intermediation. While this is certainly true, we would argue that a realistic model of network formation must include other factors. In particular, one must acknowledge that links between financial institutions are often created in a decentralized fashion within particular lines of business, such as commercial lending, foreign exchange, derivatives trading, repo desks,and the like. Building realistic models of the resulting dynamics will require a high degree of institutional knowledge and a clear under- standing of the incentives faced by the individuals who are forming (and severing) these links. We believe that this is one of the most important challenges for future research in this area.
\end{quote}

\begin{quote}
Much of the literature on financial network models presupposes complete information about the network. We have argued that network opacity is a first-order concern for agents within the network, for regulators monitoring the network, and therefore for researchers developing models. More work is needed on inference from partial observations of network data and on understanding how opacity itself may contribute to contagion.
\end{quote}

The topic of contagion in financial networks is a very suitable one because

\begin{itemize}
\item I am already familiar with advanced network concepts through my Master-thesis.
\item It should be very well suited for ABM/S due to nodes and connections resemble agents and their messages/communication channels.
\item It is a real world problem.
\item It is probably very well applicable to Pi-Calculus and category theory
\item Surely I can get very good support by martin summer throughout the Ph.D. as he also has written a paper on this topic.
\item It feels much better than the Mandelbrot topic
\end{itemize}

\subsection*{A paper: The emergence of a price system from decentralized bilateral exchange}

I've been curious about this paper since a few weeks but haven't had time to read it yet. I came across it when reading the paper "Dependently typed programming in scientific computing". The authors of that paper mentioned that people trying to reproduce Gintis paper - which made a lot of fuss due to its claims of having found a mechanism leading to equilibrium without central authorities - found that the simulation had some fundamental programming error, leading to the (wrong) results (which could be prevented by using computer aided verification, but then the results would have been different and probably not being worth the publication). So I was curious about the initial idea of this Gintis guy and wanted to read the original paper. \\

After googling for Herbert Gintis (I didn't know anything about how that guy was) I found a very interesting customer review by him on amazon.com for the book "Emergent Macroeconomics: An Agent-Based Approach to Business Fluctuations" (TODO: cite). Here is what he says about it

\begin{quote}
The theory of macroeconomic fluctuations has been a pathetic mess for a long time. Indeed, forever. The central model of the economy, the Walrasian general equilibrium model, is a purely equilibrium model, and no one has been able to derive an out-of-equilibrium mechanism of price and quantity adjustment that renders market equilibria dynamically stable, despite more than a half century of trying (see my paper, "The Dynamics of General Equilibrium", Economic Journal 117 (2007):1289-1309 for details and a proposed solution), and despite blistering attacks from within the economic establishment. See, for instance, Alan P. Kirman, "Whom or What does the Representative Individual Represent?" Journal of Economic Perspectives 6 (1992):117-136, and Franklin M. Fisher, Disequilibrium Foundations of Equilibrium Economics (Cambridge, UK: Cambridge University Press, 1983). In a recent paper (Herbert Gintis, "The Dynamics of General Equilibrium", Economic Journal 117 (2007):1289-1309) I showed using agent-based modeling techniques that the problem with the Walrasian model lies in the assumption that agents never interact, but rather that each makes decisions independently from a given system of prices. I also showed that the assumption that there exists a price structure accepted by all agents in the economy ("public prices") accounts for the chaotic nature of disequilibrium in the standard Walrasian disequilibrium models. Instead, when each agent has his own set of "private prices," even though market competition leads to a very low standard deviation of private prices (I call this situation "quasi-public prices"), such prices tend strongly to equilibrate the market system in the long run. Finally, I showed that plausible learning processes in the economy lead the system sporadically to make large excursions from equilibrium even in the absence of any global stochastic shocks (I called these "local resonances").

Because of the lack of dynamics in the standard Walrasian model, macroeconomic theories that depend on this model must perform massive simplifications in order to investigate out-of-equilibrium behaviour. The reason these models are such a mess is that they take it for granted that public prices exist (they do not) and that we can analyse the market economy as if individuals never interact, but rather interaction only with private prices. This, of course, is completely incorrect, as I explain above. However, with this assumption, it is clearly permissible simply to aggregate all economic actors of the same type into a "representative agent" having the average characteristics of that type of agent. From this is born the Keynesian consumption, investment, and government sectors, from which the standard Keynesian models flow. For the rational expectations macro models, we have a similar aggregation, with the completely crazy assumption that an aggregate "representative agent" will satisfy the condition of "rational expectations" theory, as though the aggregation of "rational agents" is prima facie an aggregate rational agent. The intellectual value of these assumptions is rather meager.

This fine book, which was in preparation at the time of appearance of my Economic Journal paper, is quite in agreement with my findings, laying blame on the "representative agent" assumption, and using agent-based modeling (abm) to investigate macroeconomic dynamics. However, whereas I took individuals as the unit of analysis, the authors allow firms to fill this role. They use empirical data on within-industry firm heterogeneity to model the population of firms, and assume asymmetric information among firms. This leads them to a financial accelerator model of financial fragility with great similarity to a model proposed by Greenwald and Stiglitz in 1993 (Bruce Greenwald and Joseph E. Stiglitz, "Financial Market Imperfections and Business Cycles", Quarterly Journal of Economics (1993):77-114). Finance is central in their model because the absence of forward markets forces firms to rely on credit to finance investment that matures only across time periods.

Based on careful industry research, the authors' abm is populated with firms whose size distribution take the form of a power law density (Zipf's Law), and firm growth rates follow a Laplace (double exponential) rather than a normal distribution. Such a distribution has `fat tails' that imply more instability than in a system with normally distributed growth densities. Indeed, they show that normally distributed shocks give rise to power law distributions and a Pareto shaped firm size distribution. This is a quite nice finding, and surprising given the degree of aggregation of their agent-based economy (they assume only two sectors, firms and banks, and no individual agents). Clearly individual interactions underlie the power law assumptions concerning firm size and the Laplace distribution of growth rates.
\end{quote}

So he is a quite critic of equilibrium theory and classical economics - this sounds intriguing. \\

\section*{2016 September 22nd}

\subsection*{A new and very interesting idea}
Finally read the Paper of Gintis (TODO: Cite). It is a very interesting paper as it doesn't provide a very complicated model but is a very simple mechanism with interesting outcomes. Probably the most interesting thing is the use of \textit{mutation} of agents behaviour like in Genetic Algorithms (Gintis cites a paper on GA). Also it is an approach of explaining the dynamics of an equilibrium process, which I am very interested in. \\
Thus this could be a very interesting thing to follow as the mutation could be interpreted as the agents change and adaption of behaviour. Also I am familiar with basics of evolutionary algorithms and optimization so this would be another idea: to incorporate evolutionary algorithms and techniques into my work. So this paper could make up a very good starting point for a research-direction: "evolutionary strategies in agent-based simulation of dynamics of equilibrium processes in decentralized trading"

\begin{itemize}
\item Evolutionary Strategies - allow to introduce a kind of unpredictability in behaviour with a natural motivation of the use: agents adopting their trading strategies to those of other, more successful agents or just adopt them because the "think" it will be better (theoretically speaking: approaching and leaving local minima).
\item Equilibrium processes - so far how an equilibrium is approach is not understood and no suitable theory has been presented. I think it is a very interesting topic to pursue and VERY well suited for agent-based simulation
\item Decentralized trading - In earlier days trading was always a kind of barter without any central authority coordinating prices. Also when looking at a stock market the prices emerge out of a "bartering" process without a central authority calling out prices. Thus it seems suitable to drop the assumption of a central authority and look for equilibrium dynamics in decentralized trading only.
\end{itemize}

I should finally read the book "Debt: the first 5.000 years" by David Graeber, I think it will be a good inspiration.

\subsection*{The principle direction of my research}

It is now clear to me that when doing computational economics I could follow down one of two roads:

\begin{enumerate}
\item Orthodox Economics: complicated but unrealistic models, continuous functions nice behaving functions, nice-behaved mathematics, static equilibrium models with centralized institutions, Gaussian-distributions,...
\item Progressive Economics: sceptic about equilibrium, looking for equilibrium dynamics, fractal nature of reality instead of mediocrity, 
\end{enumerate}

As I am very sceptic about the nice-behaved mathematics, complicated models, continuous nice behaving functions - the world does not work like that - I think that I should follow the road of progressive economics.

\subsection*{Updated TODOs}
Added two books: "Debt: the first 5000 years" and "Understanding Capitalism"

\begin{itemize}
\item Read the book "Debt: the first 5000 years" by David Graeber (TODO: cite)
\item Read the book "Understanding Capitalism" (TODO: cite)
\item Read interesting papers in my folder - need to enumerate them in a following entry.
\item Put ALL printed papers of my folders into Zotero. 
\item Read my Masterthesis again.
\item Read the following chapters of "Handbook of Market-Design" (TODO: cite) in the given order: 2,3,12,10,13,16,17
\item Rework my Reseach-Proposal and reformulate it to cover the above idea. 
This will then be the basics of the initial meeting with my supervisors. Read through ALL the research-diary entries and incorporate the essence of them and make a list of the research-ideas found so far in my diary.
\item Get basic understanding in Agda 
	\begin{itemize}
	\item Work through "Dependent Types at Work"
	\item Look at Thorsten Altenkirchs Lecture "Computer Aided Formal Verification
	\item Read online Lectures about Dependent Types: TODO
	\end{itemize}
\item Haskell Learning: Implement the core of my Masterthesis in Haskell as I know it at the moment and trying to incorporate Monads: the core is the replicated auction-mechanism where every Agent knows all others.
\end{itemize}