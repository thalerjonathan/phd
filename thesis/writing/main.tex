\documentclass[oneside]{book}

\setcounter{tocdepth}{2}
\setcounter{secnumdepth}{3}

\usepackage[toc,page]{appendix}
\usepackage{minted}
\usepackage[english]{babel}
\usepackage{graphicx}
\usepackage{hyperref}
\usepackage{amsmath} % Required for some math elements 
\usepackage{pdflscape}
\usepackage{pdfpages}

\newminted[HaskellCode]{haskell}{fontsize=\footnotesize}

\begin{document}

\begin{titlepage}
	\centering
	\includegraphics[width=0.60\textwidth]{./logo/UoN_Primary_Logo_RGB.png}\par\vspace{1cm}
	{\scshape\Large PhD Thesis\par}
	\vspace{1.5cm}
	%{\huge\bfseries Foundations Of Pure Functional \\ Agent-Based Simulation \par}
%	{\huge\bfseries Pure Functional Programming \\ In Agent-Based Simulation \par}
	{\huge\bfseries The Pure Functional Programming Paradigm \\ In Agent-Based Simulation \par}
	\vspace{2cm}
	{\Large Jonathan Thaler (4276122) \\ \itshape jonathan.thaler@nottingham.ac.uk \par}
	\vfill
	supervised by\par
	Dr. Peer-Olaf \textsc{Siebers} \\
	Dr. Thorsten \textsc{Altenkirch}

	\vfill

	{\large \today\par}
\end{titlepage}

\cleardoublepage

\section*{Abstract}
This thesis shows how to implement Agent-Based Simulations (ABS) using the \textit{pure} functional programming paradigm and what the benefits and drawbacks are when doing so. As language of choice, Haskell is used due to its modern nature, increasing use in real-world applications and \textit{pure} nature. The thesis presents various implementation techniques to ABS and then discusses concurrency and parallelism and verification and validation in ABS in a pure functional setting. Additionally the thesis briefly discusses the use of dependent types in ABS, to close the gap between specification and implementation - something the presented implementation techniques don't focus on.
Finally a case-study is presented which tries to bring together the insights of the previous chapters by replicating an agent-based model both in pure and dependently typed functional programming. The agent-based model which was selected was much discussed in ABS communities as it claimed to have solved a fundamental problem of economics but it was then found that the implementation had a number of bugs which shed doubt on the validity and correctness of the results. The thesis' case study investigates whether this failure could have happened in pure and dependent functional programming and is a further test to see of how much value functional programming is to ABS. 

%TODO: from the 2nd annual review i got the feedback that i need to come up with a more coherent thesis structure which tells a story: i need to make myself clear what story i want to tell with my phd and what research and publications i still need to do for that (identify chapters and to which extent they are already finished). Also I need to come up with a precise publication plan, focusing on writing papers early than doing research for a too long time and starting an early thesis writing. This is because any unpublished research is lost and every published research is a contribution to knowledge. "A good PhD is asking more questions than it answers."

\clearpage
\tableofcontents
\clearpage

\chapter*{Acknowledgements}
Thanks go to my first supervisor Peer-Olaf Siebers, who always very patiently reminded me that a Ph.D. is not about to change the world but learning how to do research on my own. He was a strong guidance throughout my 3 years in Nottingham and I could not have hoped for a better and more dedicated first supervisor.

I am also thankful for my second supervisor Thorsten Altenkirch, who gave strong and sometimes brutal feedback about the technical details of my approaches. Due to the fact that his main interest is a rather theoretical spin on functional programming and computing, I am deeply grateful for his strong support of my rather practical approach to functional programming.

I am in depth to the whole Functional Programming Lab at UoN, for welcoming me in their midst despite my lack of specific theoretical background. I owe them many open and deep discussions which resulted in new insights. Further, presenting at their \textit{FP Lunch} was always a challenging but highly rewarding activity, always resulting in valuable feedback.

I am especially in depth to Ivan Perez for always having an open ear for questions and valuable discussions about his research, without this Ph.D. would have probably developed a very different spin.

Many thanks go to Martin Handley and James Hey for many discussions, feedback and proof reading of my papers and my thesis.

Thanks go also to Julie Greensmith for valuable discussions and pointing me into right directions at important stages of the Ph.D.

%*******************************************************************************
%*********************************** First Chapter *****************************
%*******************************************************************************

\chapter{Introduction}  %Title of the First Chapter
I noticed that it is pretty hard to convince an agent-based economics specialist who is not a computer scientist about a pure functional approach. My conjecture is that the implementation technique and method does not matter much to them because they have very little knowledge about programming and are almost always self-taught - they don't know about software-engineering, nothing about proper software-design and architecture, nothing about software-maintenance, nothing about unit-testing,... In the end they just "hack" the simulation in whatever language they are able to: C++, Visual Basic, Java or toolboxes like Netlogo. For them it is all about to \textit{get things done somehow} and not to get things done the right way or in a beautiful way - the way and the method doesn't matter, its just a necessary evil which needs to be done. Thus if functional programming could make their lives easier, then they will definitely welcome it. But functional programming is, i think, harder to learn and harder to understand - so one needs to provide an abstraction through EDSL. So I REALLY need to come up with convincing arguments why to use pure functional approaches in ACE THEY can understand, otherwise I will be lost and not heard (not published,...). \\

What ACE economists care for:

\begin{itemize}
\item Very: Qualitative modelling with quantitative results
\item Yes: Easy reproducibility
\item Likely: Reasoning about convergence?
\item Likely: EDSL
\end{itemize}

My contributions are: pure functional framework, functional agent-model for market-simulations, EDSL for market-simulations, qualitative / implicit modelling with quanitative results, reasoning in my framework about convergence \\

IDEA: could I develop non-causal modelling (models are expressed in terms of non-directed equations, modelled in signal-relations) to allow for qualitative modelling for the agent-based economists? See hybrid modelling paper of Yampa. \textbf{THIS WOULD BE A HUGE NOVEL CONTRIBUTION TO ACE ESPECIALLY WHEN COMBINED WITH AN EDSL AND PROVIDING FULL REFERENTIAL TRANSPARENCY TO KEEP THE ABILITY TO REASON ABOUT CONVERGENCE}. This should be covered in the "EDSL"-paper.

TODO: maybe i should really focus only on market models? otherwise too much? \\

central novelty of my PhD: model specification = runnable code. possible through EDSL. but only in specific subfield of ACE: market-models. need a functional description of the model, then translate it to model specification in EDSL and then run it to see dynamics. But: model specification moves closer to functional programming languages. \\

another novelty approach: model specification through qualitative instead of quantiative approaches. is this possible? \\

WHY FUNCTIONAL? "because its the ultimate approach to scientific computing": fewer bugs due to mutable state (why? is thos shown obkectively by someone?), shorter (again as above, productivity), more expressive and closer to math, EDSL, EDSL=model=simulation, better parallelising due to referental transparency, reasoning \\

scientific results need to be reproduced, especially when they have high impact. a more formal approach of specifying the model and the simulation (model=simulation) could lead to easier sharing and easier reporduction without ambigouites \\

pure functional agent-model \& theory, EDSL framework in Haskell for ACE

\begin{enumerate}
\item Which kind of problem do we have?
\item What aim is there? Solving the problem? 
\item How the aim is achieved by enumerating VERY CLEAR objectives.
\item What the impact one expects (hypothesis) and what it is (after results).
\end{enumerate}

Note: It is not in the interest of the researcher to develop new economic theories but to research the use of functional methods (programming and specification) in agent-based computational economics (ACE).

NOTE: Get the reader’s attention early in the introduction: motivation, significance, originality and novelty.

\section{Methods}
Methods need to be selected to implement the simulations. Special emphasis will be put on functional ones which will then be compared to established methods in the field of ABM/S and ACE. \\

Claim: non-programming environments are considered to be not powerful enough to capture the complexity of ACE implementations thus a programming approach to ACE will be always required.

\section{Scenarios}
To apply and test functional methods in ACE, four scenarios of ACE are selected and then the methods applied and compared with each other to see how each of them perform in comparison. The 4 selected scenarios represent a selection of the challenges posed in ACE: from very abstract ones to very operational ones.

\section{Comparison}
Each of the selected scenarios is then implemented using the selected methods where each solution is then compared against the following criteria: 

\begin{enumerate}
\item suitability for scientific computation
\item robustness
\item error-sources
\item testability
\item stability
\item extendability
\item size of code
\item maintainability
\item time taken for development
\item verification \& correctness
\item replications \& parallelism
\item EDSL
\end{enumerate}

This will then allow to compare the different methods against each other and to show under which circumstances functional methods shine and when they should not be used.

\section{Agent-Based Modelling and Simulation (ABM/S)}
ABM/S is a method of modelling and simulating a system where the global behaviour may be unknown but the behaviour and interactions of the parts making up the system is of knowledge (Wooldrige, M. (2009). An Introduction to MultiAgent Systems. John Wiley & Sons). Those parts, called agents, are modelled and simulated out of which then the aggregate global behaviour of the whole system emerges. Thus the central aspect of ABM/S is the concept of an Agent which can be understood as a metaphor for a pro-active unit, able to spawn new Agents, and interacting with other Agents in a network of neighbours by exchange of messages. The implementation of Agents can vary and strongly depends on the programming language and the kind of domain the simulation and model is situated in.

\section{Agent-Based Economics (ACE)}
According to Leigh Tesfatsion (Tesfatsion, L. (2006). Agent-based computational economics: A constructive approach to economic theory. In Tesfatsion, L. and Judd, K. L., editors, Handbook of Computational Economics, volume 2, chapter 16, pages 831–880. Elsevier, 1 edition.), one of the leading figures, ACE is "[...] computational modelling of economic processes (including whole economies) as open-ended dynamic systems of interacting agents." - thus lending perfectly to the use of ABM/S as already the name suggests. Whereas classical economic models fall short by only looking at the average, pure rational, individual interacting in anonymous markets, the ACE approach looks at heterogeneous, non-rational individuals interacting with each other in networks (Kirman, A. (2010). Complex Economics: Individual and Collective Rationality. Routledge, London ; New York, NY.). Thus ACE can be understood as a combination of computer-science, cognitive/social science and evolutionary economics.

\section{Functional programming}
TODO: read \cite{Backus1978}

The state-of-the-art approach to implementing Agents are object-oriented methods and programming as the metaphor of an Agent as presented above lends itself very naturally to object-orientation (OO). The author of this thesis claims that OO in the hands of inexperienced or ignorant programmers is dangerous, leading to bugs and hardly maintainable and extensible code. The reason for this is that OO provides very powerful techniques of organising and structuring programs through Classes, Type Hierarchies and Objects, which, when misused, lead to the above mentioned problems. Also major problems, which experts face as well as beginners are 1. state is highly scattered across the program which disguises the flow of data in complex simulations and 2. objects don’t compose as well as functions. The reason for this is that objects always carry around some internal state which makes it obviously much more complicated as complex dependencies can be introduced according to the internal state.
All this is tackled by (pure) functional programming which abandons the concept of global state, Objects and Classes and makes data-flow explicit. This then allows to reason about correctness, termination and other properties of the program e.g. if a given function exhibits side-effects or not. Other benefits are fewer lines of code, easier maintainability and ultimately fewer bugs thus making functional programming the ideal choice for scientific computing and simulation and thus also for ACE. A very powerful feature of functional programming is Lazy evaluation. It allows to describe infinite data-structures and functions producing an infinite stream of output but which are only computed as currently needed. Thus the decision of how many is decoupled from how to (Hughes, J. (1989). Why functional programming matters. Comput. J., 32(2):98–107.).
The most powerful aspect using pure functional programming however is that it allows the design of embedded domain specific languages (EDSL). In this case one develops and programs primitives e.g. types and functions in a host language (embed) in a way that they can be combined. The combination of these primitives then looks like a language specific to a given domain, in the case of this thesis ACE. The ease of development of EDSLs in pure functional programming is also a proof of the superior extensibility and composability of pure functional languages over OO (Henderson P. (1982). Functional Geometry. Proceedings of the 1982 ACM Symposium on LISP and Functional Programming.).
One of the most compelling example to utilize pure functional programming is the reporting of Hudak (Hudak P., Jones M. (1994). Haskell vs. Ada vs. C++ vs. Awk vs. ... An Experiment in Software Prototyping Productivity. Department of Computer Science, Yale University.)  where in a prototyping contest of DARPA the Haskell prototype was by far the shortest with 85 lines of code. Also the Jury mistook the code as specification because the prototype did actually implement a small EDSL which is a perfect proof how close EDSL can get to and look like a specification.

Functional languages can best be characterized by their way computation works: instead of \textit{how} something is computed, \textit{what} is computed is described. Thus functional programming follows a declarative instead of an imperative style of programming. The key points are:
\begin{itemize}
\item No assignment statements - variables values can never change once given a value.
\item Function calls have no side-effect and will only compute the results - this makes order of execution irrelevant, as due to the lack of side-effects the logical point in \textit{time} when the function is calculated within the program-execution does not matter.
\item higher-order functions
\item lazy evaluation
\item Looping is achieved using recursion, mostly through the use of the general fold or the more specific map.
\item Pattern-matching
\end{itemize}

This alone does not really explain the \textit{real} advantages of functional programming and one must look for better motivations using functional programming languages. One motivation is given in \cite{Hughes1989} which is a great paper explaining to non-functional programmers what the significance of functional programming is and helping functional programmers putting functional languages to maximum use by showing the real power and advantages of functional languages. The main conclusion is that \textit{modularity}, which is the key to successful programming, can be achieved best using higher-order functions and lazy evaluation provided in functional languages like Haskell. \cite{Hughes1989} argues that the ability to divide problems into sub-problems depends on the ability to glue the sub-problems together which depends strongly on the programming-language and \cite{Hughes1989} argues that in this ability functional languages are superior to structured programming.

TODO: comparison of functional and object-oriented programming. My points are:
\begin{itemize}
\item The way state can be changed and treated - distributed over multiple objects - is often very difficult to understand.
\item Inheritance is a dangerous thing if not used with care because inheritance introduces very strong dependencies which cannot be changed during runtime anymore.
\item Objects don't compose very well: \url{http://zeroturnaround.com/rebellabs/why-the-debate-on-object-oriented-vs-functional-programming-is-all-about-composition/}
\item (Nearly) impossible to reason about programs
\end{itemize}

In conclusion the upsides of functional programming as opposed to OO are:
\begin{itemize}
\item Much more explicit flow of data \& control
\item Much better compose-able
\item Much better parallelism
\end{itemize}

\section{Related Research}
Tim Sweeney, CTO of Epic Games gave an invited talk about how "future programming languages could help us write better code" by "supplying stronger typing, reduce run-time failures;  and the need for pervasive concurrency support, both implicit and explicit, to effectively exploit the several forms of parallelism present in games and graphics." \cite{Sweeney2006}. Although the fields of games and agent-based simulations seem to be very different in the end, they have also very important similarities: both are simulations which perform numerical computations and update objects - in games they are called "game-objects" and in abm they are called agents but they are in fact the same thing - in a loop either concurrently or sequential. His key-points were:

\begin{itemize}
\item Dependent types as the remedy of most of the run-time failures.
\item Parallelism for numerical computation: these are pure functional algorithms, operate locally on mutable state. Haskell ST, STRef solution enables encapsulating local heaps and mutability within referentially transparent code.
\item Updating game-objects (agents) concurrently using STM: update all objects concurrently in arbitrary order, with each update wrapped in atomic block - depends on collisions if performance goes up.
\end{itemize}

\section{Related research and literature}
\label{sec:literature}

The amount of research on using pure functional programming with Haskell in the field of ABS has been moderate so far. Most of the papers are related to the field of Multi Agent Systems (MAS) and look into how agents can be specified using the belief-desire-intention paradigm \cite{de_jong_suitability_2014,sulzmann_specifying_2007,jankovic_functional_2007}.

A multi-method simulation library in Haskell called \textit{Aivika 3} is described in the technical report \cite{sorokin_aivika_2015}. It supports implementing Discrete Event Simulations (DES), System Dynamics and comes with basic features for event-driven ABS which is realised using DES under the hood. Further it provides functionality for adding GPSS to models and supports parallel and distributed simulations. It runs within the IO effect type for realising parallel and distributed simulation but also discusses generalising their approach to avoid running in IO.

In his master thesis \cite{bezirgiannis_improving_2013} the author investigates Haskells' parallel and concurrency features to implement (amongst others) \textit{HLogo}, a Haskell clone of the NetLogo \cite{wilensky_introduction_2015} simulation package, focusing on using STM for a limited form of agent-interactions. \textit{HLogo} is basically a re-implementation of NetLogos API in Haskell where agents run within an unrestricted context (known as \textit{IO}) and thus can also make use of STM functionality. The benchmarks show that this approach does indeed result in a speed-up especially under larger agent-populations. The authors' thesis can be seen as one of the first works on ABS using Haskell. Despite the concurrency and parallel aspect our work share, our approach is rather different: we avoid IO within the agents under all costs and explore the use of STM more on a conceptual level rather than implementing a ABS library and compare our case-studies with lock-based and imperative implementations.

There exists some research \cite{di_stefano_using_2005, varela_modelling_2004, sher_agent-based_2013} using the functional programming language Erlang \cite{armstrong_erlang_2010} to implement concurrent ABS. The language is inspired by the actor model \cite{agha_actors:_1986} and was created in 1986 by Joe Armstrong for Eriksson for developing distributed high reliability software in telecommunications. The actor model can be seen as quite influential to the development of the concept of agents in ABS, which borrowed it from Multi Agent Systems \cite{wooldridge_introduction_2009}. It emphasises message-passing concurrency with share-nothing semantics (no shared state between agents), which maps nicely to functional programming concepts. The mentioned papers investigate how the actor model can be used to close the conceptual gap between agent-specifications, which focus on message-passing and their implementation. Further they show that using this kind of concurrency allows to overcome some problems of low level concurrent programming as well.
Also \cite{bezirgiannis_improving_2013} ported NetLogos API to Erlang mapping agents to concurrently running processes, which interact with each other by message-passing. With some restrictions on the agent-interactions this model worked, which shows that using concurrent message-passing for parallel ABS is at least \textit{conceptually} feasible. Despite the natural mapping of ABS concepts to such an actor language, it leads to simulations, which despite same initial starting conditions, might result in different dynamics each time due to concurrency.

The work \cite{lysenko_framework_2008} discusses a framework, which allows to map Agent-Based Simulations to Graphics Processing Units (GPU). Amongst others they use the SugarScape model \cite{epstein_growing_1996} and scale it up to millions of agents on very large environment grids. They reported an impressive speed-up of a factor of 9,000. Although their work is conceptually very different we can draw inspiration from their work in terms of performance measurement and comparison of the SugarScape model.

% THIS IS MY OWN REASEARCH, DON'T CITE IT HERE
%In \cite{thaler_pure_2019} the authors showed how to implement a spatial SIR model in pure Haskell using Functional Reactive Programming \cite{hudak_arrows_2003}. They report quite low performance but mention that STM may be a way to considerably speed up the simulation. We follow their approach in implementation technique, using Functional Reactive Programming and Monadic Stream Functions \cite{perez_functional_2016} (we don't go into implementation details as it is out of the scope of this paper) and use the spatial SIR model as the first case-study.

Using functional programming for DES was discussed in \cite{jankovic_functional_2007} where the authors explicitly mention the paradigm of Functional Reactive Programming (FRP) to be very suitable to DES.

A domain-specific language for developing functional reactive agent-based simulations was presented in \cite{schneider_towards_2012,vendrov_frabjous_2014}. This language called FRABJOUS is human readable and easily understandable by domain-experts. It is not directly implemented in FRP/Haskell but is compiled to Haskell code which they claim is also readable. This supports that FRP is a suitable approach to implement ABS in Haskell. Unfortunately, the authors do not discuss their mapping of ABS to FRP on a technical level, which would be of most interest to functional programmers.

Object-oriented programming and simulation have a long history together as the former one emereged out of Simula 67 \cite{dahl_birth_2002} which was created for simulation purposes. Simula 67 already supported Discrete Event Simulation and was highly influential for today's object-oriented languages. Although the language was important and influential, in our research we look into different approaches, orthogonal to the existing object-oriented concepts.

Lustre is a formally defined, declarative and synchronous dataflow programming language for programming reactive systems \cite{halbwachs_synchronous_1991}. While it has solved some issues related to implementing ABS in Haskell it still lacks a few important features necessary for ABS. We don't see any way of implementing an environment in Lustre as we do in Chapters  \ref{ch:timedriven} and \ref{ch:eventdriven}. Also the language seems not to come with stochastic functions, which are but the very building blocks of ABS. Finally, Lustre does only support static networks, which is clearly a drawback in ABS in general where agents can be created and terminated dynamically during simulation.

The authors of \cite{botta_time_2010} discuss the problem of advancing time in message-driven agent-based socio-economic models. They formulate purely functional definitions for agents and their interactions through messages. Our architecture for synchronous agent-interaction as discussed in Chapter TODO was not directly inspired by their work but has some similarities: the use of messages and the problem of when to advance time in models with arbitrary number synchronised agent-interactions.

The authors of \cite{botta_functional_2011} are using functional programming as a specification for an agent-based model of exchange markets but leave the implementation for further research where they claim that it requires dependent types. This paper is the closest usage of dependent types in agent-based simulation we could find in the existing literature and to our best knowledge there exists no work on general concepts of implementing pure functional agent-based simulations with dependent types. As a remedy to having no related work to build on, we looked into works which apply dependent types to solve real world problems from which we then can draw inspiration from.

In his talk \cite{sweeney_next_2006}, Tim Sweeney CTO of Epic Games discussed programming languages in the development of game engines and scripting of game logic. Although the fields of games and ABS seem to be very different, Gregory \cite{gregory_game_2018} defines computer-games as \textit{"[..] soft real-time interactive agent-based computer simulations"} (p. 9) and in the end they have also very important similarities: both are simulations which perform numerical computations and update objects in a loop either concurrently or sequential. In games these objects are called \textit{game-objects} and in ABS they are called \textit{agents} but they are conceptually the same thing.  Sweeney reports that reliability suffers from dynamic failure in languages like C++ e.g. random memory overwrites, memory leaks, accessing arrays out-of-bounds, dereferencing null pointers, integer overflow, accessing uninitialized variables. He reports that 50\% of all bugs in the Game Engine Middleware Unreal can be traced back to such problems and presents dependent types as a potential rescue to those problems. The two main points Sweeney made were that dependent types could solve most of the run-time failures and that parallelism is the future for performance improvement in games. He distinguishes between pure functional algorithms which can be parallelised easily in a pure functional language and updating game-objects concurrently using software transactional memory (STM).

\chapter{Methodology}
This chapter introduces the background and methodology used in the following chapters. Roughly 50\% exists already.

\section{Agent-Based Simulation}
\label{sec:method_abs}

This thesis understands ABS as a method and methodology to model and simulate a system, where the global behaviour may be unknown but the behaviour and interactions of the parts making up the system is known. Those parts, called agents, are modelled and simulated, out of which then the aggregate global behaviour of the whole system emerges. So, the central aspect of ABS is the concept of an agent, a metaphor for a proactive unit, situated in an environment which is able to spawn new agents and interacting with other agents in some neighbourhood by the exchange of messages \cite{macal_everything_2016, odell_objects_2002, siebers_introduction_2008, wooldridge_introduction_2009}. In summary, this thesis informally assumes the following about agents:

\begin{itemize}
	\item They are uniquely addressable entities with an internal state over which they have full, exclusive control.
	\item They are proactive, which means they can initiate actions on their own. For example they can change their internal state, send messages, create new agents or terminate themselves.
	\item They are situated in an environment and can interact with it.
	\item They can interact with other agents situated in the same environment by means of messaging.
\end{itemize} 

Epstein \cite{epstein_generative_2012} identifies ABS to be particularly applicable for analysing \textit{"spatially distributed systems of heterogeneous autonomous actors with bounded information and computing capacity"}. Technically, ABS exhibits the following properties:

\begin{itemize}
	\item Linearity and non-linearity - actions of agents can lead to non-linear behaviour of the system.
	\item Time - agents act over time, which is also the source of their proactivity.
	\item State - agents encapsulate state, which can be accessed and changed during the simulation.
	\item Feedback loop - because agents act continuously and their actions influence each other and themselves in the future of subsequent time steps, feedback loops permeate every ABS. 
	\item Heterogeneity - agents can have properties (age, height, sex, etc.) where the actual values can vary arbitrarily between individuals.
	\item Interactions - agents can be modelled after interactions with an environment and other agents.
	\item Spatiality and networks - agents can be situated within arbitrary environments, like spatial environments (discrete 2D, continuous 3D, etc.) or complex networks.
\end{itemize}

There is no commonly agreed technical definition of ABS but the field draws inspiration from the closely related field of Multi-Agent Systems \cite{weiss_multiagent_2013,wooldridge_introduction_2009}. It is important to understand that Multi-Agent Systems and ABS are two different fields, where in Multi-Agent Systems the focus is geared towards technical details, implementing a system of interacting intelligent agents within a highly complex environment with the primary focus being solving AI problems.

\medskip

The field of ABS can be traced back to self-replicating von Neumann machines, cellular automata and Conway's Game of Life. The famous Schelling segregation model \cite{schelling_dynamic_1971} is regarded as a pioneering example. ABS as a discipline was first picked up by social simulation, which explores social norms, institutions, reputation, elections and economics. Axelrod \cite{axelrod_advancing_1997, axelrod_guide_2006} has called social simulation the third way of doing science which he termed the \textit{generative} approach. This is in opposition to the classical inductive (finding patterns in empirical data) and deductive (proving theorems). Consequently, the generative approach can be seen as a form of empirical research and is a natural methodology for studying social and interdisciplinary phenomena as discussed more in depth in the work of Epstein \cite{epstein_chapter_2006, epstein_generative_2012}. He gives a fundamental introduction to agent-based social social simulation and makes the strong claim that \textit{"If you didn't grow it, you didn't explain its emergence"} \footnote{This can be seen as a fundamental constructivist approach to social science, which implies that the emergent properties are actually computable. When making connections from the simulation to reality, constructible emergence raises the question whether our existence is computable or not. When pushing this further, we can suppose that the future of simulation will be simulated copies of our own existence, which potentially allows us to then simulate \textit{everything}. This idea is not actually new and an interesting treatment of it can be found in \cite{bostrom_are_2003, steinhart_theological_2010}.}. 
Epstein puts considerable emphasis on the claim that ABS is indeed a scientific instrument because hypotheses about the outcome of a simulation are empirically falsifiable. If the simulation exhibits an emergent pattern, then the model is \textit{one} way of explaining it. On the other hand, if it does not show the emergent pattern, then the hypothesis that the micro interactions amongst the agents generate the emergent pattern is falsified \footnote{This is fundamentally following Poppers theory of science \cite{popper_logic_2002}.} and we have not found an explanation \textit{yet}. In conclusion, growing a phenomena is a necessary, but not sufficient condition for explanation \cite{epstein_chapter_2006}.

% NOTE: incorporate this only when there is enough time (and energy) to go through the 3 references cited here
%This raises a number of philosophical questions \cite{frigg_philosophy_2009}, \cite{grune-yanoff_philosophy_2010}, \cite{borrill_agent-based_2011}. Although we don't want to give an in-depth discussion of the questions raised, we want to have a quick look at them as this is a foundational research-proposal for a Doctor in \textit{Philosophy} (Ph.D.).
%TODO: read above papers and give short outline philosophical questions

The first large scale social ABS model which rose to some prominence was the \textit{Sugarscape} model developed by Epstein and Axtell in 1996 \cite{epstein_growing_1996}. Their aim was to \textit{grow} an artificial society by simulation and connect observations in their simulation to the phenomenon of real-world societies. It was this model which strongly advertised object-oriented programming to implement ABS. Due to its influence and also due to the general popularity of the object-oriented paradigm which started to rise in the early-to-mid 1990s, object-oriented programming has become the de-facto standard in implementing ABS. We can distinguish between three categories of ABS implementation today: % TODO: do we need citiations here to support our claims?
\begin{enumerate}
	\item Programming from scratch using object-oriented languages, with Python, Java and C++ being the most popular.
	\item Programming with a 3rd party ABS library using object-oriented languages where RePast and DesmoJ, both in Java, are the most popular.
	\item Using a high-level ABS toolkit for non-programmers, which allow customisation through programming, if necessary. By far the most popular is NetLogo with an imperative programming approach followed by AnyLogic with an object-oriented Java approach.
\end{enumerate}

To get a better idea and deeper understanding of ABS, the next sections present two different but well-known agent-based models, to give examples of two different types: the \textit{explanatory} SIR model and the \textit{exploratory} Sugarscape model. Both are used throughout this thesis as sample cases for developing pure functional ABS implementation techniques, concepts and test beds for Software Transactional Memory and property-based testing.

\subsection{The SIR model}
\label{sec:sir_model}

The explanatory SIR model is a thoroughly studied and well understood compartment model from epidemiology \cite{kermack_contribution_1927}, which allows simulation of the dynamics of an infectious disease like influenza, tuberculosis, chicken pox, rubella and measles spreading through a population. The reason for choosing this model is its simplicity. It is easy to understand fully but complex enough to develop basic concepts of pure functional ABS, which are then extended and deepened in the much more complex Sugarscape model explained in the next section.

In this model, people in a population of size $N$ can be in either one of three states: \textit{Susceptible}, \textit{Infected} or \textit{Recovered}, at any particular time. It is assumed that initially there is at least one infected person in the population. People interact \textit{on average} with a given rate of $\beta$ other people per time unit, and become infected with a given probability $\gamma$ when interacting with an infected person. When infected, a person recovers \textit{on average} after $\delta$ time units and is then immune to further infections. An interaction between infected persons does not lead to reinfection, thus these interactions are ignored in this model. This definition gives rise to three compartments with the transitions seen in Figure \ref{fig:sir_transitions}.

\begin{figure}
	\centering
	\includegraphics[width=.7\textwidth, angle=0]{./fig/timedriven/SIR_transitions.png}
	\caption[States and transitions in the SIR compartment model]{States and transitions in the SIR compartment model.}
	\label{fig:sir_transitions}
\end{figure}

This model was also formalized using System Dynamics \cite{porter_industrial_1962}. In System Dynamics a system is modelled through differential equations, which allow expressing continuous systems, changing over time. They are solved by numerically integrating over time, which gives rise to the respective dynamics. The SIR model is modelled using the following equation, with the dynamics shown in Figure \ref{fig:sir_sd_dynamics} .

\begin{equation}
\begin{aligned}
\frac{\mathrm d S}{\mathrm d t} = -infectionRate \\
\frac{\mathrm d I}{\mathrm d t} = infectionRate - recoveryRate \\
\frac{\mathrm d R}{\mathrm d t} = recoveryRate 
\end{aligned}
\end{equation}

\begin{equation}
\begin{aligned}
infectionRate = \frac{I \beta S \gamma}{N} \\
recoveryRate = \frac{I}{\delta} 
\end{aligned}
\end{equation}

\begin{figure}
	\centering
	\includegraphics[width=0.5\textwidth, angle=0]{./fig/timedriven/SIR_SD_1000agents_150t_001dt.png}
	\caption[Dynamics of the SIR compartment model using the System Dynamics approach]{Dynamics of the SIR compartment model using the System Dynamics approach. Population Size $N$ = 1,000, contact rate $\beta =  \frac{1}{5}$, infection probability $\gamma = 0.05$, illness duration $\delta = 15$ with initially 1 infected agent. Simulation run for 150 time steps. Generated using our pure functional System Dynamics approach (see Appendix \ref{app:sd_simulation}).}
	\label{fig:sir_sd_dynamics}
\end{figure}

The approach of mapping the SIR model to an ABS is to discretise the population and model each person in the population as an individual agent. The transitions between the states are happening due to discrete events caused both by interactions amongst the agents and timeouts. The major advantage of ABS over System Dynamics is that it allows for the incorporation of spatiality and heterogeneity of a population, for example accounting for different sexes and ages. This is not directly possible with other simulation methods of System Dynamics or Discrete Event Simulation \cite{zeigler_theory_2000}.

This is directly related to a networked SIR model, where the interactions between agents are restricted by either a statically fixed or dynamically evolving network. Various network types exist, allowing for simulation of various scenarios. Very small communities where all agents are in contact with each other are modelled by a fully connected network. Real world scenarios where a few agents act as hubs are modelled by complex networks \cite{BarabasiAlbert_EmergenceScaling, Jackson2008, Newman_ComplexNetworks, WattsStrogatz_DynamicsSmallWorld}. In this thesis we do not impose restrictions on the connections among agents and always assume a fully connected network. Adding various network types to our thesis would unnecessarily complicate things in the beginning but would not constitute anything fundamentally new in terms of research. However, the use of complex networks, which in general are generated randomly, constitute an interesting direction for further research especially in the context of randomised property-based testing in ABS, which we discuss in Chapters \ref{ch:agentspec} and \ref{ch:sir_invariants}.

In the ABS classification of \cite{macal_everything_2016}, this model can be seen as an \textit{Interactive ABMS}: agents are individual heterogeneous agents with diverse set characteristics; they have autonomic, dynamic, endogenously defined behaviour; interactions happen between other agents and the environment through observed states, behaviours of other agents and the state of the environment.

\section{Case-Study II: Sugarscape}
TODO: 
we can implement everything except synchronous direct agent-interactions atm: if agent-interaction is one-way e.g. paying back a loan then this is no problem. thus the following parts of the Sugarscape are not possible with our current STM approach: mating, trading and lending  because all 3 require direct agent-to-agent interaction over multiple steps. We leave the problem of developing such an algorithm / implementation for further research.

\chapter{Functional Programming}
MacLennan \cite{maclennan_functional_1990} defines Functional Programming as a methodology and identifies it with the following properties:

\begin{enumerate}
	\item It is programming without the assignment-operator.
	\item It allows for higher levels of abstraction.
	\item It allows to develop executable specifications and prototype implementations.
	\item It is connected to computer science theory.
	\item Parallel Programming.
	\item Suitable for AI.
\end{enumerate}

The last two points don't weight as heavy today as back in 1990 as other languages came up with features for better parallel programming but they all do it by introducing functional features.

MacLennan \cite{maclennan_functional_1990} defines properties of pure expressions 
\begin{itemize}
	\item Value is independent of the evaluation order.
	\item Expressions can be evaluated in parallel.
	\item Referential transparency.
	\item No side effects.
	\item Inputs to an operation are obvious from the written form.
	\item Effects to an operation are obvious from the written form.
\end{itemize}

TODO: The question is then if we could implement in a functional style in an imperative object-oriented programming language? Or put otherwise: are these properties unique to functional programming or can we program functional in an imperative language (be it OO or not)?

Thus functional programming is identified as programming without the assignment operator and with pure expressions instead. Further characteristics are the missing of orderings as in imperative programming, caused by assignments: in functional programming the style is applicative which means we apply values to functions. The fundamental theoretical root is in the lambda calculus.

TODO: make distinction between 'applicative programming (style)', which is easily possible in imperative languages as well and 'functional programming' which is not possible in procedural languages. The question is if it is possible in OO by using some OO features to work around the limitations of procedural languages.

- cite critics

\section{Dependent Types}
Dependent types are a very powerful addition to functional programming as they allow us to express even stronger guarantees about the correctness of programs \textit{already at compile-time}. They go as far as allowing to formulate programs and types as constructive proofs which must be \textit{total} by definition \cite{thompson_type_1991, mckinna_why_2006, altenkirch_pi_2010}. 

So far no research using dependent types in agent-based simulation exists at all. We have already started to explore this for the first time and ask more specifically how we can add dependent types to our functional approach, which conceptual implications this has for ABS and what we gain from doing so. We are using Idris \cite{brady_idris_2013} as the language of choice as it is very close to Haskell with focus on real-world application and running programs as opposed to other languages with dependent types e.g. Agda and Coq which serve primarily as proof assistants.

We hypothesise, that  dependent types will allow us to push the correctness of agent-based simulations to a new, unprecedented level by narrowing the gap between model specification and implementation. The investigation of dependent types in ABS will be the main unique contribution to knowledge of my Ph.D.

In the following section \ref{sec:dep_background}, we give an introduction of the concepts behind dependent types and what they can do. Further we give a very brief overview of the foundational and philosophical concepts behind dependent types. In Section \ref{sec:dep_absconcepts} we briefly discuss ideas of how the concepts of dependent types could be applied to agent-based simulation and in Section \ref{sec:dep_vav_deptypes} we very shortly discuss the connection between Verification \& Validation and dependent types.

There exist a number of excellent introduction to dependent types which we use as main ressources for this section: \cite{thompson_type_1991, program_homotopy_2013, stump_verified_2016, brady_type-driven_2017, pierce_programming_2018}.

Generally, dependent types add the following concepts to pure functional programming:

\begin{enumerate}
	\item Types are first-class citizen - In dependently types languages, types can depend on any \textit{values}, and can be \textit{computed} at compile-time which makes them first-class citizen. This becomes apparent in Section \ref{sub:dep_vector} where we compute the return type of a function depending on its input values.

	\item Totality and termination - A total function is defined in \cite{brady_type-driven_2017} as: it terminates with a well-typed result or produces a non-empty finite prefix of a well-typed infinite result in finite time. This makes run-time overhead obsolete, as one does not need to drag around additional type-information as everything can be resolved at compile-time. Idris is turing-complete but is able to check the totality of a function under some circumstances but not in general as it would imply that it can solve the halting problem. Other dependently typed languages like Agda or Coq restrict recursion to ensure totality of all their functions - this makes them non turing-complete. All functions in Section \ref{sub:dep_vector} are total, they terminate under all inputs in finite steps.

	\item Types as \textit{constructive} proofs - Because types can depend on any values and can be computed at compile-time, they can be used as constructive proofs (see \ref{sub:dep_foundations}) which must terminate, this means a well-typed program (which is itself a proof) is always terminating which in turn means that it must consist out of total functions. Note that Idris does not restrict us to total functions but we can enforce it through compiler flags. We implement a constructive proof of showing whether two natural numbers are decidable equal in the Section \ref{sub:dep_equality}.
\end{enumerate}

\subsection{An example: Vector}
\label{sub:dep_vector}
To give a concrete example of dependent types and their concepts, we introduce the canonical example used in all tutorials on dependent types: the Vector.

In all programming languages like Haskell or in Java, there exists a List data-structure which holds a finite number of homogeneous elements, where the type of the elements can be fixed at compile-time. Using dependent types we can implement the same but adding the length of the list to the type - we call this data-structure a vector.

We define the vector as a Generalised Algebraic Data Type (GADT). A vector has a \textit{Nil} element which marks the end of a vector and a \textit{(::)} which is a recursive (inductive) definition of a linked List. We defined some vectors and we see that the length of the vector is directly encoded in its first type-variable of type Nat, natural numbers. Note that the compiler will refuse to accept \textit{testVectFail} because the type specifies that it holds 2 elements but the constructed vector only has 1 element.

\begin{HaskellCode}
data Vect : Nat -> Type -> Type where
     Nil  : Vect Z e
     (::) : (elem : e) -> (xs : Vect n e) -> Vect (S n) e
	
testVect : Vect 3 String
testVect = "Jonathan" :: "Andreas" :: "Thaler" :: Nil

testVectFail : Vect 2 Nat
testVectFail = 42 :: Nil
\end{HaskellCode}

We can now go on and implement a function \textit{append} which simply appends two vectors. Here we directly see \textit{type-level computations} as we compute the length of the resulting vector. Also this function is \textit{total}, as it covers all input cases and recurs on a \textit{structurally smaller argument}:

\begin{HaskellCode}
append : Vect n e -> Vect m e -> Vect (n + m) e
append Nil ys = ys
append (x :: xs) ys = x :: append xs ys

append testVect testVect
["Jonathan", "Andreas", "Thaler", "Jonathan", "Andreas", "Thaler"] : Vect 8 String
\end{HaskellCode}

What if we want to implement a \textit{filter} function, which, depending on a given predicate, returns a new vector which holds only the elements for which the predicates returns true? How can we compute the length of the vector at compile-time? In short: we can't, but we can make us of \textit{dependent pairs} where the \textit{type} of the second element depends on the \textit{value} of the first (dependent pairs are also known as $\Sigma$ types).

The function is total as well and works very similar to \textit{append} but uses dependent types as return, which are indicated by \textit{**}:

\begin{HaskellCode}
filter : Vect n e -> (e -> Bool) -> (k ** Vect k e)
filter [] f = (Z ** Nil)
filter (elem :: xs) f =
  case f elem of
    False => filter xs f
    True  => let (_ ** xs') = filter xs f
             in  (_ ** elem :: xs')
             
filter testVect (=="Jonathan")
(1 ** ["Jonathan"]) : (k : Nat ** Vect k String)
\end{HaskellCode}

It might seem that writing a \textit{reverse} function for a Vector is very easy, and we might give it a go by writing:
\begin{HaskellCode}
reverse : Vect n e -> Vect n e
reverse [] = []
reverse (elem :: xs) = append (reverse xs) [elem]
\end{HaskellCode}

Unfortunately the compiler complains because it cannot unify 'Vect (n + 1) e' and 'Vect (S n) e'. In the end, the compiler tells us that it cannot determine that (n + 1) is the same as (1 + n). The compiler does not know anything about the commutativity of addition which is due to how natural numbers and their addition are defined.

Lets take a detour. The natural numbers can be inductively defined by their initial element zero Z and the successor. The number 3 is then defined as the successor of successor of successor of zero:

\begin{HaskellCode}
data Nat = Z | S Nat

three : Nat 
three = S (S (S Z))
\end{HaskellCode}

Defining addition over the natural numbers is quite easy by pattern-matching over the first argument: 

\begin{HaskellCode}
plus : (n, m : Nat) -> Nat
plus Z right        = right
plus (S left) right = S (plus left right)
\end{HaskellCode}

Now we can see why the compiler cannot infer that (n + 1) is the same as (1 + n). The expression (n + 1) is translated to (plus n 1), where we pattern-match over the first argument, so we cannot reach a case in which (plus n 1) = S n. To do that we would need to define a different plus function which pattern-matches over the second argument - which is clearly the wrong way to go.

To solve this problem we can exploit the fact that dependent types allow us to perform type-level computations. This should allow us to express commutativity of addition over the natural numbers as a type. For that we define a function which takes in two natural numbers and returns a proof that addition commutes. 

\begin{HaskellCode}
plusCommutative : (left : Nat) -> (right : Nat) -> left + right = right + left
\end{HaskellCode}

We now begin to understand what it means when we speak of \textit{types as proofs}: we can actually express e.g. laws of the natural numbers in types and proof them by implementing a program which inhibits the type - we speak then of a constructive proof (see more on that below \ref{sub:dep_foundations}). Note that \textit{plusCommutative} is already implemented in Idris and we omit the actual implementation as it is beyond the scope of this introduction

Having our proof of commutativity of natural numbers, we can now implement a working (speak: correct) version of \textit{reverse}. The function \textit{rewrite} is provided by Idris: if we have a proof for x = y, the 'rewrite expr in' syntax will search for x in the required type of expr and replace it with y:

\begin{HaskellCode}
reverse : Vect n e -> Vect n e
reverse [] = []
reverse (elem :: xs) = reverseProof (append (reverse xs) [elem])
  where
    reverseProof : Vect (k + 1) a -> Vect (S k) a
    reverseProof {k} result = rewrite plusCommutative 1 k in result
\end{HaskellCode}

\subsection{Equality as type}
\label{sub:dep_equality}
On of the most powerful aspects of dependent types is that they allow us to express equality on an unprecedented level. Non-dependently typed languages have only very basic ways of expressing the equality of two elements of same type. Either we use a boolean or another data-structure which can indicate equality or not. Idris supports this type of equality as well through \textit{(==) : Eq ty $\Rightarrow$ ty $\rightarrow$ ty $\rightarrow$ Bool}. The drawback of using a boolean is that, in the end, we don't have a real evidence of equality: it doesn't tell you anything about the relationship between the inputs and the output. Even though the elements might be equal, the compiler has no means of inferring this and we can still make programming mistakes after the equality check because of this lack of compiler support. Even worse, always returning False / True or whether the inputs are \textit{not} equal is a valid implementation of (==), at least as far as the type is concerned.

As an illustrating example we want to write a function which checks if a Vector has a given length. 

\begin{HaskellCode}
exactLength : (len : Nat) -> (input : Vect n k) -> Maybe (Vect len k)
exactLength {n} len input = case n == len of
                                 True  => Just input 
                                 False => Nothing 
\end{HaskellCode}

Unfortunately this doesn't type-check ('type mismatch between n and len') because the compiler has no way of determining that $len$ is equals $n$ at compile-time. Fortunately we can solve this problem using dependent types themselves by defining \textit{decidable} equality as a type.

First we need a decidable property, meaning it either holds given with some \textit{proof} or it does not hold given some proof that it does \textit{not} hold, resulting in a contradiction. Idris defines such a decidable property already as the following:

\begin{HaskellCode}
-- Decidability. A decidable property either holds or is a contradiction.
data Dec : Type -> Type where
  -- The case where the property holds
  -- @ prf the proof
  Yes : (prf : prop) -> Dec prop

  -- The case where the property holding would be a contradiction
  -- @ contra a demonstration that prop would be a contradiction
  No  : (contra : prop -> Void) -> Dec prop
\end{HaskellCode}

With that we can implement a function which constructs a proof that two natural numbers are equal, or not. We do this simply by pattern matching over both numbers with corresponding base cases and inductions. In case they are not equal we need to construct a proof that they are actually not equal which is done by showing that given some property results in a contradiction - indicated by the type \textit{Void}. In case of \textit{zeroNotSuc} the first number is zero (Z) whereas the other one is non-zero (a successor of some k), which can never be equal, thus we return a \textit{No} instance of the decidable property for which we need to provide the contradiction. In case of \textit{sucNotZero} its just the other way around. \textit{noRec} works very similar but here we are in the induction case which says that if k equals j leads to a contradiction, (k + 1) and (j + 1) can't be equal as well (induction hypothesis).

\begin{HaskellCode}
checkEqNat : (num1 : Nat) -> (num2 : Nat) -> Dec (num1 = num2)
checkEqNat Z Z         = Yes Refl
checkEqNat Z (S k)     = No zeroNotSuc
checkEqNat (S k) Z     = No sucNotZero
checkEqNat (S k) (S j) = case checkEqNat k j of
                              Yes prf   => Yes (cong prf)
                              No contra => No (noRec contra)
                              
zeroNotSuc : (0 = S k) -> Void
zeroNotSuc Refl impossible

sucNotZero : (S k = 0) -> Void
sucNotZero Refl impossible

noRec : (contra : (k = j) -> Void) -> (S k = S j) -> Void
noRec contra Refl = contra Refl
\end{HaskellCode}  
                            
%TODO: explain cong and Refl

The important thing to understand here is that our Dec property holds much more information than just a boolean flag which indicates whether Yes/No that two elements of a type are equal: in case of Yes we have a type which says that num1 is equal to num2, which can be directly used by the compiler, both elements are treated as the same. Refl stands for reflexive and is built into Idris syntax, meaning that a value is equal to itself 'Refl : x = x'. %Further, we need to use 'cong' 

Finally we can implement a correct version of our initial \textit{exactLength} function by computing a proof of equality between both lengths at run-time using \textit{checkEqNat}. This proof can then be used by the compiler to infer that the lengths are indeed equal or not.

\begin{HaskellCode}
exactLength : (len : Nat) -> (input : Vect n k) -> Maybe (Vect len k)
exactLength {n} len input = case checkEqNat n len of
                                 -- len vanishes as compiler can unify len to n
                                 Yes Refl  => Just input 
                                 No contra => Nothing
\end{HaskellCode} 

\subsubsection{Kinds of Equality}
In type theory there are different kinds of equality \footnote{We follow in these definitions mainly \url{https://ncatlab.org/nlab/show/equality}, \url{https://ncatlab.org/nlab/show/intensional+type+theory} and \url{https://ncatlab.org/nlab/show/extensional+type+theory}.}, which in turn depend on the flavour of type theory which can be either \textit{intensional} or \textit{extensional}:

\begin{enumerate}
	\item Definitional or intensional equality: the symbols '2' and 'S(S(Z))' are said to be definitional / intensionally equal terms, because their \textit{intended meaning} is the same.
	\item Computational or judgmental equality: two terms '2 + 2' and '4' are said to be computationally equal because when the result of the addition is computed by a program then they will reduce to the same term 'S(S(Z)) + S(S(Z))' to 'S(S(S(S(Z))))'. In intensional type theory this kind of equality is treated as definitional equality, thus '2 + 2' and '4' are equal by definition.
	\item Propositional equality: when one wants to define general rules that e.g. 'a+b' and 'b+a' are equal, we are talking about a theorem, not a definition. Computational / definitional equality does not work here as to compute it one needs to substitute a and b for concrete natural numbers. In this case we are talking about extensional equality, which is a judgement, not a proposition and thus \textit{not} internal to the formal system itself. It can be internalized through \textit{propositional} equality by adding an identity type which allows to express '2+2 = 4' as a \textit{type}. If such an expression (speak: proof) holds, then this type is inhabited, if not e.g. in the case of '2+2 = 5', this type holds no element and thus no proof exists for it (see section \ref{sub:dep_foundations}).
\end{enumerate}

Still it is not very clear what \textit{intensional} and \textit{extensional} type theory means. The HOTT Book \cite{program_homotopy_2013} says the following in Chapter 1: "Extensional theory makes no distinction between judgmental and propositional equality, the intensional theory regards judgmental equality as purely definitional, and admits a much broader proof-relevant interpretation of the identity type...". This means, that extensional type theory treats objects to be equal if they have the same external properties. In this type of theory, two functions are equal if they give the same results on every input (extensional equality on the function space). Intensional type theory on the other hand allows to distinguish between internal definitions of objects. In this type of theory, two functions are equal if their (internal) definitions are the same.

%Propositional equality allows to assume that a variable x of type p is equal to y: p : x = y.
%Judgemental equality (or definitional equality) means "equal by definition" e.g. if we have a function $f : N -> N by f(x) = x^2$ then f(3) is equal to $3^2$ by definition. Whether or not two expressions are equal by definition is just a matter of expanding out the definitions, in particular it is algorithmically decidable.

Applied to our examples this means the following: We have definitional equality through $(==)$ and $Eq$. Propositional equality is exactly what we got when we introduced the identity type above in the \textit{checkEqNat} function with \textit{Dec (num1 = num2)}. The (=) in the type is built-in into Idris and defines the propositional equality. Dhe Dec type is required to indicate that the proposition may or may not be inhabited. Thus we can also follow that Idris is intensional (and so is Agda and Coq).

\subsection{Philosophical Foundations: Constructivism}
\label{sub:dep_foundations}

The main theoretical and philosophical underpinnings of dependent types as in Idris are the works of Martin-L\"of intuitionistic type theory. The view of dependently typed programs to be proofs is rooted in a deep philosophical discussion on the foundations of mathematics, which revolve around the existence of mathematical objects, with two conflicting positions known as classic vs. constructive \footnote{We follow the excellent introduction on constructive mathematics \cite{thompson_type_1991}, chapter 3.}. In general, the constructive position has been identified with realism and empirical computational content where the classical one with idealism and pragmatism.

In the classical view, the position is that to prove $\exists x. P(x)$ it is sufficient to prove that $\forall x. \neg P(x)$ leads to a contradiction. The constructive view would claim that only the contradiction is established but that a proof of existence has to supply an evidence of an $x$ and show that $P(x)$ is provable. In the end this boils down whether to use proof by contradiction or not, which is sanctioned by the law of the excluded middle which says that $A \lor \neg A$ must hold. The classic position accepts that it does and such proofs of existential statements as above, which follow directly out of the law of the excluded middle, abound in mathematics \footnote{Polynomial of degree n has n complex roots; continuous functions which change sign over a compact real interval have a zero in that interval,...}. The constructive view rejects the law of the excluded middle and thus the position that every statement is seen as true or false, independently of any evidence either way. \cite{thompson_type_1991} (p. 61): \textit{The constructive view of logic concentrates on what it means to prove or to demonstrate convincingly the validity of a statement, rather than concentrating on the abstract truth conditions which constitute the semantic foundation of classical logic}.

To prove a conjunction $A \land B$ we need prove both $A$ and $B$, to prove $A \lor B$ we need to prove one of $A, B$ and know which we have proved. This shows that the law of the excluded middle can not hold in a constructive approach because we have no means of going from a proof to its negation. Implication $A \Rightarrow B$ in constructive position is a transformation of a proof $A$ into a proof $B$: it is a function which transforms proofs of $A$ into proofs of $B$. The constructive approach also forces us to rethink negation, which is now an implication from some proof to an absurd proposition (bottom): $A \Rightarrow \perp$. Thus a negated formula has no computational context and the classical tautology $\neg \neg A \Rightarrow A$ is then obviously no longer valid.  Constructively solving this would require us to be able to effectively compute / decide whether a proposition is true or false - which amounts to solving the halting problem, which is not possible in the general case.

A very important concept in constructivism is that of finitary representation / description. Objects which are infinite e.g. infinite sets as in classic mathematics, fail to have computational computation, they are not computable. This leads to a fundamental tenet in constructive mathematics: \cite{thompson_type_1991} (p. 62): \textit{Every object in constructive mathematics is either finite [..] or has a finitary description}

Concluding, we can say that constructive mathematics is based on principles quite different from classical mathematics, with the idealistic aspects of the latter replaced by a finitary system with computational content. Objects like functions are given by rules, and the validity of an assertion is guaranteed by a proof from which we can extract relevant computational information, rather than on idealist semantic principles. 

All this is directly reflected in dependently typed programs as we introduced above: functions need to be total (finitary) and produce proofs like in \textit{checkEqNat} which allows the compiler to extract additional relevant computational information. Also the way we described the (infinite) natural numbers was in an finitary way. In the case of decidable equality, the case where it is not equal, we need to provide an actual proof of contradiction, with the type of Void which is Idris representation of $\perp$. 

\subsection{Verification, Validation and Dependent Types}
\label{sec:dep_vav_deptypes}
Dependent types allow to encode specifications on an unprecedented level, narrowing the gap between specification and implementation - ideally the code becomes the specification, making it correct-by-construction. The question is ultimately how far we can formulate model specifications in types - how far we can close the gap in the domain of ABS. Unless we cannot close that gap completely, to arrive at a sufficiently confidence in correctness, we still need to test all properties at run-time which we cannot encode at compile-time in types.

Nonetheless, dependent types should allow to substantially reduce the amount of testing which is of immense benefit when testing is costly. Especially in simulations, testing and validating a simulation can often take many hours - thus guaranteeing properties and correctness already at compile time can reduce that bottleneck substantially by reducing the number of test-runs to make.

Ultimately this leads to a very different development process than in the established object-oriented approaches, which follow a test-driven process. There one defines the necessary interface of an object with empty implementations for a given use-case first, then writes tests which cover all possible cases for the given use-case. Obviously all tests should fail because the functionality behind it was not implemented yet. Then one starts to implement the functionality behind it  step-by-step until no test-case fails. This means that one runs all tests repeatedly to both check if the test-case one is working on is not failing anymore and to make sure that old test-cases are not broken by new code. The resulting software is then trusted to be correct because no counter examples through test hypotheses, could be found. The problem is: we could forget / not think of cases, which is the easier the more complex the software becomes (and simulations are quite complex beasts). Thus in the end this is a deductive approach.

With pure functional programming and dependent types the process is now mostly constructive, type-driven (see \cite{brady_type-driven_2017}). In that approach one defines types first and is then guided by these types and the compiler in an interactive fashion towards a correct implementation, ensured at compile-time. As already noted, the ABS methodology is constructive in nature but the established object-oriented test-driven implementation approach not as much, creating an impedance mismatch. We expect that a type-driven approach using dependent types reduces that mismatch by a substantial amount.

Note that \textit{validation} is a different matter here: independent of our implementation approach we still need to validate the simulation against the real-world / ground-truth. This obviously requires to run the full simulation which could take up hours in either programming paradigm, making them absolutely equal in this respect. Also the comparison of the output to the real-world / ground-truth is completely independent to the paradigm. The fundamental difference happens in case of changes made to the code during validation: in case of the established test-driven object-oriented approach for every minor change one (should) re-run all tests, which could take up a substantial amount of additional time. Using a constructive, type-driven approach this is dramatically reduced and can often be completely omitted because the correctness of the change can be either guaranteed in the type or by informally reasoning about the code.

%-------------------------
%TODO: not sure where to put this
%ABS as a constructive / generative science, follows Poperian approach of falsification: we try to construct a model which explains a real-world (empirical) phenomenon - if validation shows that the generated dynamics match the ones of the real-world sufficiently enough, we say that we have found \textit{a} hypothesis (the model) which emergent properties explains the real-world phenomenon sufficiently enough. This is not a proof but only one possible explanation which holds for now and might be falsified in the future.
%
%When we implement our simulation things change a bit as we add another layer: the conceptual model, describing the phenomenon, which is an abstraction of reality. This description can be of many forms but can be regarded on a line between completely formal (economic models) to informal (sociology) but the implementation will follow that description. The fundamental difference here is that in this case we want our implementation to be exactly the same as the conceptual model. Contrary to the real-world, where it is not possible to find a \textit{true} model (as was argued by Popper), on this level we actually can construct an implementation which matches the conceptual model exactly because we have a description of the conceptual model. In the end we transform the conceptual model description in code, which is itself a formal description. In this translation process (speak: implementation / programming), one can make an endless number of mistakes. Generally we can distinguish between two classes of mistakes: 
%1) conceptual mistakes - wrong translation of the model specifications into code due to various reasons e.g. imprecise description, human error. The more precise an unambiguous a model description is, the less probable conceptual mistakes will be.
%2) internal mistakes - normal programming mistakes e.g. access of arrays out of bounds, ... also using correlated Random Number generators.
%
%Level 0: Real-World phenomenon
%Level 1: Conceptual model of the real-world phenomenon
%Level 2: Implementation of the conceptual model
%
%Note that we must speak of falsification and constructiveness on two different levels:
%- validation level: do the results of the conceptual model match the real-world phenomenon? the conceptual model is the hypothesis which says that its mechanics are sufficient to generate / construct the real-world phenomenon. At this level we are not interested in the implementation level anymore - the implemented model \textit{is} (seen as) the conceptual model, and one only compares its output to the real-world. If the dynamics match, then we got a valid hypothesis which works for now. If the dynamics do NOT match, then the hypothesis (the model) is falsified and one needs to adjust / change the hypothesis (model). The validation will happen by tests, there is no other way, we have no formal specification of the real-world, we can only observe empirically the phenomena, so we run tests which try to falsify the outputs of the model: assuming it will generate phenomena of the real-world and test if it does.
%- implementation \& verficiation level: in this step we are matching the code to the conceptual model. Here we are not only restricted to a test-driven approach because we have a more or less formal description of the conceptual model which we directly encode in our programming language. If the language allows to express model specifications already at compile-time then this means that the implementation narrows the gap between model specification and implementation which means it does not need to be tested at run-time because it is guaranteed for all inputs for all time. 
%
%The constructiveness of ABS and impendance mismatch: ABS methodology is constructive but the established implementation approach not too much, creating an impedance mismatch. this is especially visible in the test-driven development dependent types constructive nature could close this mismatch.
%

%todo: connection between black-box verification and dependent types
%todo: connection between white-box verification and dependent types

\chapter{Pure Functional ABS}
"how can we do ABS in functional programming?" art of iterating paper for basic foundations, conceptual paper for a high level description, pure functional epidemics for more details, sync communication in FRP for the trickiest part
50\%

\chapter{Concurrency}
- concurrency and parallelism
	-> STM vs. lock based
	-> pure parallelism in ABS


STM paper
50\%

\section{Verification in ABS}
General there are the following basic verification \& validation requirements to ABS \cite{robinson_simulation:_2014}, which all can be addressed in our \textit{pure} functional approach as described in the paper in Appendix \ref{app:pfe}:

\begin{itemize}
	%\item Modelling progress of time - achieved using functional reactive programming (FRP)
	%\item Modelling variability - achieved using FRP
	\item Fixing random number streams to allow simulations to be repeated under same conditions - ensured by \textit{pure} functional programming and Random Monads
	\item Rely only on past - guaranteed with \textit{Arrowized} FRP
	\item Bugs due to implicitly mutable state - reduced using pure functional programming
	\item Ruling out external sources of non-determinism / randomness - ensured by \textit{pure} functional programming
	\item Deterministic time-delta - ensured by \textit{pure} functional programming
	\item Repeated runs lead to same dynamics - ensured by \textit{pure} functional programming
\end{itemize}

\begin{enumerate}
	\item Run-Time robustness by compile-time guarantees - by expressing stronger guarantees already at compile-time we can restrict the classes of bugs which occur at run-time by a substantial amount due to Haskell's strong and static type system.  This implies the lack of dynamic types and dynamic casts \footnote{Note that there exist casts between different numerical types but they are all safe and can never lead to errors at run-time.} which removes a substantial source of bugs.  Note that we can still have run-time bugs in Haskell when our functions are partial.
	\item Purity - By being explicit and polymorphic in the types about side-effects and the ability to handle side-effects explicitly in a controlled way allows to rule out non-deterministic side-effects which guarantees reproducibility due to guaranteed same initial conditions and deterministic computation. Also by being explicit about side-effects e.g. Random-Numbers and State makes it easier to verify and test.
	\item Explicit Data-Flow and Immutable Data - All data must be explicitly passed to functions thus we can rule out implicit data-dependencies because we are excluding IO. This makes reasoning of data-dependencies and data-flow much easier as compared to traditional object-oriented approaches which utilize pointers or references.
	\item Declarative - describing \textit{what} a system is, instead of \textit{how} (imperative) it works. In this way it should be easier to reason about a system and its (expected) behaviour because it is more natural to reason about the behaviour of a system instead of thinking of abstract operational details.
	\item Concurrency and parallelism - due to its pure and 'stateless' nature, functional programming is extremely well suited for massively large-scale applications as it allows adding parallelism without any side-effects and provides very powerful and convenient facilities for concurrent programming. The paper of (TODO: cite my own paper on STM) explores the use Haskell for concurrent and parallel ABS in a deeper way.
\end{enumerate}

TODO: haskell-titan
TODO: Testing and Debugging Functional Reactive Programming \cite{perez_testing_2017}

Static type system eliminates a large number run-time bugs.

TODO: can we apply equational reasoning? Can we (informally) reason about various properties e.g. termination?

Follow unit-testing of the whole simulation as prototyped for towards paper.

in this we explore something new: property-based testing in ABS

\chapter{Dependent Types}
\label{ch:depTypes}

%My work is all nice and good but it solves problems the ABS community and implementations never really had. My FRP/MSF approach is quite complex and can be equally difficult to get right. Even worse, the bugs were not primarily those I am solving with FP but the REAL problem in ABS is translating the model into code. Can FP help us here? Can my pure FP approach help here? expressing invariants in FP code? can we express them in types? 

The pure functional implementation techniques have a number of technical benefits but don't help as much in closing the gap between specification and implementation as one is used from functional programming in general. Therefore we take a step back and abstract from these highly complex implementation techniques and move towards dependent types. Follow \cite{botta_time_2010} and \cite{botta_functional_2011}.

Conceptually discuss how dependent types can be made of use in ABS without going into lot of technical detail because: 1. i didn't do enough research on it and 2. dependent types seem to be nearly out of focus of the thesis.

%Linear and Dependent Types with Idris 2: more general ideas / hints / research on how it is applicable to ABS

%dependent types in ABS paper, explore totality - equilibrium correspondence idea
%About 20\% finished.

After having established the concepts of dependent types, we want to briefly discuss ideas where and how they could be made of use in ABS. We expect that dependent types will help ruling out even more classes of bugs at compile-time and encode even more invariants. Additionally by constructively implementing model specifications on the type level could allow the ABS community to reason about a model directly in code as it narrows the gap between model specification and implementation.

By definition, ABS is of constructive nature, as described by Epstein \cite{epstein_chapter_2006}: "If you can't grow it, you can't explain it" - thus an agent-based model and the simulated dynamics of it is itself a constructive proof which explain a real-world phenomenon sufficiently well. Although Epstein certainly wasn't talking about a constructive proof in any mathematical sense in this context (he was using the word \textit{generative}), dependent types \textit{might} be a perfect match and correspondence between the constructive nature of ABS and programs as proofs.

When we talk about dependently typed programs to be proofs, then we also must attribute the same to dependently typed agent-based simulations, which are then constructive proofs as well. The question is then: a constructive proof of what? It is not entirely clear \textit{what we are proving} when we are constructing dependently typed agent-based simulations. Probably the answer might be that a dependently typed agent-based simulation is then indeed a constructive proof in a mathematical sense, explaining a real-world phenomenon sufficiently well - we have closed the gap between a rather informal constructivism as mentioned above when citing Epstein who certainly didn't mean it in a constructive mathematical sense, and a formal constructivism, made possible by the use of dependent types.

In the following subsections we will discuss related work in this field (\ref{sub:dep_abs_relwork}), discuss general concepts where dependent types might be of benefit in ABS (\ref{sub:dep_abs_generalconcepts}), present a dependently typed implementation of a 2D discrete environment (\ref{sub:dep_abs_2denv}) and finally discuss potential use of dependent types in the SIR model (\ref{sub:dep_abs_sir}) and SugarScape model (\ref{sub:dep_abs_sugarscape}).

\subsection{Related Work}
\label{sub:dep_abs_relwork}
In \cite{botta_functional_2011} the authors are using functional programming as a specification for an agent-based model of exchange markets but leave the implementation for further research where they claim that it requires dependent types. This paper is the closest usage of dependent types in agent-based simulation we could find in the existing literature and to our best knowledge there exists no work on general concepts of implementing pure functional agent-based simulations with dependent types. As a remedy to having no related work to build on, we looked into works which apply dependent types to solve real world problems from which we then can draw inspiration from. 

The paper \cite{brady_correct-by-construction_2010} discusses depend types to implement correct-by-construction concurrency in the Idris language \cite{brady_idris_2013}. The authors introduce the concept of a Embedded Domain Specific Language (EDSL) for concurrently locking/unlocking and reading/writing of resources and show that an implementation and formalisation are the same thing when using dependent types. We can draw inspiration from it by taking into consideration that we might develop an EDSL in a similar fashion for specifying general commands which agents can execute. The interpreter of such a EDSL can be pure itself and doesn't have to run in the IO Monad as our previous research (see Appendix \ref{app:pfe}) has shown that ABS can be implemented pure.

In \cite{brady_idris_2011} the authors discuss systems programming with focus on network packet parsing with full dependent types in the Idris language \cite{brady_idris_2013}. Although they use an older version of it where a few features are now deprecated, they follow the same approach as in the previous paper of constructing an EDSL and writing an interpreter for the EDSL. In a longer introduction of Idris the authors discuss its ability for termination checking in case that recursive calls have an argument which is structurally smaller than the input argument in the same position and that these arguments belong to a strictly positive data type. We are particularly interested in whether we can implement an agent-based simulation which termination can be checked at compile-time - it is total.

In \cite{brady_programming_2013} the author discusses programming and reasoning with algebraic effects and dependent types in the Idris language \cite{brady_idris_2013}. They claim that monads do not compose very well as monad transformer can quickly become unwieldy when there are lots of effects to manage. As a remedy they propose algebraic effects \cite{bauer_programming_2015} and implement them in Idris and show how dependent types can be used to reason about states in effectful programs. In our previous research (see Appendix \ref{app:pfe}) we relied heavily on Monads and transformer stacks and we indeed also experienced the difficulty when using them. Algebraic effects might be a promising alternative for handling state as the global environment in which the agents live or threading of random-numbers through the simulation which is of fundamental importance in ABS. According to the authors of the paper, unfortunately, algebraic effects cannot express continuations which is but of fundamental importance for pure functional ABS as agents are on the lowest level built on continuations - synchronous agent interactions and time-stepping builds directly on continuations. Thus we need to find a different representation of agents - GADTs seem to be a natural choice as all examples build heavily on them and they are very flexible.

In \cite{fowler_dependent_2014} the authors apply dependent types to achieve safe and secure web programming. This paper shows how to implement dependent effects, which we might draw inspiration from of how to implement agent-interactions which, depending on their kind, are effectful e.g. agent-transactions or events.

In \cite{brady_state_2016} the author introduces the ST library in Idris, which allows a new way of implementing dependently typed state machines and compose them vertically (implementing a state machine in terms of others) and horizontally (using multiple state machines within a function). In addition this approach allows to manage stateful resources e.g. create new ones, delete existing ones. We can draw further inspiration from that approach on how to implement dependently typed state machines, especially composing them hierarchically, which is a common use case in agent-based models where agents behaviour is modelled through hierarchical state-machines. As with the Algebraic Effects, this approach doesn't support continuations, so it is not really an option to build our architecture for our agents on it, but it may be used internally to implement agents or other parts of the system. What we definitely can draw inspiration from is the implementation of the indexed Monad \textit{STrans} which is the main building block for the ST library.

The book \cite{brady_type-driven_2017} is a great source to learn pure functional dependently typed programming and in the advanced chapters introduces the fundamental concepts of dependent state machine and dependently typed concurrent programming on a simpler level than the papers above. One chapter discusses on how to implement a messaging protocol for concurrent programming, something we can draw inspiration from for implementing our synchronous agent interaction protocols.

In \cite{sculthorpe_safe_2009} the authors apply dependent types to FRP to avoid some run-time errors and implement a dependently typed version of the Yampa library in Agda.

The fundamental difference to all these real-world examples is that in our approach, the system evolves over time and agents act over time in a feedback loop. A fundamental question will be how we encode the monotonous increasing flow of time in types and how we can reflect in the types that agents act over time.

%An agent can be seen as a potentially infinite stream of continuations which at some point could return information to stop evaluating the next item of the stream which allows an agent to terminate.
%correspondence between temporal logics and FRP due to jeffery: is abs just another temporal logic?

%The authors of \cite{ionescu_dependently-typed_2012} discuss how to use dependent types to specify fundamental theorems of economics, unfortunately they are not computable and thus not constructive, thus leaving it more to a theoretical, specification side.
%Ionesus talk on dependently typed programming in scientific computing
%https://www.pik-potsdam.de/members/ionescu/cezar-ifl2012-slides.pdf
%Ionescus talk on Increasingly Correct Scientific Computing
%%https://www.cicm-conference.org/2012/slides/CezarIonescu.pdf
%Ionescus talk on Economic Equilibria in Type Theory
%https://www.pik-potsdam.de/members/ionescu/cezar-types11-slides.pdf
%Ionescus talk on Dependently-Typed Programming in Economic Modelling
%https://www.pik-potsdam.de/members/ionescu/ee-tt.pdf

\subsection{General Concepts}
\label{sub:dep_abs_generalconcepts}

We came up with the following ideas of how and where to apply dependent types in the context of agent-based simulation:

%Randomness is of central importance in agent-based simulation but nothing enforces from which distribution to draw. With dependent types we might to implement probabilistic types which can encode probability distributions in types already about which we can then reason and guarantee at compile-time that we draw from the correct distribution.

% encode dynamics in the types (what? feedbacks? positive/negative) on a meta-level

\paragraph{Environment Access}
Accessing e.g. discrete 2D environments involves (almost always) indexed array access which is always potentially dangerous as the indices have to be checked at run-time.

Using dependent types it should be possible to encode the environment dimensions into the types. In combination with suitable data types (finite sets) for coordinates one should be able to ensure already at compile-time that access happens only within the bounds of the environment. We have implemented this already and describe it in detail in the section \ref{sub:dep_abs_2denv}.

\paragraph{State-Machines}
Often, Agent-Based Models define their agents in terms of state-machines. It is easy to make wrong state-transitions e.g. in the SIR model when an infected agent should recover, nothing prevents one from making the transition back to susceptible. 

Using dependent types it might be possible to encode invariants and state-machines on the type level which can prevent such invalid transitions already at compile-time. This would be a huge benefit for ABS because of the popularity of state-machines in agent-based models.

\paragraph{Flow Of Time}
State-Machines often have timed transitions e.g. in the SIR model, an infected agent recovers after a given time. Nothing prevents us from introducing a bug and \textit{never} doing the transition at all.

With dependent types we might be able to encode the passing of time in the types and guarantee on a type level that an infected agent has to recover after a finite number of time steps. Also can dependent types be used to express the flow of time and that it is strongly monotonic increasing?
	
\paragraph{Existence Of Agents}
In more sophisticated models agents interact in more complex ways with each other e.g. through message exchange using agent IDs to identify target agents. The existence of an agent is not guaranteed and depends on the simulation time because agents can be created or terminated at any point during simulation. 

Dependent types could be used to implement agent IDs as a proof that an agent with the given id exists \textit{at the current time-step}. This also implies that such a proof cannot be used in the future, which is prevented by the type system as it is not safe to assume that the agent will still exist in the next step. %So it is a proof of the existence of an agent: holds always only for the current time-step or for all time, depending on the model. e.g. in the SIR model no agents are removed from / added to the system thus a proof holds for all time. 

\paragraph{Agent-Agent Interactions}
Because we are lacking method-calls as in object-oriented programming, we need to come up with different mechanics for agent-agent interaction, which are basically based upon continuations. The main use-case are multi-step interactions which happen without a time-delay e.g trading or resource exchange protocols as described in SugarScape. In these two agents interact over multiple steps, following a given protocol, which is a source of bugs when not following the required steps.

Using dependent types we might be able to encode a protocol for agent-agent interactions which e.g. ensures on the type-level that an agent has to reply to a request or that a more specific protocol has to be followed e.g. in auction- or trading-simulations.

\paragraph{Equilibrium and Totality}
For some agent-based simulations there exists equilibria, which means that from that point the dynamics won't change any more e.g. when a given type of agents vanishes from the simulation or resources are consumed. This means that at that point the dynamics won't change any more, thus one can safely terminate the simulation. Very often, despite such a global termination criterion exists, such simulations are stepped for a fixed number of time-steps or events or the termination criterion is checked at run-time in the feedback-loop. 
	
Using dependent types it might be possible to encode equilibria properties in the types in a way that the simulation automatically terminates when they are reached. This results then in a \textit{total} simulation, creating a \textit{correspondence between the equilibrium of a simulation and the totality of its implementation}. Of course this is only possible for models in which we know about their equilibria a priori or in which we can reason somehow that an equilibrium exists.

A central question in tackling this is whether to follow a model- or an agent-centric approach. The former one looks at the model and its specifications as a whole and encodes them e.g. one tries to directly find a total implementation of an agent-based model. The latter one looks only at the agent level and encodes that as dependently typed as possible and hopes that model guarantees emerge on a meta-level - put otherwise: does the totality of an implementation emerge when we follow an agent-centric approach?

\paragraph{Specifications and properties}
Using dependent types it is possible to encode model specifications and properties directly in types as described above. Other examples are to guarantee that the number of agent stays constant.

\paragraph{Hypotheses}
Models which are exploratory in nature don't have a formal ground truth where one could derive equilibria or dynamics from and validate with. In such models the researchers work with informal hypotheses which they express before running the model and then compare them informally against the resulting dynamics.

It would be of interest if dependent types could be made of use in encoding hypotheses on a more constructive and formal level directly into the implementation code. So far we have no idea how this could be done but it might be a very interesting application as it allows for a more formal and automatic testable approach to hypothesis checking.

\subsection{Dependently Typed Discrete 2D Environment}
\label{sub:dep_abs_2denv}
One of the main advantages of Agent-Based Simulation over other simulation methods e.g. System Dynamics is that agents can live within an environment. Many agent-based models place their agents within a 2D discrete NxM environment where agents either stay always on the same cell or can move freely within the environment where a cell has 0, 1 or many occupants. Ultimately this boils down to accessing a NxM matrix represented by arrays or a similar data structure. In imperative languages accessing memory always implies the danger of out-of-bounds exceptions \textit{at run-time}. With dependent types we can represent such a 2D environment using vectors which carry their length in the type (see \ref{sec:dep_background}) thus fixing the dimensions of such a 2D discrete environment in the types. This means that there is no need to drag those bounds around explicitly as data. Also by using dependent types like a finite set Fin, which depend on the dimensions we can enforce at compile-time that we can only access the data structure within bounds. If we want to we can also enforce in the types that the environment will never be an empty one where $N, M > 0$.

\begin{HaskellCode}
-- an environment has width w and height h and cells e and is never empty
-- adding Successor S to each dimension ensures that the environment is not empty
Disc2dEnv : (w : Nat) -> (h : Nat) -> (e : Type) -> Type
Disc2dEnv w h e = Vect (S w) (Vect (S h) e) 

-- the coordinates for an environment are respresented by the (Fin k) datatype
-- which represents the natural numbers as a finite set from  0..k
-- need an additional S for ensuring that our bounds are strictly less than
data Disc2dCoords : (w : Nat) -> (h : Nat) -> Type where
  MkDisc2dCoords : Fin (S w) -> Fin (S h) -> Disc2dCoords w h
  
centreCoords : Disc2dEnv w h e -> Disc2dCoords w h
centreCoords {w} {h} _ =
    let x = halfNatToFin w
        y = halfNatToFin h
    in  mkDisc2dCoords x y
  where
    halfNatToFin : (x : Nat) -> Fin (S x)
    halfNatToFin x = 
      let xh   = divNatNZ x 2 SIsNotZ 
          mfin = natToFin xh (S x)
      in  fromMaybe FZ mfin
      
-- overriding the content of a cell: no boundary checks necessary
setCell :  Disc2dCoords w h
        -> (elem : e)
        -> Disc2dEnv w h e
        -> Disc2dEnv w h e
setCell (MkDisc2dCoords colIdx rowIdx) elem env 
    = updateAt colIdx (\col => updateAt rowIdx (const elem) col) env
 
-- reading the content of a cell: no boundary checks necessary
getCell :  Disc2dCoords w h
        -> Disc2dEnv w h e
        -> e
getCell (MkDisc2dCoords colIdx rowIdx) env
    = index rowIdx (index colIdx env)
    
neumann : Vect 4 (Integer, Integer)
neumann = [         (0,  1), 
           (-1,  0),         (1,  0),
                    (0, -1)]

moore : Vect 8 (Integer, Integer)
moore = [(-1,  1), (0,  1), (1,  1),
         (-1,  0),          (1,  0),
         (-1, -1), (0, -1), (1, -1)]

filterNeighbourhood :  Disc2dCoords w h
                    -> Vect len (Integer, Integer)
                    -> Disc2dEnv w h e 
                    -> (n ** Vect n (Disc2dCoords w h, e))
filterNeighbourhood {w} {h} (MkDisc2dCoords x y) ns env =
    let xi = finToInteger x
        yi = finToInteger y
    in  filterNeighbourhood' xi yi ns env
  where
    filterNeighbourhood' :  (xi : Integer)
                         -> (yi : Integer)
                         -> Vect len (Integer, Integer)
                         -> Disc2dEnv w h e 
                         -> (n ** Vect n (Disc2dCoords w h, e))
    filterNeighbourhood' _ _ [] env = (0 ** [])
    filterNeighbourhood' xi yi ((xDelta, yDelta) :: cs) env 
      = let xd = xi - xDelta
            yd = yi - yDelta
            mx = integerToFin xd (S w)
            my = integerToFin yd (S h)
        in case mx of
            Nothing => filterNeighbourhood' xi yi cs env 
            Just x  => (case my of 
                        Nothing => filterNeighbourhood' xi yi cs env 
                        Just y  => let coord      = MkDisc2dCoords x y
                                       c          = getCell coord env
                                       (_ ** ret) = filterNeighbourhood' xi yi cs env
                                   in  (_ ** ((coord, c) :: ret)))
\end{HaskellCode}

\subsection{Dependently Typed SIR}
\label{sub:dep_abs_sir}
We plan to prototype the concepts of section \ref{sub:dep_abs_generalconcepts} in a dependently typed SIR implementation. One can object that the SIR model \cite{kermack_contribution_1927} is a very simple model but despite its simplicity it has a number of advantages. There is a theory behind it with a formal ground-truth for the dynamics which can be generated by differential equations, which allows validation of the simulation. Also, it has already many concepts of ABS in it without making it too complex: agent-behaviour as a state-machine, local agent-state (current SIR state and duration of illness), feedback, very rudimentary interaction with other agents, 2D environment if required and behaviour over time. We will also look into the SugarScape model (see \ref{sub:dep_abs_sugarscape}), which is of quite a different type and adds more complexity.

The general approach of using dependent types is to specify the general commands available for an agent, where we can follow the approach of an EDSL as described in \cite{brady_correct-by-construction_2010} and write then an interpreter for it. It is of importance that the interpreter shall be pure itself and does not make use of any IO. Applying dependent types to the SIR model, we came up with the following use-cases:

\paragraph{Environment access}
We have already introduced an implementation for a dependently typed 2D environment in section \ref{sub:dep_abs_2denv}. This can be directly used to implement a SIR on a 2D environment as we have done in the paper in Appendix \ref{app:pfe}.

\paragraph{State-Machine and Flow Of Time}
The transition through the Susceptible, Infected and Recovered states are a state-machine, thus we want to apply dependent types to restrict the valid transitions and ensure that they are enforced under the given circumstances. The transitions are restricted to: Susceptibles can only transition to Infected, Infected only to Recovered and Recovered stay in that state forever. A transition from Susceptible to Infected happens with a given probability in case the Susceptible makes contact with an Infected. The transition from Infected to Recover happens after a given number of time-steps.

The tricky thing is that all these transitions ultimately depend on stochastic events: Susceptible pick their contacts at random, uniformly distributed from all agents in the simulation, they get infected with a probability when the contact is Infected and the duration an Infected agent is ill is picked from an exponential distribution.

\paragraph{Equilibrium and totality}
The idea is to implement a total agent-based SIR simulation, where the termination does NOT depend on time (is not terminated after a finite number of time-steps, which would be trivial).  We argue that the underlying SIR model actually has a steady state.

The dynamics of the System Dynamics SIR model are in equilibrium (won't change any more) when the infected stock is 0. This might be shown formally but intuitively it is clear because only infected agents can lead to infections of susceptible agents which then make the transition to recovered after having gone through the infection phase. 

Thus an agent-based implementation of the SIR simulation has to terminate if it is implemented correctly because all infected agents will recover after a finite number of steps after then the dynamics will be in equilibrium. Thus we have the following conditions for totality:
\begin{enumerate}
	\item The simulation shall terminated when there are no more infected agents.
	\item All infected agents will recover after a finite number of time, which means that the simulation will eventually run out of infected agents. 
	
	Unfortunately this criterion alone does not suffice because when we look at the SIR+S model, which adds a cycle from Recovered back to Susceptible, we have the same termination criterion, but we cannot guarantee that it will run out of infected. We need an additional criteria.
	\item The source of infected agents is the pool of susceptible agents which is monotonous decreasing (not strictly though!) because recovered agents do NOT turn back into susceptibles.
\end{enumerate}

Thus we can conclude that a SIR model must enter a steady state after finite steps / in finite time. %\footnote{Note that there exists a SIR+S model, which adds a cycle back from Recovered to Susceptible - if we find a total implementation of the SIR model and add this transition then the simulation should become non-total, checked by the compiler.}.

By this reasoning, a non-total, correctly implemented agent-based simulations of the SIR model will eventually terminate (note that this is independent of which environment is used and which parameters are selected). Still this does not formally proof that the agent-based approach itself will terminate and so far no formal proof of the totality of it was given.

Dependent Types and Idris' ability for totality- and termination-checking should theoretically allow us to proof that an agent-based SIR implementation terminates after finite time: if an implementation of the agent-based SIR model in Idris is total it is a formal proof by construction. Note that such an implementation should not run for a limited virtual time but run unrestricted of the time and the simulation should terminate as soon as there are no more infected agents, returning the termination time as an output. Also if we find a total implementation of the SIR model and extend it to the SIR+S model, which adds a cycle from Recovered back to Susceptible, then the simulation should become again non-total as reasoned above.

The HOTT book \cite{program_homotopy_2013} states that lists, trees,... are inductive types/inductively defined structures where each of them is characterized by a corresponding \textit{induction principle}. Thus, for a constructive proof of the totality of the agent-based SIR model we need to find the induction principle of it. This leaves us with the question of what the inductive, defining structure of the agent-based SIR model is? Is it a tree where a path through the tree is one way through the simulation or is it something else? It seems that such a tree would grow and then shrink again e.g. infected agents. Can we then apply this further to (agent-based) simulation in general?

%TODO: \url{https://stackoverflow.com/questions/19642921/assisting-agdas-termination-checker/39591118}

%We hypothesize that it should be possible due to the nature of the state transitions where there are no cycles and that all infected agents will eventually reach the recovered state. 
%
%-- TODO: express in the types
%-- SUSCEPTIBLE: MAY become infected when making contact with another agent
%-- INFECTED:    WILL recover after a finite number of time-steps
%-- RECOVERED:   STAYS recovered all the time
%
%-- SIMULATION:  advanced in steps, time represented as Nat, as real numbers are not constructive and we want to be total
%--              terminates when there are no more INFECTED agents

So far we have no clear idea and understanding how to implement such a total implementation - this will be subject to quite substantial research. One might object to that undertaking and ask what we gain from it. We argue that investigating the correspondence between the equilibrium of an agent-based model and the totality of its implementation for the first time is reason enough because we expect to gain new insights from this undertaking.

\paragraph{Specifications}
The number of agents stays constant in SIR, this means no agents are created / destroyed during simulation, they only might change their state. We could conceptually specify that in the types as:
\begin{HaskellCode}
sirAgentNumberConstant : Vect s (SIRAgent Susceptible) -> 
                         Vect i (SIRAgent Infected) ->
                         Vect r (SIRAgent Recovered) -> 
                         Vect (s + i + r) (SIRAgent st)
\end{HaskellCode}

Another property of the SIR model is, that the number of susceptibles, infected and recovered might change in each step but the sum will be the same as before. We could conceptually specify that in the types as:
\begin{HaskellCode}
sirStep : Vect s (SIRAgent Susceptible) -> 
          Vect i (SIRAgent Infected) ->
          Vect r (SIRAgent Recovered) -> 
          (Vect s' (SIRAgent Susceptible),
           Vect i' (SIRAgent Infected), 
           Vect r' (SIRAgent Recovered), (s'+i'+r') = (s+i+r))
\end{HaskellCode}

\subsection{Dependently Typed Sugarscape}
\label{sub:dep_abs_sugarscape}
The other model we will employ as a use-case for the concepts of section \ref{sub:dep_abs_generalconcepts} is the SugarScape model \cite{epstein_growing_1996}. It is an exploratory model by which social scientists tried to explain phenomena found in societies in the real world. The main complexity of this model lies in the much more complex local state of the agents and the agent-agent interactions e.g. in case of trade and mating and a pro-active environment. Opposed to the SIR model agents behaviour is not modelled as a state-machine and time-semantics is not of that much importance: the simulation is stepped in unit-steps of $\Delta t = 1.0$ and in every time-step, all agents act in random order. Although there are equilibria e.g. in case all agents die out or the carrying capacity of an environment, trading prices, we think that this model is too complex for a total implementation in the cases.

\paragraph{Environment}
We have already introduced an implementation for a dependently typed 2D environment in section \ref{sub:dep_abs_2denv}. This can be directly used to implement the pro-active environment of SugarScape.

\paragraph{Existence Of Agents}
In SugarScape agents can die and be born thus on a technical level agents are added and removed dynamically during the simulation. This means we can employ proofs of existence of an agent for establishing interactions with another one. Also a proof might become invalid after a time. Also one can construct a proof only from a given time on e.g. when one wants to prove that agent X exists but agent X is only created at time t then before time t the prove cannot be constructed and is uninhabited and only inhabited from time t on.

\paragraph{Agent-Agent interactions}
In SugarScape agents interact with each other on a much more complex way than in SIR due to the complex behaviour. The two main complex use-cases are mating and trading between agents where both require multiple interaction-steps happening instantaneous without delay (that is, within 1 time-step). Both use-cases implement a protocol which we might be able to enforce using dependent types.

\paragraph{Hypotheses}
SugarScape is an exploratory model and although it is based on theoretical concepts from sociology, economics and epidemiology, it has strictly speaking no analytical or theoretical ground truth. Thus there are no means to validate this model and the researcher works by formulating hypotheses about the emergent properties of the model. So the approach the creators of SugarScape took in \cite{epstein_growing_1996} was that they started from real world phenomenon and modelled the agent-interactions for them and hypothesized that out of this the real-world phenomenon will emerge. An example is the carrying capacity of an environment, as described in the first chapter: they hypothesized that the size of the population will reach a state where it will fluctuate around some mean because the environment cannot sustain more than a given number so agents not finding enough resources will simply die. Maybe we can encode such hypotheses using dependent types.

\section{Case-Study}
perform gintis case-study: apply my developed techniques of implementing ABS and testing and concurrency / parallelism to the gintis paper (and its follow ups: the ionescu paper and the masterthesis on it). 
The aim of this is to see: 
1. we can do DES under the hood, 2. property-based tests in a different setting, 3. could pure functional programming have prevented the failure? 4. could property based tests have prevented the failure? 5. could dependent and / or types have prevented the failure? 

1. do the techniques transfer to this problem? 
2. does haskell prevent making that mistake which gintis made? 
3. how close is our implementation to ionescus functional specification? 
4. validation and verification against gintis paper using property-based testing of individual agents and the simulation as a whole. 
5. being more familiar with dependent types, would they help and where do they fit in in combination with the ionescu functional specification, which mentions dependent types at the end.

\section{Discussion}

\subsection{Other Models}
TODO: mention that we have also implemented other models, which also work without time-semantics (all agents make a move at discrete time-steps and do not really rely on some notion of time). 

\subsection{Time-Semantics}
The main reason for building our pure functional ABMS approach on top of Yampa was to leverage the powerful time-semantics of Yampa which allows us to implement important concepts of ABMS:

state-chart: agents are at all time of their life-cycle in one state and can switch between multiple states using transitions 
timed transitions: transition to another state/behaviour happens at a discrete time
rate transitions: transition happens with a given rate
message transition: transition upon receiving a given message 

\subsection{Agents as Signals}
Due to the underlying nature and motivation of Functional Reactive Programming (und im speziellen) Yampa, Agents can be seen as Signals which is generated and consumed by a Signal-Function which is the behaviour of an Agent. If an Agent does not change the OUTPUT-signal is constant, if the agent changes e.g. by sending a message, changing its state,... the OUTPUT signal changes. A dead agent has no signal at all.

\subsection{Time-Sampling}
sampling rate depends on the transition times \& rates of the model. when e.g. the contact rate is 5 then the sampling dt should be below 0.2

\subsection{System Dynamics}
can emulate system dynamics due to the parallel update-strategy and continuous time-flow semantics

\subsection{Discrete Event Simulation}
DES in FrABMS? how easily can we implement server/queue systems? do they also look like a specification? potential problem: ordering of messages is not guaranteed by now

\subsection{Advantages}
advantages:
	- no side-effects within agents leads to much safer code
	- edsl for time-semantics
	- declarative style: agent-implementation looks like a model-specification
	- reasoning and verification
	- sequential and parallel
	- powerful time-semantics
	- arrowized programming is optional and only required when utilizing yampas time-semantics. if the model does not rely on time-semantics, it can use monadic-programming by building on the existing monadic functions in the EDSL which allow to run in the State-Monad which simplifies things very much
	- when to use yampas arrowized programing: time-semantics, simple state-chart agents 
	- when not using yampas facilities: in all the other cases e.g. SugarScape is such a case as it proceeds in unit time-steps and all agents act in every time-step
	- can implement System Dynamics building on Yampas facilities with total ease	
	- get replications for free without having to worry about side-effects and can even run them in parallel without headaches
	- cant mess around with time because delta-time is hidden from you (intentional design-decision by Yampa). this would be only very difficult and cumbersome to achieve in an object-oriented approach. TODO: experiment with it in Java - how could we actually implement this? I think it is impossible: may only achieve this through complicated application of patterns and inheritance but then has the problem of how to update the dt and more important how to deal with functions like integral which accumulates a value through closures and continuations. We could do this in OO by having a general base-class e.g. ContinuousTime which provides functions like updateDt and integrate, but we could only accumulate a single integral value.
	- reproducibility statically guaranteed
	- cannot mess around with dt
	- code == specification
	- rule out serious class of bugs
	- different time-sampling leads to different results e.g. in wildfire \& SIR but not in Prisoners Dilemma. why? probabilistic time-sampling?
	- reasoning about equivalence between SD and ABS implementation in the same framework
	- recursive implementations
	
	- we can statically guarantee the reproducibility of the simulation because: no side effects possible within the agents which would result in differences between same runs (e.g. file access, networking, threading), also timedeltas are fixed and do not depend on rendering performance or userinput	
	
\subsection{Disadvantages}
disadvantages:
	- performance is low
	- reasoning about performance is very difficult
	- very steep learning curve for non-functional programmers
	- learning a new EDSL
	- think ABMS different: when to use async messages, when to use sync conversations


[ ] important: increasing sampling freqzency and increasing number of steps so that the same number of simulation steps are executed should lead to same results. but it doesnt. why?
[ ] hypothesis: if time-semantics are involved then event ordering becomes relevant for emergent patterns. there are no tine semantics in heroes and cowards but in the prisoners dilemma
[ ] can we implement different types of agents interacting with each other in the same simulation ? with different behaviour funcs, digferent state? yes, also not possible in NetLogo to my knowledge. but they must have the same messages, emvironment 

[ ] Hypothesis: we can combine with FrABS agent-based simulation and system dynamics (this has been proved by example!)

\chapter{Conclusions}
\label{ch:conclusions}

This chapter concludes the whole thesis and outlines future research. Roughly 20\% exists already.

%we now know how to engineer time- and event-driven ABS with complex state both in the agent and environment, main difficulty is direct agent-interaction (see macal classification into 4 types of ABS), compile-time guaranteed reproducibility, explicit handling of complex state (read only, read/write), concurrency explicit and limited to STM, very promising concurrency but direct agent-interactions main problem (erlang as a rescue?), main drawbacks: everything is explicit, performance

\section{Further Research}
clearly outline the ideas for further research

\subsection{A general purpose library}
generalise concepts explored into a pure functional ABS library in Haskell (called chimera)

\subsection{Dependent and linear types}
dependent types and linear types are the next big step, towards a stronger formalisation of agents and ABS,
focus on the equilibrium - totality correspondence

\subsection{Concurrent event-driven ABS}
stm based concurrency for event-driven ABS using parallel DES. challenge is the time-warp implementation using monads. in general it should be easy to roll-back agents actions but with monads we have to be careful - for some monads rolling back is not neccessary e.g. rand and reader, for others it is, and for some it is impossible e.g. IO

\renewcommand\bibname{References}

\bibliographystyle{acm}
\bibliography{../../writing/references/phdReferences}

\begin{appendices}

TODO: add full code of SIR implementation

\chapter{Validating Sugarscape in Haskell}
\label{app:validating_sugarscape}

In this chapter we look at how property-based testing can be made of use to verify the \textit{exploratory} Sugarscape model \cite{epstein_growing_1996} as introduced in Chapter \ref{sec:sugarscape}. Whereas in the chapters on testing the explanatory SIR model we had an analytical solution, the fundamental difference in the exploratory Sugarscape model is that none such analytical solutions exist. This raises the question, which properties we can actually test in such a model.

The answer lies in the very nature of exploratory models, they exist to explore and understand phenomena of the real world. Researchers come up with a model to explain the phenomena and (hopefully) with a few questions and \textit{hypotheses} about the emergent properties. The actual simulation is then used to test and refine the hypotheses. Indeed, descriptions, assumptions and hypotheses of varying formal degree abound in the Sugarscape model. Examples are: \textit{the carrying capacity becomes stable after 100 steps; when agents trade with each other, after 1000 steps the standard deviation of trading prices is less than 0.05; when there are cultures, after 2700 steps either one culture dominates the other or both are equally present}. 

We show how to use property-based testing to formalise and check such hypotheses. For this purpose we undertook a full \textit{verification} of our \href{https://github.com/thalerjonathan/haskell-sugarscape}{implementation}~\cite{thaler_sugarscape_repository} from Chapter \ref{sec:sugarscape}. We validated it against the book \cite{epstein_growing_1996} and a NetLogo implementation \cite{weaver_replicating_2009}  \footnote{Lending didn't properly work in their NetLogo code and that they didn't implement Combat.}.

\section{Property-based hypothesis testing}
The property we test for is whether \textit{the emergent property / hypothesis under test is stable under replicated runs} or not. To put it more technical, we use QuickCheck to run multiple replications with the same configuration but with different random-number streams and require that all tests pass. During the verification process we have derived and implemented property tests for the following hypotheses:

\begin{enumerate}
	\item Disease dynamics where all agents recover - when disease are turned on, if the number of initial diseases is 10, then the population is  able to rid itself completely from all disease within 100 ticks. 
	
	\item Disease dynamics where a minority recovers - when disease are turned on, if the number of initial diseases is 25, the population is not able to rid itself completely from all diseases within 1,000 ticks.
	
	\item Trading dynamics - when trading is enabled, the trading prices stabilise after 1,000 ticks with the standard deviation of the prices having dropped below 0.05.
	
	\item Cultural dynamics - when having two cultures, red and blue, after 2,700 ticks, either the red or the blue culture dominates or both are equally strong. If they dominate they make up 95\% of all agents, if they are equally strong they are both within 45\% - 55\%.
	
	\item Inheritance Gini coefficient - when agents reproduce and can die of age then inheritance of their wealth leads to an unequal wealth distribution measured using the Gini Coefficient \textit{averaging} at 0.7.
	
	\item Carrying capacity - when agents don't mate nor can die from age, due to the environment, there is an \textit{average} maximum carrying capacity of agents the environment can sustain. The capacity should be reached after 100 ticks and should be stable from then on.
		
	\item Terracing - when resources regrow immediately, after a few steps the simulation becomes static. Agents will stay on their terraces and will not move any more because they have found the best spot due to their behaviour. About 45\% will be on terraces and 95\% - 100\% are static, not moving any more.
\end{enumerate}

The hypotheses and their validation is described more in-depth in the section \ref{sec:hypotheses_testcases} below.

\subsection{Implementation}
To start with, we implement a custom data generator to produce output from a Sugarscape simulation. The generator takes the number of ticks and the scenario with which to run the simulation and returns a list of outputs, one for each tick.

\begin{HaskellCode}
sugarscapeUntil :: Int                -- ^ Number of ticks to run
                -> SugarScapeScenario -- ^ Scenario to run
                -> Gen [SimStepOut]   -- ^ Output of each step
sugarscapeUntil ticks params = do
  -- create a random-number generator
  g <- genStdGen
  -- initialise the simulation state with the given random-number generator
  -- and the scenario
  let simState = initSimulationRng g params
  -- run the simulation with the given state for number of ticks
  return (simulateUntil ticks simState)
\end{HaskellCode}

Using this generator, we can very conveniently produce Sugarscape data within a QuickCheck \texttt{Property}. Depending on the problem, we can generate only a single run or multiple replications, in case the hypothesis is assuming \textit{averages}. To see its use, we show the implementation of the \textit{Disease Dynamics (1)} hypothesis.

\begin{HaskellCode}
prop_disease_allrecover :: Property
prop_disease_allrecover = property (do
  -- after 100 ticks...
  let ticks = 100
  -- ... given Animation V-1 parameter configuration ...
  let params = mkParamsAnimationV_1
  -- ... from 1 sugarscape simulation ...
  aos <- last <*> (sugarscapeUntil ticks params)
  -- ... counting all infected agents ...
  let infected = length (filter (==False)) map (null . sugObsDiseases . snd) aos
  -- ... should result in all agents to be recovered
  return (cover 100 (infected == 0) "Diseases all recover" True)
\end{HaskellCode}

From the implementation it becomes clear, that this hypothesis states that the property has to hold \textit{for all} replications. The \textit{Inheritance Gini Coefficient (5)} hypothesis on the other hand assumes that the Gini Coefficient \textit{averages} at 0.7. We cannot average over replicated runs of the same property thus we generate multiple replications of the Sugarscape data within the property and employ a two-sided t-test with a 95\% confidence to test the hypothesis:

\begin{HaskellCode}
prop_gini :: Int      -- ^ Number of replications
          -> Double   -- ^ Confidence of the t-test
          -> Property
prop_gini repls confidence = property (do
  -- after 1000 ticks...
  let ticks = 1000
  -- ... the gini coefficient should average at 0.7 ...
  let expGini = 0.7
  -- ... given the Figure III-7 parameter configuration ...
  let params = mkParamsFigureIII_7
  -- ... from 100 replications ... 
  gini <- vectorOf repls (genGiniCoeff ticks params)
  -- on a two-tailed t-test with given confidence
  return (tTestSamples TwoTail expGini (1 - confidence) gini)
\end{HaskellCode}

%genGiniCoeff :: Int -> SugarScapeScenario -> Gen Double
%genGiniCoeff ticks params = do
%  -- generate sugarscape data
%  aos <- sugarscapeUntil ticks params
%  -- extract wealth of the agents in the last step
%  let agentWealths = map (sugObsSugLvl . snd) (last aos)
%  -- compute gini coefficient and return it
%  return (giniCoeff agentWealths)

\subsection{Running the tests}
As already pointed out in Part \ref{ch:property}, QuickCheck by default runs up to 100 test cases of a property and if all evaluate to \texttt{True} the property test succeeds. On the other hand, QuickCheck will stop at the first test case which evaluates to \texttt{False} and marks the whole property test as failed, no matter how many test cases got through already. For this reason we have used \texttt{cover} with an expected percentage of 100, meaning that we expect all tests to fall into the coverage class. This allows us to emulate failure with QuickCheck reporting the actual percentage of passed test cases.

Due to the duration even 1,000 ticks can take to compute, to get a first estimate of our hypotheses tests within reasonable time, we reduce the number of maximum successful replications required to 10 and when doing t-tests 10 replications are run there as well. 

\begin{verbatim}
SugarScape Tests
  Disease Dynamics All Recover:      OK (29.25s)
    +++ OK, passed 10 tests (100% Diseases all recover).
    
  Disease Dynamics Minority Recover: OK (536.00s)
    +++ OK, passed 10 tests (100% Diseases no recover).
    
  Trading Dynamics:                  OK (149.33s)
    +++ OK, passed 10 tests (70% Prices std less than 5.0e-2).
    Only 70% Prices std less than 5.0e-2, but expected 100%
    
  Cultural Dynamics:                 OK (996.84s)
    +++ OK, passed 10 tests (50% Cultures dominate or equal).
    Only 50% Cultures dominate or equal, but expected 100%
    
  Carrying Capacity:   OK (988.20s)
    +++ OK, passed 10 tests (90% Carrying capacity averages at 204.0).    
    Only 90% Carrying capacity averages at 204.0, but expected 100%
    
  Terracing:           OK (280.59s)
    +++ OK, passed 10 tests (80% Terracing is happening).
    Only 80% Terracing is happening, but expected 100%
    
  Inheritance Gini:    OK (7232.59s)
    +++ OK, passed 0 tests (0% Gini coefficient averages at 0.7).
    Only 0% Gini coefficient averages at 0.7, but expected 100%
\end{verbatim}

%\begin{enumerate}
%	\item Disease Dynamics all recover: \textit{+++ OK, passed 10 tests.}
%
%	\item Disease Dynamics minority recover: \textit{+++ OK, passed 10 tests.}
%		
%	\item Trading Dynamics: \textit{+++ OK, passed 10 tests; 2 failed (16\%).} (In total 12 tests (replications) were run, out of which 2 failed, which is a 16\% failure rate.)
%	
%	\item Cultural Dynamics: \textit{+++ OK, passed 10 tests; 3 failed (23\%).}
%
%	\item Inheritance Gini Coefficient: \textit{*** Failed! Passed only 0 tests; 10 failed (100\%) tests.}
%
%	\item Carrying Capacity: \textit{+++ OK, passed 10 tests; 2 failed (16\%).}
%
%	\item Terracing: \textit{+++ OK, passed 10 tests; 2 failed (16\%).}
%\end{enumerate}

How to deal with the failure of hypotheses is obviously highly model specific. A first approach is to increase the number of replications to run to 100 to get a more robust estimate of the failure rate. If the failure rate stays within reasonable ranges then one can arguably assume that the hypothesis is valid for sufficiently enough cases. On the other hand, if the failure rate escalates, then it is reasonable to deem the hypothesis invalid and refine it or even abandon it altogether.

With the exception of the Gini coefficient, we accept the failure rate of the hypotheses we presented here and deem them sufficiently valid for the task at hand. In case of the Gini coefficient, none of the replication was successful, which makes it obvious that it does \textit{not} average at 0.7. Thus the hypothesis as stated in the book does not hold and is invalid. One way to deal with it would be to simply delete it. Another, more constructive approach, is to keep it but require all replications to fail by marking it with \texttt{expectFailure} instead of \texttt{property}. In this way an invalid hypothesis is marked explicitly and acts as documentation and also as regression test.

\section{Hypotheses and test cases}
\label{sec:hypotheses_testcases}

In this section we briefly describe the process of validating our Sugarscape implementation against the specification of the Sugarscape book \cite{epstein_growing_1996} and the work of \cite{weaver_replicating_2009}.

\subsection{Terracing}
Our implementation reproduces the terracing phenomenon as described in the book and as can be seen in the NetLogo implementation as well. We implemented a property test in which we measure the closeness of agents to the ridge: counting the number of same-level sugars cells around them and if there is at least one lower then they are at the edge. If a certain percentage is at the edge then we accept terracing. The question is just how much, which we estimated from tests and resulted in 45\%. Also, in the terracing animation the agents actually never move which is because sugar immediately grows back thus there is no incentive for an agent to actually move after it has moved to the nearest largest cite in can see. Therefore we test that the coordinates of the agents after 50 steps are the same for the remaining steps.

\subsection{Carrying capacity}
Our simulation reached a steady state (variance $<$ 4 after 100 steps) with a mean around ~182. Epstein reported a carrying capacity of 224 (page 30) and the NetLogo implementations' \cite{weaver_replicating_2009} carrying capacity fluctuates around 205 which both are significantly higher than ours. Something was definitely wrong - the carrying capacity has to be around 200 (we trust in this case the NetLogo implementation and deem 224 an outlier).

After inspection of the NetLogo model we realised that we implicitly assumed that the metabolism range is \textit{continuously} uniformly randomized between 1 and 4 but this seemed not what the original authors intended: in the NetLogo model there were a few agents surviving on sugar level 1 which was never the case in ours as the probability of drawing a metabolism of exactly 1 is practically zero when drawing from a continuous range. We thus changed our implementation to draw a discrete value as the metabolism. %Note that this actually makes sense as massive floating-point number calculations were quite expensive in the mid 90s (e.g. computer games ran still on CPU only and exploited various  clever tricks to avoid the need of floating point calculations whenever possible) when SugarScape was implemented which might have been a reason for the authors to assume it implicitly.

This partly solved the problem, the carrying capacity was now around 204 which is much better than 182 but still a far cry from 210 or even 224. After adjusting the order in which agents apply the Sugarscape rules, by looking at the code of the NetLogo implementation, we arrived at a comparable carrying capacity of the NetLogo implementation: agents first make their move and harvest sugar and only after this the agents metabolism is applied (and ageing in subsequent experiments).

For regression tests we implemented a property test which tests that the carrying capacity of 100 simulation runs lies within a 95\% confidence interval of a 210 mean. These values are quite reasonable to assume, when looking at the NetLogo implementation - again we deem the reported carrying capacity of 224 in the book to be an outlier / part of other details we don't know.

One lesson learned is that even such seemingly minor things like continuous vs. discrete or order of actions an agent makes, can have substantial impact on the dynamics of a simulation.

\subsection{Wealth distribution}
By visual comparison we validated that the wealth distribution (page 32-37) becomes strongly skewed with a histogram showing a fat tail, power-law distribution where very few agents are very rich and most of the agents are quite poor. We compute the skewness and kurtosis of the distribution which is around a skewness of 1.5, clearly indicating a right skewed distribution and a kurtosis which is around 2.0 which clearly indicates the 1st histogram of Animation II-3 on page 34. Also we compute the Gini coefficient and it varies between 0.47 and 0.5 - this is accordance with Animation II-4 on page 38 which shows a gini-coefficient which stabilises around 0.5 after. 
We implemented a regression-test testing skewness, kurtosis and gini coefficients of 100 runs to be within a 95\% confidence interval of a two-sided t-test using an expected skewness of 1.5, kurtosis of 2.0 and gini coefficient of 0.48.

\subsection{Migration}
With the information provided by \cite{weaver_replicating_2009} we could replicate the waves as visible in the NetLogo implementation as well. Also we propose that a vision of 10 is not enough yet and shall be increased to 15 which makes the waves very prominent and keeps them up for much longer - agent waves are travelling back and forth between both Sugarscape peaks. We have not implemented a regression test for this property as we couldn't come up with a reasonable straightforward approach to implement it.

\subsection{Pollution and diffusion}
With the information provided by \cite{weaver_replicating_2009} we could replicate the pollution behaviour as visible in the NetLogo implementation as well. We have not implemented a regression test for this property as we couldn't come up with a reasonable straightforward approach to implement it.

%Note that we spent quite a lot of time of getting this and the terracing properties right because they form the very basics of the other ones which follow so we had to be sure that those were correct otherwise validating would have been much more difficult.

%\section{Order of Rules}
%order in which rules are applied is not specified and might have an impact on dynamics e.g. when does the agent mate with others: is it after it has harvested but before metabolism kicks in?

\subsection{Mating}
We could not replicate Figure III-1 - our dynamics first raised and then plunged to about 100 agents and go then on to recover and fluctuate around 300. This findings are in accordance with \cite{weaver_replicating_2009}, where they report similar findings - also when running their NetLogo code we find the dynamics to be qualitatively the same.

Also at first we weren't able to reproduce the cycles of population sizes. Then we realised that our agent behaviour was not correct: agents which died from age or metabolism could still engage in mating before actually dying - fixing this to the behaviour, that agents which died from age or metabolism will not engage in mating solved that and produces the same swings as in \cite{weaver_replicating_2009}. Although our bug might be obvious, the lack of specification of the order of the application of the rules is an issue in the SugarScape book.

\subsection{Inheritance}
We couldn't replicate the findings of the Sugarscape book regarding the Gini coefficient with inheritance. The authors report that they reach a gini coefficient of 0.7 and above in Animation III-4. Our Gini coefficient fluctuated around 0.35. Compared to the same configuration but without inheritance (Animation III-1) which reached a Gini coefficient of about 0.21, this is indeed a substantial increase - also with inheritance we reach a larger number of agents of around 1,000 as compared to around 300 without inheritance.
The Sugarscape book compares this to chapter II, Animation II-4 for which they report a Gini coefficient of around 0.5 which we could reproduce as well. The question remains, why it is lower (lower inequality) with inheritance?

The baseline is that this shows that inheritance indeed has an influence on the inequality in a population. Thus we deemed that our results are qualitatively the same as the make the same point. Still there must be some mechanisms going on behind the scenes which are unspecified in the original Sugarscape.

\subsection{Cultural dynamics}
We could replicate the cultural dynamics of AnimationIII-6 / Figure III-8: after 2700 steps either one culture (red / blue) dominates both hills or each hill is dominated by a different ulture. We wrote a test for it in which we run the simulation for 2.700 steps and then check if either culture dominates with a ratio of 95\% or if they are equal dominant with 45\%. Because always a few agents stay stationary on sugarlevel 1 (they have a metabolism of 1 and cant see far enough to move towards the hills, thus stay always on same spot because no improvement and grow back to 1 after 1 step), there are a few agents which never participate in the cultural process and thus no complete convergence can happen. This is accordance with \cite{weaver_replicating_2009}.

\subsection{Combat}
Unfortunately \cite{weaver_replicating_2009} didn't implement combat, so we couldn't compare it to their dynamics. Also, we weren't able to replicate the dynamics found in the Sugarscape book: the two tribes always formed a clear battlefront where some agents engage in combat, for example when one single agent strays too far from its tribe and comes into vision of the other tribe it will be killed almost always immediately. This is because crossing the sugar valley is costly: this agent wont harvest as much as the agents staying on their hill thus will be less wealthy and thus easier killed off. Also retaliation is not possible without any of its own tribe anywhere near.

We didn't see a single run where an agent of an opposite tribe "invaded" the other tribes hill and ran havoc killing off the entire tribe. We don't see how this can happen: the two tribes start in opposite corners and quickly occupy the respective sugar hills. So both tribes are acting on average the same and also because of the number of agents no single agent can gather extreme amounts of wealth - the wealth should rise in both tribes equally on average. Thus it is very unlikely that a super-wealthy agent emerges, which makes the transition to the other side and starts killing off agents at large. First: a super-wealthy agent is unlikely to emerge, second making the transition to the other side is costly and also low probability, third the other tribe is quite wealthy as well having harvested for the same time the sugar hill, thus it might be that the agent might kill a few but the closer it gets to the center of the tribe the less like is a kill due to retaliation avoidance - the agent will simply get killed by others.

Also it is unclear in case of AnimationIII-11 if the R rule also applies to agents which get killed in combat. Nothing in the book makes this clear and we left it untouched so that agents who only die from age (original R rule) are replaced. This will lead to a near extinction of the whole population quite quickly as agents kill each other off until 1 single agent is left which will never get killed in combat because there are no other agents who could kill it - instead it will enter an infinite die and  reborn cycle thanks to the R rule.

\subsection{Spice}
The book specifies for AnimationIV-1 a vision between 1-10 and a metabolism between 1-5. The last one seems to be quite strange because the maximum sugar / spice an agent can find is 4 which means that agents with metabolism of either 5 will die no matter what they do because the can never harvest enough to satisfy their metabolism. When running our implementation with this configuration the number of agents quickly drops from 400 to 105 and continues to slowly degrade below 90 after around 1000 steps.
The implementation of \cite{weaver_replicating_2009} used a slightly different configuration for AnimationIV-1, where they set vision to 1-6 and metabolism to 1-4. Their dynamics stabilise to 97 agents after around 500+ steps. When we use the same configuration as theirs, we produce the same dynamics.
Also it is worth nothing that our visual output is strikingly similar to both the book AnimationIV-1 and \cite{weaver_replicating_2009}.

\subsection{Trading}
For trading we had a look at the NetLogo implementation of \cite{weaver_replicating_2009}: there an agent engages in trading with its neighbours \textit{over multiple rounds} until either MRSs cross over or no trade has happened anymore. Because \cite{weaver_replicating_2009} were able to exactly replicate the dynamics of the trading time series we assume that their implementation is correct. We think that the fact that an agent interact with its neighbours over multiple rounds is made not very clear in the book. The only hint is found on page 102: \textit{"This process is repeated until no further gains from trades are possible."} which is not very clear and does not specify exactly what is going on: does the agent engage with all neighbours again? is the ordering random? Another hint is found on page 105 where trading is to be stopped after MRS crossover to prevent an infinite loop. Unfortunately this is missing in the Agent trade rule T on page 105. Additional information on this is found in footnote 23 on page 107. Further on page 107: \textit{"If exchange of the commodities will not cause the agents' MRSs to cross over then the transaction occurs, the agents recompute their MRSs, and bargaining begins anew."}. This is probably the clearest hint that trading could occur over multiple rounds.

We still managed to exactly replicate the trading dynamics as shown in the book in Figure IV-3, Figure IV-4 and Figure IV-5. The book is also pretty specific on the dynamics of the trading prices standard deviation: on page 109 the authors specify that at t=1000 the standard deviation will have always fallen below 0.05 (Figure IV-5), thus we implemented a property test which tests for exactly that property. Unfortunately we didn't reach the same magnitude of the trading volume where ours is much lower around 50 but it is equally erratic, so we attribute these differences to other missing specifications or different measurements because the price dynamics match that well already so we can safely assume that our trading implementation is correct.

According to the book, Carrying Capacity (Animation II-2) is increased by Trade (page 111/112). To check this it is important to compare it not against AnimationII-2 but a variation of the configuration for it where spice is enabled, otherwise the results are not comparable because carrying capacity changes substantially when spice is on the environment and trade turned off. We could replicate the findings of the book: the carrying capacity increases slightly when trading is turned on. Also does the average vision decrease and the average metabolism increase. This makes perfect sense: trading allows genetically weaker agents to survive which results in a slightly higher carrying capacity but shows a weaker genetic performance of the population.

According to the book, increasing the agent vision leads to a faster convergence towards the (near) equilibrium price (page 117/118/119, Figure IV-8 and Figure IV-9). We could replicate this behaviour as well.

According to the book, when enabling R rule and giving agents a finite life span between 60 and 100 this will lead to price dispersion: the trading prices will not converge around the equilibrium and the standard deviation will fluctuate wildly (page 120, Figure IV-10 and Figure IV-11). We could replicate this behaviour as well.

The Gini coefficient should be higher when trading is enabled (page 122, Figure IV-13) - We could replicate this behaviour.

Finite lives with sexual reproduction lead to prices which don't converge (page 123, Figure IV-14). We could reproduce this as well but it was important to set the parameters to reasonable values: increasing number of agents from 200 to 400, metabolism to 1-4 and vision to 1-6, most important the initial endowments back to 5-25 (both sugar and spice) otherwise hardly any mating would happen because the agents need too much wealth to engage (only fertile when have gathered more than initial endowment). What was kind of interesting is that in this scenario the trading volume of sugar is substantially higher than the spice volume - about 3 times as high. 

From this part, we didn't implement: Effect of Culturally Varying Preferences, page 124 - 126, Externalities and Price Disequilibrium: The effect of Pollution, page 126 - 118, On The Evolution of Foresight page 129 / 130. 

%\section{Lending (Credit)}
%Not really much information to validate was available and the \cite{weaver_replicating_2009} implementation ran into an exception so there was not much to validate against. What was unexpected was that this was the most complex behaviour to implement, with lots of subtle details to take care of (spice on/off, inheritance,...).
%Note that we implemented lending of sugar and spice, although it looks from the book (Animation IV-5) that they only implemented it for sugar.

\subsection{Diseases}
We were able to exactly replicate the behaviour of Animation V-1 and Animation V-2: in the first case the population rids itself of all diseases (maximum 10) which happens pretty quickly, in less than 100 ticks. In the second case the population fails to do so because of the much larger number of diseases (25) in circulation. We used the same parameters as in the book. 
The authors of \cite{weaver_replicating_2009} could only replicate the first animation exactly and the second was only deemed "good". Their implementation differs slightly from ours: In their case a disease can be passed to an agent who is immune to it - this is not possible in ours. In their case if an agent has already the disease, the transmitting agent selects a new disease, the other agent has not yet - this is not the case in our implementation and we think this is unreasonable to follow: it would require too much information and is also unrealistic.
We wrote regression tests which check for animation V-1 that after 100 ticks there are no more infected agents and for animation V-2 that after 1000 ticks there are still infected agents left and they dominate: there are more infected than recovered agents.

\section{Discussion}
In this appendix we showed how to use QuickCheck to formalise and check hypotheses about an \textit{exploratory} agent-based model, in which no ground truth exists. Due to ABS stochastic nature in general it became obvious that to get a good measure of a hypotheses validity we need to emulate failure using the \texttt{cover} function of QuickCheck. This allowed us to show that the hypotheses we have presented are sufficiently valid for the task at hand and can indeed be used for expressing and formalising emergent properties of the model and also as regression tests within a TDD cycle.

%What is particularly powerful is that one has complete control and insight over the changed state before and after e.g. a function was called on an agent: thus it is very easy to check if the function just tested has changed the agent-state itself or the environment: the new environment is returned after running the agent and can be checked for equality of the initial one - if the environments are not the same, one simply lets the test fail. This behaviour is very hard to emulate in OOP because one can not exclude side-effect at compile time, which means that some implicit data-change might slip away unnoticed. In FP we get this for free.

\end{appendices}

\end{document}
