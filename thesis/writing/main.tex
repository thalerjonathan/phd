\documentclass[oneside]{book}

\setcounter{tocdepth}{1}
\setcounter{secnumdepth}{3}

\usepackage[toc,page]{appendix}
\usepackage{minted}
\usepackage{graphicx}
\usepackage{hyperref}
\usepackage{amsmath} 
\usepackage{subcaption}
\usepackage{epigraph}
\usepackage{titlesec}
\usepackage{multirow}
\usepackage{setspace}
\usepackage{mathtools}
\usepackage{afterpage}


% UNUSED
%\usepackage[english]{babel}
%\usepackage{pdflscape}
%\usepackage{pdfpages}
%\usepackage{float}
%\usepackage{verbatim}
%\usepackage[]{algorithm2e}

\interfootnotelinepenalty=10000

% define HaskellCode command for nice formatting of Haskell Code
\newminted[HaskellCode]{haskell}{fontsize=\footnotesize}
% needed for clearing page after title page
\newcommand\blankpage{%
    \null
    \thispagestyle{empty}%
    \addtocounter{page}{-1}%
    \newpage}
    
\makeatletter
\titleformat{\part}[display]
  {\Huge\scshape\filright}
  {\partname~\thepart:}
  {20pt}
  {\thispagestyle{epigraph}}
\makeatother
\setlength\epigraphwidth{.6\textwidth}

\begin{document}

\begin{titlepage}
	\centering
	%\includegraphics[width=0.60\textwidth]{./logo/UoN_Primary_Logo_RGB.png}\par\vspace{1cm}
	\includegraphics[width=0.60\textwidth]{./logo/coat_of_arms.jpg}\par\vspace{1cm}
	%{\scshape\Large Ph.D. Thesis\par}
	\vspace{1.5cm}
	%{\huge\bfseries Foundations Of Pure Functional \\ Agent-Based Simulation \par}
	{\huge\bfseries Pure Functional Programming \\ in Agent-Based Simulation \par}
	%{\huge\bfseries The Pure Functional Programming Paradigm \\ in Agent-Based Simulation \par}
	\vspace{2cm}
	%{\Large Jonathan Thaler (4276122) \\ \itshape jonathan.thaler@nottingham.ac.uk \par}
	{\Large by Jonathan Thaler \par}
	\vfill
	Thesis submitted to the University of Nottingham \par
	for the degree of Doctor of Philosophy \par
	
	\vfill
	
	\small
	supervised by\par
	Dr. Peer-Olaf \textsc{Siebers} \\
	Dr. Thorsten \textsc{Altenkirch}

	\vfill
	{\large \today\par}
\end{titlepage}

\cleardoublepage
\afterpage{\blankpage}
%\doublespacing

% TOC comes after title page according to UoN thesis guidelines
\clearpage
\tableofcontents

\newpage

\thispagestyle{plain}

\section*{Abstract}
This thesis systematically investigates the use of the \textit{pure} functional programming paradigm for implementing Agent-Based Simulations (ABS) and the benefits and drawbacks when doing so. As language of choice, Haskell is used due to its modern, \textit{pure} nature and increasing use in real-world applications. First, the thesis explore \textit{how} to implement ABS pure functionally, discussing both a time- and event-driven approach. In each case arrowized Functional Reactive Programming plays a fundamental role to derive fundamental abstractions and concepts. As use cases the well known explanatory agent-based SIR and the exploratory Sugarscape model are used. Then the thesis explores \textit{why} it is of benefit to implement ABS pure functionally, where it focuses on parallelism and concurrency and property-based testing. In the parallelism part, the main focus is on how to speedup the simulation but keeping it still pure, whereas in the concurrency part, Software Transactional Memory is used at the price of purity. Property-based testing is used to show how to encode full agent specifications, model invariants and do verification of the explanatory model directly in code. Further, hypothesis testing for the exploratory Sugarscape model is shown.

\clearpage

TODO first version for others to read: until 25th July (back from SummerSim conference)
\begin{enumerate}
	\item careful read of PRINT OUT
	\begin{itemize}
		\item types / function names / monad names texttt
		\item instead of italics use texttt use it for types, functions and emphasising certain word, it looks better
		\item when citing multiple references, make sure they are in ascending order of their reference number	
		\item be consistent about hyphens and rather avoid them	
	\end{itemize}

	\item move links from footnote into references and put full implementations of existing models from phd repo in separate gitrepso. e.g. sugarscape, SIR event and time as and reconcile all code into thesis code folder?

	\item for references add hyperlinks using howpublished in case it is a blog or website (e.g. stephen diel what i wish...)	
		
	\item copy the xml definitions of the diagrams into the thesis folder

	\item introduce abbreviations once (ABS, MAS, STM, FRP, ADT, GADT, IO)?
	
	\item get rid of all e.g.

	\item use package instead of library because on hackage they are packages not libraries 

	\item avoid 'Note that...'. is annoying and can be more straightforward

	\item avoid :, can be more compact in most cases	
	
	\item Thus, However, Also, Next is usually followed by a comma	
	
	\item what about: time-flow, pro-active, multi-processing, ?
	
	\item epigraphs for all chapters?
	%\epigraph{The purpose of abstraction is not to be vague, but to create a new semantic level in which one can be absolutely precise.}{Dijkstra, EWD340}

	\item write a GOOD and STRONG abstract, the current one sucks
\end{enumerate}

TODO "cleaning up / polishing" milestone until end of August 2019
\begin{enumerate}
	\item reading by my supervisors

	\item reading by james hey
	
	\item reading by thomas schwarz for content
		
	\item proof-reading by professional lector (Karly), focus on 
	\begin{itemize}
		\item s vs 's vs. s' e.g. Agents vs. Agents's vs. Agents'
	\end{itemize}
	
	\item check for any violation of the guidelines https://www.nottingham.ac.uk/physics/currentstudents/pg-submissionguide.aspx
\end{enumerate}

% DEDICATION
\clearpage
\begin{center}
    \thispagestyle{empty}
    \vspace*{\fill}
    \textit{To my parents Irmentraud and Wolfgang. \\ For their unconditional love and support throughout all my life.}
    \vspace*{\fill}
\end{center}

% ACKNOWLEDGEMENTS
\clearpage
\chapter*{Acknowledgements}
Thanks go to my first supervisor Peer-Olaf Siebers, who always very patiently reminded me that a Ph.D. is not about to change the world but learning how to do research on my own. He was a strong guidance throughout my 3 years in Nottingham and I could not have hoped for a better and more dedicated first supervisor.

I am also thankful for my second supervisor Thorsten Altenkirch, who gave strong and sometimes brutal feedback about the technical details of my approaches. Due to the fact that his main interest is a rather theoretical spin on functional programming and computing, I am deeply grateful for his strong support of my rather practical approach to functional programming.

I am in depth to the whole Functional Programming Lab at UoN, for welcoming me in their midst despite my lack of specific theoretical background. I owe them many open and deep discussions which resulted in new insights. Further, presenting at their \textit{FP Lunch} was always a challenging but highly rewarding activity, always resulting in valuable feedback.

I am especially in depth to Ivan Perez for always having an open ear for questions and valuable discussions about his research, without this Ph.D. would have probably developed a very different spin.

Many thanks go to Martin Handley and James Hey for many discussions, feedback and proof reading of my papers and my thesis.

Thanks go also to Julie Greensmith for valuable discussions and pointing me into right directions at important stages of the Ph.D.

% PART I: Preliminaries
\epigraphhead[450]{}
\part{Preliminaries}
\label{part:preliminaries}
%*******************************************************************************
%*********************************** First Chapter *****************************
%*******************************************************************************

\chapter{Introduction}  %Title of the First Chapter
I noticed that it is pretty hard to convince an agent-based economics specialist who is not a computer scientist about a pure functional approach. My conjecture is that the implementation technique and method does not matter much to them because they have very little knowledge about programming and are almost always self-taught - they don't know about software-engineering, nothing about proper software-design and architecture, nothing about software-maintenance, nothing about unit-testing,... In the end they just "hack" the simulation in whatever language they are able to: C++, Visual Basic, Java or toolboxes like Netlogo. For them it is all about to \textit{get things done somehow} and not to get things done the right way or in a beautiful way - the way and the method doesn't matter, its just a necessary evil which needs to be done. Thus if functional programming could make their lives easier, then they will definitely welcome it. But functional programming is, i think, harder to learn and harder to understand - so one needs to provide an abstraction through EDSL. So I REALLY need to come up with convincing arguments why to use pure functional approaches in ACE THEY can understand, otherwise I will be lost and not heard (not published,...). \\

What ACE economists care for:

\begin{itemize}
\item Very: Qualitative modelling with quantitative results
\item Yes: Easy reproducibility
\item Likely: Reasoning about convergence?
\item Likely: EDSL
\end{itemize}

My contributions are: pure functional framework, functional agent-model for market-simulations, EDSL for market-simulations, qualitative / implicit modelling with quanitative results, reasoning in my framework about convergence \\

IDEA: could I develop non-causal modelling (models are expressed in terms of non-directed equations, modelled in signal-relations) to allow for qualitative modelling for the agent-based economists? See hybrid modelling paper of Yampa. \textbf{THIS WOULD BE A HUGE NOVEL CONTRIBUTION TO ACE ESPECIALLY WHEN COMBINED WITH AN EDSL AND PROVIDING FULL REFERENTIAL TRANSPARENCY TO KEEP THE ABILITY TO REASON ABOUT CONVERGENCE}. This should be covered in the "EDSL"-paper.

TODO: maybe i should really focus only on market models? otherwise too much? \\

central novelty of my PhD: model specification = runnable code. possible through EDSL. but only in specific subfield of ACE: market-models. need a functional description of the model, then translate it to model specification in EDSL and then run it to see dynamics. But: model specification moves closer to functional programming languages. \\

another novelty approach: model specification through qualitative instead of quantiative approaches. is this possible? \\

WHY FUNCTIONAL? "because its the ultimate approach to scientific computing": fewer bugs due to mutable state (why? is thos shown obkectively by someone?), shorter (again as above, productivity), more expressive and closer to math, EDSL, EDSL=model=simulation, better parallelising due to referental transparency, reasoning \\

scientific results need to be reproduced, especially when they have high impact. a more formal approach of specifying the model and the simulation (model=simulation) could lead to easier sharing and easier reporduction without ambigouites \\

pure functional agent-model \& theory, EDSL framework in Haskell for ACE

\begin{enumerate}
\item Which kind of problem do we have?
\item What aim is there? Solving the problem? 
\item How the aim is achieved by enumerating VERY CLEAR objectives.
\item What the impact one expects (hypothesis) and what it is (after results).
\end{enumerate}

Note: It is not in the interest of the researcher to develop new economic theories but to research the use of functional methods (programming and specification) in agent-based computational economics (ACE).

NOTE: Get the reader’s attention early in the introduction: motivation, significance, originality and novelty.

\section{Methods}
Methods need to be selected to implement the simulations. Special emphasis will be put on functional ones which will then be compared to established methods in the field of ABM/S and ACE. \\

Claim: non-programming environments are considered to be not powerful enough to capture the complexity of ACE implementations thus a programming approach to ACE will be always required.

\section{Scenarios}
To apply and test functional methods in ACE, four scenarios of ACE are selected and then the methods applied and compared with each other to see how each of them perform in comparison. The 4 selected scenarios represent a selection of the challenges posed in ACE: from very abstract ones to very operational ones.

\section{Comparison}
Each of the selected scenarios is then implemented using the selected methods where each solution is then compared against the following criteria: 

\begin{enumerate}
\item suitability for scientific computation
\item robustness
\item error-sources
\item testability
\item stability
\item extendability
\item size of code
\item maintainability
\item time taken for development
\item verification \& correctness
\item replications \& parallelism
\item EDSL
\end{enumerate}

This will then allow to compare the different methods against each other and to show under which circumstances functional methods shine and when they should not be used.

\section{Agent-Based Modelling and Simulation (ABM/S)}
ABM/S is a method of modelling and simulating a system where the global behaviour may be unknown but the behaviour and interactions of the parts making up the system is of knowledge (Wooldrige, M. (2009). An Introduction to MultiAgent Systems. John Wiley & Sons). Those parts, called agents, are modelled and simulated out of which then the aggregate global behaviour of the whole system emerges. Thus the central aspect of ABM/S is the concept of an Agent which can be understood as a metaphor for a pro-active unit, able to spawn new Agents, and interacting with other Agents in a network of neighbours by exchange of messages. The implementation of Agents can vary and strongly depends on the programming language and the kind of domain the simulation and model is situated in.

\section{Agent-Based Economics (ACE)}
According to Leigh Tesfatsion (Tesfatsion, L. (2006). Agent-based computational economics: A constructive approach to economic theory. In Tesfatsion, L. and Judd, K. L., editors, Handbook of Computational Economics, volume 2, chapter 16, pages 831–880. Elsevier, 1 edition.), one of the leading figures, ACE is "[...] computational modelling of economic processes (including whole economies) as open-ended dynamic systems of interacting agents." - thus lending perfectly to the use of ABM/S as already the name suggests. Whereas classical economic models fall short by only looking at the average, pure rational, individual interacting in anonymous markets, the ACE approach looks at heterogeneous, non-rational individuals interacting with each other in networks (Kirman, A. (2010). Complex Economics: Individual and Collective Rationality. Routledge, London ; New York, NY.). Thus ACE can be understood as a combination of computer-science, cognitive/social science and evolutionary economics.

\section{Functional programming}
TODO: read \cite{Backus1978}

The state-of-the-art approach to implementing Agents are object-oriented methods and programming as the metaphor of an Agent as presented above lends itself very naturally to object-orientation (OO). The author of this thesis claims that OO in the hands of inexperienced or ignorant programmers is dangerous, leading to bugs and hardly maintainable and extensible code. The reason for this is that OO provides very powerful techniques of organising and structuring programs through Classes, Type Hierarchies and Objects, which, when misused, lead to the above mentioned problems. Also major problems, which experts face as well as beginners are 1. state is highly scattered across the program which disguises the flow of data in complex simulations and 2. objects don’t compose as well as functions. The reason for this is that objects always carry around some internal state which makes it obviously much more complicated as complex dependencies can be introduced according to the internal state.
All this is tackled by (pure) functional programming which abandons the concept of global state, Objects and Classes and makes data-flow explicit. This then allows to reason about correctness, termination and other properties of the program e.g. if a given function exhibits side-effects or not. Other benefits are fewer lines of code, easier maintainability and ultimately fewer bugs thus making functional programming the ideal choice for scientific computing and simulation and thus also for ACE. A very powerful feature of functional programming is Lazy evaluation. It allows to describe infinite data-structures and functions producing an infinite stream of output but which are only computed as currently needed. Thus the decision of how many is decoupled from how to (Hughes, J. (1989). Why functional programming matters. Comput. J., 32(2):98–107.).
The most powerful aspect using pure functional programming however is that it allows the design of embedded domain specific languages (EDSL). In this case one develops and programs primitives e.g. types and functions in a host language (embed) in a way that they can be combined. The combination of these primitives then looks like a language specific to a given domain, in the case of this thesis ACE. The ease of development of EDSLs in pure functional programming is also a proof of the superior extensibility and composability of pure functional languages over OO (Henderson P. (1982). Functional Geometry. Proceedings of the 1982 ACM Symposium on LISP and Functional Programming.).
One of the most compelling example to utilize pure functional programming is the reporting of Hudak (Hudak P., Jones M. (1994). Haskell vs. Ada vs. C++ vs. Awk vs. ... An Experiment in Software Prototyping Productivity. Department of Computer Science, Yale University.)  where in a prototyping contest of DARPA the Haskell prototype was by far the shortest with 85 lines of code. Also the Jury mistook the code as specification because the prototype did actually implement a small EDSL which is a perfect proof how close EDSL can get to and look like a specification.

Functional languages can best be characterized by their way computation works: instead of \textit{how} something is computed, \textit{what} is computed is described. Thus functional programming follows a declarative instead of an imperative style of programming. The key points are:
\begin{itemize}
\item No assignment statements - variables values can never change once given a value.
\item Function calls have no side-effect and will only compute the results - this makes order of execution irrelevant, as due to the lack of side-effects the logical point in \textit{time} when the function is calculated within the program-execution does not matter.
\item higher-order functions
\item lazy evaluation
\item Looping is achieved using recursion, mostly through the use of the general fold or the more specific map.
\item Pattern-matching
\end{itemize}

This alone does not really explain the \textit{real} advantages of functional programming and one must look for better motivations using functional programming languages. One motivation is given in \cite{Hughes1989} which is a great paper explaining to non-functional programmers what the significance of functional programming is and helping functional programmers putting functional languages to maximum use by showing the real power and advantages of functional languages. The main conclusion is that \textit{modularity}, which is the key to successful programming, can be achieved best using higher-order functions and lazy evaluation provided in functional languages like Haskell. \cite{Hughes1989} argues that the ability to divide problems into sub-problems depends on the ability to glue the sub-problems together which depends strongly on the programming-language and \cite{Hughes1989} argues that in this ability functional languages are superior to structured programming.

TODO: comparison of functional and object-oriented programming. My points are:
\begin{itemize}
\item The way state can be changed and treated - distributed over multiple objects - is often very difficult to understand.
\item Inheritance is a dangerous thing if not used with care because inheritance introduces very strong dependencies which cannot be changed during runtime anymore.
\item Objects don't compose very well: \url{http://zeroturnaround.com/rebellabs/why-the-debate-on-object-oriented-vs-functional-programming-is-all-about-composition/}
\item (Nearly) impossible to reason about programs
\end{itemize}

In conclusion the upsides of functional programming as opposed to OO are:
\begin{itemize}
\item Much more explicit flow of data \& control
\item Much better compose-able
\item Much better parallelism
\end{itemize}

\section{Related Research}
Tim Sweeney, CTO of Epic Games gave an invited talk about how "future programming languages could help us write better code" by "supplying stronger typing, reduce run-time failures;  and the need for pervasive concurrency support, both implicit and explicit, to effectively exploit the several forms of parallelism present in games and graphics." \cite{Sweeney2006}. Although the fields of games and agent-based simulations seem to be very different in the end, they have also very important similarities: both are simulations which perform numerical computations and update objects - in games they are called "game-objects" and in abm they are called agents but they are in fact the same thing - in a loop either concurrently or sequential. His key-points were:

\begin{itemize}
\item Dependent types as the remedy of most of the run-time failures.
\item Parallelism for numerical computation: these are pure functional algorithms, operate locally on mutable state. Haskell ST, STRef solution enables encapsulating local heaps and mutability within referentially transparent code.
\item Updating game-objects (agents) concurrently using STM: update all objects concurrently in arbitrary order, with each update wrapped in atomic block - depends on collisions if performance goes up.
\end{itemize}
\section{Background}

\subsection{Schelling Segregation}
We follow in our implementation the original paper of Schelling as in \cite{schelling_dynamic_1971} where we focus on the \textit{Area Distribution} section (Schelling starts with movement in a linear, 1-dimensional world where agents are able to move to the nearest point which meets the agents satisfaction but this is not what we follow here). One assumes a discrete 2-dimensional lattice-world with NxM fields. Each field is either occupied by an agent of a given color (e.g. Red or Green) or is free. Each field has 8 neighbours, which denotes a Moore-Neighbourhood. In Schellings original work the lattice-world is limited at its borders but we assume a torus world which is wrapped around in both the x- and y-dimensions resulting in 8 neighbours also for fields at the border. The occupation density was set by Schelling to be about 70\%-75\% which he identifies as being a setting which allows the agents to move around freely without making the lattice-world too sparse.
Now the agents make their move sequentially one after another. In each move an agent calculates the number of neighbours which are of equal color. If the number satisfies the agents needs about the neighbourhood then the agent is regarded as being 'happy' and will stay on this field. On the other hand the agent moves to the nearest unoccupied field which satisfies its needs. An agent which moves selects an unoccupied place randomly relative from its current place within a rectangle of side-length 2r where its current place is at the center. The interpretation for that behaviour is that agents won't move too far as it could be costly. Also children might attend a school in this area or the family has friends in this area, so they don't want to break that.



Agents just move depending on their movement-strategy to another place if they are not happy on the current one - they don't care how the target place is in the present or in the future, they will decide again in the next time-step. The interpretation for that behaviour is: agents want to 'just get out' at any cost, not caring what the future place will look like - it might be better or worse but they will see then.

\subsubsection{Optimizing behaviour}
TODO: define utility

The original schelling model didn't have a move-optimizing behaviour, meaning agents are just binary: if it is happy it will not move, if it is unhappy it will move but they won't care where they move. We introduce local move-optimizing behaviours which can be interpreted as being realistic in the real-world. It is important to note that we focus on \textit{local} instead of \textit{global} move-optimization: the agents are limited in their reasoning-capabilities and have limited information available: they cannot check out \textit{every} place and pick the globally best one.\\

\subsubsection{Anticipating behaviour}
Schelling explicitly mentions in \cite{schelling_dynamic_1971} that nobody anticipates moves of others. This is what we introduce using the recursive simulation.

TODO: is this optimizing behaviour in the spirit of schellings original work? 

\paragraph{Optimizing future} Agents pick an unoccupied random place and move to it if it increases their utility in the future. The interpretation for that behaviour is: agents heard about a place which will be cool in the future.

\paragraph{Optimizing present \& future} Agents pick an unoccupied random place and move to it if it increases their utility in the now and in the future. The interpretation for that behaviour is: agents heard about a cool spot in town, check it out and move to it if they like it but they also anticipate the coolness of the place in the future and if it seems that the place is going down then they won't move there.

\subsection{Related Research}
TODO: \cite{kirman_complex_2010} mention kirman complex economics where he investigates the model more in depth

\chapter{Implementing ABS}
\label{ch:impl_abs}

In this Chapter we briefly discuss general problems and considerations, ABS implementations need to solve, independent from the programming paradigm. In general, an ABS implementation must solve the following fundamental problems:

\begin{enumerate}
	\item How to represent an agent, its local state and its interface.
	\item How to represent agent-to-agent interactions and defining and enforcing their semantics.
	\item How to represent an environment.
	\item How to represent agent-to-environment interactions and defining and enforcing their semantics.
	\item How agents and an environment can initiate actions without external stimuli.
	\item How to step the simulation.
\end{enumerate}

% agent- and environment pro-activity
We argue that the most fundamental concept of ABS is the \textit{pro-activity} of both, agents and its environment. In computer systems, pro-activity, the ability to initiate actions on its own without external stimuli, is only possible when there is some internal stimulus, most naturally represented by a continuous increasing time-flow. Due to the discrete nature of computer systems, this time-flow must be discretized in steps as well and each step must be made available to the agent, acting as the internal stimulus. This allows the agent then to perceive time and become pro-active depending on time. So we can understand an ABS as a discrete time-simulation where time is broken down into continuous, real-valued or discrete natural-valued time-steps. Independent of the representation of the time-flow we have the two fundamental choices whether the time-flow is local to the agent or whether it is a system-global time-flow. Time-flows in computer-systems can only be created through threads of execution where there are two ways of feeding time-flow into an agent. Either it has its own thread-of-execution or the system creates the illusion of its own thread-of-execution by sharing the global thread sequentially among the agents where an agent has to yield the execution back after it has executed its step. Note the similarity to an operating system with cooperative multitasking in the latter case and real multi-processing in the former.

% time- and event-driven ABS
Generally, there exist time- and event-driven approaches to ABS \cite{meyer_event-driven_2014}. In time-driven ABS, time is explicitly modelled and is the main driver of the ABS dynamics. The semantics of models using this approach, center around time. As a representative example, which will be used in Chapter \ref{ch:timedriven} on time-driven ABS, we use the agent-based SIR model \cite{macal_agent-based_2010, thaler_pure_2018}. Often such models are inspired by an underlying System Dynamics approach, where the continuous time-flow is the main driving force of the dynamics. It is clear that almost every ABS models time in some way, after all, this is the very heart of Simulation: modelling a virtual system over some (virtual) time. Still we want to distinguish clearly between different semantics of time-representation in ABS: when time is seen as a continuous flow such as in the example of the agent-based SIR model, we talk about a truly time-driven approach. In other words: if an agent behaves as a time-signal then we speak of a time-driven approach. This means that if the system is sampled with a $\Delta t = 0$ then, even though the agents are executed their behaviour must stay constant and must not change.

In the case where time advances in a discrete way either by means of events or messages, we talk about an event-driven approach. As a representative example, which will be used in Chapter \ref{ch:eventdriven} on event-driven ABS, we use the Sugarscape model. In this model time is discrete and represented by the natural numbers where agents act in every tick - time is not modelled explicitly as in the agent-based SIR case. In such a model, the underlying semantics map more naturally to a DES core, extended by ABS features. Although the Sugarscape model does not semantically map to a DES core in a strict sense, our implementation approach is very close to such and can be easily extended to a true DES core - thus it serves as a good example for the discussion of the event-driven approach. Further, using an event-driven SIR implementation, we also show how to extend it to a pure DES core, allowing to implement models with more explicit event-driven semantics as discussed in \cite{meyer_event-driven_2014}.

% agent representation
According to the definition of ABS in  Chapter \ref{sec:method_abs}, an agent is a uniquely addressable entity with an identity, an internal state it has exclusive control over and can be interacted with by means of messages. In the established object-oriented approaches to ABS all this is implemented naturally by the use of objects: an object has a clear identity, encapsulates internal state and exposes an interface through public methods through which objects can interact with each other, also called messaging. The same applies to the environment and it is by no means clear how to achieve this in a pure functional approach where we don't have objects available - this will be addressed in the subsequent Chapters \ref{ch:timedriven} and \ref{ch:eventdriven}.

Before we look into pure functional ABS implementation concepts in the next chapters, we need to discuss the concept of update strategies \cite{thaler_art_2017}. Generally, there are four strategies to approach time-driven ABS, where the differences deal with how the simulation is stepped, the agents are executed and the interaction semantics work.

% agent-to-agent interaction ant its semantics
%The semantics of messaging define when sent messages are visible to the receivers and when the receivers process them. Message-processing could happen either immediately or delayed, depending on how message-delivery works. There are two ways of message-delivery: immediate or queued. In the case of immediate message-deliver the message is sent directly to the agent without any queuing in between e.g. a direct method-call. This would allow an agent to immediately react to this message as this call of the method transfers the thread-of-execution to the agent. This is not the case in the queued message-delivery where messages are posted to the message-box of an agent and the agent pro-actively processes the message-box at regular points in time. With established OOP approaches we can have both: either a direct method-call or a message-box approach - in pure FP this is a much more subtle problem and it turns out that the problem of messaging / interacting of agents and of agents with the environment is the most subtle problem when approaching ABS from a pure functional perspective.

\section{Sequential Strategy}
\label{sec:seq_strategy}
In this strategy there exists a globally synchronized time-flow and in each time-step the simulation iterates through all the agents and updates one agent after another. Messages sent and changes to the environment made by agents are visible immediately, meaning that if an agent sends messages to other agents or changes the environment, agents which are executed after this agent will see these changes within the same time-step. There is no source of randomness and non-determinism, rendering this strategy to be completely deterministic in each step. Messages can be processed either immediately or queued depending on the semantics of the model. If the model requires to process the messages immediately the model must be free of potential infinite-loops. Often in such models, the agents are shuffled when the model semantics require to average out the advantage of being executed as first. This strategy is of fundamental importance for event-driven ABS in Chapter \ref{ch:eventdriven}. See Figure \ref{fig:strategy_seq} for a visualisation of the control flow in this strategy.

\begin{figure}[H]
	\centering
	\includegraphics[width=0.5\textwidth, angle=0]{./fig/implabs/sequential.png}
	\caption{Control flow in the Sequential Strategy.}
	\label{fig:strategy_seq}
\end{figure}

\section{Parallel Strategy}
\label{sub:par_strategy}

This strategy has a globally synchronized time-flow and in each time-step iterates through all the agents and updates them in parallel. Messages sent and changes to the environment made by agents are visible in the next global step. We can think about this strategy in a way that all agents make their moves at the same time.  If one wants to change the environment in a way that it would be visible to other agents this is regarded as a semantic error in this strategy. First, it is not logical because all actions are meant to happen at the same time and second, it would implicitly induce an ordering, violating the semantics of the model, the \textit{happens at the same time} idea.
It does not make a difference if the agents are really executed in parallel or just sequentially - due to the isolation of information, this has the same effect. Also it will make no difference if we iterate over the agents sequentially or randomly, the outcome has to be the same: the strategy is event-ordering invariant as all events and updates happen \textit{virtually at the same time}. This strategy is of fundamental importance for time-driven ABS in Chapter \ref{ch:timedriven}. See Figure \ref{fig:strategy_par} for a visualisation of the control flow in this strategy.

\begin{figure}[H]
	\centering
	\includegraphics[width=0.4\textwidth, angle=0]{./fig/implabs/parallel.png}
	\caption{Control flow in the Parallel Strategy.}
	\label{fig:strategy_par}
\end{figure}

\section{Concurrent Strategy}
This strategy has a globally synchronized time-flow but in each time-step all the agents are updated in parallel with messages sent and changes to the environment are visible immediately. So this strategy can be understood as a more general form of the \textit{parallel strategy}: all agents run at the same time but act concurrently. It is important to realize that when running agents, which are able to see actions by others immediately, in parallel, we arrive at the very definition of concurrency: parallel execution with mutual read/write access to shared data. Of course this shared data-access needs to be synchronized which in turn will introduce event-orderings in the execution of the agents. At this point we have a source of inherent non-determinism: although when one ignores any hardware-model of concurrency, at some point we need arbitration to decide which agent gets first access to a shared resource, arriving at non-deterministic solutions. This has the very important consequence that repeated runs with the same configuration of the agents and the model may lead to different results. This strategy is of fundamental importance for concurrent ABS in Chapter \ref{ch:concurrent_abs}. See Figure \ref{fig:strategy_conc} for a visualisation of the control flow in this strategy.

\begin{figure}[H]
	\centering
	\includegraphics[width=0.5\textwidth, angle=0]{./fig/implabs/concurrent.png}
	\caption{Control flow in the Concurrent Strategy.}
	\label{fig:strategy_conc}
\end{figure}

\section{Actor Strategy}
This strategy has no globally synchronized time-flow but all the agents run concurrently in parallel, with their own local time-flow. The messages and changes to the environment are visible as soon as the data arrive at the local agents - this can be immediately when running locally on a multi-processor or with a significant delay when running in a cluster over a network. Obviously this is also a non-deterministic strategy and repeated runs with the same agent- and model-configuration may (and will) lead to different results. It is of most importance to note that information and also time in this strategy is always local to an agent as each agent progresses in its own speed through the simulation. In this case one needs to explicitly \textit{observe} an agent when one wants to e.g. visualize it. This observation is then only valid for this current point in time, local to the observer but not to the agent itself, which may have changed immediately after the observation. This implies that we need to sample our agents with observations when wanting to visualize them, which would inherently lead to well known sampling issues. A solution would be to invert the problem and create an observer-agent which is known to all agents where each agent sends a \textit{'I have changed'} message with the necessary information to the observer if it has changed its internal state. This also does not guarantee that the observations will really reflect the actual state the agent is in but is a remedy against the notorious sampling. The concept of Actors was proposed by \cite{hewitt_universal_1973} for which \cite{grief_semantics_1975} and \cite{clinger_foundations_1981} developed semantics of different kinds. These works were very influential in the development of the concepts of agents and and can be regarded as foundational basics for ABS. We come back to this strategy in the context of concurrent ABS in Chapter \ref{ch:concurrent_abs}. See Figure \ref{fig:strategy_act} for a visualisation of the control flow in this strategy.

\begin{figure}[H]
	\centering
	\includegraphics[width=0.5\textwidth, angle=0]{./fig/implabs/actor.png}
	\caption{Control flow in the Actor Strategy.}
	\label{fig:strategy_act}
\end{figure}

\section{Discussion}
In the following chapters we discuss \textit{how} to implement ABS from a pure functional perspective and \textit{why} one would do so. More specifically, we show how to approach the problems discussed in this using pure functional programming (FP). The \textit{sequential} strategy will be covered in-depth in Chapter \ref{ch:eventdriven} on event-driven ABS, the \textit{parallel} one in Chapter \ref{ch:timedriven} on time-driven ABS and the \textit{concurrent} strategy is used in Chapter \ref{ch:concurrent_abs} on concurrent ABS. The \textit{actor} strategy is not used in this thesis but its implementation follows directly from the Chapters \ref{ch:timedriven} and \ref{ch:parallel_abs}: instead of globally synchronising in the main-thread, a closed feedback-loop is run in every agent thread. 

As already outlined in Chapter \ref{sec:method_abs}, the established approaches implementing ABS use object-oriented programming and thus solve the problems outlined at the start of this chapter from this perspective, which is quite well understood by now, as high quality ABS frameworks like RePast \cite{north_complex_2013} prove. In object-oriented programming an agent is mapped directly onto an object, encapsulating the agents state and providing methods, which implement the agents' actions. Object-orientation allows to expose a well-defined interface using public methods by which one can interact with the agent and query information from it. Agent objects can directly invoke other agents' methods, implicitly mutating the other agents' internal state, which makes direct agent interaction straight forward. Also with object-orientation, agents have global access to an environment e.g. through a Singleton or a simple global variable, and can mutate the environments data by direct method calls.

All these language features are not available in FP and compare to object-orientation we face seemingly severe restrictions like immutable state, recursion and a static type system. Further, we restrict ourselves deliberately to \textit{pure} FP and avoid running in the non-deterministic \textit{IO} context under all costs. The question is then how to solve these problems in FP \textit{and} use the restrictions to our advantage. In the next two chapters we show how to implement both a time-driven ABS  using the agent-based SIR model as example (Chapter \ref{ch:timedriven}) and an event-driven ABS using the Sugarscape model as example (Chapter \ref{ch:eventdriven}). In both we present fundamental concepts of how to engineer an ABS from a pure FP perspective. This will then be used in subsequent chapters to discuss \textit{why} one would follow an FP approach, identifying its benefits, advantages and also drawbacks over object-oriented approaches. 

%An implementation of an ABS must solve two fundamental problems:
%
%\begin{enumerate}
%	\item \textbf{Source of pro-activity} How can an agent initiate actions without the external stimuli of messages?
%	\item \textbf{Semantics of Messaging} When is a message \textit{m}, sent by agent \textit{A} to agent \textit{B}, visible and processed by \textit{B}?
%\end{enumerate}
%
%In computer systems, pro-activity, the ability to initiate actions on its own without external stimuli, is only possible when there is some internal stimulus, most naturally represented by a continuous increasing time-flow. Due to the discrete nature of computer-system, this time-flow must be discretized in steps as well and each step must be made available to the agent, acting as the internal stimulus. This allows the agent then to perceive time and become pro-active depending on time. So we can understand an ABS as a discrete time-simulation where time is broken down into continuous, real-valued or discrete natural-valued time-steps. Independent of the representation of the time-flow we have the two fundamental choices whether the time-flow is local to the agent or whether it is a system-global time-flow. Time-flows in computer-systems can only be created through threads of execution where there are two ways of feeding time-flow into an agent. Either it has its own thread-of-execution or the system creates the illusion of its own thread-of-execution by sharing the global thread sequentially among the agents where an agent has to yield the execution back after it has executed its step. Note the similarity to an operating system with cooperative multitasking in the latter case and real multi-processing in the former.
%
%The semantics of messaging define when sent messages are visible to the receivers and when the receivers process them. Message-processing could happen either immediately or delayed, depending on how message-delivery works. There are two ways of message-delivery: immediate or queued. In the case of immediate message-deliver the message is sent directly to the agent without any queuing in between e.g. a direct method-call. This would allow an agent to immediately react to this message as this call of the method transfers the thread-of-execution to the agent. This is not the case in the queued message-delivery where messages are posted to the message-box of an agent and the agent pro-actively processes the message-box at regular points in time.

% PART II: Towards pure functional ABS
\epigraphhead[450]{}
\part{Implementation techniques}
\label{part:implementation}
\chapter{Pure Functional Time-Driven ABS}
\label{ch:timedriven}

In the following, we derive a pure functional approach for an agent-based SIR model in which we pose solutions to the previously mentioned problems. We start out with a first approach in Yampa and show its limitations. Then we generalise it to a more powerful approach, which utilises Monadic Stream Functions, a generalisation of FRP. Finally we add a structured environment, making the example more interesting and showing the real strength of ABS over other simulation methodologies like System Dynamics and Discrete Event Simulation \footnote{The code of all steps can be accessed freely through the following URL: \url{https://github.com/thalerjonathan/phd/tree/master/public/purefunctionalepidemics/code}}.

\subsection{The SIR model}
\label{sec:sir_model}

The explanatory SIR model is a thoroughly studied and well understood compartment model from epidemiology \cite{kermack_contribution_1927}, which allows simulation of the dynamics of an infectious disease like influenza, tuberculosis, chicken pox, rubella and measles spreading through a population. The reason for choosing this model is its simplicity. It is easy to understand fully but complex enough to develop basic concepts of pure functional ABS, which are then extended and deepened in the much more complex Sugarscape model explained in the next section.

In this model, people in a population of size $N$ can be in either one of three states: \textit{Susceptible}, \textit{Infected} or \textit{Recovered}, at any particular time. It is assumed that initially there is at least one infected person in the population. People interact \textit{on average} with a given rate of $\beta$ other people per time unit, and become infected with a given probability $\gamma$ when interacting with an infected person. When infected, a person recovers \textit{on average} after $\delta$ time units and is then immune to further infections. An interaction between infected persons does not lead to reinfection, thus these interactions are ignored in this model. This definition gives rise to three compartments with the transitions seen in Figure \ref{fig:sir_transitions}.

\begin{figure}
	\centering
	\includegraphics[width=.7\textwidth, angle=0]{./fig/timedriven/SIR_transitions.png}
	\caption[States and transitions in the SIR compartment model]{States and transitions in the SIR compartment model.}
	\label{fig:sir_transitions}
\end{figure}

This model was also formalized using System Dynamics \cite{porter_industrial_1962}. In System Dynamics a system is modelled through differential equations, which allow expressing continuous systems, changing over time. They are solved by numerically integrating over time, which gives rise to the respective dynamics. The SIR model is modelled using the following equation, with the dynamics shown in Figure \ref{fig:sir_sd_dynamics} .

\begin{equation}
\begin{aligned}
\frac{\mathrm d S}{\mathrm d t} = -infectionRate \\
\frac{\mathrm d I}{\mathrm d t} = infectionRate - recoveryRate \\
\frac{\mathrm d R}{\mathrm d t} = recoveryRate 
\end{aligned}
\end{equation}

\begin{equation}
\begin{aligned}
infectionRate = \frac{I \beta S \gamma}{N} \\
recoveryRate = \frac{I}{\delta} 
\end{aligned}
\end{equation}

\begin{figure}
	\centering
	\includegraphics[width=0.5\textwidth, angle=0]{./fig/timedriven/SIR_SD_1000agents_150t_001dt.png}
	\caption[Dynamics of the SIR compartment model using the System Dynamics approach]{Dynamics of the SIR compartment model using the System Dynamics approach. Population Size $N$ = 1,000, contact rate $\beta =  \frac{1}{5}$, infection probability $\gamma = 0.05$, illness duration $\delta = 15$ with initially 1 infected agent. Simulation run for 150 time steps. Generated using our pure functional System Dynamics approach (see Appendix \ref{app:sd_simulation}).}
	\label{fig:sir_sd_dynamics}
\end{figure}

The approach of mapping the SIR model to an ABS is to discretise the population and model each person in the population as an individual agent. The transitions between the states are happening due to discrete events caused both by interactions amongst the agents and timeouts. The major advantage of ABS over System Dynamics is that it allows for the incorporation of spatiality and heterogeneity of a population, for example accounting for different sexes and ages. This is not directly possible with other simulation methods of System Dynamics or Discrete Event Simulation \cite{zeigler_theory_2000}.

This is directly related to a networked SIR model, where the interactions between agents are restricted by either a statically fixed or dynamically evolving network. Various network types exist, allowing for simulation of various scenarios. Very small communities where all agents are in contact with each other are modelled by a fully connected network. Real world scenarios where a few agents act as hubs are modelled by complex networks \cite{BarabasiAlbert_EmergenceScaling, Jackson2008, Newman_ComplexNetworks, WattsStrogatz_DynamicsSmallWorld}. In this thesis we do not impose restrictions on the connections among agents and always assume a fully connected network. Adding various network types to our thesis would unnecessarily complicate things in the beginning but would not constitute anything fundamentally new in terms of research. However, the use of complex networks, which in general are generated randomly, constitute an interesting direction for further research especially in the context of randomised property-based testing in ABS, which we discuss in Chapters \ref{ch:agentspec} and \ref{ch:sir_invariants}.

In the ABS classification of \cite{macal_everything_2016}, this model can be seen as an \textit{Interactive ABMS}: agents are individual heterogeneous agents with diverse set characteristics; they have autonomic, dynamic, endogenously defined behaviour; interactions happen between other agents and the environment through observed states, behaviours of other agents and the state of the environment.

\section{First step: pure computation}
\label{sec:timedriven_firststep}
As described in Chapter \ref{sec:back_frp}, Arrowized FRP \cite{hughes_generalising_2000} is a way to implement systems  with continuous and discrete time-semantics where the central concept is the signal function, which can be understood as a process over time, mapping an input- to an output-signal. Technically speaking, a signal function is a continuation which allows to capture state using closures and hides away the $\Delta t$, which means that it is never exposed explicitly to the programmer, meaning it cannot be manipulated. As already pointed out, agents need to perceive time, which means that the concept of processes over time is an ideal match for our agents and our system as a whole, thus we will implement them and the whole system as signal functions.

We start by defining the SIR states as ADT and our agents as signal functions (SF) which receive the SIR states of all agents form the previous step as input and outputs the current SIR state of the agent. This definition, and the fact that Yampa is not monadic, guarantees already at compile, that the agents are isolated from each other, enforcing the \textit{parallel} lock-step semantics of the model.

\begin{HaskellCode}
data SIRState = Susceptible | Infected | Recovered

type SIRAgent = SF [SIRState] SIRState 

sirAgent :: RandomGen g => g -> SIRState -> SIRAgent
sirAgent g Susceptible = susceptibleAgent g
sirAgent g Infected    = infectedAgent g
sirAgent _ Recovered   = recoveredAgent
\end{HaskellCode}

Depending on the initial state we return the corresponding behaviour. Note that we are passing a random-number generator instead of running in the Random Monad because signal functions as implemented in Yampa are not capable of being monadic. 

We see that the recovered agent ignores the random-number generator because a recovered agent does nothing, stays immune forever and can not get infected again in this model. Thus a recovered agent is a consuming state from which there is no escape, it simply acts as a sink which returns constantly \textit{Recovered}:

\begin{HaskellCode}
recoveredAgent :: SIRAgent
recoveredAgent = arr (const Recovered)
\end{HaskellCode}

Next, we implement the behaviour of a susceptible agent. It makes contact \textit{on average} with $\beta$ other random agents. For every \textit{infected} agent it gets into contact with, it becomes infected with a probability of $\gamma$. If an infection happens, it makes the transition to the \textit{Infected} state. To make contact, it gets fed the states of all agents in the system from the previous time-step, so it can draw random contacts - this is one, very naive way of implementing the interactions between agents.

Thus a susceptible agent behaves as susceptible until it becomes infected. Upon infection an \textit{Event} is returned, which results in switching into the \textit{infectedAgent} SF, which causes the agent to behave as an infected agent from that moment on. When an infection event occurs we change the behaviour of an agent using the Yampa combinator \textit{switch}, which is quite elegant and expressive as it makes the change of behaviour at the occurrence of an event explicit. Note that to make contact \textit{on average}, we use Yampas \textit{occasionally} function which requires us to carefully select the right $\Delta t$ for sampling the system as will be shown in results. 

Note the use of \textit{iPre :: a $\rightarrow$ SF a a}, which delays the input signal by one sample, taking an initial value for the output at time zero. The reason for it is that we need to delay the transition from susceptible to infected by one step due to the semantics of the \textit{switch} combinator: whenever the switching event occurs, the signal function into which is switched will be run at the time of the event occurrence. This means that a susceptible agent could make a transition to recovered within one time-step, which we want to prevent, because the semantics should be that only one state-transition can happen per time-step.

\begin{HaskellCode}
susceptibleAgent :: RandomGen g => g -> SIRAgent
susceptibleAgent g 
    = switch 
      -- delay switching by 1 step to prevent against transition
      -- from Susceptible to Recovered within one time-step
      (susceptible g >>> iPre (Susceptible, NoEvent)) 
      (const (infectedAgent g))
  where
    susceptible :: RandomGen g => g -> SF [SIRState] (SIRState, Event ())
    susceptible g = proc as -> do
      makeContact <- occasionally g (1 / contactRate) () -< ()
      if isEvent makeContact
        then (do
          -- draw random element from the list
          a <- drawRandomElemSF g -< as
          case a of
            Infected -> do
              -- returns True with given probability
              i <- randomBoolSF g infectivity -< ()
              if i
                then returnA -< (Infected, Event ())
                else returnA -< (Susceptible, NoEvent)
             _       -> returnA -< (Susceptible, NoEvent))
        else returnA -< (Susceptible, NoEvent)
\end{HaskellCode}

To deal with randomness in an FRP way, we implemented additional signal functions built on the \textit{noiseR} function provided by Yampa. This is an example for the stream character and statefulness of a signal function as it allows to keep track of the changed random-number generator internally through the use of continuations and closures. Here we provide the implementation of \textit{randomBoolSF}. \textit{drawRandomElemSF} works similar but takes a list as input and returns a randomly chosen element from it:

\begin{HaskellCode}
randomBoolSF :: RandomGen g => g -> Double -> SF () Bool
randomBoolSF g p = proc _ -> do
  r <- noiseR ((0, 1) :: (Double, Double)) g -< ()
  returnA -< (r <= p)
\end{HaskellCode}

An infected agent recovers \textit{on average} after $\delta$ time units. This is implemented by drawing the duration from an exponential distribution \cite{borshchev_system_2004} with $\lambda = \frac{1}{\delta}$ and making the transition to the \textit{Recovered} state after this duration. Thus the infected agent behaves as infected until it recovers, on average after the illness duration, after which it behaves as a recovered agent by switching into \textit{recoveredAgent}. As in the case of the susceptible agent, we use the \textit{occasionally} function to generate the event when the agent recovers. Note that the infected agent ignores the states of the other agents as its behaviour is completely independent of them.

\begin{HaskellCode}
infectedAgent :: RandomGen g => g -> SIRAgent
infectedAgent g 
    = switch 
      -- delay switching by 1 step 
      (infected >>> iPre (Infected, NoEvent))
      (const recoveredAgent)
  where
    infected :: SF [SIRState] (SIRState, Event ())
    infected = proc _ -> do
      recEvt <- occasionally g illnessDuration () -< ()
      let a = event Infected (const Recovered) recEvt
      returnA -< (a, recEvt)
\end{HaskellCode}

For running the simulation we use Yampas function \textit{embed}:

\begin{HaskellCode}
runSimulation :: RandomGen g => g -> Time -> DTime -> [SIRState] -> [[SIRState]]
runSimulation g t dt as 
    = embed (stepSimulation sfs as) ((), dts)
  where
    steps     = floor (t / dt)
    dts       = replicate steps (dt, Nothing)
    n         = length as
    (rngs, _) = rngSplits g n [] -- unique rngs for each agent
    sfs       = zipWith sirAgent rngs as
\end{HaskellCode}

What we need to implement next is a closed feedback-loop - the heart of every agent-based simulation. Fortunately, \cite{nilsson_functional_2002, courtney_yampa_2003} discusses implementing this in Yampa. The function \textit{stepSimulation} is an implementation of such a closed feedback-loop. It takes the current signal functions and states of all agents, runs them all in parallel and returns this step's new agent states. Note the use of \textit{notYet}, which is required because in Yampa switching occurs immediately at $t = 0$. If we don't delay the switching at $t = 0$ until the next step, we would enter an infinite switching loop - \textit{notYet} simply delays the first switching until the next time-step.

\begin{HaskellCode}
stepSimulation :: [SIRAgent] -> [SIRState] -> SF () [SIRState]
stepSimulation sfs as =
    dpSwitch
      -- feeding the agent states to each SF
      (\_ sfs' -> (map (\sf -> (as, sf)) sfs'))
      -- the signal functions
      sfs
      -- switching event, ignored at t = 0         
      (switchingEvt >>> notYet)
      -- recursively switch back into stepSimulation         
      stepSimulation                            
  where
    switchingEvt :: SF ((), [SIRState]) (Event [SIRState])
    switchingEvt = arr (\ (_, newAs) -> Event newAs)
\end{HaskellCode}

Yampa provides the \textit{dpSwitch} combinator for running signal functions in parallel, which has the following type-signature:

\begin{HaskellCode}
dpSwitch :: Functor col
         -- routing function
         => (forall sf. a -> col sf -> col (b, sf))
         -- SF collection
         -> col (SF b c)
         -- SF generating switching event     
         -> SF (a, col c) (Event d)
         -- continuation to invoke upon event           
         -> (col (SF b c) -> d -> SF a (col c))
         -> SF a (col c)
\end{HaskellCode}

Its first argument is the pairing-function, which pairs up the input to the signal functions - it has to preserve the structure of the signal function collection. The second argument is the collection of signal functions to run. The third argument is a signal function generating the switching event. The last argument is a function, which generates the continuation after the switching event has occurred. \textit{dpSwitch} returns a new signal function, which runs all the signal functions in parallel and switches into the continuation when the switching event occurs. The d in \textit{dpSwitch} stands for decoupled which guarantees that it delays the switching until the next time-step: the function into which we switch is only applied in the next step, which prevents an infinite loop if we switch into a recursive continuation.

Conceptually, \textit{dpSwitch} allows us to recursively switch back into the \textit{stepSimulation} with the continuations and new states of all the agents after they were run in parallel. 

\subsection{Results}
The dynamics generated by this step can be seen in Figure \ref{fig:sir_abs_dynamics_frp}. 

\begin{figure}
\begin{center}
	\begin{tabular}{c c}
		\begin{subfigure}[b]{0.4\textwidth}
			\centering
			\includegraphics[width=1\textwidth, angle=0]{./fig/timedriven/SIR_Yampa/SIR_Yampa_dt01.png}
			\caption{$\Delta t = 0.1$}
			\label{fig:sir_abs_approximating_01dt_1000agents}
		\end{subfigure}
		
		&
    	
		\begin{subfigure}[b]{0.4\textwidth}
			\centering
			\includegraphics[width=1\textwidth, angle=0]{./fig/timedriven/SIR_Yampa/SIR_Yampa_dt001.png}
			\caption{$\Delta t = 0.01$}
			\label{fig:sir_abs_approximating_001dt_1000agents}
		\end{subfigure}
	\end{tabular}
	
	\caption{FRP simulation of agent-based SIR showing the influence of different $\Delta t$. Population size of 1,000 with contact rate $\beta = \frac{1}{5}$, infection probability $\gamma = 0.05$, illness duration $\delta = 15$ with initially 1 infected agent. Simulation run for 150 time-steps with respective $\Delta t$.} 
	\label{fig:sir_abs_dynamics_frp}
\end{center}
\end{figure}

By following the FRP approach we assume a continuous flow of time, which means that we need to select a \textit{correct} $\Delta t$, otherwise we would end up with wrong dynamics. The selection of a correct $\Delta t$ depends in our case on \textit{occasionally} in the \textit{susceptible} behaviour, which randomly generates an event on average with \textit{contact rate} following the exponential distribution. To arrive at the correct dynamics, this requires us to sample \textit{occasionally}, and thus the whole system, with small enough $\Delta t$ which matches the frequency of events generated by \textit{contact rate}. If we choose a too large $\Delta t$, we loose events, which will result in wrong dynamics as can be seen in Figure \ref{fig:sir_abs_approximating_01dt_1000agents}. This issue is known as under-sampling and is described in Figure \ref{fig:sampling_issue}.

\begin{figure}
\begin{center}
	\begin{tabular}{c}
		\begin{subfigure}[b]{0.5\textwidth}
			\centering
			\includegraphics[width=1\textwidth, angle=0]{./fig/timedriven/undersampling.png}
			\caption{Under-sampling}
			\label{fig:undersampling}
		\end{subfigure}
		
		\\
		
		\begin{subfigure}[b]{0.5\textwidth}
			\centering
			\includegraphics[width=1\textwidth, angle=0]{./fig/timedriven/supersampling.png}
			\caption{Super-sampling}
			\label{fig:supersampling}
		\end{subfigure}
	\end{tabular}
	
	\caption{A visual explanation of under-sampling and super-sampling. The black dots represent the time-steps of the simulation. The red dots represent virtual events which occur at specific points in continuous time. In the case of under-sampling, 3 events occur in between the two time steps but \textit{occasionally} only captures the first one. By increasing the sampling frequency either through a smaller $\Delta t$ or super-sampling all 3 events can be captured.} 
	\label{fig:sampling_issue}
\end{center}
\end{figure}

For tackling this issue we have three options. The first one is to use a smaller $\Delta t$ as can be seen in \ref{fig:sir_abs_approximating_001dt_1000agents}, which results in the whole system being sampled more often, thus reducing performance. The second option is to step the simulation with $\Delta t = 1$ and in each step, instead of using \textit{occasionally}, to make a number of contacts drawn from the exponential distribution. Note that if we follow this option, we abandon the time-driven approach altogether because we don't abstract away from $\Delta t$ and violate the fundamental abstraction of FRP which assumes that time is continuous and signal functions are running conceptually infinitely fast and infinitely often \cite{winograd-cort_wormholes:_2012}. We will come back to this approach in the even-driven approach to ABS in Chapter \ref{sec:eventdriven_sir}. This leaves us with the third option to implement super-sampling and apply it to \textit{occasionally}, which allows us then to run the whole simulation with $\Delta t = 1.0$ and only sample the \textit{occasionally} function with a much higher frequency.

In Yampa there exists a function \textit{embed} which allows to run a given signal-function with provided $\Delta t$ but the problem is that this function does not really help because it does not return a signal-function. What we need is a signal-function which takes the number of super-samples \textit{n}, the signal-function \textit{sf} to sample and returns a new signal-function which performs super-sampling on it. We provide a full implementation of such a function, which also gives an insight into how signal functions are implemented in Yampa:

\begin{HaskellCode}
import FRP.Yampa.InternalCore

-- SF is the signal-function defined for time t = 0 and returns
-- a continuation of type SF' which is the signal-function 
-- defined for t > 0: it receives an additional time-delta
-- data SF a b  = SF { sfTF :: a -> (SF' a b, b) }
-- data SF' a b = DTime -> a -> (SF' a b, b)

superSampling :: Int -> SF a b -> SF a [b]
superSampling n sf0 = SF { sfTF = tf0 }
  where
    -- no supersampling at time 0
    tf0 :: a -> (SF' a b, [b])
    tf0 a0 = (tfCont, [b0])
      where
        (sf', b0) = sfTF sf0 a0 -- running a SF
        tfCont    = superSamplingAux sf'

    superSamplingAux :: SF' a [b]
    superSamplingAux sf' = SF' tf
      where
      	tf0 :: DTime -> a -> (SF' a b, [b])
        tf dt a = (tf', bs)
          where
            (sf'', bs) = superSampleRun n dt sf' a
            tf'        = superSamplingAux sf''

    superSampleRun :: Int -> DTime -> SF' a b -> a -> (SF' a b, [b])
    superSampleRun n dt sf a 
        | n <= 1    = superSampleMulti 1 dt sf a []
        | otherwise = (sf', reverse bs)  -- reverse due to accumulator
      where
        superDt = dt / fromIntegral n
        (sf', bs) = superSampleMulti n superDt sf a []

    superSampleMulti :: Int -> DTime -> SF' a b -> a -> [b] -> (SF' a b, [b])
    superSampleMulti 0 _ sf _ acc  = (sf, acc)
    superSampleMulti n dt sf a acc = superSampleMulti (n-1) dt sf' a (b:acc) 
      where
        (sf', b) = sfTF' sf dt a -- running a SF'
\end{HaskellCode}

It evaluates the \textit{SF} argument for \textit{n} times, each with $\Delta t = \frac{\Delta t}{n}$ and the same input argument \textit{a} for all \textit{n} evaluations. At time 0 no super-sampling is performed and just a single output of the \textit{SF} argument is calculated. A list of \textit{b} is returned with length of \textit{n} containing the result of the \textit{n} evaluations of the \textit{SF} argument. If 0 or less super samples are requested exactly one is calculated. We could then wrap the occasionally function which would then generate a list of events. 

\subsection{Discussion}
We can conclude that our first step already introduced most of the fundamental concepts of ABS:
\begin{itemize}
	\item Time - the simulation occurs over virtual time which is modelled explicitly, divided into \textit{fixed} $\Delta t$, where at each step all agents are executed.
	\item Agents - we implement each agent as an individual, with the behaviour depending on its state. It is clear to see that agents behave as signals: when the system is sampled with $\Delta t = 0$ then their behaviour will stay constant and won't change because it is completely determined by the flow of time. 
	\item Feedback - the output state of the agent in the current time-step $t$ is the input state for the next time-step $t + \Delta t$.
	\item Environment - as environment we implicitly assume a fully-connected network (complete graph) where every agent 'knows' every other agent, including itself and thus can make contact with all of them.
	\item Stochasticity - it is an inherently stochastic simulation, which is indicated by the random-number generator and the usage of \textit{occasionally}, \textit{randomBoolSF} and \textit{drawRandomElemSF}.
	\item Deterministic - repeated runs with the same initial random-number generator result in same dynamics. This may not come as a surprise but in Haskell we can guarantee that property statically already at compile time because our simulation runs \textit{not} in the IO Monad. This guarantees that no external, uncontrollable sources of non-determinism can interfere with the simulation.
	\item Parallel, lock-step semantics - the simulation implements a \textit{parallel} update-strategy where in each step the agents are run isolated in parallel and don't see the actions of the others until the next step.
\end{itemize}

Using FRP in the instance of Yampa results in a clear, expressive and robust implementation. State is implicitly encoded, depending on which signal function is active. By using explicit time-semantics with \textit{occasionally} we can achieve extremely fine grained stochastics by sampling the system with small $\Delta t$: we are treating it as a truly continuous time-driven agent-based system.

A very severe problem, hard to find with testing but detectable with in-depth validation analysis, is the fact that in the \textit{susceptible} agent the same random-number generator is used in \textit{occasionally}, \textit{drawRandomElemSF} and \textit{randomBoolSF}. This means that all three stochastic functions, which should be independent from each other, are inherently correlated. This is something one wants to prevent under all circumstances in a simulation, as it can invalidate the dynamics on a very subtle level, and indeed we have tested the influence of the correlation in this example and it has an impact. We left this severe bug in for explanatory reasons, as it shows an example where functional programming actually encourages very subtle bugs if one is not careful. A possible but not very elegant solution would be to simply split the initial random-number generator in \textit{sirAgent} three times (using one of the splited generators for the next split) and pass three random-number generators to \textit{susceptible}. A much more elegant solution would be to use the Random Monad which is not possible because Yampa is not monadic.

So far we have an acceptable implementation of an agent-based SIR approach. What we are lacking at the moment is a general treatment of an environment and an elegant solution to the random number correlation. In the next step we make the transition to Monadic Stream Functions as introduced in Dunai \cite{perez_functional_2016}, which allows FRP within a monadic context and gives us a way for an elegant solution to the random number correlation.

\section{Second step: going monadic}
\label{sec:timedriven_monadic}

A part of the library Dunai is BearRiver, a wrapper re-implementing Yampa on top of Dunai, which allows us to easily replace Yampa with MSFs. This will enable us to run arbitrary monadic computations in a signal function, solving the problem of correlated random numbers through the use of the \textit{Rand} Monad.

\subsection{Identity Monad}
We start by making the transition to BearRiver by simply replacing Yampas signal function by BearRivers', which is the same but takes an additional type parameter \textit{m}, indicating the monadic context. If we replace this type parameter with the \textit{Identity} Monad, we should be able to keep the code exactly the same, because BearRiver re-implements all necessary functions we are using from Yampa. We simply re-define the agent signal function, introducing the monad stack our SIR implementation runs in:

\begin{HaskellCode}
type SIRMonad = Identity
type SIRAgent = SF SIRMonad [SIRState] SIRState
\end{HaskellCode}

\subsection{Rand Monad}
Using the \textit{Identity} Monad does not gain us anything but it is a first step towards a more general solution. Our next step is to replace the \textit{Identity} Monad by the \textit{Rand} Monad, which will allow us to run the whole simulation within the \textit{Rand} Monad with the full features of FRP, finally solving the problem of correlated random numbers in an elegant way. We start by re-defining the SIRMonad and SIRAgent:

\begin{HaskellCode}
type SIRMonad g = Rand g
type SIRAgent g = SF (SIRMonad g) [SIRState] SIRState
\end{HaskellCode}

To access the \textit{Rand} Monad functionality within the MSF context, overloaded functions are used. For the function \textit{occasionally}, there exists a monadic pendant \textit{occasionallyM} which requires a \textit{MonadRandom} type class. Because we are now running within a \textit{MonadRandom} instance we simply replace \textit{occasionally} with \textit{occasionallyM}. 

\begin{HaskellCode}
occasionallyM :: MonadRandom m => Time -> b -> SF m a (Event b)
-- can be used through the use of arrM and lift
randomBoolM :: RandomGen g => Double -> Rand g Bool
-- this can be used directly as a SF with the arrow notation
drawRandomElemSF :: MonadRandom m => SF m [a] a
\end{HaskellCode}

\subsection{Discussion} 
Running in the Random Monad elegantly solves the problem of correlated random numbers and guarantees that we will not have correlated stochastics as discussed in the previous section. In the next step we introduce the concept of an explicit discrete 2D environment.

\section{Third Step: Adding an environment}
\label{sec:adding_env}
So far we have implicitly assumed a fully connected network amongst agents, where each agent can see and knows every other agent. This is a valid environment and in accordance with the System Dynamics inspired implementation of the SIR model but does not show the real advantage of ABS to situate agents within arbitrary environments. Often, agents are situated within a discrete 2D environment \cite{epstein_growing_1996} which is simply a finite $N x M$ grid with either a Moore or von Neumann neighbourhood (Figure \ref{fig:abs_neighbourhoods}). Agents are either static or can move freely around with cells allowing either single or multiple occupants.

We can directly map the SIR model to a discrete 2D environment by placing the agents on a corresponding 2D grid with an unrestricted neighbourhood. The behaviour of the agents is the same but they select their interactions directly from the shared read-only environment, which will be passed to the agents as input. This allows agents to read the states of all their neighbours, which tells them if a neighbour is infected or not. To show the benefit over the System Dynamics approach  and for purposes of a more interesting approach, we restrict the neighbourhood to Moore (Figure \ref{fig:moore_neighbourhood}).

\begin{figure}
\begin{center}
	\begin{tabular}{c c}
		\begin{subfigure}[b]{0.3\textwidth}
			\centering
			\includegraphics[width=0.5\textwidth, angle=0]{./fig/timedriven/neumann.png}
			\caption{von Neumann}
			\label{fig:neumann_neighbourhood}
		\end{subfigure}
    	&
		\begin{subfigure}[b]{0.3\textwidth}
			\centering
			\includegraphics[width=0.5\textwidth, angle=0]{./fig/timedriven/moore.png}
			\caption{Moore}
			\label{fig:moore_neighbourhood}
		\end{subfigure}
    \end{tabular}
	\caption{Common neighbourhoods in discrete 2D environments of Agent-Based Simulation.}
	\label{fig:abs_neighbourhoods}
\end{center}
\end{figure}

We also implemented this spatial approach in Java using the well known ABS library RePast \cite{north_complex_2013}, to have a comparison with a state of the art approach and came to the same results as shown in Figure \ref{fig:sir_dunai}. This supports, that our pure functional approach can produce such results as well and compares positively to the state of the art in the ABS field.

\subsection{Implementation}
We start by defining the discrete 2D environment for which we use an indexed two dimensional array. Each cell stores the agent state of the last time-step, thus we use the \textit{SIRState} as type for our array data. Also, we re-define the agent signal function to take the structured environment \textit{SIREnv} as input instead of the list of all agents as in our previous approach. As output we keep the \textit{SIRState}, which is the state the agent is currently in. Also we run in the \textit{Rand} Monad as introduced before to avoid the random number correlation. 

\begin{HaskellCode}
type Disc2dCoord = (Int, Int)
type SIREnv      = Array Disc2dCoord SIRState

type SIRAgent g  = SF (Rand g) SIREnv SIRState
\end{HaskellCode}

Note that the environment is not returned as output because the agents do not directly manipulate the environment but only read from it. Again, this enforces the semantics of the \textit{parallel} update-strategy through the types where the agents can only see the previous state of the environment and see the actions of other agents reflected in the environment only in the next step.

Note that we could have chosen to use a \textit{StateT} transformer with the \textit{SIREnv} as state, instead of passing it as input, with the agents then able to arbitrarily read/write, but this would have violated the semantics of our model because actions of agents would have become visible within the same time-step.

The implementation of the susceptible, infected and recovered agents are almost the same with only the neighbour querying now slightly different. 

Stepping the simulation needs a new approach because in each step we need to collect the agent outputs and update the environment for the next step. For this we implemented a separate MSF, which receives the coordinates for every agent to be able to update the state in the environment after the agent was run. Note that we need use \textit{mapM} to run the agents because we are running now in the context of the \textit{Rand} Monad. This has the consequence that the agents are in fact run sequentially one after the other but because they cannot see the other agents actions nor observe changes in the shared read-only environment, it is \textit{conceptually} a \textit{parallel} update-strategy where agents run in lock-step, isolated from each other at conceptually the same time.
  
\begin{HaskellCode}
simulationStep :: RandomGen g => [(SIRAgent g, Disc2dCoord)]
               -> SIREnv -> SF (Rand g) () SIREnv
simulationStep sfsCoords env = MSF (\_ -> do
  let (sfs, coords) = unzip sfsCoords 
  -- run agents sequentially but with shared, read-only environment
  ret <- mapM (`unMSF` env) sfs
  -- construct new environment from all agent outputs for next step
  let (as, sfs') = unzip ret
      env' = foldr (\ (a, coord) envAcc -> updateCell coord a envAcc) 
               env (zip as coords)

      sfsCoords' = zip sfs' coords
      cont       = simulationStep sfsCoords' env'
  return (env', cont))
 
updateCell :: Disc2dCoord -> SIRState -> SIREnv -> SIREnv
\end{HaskellCode}

\subsection{Results}
We implemented rendering of the environments using the gloss library which enabled us to cycle arbitrarily through the steps and inspect the spreading of the disease over time visually as seen in Figure \ref{fig:sir_dunai}.

\begin{figure}
\begin{center}
	\begin{tabular}{c c}
		\begin{subfigure}[b]{0.4\textwidth}
			\centering
			\includegraphics[width=1\textwidth, angle=0]{./fig/timedriven/SIR_Dunai/SIR_Dunai_dt001_environment.png}
			\caption{Environment at $t = 50$}
			\label{fig:sir_dunai_env}
		\end{subfigure}
    	
    	&
  
		\begin{subfigure}[b]{0.43\textwidth}
			\centering
			\includegraphics[width=1\textwidth, angle=0]{./fig/timedriven/SIR_Dunai/SIR_Dunai_dt001.png}
			\caption{Dynamics over time}
			\label{fig:sir_dunai_env_dynamics}
		\end{subfigure}
	\end{tabular}
	
	\caption{Simulating the agent-based SIR model on a 21x21 2D grid with Moore neighbourhood (Figure \ref{fig:moore_neighbourhood}), a single infected agent at the center and same SIR parameters as in Figure \ref{fig:sir_sd_dynamics}. Simulation run until $t = 200$ with fixed $\Delta t = 0.01$. Last infected agent recovers around $t = 194$. The susceptible agents are rendered as blue hollow circles for better contrast.}
	\label{fig:sir_dunai}
\end{center}
\end{figure}

Note that the dynamics of the spatial SIR simulation, which are seen in Figure \ref{fig:sir_dunai_env_dynamics}, look quite different from the reference dynamics of Figure \ref{fig:sir_sd_dynamics}. This is due to a much more restricted neighbourhood, resulting in far fewer infected agents at a time and a lower number of recovered agents at the end of the epidemic, meaning that fewer agents got infected overall.

\subsection{Discussion}
Introducing a structured environment with a Moore neighbourhood, shows the ability of ABS to place the heterogeneous agents in a generic environment, which is the fundamental advantage of an agent-based approach over other simulation methodologies and allows us to simulate much more realistic scenarios.

Note, that an environment is not restricted to be a discrete 2D grid and can be anything from a continuous N-dimensional space to a complex network - one only needs to change the type of the environment and agent input and provide corresponding neighbourhood querying functions. 

\section{Discussion}

\subsection{Other Models}
TODO: mention that we have also implemented other models, which also work without time-semantics (all agents make a move at discrete time-steps and do not really rely on some notion of time). 

\subsection{Time-Semantics}
The main reason for building our pure functional ABMS approach on top of Yampa was to leverage the powerful time-semantics of Yampa which allows us to implement important concepts of ABMS:

state-chart: agents are at all time of their life-cycle in one state and can switch between multiple states using transitions 
timed transitions: transition to another state/behaviour happens at a discrete time
rate transitions: transition happens with a given rate
message transition: transition upon receiving a given message 

\subsection{Agents as Signals}
Due to the underlying nature and motivation of Functional Reactive Programming (und im speziellen) Yampa, Agents can be seen as Signals which is generated and consumed by a Signal-Function which is the behaviour of an Agent. If an Agent does not change the OUTPUT-signal is constant, if the agent changes e.g. by sending a message, changing its state,... the OUTPUT signal changes. A dead agent has no signal at all.

\subsection{Time-Sampling}
sampling rate depends on the transition times \& rates of the model. when e.g. the contact rate is 5 then the sampling dt should be below 0.2

\subsection{System Dynamics}
can emulate system dynamics due to the parallel update-strategy and continuous time-flow semantics

\subsection{Discrete Event Simulation}
DES in FrABMS? how easily can we implement server/queue systems? do they also look like a specification? potential problem: ordering of messages is not guaranteed by now

\subsection{Advantages}
advantages:
	- no side-effects within agents leads to much safer code
	- edsl for time-semantics
	- declarative style: agent-implementation looks like a model-specification
	- reasoning and verification
	- sequential and parallel
	- powerful time-semantics
	- arrowized programming is optional and only required when utilizing yampas time-semantics. if the model does not rely on time-semantics, it can use monadic-programming by building on the existing monadic functions in the EDSL which allow to run in the State-Monad which simplifies things very much
	- when to use yampas arrowized programing: time-semantics, simple state-chart agents 
	- when not using yampas facilities: in all the other cases e.g. SugarScape is such a case as it proceeds in unit time-steps and all agents act in every time-step
	- can implement System Dynamics building on Yampas facilities with total ease	
	- get replications for free without having to worry about side-effects and can even run them in parallel without headaches
	- cant mess around with time because delta-time is hidden from you (intentional design-decision by Yampa). this would be only very difficult and cumbersome to achieve in an object-oriented approach. TODO: experiment with it in Java - how could we actually implement this? I think it is impossible: may only achieve this through complicated application of patterns and inheritance but then has the problem of how to update the dt and more important how to deal with functions like integral which accumulates a value through closures and continuations. We could do this in OO by having a general base-class e.g. ContinuousTime which provides functions like updateDt and integrate, but we could only accumulate a single integral value.
	- reproducibility statically guaranteed
	- cannot mess around with dt
	- code == specification
	- rule out serious class of bugs
	- different time-sampling leads to different results e.g. in wildfire \& SIR but not in Prisoners Dilemma. why? probabilistic time-sampling?
	- reasoning about equivalence between SD and ABS implementation in the same framework
	- recursive implementations
	
	- we can statically guarantee the reproducibility of the simulation because: no side effects possible within the agents which would result in differences between same runs (e.g. file access, networking, threading), also timedeltas are fixed and do not depend on rendering performance or userinput	
	
\subsection{Disadvantages}
disadvantages:
	- performance is low
	- reasoning about performance is very difficult
	- very steep learning curve for non-functional programmers
	- learning a new EDSL
	- think ABMS different: when to use async messages, when to use sync conversations


[ ] important: increasing sampling freqzency and increasing number of steps so that the same number of simulation steps are executed should lead to same results. but it doesnt. why?
[ ] hypothesis: if time-semantics are involved then event ordering becomes relevant for emergent patterns. there are no tine semantics in heroes and cowards but in the prisoners dilemma
[ ] can we implement different types of agents interacting with each other in the same simulation ? with different behaviour funcs, digferent state? yes, also not possible in NetLogo to my knowledge. but they must have the same messages, emvironment 

[ ] Hypothesis: we can combine with FrABS agent-based simulation and system dynamics (this has been proved by example!)
\chapter{Pure Functional Event-Driven ABS}
\label{ch:eventdriven}

In this chapter we build on the previous discussion of update-strategies in Chapter \ref{ch:impl_abs} and the implementation techniques presented in the time-driven approach of Chapter \ref{ch:timedriven} to develop concepts for event-driven ABS in a pure functional way. 

In event-driven ABS \cite{meyer_event-driven_2014}, the simulation is advanced through events: agents and the environment schedule events into the future and react to incoming events scheduled by themselves, other agents, the environment or the simulation kernel. Time is discrete in this approach: it advances step-wise from event to event where each event has an accociated time-stamp which indicates the virtual simulation time when it is scheduled. This implies that time could stay constant e.g. when an event is scheduled with a time-delay of 0 the virtual simulation time does not advance. Because agents can adopt and change their state and behaviour when processing an event this means that even if time does not advance, agents can change. This non-signal behaviour is the fundamental difference to the time-driven approach in Chapter \ref{ch:timedriven}. Further, we exploit this mechanism to implement direct agent-interactions in pure functional ABS as discussed in our use-case of the Sugarscape model.

The event-driven approach makes the simulation kernel technically quite close to a Discrete Event Simulation (DES) \cite{zeigler_theory_2000}. Due to the necessity of imposing a correct ordering of events in this type of ABS, we are forced to step it event by event, with the \textit{sequential} update-strategy. Note that there exists also Parallel DES (PDES) \cite{fujimoto_parallel_1990}, which processes events in parallel and deals with inconsistencies by reverting to consistent states - we hypothesize that a pure functional approach could be beneficial due to persistent data-structures and explicit handling of side-effects but we leave this for further research.

We use the Sugarscape model to develop pure functional concepts for event-driven ABS \footnote{The code of all steps can be accessed freely through the following URL: \url{https://github.com/thalerjonathan/phd/tree/master/public/towards/SugarScape/sequential}}. We chose this model for the following reasons: it is quite well known in the ABS community; it was highly influential in sparking the interest in ABS; it is quite complex with non-trivial agent-interactions; the original implementation used object-oriented techniques (Objective C) and explicitly advocates them as a good fit to ABS which begged the question whether and how well a pure functional implementation is possible. The Sugarscape model is not a classic event-driven model as in it the agents do schedule events they don't do this into the future - events in Sugarscape don't have associated time-stamps. Still the underlying concepts are the same as in event-driven ABS and it is trivial to add time-stamps, moving towards a real event-driven ABS with DES character.

\section{Case-Study II: Sugarscape}
TODO: 
we can implement everything except synchronous direct agent-interactions atm: if agent-interaction is one-way e.g. paying back a loan then this is no problem. thus the following parts of the Sugarscape are not possible with our current STM approach: mating, trading and lending  because all 3 require direct agent-to-agent interaction over multiple steps. We leave the problem of developing such an algorithm / implementation for further research.

\section{Synchronised Agent-Interactions}
Following towards papers SugarScape implementation

% direct MSF call, Problem is recursive nature. maybe try it with gintis Implementation. I just learned that what i want to achieve is actually: https://en.wikipedia.org/wiki/This_(computer_programming)#Open_recursion AND OPEN RECURSION IS PRETTY BAD


\section{Discussion}

\subsection{Other Models}
TODO: mention that we have also implemented other models, which also work without time-semantics (all agents make a move at discrete time-steps and do not really rely on some notion of time). 

\subsection{Time-Semantics}
The main reason for building our pure functional ABMS approach on top of Yampa was to leverage the powerful time-semantics of Yampa which allows us to implement important concepts of ABMS:

state-chart: agents are at all time of their life-cycle in one state and can switch between multiple states using transitions 
timed transitions: transition to another state/behaviour happens at a discrete time
rate transitions: transition happens with a given rate
message transition: transition upon receiving a given message 

\subsection{Agents as Signals}
Due to the underlying nature and motivation of Functional Reactive Programming (und im speziellen) Yampa, Agents can be seen as Signals which is generated and consumed by a Signal-Function which is the behaviour of an Agent. If an Agent does not change the OUTPUT-signal is constant, if the agent changes e.g. by sending a message, changing its state,... the OUTPUT signal changes. A dead agent has no signal at all.

\subsection{Time-Sampling}
sampling rate depends on the transition times \& rates of the model. when e.g. the contact rate is 5 then the sampling dt should be below 0.2

\subsection{System Dynamics}
can emulate system dynamics due to the parallel update-strategy and continuous time-flow semantics

\subsection{Discrete Event Simulation}
DES in FrABMS? how easily can we implement server/queue systems? do they also look like a specification? potential problem: ordering of messages is not guaranteed by now

\subsection{Advantages}
advantages:
	- no side-effects within agents leads to much safer code
	- edsl for time-semantics
	- declarative style: agent-implementation looks like a model-specification
	- reasoning and verification
	- sequential and parallel
	- powerful time-semantics
	- arrowized programming is optional and only required when utilizing yampas time-semantics. if the model does not rely on time-semantics, it can use monadic-programming by building on the existing monadic functions in the EDSL which allow to run in the State-Monad which simplifies things very much
	- when to use yampas arrowized programing: time-semantics, simple state-chart agents 
	- when not using yampas facilities: in all the other cases e.g. SugarScape is such a case as it proceeds in unit time-steps and all agents act in every time-step
	- can implement System Dynamics building on Yampas facilities with total ease	
	- get replications for free without having to worry about side-effects and can even run them in parallel without headaches
	- cant mess around with time because delta-time is hidden from you (intentional design-decision by Yampa). this would be only very difficult and cumbersome to achieve in an object-oriented approach. TODO: experiment with it in Java - how could we actually implement this? I think it is impossible: may only achieve this through complicated application of patterns and inheritance but then has the problem of how to update the dt and more important how to deal with functions like integral which accumulates a value through closures and continuations. We could do this in OO by having a general base-class e.g. ContinuousTime which provides functions like updateDt and integrate, but we could only accumulate a single integral value.
	- reproducibility statically guaranteed
	- cannot mess around with dt
	- code == specification
	- rule out serious class of bugs
	- different time-sampling leads to different results e.g. in wildfire \& SIR but not in Prisoners Dilemma. why? probabilistic time-sampling?
	- reasoning about equivalence between SD and ABS implementation in the same framework
	- recursive implementations
	
	- we can statically guarantee the reproducibility of the simulation because: no side effects possible within the agents which would result in differences between same runs (e.g. file access, networking, threading), also timedeltas are fixed and do not depend on rendering performance or userinput	
	
\subsection{Disadvantages}
disadvantages:
	- performance is low
	- reasoning about performance is very difficult
	- very steep learning curve for non-functional programmers
	- learning a new EDSL
	- think ABMS different: when to use async messages, when to use sync conversations


[ ] important: increasing sampling freqzency and increasing number of steps so that the same number of simulation steps are executed should lead to same results. but it doesnt. why?
[ ] hypothesis: if time-semantics are involved then event ordering becomes relevant for emergent patterns. there are no tine semantics in heroes and cowards but in the prisoners dilemma
[ ] can we implement different types of agents interacting with each other in the same simulation ? with different behaviour funcs, digferent state? yes, also not possible in NetLogo to my knowledge. but they must have the same messages, emvironment 

[ ] Hypothesis: we can combine with FrABS agent-based simulation and system dynamics (this has been proved by example!)

% PART III: Why 1: Parallel Computation
\epigraphhead[450]{}
\part{Parallel computation}
\label{part:parallel}
\chapter{Parallel ABS}
\label{ch:parallel_abs}
Functional programming as in Haskell is well known and accepted as a remedy against the problems of imperative programming in implementing parallel software TODO: cite ?. The reason for it is clear: immutable data and explicit control of side-effects removes a large class of bugs due to data-conflicts, data-races, and blablabla TODO: we are claiming things here, which we need to clearly back up, also data-races ARE possible in Haskell! A fundamental benefit and strength of Haskell is, that it clearly distinguishes between parallelism and concurrency \cite{jones_tackling_2002} and it is very important for us to do so as well:

\begin{itemize}
	\item \textbf{Parallelism} - In parallelism, code runs in parallel without interfering with other code through shared data (references, mutexes, semaphores,...). An example is the function \textit{map :: (a $\rightarrow$ b) $\rightarrow$ [a] $\rightarrow$ [b]}, which maps each element of type \textit{a} to \textit{b} using the function \textit{(a $\rightarrow$ b)}. It is a pure function and thus no sharing of data either through some monadic context or through the function \textit{(a $\rightarrow$ b)} is possible. This allows to run it in parallel: each function evaluation \textit{(a $\rightarrow$ b)} could potentially be executed at the same time, if we had enough CPU cures. Whether it runs actually in parallel or not, has no influence on the outcome, it is not subject to any non-deterministic influences. Thus we identify parallelism with pure and deterministic execution of data-transformations (data-parallelism).
	
	\item \textbf{Concurrency} - In concurrency, code runs in parallel but can potentially interfere with other code through shared data (references, mutexes, semaphores, ...). An example are two threads, running in parallel, which share data through \textit{IORefs}. In concurrency there is no option: code has to run in parallel through the use of threads but now the outcome of the program very much depends on the ordering in which the threads are scheduled. This gives rise to very different access patterns to the shared data, with the potential for race conditions, dirty reads and so on... Ordering suddenly becomes important and the challenge of implementing concurrent programs, is to write the program in a way that despite of these non-deterministic influences it is still a correctly working program. Thus we identify concurrency with impure and non-deterministic execution of imperative-style (ordered) monadic command execution.
\end{itemize}

In this chapter we investigate the application of both parallelism and concurrency to our pure functional ABS approach. In general, we want to test if the benefits of parallel and concurrent programming in Haskell are transferable to pure functional ABS. In particular we are interested in speeding up the existing implementations and derive general concepts from that. Note that we use the term \textit{parallel} to identify both \textit{parallelism} and \textit{concurrency} and we distinguish between them whenever necessary with their respective terms.

\section{Parallelism in ABS}
The promise of parallelism in Haskell is compelling: speeding up the execution but retaining all static compile-time guarantees about determinism. In other words, using parallelism could give us a substantial performance improvement without sacrificing the static guarantees of reproducible outputs from repeated runs with initial conditions.

Generally, parallelism can be applied whenever the execution of code is order-independent, that is referential transparent, and has no implicit or explicit side-effects. Without going into too much technical detail, in this section we outline the parallelism techniques available in Haskell and briefly discuss how they can be used in ABS in general. We also discuss if and how parallelism can be added to our previously discussed use-cases of Chapters \ref{sec:timedriven_firststep}, \ref{sec:adding_env} and Sugarscape TODO and report the performance improvements where applicable.

\subsection{Parallelism in Haskell}
parallelism in haskell builds on laziness

We follow the book \cite{marlow_parallel_2013}, which can be seen as the main source for parallelism and concurrency in Haskell and refer to it for an in-depth discussions of parallel Haskell.

\paragraph{Evaluation parallelism}
The basis are the following functions: \textit{rpar :: a -> Eval a} and \textit{rseq :: a -> Eval a}, where Eval is a Monad which can be run with \textit{runEval :: Eval a -> a}. rpar runs the computation in parallel and immediately returns without waiting for the evaluation of the thunk - this will happen behind the sences. rseq runs the computation in parallel as well but waits for the evaluation to WFNF. Using this we can start evaluating multiple expressions in parallel with rpar and then wait for their result with rseq. Note that both cases evaluate their argument to weak head normal form (WHNF), thus if the argument is already in WHNF, then the computation does nothing. This becomes important when to understand how far we can to in evaluation of parallelism. TODO: need to say a bit about haskell as a lazy language. TODO: isnt this all the very basics of haskells parallelism together with lazy evaluation?

Put short, evaluation parallelism allows to build functions which run in parallel e.g. a parallel version of \textit{map}, which is called \textit{parMap}. This is achieved using the \textit{Eval Monad} which is run using a parallel evaluation strategy, arriving at a pure value - the evaluation of the \textit{Eval Monad} itself is pure and does not require the IO (this is exactly what we expect from parallelism: to be deterministic). Obviously this gives huge potential for speeding up programs because maps are omnipresent in a lot of functional code. Not only parmap! explain a little bit more in detail without going into too much technical stuff.


Very important: "In the previous two chapters, we looked at the Eval monad and Strategies, which work in conjunction with lazy evaluation to express parallelism. A Strategy consumes a lazy data structure and evaluates parts of it in parallel. This model has some advantages: it allows the decoupling of the algorithm from the parallelism, and it allows parallel evaluation strategies to be built compositionally. But Strategies and Eval are not always the most convenient or effective way to express parallelism. We might not want to build a lazy data structure, for example. Lazy evaluation brings the nice modularity properties that we get with Strategies, but on the flip side, lazy evaluation can make it tricky to understand and diagnose performance."
Its laziness which allows that.


%https://www.oreilly.com/library/view/parallel-and-concurrent/9781449335939/ch02.html
%https://www.oreilly.com/library/view/parallel-and-concurrent/9781449335939/ch03.html

\paragraph{Data-flow parallelism}
Par Monad: how does it work? can express data-flow networks where tasks are forked and then results are synchronised. all this happens deterministically by building on the same mechanics the Eval monad is using thus technically speaking they are equivalent. 

can use both par and eval monad but which is applicable? par seems to require strict data, eval works on lazy data-structure. can we use both inside an msf? what is the Advantage over simple rpar or rseq?\\

%https://www.oreilly.com/library/view/parallel-and-concurrent/9781449335939/ch04.html

\paragraph{Data-structure parallelism}
An environment could be organised and accessed through such a data-structure, which could potentially lead to big speed ups. Agents could locally read the data-structure data-parallel and the simulation kernel could feed the output of the agents data-parallel back into this structure.

%https://learning.oreilly.com/library/view/parallel-and-concurrent/9781449335939/ch05.html

general solution we opt for is  to run agents in parallel in our approaches. in other abs models we could apply data-structure parallelism and/or data-flow parallelism with huge Performance potential but thats always highly model dependent thus we dont go in depth here

\subsection{Use-Cases}

\subsubsection{Non-Monadic SIR}
Although \textit{parMap} can be applied in all cases where a map us used, we are particularly interested in running agents in parallel. With \textit{parMap} this should become possible in our non-monadic SIR implementation built on Yampa from Chapter \ref{sec:timedriven_firststep}. Even thought the Eval Monad is used under the hood and Yampa is non-monadic, it is still applicable because running the monad is pure, resulting in a pure result - \textit{parMap} is a pure function. TODO: how can we apply this?

Inspired by the work of \cite{perez_60_2014}, which shows the potential of speeding up real-world Haskell programs using Yampa We conducted a comparison of an implementation which makes use of evaluation parallelism to run agents in parallel.

OK, rephrase: compare performance of non-parallel implementation WITH threaded an -N option to non-parallel implementation without threaded and / or N1 to make sure that no performance improvement happens automatically by using threaded e.g. GCs or something else...
I observed the behaviour in the following code: https://github.com/thalerjonathan/phd/tree/master/public/purefunctionalepidemics/code/SIR_Yampa

I analysed a bit more using the threadscope tool. I ran the same program twice with different ghc-options:
1. -O2 -Wall -Werror -eventlog 
2. -O2 -Wall -Werror -eventlog -threaded -with-rtsopts=-N

When looking at the event logs with threadscope it becomes appartent, that parallel garbage collection is the cause of the CPU usage above 100%:
-  In the single-threaded case 0 sparks are created and everything runs indeed only on one core. There are two Garbage Collectors (Gen0 and Gen1) but nothing runs in parallel (Par collections are 0 for both).
- In the multi-threaded case also 0 sparks are created but now 8 cores are used: all 'running' activity happens on only 1 core as expected but garbage collection happens on all 8 cores: the diagrams and the number of Par collections clearly indicates that. The time spent on parallel GC work is 10.76% (0 is completely serial and 100% is completely parallel).

Now when we compare the timing between both runs we see the following: 
- single-threaded: 11.68s total, 7.35s mutator, 4.34s GC,
- multi-threaded: 10.70s total, 7.03s mutator, 3.68s GC

This adds up: the ~ 10\% of parallel GC work done in multi-threaded are also the ~ 10\% it is faster over the single-threaded one. Of course I only did a single run in each case but I think the analysis is still valid and the point was made: when running a Haskell program which does not use any parallel features, running it with the -threaded option can lead to an increase in performance due to parallel GC.

% https://www.reddit.com/r/haskell/comments/2jbl78/from_60_frames_per_second_to_500_in_haskell/\\

The idea:
Using compositional parallelism we can add evaluation parallelism for agent execution, without needing to re-implement dpSwitch. Also re-implementing switch functions would not get us very far because of WHNF evaluation it is the wrong end to start parallel evaluation: probably only the arguments would be evaluated but not the agent behaviour. The solution is to add evaluation parallelism in the agent-output collection phase: where the recursive switch into the step Simulation function happens. There we use a parMap to evaulate the outputs of all agents in parallel simply using a parMap with id, which due to compositional parallelism because of lazy evaluation, should then run the whole agent when it is evaluated in parallel because the output is forced in parallel evaluation.

Our use case:
Unfortunately in our non-monadic Yampa implementation we see a negligible speedup of less than 10\% between running it on 1 or 8 cores and this difference is probably due to garbage collection. When analysing the problem more in-depth it becomes clear that 50\% of the parallel evaluation sparks (todo explain) are duplications and get never evaluated, which is due to the thunk being already evaluated before thus no need to run it actually in parallel. Unfortunately this seems reasonable in this example: the way the agent-behaviour is implemented forces the values, including the output, due to lots of comparisons, which results basically in a strict behaviour with the output already evaluated for many agents. It seems that it depends on the current state the agent is in otherwise we could not explain why some sparks are duplications and others not. Further it seems, that although work happens in parallel, the overhead eats up the benefit and thus we arrive at roughly the same performance of the non-parallel version. This might be completely different for much more computational intensive agent behaviour with a more complex agent-output data-structure - but we leave this for further research.

\subsubsection{Monadic SIR}
Unfortunately \textit{parMap} is not applicable to the monadic SIR version of Chapter \ref{sec:adding_env} because of the use of mapM, which cannot be replaced due to its inherent sequential nature: mapM runs monadic actions which have side-effects thus ordering matters. Even if the implementation in that chapter behaves as if the agents are run in parallel, technically they are run sequentially because of the need for the Random Monad effect. This leaves us basically without any options of parallelism for the monadic SIR model, we will come back to this use-case in the concurrency section, where we will show that by using concurrency it is possible to achieve a substantial speed up of orders of magnitude.

TODO: the marlow book says: don't do repeated calls to runPar, so although the agent can in fact do that it should avoid it and if there is heavy parallel work in each agent then one should consider running the agent in the par monad with a single runPar outside

NOTE: running the agents in parallel with par doesn't work because we use mapM and are thus monadic, which involves sequencing. so this is really out of the window here. Also we cannot put a Par in a transformer stack because the library doesn't support it, what actually makes sense. But we can do the following: we can run an agents MSF only within the Par monad which gives agents the ability to spawn data-flow parallel computations - random-number streams are handled like in the non-monadic version. Note that this is only possible with the MSFs of dunai and not the SF because the latter one adds already the a ReaderT DTime which makes it impossible already. 
What is actually possible would be to write a combined monad for Par and ReaderT because the latter one is a read-only value and could thus potentially run in parallel - we leave this for further research. There exists also a combination of the Par with the Rand monad, so if the time-driven approach is not needed then this could be used to give the agents the ability to both draw random numbers AND do deterministic data-parallel computations. The agents can then be run in parallel through the par monad.

TODO: try the same thing as in the non-monadic SIR: parMap rpar id evaluating the output. Hypothesis is that it hould not show any parallelism because of monadic code.

\subsubsection{Sugarscape}
The same case as in the monadic SIR: running the agents with evaluation or data-flow parallelism is not possible in a monadic context  \textit{in general}. We have shortly discussed how it could be achieved in specific circumstances where then agents are running in the Par monad only, but this is highly model specific and for the Sugarscape this approach does not work. 

There is though one tiny thing we could optimise.

use parmap for updating Pollution/regrow resources. still agents can't be run in parallel because of monadic effects, we show in the concurrent section how we can use concurrency to achieve a substantial speed up using STM.

compare Environment parallelism between sequential and concurrent sugarscape: should see alarger speedup in conc bcs the sequential percentage is larger there

unfortunately its a Map datastructure, so we cannot operate in parallel e.g. map. but we can compute pollution because it uses map

\subsection{Parallel Runs}
Often one needs to perform a large number of runs of the same simulation. The most prominent use-cases for this are:

\begin{itemize}
	\item Parameter Sweeps / Variations - To explore the parameter space and the dynamics under varying parameter configurations, the same simulation is run with varying parameters and the results recorded for statistical analysis.
	
	\item Stochastic replications - Due to ABS stochastic nature, running a simulation only once does not allow to generalise or predict overall behaviour - one might have just hit an (un)fortunate special case. To counter this problem, in ABS multiple replications of the  simulation are run with same initial model parameters but with different random-number streams. All the results are collected and analysed stochastically (averaged, median,...) from which then more general properties can be derived.
\end{itemize}

In each case thousands of runs of the same simulation with different model parameters and / or varying random-number streams are needed, requiring a considerable amount of computing power.

Parallelism is a remedy to this problem because in each of these cases individual runs do not interfere with each other and thus can be seen as isolated from each other, like referential, pure computations. Our approaches shown in the Part II make this very explicit: the top level functions can always be made pure computations because we are ruling out IO (so far) and thus even though Monads are employed in many cases, they are still pure. A benefit of our approach is that it is guaranteed at compile time, that individual runs do not interfere with each other and thus there is no danger that parallel runs influence each other. 

All this allows to implement parameter sweeps and stochastic replications both through evaluation and data-flow parallelism making another very compelling use-case - probably the most striking one - for the use of parallelism in ABS. We hypothesize that data-flow parallelism is better suited for this task because it makes parallelism more explicit as it is indeed a data-flow problem: we pass parameters to single replications which are run and return their results. To apply this we simply run the top level replication logic in the Par Monad where replications are run in parallel by forking tasks and results are handed back through IVars. If we want the convenience of having a monadic random-number generator within the Par monad as well, one can use the combined ParRand monad which provides both.

\subsection{Reflection}
In general we aimed at running agents in parallel using the various techniques. Because of the quite sequential nature of the agent behaviours themselves, there is much less potential for parallelism \textit{within} an agent, thus the obvious idea was to run them all in parallel because they are an obvious unit of partitioning, have considerable workload and can indeed be run in parallel under given circumstances.
Unfortunately it is not possible applying parallelism in case the agents run within a monadic context: we have side-effects which imposes ordering e.g. in the case of a

We see a direct consequence of this that types also reflect the semantics of our model: when our agents are pure they can be run indeed in parallel and independent from each other, if they are monadic, then this is not applicable to parallelism. In the next section, we show how to approach this problem and come up with a solution where we can run monadic agents in parallel. This is obviously only possible within a concurrent setting which means we have to sacrifice determinism in our solution. Still we reach considerable speed ups using Software Transactional Memory.

\section{Concurrent ABS}
In an ideal world, we would like to solve all our problems using parallelism but unfortunately, it can't be applied to all parallel problems and ABS is no exception. As soon as there are data-dependencies, like we have them in the Sugarscape model in the form of the read/write environment and synchronous agent-interactions, we cannot avoid concurrency.

Traditional approaches to concurrency follow a lock-based approach, where sections which access shared data are synchronised through synchronisation primitives like mutexes, semaphores, monitors,... The lock-based path is a well trodden one, with all problems and benefits well established. In this chapter we want to follow a different path and look into using Software Transactional Memory (STM) for implementing concurrent ABS, which promises to overcome the problems of lock-based approaches. Although STM exists in other languages as well, Haskell was one of the first to natively build it into its core, thus it is a natural choice to follow that direction when already investigating pure functional ABS.

Unfortunately, as soon as we employ concurrency, we lose all static guarantees about reproducibility and the use of STM is no exception. Still, STM has the unique benefit that it can guarantee the lack of persistent side-effects at compile time, allowing unproblematic retries of transactions, something of fundamental importance in STM as will be described below. This has also another \textit{very} compelling advantage of STM over unrestricted lock-based approaches: by using STM, we can reduce the side-effects allowed substantially and guarantee at compile time, that the differences between runs of same initial conditions will only stem from the fact that we run the simulation concurrently - \textit{and from nothing else}. All this makes the use of STM together with Haskell very compelling and to our best knowledge we are the very first to investigate the use of STM for implementing concurrent ABS in a systematic way.

The paper \cite{discolo_lock_2006} gives a good indication how difficult and complex constructing a correct concurrent program is and shows how much easier, concise and less error-prone an STM implementation is over traditional locking with mutexes and semaphores. More important, it shows that STM consistently outperforms the lock-based implementations. We follow this work and compare the performance of lock-based and STM implementations and hypothesise that the reduced complexity and increased performance will be directly applicable to ABS as well.

We present two case studies using the already introduced SIR (Chapter \ref{sec:sir_model}) and Sugarscape (Chapter \ref{sec:sugarscape}) models. We compare the performance of lock-based and STM implementations in each case where we investigate both the scaling performance under increasing number of CPUs and under increasing number of agents. We show that the STM implementations consistently outperform the lock-based ones and scale much better to increasing number of CPUs both on local machines and on Amazon Cloud Services.

\section{Software Transactional Memory}
Software Transactional Memory (STM) was introduced by \cite{shavit_software_1995} in 1995 as an alternative to lock-based synchronisation in concurrent programming which, in general, is notoriously difficult to get right. This is because reasoning about the interactions of multiple concurrently running threads and low level operational details of synchronisation primitives is \textit{very hard}. The main problems are:

\begin{itemize}
	\item Race conditions due to forgotten locks;
	\item Deadlocks resulting from inconsistent lock ordering;
	\item Corruption caused by uncaught exceptions;
	\item Lost wake-ups induced by omitted notifications.
\end{itemize}

Worse, concurrency does not compose. It is very difficult to write two functions (or methods in an object) acting on concurrent data which can be composed into a larger concurrent behaviour. The reason for it is that one has to know about internal details of locking, which breaks encapsulation and makes composition dependent on knowledge about their implementation. Therefore, it is impossible to compose two  functions e.g. where one withdraws some amount of money from an account and the other deposits this amount of money into a different account: one ends up with a temporary state where the money is in none of either accounts, creating an inconsistency - a potential source for errors because threads can be rescheduled at any time.

STM promises to solve all these problems for a low cost by executing actions \textit{atomically}, where modifications made in such an action are invisible to other threads and changes by other threads are invisible as well until actions are committed - STM actions are atomic and isolated. When an STM action exits, either one of two outcomes happen: if no other thread has modified the same data as the thread running the STM action, then the modifications performed by the action will be committed and become visible to the other threads. If other threads have modified the data then the modifications will be discarded, the action block rolled-back and automatically restarted.

STM in Haskell is implemented using optimistic synchronisation, which means that instead of locking access to shared data, each thread keeps a transaction log for each read and write to shared data it makes. When the transaction exits, the thread checks whether it has a consistent view to the shared data or not: whether other threads have written to memory it has read. % This might look like a serious overhead but the implementations are very mature by now, being very performant and the benefits outweigh its costs by far.

In the paper \cite{heindl_modeling_2009} the authors use a model of STM to simulate optimistic and pessimistic STM behaviour under various scenarios using the AnyLogic simulation package. They conclude that optimistic STM may lead to 25\% less retries of transactions. The authors of \cite{perfumo_limits_2008} analyse several Haskell STM programs with respect to their transactional behaviour. They identified the roll-back rate as one of the key metric which determines the scalability of an application. Although STM might promise better performance, they also warn of the overhead it introduces which could be quite substantial in particular for programs which do not perform much work inside transactions as their commit overhead appears to be high.

\subsection{STM in Haskell}
The work of \cite{harris_composable_2005, harris_transactional_2006} added STM to Haskell, which was one of the first programming languages to incorporate STM into its main core and added the ability to composable operations. There exist various implementations of STM in other languages as well (Python, Java, C\#, C/C++, etc) but we argue, that it is in Haskell with its type-system and the way how side-effects are treated where it truly shines.

In the Haskell implementation, STM actions run within the \textit{STM} context. This restricts the operations to only STM primitives as shown below, which allows to enforce that STM actions are always repeatable without persistent side-effects because such persistent side-effects (e.g. writing to a file, launching a missile) are not possible in an \textit{STM} context. This is also the fundamental difference to  \textit{IO}, where all bets are off because \textit{everything} is possible as there are basically no restrictions because \textit{IO} can run everything.

Thus the ability to \textit{restart} a block of actions without any visible effects is only possible due to the nature of Haskells type-system: by restricting the effects to STM only, ensures that no uncontrolled effects, which cannot be rolled-back, occur.

STM comes with a number of primitives to share transactional data. Amongst others the most important ones are:

\begin{itemize}
	\item \textit{TVar} - A transactional variable which can be read and written arbitrarily;
	\item \textit{TArray} - A transactional array where each cell is an individual shared data, allowing much finer-grained transactions instead of e.g. having the whole array in a \textit{TVar};
	\item \textit{TChan} - A transactional channel, representing an unbounded FIFO channel;
	\item \textit{TMVar} - A transactional \textit{synchronising} variable which is either empty of full. To read from an empty or write to a full \textit{TMVar} will cause the current thread to retry its transaction.
\end{itemize}

% NOTE: too technical
%To run an \textit{STM} action the function \textit{atomically :: STM a $\to$ IO a} is provided, which can be seen as the STM effect-runner as it performs a series of \textit{STM} actions atomically within an \textit{IO} context. It takes the STM action which returns a value of type \textit{a} and returns an \textit{IO} action which returns a value of type \textit{a}. This \textit{IO} action can only be executed within an \textit{IO} context.

\section{STM in ABS}
\label{sec:stm_abs}
In this section we give a short overview of how we apply STM in our ABS. In both case-studies we fundamentally follow a time-driven, parallel approach as introduced in Chapter \ref{sub:par_strategy}, where the simulation is advanced by a given $\Delta t$ and in each step all agents are executed. To employ parallelism, each agent runs within its own thread and agents are executed in lock-step, synchronising between each $\Delta t$, which is controlled by the main thread. See Figure \ref{fig:stm_abs_structure} for a visualisation of our concurrent, time-driven lock-step approach.

By running each agent in a thread will guarantee the execution in parallel even if the agent has a monadic context. This is forces us to evaluate each agents monadic context separately instead of running them all in a common context. Note that ultimately we are ending up in the \textit{IO} context because \textit{STM} can be only transacted from within an \textit{IO} context due to non-deterministic side-effects. This is no contradiction to our original claim: yes we are running in IO but not the agent behaviour itself, which is a fundamental difference.

An agent thread will block until the main-thread sends the next $\Delta t$ and runs the \textit{STM} action atomically with the given $\Delta t$. When the \textit{STM} action has been committed, the thread will send the output of the agent action to the main-thread to signal it has finished. The main thread awaits the results of all agents to collect them for output of the current step e.g. visualisation or writing to a file.

As will be described in subsequent sections, central to both case-studies is an environment which is shared between the agents using a \textit{TVar} or \textit{TArray} primitive through which the agents communicate concurrently with each other. To get the environment in each step for visualisation purposes, the main thread can access the \textit{TVar} and \textit{TArray} as well. 

\begin{figure}
	\centering
	\includegraphics[width=1.0\textwidth, angle=0]{./fig/concurrentabs/stm_abs.png}
	\caption{Diagram of the parallel time-driven lock-step approach.}
	\label{fig:stm_abs_structure}
\end{figure}

\subsection{Adding and running the STM Monad}
We briefly show how to add STM to agents and run them within their own threads. We use the SIR implementation as example - applying it to the Sugarscape implementation works exactly the same way and is left as a trivial exercise to the reader.

The first step is to simply add the \textit{STM} to the existing transformer stack as the \textit{innermost} monad. The reason why we make it the innermost is to guarantee that in case of a retry \textit{all} outer monadic effects are retried as well - if the STM would be placed on a higher stack level, the levels below would not be subject to a retry. For monads like the \textit{ReaderT} this would not matter because they are read-only but for a StateT this fact would matter a lot. Note that STM does not provide a transformer instance, so this is not an option anyway. If STM would provide a transformer then we could make \textit{IO} the innermost monad and do \textit{IO} within STM, which should be prevented under all circumstances because then rolling back a transaction cannot guarantee to undo the effects. To better understand the semantics of retries consider the following example:

\begin{HaskellCode}

\end{HaskellCode}



\begin{HaskellCode}
innerSTMAction :: RandomGen g => StateT SomeState (RandT g STM) SomeResult

let randAction = runStateT innerSTMAction initState
let stmAction  = runRandT randAction (mkStdGen 42)
let ioAction   = atomically stmAction
((someResult, someState), g) <- ioAction
\end{HaskellCode}

In this case the STM is the \textit{innermost} monad thus it will be run last. This means that all the outer monads are subject to re-computation due to retries.

\begin{HaskellCode}
outerSTMAction :: STMT (StateT Environment (Rand g)) SomeResult

let ioAction = runSTMT outerSTMAction
stateAction <- ioAction
let randAction = runStateT stateAction initState
let ((someResult, someState), g) = runRandT randAction (mkStdGen 42)
\end{HaskellCode}

In this case, the STM is the \textit{outermost} monad, thus it will be run first. This means that it will return a StateT computation which will be computed \textit{after} the STM has transacted. The computation construction is subject to the retries but the computation itself won't be repeated in case of retries.

TODO: add STM

\begin{HaskellCode}
agentThread :: RandomGen g 
            => Int
            -> SIRAgent g
            -> g
            -> MVar SIRState
            -> MVar DTime
            -> IO ()
agentThread 0 _ _ _ _ = return () -- all steps computed, terminate thread
agentThread n sf rng retVar dtVar = do
  -- wait for dt to compute current step
  dt <- takeMVar dtVar

  -- compute output of current step
  let sfReader = unMSF sf ()
      sfRand   = runReaderT sfReader dt
      sfSTM    = runRandT sfRand rng
  ((ret, sf'), rng') <- atomically sfSTM -- run the STM action atomically within IO

  -- post result to main thread
  putMVar retVar ret
  
  -- to next step
  agentThread (n - 1) sf' rng retVar dtVar
\end{HaskellCode}

\begin{HaskellCode}
simulationStep :: TVar SIREnv
               -> [MVar DTime]
               -> [MVar SIRState]
               -> DTime
               -> IO SIREnv
simulationStep env dtVars retVars dt = do
  -- tell all threads to continue with the corresponding DTime
  mapM_ (`putMVar` dt) dtVars
  -- wait for results but ignore them, SIREnv contains all states
  mapM_ takeMVar retVars
  -- return state of environment when step has finished
  readTVarIO env
\end{HaskellCode}

\section{Discussion}

\subsection{Other Models}
TODO: mention that we have also implemented other models, which also work without time-semantics (all agents make a move at discrete time-steps and do not really rely on some notion of time). 

\subsection{Time-Semantics}
The main reason for building our pure functional ABMS approach on top of Yampa was to leverage the powerful time-semantics of Yampa which allows us to implement important concepts of ABMS:

state-chart: agents are at all time of their life-cycle in one state and can switch between multiple states using transitions 
timed transitions: transition to another state/behaviour happens at a discrete time
rate transitions: transition happens with a given rate
message transition: transition upon receiving a given message 

\subsection{Agents as Signals}
Due to the underlying nature and motivation of Functional Reactive Programming (und im speziellen) Yampa, Agents can be seen as Signals which is generated and consumed by a Signal-Function which is the behaviour of an Agent. If an Agent does not change the OUTPUT-signal is constant, if the agent changes e.g. by sending a message, changing its state,... the OUTPUT signal changes. A dead agent has no signal at all.

\subsection{Time-Sampling}
sampling rate depends on the transition times \& rates of the model. when e.g. the contact rate is 5 then the sampling dt should be below 0.2

\subsection{System Dynamics}
can emulate system dynamics due to the parallel update-strategy and continuous time-flow semantics

\subsection{Discrete Event Simulation}
DES in FrABMS? how easily can we implement server/queue systems? do they also look like a specification? potential problem: ordering of messages is not guaranteed by now

\subsection{Advantages}
advantages:
	- no side-effects within agents leads to much safer code
	- edsl for time-semantics
	- declarative style: agent-implementation looks like a model-specification
	- reasoning and verification
	- sequential and parallel
	- powerful time-semantics
	- arrowized programming is optional and only required when utilizing yampas time-semantics. if the model does not rely on time-semantics, it can use monadic-programming by building on the existing monadic functions in the EDSL which allow to run in the State-Monad which simplifies things very much
	- when to use yampas arrowized programing: time-semantics, simple state-chart agents 
	- when not using yampas facilities: in all the other cases e.g. SugarScape is such a case as it proceeds in unit time-steps and all agents act in every time-step
	- can implement System Dynamics building on Yampas facilities with total ease	
	- get replications for free without having to worry about side-effects and can even run them in parallel without headaches
	- cant mess around with time because delta-time is hidden from you (intentional design-decision by Yampa). this would be only very difficult and cumbersome to achieve in an object-oriented approach. TODO: experiment with it in Java - how could we actually implement this? I think it is impossible: may only achieve this through complicated application of patterns and inheritance but then has the problem of how to update the dt and more important how to deal with functions like integral which accumulates a value through closures and continuations. We could do this in OO by having a general base-class e.g. ContinuousTime which provides functions like updateDt and integrate, but we could only accumulate a single integral value.
	- reproducibility statically guaranteed
	- cannot mess around with dt
	- code == specification
	- rule out serious class of bugs
	- different time-sampling leads to different results e.g. in wildfire \& SIR but not in Prisoners Dilemma. why? probabilistic time-sampling?
	- reasoning about equivalence between SD and ABS implementation in the same framework
	- recursive implementations
	
	- we can statically guarantee the reproducibility of the simulation because: no side effects possible within the agents which would result in differences between same runs (e.g. file access, networking, threading), also timedeltas are fixed and do not depend on rendering performance or userinput	
	
\subsection{Disadvantages}
disadvantages:
	- performance is low
	- reasoning about performance is very difficult
	- very steep learning curve for non-functional programmers
	- learning a new EDSL
	- think ABMS different: when to use async messages, when to use sync conversations


[ ] important: increasing sampling freqzency and increasing number of steps so that the same number of simulation steps are executed should lead to same results. but it doesnt. why?
[ ] hypothesis: if time-semantics are involved then event ordering becomes relevant for emergent patterns. there are no tine semantics in heroes and cowards but in the prisoners dilemma
[ ] can we implement different types of agents interacting with each other in the same simulation ? with different behaviour funcs, digferent state? yes, also not possible in NetLogo to my knowledge. but they must have the same messages, emvironment 

[ ] Hypothesis: we can combine with FrABS agent-based simulation and system dynamics (this has been proved by example!)

% PART IV: Why 2: Property-Based Testing
\epigraphhead[450]{}
\part{Property-based testing}
\label{part:property}
TODO REFINE
- don't use resize population and always use normal size to be consistent and feed random population as random parameter
- use TimeRange and UnitRange consistently everywhere
- WORDING: don't use replications in terms of QuickCheck! we test properties with random test-cases
- IMPROVE EXPLANATION: more explanations, its too flat, code straight in the face, prepare it gently as it is not directly clear what the intention is. discuss it more generally, otherwise it reads like an advanced technical report and not like a thesis.
- explain more how and why and when we use cover and checkCoverage. Also when using cover we cannot use random model parameters as it changes distributions
- only use cover with/without checkCoverage instead of maxFailPercent => rework sugarscape hypotheses chapter; need not much change only remove maxFailPercent and use cover and run all cases again.

\bigskip

%*******************************************************************************
%*********************************** First Chapter *****************************
%*******************************************************************************

\chapter{Introduction}  %Title of the First Chapter
I noticed that it is pretty hard to convince an agent-based economics specialist who is not a computer scientist about a pure functional approach. My conjecture is that the implementation technique and method does not matter much to them because they have very little knowledge about programming and are almost always self-taught - they don't know about software-engineering, nothing about proper software-design and architecture, nothing about software-maintenance, nothing about unit-testing,... In the end they just "hack" the simulation in whatever language they are able to: C++, Visual Basic, Java or toolboxes like Netlogo. For them it is all about to \textit{get things done somehow} and not to get things done the right way or in a beautiful way - the way and the method doesn't matter, its just a necessary evil which needs to be done. Thus if functional programming could make their lives easier, then they will definitely welcome it. But functional programming is, i think, harder to learn and harder to understand - so one needs to provide an abstraction through EDSL. So I REALLY need to come up with convincing arguments why to use pure functional approaches in ACE THEY can understand, otherwise I will be lost and not heard (not published,...). \\

What ACE economists care for:

\begin{itemize}
\item Very: Qualitative modelling with quantitative results
\item Yes: Easy reproducibility
\item Likely: Reasoning about convergence?
\item Likely: EDSL
\end{itemize}

My contributions are: pure functional framework, functional agent-model for market-simulations, EDSL for market-simulations, qualitative / implicit modelling with quanitative results, reasoning in my framework about convergence \\

IDEA: could I develop non-causal modelling (models are expressed in terms of non-directed equations, modelled in signal-relations) to allow for qualitative modelling for the agent-based economists? See hybrid modelling paper of Yampa. \textbf{THIS WOULD BE A HUGE NOVEL CONTRIBUTION TO ACE ESPECIALLY WHEN COMBINED WITH AN EDSL AND PROVIDING FULL REFERENTIAL TRANSPARENCY TO KEEP THE ABILITY TO REASON ABOUT CONVERGENCE}. This should be covered in the "EDSL"-paper.

TODO: maybe i should really focus only on market models? otherwise too much? \\

central novelty of my PhD: model specification = runnable code. possible through EDSL. but only in specific subfield of ACE: market-models. need a functional description of the model, then translate it to model specification in EDSL and then run it to see dynamics. But: model specification moves closer to functional programming languages. \\

another novelty approach: model specification through qualitative instead of quantiative approaches. is this possible? \\

WHY FUNCTIONAL? "because its the ultimate approach to scientific computing": fewer bugs due to mutable state (why? is thos shown obkectively by someone?), shorter (again as above, productivity), more expressive and closer to math, EDSL, EDSL=model=simulation, better parallelising due to referental transparency, reasoning \\

scientific results need to be reproduced, especially when they have high impact. a more formal approach of specifying the model and the simulation (model=simulation) could lead to easier sharing and easier reporduction without ambigouites \\

pure functional agent-model \& theory, EDSL framework in Haskell for ACE

\begin{enumerate}
\item Which kind of problem do we have?
\item What aim is there? Solving the problem? 
\item How the aim is achieved by enumerating VERY CLEAR objectives.
\item What the impact one expects (hypothesis) and what it is (after results).
\end{enumerate}

Note: It is not in the interest of the researcher to develop new economic theories but to research the use of functional methods (programming and specification) in agent-based computational economics (ACE).

NOTE: Get the reader’s attention early in the introduction: motivation, significance, originality and novelty.

\section{Methods}
Methods need to be selected to implement the simulations. Special emphasis will be put on functional ones which will then be compared to established methods in the field of ABM/S and ACE. \\

Claim: non-programming environments are considered to be not powerful enough to capture the complexity of ACE implementations thus a programming approach to ACE will be always required.

\section{Scenarios}
To apply and test functional methods in ACE, four scenarios of ACE are selected and then the methods applied and compared with each other to see how each of them perform in comparison. The 4 selected scenarios represent a selection of the challenges posed in ACE: from very abstract ones to very operational ones.

\section{Comparison}
Each of the selected scenarios is then implemented using the selected methods where each solution is then compared against the following criteria: 

\begin{enumerate}
\item suitability for scientific computation
\item robustness
\item error-sources
\item testability
\item stability
\item extendability
\item size of code
\item maintainability
\item time taken for development
\item verification \& correctness
\item replications \& parallelism
\item EDSL
\end{enumerate}

This will then allow to compare the different methods against each other and to show under which circumstances functional methods shine and when they should not be used.

\section{Agent-Based Modelling and Simulation (ABM/S)}
ABM/S is a method of modelling and simulating a system where the global behaviour may be unknown but the behaviour and interactions of the parts making up the system is of knowledge (Wooldrige, M. (2009). An Introduction to MultiAgent Systems. John Wiley & Sons). Those parts, called agents, are modelled and simulated out of which then the aggregate global behaviour of the whole system emerges. Thus the central aspect of ABM/S is the concept of an Agent which can be understood as a metaphor for a pro-active unit, able to spawn new Agents, and interacting with other Agents in a network of neighbours by exchange of messages. The implementation of Agents can vary and strongly depends on the programming language and the kind of domain the simulation and model is situated in.

\section{Agent-Based Economics (ACE)}
According to Leigh Tesfatsion (Tesfatsion, L. (2006). Agent-based computational economics: A constructive approach to economic theory. In Tesfatsion, L. and Judd, K. L., editors, Handbook of Computational Economics, volume 2, chapter 16, pages 831–880. Elsevier, 1 edition.), one of the leading figures, ACE is "[...] computational modelling of economic processes (including whole economies) as open-ended dynamic systems of interacting agents." - thus lending perfectly to the use of ABM/S as already the name suggests. Whereas classical economic models fall short by only looking at the average, pure rational, individual interacting in anonymous markets, the ACE approach looks at heterogeneous, non-rational individuals interacting with each other in networks (Kirman, A. (2010). Complex Economics: Individual and Collective Rationality. Routledge, London ; New York, NY.). Thus ACE can be understood as a combination of computer-science, cognitive/social science and evolutionary economics.

\section{Functional programming}
TODO: read \cite{Backus1978}

The state-of-the-art approach to implementing Agents are object-oriented methods and programming as the metaphor of an Agent as presented above lends itself very naturally to object-orientation (OO). The author of this thesis claims that OO in the hands of inexperienced or ignorant programmers is dangerous, leading to bugs and hardly maintainable and extensible code. The reason for this is that OO provides very powerful techniques of organising and structuring programs through Classes, Type Hierarchies and Objects, which, when misused, lead to the above mentioned problems. Also major problems, which experts face as well as beginners are 1. state is highly scattered across the program which disguises the flow of data in complex simulations and 2. objects don’t compose as well as functions. The reason for this is that objects always carry around some internal state which makes it obviously much more complicated as complex dependencies can be introduced according to the internal state.
All this is tackled by (pure) functional programming which abandons the concept of global state, Objects and Classes and makes data-flow explicit. This then allows to reason about correctness, termination and other properties of the program e.g. if a given function exhibits side-effects or not. Other benefits are fewer lines of code, easier maintainability and ultimately fewer bugs thus making functional programming the ideal choice for scientific computing and simulation and thus also for ACE. A very powerful feature of functional programming is Lazy evaluation. It allows to describe infinite data-structures and functions producing an infinite stream of output but which are only computed as currently needed. Thus the decision of how many is decoupled from how to (Hughes, J. (1989). Why functional programming matters. Comput. J., 32(2):98–107.).
The most powerful aspect using pure functional programming however is that it allows the design of embedded domain specific languages (EDSL). In this case one develops and programs primitives e.g. types and functions in a host language (embed) in a way that they can be combined. The combination of these primitives then looks like a language specific to a given domain, in the case of this thesis ACE. The ease of development of EDSLs in pure functional programming is also a proof of the superior extensibility and composability of pure functional languages over OO (Henderson P. (1982). Functional Geometry. Proceedings of the 1982 ACM Symposium on LISP and Functional Programming.).
One of the most compelling example to utilize pure functional programming is the reporting of Hudak (Hudak P., Jones M. (1994). Haskell vs. Ada vs. C++ vs. Awk vs. ... An Experiment in Software Prototyping Productivity. Department of Computer Science, Yale University.)  where in a prototyping contest of DARPA the Haskell prototype was by far the shortest with 85 lines of code. Also the Jury mistook the code as specification because the prototype did actually implement a small EDSL which is a perfect proof how close EDSL can get to and look like a specification.

Functional languages can best be characterized by their way computation works: instead of \textit{how} something is computed, \textit{what} is computed is described. Thus functional programming follows a declarative instead of an imperative style of programming. The key points are:
\begin{itemize}
\item No assignment statements - variables values can never change once given a value.
\item Function calls have no side-effect and will only compute the results - this makes order of execution irrelevant, as due to the lack of side-effects the logical point in \textit{time} when the function is calculated within the program-execution does not matter.
\item higher-order functions
\item lazy evaluation
\item Looping is achieved using recursion, mostly through the use of the general fold or the more specific map.
\item Pattern-matching
\end{itemize}

This alone does not really explain the \textit{real} advantages of functional programming and one must look for better motivations using functional programming languages. One motivation is given in \cite{Hughes1989} which is a great paper explaining to non-functional programmers what the significance of functional programming is and helping functional programmers putting functional languages to maximum use by showing the real power and advantages of functional languages. The main conclusion is that \textit{modularity}, which is the key to successful programming, can be achieved best using higher-order functions and lazy evaluation provided in functional languages like Haskell. \cite{Hughes1989} argues that the ability to divide problems into sub-problems depends on the ability to glue the sub-problems together which depends strongly on the programming-language and \cite{Hughes1989} argues that in this ability functional languages are superior to structured programming.

TODO: comparison of functional and object-oriented programming. My points are:
\begin{itemize}
\item The way state can be changed and treated - distributed over multiple objects - is often very difficult to understand.
\item Inheritance is a dangerous thing if not used with care because inheritance introduces very strong dependencies which cannot be changed during runtime anymore.
\item Objects don't compose very well: \url{http://zeroturnaround.com/rebellabs/why-the-debate-on-object-oriented-vs-functional-programming-is-all-about-composition/}
\item (Nearly) impossible to reason about programs
\end{itemize}

In conclusion the upsides of functional programming as opposed to OO are:
\begin{itemize}
\item Much more explicit flow of data \& control
\item Much better compose-able
\item Much better parallelism
\end{itemize}

\section{Related Research}
Tim Sweeney, CTO of Epic Games gave an invited talk about how "future programming languages could help us write better code" by "supplying stronger typing, reduce run-time failures;  and the need for pervasive concurrency support, both implicit and explicit, to effectively exploit the several forms of parallelism present in games and graphics." \cite{Sweeney2006}. Although the fields of games and agent-based simulations seem to be very different in the end, they have also very important similarities: both are simulations which perform numerical computations and update objects - in games they are called "game-objects" and in abm they are called agents but they are in fact the same thing - in a loop either concurrently or sequential. His key-points were:

\begin{itemize}
\item Dependent types as the remedy of most of the run-time failures.
\item Parallelism for numerical computation: these are pure functional algorithms, operate locally on mutable state. Haskell ST, STRef solution enables encapsulating local heaps and mutability within referentially transparent code.
\item Updating game-objects (agents) concurrently using STM: update all objects concurrently in arbitrary order, with each update wrapped in atomic block - depends on collisions if performance goes up.
\end{itemize}

\section{Property-based testing}
\label{sec:proptesting}
Property-based testing allows to formulate \textit{functional specifications} in code which then a property-based testing library tries to falsify by \textit{automatically} generating test data, covering as much cases as possible. When a case is found for which the property fails, the library then reduces the test data to its simplest form for which the test still fails, for example shrinking a list to a smaller size. It is clear to see that this kind of testing is especially suited to ABS, because we can formulate specifications, meaning we describe \textit{what} to test instead of \textit{how} to test. Also the deductive nature of falsification in property-based testing suits very well the constructive and exploratory nature of ABS. Further, the automatic test generation can make testing of large scenarios in ABS feasible because it does not require the programmer to specify all test cases by hand, as is required in traditional unit tests.

Property-based testing was introduced in \cite{claessen_quickcheck_2000,claessen_testing_2002} where the authors present the QuickCheck library in Haskell, which tries to falsify the specifications by \textit{randomly} sampling the test space. %We argue, that the stochastic sampling nature of this approach is particularly well suited to ABS, because it is itself almost always driven by stochastic events and randomness in the agents behaviour, thus this correlation should make it straightforward to map ABS to property-testing.
%The main challenge when using QuickCheck, as will be shown later, is to write \textit{custom} test data generators for agents and the environment which cover the space sufficiently enough to not miss out on important test cases.
According to the authors of QuickCheck \textit{"The major limitation is that there is no measurement of test coverage."} \cite{claessen_quickcheck_2000}. Although QuickCheck provides help to report the distribution of test cases it is not able to measure the coverage of tests in general. This could lead to the case that test cases which would fail are never tested because of the stochastic nature of QuickCheck. Fortunately, the library provides mechanisms for the developer to measure coverage in specific test cases where the data and its expected distribution is known to the developer. This is a powerful tool for testing randomness in ABS as will be shown in the next chapters.

\medskip

As a remedy for the potential coverage problems of QuickCheck, there exists also a deterministic property-testing library called SmallCheck \cite{runciman_smallcheck_2008}, which instead of randomly sampling the test space, enumerates test cases exhaustively up to some depth. It is based on two observations, derived from model-checking, that (1) \textit{"If a program fails to meet its specification in some cases, it almost always fails in some simple case"} and (2) \textit{"If a program does not fail in any simple case, it hardly ever fails in any case} \cite{runciman_smallcheck_2008}. This non-stochastic approach to property-based testing might be a complementary addition in some cases where the tests are of non-stochastic nature with a search space too large to test manually by unit testing but small enough to enumerate exhaustively. The main difficulty and weakness of using SmallCheck is to reduce the dimensionality of the test case depth search to prevent combinatorial explosion, which would lead to exponential number of cases. Thus one can see QuickCheck and SmallCheck as complementary instead of in opposition to each other.

\subsection{A brief overview of QuickCheck}
To give a good understanding of how property-based testing works with \\ QuickCheck, we give a few examples of property tests on lists, which are directly expressed as functions in Haskell. Such a function has to return a \texttt{Bool} which indicates \texttt{True} in case the test succeeds or \texttt{False} if not and can take input arguments which data is automatically generated by QuickCheck.

\begin{HaskellCode}
-- append operator (++) is associative
append_associative :: [Int] -> [Int] -> [Int] -> Bool
append_associative xs ys zs = (xs ++ ys) ++ zs == xs ++ (ys ++ zs)

-- The reverse of a reversed list is the original list
reverse_reverse :: [Int] -> Bool
reverse_reverse xs = reverse (reverse xs) == xs

-- reverse is distributive over append (++)
-- This test fails for explanatory reasons, for a correct 
-- property xs and ys need to be swapped on the right-hand side!
reverse_distributive :: [Int] -> [Int] -> Bool
reverse_distributive xs ys = reverse (xs ++ ys) == reverse xs ++ reverse ys

-- running the tests
main :: IO ()
main = do
  quickCheck append_associative
  quickCheck reverse_reverse
  quickCheck reverse_distributive
\end{HaskellCode}

When we run the tests using \textit{main}, we get the following output:

\begin{verbatim}
+++ OK, passed 100 tests.
+++ OK, passed 100 tests.
*** Failed! Falsifiable (after 5 tests and 6 shrinks):    
[0]
[1]
\end{verbatim}

We see that QuickCheck generates 100 test cases for each property test and it does this by generating random data for the input arguments. We have not specified any data for our input arguments because QuickCheck is able to provide a suitable data generator through type inference. For lists and all the existing Haskell types there exist custom data generators already. We have to use a monomorphic list, in our case \texttt{Int}, and cannot use polymorphic lists because QuickCheck would not know how to generate data for a polymorphic type. Still, by appealing to genericity and polymorphism, we get the guarantee that the test case is the same for all types of a lists.

QuickCheck generates 100 test cases by default and requires all of them to pass. If there is a test case which fails, the overall property test fails and QuickCheck shrinks the input to a minimal size, which still fails and reports it as a counter example. This is the case in the last property test \texttt{reverse\_distributive} which is wrong as \textit{xs} and \textit{ys} need to be swapped on the right-hand side. In this run, QuickCheck found a counter example to the property after 5 tests and applied 6 shrinks to find the minimal failing example of \texttt{xs = [0]} and \texttt{ys = [1]}. If we swap \texttt{xs} and \texttt{ys}, the property test passes 100 test cases just like the other two did. It is possible to configure QuickCheck to generate more or less random test cases, which can be used to increase the coverage if the sampling space is quite large - this will become useful later.

\subsubsection{Generators}
QuickCheck comes with a lot of data generators for existing types like \texttt{String, Int, Double, []}, but in case one wants to randomize custom data types one has to write custom data generators. There are two ways to do this. Either fix them at compile time by writing an \texttt{Arbitrary} instance or write a run-time generator running in the \texttt{Gen} Monad. The advantage of having an \texttt{Arbitrary} instance is that the custom data type can then be used as random argument to a function as in the examples above.

Lets implement a custom data generator for the \texttt{SIRState} for both cases. We start with the run-time option, running in the \texttt{Gen} Monad:

\begin{HaskellCode}
genSIRState :: Gen SIRState
genSIRState = elements [Susceptible, Infected, Recovered]
\end{HaskellCode}

This implementation makes use of the \texttt{elements :: [a] $\rightarrow$ Gen a} functions, which picks a random element from a non-empty list with uniform probability. If a skewed distribution is needed, one can use the \texttt{frequency :: [(Int, Gen a)] $\rightarrow$ Gen a} function, where a frequency can be specified for each element. For example generating on average 80\% \texttt{Susceptible}, 15\% \texttt{Infected} and 5\% \texttt{Recovered} can be achieved using this function:

\begin{HaskellCode}
genSIRState :: Gen SIRState
genSIRState = frequency [(80, Susceptible), (15, Infected), (5, Recovered)]
\end{HaskellCode}

Implementing an \texttt{Arbitrary} instance is straightforward, one only needs to implement the \texttt{arbitrary :: Gen a} method:

\begin{HaskellCode}
instance Arbitrary SIRState where
  arbitrary = genSIRState
\end{HaskellCode}

When we have a random \texttt{Double} as input to a function but want to restrict its random range to (0,1) because it reflects a probability, we can do this easily with \texttt{newtype} and implementing an \texttt{Arbitrary} instance. The same can be done for limiting the simulation duration to a lower range than the full \texttt{Double} range.

\begin{HaskellCode}
newtype Probability = P Double
newtype TimeRange   = T Double

instance Arbitrary Probability where
  arbitrary = P <$> choose (0, 1)
  
instance Arbitrary TimeRange where
  arbitrary = T <$> choose (0, 50)
\end{HaskellCode}

The simulations we run all rely on a random-number generator, thus we need a randomly initialised random-number generator each time we run a simulation. This can be easily achieved by drawing a seed from the full \texttt{Int} range and creating an \texttt{StdGen} from it:

\begin{HaskellCode}
genStdGen :: Gen StdGen
-- min/maxBound are defined in the Haskell Prelude and
-- define the smallest and largest value of a Bounded type 
genStdGen = mkStdGen <$> choose (minBound, maxBound)

instance Arbitrary StdGen where
  arbitrary = genStdGen
\end{HaskellCode}
%$

This generator then can be used to write another custom data generator which generates simulation runs. Here we give an example for the time-driven SIR:

\begin{HaskellCode}
genTimeSIR :: [SIRState]  -- ^ Population
           -> Double      -- ^ Contact rate (beta)
           -> Double      -- ^ Infectivity (gamma)
           -> Double      -- ^ Illness duration (delta)
           -> Double      -- ^ Time Delta
           -> Double      -- ^ Time Limit
           -> Gen [(Double, (Int, Int, Int))]
genTimeSIR as beta gamma delta dt tMax 
  = runTimeSIR as beta gamma delta dt tMax <$> genStdGen
\end{HaskellCode}
%$

\subsubsection{Distributions}
As already mentioned, QuickCheck provides functions to measure the coverage of test cases. This can be done using the 
\texttt{label :: Testable prop $\Rightarrow$ String $\rightarrow$ prop $\rightarrow$ Property} function. It takes a \texttt{String} as first argument and a testable property and constructs a \texttt{Property}. QuickCheck collects all generated labels, counts their occurrences and reports their distribution. For example it could be used to get a rough idea about the length of the random lists created in the \texttt{reverse\_reverse} property shown above:

\begin{HaskellCode}
reverse_reverse_label :: [Int] -> Property
reverse_reverse_label xs  
  = label ("length of random-list is " ++ show (length xs)) 
          (reverse (reverse xs) == xs)
\end{HaskellCode}
%$
When running the test, we see the following output:

\begin{verbatim}
+++ OK, passed 100 tests:
 5% length of random-list is 27
 5% length of random-list is 0
 4% length of random-list is 19
 ...
\end{verbatim}

\subsubsection{Coverage}
The most powerful functions to work with test-case distributions though are \texttt{cover} and \texttt{checkCoverage}. The function \texttt{cover :: Testable prop $\Rightarrow$ Double $\rightarrow$ Bool $\rightarrow$ String $\rightarrow$ prop $\rightarrow$ Property} allows to explicitly specify that a given percentage of successful test cases belong to a given class. The first argument is the expected percentage; the second argument is a \texttt{Bool} indicating whether the current test case belongs to the class or not; the third argument is a label for the coverage; the fourth argument is the property which needs to hold for the test case to succeed. 

Lets look at an example where we use \texttt{cover} to express that we expect 15\% of all test cases to have a random list with at least 50 elements.

\begin{HaskellCode}
reverse_reverse_cover :: [Int] -> Property
reverse_reverse_cover xs  
  = cover 15 (length xs >= 50) "Length of random list at least 50"
             (reverse (reverse xs) == xs)
\end{HaskellCode}

When repeatedly running the test, we see the following output:

\begin{verbatim}
+++ OK, passed 100 tests (10% length of random list at least 50).
Only 10% Length of random-list at least 50, but expected 15%.
+++ OK, passed 100 tests (21% length of random list at least 50).
\end{verbatim}

As can be seen, QuickCheck runs the default 100 test cases and prints a warning if the expected coverage is not reached. This is a useful feature but it is up to us to decide whether 100 test cases are suitable and whether we can really claim that the given coverage will be reached or not. Fortunately, QuickCheck provides the powerful function \texttt{checkCoverage :: Testable prop $\Rightarrow$ prop $\rightarrow$ Property} which does this for us. When \texttt{checkCoverage} is used, QuickCheck will run an increasing number of test cases until it can decide whether the percentage in \texttt{cover} was reached or cannot be reached at all. The way QuickCheck does it, is by using sequential statistical hypothesis testing \cite{wald_sequential_1992}, thus if QuickCheck comes to the conclusion that the given percentage can or cannot be reached, it is based on a robust statistical test giving strong confidence in the result.

When we run the example from above but now with \texttt{checkCoverage} we get the following output:

\begin{verbatim}
+++ OK, passed 12800 tests 
    (15.445% length of random-list at least 50).
\end{verbatim}

We see that after QuickCheck has run 12,800 tests it came to the statistically robust conclusion that indeed at least 15\% of the test cases have a random list with at least 50 elements. 

\subsubsection{Emulating failure}
As already mentioned, \textit{all} test cases have to pass for the whole property test to succeed. If just a single test case fails, the whole property test fails. This requirement is sometimes too strong, especially when we are dealing with stochastic systems like ABS.

The function \texttt{cover} can be used to emulate failure of test cases and get a measure of failure. Instead of computing the \texttt{True/False} property in the last \texttt{prop} argument, we set the last argument always to \texttt{True} and compute the \texttt{True/False} property in the second \texttt{Bool} argument, indicating whether the test case belongs to the class of passed tests or not. This has the effect that \textit{all} test cases are successful but that we get a distribution of failed and successful ones. In combination with \texttt{checkCoverage}, this is a particularly powerful pattern for testing ABS, which allows us to test hypotheses and statistical tests on distributions as will be shown in the following chapters.

\input{./tex/property/abstesting.tex}

\chapter{Testing Agent specifications}
\label{ch:agentspec}

In this chapter we are showing how to use QuickCheck to encode full agent-specifications directly in code as property-tests. These properties serve then both as formal specification and tests at the same time - a fundamental strength of property-based testing, not possible with unit-testing in this strong expressive form. Besides the high expressivity, QuickCheck also allows us to state statistical coverage for certain cases, which allows to express statistical properties of the agents behaviour, something also not directly possible with unit-testing. This is a very strong indication that property-based testing is a natural fit to test agent-based simulation. We discuss both time- and event-driven  implementations of the agent-based SIR model as introduced in Chapter \ref{sec:sir_model}.

\section{Event-driven specification}
In this section we present how QuickCheck can be used to test event-driven agents by expressing their \textit{specification} as property-tests in the case of the event-driven SIR implementation from chapter \ref{sec:eventdriven_basics}.

In general, testing event-driven agents is fundamentally different and more complex than testing time-driven agents, as their interface surface is generally much larger: events form the input to the agents to which they react with new events - the dependencies between those can be quite complex and deep. Using property-based tests we can encode the invariants and end up with an actual specification of their behaviour, acting as documentation, regression test within a TDD and property tests.

Note that the concepts presented here are applicable with slight adjustments to the Sugarscape implementation as well but we focused on the SIR one as its specification is shorter and does not require as much in-depth details - after all we are interested in deriving concepts, not dealing with specific technicalities.

With event-driven ABS a good starting point in specifying and then testing the system is simply relating the input events to expected output events. In the SIR implementation we have only three events, making it feasible to give a full formal specification - note that the Sugarscape implementation has more than 16 events, which makes it much harder to test it with sufficient coverage, giving a good reason to primarily focus on the SIR implementation. 

\subsection{Deriving the specification}
We start by giving the full \textit{specification} of the susceptible, infected and recovered agent by stating the input-to-output event relations. The susceptible agent is specified as follows:

\begin{enumerate}
	\item \textit{MakeContact} - If the agent receives this event it will output $\beta$  \textit{Contact ai Susceptible} events, where ai is the agents self id. The events have to be scheduled immediately without delay, thus having the current time as scheduling timestamp. The receivers of the events are uniformly randomly chosen from the agent population. The agent doesn't change its state, stays \textit{Susceptible} and does not schedule any other events than the ones mentioned.
	
	\item \textit{Contact \_ Infected} - If the agent receives this event there is a chance of uniform probability $\gamma$ (infectivity) that the agent becomes \textit{Infected}. If this happens, the agent will schedule a \textit{Recover} event to itself into the future, where the time is drawn randomly from the exponential distribution with $\lambda = \delta$ (illness duration). If the agent does not become infected, it will not change its state, stays \textit{Susceptible} and does not schedule any events.
	
	\item \textit{Contact \_ \_} or \textit{Recover} - If the agent receives any of these (other) events it will not change its state, stays \textit{Susceptible} and does not schedule any events.
\end{enumerate}

This specification implicitly covers that a susceptible agent can never transition from a \textit{Susceptible} to a \textit{Recovered} state within a single event - it can only make the transition to \textit{Infected} or stays \textit{Susceptible}. The infected agents are specified as follows:

\begin{enumerate}
	\item \textit{Recover} - If the agent receives this, it will not schedule any events and make the transition to the \textit{Recovered} state.
	
	\item \textit{Contact sender Susceptible} - If the agent receives this, it will reply immediately with \textit{Contact ai Infected} to \textit{sender}, where \textit{ai} is the infected agents' id and the scheduling timestamp is the current time. It will not schedule any events and stays \textit{Infected}.
	
	\item In case of any other event, the agent will not schedule any events and stays \textit{Infected}.
\end{enumerate}

This specification implicitly covers that an infected agent never goes back to the \textit{Susceptible} state - it can only make the transition to \textit{Recovered} or stay \textit{Infected}. From the specification of the susceptible agent it becomes clear that a susceptible agent who became infected, will always recover as the transition to \textit{Infected} includes the scheduling of \textit{Recovered} to itself. 

\medskip

The \textit{recovered} agents specification is very simple. It stays \textit{Recovered} forever and does not schedule any events.

\medskip

The question is now how to put these into a property-test with QuickCheck. We focus on the susceptible agent, as it it the most complex one, which concepts can then be easily applied to the other two. Generally speaking, we create a random \textit{susceptible} agent and a random event, feed it to the agent to get the output and check the invariants accordingly to input and output. % In the specification there are stated three probabilities regarding $\beta$ (contact rate), $\gamma$ (infectivity) and $\delta$ (illness duration). We will only check one, $\gamma$ (infectivity) using the coverage features of QuickCheck and write additional property-tests for the other two. The reason for that is, that checking $\gamma$ is natural with the invariant checking whereas the others need a slightly different approach and are more obviously stated in separate property-tests.

\subsection{Encoding invariants}
We start by encoding the invariants of the susceptible agent directly into Haskell, implementing a function which takes all necessary parameters and returns a \textit{Bool} indicating whether the invariants hold or not. The encoding is straightforward when using pattern matching and it nearly reads like a formal specification due to the declarative nature of functional programming.

\begin{HaskellCode}
susceptibleProps :: SIREvent              -- ^ Random event sent to agent
                 -> SIRState              -- ^ Output state of the agent
                 -> [QueueItem SIREvent]  -- ^ Events the agent scheduled
                 -> AgentId               -- ^ Agent id of the agent
                 -> Bool
-- received Recover => stay Susceptible, no event scheduled
susceptibleProps Recover Susceptible es _ = null es
-- received MakeContact => stay Susceptible, check events
susceptibleProps MakeContact Susceptible es ai
  = checkMakeContactInvariants ai es cor 
-- received Contact _ Recovered => stay Susceptible, no event scheduled
susceptibleProps (Contact _ Recovered) Susceptible es _ = null es
-- received Contact _ Susceptible => stay Susceptible, no event scheduled
susceptibleProps (Contact _ Susceptible) Susceptible es _  = null es
-- received Contact _ Infected, didn't get Infected, no event scheduled
susceptibleProps (Contact _ Infected) Susceptible es _ = null es
-- received Contact _ Infected AND got infected, check events
susceptibleProps (Contact _ Infected) Infected es ai
  = checkInfectedInvariants ai es
-- all other cases are invalid and result in a failed test case
susceptibleProps _ _ _ _ = False
\end{HaskellCode}

Next, we give the implementation for the \textit{checkMakeContactInvariants} and \textit{checkInfectedInvariants} functions. The function \textit{checkMakeContactInvariants} encodes the invariants which have to hold when the susceptible agent receives a \textit{MakeContact} event. The \textit{checkInfectedInvariants} function encodes the invariants which have to hold when the susceptible agent got \textit{Infected}. Both implementations read like a formal specification, again thanks to the declarative nature of functional programming and pattern matching:

\begin{HaskellCode}
checkInfectedInvariants :: AgentId              -- ^ Agent id of the agent 
                        -> [QueueItem SIREvent] -- ^ Events the agent scheduled
                        -> Bool
checkInfectedInvariants sender 
  -- expect exactly one Recovery event
  [QueueItem receiver (Event Recover) t'] 
  -- receiver is sender (self) and scheduled into the future
  = sender == receiver && t' >= t 
-- all other cases are invalid
checkInfectedInvariants _ _ = False
\end{HaskellCode}

The \textit{checkMakeContactInvariants} is a bit more complex but reads as a formal specification as well:

\begin{HaskellCode}
checkMakeContactInvariants :: AgentId              -- ^ Agent id of the agent 
                           -> [QueueItem SIREvent] -- ^ Events the agent scheduled
                           -> Int                  -- ^ Contact Rate
                           -> Bool
checkMakeContactInvariants sender es contactRate
    -- make sure there has to be exactly one MakeContact event and
    -- exactly contactRate Contact events
    = invOK && hasMakeCont && numCont == contactRate
  where
    (invOK, hasMakeCont, numCont) 
      = foldr checkMakeContactInvariantsAux (True, False, 0) es

    checkMakeContactInvariantsAux :: QueueItem SIREvent 
                                  -> (Bool, Bool, Int)
                                  -> (Bool, Bool, Int)
    checkMakeContactInvariantsAux 
        (QueueItem (Contact sender' Susceptible) receiver t') (b, mkb, n)
      = (b && sender == sender'    -- the sender in Contact must be the Susceptible agent
           && receiver `elem` ais  -- the receiver of Contact must be in the agent ids
           && t == t', mkb, n+1)   -- the Contact event is scheduled immediately
    checkMakeContactInvariantsAux 
        (QueueItem MakeContact receiver t') (b, mkb, n) 
      = (b && receiver == sender   -- the receiver of MakeContact is the Susceptible agent itself
           && t' == t + 1          -- the MakeContact event is scheduled 1 time-unit into the future
           &&  not mkb, True, n)   -- there can only be one MakeContact event
    checkMakeContactInvariantsAux _ (_, _, _) 
      = (False, False, 0)          -- other patterns are invalid
\end{HaskellCode}

What is left is to actually write a property-test using QuickCheck. We are making heavy use of random parameters to express that the properties have to hold invariant of the model parameters. We make use of additional data-generator modifiers: \textit{Positive} ensures that the value generated is positive; \textit{NonEmptyList} ensures that the randomly generated list is non-empty.

\begin{HaskellCode}
prop_susceptible_invariants :: Positive Int         -- ^ Contact rate
                            -> Probability          -- ^ Infectivity
                            -> Positive Double      -- ^ Illness duration
                            -> Positive Double      -- ^ Current simulation time
                            -> NonEmptyList AgentId -- ^ Agent ids of the population
                            -> Gen Property
prop_susceptible_invariants 
  (Positive cor) (P inf) (Positive ild) (Positive t) (NonEmpty ais) = do
  -- generate random event, requires the population agent ids
  evt <- genEvent ais
  -- run susceptible random agent with given parameters
  (ai, ao, es) <- genRunSusceptibleAgent cor inf ild t ais evt
  -- check properties
  return $ property $ susceptibleProps evt ao es ai
\end{HaskellCode}

When running this property-test all 100 test cases pass. Due to the large random sampling space with 5 parameters, we increase the number of test cases to generate to 100,000 - still all test cases pass.

\subsection{Encoding transition probabilities}
In the specifications above there are probabilistic state-transitions, for example an infected agent \textit{will} recover after a given time, which is randomly distributed with the exponential distribution. The susceptible agent \textit{might} become infected, depending on the events it receives and the infectivity ($\gamma$) parameter. We look now into how we can encode these probabilistic properties using the powerful \textit{cover} and \textit{checkCoverage} feature of QuickCheck.

\subsubsection{Susceptible agent}
We follow the same approach as in encoding the invariants of the susceptible agent but instead of checking the invariants, we compute the probability for each case. Note that in this property-test we cannot randomise the model parameters because this would lead to random coverage. This might seem like a disadvantage but we do not really have a choice here - still, the model parameters can be adjusted arbitrarily and the property (must) still hold. %Note that we do not provide the details of computing the probabilities of each input-to-output case as it is quite technical and of not much importance - it is only a matter of multiplication and divisions amongst the event-frequencies and model parameters.
We make use of the \textit{cover} function together with \textit{checkCoverage}, which ensures that we get a statistical robust estimate whether the expected percentages can be reached or not. Implementing this property-test is then simply a matter of computing the probabilities and of case analysis over the random input event and the agents output.

\begin{HaskellCode}
...
case evt of 
  Recover -> 
    cover recoverPerc True 
     ("Susceptible receives Recover, expected " ++ show recoverPerc) True
...
\end{HaskellCode}

Note the usage pattern of \textit{cover}: we always include the test case into the coverage class and all test cases pass. The reason for this is that we are just interested in testing the coverage, which is in fact the property we want to test. We could have combined this test into the previous one but then we couldn't have use randomised model parameters. For this reason, and to keep the concerns separated we opted for two different tests, which makes them also much more readable.

%\begin{HaskellCode}
%prop_susceptible_proabilities :: Positive Double      -- ^ Current simulation time
%                              -> NonEmptyList AgentId -- ^ Agent ids of the population
%                              -> Property
%prop_susceptible_proabilities (Positive t) (NonEmpty ais) = checkCoverage (do
%  -- fixed model parameters, otherwise random coverage
%  let cor = 5
%      inf = 0.05
%      ild = 15.0
%
%   -- compute distributions for all cases
%  let recoverPerc       = ...
%      makeContPerc      = ...
%      contactRecPerc    = ...
%      contactSusPerc    = ...
%      contactInfSusPerc = ...
%      contactInfInfPerc = ...
%
%  -- generate a random event
%  evt <- genEvent ais
%  -- run susceptible random agent with given parameters
%  (_, ao, _) <- genRunSusceptibleAgent cor inf ild t ais evt
%
%  -- encode expected distributions
%  return $ property $
%    case evt of 
%      Recover -> 
%        cover recoverPerc True 
%          ("Susceptible receives Recover, expected " ++ 
%           show recoverPerc) True
%      MakeContact -> 
%        cover makeContPerc True 
%          ("Susceptible receives MakeContact, expected " ++ 
%           show makeContPerc) True
%      (Contact _ Recovered) -> 
%        cover contactRecPerc True 
%          ("Susceptible receives Contact * Recovered, expected " ++ 
%           show contactRecPerc) True
%      (Contact _ Susceptible) -> 
%        cover contactSusPerc True 
%          ("Susceptible receives Contact * Susceptible, expected " ++ 
%           show contactSusPerc) True
%      (Contact _ Infected) -> 
%        case ao of
%          Susceptible ->
%            cover contactInfSusPerc True 
%              ("Susceptible receives Contact * Infected, stays Susceptible " ++
%               ", expected " ++ show contactInfSusPerc) True
%          Infected ->
%            cover contactInfInfPerc True 
%              ("Susceptible receives Contact * Infected, becomes Infected, " ++
%               ", expected " ++ show contactInfInfPerc) True
%          _ ->
%            cover 0 True "Impossible Case, expected 0" True
%\end{HaskellCode}

When running the property-test we get the following output:

\begin{footnotesize}
\begin{verbatim}
+++ OK, passed 819200 tests:
33.3582% Susceptible receives MakeContact, expected 33.33%
33.2578% Susceptible receives Recover, expected 33.33%
11.1643% Susceptible receives Contact * Recovered, expected 11.11%
11.1096% Susceptible receives Contact * Susceptible, expected 11.11%
10.5616% Susceptible receives Contact * Infected, stays Susceptible, expected 10.56%
 0.5485% Susceptible receives Contact * Infected, becomes Infected, expected 0.56%
\end{verbatim}
\end{footnotesize}

After 819,200 (!) test cases QuickCheck comes to the conclusion that the distributions generated by the test cases reflect the expected distributions and passes the property-test. We see that the values do not match exactly in some cases but by using sequential statistical hypothesis testing QuickCheck is able to conclude that the coverage are statistically equal.

\subsubsection{Infected agent}
We want to write a property-test which checks whether the transition from \textit{Infected} to \textit{Recovered} actually follows the exponential distribution with a fixed $\delta$ (illness duration). The idea is to compute the expected probability for agents having an illness duration of less or equal $\delta$. This probability is given by the cumulative density function (CDF) of the exponential distribution. The question is how to get the infected illness duration. This is simply achieved by infecting a susceptible agent and taking the scheduling time of the \textit{Recover} event. We have written a custom data-generator for this:

\begin{HaskellCode}
getInfectedAgentDuration :: Double -> Gen (SIRState, Double)
getInfectedAgentDuration ild = do
  -- with these parameters the susceptible agent WILL become infected
  (_, ao, es) <- genRunSusceptibleAgent 1 1 ild 0 [0] (Contact 0 Infected)
  return (ao, recoveryTime es)
  where
    -- expect exactly one event: Recover
    recoveryTime :: [QueueItem SIREvent] -> Double
    recoveryTime [QueueItem Recover _ t]  = t
    recoveryTime _ = 0
\end{HaskellCode}

Encoding the probability check into a property-test is straightforward:

\begin{HaskellCode}
prop_infected_duration :: Property
prop_infected_duration = checkCoverage (do
  -- fixed model parameter, otherwise random coverage
  let ild  = 15
  -- compute probability drawing a random value less or equal
  -- ild from the exponential distribution (follows the CDF)
  let prob = 100 * expCDF (1 / ild) ild

  -- run random susceptible agent to become infected and
  -- return agents state and recovery time
  (ao, dur) <- getInfectedAgentDuration ild

  return (cover prob (dur <= ild) 
            ("Infected agent recovery time is less or equals " ++ show ild ++ 
             ", expected at least " ++ show prob) 
            (ao == Infected)) -- final state has to Infected
\end{HaskellCode}

When running the property-test we get the following output.

\begin{footnotesize}
\begin{verbatim}
+++ OK, passed 3200 tests 
    (63.62% Infected agent recovery time is less or equals 15.0, 
     expected at least 63.21%).
\end{verbatim}
\end{footnotesize}

QuickCheck is able to determine after only 3,200 test cases that the expected coverage is met and passes the property-test.

\section{Time-driven specification}
\label{sec:timedriven_specification}
The time-driven SIR agents have a very small interface: they only receive the agent population from the previous step and output their state in the current step. We can also assume an implicit forward flow of time, statically guaranteed by Yampas arrowized FRP. Thus a specification in time-driven approach is given in terms of probabilities and timeouts, rather than in events as in the event-driven testing as presented before.

\begin{itemize}
	\item Susceptible agent - makes \textit{on average} contact with $\beta$ (contact rate) agents per time unit. The distribution follows the exponential distribution with $\lambda = \frac{1}{\beta}$. If a susceptible agent gets into contact with an infected agent, it will become infected with a uniform probability of $\gamma$ (infectivity).
	
	\item Infected agent - \textit{will} recover \textit{on average} after $\delta$ (illness duration) time units. The distribution follows the exponential distribution with $\lambda = \delta$.

	\item Recovered agent - stays recovered \textit{forever}.
\end{itemize}

\subsection{Specifications of the susceptible agent}
We cannot directly observe that a susceptible agent makes contact with other agents like we can in the event-driven approach but only indirectly through its change of state: a susceptible agent \textit{might} become infected if there are infected agents in the population.
Thus when we run a susceptible agent for some time, we have 3 possible outcomes of the agents output stream: 1. the agent did not get infected and thus all elements of the stream are \textit{Susceptible}; 2. the agent got infected thus up to a given index in the stream all elements are \textit{Susceptible} and change to \textit{Infected} after; 3. the agent got \textit{Infected} and then \textit{Recovered} thus the stream is the same as in infected but there is a second index after which all elements change to \textit{Recovered}. Encoding them in code is straightforward:

\begin{HaskellCode}
susceptibleInvariants :: [SIRState] -- ^ The output stream of the susceptible agent 
                      -> Bool       -- ^ The population contains an infected agent
                      -> Bool       -- ^ True in case the invariant holds
susceptibleInvariants aos infInPop
    -- Susceptible became Infected and then Recovered
    | isJust recIdxMay 
      = infIdx < recIdx &&  -- agent has to become infected before recovering
        all (==Susceptible) (take infIdx aos) && 
        all (==Infected) (take (recIdx - infIdx) (drop infIdx aos)) && 
        all (==Recovered) (drop recIdx aos) &&
        infInPop  -- can only happen if there are infected in the population

    -- Susceptible became Infected
    | isJust infIdxMay 
      = all (==Susceptible) (take infIdx aos) &&
        all (==Infected) (drop infIdx aos) &&
        infInPop -- can only happen if there are infected in the population

    -- Susceptible stayed Susceptible
    | otherwise = all (==Susceptible) aos
  where
    -- look for the first element when agent became Infected
    infIdxMay = elemIndex Infected aos
    -- look for the first element when agent became Recovered
    recIdxMay = elemIndex Recovered aos

    infIdx = fromJust infIdxMay
    recIdx = fromJust recIdxMay
\end{HaskellCode}

Putting this into a property-test is also straightforward. We generate a random population, run a random susceptible agent with a sampling rate of $\Delta t = 0.01$ and check the invariants on its output stream. These invariants all have to hold independently from the (positive) duration we run the random susceptible agent for, thus we run it for a random amount of time units. The invariants also have to hold for arbitrary positive beta (contact rate), gamma (infectivity) and delta (illness duration). At the same time, we want to get an idea of the percentage of agents which stayed susceptible, became infected or made the transition to recovered, thus we \textit{label} all our test cases accordingly.

\begin{HaskellCode}
prop_susceptible_invariants :: Positive Double  -- ^ beta, contact rate
                            -> Probability      -- ^ gamma, infectivity within (0,1) range
                            -> Positive Double  -- ^ delta, illness duration
                            -> TimeRange        -- ^ simulation duration, within (0,50) range
                            -> [SIRState]       -- ^ random population
                            -> Property
prop_susceptible_invariants
      (Positive cor) (P inf) (Positive ild) (T t) as = property (do  
    -- population contains an infected agent True/False
    let infInPop = Infected `elem` as

    -- run a random susceptible agent for random time units with 
    -- sampling rate dt 0.01 and return its stream of output
    aos <- genSusceptible cor inf ild as t 0.01

    return 
        -- label all test cases
        label (labelTestCase aos) 
        -- check invariants on output stream
        (property (susceptibleInvariants aos infInPop))
  where
    labelTestCase :: [SIRState] -> String
    labelTestCase aos
      | Recovered `elem` aos = "Susceptible -> Infected -> Recovered"
      | Infected `elem` aos  = "Susceptible -> Infected"
      | otherwise            = "Susceptible"
\end{HaskellCode}

Due to the high dimensionality of the random sampling space, we run 10,000 tests - all succeed as expected.

\begin{verbatim}
SIR Agent Specifications Tests
  Susceptible agents invariants: OK (12.72s)
    +++ OK, passed 10000 tests:
    55.78% Susceptible -> Infected -> Recovered
    37.19% Susceptible -> Infected
     7.03% Susceptible
\end{verbatim}

This test so far did not state anything about the probability of a susceptible agent getting infected. The probability for it is bimodal (see next Chapter) due to the combined probabilities of the exponential distribution of the contact rate and the uniform distribution of the infectivity. Unfortunately, the bimodality makes it not possible to compute a coverage percentage of infected in this case, as we did in the event-driven test because the bimodal distribution can only be described in terms of a distribution and not a single probability. This was possible in the even-driven approach because we decoupled the production of the \textit{Contact \_ Infected} event from the infection: both were uniform distributed, thus we could compute a coverage percentage. Thus we see that different approaches also allow different explicitness of testing.

\subsection{Probabilities of the infected agent}
An infected agent \textit{will} recover after \textit{finite} time, thus we assume that there exists an index in the output stream, where the elements will change to \textit{Recovered}. From the index we can compute the time of recovery, knowing the fixed sampling rate $\Delta t$.

\begin{HaskellCode}
infectedInvariant :: [SIRState]     -- ^ The stream of outputs from the infected agent
                  -> Double         -- ^ Sampling rate dt
                  -> Maybe Double   -- ^ Just recovery time, or Nothing if not recovered
infectedInvariant aos dt  = do
  -- search for the index of the first Recovery element
  recIdx <- elemIndex Recovered aos
  -- all elements up to the index need to be Infected,
  -- because the agent cannot go back to Susceptible
  if all (==Infected) (take recIdx aos)
    then Just (dt * recIdx)
    else Nothing
\end{HaskellCode}

To put this into a property-test, we follow a similar approach as in the event-driven case of the infected agents invariants. We employ the CDF of the exponential distribution to get the probability of an agent recovering within $\delta$ (illness duration) time steps. We then run a random infected agent for an \textit{unlimited} time with a sampling rate o f $\Delta t = 0.01$ and search in its potentially infinite output stream for the first occurrence of an \textit{Infected} element to compute the recovery time, as shown in the invariant above. The code is conceptually exactly the same as in the event-driven case, so we do not repeat the property-test here.

%\begin{HaskellCode}
%prop_infected_invariants :: [SIRState] -> Property
%prop_infected_invariants as = checkCoverage (do
%  -- delta, illnes duration
%  let illnessDuration = 15.0
%  -- compute perc of agents which recover in less or equal 
%  -- illnessDuration time units. Follows the exponential distribution
%  -- thus we use the CDF to compute the probability.
%  let prob = 100 * expCDF (1 / illnessDuration) illnessDuration
%  -- fixed sampling rate
%  let dt = 0.01
%
%  -- run a random infected agent without time-limit (0) and sampling rate
%  -- of 0.01 and return its infinite output stream 
%  aos <- genInfected illnessDuration as 0 dt
%
%  -- compute the recovery time
%  let dur = infectedInvariant aos dt
%
%  return (cover prob (fromJust dur <= illnessDuration)
%            ("infected agents have an illness duration of " ++ show illnessDuration ++
%             " or less, expected " ++ show prob) (isJust dur)
%\end{HaskellCode}

When running the test we get the following output, indicating that QuickCheck finds the coverage satisfied after 3,200 test cases:

\begin{verbatim}
+++ OK, passed 3200 tests (62.28% infected agents have an illness 
    duration of 15.0 or less, expected 63.21).
\end{verbatim}

The fact that we run the random infected agent without time-limit explicitly expresses the invariant that an infected agent \textit{will} recover in \textit{finite} time steps: a correct implementation will produce a stream which contains an index after which all elements are \textit{Infected}, thus resulting in \textit{Just} recovery time. This is also a direct expression of the fact that the CDF of the exponential distribution reaches 1.0 at infinity. An approach which would guarantee the termination would be to limit the time to run the infected agent to \textit{illnessDuration} and evaluate the property always to True. This approach guarantees termination but removes an important part of the specification - we decided to follow the initial approach to make the specification really clear, and in practice it has turned out to terminate within a very short time (see below).

\subsection{The non-computability of the recovered agent test}
The property-test for the recovered agent is trivial: we run a random recovered agent for a random number of time units with $\Delta t = 0.01$ and require that all elements in the output stream are \textit{Recovered}. Of course this is no proof that the recovered agent stays recovered \textit{forever} as this would take \textit{forever} to test and is thus not computable.  Here we are hitting the limits of what is possible with random black-box testing: without looking at the actual implementation it is not possible to prove that the recovered agent is really behaving as specified. We made this fact very clear at the beginning of this part: property-based testing is not a proof for the correctness but is only a support for raising the confidence in the correctness by constructing cases which show that the behaviour is not incorrect.

To be really sure that the recovered agent behaves as specified we need to employ white-box verification and look at the actual implementation. It is immediately obvious that the implementation follows the specification and actually \textit{is} the specification, and we can even regard it as a very concise proof that it will stay recovered \textit{forever}:

\begin{HaskellCode}
recoveredAgent :: SIRAgent
recoveredAgent = constant Recovered
\end{HaskellCode}

The signal function \textit{constant} is the \textit{const} function lifted into an arrow: \textit{constant b = arr (const b)}. This should be proof enough that a recovered agent will stay recovered \textit{forever}. We discuss the topic of computability in pure functional ABS in a slightly different context in Appendix \ref{app:equilibrium_totality}.

\section{Discussion}
In this section we have shown how to express the specifications of both the event- and time-driven agent behaviour directly in code as properties and how to implement property-tests in QuickCheck for them. The approach to event-driven properties was to establish a correspondence between an input-event the current agent-state and the output events and the new agents state. In case of the time-driven agent, the properties are expressed in terms of a potentially infinite stream of agent output-states. Although both implementations follow the same underlying model, the technical details of the properties differ substantially. The reason for this is that although property-based testing is a black-box verification technique, the implementation often requires substantial knowledge of the internal details as can be seen especially in the event-driven case.

The resulting properties are highly expressive due to pattern matching and declarative programming and can be regarded as a kind of formal specification. Together with the properties which check the state-transition probabilities, we claim that the property-tests shown in this chapter fully specify both the event- and time-driven agent behaviour. This is a first example emphasising the usefulness of QuickCheck for testing ABS, providing a first strong evidence for the hypothesis that randomised property-testing is a good match for testing ABS.

%- time-driven is rather straight forward: just feed the population and increment the time, the agents are much more a black-box than in the event-driven approach, where there is much revealed about their inner workings through the events, thus we have to be much more explicit in event-driven. Note that we were not able to state a coverage for the infection of susceptible agents because the distribution is bimodal due to the combined probabliities of the exponential and uniform distriubtion. In event-driven we were able to state a coverage because we decoupled the process of making contact from infection: the makeContact events followed a uniform distribution as well, generated by oureselves, so we could state an expected coverage.

Curiously, the implementation of all the specifications and property-tests is a lot larger than the original implementations. Still, that is not the point here: we showed how to implement a full specification of an ABS model as a property-based test and we succeeded! This is definitely a strong indication that our hypothesis that randomised property-based testing is a suitable tool for testing ABS is valid. With unit tests we would be quite lost here: even for the SIR model, it is hard to enumerate all possible interactions and cases but by stating invariants as properties and generating random test-cases we make sure they are checked.

We have not looked into more complex testing patterns like the synchronous agent-interactions of Sugarscape. We didn't look into testing full agent and interacting agent behaviour using property-tests due to its complexity which would justify a part paper alone. Due to its inherent stateful nature with complex dependencies between valid states and agents actions we need a more sophisticated approach as outlined in \cite{de_vries_-depth_2019}, where the authors show how to build a meta-model and commands which allow to specify properties and valid state-transitions which can be generated automatically. We leave this for further research.

What is particularly powerful is that one has complete control and insight over the changed state before and after e.g. a function was called on an agent: thus it is very easy to check if the function just tested has changed the agent-state itself or the environment: the new environment is returned after running the agent and can be checked for equality of the initial one - if the environments are not the same, one simply lets the test fail. This behaviour is very hard to emulate in OOP because one can not exclude side-effect at compile time, which means that some implicit data-change might slip away unnoticed. In FP we get this for free.

Note that by exploiting lazy evaluation in the time-driven tests we scratch on what is conveniently possible in established approaches to ABS: we can let the simulation run potentially forever as in the case of the infected agent and rely on the correctness of the implementation to terminate in finite step when consuming the potentially infinite stream.

Note that we did not include an explicit environment in our agent specification tests and assumed a full connected network where all agents can make contact with each other. We think that property-based testing is highly useful there as well, especially when dealing with random environment like in Sugarscape or social and random networks \cite{jackson_social_2008, easley_networks_2010}. We leave this for further research but we hypothesise that for the SIR model all properties presented here should still hold under different environments.


\chapter{Testing SIR Invariants}
\label{ch:sir_invariants}

So far, our tests were stateless: only one computational step of an agent was considered by feeding a single event and ignoring the agent continuation. Also the events didn't contain any notion of time as they would carry within the queue. Feeding follow-up events into the continuation would make testing inherently stateful as we introduce history into the system. Such tests would allow to test the full life-cycle of one  agent or a full population.

In this chapter we will discuss how we can encode properties and specifications which require stateful testing. We define stateful testing here as: evolving a simulation state consisting of one or more agents over multiple events. Note that this also includes running the whole simulation.

\section{Deriving the invariants}
By informally reasoning about the agent-specification and by realising that they are in fact a state-machine with a on-directional flow from Susceptible to Infected to Recovered, we can come up with a few invariants which have to hold for any SIR simulation run independent of the random-number stream and the population:

\begin{enumerate}
	\item Simulation time is monotonic increasing. For each event an output of the current number of S,I and R agents is generated together with the time when the events occurred. This event time can stay the same between steps, will eventually increase but must never decrease. Obviously this invariant is a fundamental assumption in most simulations: time advances into the future and does not flow backwards.
	
	\item The number of total agents $N$ stays constant in SIR. The SIR model does not specify the dynamic creation or removal of agents during simulation. This is in contrast to the Sugarscape where, depending on the model parameters, this can be very well the case.
	
	\item The number of \textit{Susceptible} agents $S$ is monotonic decreasing. Susceptile agents \textit{might} become infected, reducing the total number of susceptible agents but they can never increase because neither an infected nor recovered agent can go back to susceptible.
	
	\item The number of \textit{Recovered} agents $R$ is monotonic increasing. This is due to infected agents \textit{will} recover, leading to an increase of recovered agents but once the recovered state is reached, there is no escape from it.
	
	\item The number of \textit{Infected} agents respects the invariant of the equation $I = N - (S + R)$ for every step. This follows directly from the first property which says $N = S + I + R$.
\end{enumerate}

\section{Encoding the invariants}
All of those properties are easily expressed directly in code (TODO: add a note in the benefits how extremely well this is encoded in functional programming).

\begin{HaskellCode}
sirInvariants :: Int -> [(Time, (Int, Int, Int))] -> Bool
sirInvariants n aos = timeInc && aConst && susDec && recInc && infInv
  where
    (ts, sirs)  = unzip aos
    (ss, _, rs) = unzip3 sirs

    -- 1. time is monotonic increasing
    timeInc = mono (<=)  ts
    -- 2. number of agents N stays constant in each step
    aConst = all agentCountInv sirs
    -- 3. number of susceptible S is monotonic decreasing
    susDec = mono (>=) ss
    -- 4. number of recovered R is monotonic increasing
    recInc = mono (<=)  rs
    -- 5. number of infected I = N - (S + R)
    infInv = all infectedInv sirs

    agentCountInv :: (Int, Int, Int) -> Bool
    agentCountInv (s,i,r) = s + i + r == n

    infectedInv :: (Int, Int, Int) -> Bool
    infectedInv (s,i,r) = i == n - (s + r)

    mono :: (Ord a, Num a) => (a -> a -> Bool) -> [a] -> Bool
    mono f xs = all (uncurry f) (pairs xs)

    pairs :: [a] -> [(a,a)]
    pairs xs = zip xs (tail xs)
\end{HaskellCode}

Putting this property into a QuickCheck test is straightforward. Note that we use randomise the model parameters $\beta$ (contact rate), $\gamma$ (infectivity) and $\delta$ (illness duration) because the properties have to hold for all positive, finite model parameters.

\begin{HaskellCode}
prop_sir_invariants :: Positive Double -- ^ Random beta, contact rate
                    -> Positive Double -- ^ Random gamma, infectivity
                    -> Positive Double -- ^ Random delta, illness duration
                    -> Property
prop_sir_invariants (Positive cor) (Positive inf) (Positive ild) = property (do
  -- generate population with size of up to 1000
  ss <- resize 1000 (listOf genSIRState)
  -- total agent count
  let n = length ss

  -- run the SIR simulation with a new RNG for up to t = 150 
  ret <- genSimulationSIR ss cor inf ild (-1) 150
  -- check invariants and return result
  return (sirInvariants n ret)
\end{HaskellCode}

Unsurprisingly all 100 tests pass. Note that we put a time-limit of 150 on the simulations to run, meaning that if a simulation does not terminate before that limit, it will be terminated at $t=150$. This is actually not necessary because we can reason that the SIR simulation \textit{will always} reach an equilibrium in finite steps thus not requiring an actual time-limit - we discuss this more in-depth in Chapter \ref{ch:equilibrium_totality}.

\subsection{Random Event Sampling}
An interesting question is whether or not these properties depend on correct interdependencies of events the agents send to each other in reaction to events they receive. Put in other words: do these invariants also hold under \textit{random event sampling}? To test this, instead of using the actual SIR implementation, which inserts the events generated by the agents into the event-queue, we wrote a new SIR kernel. It completely ignores the events generated by the agents and instead makes use of an infinite stream of random queue-elements from which it executes a given number, 100,000 in our case. Note that queue-elements contain a time-stamp, the receiver agent id and the actual event: the time-stamp is ensured to be increasing, to hold up the monotonic time property, the receiver agent id is drawn randomly from the constant list of all agents in the simulation and the actual event is generated completely randomly. As it turns out, the implementation of the agents ensure that the SIR properties are also invariant under \textit{random event sampling} - all tests pass.

\section{Time-driven}
We can expect that the invariants above also hold for the time-driven implementation. The property-test is exactly the same, with the time-driven implementation running instead of the even-driven one. A big difference is that is not necessary to check the property of monotonic increasing time, as it is an invariant statically guaranteed by arrowized FRP through the Yampa implementation. Due to the fact that the flow of time is always implicitly forward and no time-variable is explicitly made accessible within the code, it is not possible to violate the forward flow of time. Because of this, there is no need to check this property explicitly.

When we run the property-test we get a big surprise though: after a few test-cases the property-test fails due to a violation of the invariants! After a little bit of investigation it becomes clear that the invariant \textit{(3) number of susceptible agents is monotonic decreasing} is violated: in the failing test-case the number of susceptible agents is monotonic decreasing with the exception of one step where it \textit{increases} by 1 just to decrease by 1 in the next step. A coverage test reveals that this happens in about 66\% of 1,000 test-cases. How can this happen?

\medskip

The source of the problem is the use of \textit{dpSwitch} in the implementation of the \textit{stepSimulation} function as shown in Chapter \ref{sec:timedriven_firststep}. The d in \textit{dpSwitch} stands for delayed observation, which means that the output of the switch at time of switching is the output of the \textit{old} signal functions \cite{courtney_yampa_2003}. Speaking more technically: \textit{dpSwitch} is non-strict in its switching event, which ultimately results in old signal functions being run on new output which were but produced by those old signal functions: the output of time-step t is only visible in the time-step t=t+dt but the signal functions at time-step t are actually one step ahead. This is particularly visible at $t = 0$ and $t = \Delta t$, where the outputs are the same but the signal functions are not: the one at $t = \Delta t$ has changed already.

This has the desired effect that in the case of our SIR implementation, the population the agents see at time $t = t + \Delta t$ is the one from the previous step t, generated with the signal functions at time $t$. Due to the semantics of \textit{dpSwitch}, in the next switching event those signal functions from time $t$ are run again with the new input to produce the next output \textit{and} the new signal functions - in each step output is produced but due to the delay of \textit{dpSwitch} and the use of \textit{notYet}, we get this alternating behaviour.

This leads to trouble if the very rare case happens when a susceptible agent makes the transition from susceptible to recovered within one time-step. This is indeed possible due to the semantics of \textit{switch}, which is employed to make the state-transitions. In case a switching event occurs, \textit{switch} runs the signal function into which was switched immediately, which makes it highly unlikely but possible, that the susceptible agent, which has just switched into infected  recovers immediately by making the \textit{switch} to recovered. Why does this violate the property then?

\medskip

Lets assume that at $t = t + \Delta t$ the agents receive as input the population from time $t$ which contains, say 42, susceptible agents. A susceptible agent then makes the highly unlikely transition to recovered, reducing the number of susceptible agents by 1 to 41. In the next step the old signal function is run again but with the new input, which is slightly different, thus leading to slightly different probabilities. A susceptible agent has become recovered, which reduces probabilities of a susceptible agent becoming infected. When now the old signal function is run, it leads to a different output due to different probabilities: the susceptible agent stays susceptible instead of becoming infected (and then recovering). 

One way to solve this problem would be to use \textit{pSwitch}. It is the non-delayed, stricht version of \textit{dpSwitch}, which output at time of switching is the output of the \textit{new} signal functions. Using \textit{pSwitch} instead of \textit{dpSwitch} solves the problem because it will use the new signal functions in the second run because it is strict. Indeed, when using this, the property-test passes. This comes though at a high cost: due to \textit{pSwitch} strict semantics, which runs all the signal functions before and at time of switching, all agents are run twice in each step! This is clearly an unacceptable solution, especially because the time-driven approach already suffers severe performance problems. A more performant solution is to delay the susceptible agents output, as we have done already in \ref{sec:timedriven_firststep}. This solves the problem as well and the property-test passes. 

\section{Comparing time- and event-driven}
TODO: run event-driven properly with full random values, compare three runs:
-> MakeContact 1.0 dt with fixed number of contacts
-> MakeContact 1.0 dt with number of contacts exponentially distributed
-> MakeContact exponential dt with fixed number of contacts

Having two conceptually different implementations of the same model, an obvious idea is to compare the outputs of the two and verify whether both are producing the same dynamics or not. Speaking more technically, we are checking whether both simulations produce the same distributions under random model parameters and simulation time. We use QuickCheck to generate random values for $\beta$ (contact rate), $\gamma$ (infectivity) and $\delta$ (illness duration) as well as a random population. Then both simulation types are run with the same random parameters and random duration for 100 replications, collecting the output in the final step. The samples of these replications are then compared using a Mann-Whitney test with a 95\% confidence (p-value of 0.05). The reason for choosing this statistical test over e.g. a two sample t-test is that the Mann-Whitney test does not require the samples to be normally distributed, whereas the t-test assumes this. We know that the both implementations produce a bi-modal distribution as already discussed in \cite{macal_agent-based_2010} thus we cannot employ a two sample t-test to compare the distributions.
Our initial assumption is that we will reach a coverage of 90-95\%, meaning that this percentage of tests pass, indicating similar distributions. When running the default 100 random-test case though only in 81\% of the test-cases the difference in the distributions are statistically not significant and thus pass the Mann-Whitney test. Although this gives some robust evidence that both implementations produce similar distributions, they are not as close as we initially assumed. We look into this more in-depth in the verification against the SD specification in Chapter \ref{ch:prop_sirspec}.

TODO: more discussion about this! this is the place where we can really introduce that. also macal only discusses event-driven and not time-driven, why can we then assume that time-driven also produces a bi-modal distribution? show some histograms of failing test-cases with t-tests

\section{Discussion}
TODO

In the case of the time-driven implementation we saw that our initial assumption, that the invariants will hold for this implementation as well was wrong: QuickCheck revealed a \textit{very} subtle bug in our implementation. Although the probability of this bug is very low, QuickCheck found it due to its random testing nature. This is another \textit{strong} evidence, that random property-based testing is an \textit{excellent} approach for testing Agent-Based Simulations. On the other hand, this bug revealed the difficulties in getting the subtle semantics of FRP right to implement pure functional ABS. This is a strong case that in general an event-driven approach should be preferred, which is also much faster and also not subject to the sampling issues discussed in Chapter \ref{sec:timedriven_firststep}.

\chapter{Testing the SIR model specification}
\label{ch:prop_sirspec}

In the previous chapters we have established the correctness of our event- and time-driven implementation up to our informal specification, we derived from the formal SD specification from Chapter \ref{sec:sir_model}. What we are lacking is a verification whether the implementations also match the formal SD specification or not. In the process of verification, we need to make sure it is correct up to some specification. We aim at connecting the agent-based implementation to the SD specification, by formalising it into properties within a property-test. The SD specification can be given through the differential equations shown in Chapter \ref{sec:sir_model}, which we repeat here:

\begin{equation}
\begin{split}
\frac{\mathrm d S}{\mathrm d t} = -infectionRate \\
\frac{\mathrm d I}{\mathrm d t} = infectionRate - recoveryRate \\
\frac{\mathrm d R}{\mathrm d t} = recoveryRate 
\end{split}
\quad
\begin{split}
infectionRate = \frac{I \beta S \gamma}{N} \\
recoveryRate = \frac{I}{\delta} 
\end{split}
\end{equation}
\label{eq:sir_delta_rates}

Solving these equations is done by integrating over time. In the SD terminology, the integrals are called \textit{Stocks} and the values over which is integrated over time are called \textit{Flows}. At $t = 0$ a single agent is infected because if there wouldn't be any infected agents, the system would immediately reach equilibrium - this is also the formal definition of the steady state of the system: as soon as $I(t) = 0$ the system won't change any more.

\begin{align}
S(t) &= N - I(0) + \int_0^t -infectionRate\, \mathrm{d}t \\
I(0) &= 1 \\
I(t) &= \int_0^t infectionRate - recoveryRate\, \mathrm{d}t \\
R(t) &= \int_0^t recoveryRate\, \mathrm{d}t
\end{align}

\section{Deriving a property}

TODO: make clear that we compare the numbers of susceptible, infected and recovered in the last step of the ABS and SD implementations.

The goal is now to derive a property which connects those equations with our implementation. We have to be careful and realis a fundamental difference between the SD and ABS implementations: SD is deterministic and continuous, ABS is stochastic and discrete. Thus we cannot compare single runs but we can only compare averages: stated informally, the property we want to implement is that the ABS dynamics matches the SD ones \textit{on average}, independent of the finite population size, model parameters $\beta$ (contact rate), $\gamma$ (infectivity) and $\delta$ (illness duration) and duration of the simulation. To be able to compare averages, we run 100 replications of the ABS simulation with same parameters except a different random-number generator in each replication. We then run a two-sided t-test on the replication values with the expected values from the SD dynamics.

\begin{HaskellCode}
compareSDToABS :: Int     -- ^ Initial number of susceptibles
               -> Int     -- ^ Initial number of infected
               -> Int     -- ^ Initial number of recovered
               -> [Int]   -- ^ Final Number of susceptibles in replications
               -> [Int]   -- ^ Final Number of infected in replications
               -> [Int]   -- ^ Final Number of recovered in replications
               -> Double  -- ^ beta (contact rate)
               -> Double  -- ^ gamma (infectivity)
               -> Double  -- ^ delta (illness duration)
               -> Time    -- ^ duration of simulation
               -> Bool
compareSDToABS s0 r0 i0
               ss is rs
               beta gamma delta t = sTest && iTest && rTest
  where
    -- run SD simulation to get expected averages
    (s, i, r) = simulateSD s0 i0 r0 beta gamma delta t
    
    confidence = 0.95
    sTest = tTestSamples TwoTail s (1 - confidence) ss
    iTest = tTestSamples TwoTail i (1 - confidence) is
    rTest = tTestSamples TwoTail r (1 - confidence) rs
\end{HaskellCode}

The implementation of \textit{simulateSD} is discussed in-depth in the Appendix \ref{app:sdSimulation}. We are very well aware that comparing the output against an SD simulation is dangerous: after all, why should be trust the SD implementation? As outlined in the Appendix \ref{app:sdSimulation}, great care has been taken to ensure the correctness: the formulas from the SIR specification are directly encoded in code, allowed by Yampas arrowized FRP which guarantees that at least that translation step is correct - we then only rely on a small enough sampling rate and the correctness of the Yampa library. The former one is very well in our reach and we pick a sufficiently small samply rate; the latter one is beyond our reach but we expect the library to me mature enough to be correct for our purposes.

\section{Running the tests}
Implementing a property-test is straight-forward. Here we give the implementation for the time-driven SIR implementation, the implementation for the event-driven SIR implementation is exactly the same with the exception of \textit{genTimeSIRRepls}. We again make use of the \textit{checkCoverage} feature of QuickCheck to get statistical robust results: we expect that in 90\% of all test-cases the SD and ABS dynamics match \textit{on average}. QuickCheck will run as many tests as necessary to reach a statistically robust result which either allows to reject or accept this hypothesis.

\begin{HaskellCode}
prop_sir_time_spec :: Positive Double  -- ^ contact rate
                   -> UnitRange        -- ^ infectivity, within range (0,1)
                   -> Positive Double  -- ^ illness duration
                   -> TimeRange        -- ^ time to run
                   -> Property
prop_sir_time_spec 
    (Positive cor) (UnitRange inf) (Positive ild) (TimeRange t) = checkCoverage (do
  -- running 100 replications for the ABS SIR implementation
  let repls = 100
  -- generate large random population
  as <- resize 1000 (listOf genSIRState)
  -- run replications of time-driven SIR implementation
  (ss, is, rs) <- unzip3 <$> genTimeSIRRepls repls as cor inf ild t
  -- check if they match 
  let prop = compareSDToABS as ss is rs cor inf ild t
  -- we expect 80% to pass and use checkCoverage to get statistical robust result
  return $ cover 80 prop "SIR time-driven passes t-test with simulated SD" True
\end{HaskellCode}

TODO: run the tests

\section{Discussion}
By using QuickCheck, we showed how to connect the ABS implementation to the SD specification by deriving a property, based on the SD specification. This property is directly expressed in code and tested through generating random test-cases with random agent populations. We assumed that the underlying SIR implementation, more specific, that all agent behaviour, is correct - we explore testing of individual agent behaviour in the later chapters.

Although our initial idea of matching the ABS implementation to the SD specifications has not worked out in an exact way, we still showed a way of formalizing and expressing these relations in code and testing them using QuickCheck. By allowing failure in our tests using the \textit{maxFailPercent} parameter, we confirmed the importance of selecting an optimal $\Delta t$ as already pointed out in Chapter \ref{sub:timedriven_results}. By having measure of failure, we can use a property-test also as a way of systematically searching for the smallest optimal $\Delta t$, relieving us from making conservative guesses. For the event-driven implementation, there is no such issue and we have verified that it produces qualitatively the same results.

The results showed that the ABS implementation comes close to the original SD specification but does not match it exactly - it is indeed richer in its dynamics as \cite{macal_agent-based_2010, figueredo_comparing_2014} have already shown. Our approach might work out better for a different model, which has a better behaved underlying specification than the bimodal SIR. % for which a proper statistical analysis is not the aim and focus of this thesis is left for other researchers to dwell upon.

\chapter{Hypotheses in Sugarscape}
\label{ch:prop_exploratory}

TODO: replace maxFailPercent with cover and checkCoverage and configure quickcheck to run only 10 tests

In this chapter we look at how property-based testing can be made of use to verify the \textit{exploratory} Sugarscape model \cite{epstein_growing_1996} as already introduced in Chapter \ref{sec:sugarscape}. Whereas in the previous chapter on testing the explanatory SIR case-study we had an analytical solution, the fundamental difference in the exploratory Sugarscape model is that none such analytical solutions exist. This raises the question, which properties we can actually test in such a mode.

The answer lies in the very nature of exploratory models: they exist to explore and understand phenomena of the real world. Researchers come up with a model to explain the phenomena and then (hopefully) come up with a few questions and  \textit{hypotheses} about the emergent properties. The actual simulation is then used to test and refine the hypotheses. Indeed, descriptions, assumptions and hypotheses of varying formal degree abound in the Sugarscape model. Examples are: \textit{the carrying capacity becomes stable after 100 steps; when agents trade with each other, after 1000 steps the standard deviation of trading prices is less than 0.05; when there are cultures, after 2700 steps either one culture dominates the other or both are equally present}. 

We show how to use property-testing to formalise and check such hypotheses. For this purpose we undertook a full \textit{verification} of our implementation \footnote{The code can be accessed freely from \url{https://github.com/thalerjonathan/phd/tree/master/public/towards/SugarScape/sequential}} from Chapter \ref{sec:sugarscape}. We validated it against the book \cite{epstein_growing_1996} and a NetLogo implementation \cite{weaver_replicating_2009} \footnote{\url{https://www2.le.ac.uk/departments/interdisciplinary-science/research/replicating-sugarscape}}. A longer report on the details of this validation process is attached as Appendix \ref{app:validating_sugarscape}, in this section we focus on QuickChecks role in this process.

The property we test for is whether \textit{the emergent property / hypothesis under test is stable under replicated runs} or not. To put it more technical, we use QuickCheck to run multiple replications with the same configuration but with different random-number streams and require that the tests all pass. During the verification process described in Appendix \ref{app:validating_sugarscape} we have derived and implemented property-tests for the following hypotheses:

\begin{enumerate}
	\item Disease Dynamics all recover - When disease are turned on, if the number of initial diseases is 10, then the population is  able to rid itself completely from all disease within 100 ticks. 
	
	\item Disease Dynamics minority recover - When disease are turned on, if the number of initial diseases is 25, the population is not able to rid itself completely from all diseases within 1,000 ticks.
	
	\item Trading Dynamics - When trading is enabled, the trading prices stabilise after 1,000 ticks with the standard deviation of the prices having dropped below 0.05.
	
	\item Cultural Dynamics - When having two cultures, red and green, after 2,700 ticks, either the red or the blue culture dominates or both are equally strong. If they dominate they make up 95\% of all agents, if they are equally strong they are both within 45\% - 55\%.
	
	\item Inheritance Gini Coefficient - According to the book, when agents reproduce and can die of age then inheritance of their wealth leads to an unequal wealth distribution measured using the Gini Coefficient \textit{averaging} at 0.7.
	
	\item Carrying Capacity - When agents don't mate nor can die from age (chapter II), due to the environment, there is an \textit{average} maximum carrying capacity of agents the environment can sustain. The capacity should be reached after 100 ticks and should be stable from then on.
		
	\item Terracing - When resources regrow immediately, after a few steps the simulation becomes static. Agents will stay on their terraces and will not move any more because they have found the best spot due to their behaviour. About 45\% will be on terraces and 95\% - 100\% are static and not moving any more.
\end{enumerate}

\section{Implementation}
To implement this, we implement a custom data-generator to produce output from a Sugarscape simulation. The generator takes the number of ticks and the scenario with which to run the simulation and returns a list of outputs, one for each tick.

\begin{HaskellCode}
sugarscapeUntil :: Int -> SugarScapeScenario -> Gen [SimStepOut]
sugarscapeUntil ticks params = do
  -- create a random-number generators
  g <- genStdGen
  -- initialise the simulation state with the given random-number generator
  -- and the parameters
  let (simState, _, _) = initSimulationRng g params
  -- run the simulation with the given state for number of ticks
  return (simulateUntil ticks simState)
\end{HaskellCode}

Using this generator, we can very conveniently produce sugarscape data within a property. Depending on the problem, we can generate only a single run or multiple replications, in case the hypothesis is assuming \textit{averages}. To see its use, we show the encoding of the \textit{Disease Dynamics (1)} hypothesis. Its type is \textit{Property}, which is required by QuickChecks top-level testing function. To generate a property, the \textit{property} function is used which takes a \textit{Gen Bool} computation or a simple \textit{Bool} function as predicate to indicate success (True) or failure (False).

\begin{HaskellCode}
prop_disease_allrecover :: Property
prop_disease_allrecover = property (do
  -- after 100 ticks...
  let ticks  = 100
  -- ... given Animation V-1 parameter configuration ...
  let params = mkParamsAnimationV_1
  -- ... from 1 sugarscape simulation ...
  aos <- sugarscapeLast ticks params
  -- ... counting all infected agents ...
  let infected = length (filter (==False)) map (null . sugObsDiseases . snd) aos
  -- ... should result in all agents to be recovered
  return (infected == 0))
\end{HaskellCode}

From the implementation it becomes clear, that this hypothesis states that the property has to hold \textit{for all} replications. The \textit{Inheritance Gini Coefficient (5)} hypothesis on the other hand assumes that the Gini Coefficient \textit{averages} at 0.7. We cannot average over replicated runs of the same property thus we generate multiple replications of the sugarscape data within the property and employ a two-sided T-Test with a 95\% confidence to test the hypothesis:

\begin{HaskellCode}
prop_gini :: Int      -- ^ Number of replications
          -> Double   -- ^ Confidence of the t-test
          -> Property
prop_gini repls confidence = property (do
  -- after 1000 ticks...
  let ticks = 1000
  -- ... the gini coefficient should average at 0.7 ...
  let expGini = 0.7
  -- ... given the Figure III-7 parameter configuration ...
  let params = mkParamsFigureIII_7
  -- ... from repls replciations ... 
  gini <- vectorOf repls (genGiniCoeff ticks params)
  -- on a two-tailed t-test with given confidence
  let giniTTest = tTestSamples TwoTail expGini (1 - confidence) gini
  return giniTTest
  
genGiniCoeff :: Int -> SugarScapeScenario -> Gen Double
genGiniCoeff ticks params = do
  -- generate sugarscape data
  aos <- sugarscapeUntil ticks params
  -- extract wealth of the agents in the last step
  let agentWealths = map (sugObsSugLvl . snd) (last aos)
  -- compute gini coefficient and return it
  return (giniCoeff agentWealths)
\end{HaskellCode}

\section{Running the tests}
As already pointed out, QuickCheck tries to run by default up to 100 replications  of a property and if all evaluate to \textit{True} the property-test succeeds. On the other hand, QuickCheck will stop at the first predicate which evalutes to \textit{False} and marks the whole property-test as failed, no matter how many replications got through already. 

Due to the duration even 1,000 ticks can take to compute, to get a first estimate of our hypotheses tests within reasonable time, we reduce the number of maximum successful replications required to 10 and when doing t-tests 10 replications are run there as well. Unfortunately, when running the tests only the Disease Dynamics (1) and (2) go through, all the other tests fail. It is important to understand that QuickCheck is always initialised with a new random-number seed when run, thus we might have just been unlucky. Unfortunately, when we run it again, this happens again, thus there seems to be something wrong with our approach.

\subsection{Allowing failure}
It is arguably the case that the binary approach of QuickCheck, where the whole property-test fails when a single replication fails, is too strict for testing ABS in general and our hypotheses in particular. The reason for that is, that due to ABS stochastic nature, the hypotheses might hold for a large number of replications but not strictly for all.

As a remedy, we can use \textit{maxFailPercent} \footnote{As of the time of writing this thesis (2nd April 2019), this only exists as a pull request \url{https://github.com/nick8325/quickcheck/pull/239} and has not been merged into the main branch of QuickCheck. Thus we use the QuickCheck from \url{https://github.com/stevana/quickcheck/tree/feat/max-failed-percent} who has provided the implementation of \textit{maxFailPercent}.} as a configuration argument to QuickCheck, which allows the failure of a given percentage of replications. The argument behaves in a way that it tries to run up to the 100 default (or whatever the configuration is) successful replications but fails the overall property-test if the percentage of failed replications is reached. By switching from a binary PASS/FAIL to a more probabilistic measure, reflecting reliability, we have now a more appropriate tool for testing the suitability of our hypotheses. 

We run the tests again with 10 replications each but now allowing 100\% of failure in each case to see how reliable each hypothesis is. In one specific run we get the following result:

\begin{enumerate}
	\item Disease Dynamics all recover: \textit{+++ OK, passed 10 tests.}

	\item Disease Dynamics minority recover: \textit{+++ OK, passed 10 tests.}
		
	\item Trading Dynamics: \textit{+++ OK, passed 10 tests; 2 failed (16\%).} (In total 12 tests (replications) were run, out of which 2 failed, which is a 16\% failure rate.)
	
	\item Cultural Dynamics: \textit{+++ OK, passed 10 tests; 3 failed (23\%).}

	\item Inheritance Gini Coefficient: \textit{*** Failed! Passed only 0 tests; 10 failed (100\%) tests.}

	\item Carrying Capacity: \textit{+++ OK, passed 10 tests; 2 failed (16\%).}

	\item Terracing: \textit{+++ OK, passed 10 tests; 2 failed (16\%).}
\end{enumerate}

How to deal with the failure of the hypotheses is obviously highly model specific. A first approach is to increase the number of replications to run to 100 to get a more robust estimate of the failure rate. If the failure rate stays within reasonable ranges then one can arguably assume that the hypothesis is valid for sufficiently enough cases. On the other hand, if the failure rate escalates, then it is reasonable to deem the hypothesis invalid and refine it or even abandon it altogether.

With the exception of the Gini Coefficient, we accept the failure rate of the hypotheses we presented here and deem them sufficiently valid for the task at hand. In case of the Gini Coefficient, none of the replication was successful, which makes it obvious that it does \textit{not} average at 0.7. Thus the hypothesis as stated in the book does not hold and is invalid. One way to deal with it would be to simply delete it. Another, more constructive approach, is to keep it but require all replications to fail by marking it with \textit{expectFailure} instead of \textit{property}. In this way an invalid hypothesis is marked explicitly and acts as documentation and also as test.

Note that we disabled shrinking for hypothesis testing as it has no meaning here. The only thing which would be shrunk would be the seed of the random-number generator, which has no intrinsic meaning.

\section{Discussion}
In this chapter we showed how to use QuickCheck to formalise and check hypotheses about an \textit{exploratory} agent-based model, in which no ground truth exists. Due to ABS stochastic nature in general it became obvious that to get a good measure of a hypotheses validity we need to allow failure using the \textit{maxFailPercent} argument of QuickCheck. This allowed us to show that the hypotheses we have presented are sufficiently valid for the task at hand and can indeed be used for expressing and formalising emergent properties of the model and also as regression tests within a TDD cycle.

\chapter{The Equilibrium-Totality Correspondence}
\label{ch:equilibrium_totality}

TODO UNFINISHED BUT OPTIONAL
- write NON-DEPENDENTLY TYPED ARGUMENT
- NEEDS SUBSTANTIAL RESEARCH: IMPLEMENT A TOTAL SIR IMPLEMENTATION IN IDRIS
- write a short intro into dependent types
- the question will remain: does this chapter really belong in this thesis?

In the property-tests of Chapter \ref{ch:sir_invariants} and \ref{sec:timedriven_specification} we limited the time an individual simulation is run to a random range between 0 and 50. In this context, the decision to do so was practical to guarantee that we will actually terminate as both the event- and time-driven implementations would run forever if no time- and/or event-limit is specified \footnote{Note that the event-driven implementation would terminate if the event-queue is empty but in the case of the SIR this will never be the case due to susceptible agents keep scheduling \textit{MakeContact}, resulting in an infinite stream of events.}.

However, restricting the simulation to a time- and/or event-limit is not necessary in a correct SIR implementation because it  \textit{will} reach an equilibrium \textit{within finite time} at which point the simulation can be terminated. This is the case as soon as there are no more infected agents: intuitively this is clear because only infected agents can lead to infections of susceptible agents which then make the transition to recovered after having gone through the infection phase. The infected agents themselves \textit{will} recover within finite time. Thus we can conclude that a correct implementation of the SIR model must enter a steady state in finite time.

Using this informal reasoning, we change the property-test from Chapter \ref{ch:sir_invariants} to encode this property implicitly.

\begin{HaskellCode}
prop_sir_invariants :: Positive Int    -- ^ beta, contact rate
                    -> Property        -- ^ gamma, infectivity in range (0,1)
                    -> Positive Double -- ^ delta, illness duration
                    -> [SIRState]      -- ^ population
                    -> Property
prop_sir_invariants 
    (Positive cor) (P inf) (Positive ild) as  = property (do
  -- CHANGED: run the SIR simulation with UNRESTRICTED time
  ret <- genSimulationSIR ss cor inf ild 0
  -- CHANGED: take data as long as not in equilibrium
  let ret' = takeWhile ((>0).snd3.snd) ret
  -- check invariants and return result
  return (sirInvariants (length as) ret')
\end{HaskellCode}

Unfortunately this code is dangerous: generally, we cannot distinguish between a very long or infinitely running simulation. It might be the case that there is a bug in our implementation which would violate the property that all infected agents eventually recover, in which case \textit{takeWhile} might run forever. This means that we cannot write a property-test, which could tell us whether this property holds or not for both our time- and event-driven implementations - it is in general non-computable. Obviously this is nothing new and was established in the 1930s through the work of Turing \cite{turing_computable_1937}. The question is now: what can we do about it?

The solution is to abandon the power of general recursion and Turing-completeness and switch to a different kind of pure functional programming language in which programs can be checked for totality by the compiler. These languages have a different kind of type system, called dependent types. Generally, dependent types add the following concepts to pure functional programming:

\begin{enumerate}
	\item Totality and termination - A total function is defined in \cite{brady_type-driven_2017} as one that terminates with a well-typed result or produces a non-empty finite prefix of a well-typed infinite result in finite time. In dependently typed languages which abandon Turing completeness this can be checked at compile time under certain circumstances.
	
	\item Types are first-class citizen - In dependently typed languages, types can depend on any \textit{values}, and can be \textit{computed} at compile-time which makes them first-class citizen. This allows to compute the return type of a function depending on its input values. Note that this requires totality, otherwise type-checking would non-decidable and potentially non-terminating.
	
	\item Types as \textit{constructive} proofs - Because types can depend on any values and can be computed at compile-time, they can be used as constructive proofs (see \ref{sub:dep_foundations}) which must terminate, this means a well-typed program (which is itself a proof) is always terminating which in turn means that it must consist out of total functions.
\end{enumerate}

There exist a number of excellent introduction to dependent types which we use as main ressources for this section: \cite{thompson_type_1991, program_homotopy_2013, stump_verified_2016, brady_type-driven_2017, pierce_programming_2018}. We are using Idris \cite{brady_idris_2013} as the language of choice as it is very close to Haskell with focus on real-world application and running programs as opposed to other languages with dependent types e.g. Agda and Coq which serve primarily as proof assistants.

Dependent types are a very powerful addition to functional programming as they allow us to express even stronger guarantees about the correctness of programs \textit{already at compile-time}. They go as far as allowing to formulate programs and types as constructive proofs which must be \textit{total} by definition \cite{thompson_type_1991, mckinna_why_2006, altenkirch_pi_2010}. 

So far no research using dependent types in agent-based simulation exists at all. We have already started to explore this for the first time and ask more specifically how we can add dependent types to our functional approach, which conceptual implications this has for ABS and what we gain from doing so. We are using Idris \cite{brady_idris_2013} as the language of choice as it is very close to Haskell with focus on real-world application and running programs as opposed to other languages with dependent types e.g. Agda and Coq which serve primarily as proof assistants.

We hypothesise, that  dependent types will allow us to push the correctness of agent-based simulations to a new, unprecedented level by narrowing the gap between model specification and implementation. The investigation of dependent types in ABS will be the main unique contribution to knowledge of my Ph.D.\\

There is a strong relation between property-based tests and dependent types: in property-based testing we express specifications / properties / laws in code and test their invariance at run-time by random sampling the space. In dependent-types it is possible to express such properties already statically in types. This is the subject of the next part of the thesis which tries to move towards dependent types in ABS.



%Idris is Turing-complete but is able to check the totality of a function under some circumstances but not in general as it would imply that it can solve the halting problem.

\section{A total SIR implementation}
In this section we want to implement a total agent-based SIR simulation, where the termination does NOT depend on time (is not terminated after a finite number of time-steps, which would be trivial).

Dependent Types and Idris' ability for totality- and termination-checking should theoretically allow us to proof that an agent-based SIR implementation terminates after finite time: if an implementation of the agent-based SIR model in Idris is total it is a formal proof by construction. Note that such an implementation should not run for a limited virtual time but run unrestricted of the time and the simulation should terminate as soon as there are no more infected agents, returning the termination time as an output. Also if we find a total implementation of the SIR model and extend it to the SIR+S model, which adds a cycle from Recovered back to Susceptible, then the simulation should become again non-total as reasoned above.

The HOTT book \cite{program_homotopy_2013} states that lists, trees,... are inductive types/inductively defined structures where each of them is characterized by a corresponding \textit{induction principle}. Thus, for a constructive proof of the totality of the agent-based SIR model we need to find the induction principle of it. This leaves us with the question of what the inductive, defining structure of the agent-based SIR model is? Is it a tree where a path through the tree is one way through the simulation or is it something else? It seems that such a tree would grow and then shrink again e.g. infected agents. Can we then apply this further to (agent-based) simulation in general?

By this reasoning, a non-total, correctly implemented agent-based simulations of the SIR model will eventually terminate (note that this is independent of which environment is used and which parameters are selected). Still this does not formally proof that the agent-based approach itself will terminate and so far no formal proof of the totality of it was given.

Thus an agent-based implementation of the SIR simulation has to terminate if it is implemented correctly because all infected agents will recover after a finite number of steps after then the dynamics will be in equilibrium. Thus we have the following conditions for totality:

\begin{enumerate}
	\item The simulation shall be terminated when there are no more infected agents.
	
	\item All infected agents will recover after a finite number of time, which means that the simulation will eventually run out of infected agents. 
	
	Unfortunately this criterion alone does not suffice because when we look at the SIR+S model, which adds a cycle from Recovered back to Susceptible, we have the same termination criterion, but we cannot guarantee that it will run out of infected. We need an additional criteria.

	\item The source of infected agents is the pool of susceptible agents which is monotonic decreasing (not strictly though!) because recovered agents do NOT turn back into susceptibles.
	
	\item finite number of initial susceptibles
	
	\item finite number of initial infected
	
	\item finite illness duration
	
	\item finite contact rate
\end{enumerate}

A central question in tackling this is whether to follow a model- or an agent-centric approach. The former one looks at the model and its specifications as a whole and encodes them e.g. one tries to directly find a total implementation of an agent-based model. The latter one looks only at the agent level and encodes that as dependently typed as possible and hopes that model guarantees emerge on a meta-level - put otherwise: does the totality of an implementation emerge when we follow an agent-centric approach?

%TODO: \url{https://stackoverflow.com/questions/19642921/assisting-agdas-termination-checker/39591118}

%The authors of \cite{ionescu_dependently-typed_2012} discuss how to use dependent types to specify fundamental theorems of economics, unfortunately they are not computable and thus not constructive, thus leaving it more to a theoretical, specification side.
%Ionesus talk on dependently typed programming in scientific computing
%https://www.pik-potsdam.de/members/ionescu/cezar-ifl2012-slides.pdf
%Ionescus talk on Increasingly Correct Scientific Computing
%%https://www.cicm-conference.org/2012/slides/CezarIonescu.pdf
%Ionescus talk on Economic Equilibria in Type Theory
%https://www.pik-potsdam.de/members/ionescu/cezar-types11-slides.pdf
%Ionescus talk on Dependently-Typed Programming in Economic Modelling
%https://www.pik-potsdam.de/members/ionescu/ee-tt.pdf

\section{Constructivism in ABS}
\label{sub:dep_foundations}
The main theoretical and philosophical underpinnings of dependent types as in Idris are the works of Martin-L\"of intuitionistic type theory. The view of dependently typed programs to be proofs is rooted in a deep philosophical discussion on the foundations of mathematics, which revolve around the existence of mathematical objects, with two conflicting positions known as classic vs. constructive \footnote{We follow the excellent introduction on constructive mathematics \cite{thompson_type_1991}, chapter 3.}. In general, the constructive position has been identified with realism and empirical computational content where the classical one with idealism and pragmatism.

In the classical view, the position is that to prove $\exists x. P(x)$ it is sufficient to prove that $\forall x. \neg P(x)$ leads to a contradiction. The constructive view would claim that only the contradiction is established but that a proof of existence has to supply an evidence of an $x$ and show that $P(x)$ is provable. In the end this boils down whether to use proof by contradiction or not, which is sanctioned by the law of the excluded middle which says that $A \lor \neg A$ must hold. The classic position accepts that it does and such proofs of existential statements as above, which follow directly out of the law of the excluded middle, abound in mathematics \footnote{Polynomial of degree n has n complex roots; continuous functions which change sign over a compact real interval have a zero in that interval,...}. The constructive view rejects the law of the excluded middle and thus the position that every statement is seen as true or false, independently of any evidence either way. \cite{thompson_type_1991} (p. 61): \textit{The constructive view of logic concentrates on what it means to prove or to demonstrate convincingly the validity of a statement, rather than concentrating on the abstract truth conditions which constitute the semantic foundation of classical logic}.

To prove a conjunction $A \land B$ we need prove both $A$ and $B$, to prove $A \lor B$ we need to prove one of $A, B$ and know which we have proved. This shows that the law of the excluded middle can not hold in a constructive approach because we have no means of going from a proof to its negation. Implication $A \Rightarrow B$ in constructive position is a transformation of a proof $A$ into a proof $B$: it is a function which transforms proofs of $A$ into proofs of $B$. The constructive approach also forces us to rethink negation, which is now an implication from some proof to an absurd proposition (bottom): $A \Rightarrow \perp$. Thus a negated formula has no computational context and the classical tautology $\neg \neg A \Rightarrow A$ is then obviously no longer valid.  Constructively solving this would require us to be able to effectively compute / decide whether a proposition is true or false - which amounts to solving the halting problem, which is not possible in the general case.

A very important concept in constructivism is that of finitary representation / description. Objects which are infinite e.g. infinite sets as in classic mathematics, fail to have computational computation, they are not computable. This leads to a fundamental tenet in constructive mathematics: \cite{thompson_type_1991} (p. 62): \textit{Every object in constructive mathematics is either finite [..] or has a finitary description}

Concluding, we can say that constructive mathematics is based on principles quite different from classical mathematics, with the idealistic aspects of the latter replaced by a finitary system with computational content. Objects like functions are given by rules, and the validity of an assertion is guaranteed by a proof from which we can extract relevant computational information, rather than on idealist semantic principles. 

All this is directly reflected in dependently typed programs as we introduced above: functions need to be total (finitary) and produce proofs like in \textit{checkEqNat} which allows the compiler to extract additional relevant computational information. Also the way we described the (infinite) natural numbers was in an finitary way. In the case of decidable equality, the case where it is not equal, we need to provide an actual proof of contradiction, with the type of Void which is Idris representation of $\perp$. 

\subsection{Verification, Validation and Dependent Types}
\label{sec:dep_vav_deptypes}
Dependent types allow to encode specifications on an unprecedented level, narrowing the gap between specification and implementation - ideally the code becomes the specification, making it correct-by-construction. The question is ultimately how far we can formulate model specifications in types - how far we can close the gap in the domain of ABS. Unless we cannot close that gap completely, to arrive at a sufficiently confidence in correctness, we still need to test all properties at run-time which we cannot encode at compile-time in types.

Nonetheless, dependent types should allow to substantially reduce the amount of testing which is of immense benefit when testing is costly. Especially in simulations, testing and validating a simulation can often take many hours - thus guaranteeing properties and correctness already at compile time can reduce that bottleneck substantially by reducing the number of test-runs to make.

Ultimately this leads to a very different development process than in the established object-oriented approaches, which follow a test-driven process. There one defines the necessary interface of an object with empty implementations for a given use-case first, then writes tests which cover all possible cases for the given use-case. Obviously all tests should fail because the functionality behind it was not implemented yet. Then one starts to implement the functionality behind it  step-by-step until no test-case fails. This means that one runs all tests repeatedly to both check if the test-case one is working on is not failing anymore and to make sure that old test-cases are not broken by new code. The resulting software is then trusted to be correct because no counter examples through test hypotheses, could be found. The problem is: we could forget / not think of cases, which is the easier the more complex the software becomes (and simulations are quite complex beasts). Thus in the end this is a deductive approach.

With pure functional programming and dependent types the process is now mostly constructive, type-driven (see \cite{brady_type-driven_2017}). In that approach one defines types first and is then guided by these types and the compiler in an interactive fashion towards a correct implementation, ensured at compile-time. As already noted, the ABS methodology is constructive in nature but the established object-oriented test-driven implementation approach not as much, creating an impedance mismatch. We expect that a type-driven approach using dependent types reduces that mismatch by a substantial amount.
Models like the Sugarscape are exploratory in nature and don't have a formal ground truth where one could derive equilibria or dynamics from and validate with. In such models the researchers work with informal hypotheses which they express before running the model and then compare them informally against the resulting dynamics.

It would be of interest if dependent types could be made of use in encoding hypotheses on a more constructive and formal level directly into the implementation code. So far we have no idea how this could be done but it might be a very interesting application as it allows for a more formal and automatic testable approach to hypothesis checking.

Note that \textit{validation} is a different matter here: independent of our implementation approach we still need to validate the simulation against the real-world / ground-truth. This obviously requires to run the full simulation which could take up hours in either programming paradigm, making them absolutely equal in this respect. Also the comparison of the output to the real-world / ground-truth is completely independent to the paradigm. The fundamental difference happens in case of changes made to the code during validation: in case of the established test-driven object-oriented approach for every minor change one (should) re-run all tests, which could take up a substantial amount of additional time. Using a constructive, type-driven approach this is dramatically reduced and can often be completely omitted because the correctness of the change can be either guaranteed in the type or by informally reasoning about the code.

%TODO: not sure where to put this
ABS as a constructive / generative science, follows Poperian approach of falsification: we try to construct a model which explains a real-world (empirical) phenomenon - if validation shows that the generated dynamics match the ones of the real-world sufficiently enough, we say that we have found \textit{a} hypothesis (the model) which emergent properties explains the real-world phenomenon sufficiently enough. This is not a proof but only one possible explanation which holds for now and might be falsified in the future.

When we implement our simulation things change a bit as we add another layer: the conceptual model, describing the phenomenon, which is an abstraction of reality. This description can be of many forms but can be regarded on a line between completely formal (economic models) to informal (sociology) but the implementation will follow that description. The fundamental difference here is that in this case we want our implementation to be exactly the same as the conceptual model. Contrary to the real-world, where it is not possible to find a \textit{true} model (as was argued by Popper), on this level we actually can construct an implementation which matches the conceptual model exactly because we have a description of the conceptual model. In the end we transform the conceptual model description in code, which is itself a formal description. In this translation process (speak: implementation / programming), one can make an endless number of mistakes. Generally we can distinguish between two classes of mistakes: 
1) conceptual mistakes - wrong translation of the model specifications into code due to various reasons e.g. imprecise description, human error. The more precise an unambiguous a model description is, the less probable conceptual mistakes will be.
2) internal mistakes - normal programming mistakes e.g. access of arrays out of bounds, ... also using correlated Random Number generators.
%
Level 0: Real-World phenomenon
Level 1: Conceptual model of the real-world phenomenon
Level 2: Implementation of the conceptual model
%
Note that we must speak of falsification and constructiveness on two different levels:
- validation level: do the results of the conceptual model match the real-world phenomenon? the conceptual model is the hypothesis which says that its mechanics are sufficient to generate / construct the real-world phenomenon. At this level we are not interested in the implementation level anymore - the implemented model \textit{is} (seen as) the conceptual model, and one only compares its output to the real-world. If the dynamics match, then we got a valid hypothesis which works for now. If the dynamics do NOT match, then the hypothesis (the model) is falsified and one needs to adjust / change the hypothesis (model). The validation will happen by tests, there is no other way, we have no formal specification of the real-world, we can only observe empirically the phenomena, so we run tests which try to falsify the outputs of the model: assuming it will generate phenomena of the real-world and test if it does.
- implementation \& verficiation level: in this step we are matching the code to the conceptual model. Here we are not only restricted to a test-driven approach because we have a more or less formal description of the conceptual model which we directly encode in our programming language. If the language allows to express model specifications already at compile-time then this means that the implementation narrows the gap between model specification and implementation which means it does not need to be tested at run-time because it is guaranteed for all inputs for all time. 

The constructiveness of ABS and impendance mismatch: ABS methodology is constructive but the established implementation approach not too much, creating an impedance mismatch. this is especially visible in the test-driven development dependent types constructive nature could close this mismatch.

\section{Discussion}
In this chapter we have shown how to  encode equilibria properties in the types in a way that the simulation automatically terminates when they are reached. This results then in a \textit{total} simulation, creating a \textit{correspondence between the equilibrium of a simulation and the totality of its implementation}. Of course this is only possible for models in which we know about their equilibria a priori or in which we can reason somehow that an equilibrium exists.

Models like the Sugarscape are exploratory in nature and don't have a formal ground truth where one could derive equilibria or dynamics from and validate with. In such models the researchers work with informal hypotheses which they express before running the model and then compare them informally against the resulting dynamics.

It would be of interest if dependent types could be made of use in encoding hypotheses on a more constructive and formal level directly into the implementation code. So far we have no idea how this could be done but it might be a very interesting application as it allows for a more formal and automatic testable approach to hypothesis checking.

Often, Agent-Based Models define their agents in terms of state-machines. It is easy to make wrong state-transitions e.g. in the SIR model when an infected agent should recover, nothing prevents one from making the transition back to susceptible. 

Using dependent types it might be possible to encode invariants and state-machines on the type level which can prevent such invalid transitions already at compile-time. This would be a huge benefit for ABS because of the popularity of state-machines in agent-based models.

State-Machines often have timed transitions e.g. in the SIR model, an infected agent recovers after a given time. Nothing prevents us from introducing a bug and \textit{never} doing the transition at all.

With dependent types we might be able to encode the passing of time in the types and guarantee on a type level that an infected agent has to recover after a finite number of time steps. Also can dependent types be used to express the flow of time and that it is strongly monotonic increasing?
	
In more sophisticated models agents interact in more complex ways with each other e.g. through message exchange using agent IDs to identify target agents. The existence of an agent is not guaranteed and depends on the simulation time because agents can be created or terminated at any point during simulation. 

Dependent types could be used to implement agent IDs as a proof that an agent with the given id exists \textit{at the current time-step}. This also implies that such a proof cannot be used in the future, which is prevented by the type system as it is not safe to assume that the agent will still exist in the next step. %So it is a proof of the existence of an agent: holds always only for the current time-step or for all time, depending on the model. e.g. in the SIR model no agents are removed from / added to the system thus a proof holds for all time. 


In case of an SD this will take forever to reach 0 due to the dynamics of the equations and floating point arithmetic is another difficulty. On the other hand, due to ABS discrete nature this is not an issue anymore: agents are discrete and as soon as we hit 0 infected agents - which, due to Integer representation, can be exact - an equilibrium is reached.

Note that there exists a SIR+S model, which adds a cycle back from Recovered to Susceptible - if we add this cycle in our total implementation, this should make it immediately non-total as an important criteria for totality gets violated: the source of susceptible is not finite anymore and we might run in non-stationary cycles like in a prey-predator model with Lotka-Volterra equations.


TODO: connect to property-based testing and put emphasise on the constructive nature and hypothesis testing: this is a popperian approach.

FP is the first step towards a more structural understanding of ABS implementations where dependent types should allow us to develop this even further. we leave this for further research and outline only broadly the ideas we want to follow.

Linear and Dependent Types with Idris 2: more general ideas / hints / research on how it is applicable to ABS

By definition, ABS is of constructive nature, as described by Epstein \cite{epstein_chapter_2006}: "If you can't grow it, you can't explain it" - thus an agent-based model and the simulated dynamics of it is itself a constructive proof which explain a real-world phenomenon sufficiently well. Although Epstein certainly wasn't talking about a constructive proof in any mathematical sense in this context (he was using the word \textit{generative}), dependent types \textit{might} be a perfect match and correspondence between the constructive nature of ABS and programs as proofs.

When we talk about dependently typed programs to be proofs, then we also must attribute the same to dependently typed agent-based simulations, which are then constructive proofs as well. The question is then: a constructive proof of what? It is not entirely clear \textit{what we are proving} when we are constructing dependently typed agent-based simulations. Probably the answer might be that a dependently typed agent-based simulation is then indeed a constructive proof in a mathematical sense, explaining a real-world phenomenon sufficiently well - we have closed the gap between a rather informal constructivism as mentioned above when citing Epstein who certainly didn't mean it in a constructive mathematical sense, and a formal constructivism, made possible by the use of dependent types.

\section{Discussion}

\subsection{Other Models}
TODO: mention that we have also implemented other models, which also work without time-semantics (all agents make a move at discrete time-steps and do not really rely on some notion of time). 

\subsection{Time-Semantics}
The main reason for building our pure functional ABMS approach on top of Yampa was to leverage the powerful time-semantics of Yampa which allows us to implement important concepts of ABMS:

state-chart: agents are at all time of their life-cycle in one state and can switch between multiple states using transitions 
timed transitions: transition to another state/behaviour happens at a discrete time
rate transitions: transition happens with a given rate
message transition: transition upon receiving a given message 

\subsection{Agents as Signals}
Due to the underlying nature and motivation of Functional Reactive Programming (und im speziellen) Yampa, Agents can be seen as Signals which is generated and consumed by a Signal-Function which is the behaviour of an Agent. If an Agent does not change the OUTPUT-signal is constant, if the agent changes e.g. by sending a message, changing its state,... the OUTPUT signal changes. A dead agent has no signal at all.

\subsection{Time-Sampling}
sampling rate depends on the transition times \& rates of the model. when e.g. the contact rate is 5 then the sampling dt should be below 0.2

\subsection{System Dynamics}
can emulate system dynamics due to the parallel update-strategy and continuous time-flow semantics

\subsection{Discrete Event Simulation}
DES in FrABMS? how easily can we implement server/queue systems? do they also look like a specification? potential problem: ordering of messages is not guaranteed by now

\subsection{Advantages}
advantages:
	- no side-effects within agents leads to much safer code
	- edsl for time-semantics
	- declarative style: agent-implementation looks like a model-specification
	- reasoning and verification
	- sequential and parallel
	- powerful time-semantics
	- arrowized programming is optional and only required when utilizing yampas time-semantics. if the model does not rely on time-semantics, it can use monadic-programming by building on the existing monadic functions in the EDSL which allow to run in the State-Monad which simplifies things very much
	- when to use yampas arrowized programing: time-semantics, simple state-chart agents 
	- when not using yampas facilities: in all the other cases e.g. SugarScape is such a case as it proceeds in unit time-steps and all agents act in every time-step
	- can implement System Dynamics building on Yampas facilities with total ease	
	- get replications for free without having to worry about side-effects and can even run them in parallel without headaches
	- cant mess around with time because delta-time is hidden from you (intentional design-decision by Yampa). this would be only very difficult and cumbersome to achieve in an object-oriented approach. TODO: experiment with it in Java - how could we actually implement this? I think it is impossible: may only achieve this through complicated application of patterns and inheritance but then has the problem of how to update the dt and more important how to deal with functions like integral which accumulates a value through closures and continuations. We could do this in OO by having a general base-class e.g. ContinuousTime which provides functions like updateDt and integrate, but we could only accumulate a single integral value.
	- reproducibility statically guaranteed
	- cannot mess around with dt
	- code == specification
	- rule out serious class of bugs
	- different time-sampling leads to different results e.g. in wildfire \& SIR but not in Prisoners Dilemma. why? probabilistic time-sampling?
	- reasoning about equivalence between SD and ABS implementation in the same framework
	- recursive implementations
	
	- we can statically guarantee the reproducibility of the simulation because: no side effects possible within the agents which would result in differences between same runs (e.g. file access, networking, threading), also timedeltas are fixed and do not depend on rendering performance or userinput	
	
\subsection{Disadvantages}
disadvantages:
	- performance is low
	- reasoning about performance is very difficult
	- very steep learning curve for non-functional programmers
	- learning a new EDSL
	- think ABMS different: when to use async messages, when to use sync conversations


[ ] important: increasing sampling freqzency and increasing number of steps so that the same number of simulation steps are executed should lead to same results. but it doesnt. why?
[ ] hypothesis: if time-semantics are involved then event ordering becomes relevant for emergent patterns. there are no tine semantics in heroes and cowards but in the prisoners dilemma
[ ] can we implement different types of agents interacting with each other in the same simulation ? with different behaviour funcs, digferent state? yes, also not possible in NetLogo to my knowledge. but they must have the same messages, emvironment 

[ ] Hypothesis: we can combine with FrABS agent-based simulation and system dynamics (this has been proved by example!)

% PART V: Discussion
\epigraphhead[450]{}
\part{Discussion and Conclusion}
\label{part:discussion}
\section{Discussion}

\subsection{Other Models}
TODO: mention that we have also implemented other models, which also work without time-semantics (all agents make a move at discrete time-steps and do not really rely on some notion of time). 

\subsection{Time-Semantics}
The main reason for building our pure functional ABMS approach on top of Yampa was to leverage the powerful time-semantics of Yampa which allows us to implement important concepts of ABMS:

state-chart: agents are at all time of their life-cycle in one state and can switch between multiple states using transitions 
timed transitions: transition to another state/behaviour happens at a discrete time
rate transitions: transition happens with a given rate
message transition: transition upon receiving a given message 

\subsection{Agents as Signals}
Due to the underlying nature and motivation of Functional Reactive Programming (und im speziellen) Yampa, Agents can be seen as Signals which is generated and consumed by a Signal-Function which is the behaviour of an Agent. If an Agent does not change the OUTPUT-signal is constant, if the agent changes e.g. by sending a message, changing its state,... the OUTPUT signal changes. A dead agent has no signal at all.

\subsection{Time-Sampling}
sampling rate depends on the transition times \& rates of the model. when e.g. the contact rate is 5 then the sampling dt should be below 0.2

\subsection{System Dynamics}
can emulate system dynamics due to the parallel update-strategy and continuous time-flow semantics

\subsection{Discrete Event Simulation}
DES in FrABMS? how easily can we implement server/queue systems? do they also look like a specification? potential problem: ordering of messages is not guaranteed by now

\subsection{Advantages}
advantages:
	- no side-effects within agents leads to much safer code
	- edsl for time-semantics
	- declarative style: agent-implementation looks like a model-specification
	- reasoning and verification
	- sequential and parallel
	- powerful time-semantics
	- arrowized programming is optional and only required when utilizing yampas time-semantics. if the model does not rely on time-semantics, it can use monadic-programming by building on the existing monadic functions in the EDSL which allow to run in the State-Monad which simplifies things very much
	- when to use yampas arrowized programing: time-semantics, simple state-chart agents 
	- when not using yampas facilities: in all the other cases e.g. SugarScape is such a case as it proceeds in unit time-steps and all agents act in every time-step
	- can implement System Dynamics building on Yampas facilities with total ease	
	- get replications for free without having to worry about side-effects and can even run them in parallel without headaches
	- cant mess around with time because delta-time is hidden from you (intentional design-decision by Yampa). this would be only very difficult and cumbersome to achieve in an object-oriented approach. TODO: experiment with it in Java - how could we actually implement this? I think it is impossible: may only achieve this through complicated application of patterns and inheritance but then has the problem of how to update the dt and more important how to deal with functions like integral which accumulates a value through closures and continuations. We could do this in OO by having a general base-class e.g. ContinuousTime which provides functions like updateDt and integrate, but we could only accumulate a single integral value.
	- reproducibility statically guaranteed
	- cannot mess around with dt
	- code == specification
	- rule out serious class of bugs
	- different time-sampling leads to different results e.g. in wildfire \& SIR but not in Prisoners Dilemma. why? probabilistic time-sampling?
	- reasoning about equivalence between SD and ABS implementation in the same framework
	- recursive implementations
	
	- we can statically guarantee the reproducibility of the simulation because: no side effects possible within the agents which would result in differences between same runs (e.g. file access, networking, threading), also timedeltas are fixed and do not depend on rendering performance or userinput	
	
\subsection{Disadvantages}
disadvantages:
	- performance is low
	- reasoning about performance is very difficult
	- very steep learning curve for non-functional programmers
	- learning a new EDSL
	- think ABMS different: when to use async messages, when to use sync conversations


[ ] important: increasing sampling freqzency and increasing number of steps so that the same number of simulation steps are executed should lead to same results. but it doesnt. why?
[ ] hypothesis: if time-semantics are involved then event ordering becomes relevant for emergent patterns. there are no tine semantics in heroes and cowards but in the prisoners dilemma
[ ] can we implement different types of agents interacting with each other in the same simulation ? with different behaviour funcs, digferent state? yes, also not possible in NetLogo to my knowledge. but they must have the same messages, emvironment 

[ ] Hypothesis: we can combine with FrABS agent-based simulation and system dynamics (this has been proved by example!)
\section{Conclusions}
\label{sec:conclusions}

Our approach is radically different from traditional approaches in the ABS community. First it builds on the already quite powerful FRP paradigm. Second, due to our continuous time approach, it forces one to think properly of time-semantics of the model and how small $\Delta t$ should be. Third it requires to think about agent interactions in a new way instead of being just method-calls.

Because no part of the simulation runs in the IO Monad and we do not use unsafePerformIO we can rule out a serious class of bugs caused by implicit data-dependencies and side-effects which can occur in traditional imperative implementations.

Also we can statically guarantee the reproducibility of the simulation, which means that repeated runs with the same initial conditions are guaranteed to result in the same dynamics. Although we allow side-effects within agents, we restrict them to only the Random and State Monad in a controlled, deterministic way and never use the IO Monad which guarantees the absence of non-deterministic side effects within the agents and other parts of the simulation.

Determinism is also ensured by fixing the $\Delta t$ and not making it dependent on the performance of e.g. a rendering-loop or other system-dependent sources of non-determinism as described by \cite{perez_testing_2017}. Also by using FRP we gain all the benefits from it and can use research on testing, debugging and exploring FRP systems \cite{perez_testing_2017, perez_back_2017}.

\subsection*{Issues}
Currently, the performance of the system is not comparable to imperative implementations but our research was not focusing on this aspect. We leave the investigation and optimization of the performance aspect of our approach for further research.

Despite the strengths and benefits we get by leveraging on FRP, there are errors that are not raised at compile time, e.g. we can still have infinite loops and run-time errors. This was for example investigated in \cite{sculthorpe_safe_2009} where the authors use dependent types to avoid some run-time errors in FRP. We suggest that one could go further and develop a domain specific type system for FRP that makes the FRP based ABS more predictable and that would support further mathematical analysis of its properties. Furthermore, moving to dependent types would pose a unique benefit over the traditional object-oriented approach and should allow us to express and guarantee even more properties at compile time. We leave this for further research.

In our pure functional approach, agent identity is not as clear as in traditional object-oriented programming, where an agent can be hidden behind a polymorphic interface which is much more abstract than in our approach. Also the identity of an agent is much clearer in object-oriented programming due to the concept of object-identity and the encapsulation of data and methods.

We can conclude that the main difficulty of a pure functional approach evolves around the communication and interaction between agents, which is a direct consequence of the issue with agent identity. Agent interaction is straight-forward in object-oriented programming, where it is achieved using method-calls mutating the internal state of the agent, but that comes at the cost of a new class of bugs due to implicit data flow. In pure functional programming these data flows are explicit but our current approach of feeding back the states of all agents as inputs is not very general and we have added further mechanisms of agent interaction which we had to omit due to lack of space.

% BIB
\renewcommand\bibname{References}
\bibliographystyle{acm} %apalike
\bibliography{bibliography}

% PART VI: Appendix
%\epigraphhead[450]{}
%\part{Appendices}
\begin{appendices}

TODO: add full code of SIR implementation

\chapter{Validating Sugarscape in Haskell}
\label{app:validating_sugarscape}

In this chapter we look at how property-based testing can be made of use to verify the \textit{exploratory} Sugarscape model \cite{epstein_growing_1996} as introduced in Chapter \ref{sec:sugarscape}. Whereas in the chapters on testing the explanatory SIR model we had an analytical solution, the fundamental difference in the exploratory Sugarscape model is that none such analytical solutions exist. This raises the question, which properties we can actually test in such a model.

The answer lies in the very nature of exploratory models, they exist to explore and understand phenomena of the real world. Researchers come up with a model to explain the phenomena and (hopefully) with a few questions and \textit{hypotheses} about the emergent properties. The actual simulation is then used to test and refine the hypotheses. Indeed, descriptions, assumptions and hypotheses of varying formal degree abound in the Sugarscape model. Examples are: \textit{the carrying capacity becomes stable after 100 steps; when agents trade with each other, after 1000 steps the standard deviation of trading prices is less than 0.05; when there are cultures, after 2700 steps either one culture dominates the other or both are equally present}. 

We show how to use property-based testing to formalise and check such hypotheses. For this purpose we undertook a full \textit{verification} of our \href{https://github.com/thalerjonathan/haskell-sugarscape}{implementation}~\cite{thaler_sugarscape_repository} from Chapter \ref{sec:sugarscape}. We validated it against the book \cite{epstein_growing_1996} and a NetLogo implementation \cite{weaver_replicating_2009}  \footnote{Lending didn't properly work in their NetLogo code and that they didn't implement Combat.}.

\section{Property-based hypothesis testing}
The property we test for is whether \textit{the emergent property / hypothesis under test is stable under replicated runs} or not. To put it more technical, we use QuickCheck to run multiple replications with the same configuration but with different random-number streams and require that all tests pass. During the verification process we have derived and implemented property tests for the following hypotheses:

\begin{enumerate}
	\item Disease dynamics where all agents recover - when disease are turned on, if the number of initial diseases is 10, then the population is  able to rid itself completely from all disease within 100 ticks. 
	
	\item Disease dynamics where a minority recovers - when disease are turned on, if the number of initial diseases is 25, the population is not able to rid itself completely from all diseases within 1,000 ticks.
	
	\item Trading dynamics - when trading is enabled, the trading prices stabilise after 1,000 ticks with the standard deviation of the prices having dropped below 0.05.
	
	\item Cultural dynamics - when having two cultures, red and blue, after 2,700 ticks, either the red or the blue culture dominates or both are equally strong. If they dominate they make up 95\% of all agents, if they are equally strong they are both within 45\% - 55\%.
	
	\item Inheritance Gini coefficient - when agents reproduce and can die of age then inheritance of their wealth leads to an unequal wealth distribution measured using the Gini Coefficient \textit{averaging} at 0.7.
	
	\item Carrying capacity - when agents don't mate nor can die from age, due to the environment, there is an \textit{average} maximum carrying capacity of agents the environment can sustain. The capacity should be reached after 100 ticks and should be stable from then on.
		
	\item Terracing - when resources regrow immediately, after a few steps the simulation becomes static. Agents will stay on their terraces and will not move any more because they have found the best spot due to their behaviour. About 45\% will be on terraces and 95\% - 100\% are static, not moving any more.
\end{enumerate}

The hypotheses and their validation is described more in-depth in the section \ref{sec:hypotheses_testcases} below.

\subsection{Implementation}
To start with, we implement a custom data generator to produce output from a Sugarscape simulation. The generator takes the number of ticks and the scenario with which to run the simulation and returns a list of outputs, one for each tick.

\begin{HaskellCode}
sugarscapeUntil :: Int                -- ^ Number of ticks to run
                -> SugarScapeScenario -- ^ Scenario to run
                -> Gen [SimStepOut]   -- ^ Output of each step
sugarscapeUntil ticks params = do
  -- create a random-number generator
  g <- genStdGen
  -- initialise the simulation state with the given random-number generator
  -- and the scenario
  let simState = initSimulationRng g params
  -- run the simulation with the given state for number of ticks
  return (simulateUntil ticks simState)
\end{HaskellCode}

Using this generator, we can very conveniently produce Sugarscape data within a QuickCheck \texttt{Property}. Depending on the problem, we can generate only a single run or multiple replications, in case the hypothesis is assuming \textit{averages}. To see its use, we show the implementation of the \textit{Disease Dynamics (1)} hypothesis.

\begin{HaskellCode}
prop_disease_allrecover :: Property
prop_disease_allrecover = property (do
  -- after 100 ticks...
  let ticks = 100
  -- ... given Animation V-1 parameter configuration ...
  let params = mkParamsAnimationV_1
  -- ... from 1 sugarscape simulation ...
  aos <- last <*> (sugarscapeUntil ticks params)
  -- ... counting all infected agents ...
  let infected = length (filter (==False)) map (null . sugObsDiseases . snd) aos
  -- ... should result in all agents to be recovered
  return (cover 100 (infected == 0) "Diseases all recover" True)
\end{HaskellCode}

From the implementation it becomes clear, that this hypothesis states that the property has to hold \textit{for all} replications. The \textit{Inheritance Gini Coefficient (5)} hypothesis on the other hand assumes that the Gini Coefficient \textit{averages} at 0.7. We cannot average over replicated runs of the same property thus we generate multiple replications of the Sugarscape data within the property and employ a two-sided t-test with a 95\% confidence to test the hypothesis:

\begin{HaskellCode}
prop_gini :: Int      -- ^ Number of replications
          -> Double   -- ^ Confidence of the t-test
          -> Property
prop_gini repls confidence = property (do
  -- after 1000 ticks...
  let ticks = 1000
  -- ... the gini coefficient should average at 0.7 ...
  let expGini = 0.7
  -- ... given the Figure III-7 parameter configuration ...
  let params = mkParamsFigureIII_7
  -- ... from 100 replications ... 
  gini <- vectorOf repls (genGiniCoeff ticks params)
  -- on a two-tailed t-test with given confidence
  return (tTestSamples TwoTail expGini (1 - confidence) gini)
\end{HaskellCode}

%genGiniCoeff :: Int -> SugarScapeScenario -> Gen Double
%genGiniCoeff ticks params = do
%  -- generate sugarscape data
%  aos <- sugarscapeUntil ticks params
%  -- extract wealth of the agents in the last step
%  let agentWealths = map (sugObsSugLvl . snd) (last aos)
%  -- compute gini coefficient and return it
%  return (giniCoeff agentWealths)

\subsection{Running the tests}
As already pointed out in Part \ref{ch:property}, QuickCheck by default runs up to 100 test cases of a property and if all evaluate to \texttt{True} the property test succeeds. On the other hand, QuickCheck will stop at the first test case which evaluates to \texttt{False} and marks the whole property test as failed, no matter how many test cases got through already. For this reason we have used \texttt{cover} with an expected percentage of 100, meaning that we expect all tests to fall into the coverage class. This allows us to emulate failure with QuickCheck reporting the actual percentage of passed test cases.

Due to the duration even 1,000 ticks can take to compute, to get a first estimate of our hypotheses tests within reasonable time, we reduce the number of maximum successful replications required to 10 and when doing t-tests 10 replications are run there as well. 

\begin{verbatim}
SugarScape Tests
  Disease Dynamics All Recover:      OK (29.25s)
    +++ OK, passed 10 tests (100% Diseases all recover).
    
  Disease Dynamics Minority Recover: OK (536.00s)
    +++ OK, passed 10 tests (100% Diseases no recover).
    
  Trading Dynamics:                  OK (149.33s)
    +++ OK, passed 10 tests (70% Prices std less than 5.0e-2).
    Only 70% Prices std less than 5.0e-2, but expected 100%
    
  Cultural Dynamics:                 OK (996.84s)
    +++ OK, passed 10 tests (50% Cultures dominate or equal).
    Only 50% Cultures dominate or equal, but expected 100%
    
  Carrying Capacity:   OK (988.20s)
    +++ OK, passed 10 tests (90% Carrying capacity averages at 204.0).    
    Only 90% Carrying capacity averages at 204.0, but expected 100%
    
  Terracing:           OK (280.59s)
    +++ OK, passed 10 tests (80% Terracing is happening).
    Only 80% Terracing is happening, but expected 100%
    
  Inheritance Gini:    OK (7232.59s)
    +++ OK, passed 0 tests (0% Gini coefficient averages at 0.7).
    Only 0% Gini coefficient averages at 0.7, but expected 100%
\end{verbatim}

%\begin{enumerate}
%	\item Disease Dynamics all recover: \textit{+++ OK, passed 10 tests.}
%
%	\item Disease Dynamics minority recover: \textit{+++ OK, passed 10 tests.}
%		
%	\item Trading Dynamics: \textit{+++ OK, passed 10 tests; 2 failed (16\%).} (In total 12 tests (replications) were run, out of which 2 failed, which is a 16\% failure rate.)
%	
%	\item Cultural Dynamics: \textit{+++ OK, passed 10 tests; 3 failed (23\%).}
%
%	\item Inheritance Gini Coefficient: \textit{*** Failed! Passed only 0 tests; 10 failed (100\%) tests.}
%
%	\item Carrying Capacity: \textit{+++ OK, passed 10 tests; 2 failed (16\%).}
%
%	\item Terracing: \textit{+++ OK, passed 10 tests; 2 failed (16\%).}
%\end{enumerate}

How to deal with the failure of hypotheses is obviously highly model specific. A first approach is to increase the number of replications to run to 100 to get a more robust estimate of the failure rate. If the failure rate stays within reasonable ranges then one can arguably assume that the hypothesis is valid for sufficiently enough cases. On the other hand, if the failure rate escalates, then it is reasonable to deem the hypothesis invalid and refine it or even abandon it altogether.

With the exception of the Gini coefficient, we accept the failure rate of the hypotheses we presented here and deem them sufficiently valid for the task at hand. In case of the Gini coefficient, none of the replication was successful, which makes it obvious that it does \textit{not} average at 0.7. Thus the hypothesis as stated in the book does not hold and is invalid. One way to deal with it would be to simply delete it. Another, more constructive approach, is to keep it but require all replications to fail by marking it with \texttt{expectFailure} instead of \texttt{property}. In this way an invalid hypothesis is marked explicitly and acts as documentation and also as regression test.

\section{Hypotheses and test cases}
\label{sec:hypotheses_testcases}

In this section we briefly describe the process of validating our Sugarscape implementation against the specification of the Sugarscape book \cite{epstein_growing_1996} and the work of \cite{weaver_replicating_2009}.

\subsection{Terracing}
Our implementation reproduces the terracing phenomenon as described in the book and as can be seen in the NetLogo implementation as well. We implemented a property test in which we measure the closeness of agents to the ridge: counting the number of same-level sugars cells around them and if there is at least one lower then they are at the edge. If a certain percentage is at the edge then we accept terracing. The question is just how much, which we estimated from tests and resulted in 45\%. Also, in the terracing animation the agents actually never move which is because sugar immediately grows back thus there is no incentive for an agent to actually move after it has moved to the nearest largest cite in can see. Therefore we test that the coordinates of the agents after 50 steps are the same for the remaining steps.

\subsection{Carrying capacity}
Our simulation reached a steady state (variance $<$ 4 after 100 steps) with a mean around ~182. Epstein reported a carrying capacity of 224 (page 30) and the NetLogo implementations' \cite{weaver_replicating_2009} carrying capacity fluctuates around 205 which both are significantly higher than ours. Something was definitely wrong - the carrying capacity has to be around 200 (we trust in this case the NetLogo implementation and deem 224 an outlier).

After inspection of the NetLogo model we realised that we implicitly assumed that the metabolism range is \textit{continuously} uniformly randomized between 1 and 4 but this seemed not what the original authors intended: in the NetLogo model there were a few agents surviving on sugar level 1 which was never the case in ours as the probability of drawing a metabolism of exactly 1 is practically zero when drawing from a continuous range. We thus changed our implementation to draw a discrete value as the metabolism. %Note that this actually makes sense as massive floating-point number calculations were quite expensive in the mid 90s (e.g. computer games ran still on CPU only and exploited various  clever tricks to avoid the need of floating point calculations whenever possible) when SugarScape was implemented which might have been a reason for the authors to assume it implicitly.

This partly solved the problem, the carrying capacity was now around 204 which is much better than 182 but still a far cry from 210 or even 224. After adjusting the order in which agents apply the Sugarscape rules, by looking at the code of the NetLogo implementation, we arrived at a comparable carrying capacity of the NetLogo implementation: agents first make their move and harvest sugar and only after this the agents metabolism is applied (and ageing in subsequent experiments).

For regression tests we implemented a property test which tests that the carrying capacity of 100 simulation runs lies within a 95\% confidence interval of a 210 mean. These values are quite reasonable to assume, when looking at the NetLogo implementation - again we deem the reported carrying capacity of 224 in the book to be an outlier / part of other details we don't know.

One lesson learned is that even such seemingly minor things like continuous vs. discrete or order of actions an agent makes, can have substantial impact on the dynamics of a simulation.

\subsection{Wealth distribution}
By visual comparison we validated that the wealth distribution (page 32-37) becomes strongly skewed with a histogram showing a fat tail, power-law distribution where very few agents are very rich and most of the agents are quite poor. We compute the skewness and kurtosis of the distribution which is around a skewness of 1.5, clearly indicating a right skewed distribution and a kurtosis which is around 2.0 which clearly indicates the 1st histogram of Animation II-3 on page 34. Also we compute the Gini coefficient and it varies between 0.47 and 0.5 - this is accordance with Animation II-4 on page 38 which shows a gini-coefficient which stabilises around 0.5 after. 
We implemented a regression-test testing skewness, kurtosis and gini coefficients of 100 runs to be within a 95\% confidence interval of a two-sided t-test using an expected skewness of 1.5, kurtosis of 2.0 and gini coefficient of 0.48.

\subsection{Migration}
With the information provided by \cite{weaver_replicating_2009} we could replicate the waves as visible in the NetLogo implementation as well. Also we propose that a vision of 10 is not enough yet and shall be increased to 15 which makes the waves very prominent and keeps them up for much longer - agent waves are travelling back and forth between both Sugarscape peaks. We have not implemented a regression test for this property as we couldn't come up with a reasonable straightforward approach to implement it.

\subsection{Pollution and diffusion}
With the information provided by \cite{weaver_replicating_2009} we could replicate the pollution behaviour as visible in the NetLogo implementation as well. We have not implemented a regression test for this property as we couldn't come up with a reasonable straightforward approach to implement it.

%Note that we spent quite a lot of time of getting this and the terracing properties right because they form the very basics of the other ones which follow so we had to be sure that those were correct otherwise validating would have been much more difficult.

%\section{Order of Rules}
%order in which rules are applied is not specified and might have an impact on dynamics e.g. when does the agent mate with others: is it after it has harvested but before metabolism kicks in?

\subsection{Mating}
We could not replicate Figure III-1 - our dynamics first raised and then plunged to about 100 agents and go then on to recover and fluctuate around 300. This findings are in accordance with \cite{weaver_replicating_2009}, where they report similar findings - also when running their NetLogo code we find the dynamics to be qualitatively the same.

Also at first we weren't able to reproduce the cycles of population sizes. Then we realised that our agent behaviour was not correct: agents which died from age or metabolism could still engage in mating before actually dying - fixing this to the behaviour, that agents which died from age or metabolism will not engage in mating solved that and produces the same swings as in \cite{weaver_replicating_2009}. Although our bug might be obvious, the lack of specification of the order of the application of the rules is an issue in the SugarScape book.

\subsection{Inheritance}
We couldn't replicate the findings of the Sugarscape book regarding the Gini coefficient with inheritance. The authors report that they reach a gini coefficient of 0.7 and above in Animation III-4. Our Gini coefficient fluctuated around 0.35. Compared to the same configuration but without inheritance (Animation III-1) which reached a Gini coefficient of about 0.21, this is indeed a substantial increase - also with inheritance we reach a larger number of agents of around 1,000 as compared to around 300 without inheritance.
The Sugarscape book compares this to chapter II, Animation II-4 for which they report a Gini coefficient of around 0.5 which we could reproduce as well. The question remains, why it is lower (lower inequality) with inheritance?

The baseline is that this shows that inheritance indeed has an influence on the inequality in a population. Thus we deemed that our results are qualitatively the same as the make the same point. Still there must be some mechanisms going on behind the scenes which are unspecified in the original Sugarscape.

\subsection{Cultural dynamics}
We could replicate the cultural dynamics of AnimationIII-6 / Figure III-8: after 2700 steps either one culture (red / blue) dominates both hills or each hill is dominated by a different ulture. We wrote a test for it in which we run the simulation for 2.700 steps and then check if either culture dominates with a ratio of 95\% or if they are equal dominant with 45\%. Because always a few agents stay stationary on sugarlevel 1 (they have a metabolism of 1 and cant see far enough to move towards the hills, thus stay always on same spot because no improvement and grow back to 1 after 1 step), there are a few agents which never participate in the cultural process and thus no complete convergence can happen. This is accordance with \cite{weaver_replicating_2009}.

\subsection{Combat}
Unfortunately \cite{weaver_replicating_2009} didn't implement combat, so we couldn't compare it to their dynamics. Also, we weren't able to replicate the dynamics found in the Sugarscape book: the two tribes always formed a clear battlefront where some agents engage in combat, for example when one single agent strays too far from its tribe and comes into vision of the other tribe it will be killed almost always immediately. This is because crossing the sugar valley is costly: this agent wont harvest as much as the agents staying on their hill thus will be less wealthy and thus easier killed off. Also retaliation is not possible without any of its own tribe anywhere near.

We didn't see a single run where an agent of an opposite tribe "invaded" the other tribes hill and ran havoc killing off the entire tribe. We don't see how this can happen: the two tribes start in opposite corners and quickly occupy the respective sugar hills. So both tribes are acting on average the same and also because of the number of agents no single agent can gather extreme amounts of wealth - the wealth should rise in both tribes equally on average. Thus it is very unlikely that a super-wealthy agent emerges, which makes the transition to the other side and starts killing off agents at large. First: a super-wealthy agent is unlikely to emerge, second making the transition to the other side is costly and also low probability, third the other tribe is quite wealthy as well having harvested for the same time the sugar hill, thus it might be that the agent might kill a few but the closer it gets to the center of the tribe the less like is a kill due to retaliation avoidance - the agent will simply get killed by others.

Also it is unclear in case of AnimationIII-11 if the R rule also applies to agents which get killed in combat. Nothing in the book makes this clear and we left it untouched so that agents who only die from age (original R rule) are replaced. This will lead to a near extinction of the whole population quite quickly as agents kill each other off until 1 single agent is left which will never get killed in combat because there are no other agents who could kill it - instead it will enter an infinite die and  reborn cycle thanks to the R rule.

\subsection{Spice}
The book specifies for AnimationIV-1 a vision between 1-10 and a metabolism between 1-5. The last one seems to be quite strange because the maximum sugar / spice an agent can find is 4 which means that agents with metabolism of either 5 will die no matter what they do because the can never harvest enough to satisfy their metabolism. When running our implementation with this configuration the number of agents quickly drops from 400 to 105 and continues to slowly degrade below 90 after around 1000 steps.
The implementation of \cite{weaver_replicating_2009} used a slightly different configuration for AnimationIV-1, where they set vision to 1-6 and metabolism to 1-4. Their dynamics stabilise to 97 agents after around 500+ steps. When we use the same configuration as theirs, we produce the same dynamics.
Also it is worth nothing that our visual output is strikingly similar to both the book AnimationIV-1 and \cite{weaver_replicating_2009}.

\subsection{Trading}
For trading we had a look at the NetLogo implementation of \cite{weaver_replicating_2009}: there an agent engages in trading with its neighbours \textit{over multiple rounds} until either MRSs cross over or no trade has happened anymore. Because \cite{weaver_replicating_2009} were able to exactly replicate the dynamics of the trading time series we assume that their implementation is correct. We think that the fact that an agent interact with its neighbours over multiple rounds is made not very clear in the book. The only hint is found on page 102: \textit{"This process is repeated until no further gains from trades are possible."} which is not very clear and does not specify exactly what is going on: does the agent engage with all neighbours again? is the ordering random? Another hint is found on page 105 where trading is to be stopped after MRS crossover to prevent an infinite loop. Unfortunately this is missing in the Agent trade rule T on page 105. Additional information on this is found in footnote 23 on page 107. Further on page 107: \textit{"If exchange of the commodities will not cause the agents' MRSs to cross over then the transaction occurs, the agents recompute their MRSs, and bargaining begins anew."}. This is probably the clearest hint that trading could occur over multiple rounds.

We still managed to exactly replicate the trading dynamics as shown in the book in Figure IV-3, Figure IV-4 and Figure IV-5. The book is also pretty specific on the dynamics of the trading prices standard deviation: on page 109 the authors specify that at t=1000 the standard deviation will have always fallen below 0.05 (Figure IV-5), thus we implemented a property test which tests for exactly that property. Unfortunately we didn't reach the same magnitude of the trading volume where ours is much lower around 50 but it is equally erratic, so we attribute these differences to other missing specifications or different measurements because the price dynamics match that well already so we can safely assume that our trading implementation is correct.

According to the book, Carrying Capacity (Animation II-2) is increased by Trade (page 111/112). To check this it is important to compare it not against AnimationII-2 but a variation of the configuration for it where spice is enabled, otherwise the results are not comparable because carrying capacity changes substantially when spice is on the environment and trade turned off. We could replicate the findings of the book: the carrying capacity increases slightly when trading is turned on. Also does the average vision decrease and the average metabolism increase. This makes perfect sense: trading allows genetically weaker agents to survive which results in a slightly higher carrying capacity but shows a weaker genetic performance of the population.

According to the book, increasing the agent vision leads to a faster convergence towards the (near) equilibrium price (page 117/118/119, Figure IV-8 and Figure IV-9). We could replicate this behaviour as well.

According to the book, when enabling R rule and giving agents a finite life span between 60 and 100 this will lead to price dispersion: the trading prices will not converge around the equilibrium and the standard deviation will fluctuate wildly (page 120, Figure IV-10 and Figure IV-11). We could replicate this behaviour as well.

The Gini coefficient should be higher when trading is enabled (page 122, Figure IV-13) - We could replicate this behaviour.

Finite lives with sexual reproduction lead to prices which don't converge (page 123, Figure IV-14). We could reproduce this as well but it was important to set the parameters to reasonable values: increasing number of agents from 200 to 400, metabolism to 1-4 and vision to 1-6, most important the initial endowments back to 5-25 (both sugar and spice) otherwise hardly any mating would happen because the agents need too much wealth to engage (only fertile when have gathered more than initial endowment). What was kind of interesting is that in this scenario the trading volume of sugar is substantially higher than the spice volume - about 3 times as high. 

From this part, we didn't implement: Effect of Culturally Varying Preferences, page 124 - 126, Externalities and Price Disequilibrium: The effect of Pollution, page 126 - 118, On The Evolution of Foresight page 129 / 130. 

%\section{Lending (Credit)}
%Not really much information to validate was available and the \cite{weaver_replicating_2009} implementation ran into an exception so there was not much to validate against. What was unexpected was that this was the most complex behaviour to implement, with lots of subtle details to take care of (spice on/off, inheritance,...).
%Note that we implemented lending of sugar and spice, although it looks from the book (Animation IV-5) that they only implemented it for sugar.

\subsection{Diseases}
We were able to exactly replicate the behaviour of Animation V-1 and Animation V-2: in the first case the population rids itself of all diseases (maximum 10) which happens pretty quickly, in less than 100 ticks. In the second case the population fails to do so because of the much larger number of diseases (25) in circulation. We used the same parameters as in the book. 
The authors of \cite{weaver_replicating_2009} could only replicate the first animation exactly and the second was only deemed "good". Their implementation differs slightly from ours: In their case a disease can be passed to an agent who is immune to it - this is not possible in ours. In their case if an agent has already the disease, the transmitting agent selects a new disease, the other agent has not yet - this is not the case in our implementation and we think this is unreasonable to follow: it would require too much information and is also unrealistic.
We wrote regression tests which check for animation V-1 that after 100 ticks there are no more infected agents and for animation V-2 that after 1000 ticks there are still infected agents left and they dominate: there are more infected than recovered agents.

\section{Discussion}
In this appendix we showed how to use QuickCheck to formalise and check hypotheses about an \textit{exploratory} agent-based model, in which no ground truth exists. Due to ABS stochastic nature in general it became obvious that to get a good measure of a hypotheses validity we need to emulate failure using the \texttt{cover} function of QuickCheck. This allowed us to show that the hypotheses we have presented are sufficiently valid for the task at hand and can indeed be used for expressing and formalising emergent properties of the model and also as regression tests within a TDD cycle.

%What is particularly powerful is that one has complete control and insight over the changed state before and after e.g. a function was called on an agent: thus it is very easy to check if the function just tested has changed the agent-state itself or the environment: the new environment is returned after running the agent and can be checked for equality of the initial one - if the environments are not the same, one simply lets the test fail. This behaviour is very hard to emulate in OOP because one can not exclude side-effect at compile time, which means that some implicit data-change might slip away unnoticed. In FP we get this for free.

\end{appendices}

\end{document}