\chapter{Background}
\label{ch:background}

\section{Related research and literature}
\label{sec:literature}

The amount of research on using pure functional programming with Haskell in the field of ABS has been moderate so far. Most of the papers are related to the field of Multi Agent Systems (MAS) and look into how agents can be specified using the belief-desire-intention paradigm \cite{de_jong_suitability_2014,sulzmann_specifying_2007,jankovic_functional_2007}.

A multi-method simulation library in Haskell called \textit{Aivika 3} is described in the technical report \cite{sorokin_aivika_2015}. It supports implementing Discrete Event Simulations (DES), System Dynamics and comes with basic features for event-driven ABS which is realised using DES under the hood. Further it provides functionality for adding GPSS to models and supports parallel and distributed simulations. It runs within the IO effect type for realising parallel and distributed simulation but also discusses generalising their approach to avoid running in IO.

In his master thesis \cite{bezirgiannis_improving_2013} the author investigates Haskells' parallel and concurrency features to implement (amongst others) \textit{HLogo}, a Haskell clone of the NetLogo \cite{wilensky_introduction_2015} simulation package, focusing on using STM for a limited form of agent-interactions. \textit{HLogo} is basically a re-implementation of NetLogos API in Haskell where agents run within an unrestricted context (known as \textit{IO}) and thus can also make use of STM functionality. The benchmarks show that this approach does indeed result in a speed-up especially under larger agent-populations. The authors' thesis can be seen as one of the first works on ABS using Haskell. Despite the concurrency and parallel aspect our work share, our approach is rather different: we avoid IO within the agents under all costs and explore the use of STM more on a conceptual level rather than implementing a ABS library and compare our case-studies with lock-based and imperative implementations.

There exists some research \cite{di_stefano_using_2005, varela_modelling_2004, sher_agent-based_2013} using the functional programming language Erlang \cite{armstrong_erlang_2010} to implement concurrent ABS. The language is inspired by the actor model \cite{agha_actors:_1986} and was created in 1986 by Joe Armstrong for Eriksson for developing distributed high reliability software in telecommunications. The actor model can be seen as quite influential to the development of the concept of agents in ABS, which borrowed it from Multi Agent Systems \cite{wooldridge_introduction_2009}. It emphasises message-passing concurrency with share-nothing semantics (no shared state between agents), which maps nicely to functional programming concepts. The mentioned papers investigate how the actor model can be used to close the conceptual gap between agent-specifications, which focus on message-passing and their implementation. Further they show that using this kind of concurrency allows to overcome some problems of low level concurrent programming as well.
Also \cite{bezirgiannis_improving_2013} ported NetLogos API to Erlang mapping agents to concurrently running processes, which interact with each other by message-passing. With some restrictions on the agent-interactions this model worked, which shows that using concurrent message-passing for parallel ABS is at least \textit{conceptually} feasible. Despite the natural mapping of ABS concepts to such an actor language, it leads to simulations, which despite same initial starting conditions, might result in different dynamics each time due to concurrency.

The work \cite{lysenko_framework_2008} discusses a framework, which allows to map Agent-Based Simulations to Graphics Processing Units (GPU). Amongst others they use the SugarScape model \cite{epstein_growing_1996} and scale it up to millions of agents on very large environment grids. They reported an impressive speed-up of a factor of 9,000. Although their work is conceptually very different we can draw inspiration from their work in terms of performance measurement and comparison of the SugarScape model.

% THIS IS MY OWN REASEARCH, DON'T CITE IT HERE
%In \cite{thaler_pure_2019} the authors showed how to implement a spatial SIR model in pure Haskell using Functional Reactive Programming \cite{hudak_arrows_2003}. They report quite low performance but mention that STM may be a way to considerably speed up the simulation. We follow their approach in implementation technique, using Functional Reactive Programming and Monadic Stream Functions \cite{perez_functional_2016} (we don't go into implementation details as it is out of the scope of this paper) and use the spatial SIR model as the first case-study.

Using functional programming for DES was discussed in \cite{jankovic_functional_2007} where the authors explicitly mention the paradigm of Functional Reactive Programming (FRP) to be very suitable to DES.

A domain-specific language for developing functional reactive agent-based simulations was presented in \cite{schneider_towards_2012,vendrov_frabjous_2014}. This language called FRABJOUS is human readable and easily understandable by domain-experts. It is not directly implemented in FRP/Haskell but is compiled to Haskell code which they claim is also readable. This supports that FRP is a suitable approach to implement ABS in Haskell. Unfortunately, the authors do not discuss their mapping of ABS to FRP on a technical level, which would be of most interest to functional programmers.

Object-oriented programming and simulation have a long history together as the former one emereged out of Simula 67 \cite{dahl_birth_2002} which was created for simulation purposes. Simula 67 already supported Discrete Event Simulation and was highly influential for today's object-oriented languages. Although the language was important and influential, in our research we look into different approaches, orthogonal to the existing object-oriented concepts.

Lustre is a formally defined, declarative and synchronous dataflow programming language for programming reactive systems \cite{halbwachs_synchronous_1991}. While it has solved some issues related to implementing ABS in Haskell it still lacks a few important features necessary for ABS. We don't see any way of implementing an environment in Lustre as we do in Chapters  \ref{ch:timedriven} and \ref{ch:eventdriven}. Also the language seems not to come with stochastic functions, which are but the very building blocks of ABS. Finally, Lustre does only support static networks, which is clearly a drawback in ABS in general where agents can be created and terminated dynamically during simulation.

The authors of \cite{botta_time_2010} discuss the problem of advancing time in message-driven agent-based socio-economic models. They formulate purely functional definitions for agents and their interactions through messages. Our architecture for synchronous agent-interaction as discussed in Chapter TODO was not directly inspired by their work but has some similarities: the use of messages and the problem of when to advance time in models with arbitrary number synchronised agent-interactions.

The authors of \cite{botta_functional_2011} are using functional programming as a specification for an agent-based model of exchange markets but leave the implementation for further research where they claim that it requires dependent types. This paper is the closest usage of dependent types in agent-based simulation we could find in the existing literature and to our best knowledge there exists no work on general concepts of implementing pure functional agent-based simulations with dependent types. As a remedy to having no related work to build on, we looked into works which apply dependent types to solve real world problems from which we then can draw inspiration from.

In his talk \cite{sweeney_next_2006}, Tim Sweeney CTO of Epic Games discussed programming languages in the development of game engines and scripting of game logic. Although the fields of games and ABS seem to be very different, Gregory \cite{gregory_game_2018} defines computer-games as \textit{"[..] soft real-time interactive agent-based computer simulations"} (p. 9) and in the end they have also very important similarities: both are simulations which perform numerical computations and update objects in a loop either concurrently or sequential. In games these objects are called \textit{game-objects} and in ABS they are called \textit{agents} but they are conceptually the same thing.  Sweeney reports that reliability suffers from dynamic failure in languages like C++ e.g. random memory overwrites, memory leaks, accessing arrays out-of-bounds, dereferencing null pointers, integer overflow, accessing uninitialized variables. He reports that 50\% of all bugs in the Game Engine Middleware Unreal can be traced back to such problems and presents dependent types as a potential rescue to those problems. The two main points Sweeney made were that dependent types could solve most of the run-time failures and that parallelism is the future for performance improvement in games. He distinguishes between pure functional algorithms which can be parallelised easily in a pure functional language and updating game-objects concurrently using software transactional memory (STM).

\section{Agent-Based Simulation}
\label{sec:method_abs}

This thesis understands ABS as a method and methodology to model and simulate a system, where the global behaviour may be unknown but the behaviour and interactions of the parts making up the system is known. Those parts, called agents, are modelled and simulated, out of which then the aggregate global behaviour of the whole system emerges. So, the central aspect of ABS is the concept of an agent, a metaphor for a proactive unit, situated in an environment which is able to spawn new agents and interacting with other agents in some neighbourhood by the exchange of messages \cite{macal_everything_2016, odell_objects_2002, siebers_introduction_2008, wooldridge_introduction_2009}. In summary, this thesis informally assumes the following about agents:

\begin{itemize}
	\item They are uniquely addressable entities with an internal state over which they have full, exclusive control.
	\item They are proactive, which means they can initiate actions on their own. For example they can change their internal state, send messages, create new agents or terminate themselves.
	\item They are situated in an environment and can interact with it.
	\item They can interact with other agents situated in the same environment by means of messaging.
\end{itemize} 

Epstein \cite{epstein_generative_2012} identifies ABS to be particularly applicable for analysing \textit{"spatially distributed systems of heterogeneous autonomous actors with bounded information and computing capacity"}. Technically, ABS exhibits the following properties:

\begin{itemize}
	\item Linearity and non-linearity - actions of agents can lead to non-linear behaviour of the system.
	\item Time - agents act over time, which is also the source of their proactivity.
	\item State - agents encapsulate state, which can be accessed and changed during the simulation.
	\item Feedback loop - because agents act continuously and their actions influence each other and themselves in the future of subsequent time steps, feedback loops permeate every ABS. 
	\item Heterogeneity - agents can have properties (age, height, sex, etc.) where the actual values can vary arbitrarily between individuals.
	\item Interactions - agents can be modelled after interactions with an environment and other agents.
	\item Spatiality and networks - agents can be situated within arbitrary environments, like spatial environments (discrete 2D, continuous 3D, etc.) or complex networks.
\end{itemize}

There is no commonly agreed technical definition of ABS but the field draws inspiration from the closely related field of Multi-Agent Systems \cite{weiss_multiagent_2013,wooldridge_introduction_2009}. It is important to understand that Multi-Agent Systems and ABS are two different fields, where in Multi-Agent Systems the focus is geared towards technical details, implementing a system of interacting intelligent agents within a highly complex environment with the primary focus being solving AI problems.

\medskip

The field of ABS can be traced back to self-replicating von Neumann machines, cellular automata and Conway's Game of Life. The famous Schelling segregation model \cite{schelling_dynamic_1971} is regarded as a pioneering example. ABS as a discipline was first picked up by social simulation, which explores social norms, institutions, reputation, elections and economics. Axelrod \cite{axelrod_advancing_1997, axelrod_guide_2006} has called social simulation the third way of doing science which he termed the \textit{generative} approach. This is in opposition to the classical inductive (finding patterns in empirical data) and deductive (proving theorems). Consequently, the generative approach can be seen as a form of empirical research and is a natural methodology for studying social and interdisciplinary phenomena as discussed more in depth in the work of Epstein \cite{epstein_chapter_2006, epstein_generative_2012}. He gives a fundamental introduction to agent-based social social simulation and makes the strong claim that \textit{"If you didn't grow it, you didn't explain its emergence"} \footnote{This can be seen as a fundamental constructivist approach to social science, which implies that the emergent properties are actually computable. When making connections from the simulation to reality, constructible emergence raises the question whether our existence is computable or not. When pushing this further, we can suppose that the future of simulation will be simulated copies of our own existence, which potentially allows us to then simulate \textit{everything}. This idea is not actually new and an interesting treatment of it can be found in \cite{bostrom_are_2003, steinhart_theological_2010}.}. 
Epstein puts considerable emphasis on the claim that ABS is indeed a scientific instrument because hypotheses about the outcome of a simulation are empirically falsifiable. If the simulation exhibits an emergent pattern, then the model is \textit{one} way of explaining it. On the other hand, if it does not show the emergent pattern, then the hypothesis that the micro interactions amongst the agents generate the emergent pattern is falsified \footnote{This is fundamentally following Poppers theory of science \cite{popper_logic_2002}.} and we have not found an explanation \textit{yet}. In conclusion, growing a phenomena is a necessary, but not sufficient condition for explanation \cite{epstein_chapter_2006}.

% NOTE: incorporate this only when there is enough time (and energy) to go through the 3 references cited here
%This raises a number of philosophical questions \cite{frigg_philosophy_2009}, \cite{grune-yanoff_philosophy_2010}, \cite{borrill_agent-based_2011}. Although we don't want to give an in-depth discussion of the questions raised, we want to have a quick look at them as this is a foundational research-proposal for a Doctor in \textit{Philosophy} (Ph.D.).
%TODO: read above papers and give short outline philosophical questions

The first large scale social ABS model which rose to some prominence was the \textit{Sugarscape} model developed by Epstein and Axtell in 1996 \cite{epstein_growing_1996}. Their aim was to \textit{grow} an artificial society by simulation and connect observations in their simulation to the phenomenon of real-world societies. It was this model which strongly advertised object-oriented programming to implement ABS. Due to its influence and also due to the general popularity of the object-oriented paradigm which started to rise in the early-to-mid 1990s, object-oriented programming has become the de-facto standard in implementing ABS. We can distinguish between three categories of ABS implementation today: % TODO: do we need citiations here to support our claims?
\begin{enumerate}
	\item Programming from scratch using object-oriented languages, with Python, Java and C++ being the most popular.
	\item Programming with a 3rd party ABS library using object-oriented languages where RePast and DesmoJ, both in Java, are the most popular.
	\item Using a high-level ABS toolkit for non-programmers, which allow customisation through programming, if necessary. By far the most popular is NetLogo with an imperative programming approach followed by AnyLogic with an object-oriented Java approach.
\end{enumerate}

To get a better idea and deeper understanding of ABS, the next sections present two different but well-known agent-based models, to give examples of two different types: the \textit{explanatory} SIR model and the \textit{exploratory} Sugarscape model. Both are used throughout this thesis as sample cases for developing pure functional ABS implementation techniques, concepts and test beds for Software Transactional Memory and property-based testing.

\subsection{The SIR model}
\label{sec:sir_model}

The explanatory SIR model is a thoroughly studied and well understood compartment model from epidemiology \cite{kermack_contribution_1927}, which allows simulation of the dynamics of an infectious disease like influenza, tuberculosis, chicken pox, rubella and measles spreading through a population. The reason for choosing this model is its simplicity. It is easy to understand fully but complex enough to develop basic concepts of pure functional ABS, which are then extended and deepened in the much more complex Sugarscape model explained in the next section.

In this model, people in a population of size $N$ can be in either one of three states: \textit{Susceptible}, \textit{Infected} or \textit{Recovered}, at any particular time. It is assumed that initially there is at least one infected person in the population. People interact \textit{on average} with a given rate of $\beta$ other people per time unit, and become infected with a given probability $\gamma$ when interacting with an infected person. When infected, a person recovers \textit{on average} after $\delta$ time units and is then immune to further infections. An interaction between infected persons does not lead to reinfection, thus these interactions are ignored in this model. This definition gives rise to three compartments with the transitions seen in Figure \ref{fig:sir_transitions}.

\begin{figure}
	\centering
	\includegraphics[width=.7\textwidth, angle=0]{./fig/timedriven/SIR_transitions.png}
	\caption[States and transitions in the SIR compartment model]{States and transitions in the SIR compartment model.}
	\label{fig:sir_transitions}
\end{figure}

This model was also formalized using System Dynamics \cite{porter_industrial_1962}. In System Dynamics a system is modelled through differential equations, which allow expressing continuous systems, changing over time. They are solved by numerically integrating over time, which gives rise to the respective dynamics. The SIR model is modelled using the following equation, with the dynamics shown in Figure \ref{fig:sir_sd_dynamics} .

\begin{equation}
\begin{aligned}
\frac{\mathrm d S}{\mathrm d t} = -infectionRate \\
\frac{\mathrm d I}{\mathrm d t} = infectionRate - recoveryRate \\
\frac{\mathrm d R}{\mathrm d t} = recoveryRate 
\end{aligned}
\end{equation}

\begin{equation}
\begin{aligned}
infectionRate = \frac{I \beta S \gamma}{N} \\
recoveryRate = \frac{I}{\delta} 
\end{aligned}
\end{equation}

\begin{figure}
	\centering
	\includegraphics[width=0.5\textwidth, angle=0]{./fig/timedriven/SIR_SD_1000agents_150t_001dt.png}
	\caption[Dynamics of the SIR compartment model using the System Dynamics approach]{Dynamics of the SIR compartment model using the System Dynamics approach. Population Size $N$ = 1,000, contact rate $\beta =  \frac{1}{5}$, infection probability $\gamma = 0.05$, illness duration $\delta = 15$ with initially 1 infected agent. Simulation run for 150 time steps. Generated using our pure functional System Dynamics approach (see Appendix \ref{app:sd_simulation}).}
	\label{fig:sir_sd_dynamics}
\end{figure}

The approach of mapping the SIR model to an ABS is to discretise the population and model each person in the population as an individual agent. The transitions between the states are happening due to discrete events caused both by interactions amongst the agents and timeouts. The major advantage of ABS over System Dynamics is that it allows for the incorporation of spatiality and heterogeneity of a population, for example accounting for different sexes and ages. This is not directly possible with other simulation methods of System Dynamics or Discrete Event Simulation \cite{zeigler_theory_2000}.

This is directly related to a networked SIR model, where the interactions between agents are restricted by either a statically fixed or dynamically evolving network. Various network types exist, allowing for simulation of various scenarios. Very small communities where all agents are in contact with each other are modelled by a fully connected network. Real world scenarios where a few agents act as hubs are modelled by complex networks \cite{BarabasiAlbert_EmergenceScaling, Jackson2008, Newman_ComplexNetworks, WattsStrogatz_DynamicsSmallWorld}. In this thesis we do not impose restrictions on the connections among agents and always assume a fully connected network. Adding various network types to our thesis would unnecessarily complicate things in the beginning but would not constitute anything fundamentally new in terms of research. However, the use of complex networks, which in general are generated randomly, constitute an interesting direction for further research especially in the context of randomised property-based testing in ABS, which we discuss in Chapters \ref{ch:agentspec} and \ref{ch:sir_invariants}.

In the ABS classification of \cite{macal_everything_2016}, this model can be seen as an \textit{Interactive ABMS}: agents are individual heterogeneous agents with diverse set characteristics; they have autonomic, dynamic, endogenously defined behaviour; interactions happen between other agents and the environment through observed states, behaviours of other agents and the state of the environment.

\section{Case-Study II: Sugarscape}
TODO: 
we can implement everything except synchronous direct agent-interactions atm: if agent-interaction is one-way e.g. paying back a loan then this is no problem. thus the following parts of the Sugarscape are not possible with our current STM approach: mating, trading and lending  because all 3 require direct agent-to-agent interaction over multiple steps. We leave the problem of developing such an algorithm / implementation for further research.

\chapter{Functional Programming}
MacLennan \cite{maclennan_functional_1990} defines Functional Programming as a methodology and identifies it with the following properties:

\begin{enumerate}
	\item It is programming without the assignment-operator.
	\item It allows for higher levels of abstraction.
	\item It allows to develop executable specifications and prototype implementations.
	\item It is connected to computer science theory.
	\item Parallel Programming.
	\item Suitable for AI.
\end{enumerate}

The last two points don't weight as heavy today as back in 1990 as other languages came up with features for better parallel programming but they all do it by introducing functional features.

MacLennan \cite{maclennan_functional_1990} defines properties of pure expressions 
\begin{itemize}
	\item Value is independent of the evaluation order.
	\item Expressions can be evaluated in parallel.
	\item Referential transparency.
	\item No side effects.
	\item Inputs to an operation are obvious from the written form.
	\item Effects to an operation are obvious from the written form.
\end{itemize}

TODO: The question is then if we could implement in a functional style in an imperative object-oriented programming language? Or put otherwise: are these properties unique to functional programming or can we program functional in an imperative language (be it OO or not)?

Thus functional programming is identified as programming without the assignment operator and with pure expressions instead. Further characteristics are the missing of orderings as in imperative programming, caused by assignments: in functional programming the style is applicative which means we apply values to functions. The fundamental theoretical root is in the lambda calculus.

TODO: make distinction between 'applicative programming (style)', which is easily possible in imperative languages as well and 'functional programming' which is not possible in procedural languages. The question is if it is possible in OO by using some OO features to work around the limitations of procedural languages.

- cite critics

\section{Methodology}
%- Methodology is the justification for using a particular research method.
%- make clear what our method is (Method is simply a research tool, a component of research – say for example, a qualitative method such as interviews) and then justify it => this is then the methodology 
%- discussion of methodology is missing: what is the scientific approach we used in our thesis to address the aims and answer hypotheses? Basically we perform use-cases and discuss them\\

In this section we briefly motivate and justify our methods, to point out the scientific approach used in this thesis to address the aims and answer hypotheses put forward in the Introduction Chapter \ref{ch:intro}. Fundamentally, the method we use is developing concepts step-by-step using the two well known agent-based models SIR \ref{sec:sir_model} and Sugarscape \ref{sec:sugarscape}. We put our approach into a broader context of how to implement ABS from a programming language agnostic view, discussed in Chapter \ref{ch:impl_abs} which serves as underlying assumptions and general direction to follow.

The first part of our method is dedicated to develop how to implement ABS in a pure functional way. The aim is to develop a robust, maintainable and extensible solution to existing models at which example we show concepts which can be adopted to ABS in general. The overall goal is a clear representation of agents with their local (immutable) state, a way for the agents to interact with an (active) environment and interactions amongst agents where necessary. It is quite important to state this clearly as we could follow a rather completely data-driven approach, which would have been very easy in pure functional programming: represent all agents and the simulation state as a big data-structure which is passed in and out pure functions (thus read/write). Indeed it would work but we would probably end up in a difficult to understand data-flow (everything is read/write) and what is worse: we don't arrive at very general solutions as we would not abstract out concepts, existing in ABS already which could then be transferred easily. Note that a more data-driven approach has indeed its value, depending on the model! In Chapter \ref{ch:gintis_case} we briefly introduce the field of ACE, where agents are almost always completely represented by data and very simple behaviour and do not interact in a complex way as in Sugarscape. Examples are ZI traders or bilateral traders which are all simply represented by data and interact with each other quite indirectly through a trading and bartering process.

The second part of our method is dedicated to show the benefits of using the previously developed pure functional approach to ABS. It is split into two parts where in the first we investigate the hypothesis that pure functional programming makes it easy to apply parallel computation using parallelism and concurrency to ABS. The second part answers another central hypothesis namely that randomize property-based testing is a good match to test stochastic ABS implementations.

The concepts we derive are driven by the hypotheses and aims from the introduction and we continuously refer back to them, especially in the respective discussions and the final conclusion and discussion. By doing this we are able to qualitatively assess whether we have achieved our aim and answered the hypotheses in a satisfactory way, which is reviewed more in-depth and holistic in the final discussion and conclusion chapters.