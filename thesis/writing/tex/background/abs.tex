\section{Agent-Based Simulation}
\label{sec:method_abs}

%TODO RESTRUCTURING
%- classification according to \cite{macal_everything_2016}: macal paper \cite{macal_everything_2016}: very good survey/review paper on ABMS in General. fp can help with challenges h2, h4 and h5. also fp can help macals added transparency challenge, my thesis in general also adresses the knowledge challenge of macal "lack of abms educational...", note that we do NOT address ease-of-use as our approach is not easy to use. also the yampa approach can be seen as a hybrid approach of ABS/SD as posed as Research Challenge by macal. further STM might be one way of tackling large-scale ABS as identified as Research Challenge by macal. also this paper supports that ABS is a fundamentally new technique that offers the Potential to solve problems that are not robustly addressed by other methods
%\bigskip

This thesis understands Agent-Based Simulation (ABS) as a methodology to model and simulate a system, where the global behaviour may be unknown but the behaviour and interactions of the parts making up the system is known. Those parts, called agents, are modelled and simulated, out of which then the aggregate global behaviour of the whole system emerges. So, the central aspect of ABS is the concept of an agent, a metaphor for a pro-active unit, situated in an environment, able to spawn new agents and interacting with other agents in some neighbourhood by exchange of messages \cite{macal_everything_2016, odell_objects_2002, siebers_introduction_2008, wooldridge_introduction_2009}. Summarising, this thesis informally assumes the following about agents:

\begin{itemize}
	\item They are uniquely addressable entities with some internal state over which they have full, exclusive control.
	\item They are pro-active, which means they can initiate actions on their own e.g. change their internal state, send messages, create new agents, terminate themselves.
	\item They are situated in an environment and can interact with it.
	\item They can interact with other agents situated in the same environment by means of messaging.
\end{itemize} 

Epstein \cite{epstein_generative_2012} identifies ABS to be especially applicable for analysing \textit{"spatially distributed systems of heterogeneous autonomous actors with bounded information and computing capacity"}. They exhibit the following properties:

\begin{itemize}
	\item Linearity \& Non-Linearity - actions of agents can lead to non-linear behaviour of the system.
	\item Time - agents act over time, which is also the source of their pro-activity.
	\item State - agents encapsulate some state, which can be accessed and changed during the simulation.
	\item Feedback-Loop - because agents act continuously and their actions influence each other and themselves in the future of subsequent time-steps, feedback-loops are a fundamental concept, inherent to every ABS. 
	\item Heterogeneity - agents can have properties (age, height, sex,...) where the actual values can vary arbitrarily between agents.
	\item Interactions - agents can be modelled after interactions with an environment and other agents.
	\item Spatiality \& Networks - agents can be situated within arbitrary environments, like spatial environments (discrete 2D, continuous 3D,...) or complex networks.
\end{itemize}

Note that there doesn't exist a commonly agreed technical definition of ABS but the field draws inspiration from the closely related field of Multi-Agent Systems (MAS) \cite{weiss_multiagent_2013,wooldridge_introduction_2009}. It is important to understand that MAS and ABS are two different fields where in MAS the focus is much more on technical details, implementing a system of interacting intelligent agents within a highly complex environment with the focus primarily on solving AI problems.

\medskip

The field of ABS can be traced back to self-replicating von Neumann machines, cellular automata and Conway's Game of Life. The famous Schelling segregation model \cite{schelling_dynamic_1971} is regarded as a pioneering example. The most prominent topics which are explored in social simulation are social norms, institutions, reputation, elections and economics.

Axelrod \cite{axelrod_advancing_1997, axelrod_guide_2006} has called social simulation the third way of doing science, which he termed the \textit{generative} approach, which is in opposition to the classical inductive (finding patterns in empirical data) and deductive (proving theorems). Thus, the generative approach can be seen as a form of empirical research and is a natural environment for studying social and interdisciplinary phenomena as discussed more in-depth in the work of Epstein \cite{epstein_chapter_2006, epstein_generative_2012}. He gives a fundamental introduction to agent-based social social simulation and makes the strong claim that \textit{"If you didn't grow it, you didn't explain its emergence"} \footnote{Note the fundamental constructivist approach to social science, which implies that the emergent properties are actually computable. When making connections from the simulation to reality, constructible emergence raises the question whether our existence is computable or not. When pushing this further, we can conjecture that the future of simulation will be simulated copies of our own existence, which potentially allows to simulate \textit{everything}. This idea is not new and an interesting treatment of it can be found in \cite{bostrom_are_2003, steinhart_theological_2010}.}. Epstein puts much emphasis on the claim that ABS is indeed a scientific instrument as hypotheses which are investigated are empirical falsifiable: the simulation exhibits the emergent pattern in which case the model is \textit{one} way of explaining it or it simply does not show the emergent pattern, in which case the hypothesis, that the model (the micro-interactions amongst the agents) generates the emergent pattern is falsified \footnote{This is fundamentally following Poppers theory of science \cite{popper_logic_2002}.} - we haven't found an explanation \textit{yet}. So in summary, growing a phenomena is a necessary, but not sufficient condition for explanation \cite{epstein_chapter_2006}.

% NOTE: incorporate this only when there is enough time (and energy) to go through the 3 references cited here
%This raises a number of philosophical questions \cite{frigg_philosophy_2009}, \cite{grune-yanoff_philosophy_2010}, \cite{borrill_agent-based_2011}. Although we don't want to give an in-depth discussion of the questions raised, we want to have a quick look at them as this is a foundational research-proposal for a Doctor in \textit{Philosophy} (Ph.D.).
%TODO: read above papers and give short outline philosophical questions

The first large scale ABS model which rose to some prominence was the \textit{Sugarscape} model developed by Epstein and Axtell in 1996 \cite{epstein_growing_1996}. Their aim was to \textit{grow} an artificial society by simulation and connect observations in their simulation to phenomenon of real-world societies. It was this simulation which strongly advertised object-oriented programming to implement ABS and due to this and also due to the general popularity of object-oriented which started to rise in the early-mid 90s, it is reasonable to say that the established, state-of-the-art approach to implement ABS falls into three categories: % TODO: do we need citiations here to support our claims?
\begin{enumerate}
	\item Programming from scratch using object-oriented languages with Python, Java and C++ being the most popular ones.
	\item Programming using a 3rd party ABS library using object-oriented languages where RePast and DesmoJ, both in Java, are the most popular ones.
	\item Using a high-level ABS tool-kit for non-programmers, which allow customization through programming if necessary. By far the most popular one is NetLogo with an imperative programming approach followed by AnyLogic with an object-oriented Java approach.
\end{enumerate}

To get a deeper idea and understanding of ABS, the next sections present two different, well-known agent-based models to give examples of two different types: the explanatory SIR model and the exploratory Sugarscape model. Both are used throughout the thesis as use-cases for developing the pure functional ABS techniques.

% TODO: this is a nice blog: https://drewdevault.com/2018/07/09/Simple-correct-fast.html
% TODO: \cite{vipindeep_list_2005}
% TODO: software errors can be costly %https://raygun.com/blog/costly-software-errors-history/
% TODO: bugs per loc %https://www.mayerdan.com/ruby/2012/11/11/bugs-per-line-of-code-ratio

\subsection{The SIR model}
\label{sec:sir_model}

The explanatory SIR model is a thoroughly studied and well understood compartment model from epidemiology \cite{kermack_contribution_1927}, which allows simulation of the dynamics of an infectious disease like influenza, tuberculosis, chicken pox, rubella and measles spreading through a population. The reason for choosing this model is its simplicity. It is easy to understand fully but complex enough to develop basic concepts of pure functional ABS, which are then extended and deepened in the much more complex Sugarscape model explained in the next section.

In this model, people in a population of size $N$ can be in either one of three states: \textit{Susceptible}, \textit{Infected} or \textit{Recovered}, at any particular time. It is assumed that initially there is at least one infected person in the population. People interact \textit{on average} with a given rate of $\beta$ other people per time unit, and become infected with a given probability $\gamma$ when interacting with an infected person. When infected, a person recovers \textit{on average} after $\delta$ time units and is then immune to further infections. An interaction between infected persons does not lead to reinfection, thus these interactions are ignored in this model. This definition gives rise to three compartments with the transitions seen in Figure \ref{fig:sir_transitions}.

\begin{figure}
	\centering
	\includegraphics[width=.7\textwidth, angle=0]{./fig/timedriven/SIR_transitions.png}
	\caption[States and transitions in the SIR compartment model]{States and transitions in the SIR compartment model.}
	\label{fig:sir_transitions}
\end{figure}

This model was also formalized using System Dynamics \cite{porter_industrial_1962}. In System Dynamics a system is modelled through differential equations, which allow expressing continuous systems, changing over time. They are solved by numerically integrating over time, which gives rise to the respective dynamics. The SIR model is modelled using the following equation, with the dynamics shown in Figure \ref{fig:sir_sd_dynamics} .

\begin{equation}
\begin{aligned}
\frac{\mathrm d S}{\mathrm d t} = -infectionRate \\
\frac{\mathrm d I}{\mathrm d t} = infectionRate - recoveryRate \\
\frac{\mathrm d R}{\mathrm d t} = recoveryRate 
\end{aligned}
\end{equation}

\begin{equation}
\begin{aligned}
infectionRate = \frac{I \beta S \gamma}{N} \\
recoveryRate = \frac{I}{\delta} 
\end{aligned}
\end{equation}

\begin{figure}
	\centering
	\includegraphics[width=0.5\textwidth, angle=0]{./fig/timedriven/SIR_SD_1000agents_150t_001dt.png}
	\caption[Dynamics of the SIR compartment model using the System Dynamics approach]{Dynamics of the SIR compartment model using the System Dynamics approach. Population Size $N$ = 1,000, contact rate $\beta =  \frac{1}{5}$, infection probability $\gamma = 0.05$, illness duration $\delta = 15$ with initially 1 infected agent. Simulation run for 150 time steps. Generated using our pure functional System Dynamics approach (see Appendix \ref{app:sd_simulation}).}
	\label{fig:sir_sd_dynamics}
\end{figure}

The approach of mapping the SIR model to an ABS is to discretise the population and model each person in the population as an individual agent. The transitions between the states are happening due to discrete events caused both by interactions amongst the agents and timeouts. The major advantage of ABS over System Dynamics is that it allows for the incorporation of spatiality and heterogeneity of a population, for example accounting for different sexes and ages. This is not directly possible with other simulation methods of System Dynamics or Discrete Event Simulation \cite{zeigler_theory_2000}.

This is directly related to a networked SIR model, where the interactions between agents are restricted by either a statically fixed or dynamically evolving network. Various network types exist, allowing for simulation of various scenarios. Very small communities where all agents are in contact with each other are modelled by a fully connected network. Real world scenarios where a few agents act as hubs are modelled by complex networks \cite{BarabasiAlbert_EmergenceScaling, Jackson2008, Newman_ComplexNetworks, WattsStrogatz_DynamicsSmallWorld}. In this thesis we do not impose restrictions on the connections among agents and always assume a fully connected network. Adding various network types to our thesis would unnecessarily complicate things in the beginning but would not constitute anything fundamentally new in terms of research. However, the use of complex networks, which in general are generated randomly, constitute an interesting direction for further research especially in the context of randomised property-based testing in ABS, which we discuss in Chapters \ref{ch:agentspec} and \ref{ch:sir_invariants}.

In the ABS classification of \cite{macal_everything_2016}, this model can be seen as an \textit{Interactive ABMS}: agents are individual heterogeneous agents with diverse set characteristics; they have autonomic, dynamic, endogenously defined behaviour; interactions happen between other agents and the environment through observed states, behaviours of other agents and the state of the environment.

\section{Case-Study II: Sugarscape}
TODO: 
we can implement everything except synchronous direct agent-interactions atm: if agent-interaction is one-way e.g. paying back a loan then this is no problem. thus the following parts of the Sugarscape are not possible with our current STM approach: mating, trading and lending  because all 3 require direct agent-to-agent interaction over multiple steps. We leave the problem of developing such an algorithm / implementation for further research.