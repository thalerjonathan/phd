\subsection{Sugarscape}
\label{sec:sugarscape}

The seminal Sugarscape model was one of the first models in ABS, developed by Epstein and Axtell in 1996 \cite{epstein_growing_1996}. Their aim was to \textit{grow} an artificial society by simulation and connect observations in their simulation to phenomena observed in real-world societies, making it an \textit{exploratory} model. In the model a population of agents move around in a discrete 2D environment, where sugar and spice grow, and interact with each other and the environment in multiple ways. The main features of this model are (amongst others): searching, harvesting and consuming of resources, wealth and age distributions, population dynamics under sexual reproduction, cultural processes and transmission, combat and assimilation, bilateral decentralized trading (bartering) between agents with endogenous demand and supply, disease processes transmission and immunology.

The reasons for choosing the Sugarscape model as use case in this thesis are: it is quite well known in the ABS community, it was highly influential in sparking the overall interest in ABS, it is quite complex with non-trivial agent interactions, the original implementation was done in Object Pascal and C with about 20,000 lines of code which includes GUI, graphs and plotting, where the authors used Object Pascal for programming the agents and C for low-level graphics \cite{axtell_aligning_1996}. The authors explicitly advocate object-oriented programming as a good fit with ABS which begged the question whether and how well a pure functional implementation is possible. 

In the ABS classification of \cite{macal_everything_2016}, the Sugarscape can be seen as an \textit{Adaptive ABMS}: agents are individual heterogeneous agents with diverse set characteristics: they have autonomic, dynamic, endogenously defined behaviour, interactions happen between other agents and the environment through observed states, behaviours of other agents and the state of the environment, agents can change their behaviour during the simulation through observing their own state and learning and populations can adjust their composition.

The full specification of the Sugarscape model itself fills a small book of about 200 pages, so we will only give a very brief overview of the model in terms of activities that occur. See Figure \ref{fig:sugarscape_activities} for a diagrammatic explanation of the activity flow in a single tick, as summarised in the text below.

Generally, the Sugarscape is stepped in discrete, natural number time steps, also called ticks, where in each tick the following actions happen:

\begin{enumerate}
	\item Shuffle all agents and process them sequentially. The reason why the agents are shuffled is to even-out the odds of being scheduled at a specific position - it is equally probable to be scheduled in any position. The semantics of the model follow the sequential update strategy (see Chapter \ref{ch:impl_abs}), thus requiring stepping the agents sequentially. Ideally, though, one wants to avoid any biases in ordering and pretend that agents act conceptually or statistically at the same time. The shuffling allows to do this by \textit{running the agents sequentially making their behaviour appear statistically in parallel}. Every agent executes the following actions, where agents executed after the agent in the same tick can already see the changes and interactions of preceding agents. Each agent behaves as follows:
	
	\begin{enumerate}
		\item The agent ages by 1 tick. An agent might have a maximum age, which when reached will result in the removal of the agent from the simulation (see below).
	
		\item Move to the nearest unoccupied site in sight with highest resource. In the event of combat, sites that are also occupied with agents from a different tribe are potential targets. Harvest all the resources on the site and in case of combat, additionally reap the enemies resources, or gather a combat reward. This is one of the primary reasons why the Sugarscape model needs to be stepped sequentially: because only one agent can occupy a site at a time, it would lead to conflicts when agents actually act at the same time.

		\item Apply the agents' metabolism. Each agent needs to consume a given number of resources in each tick to satisfy its metabolism. The gathered resources can be stocked up during the harvesting process, but if the agent does not have enough resources to satisfy its metabolism, it will be removed from the simulation (see below).
		
		\item Apply pollution from the environment through the agent. Depending on how much the agent has harvested during its movement and consumed in its metabolism process, it will leave a small fraction of pollution in the environment.
		
		\item Check if the agent has died from age or starved to death, in the event it removes itself from the simulation and does not execute the next steps (the previous steps are executed independently from the age of the agent). Depending on the model configuration this could also lead to the respawning of a new agent which replaces the dead agent.
		
		\item Engage with other neighbours in mating, which involves multiple synchronous interaction steps happening in the same tick: exchange of information and both agents agreeing on the mating action. If both agents agree to mate, the initiating agent spawns a new agent, with characteristics inherited from both parents. See Figure \ref{fig:sugarscape_visualisation_trading_mating}.
		
		\item Engage in the cultural process, where cultural tags are picked up from other agents and passed on to other agents. This action is a one-way interaction where the neighbours do not reply synchronously.
		
		\item Engage in trading with neighbours where the initiating agent offers a given resource (sugar / spice) in exchange for another resource (spice / sugar). The agent asks every neighbour and a trade will take place if it makes both agents better off. This action involves multiple synchronous interaction steps within the same tick because of the exchange of information and agreeing on the final transaction. See Figure \ref{fig:sugarscape_visualisation_trading_mating}.
		
		\item Engage in lending and borrowing, where the agent offers loans to neighbours. This action also involves multiple synchronous interaction steps within the same tick because of exchange of information and agreeing on the final transaction.
		
		\item Engage in disease processes, where the agent passes on diseases it has to other neighbour agents. This action is a one-way interaction where the neighbours do not reply synchronously.
	\end{enumerate}
		
	\item Run the environment which consists of an NxN discrete grid
		\begin{enumerate}
			\item Re-grow resources on each site according to the model configuration: either with a given rate per tick, as seen in Figure \ref{fig:sugarscape_visualisation_normal}, or immediately. Depending on whether seasons are enabled (see Figure \ref{fig:sugarscape_visualisation_seasons}) the re-growing rate varies in different regions of the environment.
			
			\item Apply diffusion of pollution where the pollution generated by the agents spreads out slowly across the whole environment, see Figure \ref{fig:sugarscape_visualisation_pollution}.
		\end{enumerate}
\end{enumerate}

\begin{figure}
	\centering
	\includegraphics[width=.9\textwidth, angle=0]{./fig/background/abs/sugarscape_activity.png}
	\caption[Activity flow in Sugarscape]{Activity flow in Sugarscape in a single tick.}
	\label{fig:sugarscape_activities}
\end{figure}

In Figure \ref{fig:sugarscape_visualisation} visualisations of our Sugarscape implementation as discussed in Chapter \ref{sec:advanced_eventdriven_ABS} are shown.

\begin{figure}[H]
\begin{center}
	\begin{tabular}{c c}
		\begin{subfigure}[b]{0.4\textwidth}
			\centering
			\includegraphics[width=1\textwidth, angle=0]{./fig/background/abs/sugarscape_normal.png}
			\caption{\textit{Animation II-2}: resource growback and infinite agent age.}
			\label{fig:sugarscape_visualisation_normal}
		\end{subfigure}
		
		&
    	
		\begin{subfigure}[b]{0.4\textwidth}
			\centering
			\includegraphics[width=1\textwidth, angle=0]{./fig/background/abs/sugarscape_pollution.png}
			\caption{\textit{Animation II-8}: active pollution with diffusion.}
			\label{fig:sugarscape_visualisation_pollution}
		\end{subfigure}
	\end{tabular}

	
	\begin{tabular}{c c}
		\begin{subfigure}[b]{0.4\textwidth}
			\centering
			\includegraphics[width=1\textwidth, angle=0]{./fig/background/abs/sugarscape_seasons.png}
			\caption{\textit{Animation II-7}: seasons, agents trying to migrate.}
			\label{fig:sugarscape_visualisation_seasons}
		\end{subfigure}	
		
		& 
		
		\begin{subfigure}[b]{0.4\textwidth}
			\centering
			\includegraphics[width=1\textwidth, angle=0]{./fig/background/abs/sugarscape_trading_mating.png}
			\caption{\textit{Figure IV-14}: trading, mating and finite life spans.}
			\label{fig:sugarscape_visualisation_trading_mating}
		\end{subfigure}
	\end{tabular}
	
	\caption[Visualisation of the Sugarscape implementation]{Visualisation of the Sugarscape implementation (see Chapter \ref{sec:advanced_eventdriven_ABS}). The naming of the respective \textit{Animation} and \textit{Figure} is taken from \cite{epstein_growing_1996}.} 
	\label{fig:sugarscape_visualisation}
\end{center}
\end{figure}