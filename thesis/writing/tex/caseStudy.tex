\section{Case-Study}
perform gintis case-study: apply my developed techniques of implementing ABS and testing and concurrency / parallelism to the gintis paper (and its follow ups: the ionescu paper and the masterthesis on it). 
The aim of this is to see: 
1. we can do DES under the hood, 2. property-based tests in a different setting, 3. could pure functional programming have prevented the failure? 4. could property based tests have prevented the failure? 5. could dependent and / or types have prevented the failure? 

1. do the techniques transfer to this problem? 
2. does haskell prevent making that mistake which gintis made? 
3. how close is our implementation to ionescus functional specification? 
4. validation and verification against gintis paper using property-based testing of individual agents and the simulation as a whole. 
5. being more familiar with dependent types, would they help and where do they fit in in combination with the ionescu functional specification, which mentions dependent types at the end.

after re-reading ionescu paper: too complex and out of scope, but ionescu work more directly applicable in a pure functional implementation than in e.g. c++ (that was what they used).

we base our implementation on the existing gintis code from https://people.umass.edu/gintis/ 
also we make use of the masterthesis on gintis work which revealed a few bugs

reasoning about termination is easier