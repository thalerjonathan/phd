\chapter{Concurrent ABS}
\label{ch:concurrent_abs}
Functional programming as in Haskell is well known and accepted as a remedy against the problems of imperative programming in implementing parallel software TODO: cite ?. The reason for it is clear: immutable data and explicit control of side-effects removes a large class of bugs due to data-conflicts, data-races, and blablabla TODO: we are claiming things here, which we need to clearly back up, also data-races ARE possible in Haskell! A fundamental benefit and strength of Haskell is, that it clearly distinguishes between parallelism and concurrency \cite{jones_tackling_2002} and it is very important for us to do so as well:

\begin{itemize}
	\item \textbf{Parallelism} - In parallelism, code runs in parallel without interfering with other code through shared data (references, mutexes, semaphores,...). An example is the function \textit{map :: (a $\rightarrow$ b) $\rightarrow$ [a] $\rightarrow$ [b]}, which maps each element of type \textit{a} to \textit{b} using the function \textit{(a $\rightarrow$ b)}. It is a pure function and thus no sharing of data either through some monadic context or through the function \textit{(a $\rightarrow$ b)} is possible. This allows to run it in parallel: each function evaluation \textit{(a $\rightarrow$ b)} could potentially be executed at the same time, if we had enough CPU cures. Whether it runs actually in parallel or not, has no influence on the outcome, it is not subject to any non-deterministic influences. Thus we identify parallelism with pure and deterministic execution of data-transformations (data-parallelism).
	
	\item \textbf{Concurrency} - In concurrency, code runs in parallel but can potentially interfere with other code through shared data (references, mutexes, semaphores, ...). An example are two threads, running in parallel, which share data through \textit{IORefs}. In concurrency there is no option: code has to run in parallel through the use of threads but now the outcome of the program very much depends on the ordering in which the threads are scheduled. This gives rise to very different access patterns to the shared data, with the potential for race conditions, dirty reads and so on... The challenge of implementing concurrent programs, is to write the program in a way that despite of these non-deterministic influences it is still a correctly working program. Thus we identify concurrency with impure and non-deterministic execution of imperative-style monadic command execution.
\end{itemize}

There is obvious potential for adding (data-)parallelism to ABS e.g. using data-parallel data-structures for the environment so cells can be updated in parallel, in time-driven ABS agents can be updated in parallel using parMap because they all act conceptually at the same time as shown already in Yampa \footnote{\url{https://www.reddit.com/r/haskell/comments/2jbl78/from_60_frames_per_second_to_500_in_haskell/}}. Despite the potential for parallelism, in this chapter we focus on concurrency only and refer to the book \cite{marlow_parallel_2013} for an in-depth discussions of the mechanism for parallelism in Haskell. The reason for focusing on concurrency and leaving parallelism out is simple: the Ph.D. doesn't provide enough time to explore both in equal depth and the application of STM to implement concurrent ABS looks very much more interesting and challenging probably because it is also a complete novelty.

\medskip

More specifically, in this chapter, we look into using Software Transactional Memory (STM) for implementing concurrent ABS. The benefits of STM is that it allows to overcome the problems of lock-based approaches. Although STM exists in other languages as well, Haskell was one of the first to natively build it into its core. Further, it has the unique benefit that it can guarantee the lack of persistent side-effects at compile time, allowing unproblematic retries of transactions, something of fundamental importance in STM. This makes the use of STM in Haskell very compelling.

The paper \cite{discolo_lock_2006} gives a good indication how difficult and complex constructing a correct concurrent program is and shows how much easier, concise and less error-prone an STM implementation is over traditional locking with mutexes and semaphores. Further it shows that STM consistently outperforms the lock-based implementation. We follow this work and compare the performance of lock-based and STM implementations and hypothesise that the reduced complexity and increased performance will be directly applicable to ABS as well.

We present two case studies using the already introduced SIR (Chapter \ref{sec:sir_model}) and Sugarscape (Chapter \ref{sec:sugarscape}) models, where we compare the performance of lock-based and STM implementations in two different well known Agent-Based Models, where we investigate both the scaling performance under increasing number of CPUs and the scaling performance under increasing number of agents. We show that the STM implementations consistently outperform the lock-based ones and scale much better to increasing number of CPUs both on local machines and on Amazon Cloud Services.

\section{Software Transactional Memory}
Software Transactional Memory (STM) was introduced by \cite{shavit_software_1995} in 1995 as an alternative to lock-based synchronisation in concurrent programming which, in general, is notoriously difficult to get right. This is because reasoning about the interactions of multiple concurrently running threads and low level operational details of synchronisation primitives is \textit{very hard}. The main problems are:

\begin{itemize}
	\item Race conditions due to forgotten locks;
	\item Deadlocks resulting from inconsistent lock ordering;
	\item Corruption caused by uncaught exceptions;
	\item Lost wake-ups induced by omitted notifications.
\end{itemize}

Worse, concurrency does not compose. It is very difficult to write two functions (or methods in an object) acting on concurrent data which can be composed into a larger concurrent behaviour. The reason for it is that one has to know about internal details of locking, which breaks encapsulation and makes composition dependent on knowledge about their implementation. Therefore, it is impossible to compose two  functions e.g. where one withdraws some amount of money from an account and the other deposits this amount of money into a different account: one ends up with a temporary state where the money is in none of either accounts, creating an inconsistency - a potential source for errors because threads can be rescheduled at any time.

STM promises to solve all these problems for a low cost by executing actions \textit{atomically}, where modifications made in such an action are invisible to other threads and changes by other threads are invisible as well until actions are committed - STM actions are atomic and isolated. When an STM action exits, either one of two outcomes happen: if no other thread has modified the same data as the thread running the STM action, then the modifications performed by the action will be committed and become visible to the other threads. If other threads have modified the data then the modifications will be discarded, the action block rolled-back and automatically restarted.

STM in Haskell is implemented using optimistic synchronisation, which means that instead of locking access to shared data, each thread keeps a transaction log for each read and write to shared data it makes. When the transaction exits, the thread checks whether it has a consistent view to the shared data or not: whether other threads have written to memory it has read. % This might look like a serious overhead but the implementations are very mature by now, being very performant and the benefits outweigh its costs by far.

In the paper \cite{heindl_modeling_2009} the authors use a model of STM to simulate optimistic and pessimistic STM behaviour under various scenarios using the AnyLogic simulation package. They conclude that optimistic STM may lead to 25\% less retries of transactions. The authors of \cite{perfumo_limits_2008} analyse several Haskell STM programs with respect to their transactional behaviour. They identified the roll-back rate as one of the key metric which determines the scalability of an application. Although STM might promise better performance, they also warn of the overhead it introduces which could be quite substantial in particular for programs which do not perform much work inside transactions as their commit overhead appears to be high.

\subsection{STM in Haskell}
The work of \cite{harris_composable_2005, harris_transactional_2006} added STM to Haskell, which was one of the first programming languages to incorporate STM into its main core and added the ability to composable operations. There exist various implementations of STM in other languages as well (Python, Java, C\#, C/C++, etc) but we argue, that it is in Haskell with its type-system and the way how side-effects are treated where it truly shines.

In the Haskell implementation, STM actions run within the \textit{STM} context. This restricts the operations to only STM primitives as shown below, which allows to enforce that STM actions are always repeatable without persistent side-effects because such persistent side-effects (e.g. writing to a file, launching a missile) are not possible in an \textit{STM} context. This is also the fundamental difference to  \textit{IO}, where all bets are off because \textit{everything} is possible as there are basically no restrictions because \textit{IO} can run everything.

Thus the ability to \textit{restart} a block of actions without any visible effects is only possible due to the nature of Haskells type-system: by restricting the effects to STM only, ensures that no uncontrolled effects, which cannot be rolled-back, occur.

STM comes with a number of primitives to share transactional data. Amongst others the most important ones are:

\begin{itemize}
	\item \textit{TVar} - A transactional variable which can be read and written arbitrarily;
	\item \textit{TArray} - A transactional array where each cell is an individual shared data, allowing much finer-grained transactions instead of e.g. having the whole array in a \textit{TVar};
	\item \textit{TChan} - A transactional channel, representing an unbounded FIFO channel;
	\item \textit{TMVar} - A transactional \textit{synchronising} variable which is either empty of full. To read from an empty or write to a full \textit{TMVar} will cause the current thread to retry its transaction.
\end{itemize}

% NOTE: too technical
%To run an \textit{STM} action the function \textit{atomically :: STM a $\to$ IO a} is provided, which can be seen as the STM effect-runner as it performs a series of \textit{STM} actions atomically within an \textit{IO} context. It takes the STM action which returns a value of type \textit{a} and returns an \textit{IO} action which returns a value of type \textit{a}. This \textit{IO} action can only be executed within an \textit{IO} context.

\section{STM in ABS}
\label{sec:stm_abs}
In this section we give a short overview of how we apply STM in our ABS. In both case-studies we fundamentally follow a time-driven, parallel approach as introduced in Chapter \ref{sub:par_strategy}, where the simulation is advanced by a given $\Delta t$ and in each step all agents are executed. To employ parallelism, each agent runs within its own thread and agents are executed in lock-step, synchronising between each $\Delta t$, which is controlled by the main thread. See Figure \ref{fig:stm_abs_structure} for a visualisation of our concurrent, time-driven lock-step approach.

By running each agent in a thread will guarantee the execution in parallel even if the agent has a monadic context. This is forces us to evaluate each agents monadic context separately instead of running them all in a common context. Note that ultimately we are ending up in the \textit{IO} context because \textit{STM} can be only transacted from within an \textit{IO} context due to non-deterministic side-effects. This is no contradiction to our original claim: yes we are running in IO but not the agent behaviour itself, which is a fundamental difference.

An agent thread will block until the main-thread sends the next $\Delta t$ and runs the \textit{STM} action atomically with the given $\Delta t$. When the \textit{STM} action has been committed, the thread will send the output of the agent action to the main-thread to signal it has finished. The main thread awaits the results of all agents to collect them for output of the current step e.g. visualisation or writing to a file.

As will be described in subsequent sections, central to both case-studies is an environment which is shared between the agents using a \textit{TVar} or \textit{TArray} primitive through which the agents communicate concurrently with each other. To get the environment in each step for visualisation purposes, the main thread can access the \textit{TVar} and \textit{TArray} as well. 

\begin{figure}
	\centering
	\includegraphics[width=1.0\textwidth, angle=0]{./fig/concurrentabs/stm_abs.png}
	\caption{Diagram of the parallel time-driven lock-step approach.}
	\label{fig:stm_abs_structure}
\end{figure}

\subsection{Adding and running the STM Monad}
We briefly show how to add STM to agents and run them within their own threads. We use the SIR implementation as example - applying it to the Sugarscape implementation works exactly the same way and is left as a trivial exercise to the reader.

The first step is to simply add the \textit{STM} to the existing transformer stack as the \textit{innermost} monad. The reason why we make it the innermost is to guarantee that in case of a retry \textit{all} outer monadic effects are retried as well - if the STM would be placed on a higher stack level, the levels below would not be subject to a retry. For monads like the \textit{ReaderT} this would not matter because they are read-only but for a StateT this fact would matter a lot. Note that STM does not provide a transformer instance, so this is not an option anyway. If STM would provide a transformer then we could make \textit{IO} the innermost monad and do \textit{IO} within STM, which should be prevented under all circumstances because then rolling back a transaction cannot guarantee to undo the effects. To better understand the semantics of retries consider the following example:

\begin{HaskellCode}

\end{HaskellCode}



\begin{HaskellCode}
innerSTMAction :: RandomGen g => StateT SomeState (RandT g STM) SomeResult

let randAction = runStateT innerSTMAction initState
let stmAction  = runRandT randAction (mkStdGen 42)
let ioAction   = atomically stmAction
((someResult, someState), g) <- ioAction
\end{HaskellCode}

In this case the STM is the \textit{innermost} monad thus it will be run last. This means that all the outer monads are subject to re-computation due to retries.

\begin{HaskellCode}
outerSTMAction :: STMT (StateT Environment (Rand g)) SomeResult

let ioAction = runSTMT outerSTMAction
stateAction <- ioAction
let randAction = runStateT stateAction initState
let ((someResult, someState), g) = runRandT randAction (mkStdGen 42)
\end{HaskellCode}

In this case, the STM is the \textit{outermost} monad, thus it will be run first. This means that it will return a StateT computation which will be computed \textit{after} the STM has transacted. The computation construction is subject to the retries but the computation itself won't be repeated in case of retries.

TODO: add STM

\begin{HaskellCode}
agentThread :: RandomGen g 
            => Int
            -> SIRAgent g
            -> g
            -> MVar SIRState
            -> MVar DTime
            -> IO ()
agentThread 0 _ _ _ _ = return () -- all steps computed, terminate thread
agentThread n sf rng retVar dtVar = do
  -- wait for dt to compute current step
  dt <- takeMVar dtVar

  -- compute output of current step
  let sfReader = unMSF sf ()
      sfRand   = runReaderT sfReader dt
      sfSTM    = runRandT sfRand rng
  ((ret, sf'), rng') <- atomically sfSTM -- run the STM action atomically within IO

  -- post result to main thread
  putMVar retVar ret
  
  -- to next step
  agentThread (n - 1) sf' rng retVar dtVar
\end{HaskellCode}

\begin{HaskellCode}
simulationStep :: TVar SIREnv
               -> [MVar DTime]
               -> [MVar SIRState]
               -> DTime
               -> IO SIREnv
simulationStep env dtVars retVars dt = do
  -- tell all threads to continue with the corresponding DTime
  mapM_ (`putMVar` dt) dtVars
  -- wait for results but ignore them, SIREnv contains all states
  mapM_ takeMVar retVars
  -- return state of environment when step has finished
  readTVarIO env
\end{HaskellCode}

\section{Discussion}

\subsection{Other Models}
TODO: mention that we have also implemented other models, which also work without time-semantics (all agents make a move at discrete time-steps and do not really rely on some notion of time). 

\subsection{Time-Semantics}
The main reason for building our pure functional ABMS approach on top of Yampa was to leverage the powerful time-semantics of Yampa which allows us to implement important concepts of ABMS:

state-chart: agents are at all time of their life-cycle in one state and can switch between multiple states using transitions 
timed transitions: transition to another state/behaviour happens at a discrete time
rate transitions: transition happens with a given rate
message transition: transition upon receiving a given message 

\subsection{Agents as Signals}
Due to the underlying nature and motivation of Functional Reactive Programming (und im speziellen) Yampa, Agents can be seen as Signals which is generated and consumed by a Signal-Function which is the behaviour of an Agent. If an Agent does not change the OUTPUT-signal is constant, if the agent changes e.g. by sending a message, changing its state,... the OUTPUT signal changes. A dead agent has no signal at all.

\subsection{Time-Sampling}
sampling rate depends on the transition times \& rates of the model. when e.g. the contact rate is 5 then the sampling dt should be below 0.2

\subsection{System Dynamics}
can emulate system dynamics due to the parallel update-strategy and continuous time-flow semantics

\subsection{Discrete Event Simulation}
DES in FrABMS? how easily can we implement server/queue systems? do they also look like a specification? potential problem: ordering of messages is not guaranteed by now

\subsection{Advantages}
advantages:
	- no side-effects within agents leads to much safer code
	- edsl for time-semantics
	- declarative style: agent-implementation looks like a model-specification
	- reasoning and verification
	- sequential and parallel
	- powerful time-semantics
	- arrowized programming is optional and only required when utilizing yampas time-semantics. if the model does not rely on time-semantics, it can use monadic-programming by building on the existing monadic functions in the EDSL which allow to run in the State-Monad which simplifies things very much
	- when to use yampas arrowized programing: time-semantics, simple state-chart agents 
	- when not using yampas facilities: in all the other cases e.g. SugarScape is such a case as it proceeds in unit time-steps and all agents act in every time-step
	- can implement System Dynamics building on Yampas facilities with total ease	
	- get replications for free without having to worry about side-effects and can even run them in parallel without headaches
	- cant mess around with time because delta-time is hidden from you (intentional design-decision by Yampa). this would be only very difficult and cumbersome to achieve in an object-oriented approach. TODO: experiment with it in Java - how could we actually implement this? I think it is impossible: may only achieve this through complicated application of patterns and inheritance but then has the problem of how to update the dt and more important how to deal with functions like integral which accumulates a value through closures and continuations. We could do this in OO by having a general base-class e.g. ContinuousTime which provides functions like updateDt and integrate, but we could only accumulate a single integral value.
	- reproducibility statically guaranteed
	- cannot mess around with dt
	- code == specification
	- rule out serious class of bugs
	- different time-sampling leads to different results e.g. in wildfire \& SIR but not in Prisoners Dilemma. why? probabilistic time-sampling?
	- reasoning about equivalence between SD and ABS implementation in the same framework
	- recursive implementations
	
	- we can statically guarantee the reproducibility of the simulation because: no side effects possible within the agents which would result in differences between same runs (e.g. file access, networking, threading), also timedeltas are fixed and do not depend on rendering performance or userinput	
	
\subsection{Disadvantages}
disadvantages:
	- performance is low
	- reasoning about performance is very difficult
	- very steep learning curve for non-functional programmers
	- learning a new EDSL
	- think ABMS different: when to use async messages, when to use sync conversations


[ ] important: increasing sampling freqzency and increasing number of steps so that the same number of simulation steps are executed should lead to same results. but it doesnt. why?
[ ] hypothesis: if time-semantics are involved then event ordering becomes relevant for emergent patterns. there are no tine semantics in heroes and cowards but in the prisoners dilemma
[ ] can we implement different types of agents interacting with each other in the same simulation ? with different behaviour funcs, digferent state? yes, also not possible in NetLogo to my knowledge. but they must have the same messages, emvironment 

[ ] Hypothesis: we can combine with FrABS agent-based simulation and system dynamics (this has been proved by example!)