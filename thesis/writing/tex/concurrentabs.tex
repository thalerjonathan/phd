\chapter{Concurrent ABS}
\label{ch:concurrent_abs}
Functional programming as in Haskell is well known and accepted as a remedy against the problems of imperative programming in implementing parallel software TODO: cite ?. The reason for it is clear: immutable data and explicit control of side-effects removes a large class of bugs due to data-conflicts, data-races, and blablabla TODO: we are claiming things here, which we need to clearly back up, also data-races ARE possible in Haskell! A fundamental benefit and strength of Haskell is, that it clearly distinguishes between parallelism and concurrency \cite{jones_tackling_2002} and it is very important for us to do so as well:

\begin{itemize}
	\item \textbf{Parallelism} - In parallelism, code runs in parallel without interfering with other code through shared data (references, mutexes, semaphores,...). An example is the function \textit{map :: (a $\rightarrow$ b) $\rightarrow$ [a] $\rightarrow$ [b]}, which maps each element of type \textit{a} to \textit{b} using the function \textit{(a $\rightarrow$ b)}. It is a pure function and thus no sharing of data either through some monadic context or through the function \textit{(a $\rightarrow$ b)} is possible. This allows to run it in parallel: each function evaluation \textit{(a $\rightarrow$ b)} could potentially be executed at the same time, if we had enough CPU cures. Whether it runs actually in parallel or not, has no influence on the outcome, it is not subject to any non-deterministic influences. Thus we identify parallelism with pure and deterministic execution of data-transformations (data-parallelism).
	
	\item \textbf{Concurrency} - In concurrency, code runs in parallel but can potentially interfere with other code through shared data (references, mutexes, semaphores, ...). An example are two threads, running in parallel, which share data through \textit{IORefs}. In concurrency there is no option: code has to run in parallel through the use of threads but now the outcome of the program very much depends on the ordering in which the threads are scheduled. This gives rise to very different access patterns to the shared data, with the potential for race conditions, dirty reads and so on... The challenge of implementing concurrent programs, is to write the program in a way that despite of these non-deterministic influences it is still a correctly working program. Thus we identify concurrency with impure and non-deterministic execution of imperative-style monadic command execution.
\end{itemize}

There is obvious potential for adding (data-)parallelism to ABS e.g. using data-parallel data-structures for the environment so cells can be updated in parallel, in time-driven ABS agents can be updated in parallel using parMap because they all act conceptually at the same time as shown already in Yampa \footnote{\url{https://www.reddit.com/r/haskell/comments/2jbl78/from_60_frames_per_second_to_500_in_haskell/}}.

Despite the potential for parallelism, in this chapter we focus on concurrency only and refer to the book \cite{marlow_parallel_2013} for an in-depth discussions of the mechanism for parallelism in Haskell. The reason for focusing on concurrency and leaving parallelism out is simple: the Ph.D. doesn't provide enough time to explore both in equal depth and the application of STM to implement concurrent ABS looks very much more interesting and challenging probably because it is also a complete novelty.

\chapter{Concurrency}
- concurrency and parallelism
	-> STM vs. lock based
	-> pure parallelism in ABS


STM paper
50\%

\section{Discussion}

\subsection{Other Models}
TODO: mention that we have also implemented other models, which also work without time-semantics (all agents make a move at discrete time-steps and do not really rely on some notion of time). 

\subsection{Time-Semantics}
The main reason for building our pure functional ABMS approach on top of Yampa was to leverage the powerful time-semantics of Yampa which allows us to implement important concepts of ABMS:

state-chart: agents are at all time of their life-cycle in one state and can switch between multiple states using transitions 
timed transitions: transition to another state/behaviour happens at a discrete time
rate transitions: transition happens with a given rate
message transition: transition upon receiving a given message 

\subsection{Agents as Signals}
Due to the underlying nature and motivation of Functional Reactive Programming (und im speziellen) Yampa, Agents can be seen as Signals which is generated and consumed by a Signal-Function which is the behaviour of an Agent. If an Agent does not change the OUTPUT-signal is constant, if the agent changes e.g. by sending a message, changing its state,... the OUTPUT signal changes. A dead agent has no signal at all.

\subsection{Time-Sampling}
sampling rate depends on the transition times \& rates of the model. when e.g. the contact rate is 5 then the sampling dt should be below 0.2

\subsection{System Dynamics}
can emulate system dynamics due to the parallel update-strategy and continuous time-flow semantics

\subsection{Discrete Event Simulation}
DES in FrABMS? how easily can we implement server/queue systems? do they also look like a specification? potential problem: ordering of messages is not guaranteed by now

\subsection{Advantages}
advantages:
	- no side-effects within agents leads to much safer code
	- edsl for time-semantics
	- declarative style: agent-implementation looks like a model-specification
	- reasoning and verification
	- sequential and parallel
	- powerful time-semantics
	- arrowized programming is optional and only required when utilizing yampas time-semantics. if the model does not rely on time-semantics, it can use monadic-programming by building on the existing monadic functions in the EDSL which allow to run in the State-Monad which simplifies things very much
	- when to use yampas arrowized programing: time-semantics, simple state-chart agents 
	- when not using yampas facilities: in all the other cases e.g. SugarScape is such a case as it proceeds in unit time-steps and all agents act in every time-step
	- can implement System Dynamics building on Yampas facilities with total ease	
	- get replications for free without having to worry about side-effects and can even run them in parallel without headaches
	- cant mess around with time because delta-time is hidden from you (intentional design-decision by Yampa). this would be only very difficult and cumbersome to achieve in an object-oriented approach. TODO: experiment with it in Java - how could we actually implement this? I think it is impossible: may only achieve this through complicated application of patterns and inheritance but then has the problem of how to update the dt and more important how to deal with functions like integral which accumulates a value through closures and continuations. We could do this in OO by having a general base-class e.g. ContinuousTime which provides functions like updateDt and integrate, but we could only accumulate a single integral value.
	- reproducibility statically guaranteed
	- cannot mess around with dt
	- code == specification
	- rule out serious class of bugs
	- different time-sampling leads to different results e.g. in wildfire \& SIR but not in Prisoners Dilemma. why? probabilistic time-sampling?
	- reasoning about equivalence between SD and ABS implementation in the same framework
	- recursive implementations
	
	- we can statically guarantee the reproducibility of the simulation because: no side effects possible within the agents which would result in differences between same runs (e.g. file access, networking, threading), also timedeltas are fixed and do not depend on rendering performance or userinput	
	
\subsection{Disadvantages}
disadvantages:
	- performance is low
	- reasoning about performance is very difficult
	- very steep learning curve for non-functional programmers
	- learning a new EDSL
	- think ABMS different: when to use async messages, when to use sync conversations


[ ] important: increasing sampling freqzency and increasing number of steps so that the same number of simulation steps are executed should lead to same results. but it doesnt. why?
[ ] hypothesis: if time-semantics are involved then event ordering becomes relevant for emergent patterns. there are no tine semantics in heroes and cowards but in the prisoners dilemma
[ ] can we implement different types of agents interacting with each other in the same simulation ? with different behaviour funcs, digferent state? yes, also not possible in NetLogo to my knowledge. but they must have the same messages, emvironment 

[ ] Hypothesis: we can combine with FrABS agent-based simulation and system dynamics (this has been proved by example!)