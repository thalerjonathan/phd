\chapter{Discussion}
\label{ch:discussion}
% starting point
This thesis started out by challenging the established views that \textit{"[..] object-oriented programming to be a particularly natural development environment for Sugarscape specifically and artificial societies generally [..]"} \cite{epstein_growing_1996} (p. 179) and that \textit{agents map naturally to objects} \cite{north_managing_2007}. As an alternative, a radical different approach to implementing ABS was proposed, using the pure functional programming paradigm. As language of choice, Haskell was motivated due to its  pure functional features, matureness and increasing relevance to real-world applications. 

% motivation
The conjecture was that by using Haskell, one can directly transfer the promises made by pure functional programming to ABS as well, directly gaining a few highly important benefits.  The relevance of each of these promises to ABS was already pointed out in the respective chapters and it is quite obvious that these benefits would clearly be of substantial value in ABS. The common baseline is that all those benefits support implementing ABS which are more likely to be correct, something of fundamental value in simulation.

\begin{enumerate}
	\item The static strong type system allows to remove a substantial number and class of bugs at run time and if one programs careful one can even guarantee that no bugs as in crashes or exceptions will occur at run time. This is particularly the case for purely computational problems, without \textit{IO} \footnote{Obviously IO is involved in all ABS, otherwise the results are not observable. By purely computational, we mean the lack of \textit{IO} in the agents, as fundamental part of the model specification, other than visualisation and exporting to an output file.}, as ABS almost always are. 
	
	\item Explicit handling and control of side effects delivers even more static guarantees at compile time and allows to deal with deterministic side effects (random-number streams, read-/write only contexts, state) in a referential transparent way. In combination with strong static typing this allows to reduce logical bugs, subject to the domain of the problem, by dramatically reducing available, valid operations on data - after all stateful applications are a fact, the challenge is how to deal with state. As ABS is an inherently stateful problem due to agents and the environment, this should increase the correctness of an ABS implementation as well. Further, this should allow to produce an implementation which is guaranteed to reproducible at compile time, where runs with same initial conditions \textit{will} result in same dynamics.
	
	\item Parallel and concurrent programming is claimed to be a lot easier, less painful and less error prone in functional programming in general and in Haskell in particular due to immutable data and the explicit handling of side effects. The concept of Software Transactional Memory, which allows to express a problem as a data-flow problem is highly promising. Besides, data-parallel programming promises to speedup code without the need for changing any of the logic or types. This seemed to offer a straightforward way of speeding up ABS implementations either through data parallelism or concurrency. This has always been quite difficult to achieve in traditional object-oriented ABS and pure functional programming seems to offer a solution.
	
	\item The data-centric declarative style, referential transparency and immutability of data makes testing substantially easier due to composability: functions can be easily tested in isolation from each other even if they involve side effects. This opens the door for randomised property-based testing which intuitively seemed to be a perfect match to test ABS implementations which are almost always stochastic in nature.
\end{enumerate}

The central question which needed to be answered first was \textit{how} ABS could be done pure functionally, as there didn't exist any research which offered a systematic solution to that problem. More specifically, it was unclear how to represent agents, how to express agent identity, local agents state, changing behaviour and interactions amongst agents and the environment. After all, this is straightforward in object-oriented programming due to method calls and mutable shared state encapsulated in objects. This thesis solution was to use arrowized FRP, both in the pure implementation of Yampa and the monadic version as in the library Dunai. Building on top of them allowed to implement pro-activity of agents, encapsulation of local agent state, an environment as shared mutable state and synchronous agent-interactions based on an event-driven approach. The central concept behind these approaches are Signal Functions (SF), generalised in Dunai to Monadic Stream Functions (MSF), which are implemented using closures and continuations, fundamental building blocks and concepts of pure functional programming. With theses techniques it became possible to implement time-driven models like the agent-based SIR as introduced in Chapter \ref{sec:sir_model} and the highly complex event-driven Sugarscape model as introduced in Chapter \ref{sec:sugarscape}. The fact that the developed concepts can manage the complexity of a full Sugarscape implementation is proof that they are suitable for most agent-based models, running on single machines.

The thesis then turned to the additional opportunities a pure functional approach would allow, with focus on parallelism and concurrency and property-based testing. By applying compositional parallelism it was possible to speed up the non-monadic time-driven ABS by a substantial factor, without losing static guarantees about reproducibility. The same was not possible with a monadic approach, which motivated the transition to concurrency using Software Transactional Memory, where a substantial speedup was achieved in a monadic, event-driven implementation sacrificing purity. Still, by limiting the side-effects to Software Transactional Memory only guarantees that the differences of the dynamics of repeated runs, come only from the side-effects of concurrency and nothing else.

The last step was to apply property-based testing to pure functional ABS, to test the hypothesis whether stochastic ABS and random property-based testing are a good match. Indeed, it has turned out to be a perfect match: it was possible to fully specify the agents behaviour of the time- and event-driven SIR model directly in code and verify them with QuickCheck. Further, other important techniques relevant for ABS like validation against specifications and comparisons of different implementations of the same model are easily done with property-based testing. Also, by showing how to perform hypothesis testing with property-based testing in the case of the Sugarscape model, we were able to show how to apply it to exploratory models which have no formal ground truth as well.

We argue that the thesis has successfully demonstrated in the respective chapters that all promises can successfully be transferred to pure functional ABS implementations in a reasonable, robust and maintainable way. Below we discuss respective points more in depth and give a holistic reflection and critique of our pure functional approach in general.

\section{Benefits}
TODO: not happy with that, it doesnt feel right, too short and detached, maybe can integrate it into the previous part.

We have discussed the benefits of pure functional ABS in the respective chapters already at length and recap them here in a broader context.

%After having established \textit{how} to do pure functional ABS in a robust and maintainable way the question was: does our approach leverage and deliver the promises of functional programming as well: what are its benefits? 

\subsection{Static guarantees}
Probably the biggest strength is that we can guarantee reproducibility at compile time: given identical initial conditions, repeated runs of the simulation will lead to same outputs. This is of fundamental importance in simulation and addressed in the Sugarscape model: \textit{"... when the sequence of random numbers is specified ex ante the model is deterministic. Stated yet another way, model output is invariant from run to run when all aspects of the model are kept constant including the stream of random numbers."} (page 28, footnote 16) - the pure functional approach can guarantee that \textit{at compile time}.

Further, we can enforce update semantics to a certain degree by dramatically restricting the state agents can manipulate. For example, in the time-driven approach the fact that the environment is provided as read-only and agents cannot interfere with other agents due to purity and complete lack of side effects, factually enforces the parallel update strategy: there is simply no way for agents to misbehave and violate semantics of updates by mutating state they are not allowed to.

%Summarising the benefits of static guarantees, we can directly relate them to the Verification \& Validation requirements of ABS as outlined by \cite{robinson_simulation:_2014}.
%
%\begin{itemize}
%	\item Modelling progress of time and variability - This is achieved by using Functional Reactive Programming (FRP), which allows the implementation of continuous and discrete time systems.
%
%	\item Fixing random number streams to allow simulations to be repeated under same conditions; Deterministic time-delta; Repeated runs lead to same dynamics;  Ruling out external sources of non-determinism / randomness - ensured by \textit{pure} functional programming and Random Monads.
%	
%	\item Relying only on past - This is guaranteed by \textit{Arrowized} FRP, which addressed exactly that problem of making it impossible for the programmer to define acausal systems, guaranteed at compile time.
%	
%	\item Bugs due to implicitly mutable state - This is reduced using pure functional programming.
%\end{itemize}

%%%%%%%%%%%
% PARALLELISM 
\subsection{Parallelism and concurrency}
Adding data parallelism is easy and often requires simply swapping out a data structure or library function against its parallel version. Concurrency, although still hard, is less painful to address and add in a pure functional setting due to immutable data and explicit side effects. Further, the benefits of implementing concurrent ABS based on STM has been shown at length in the respective chapter which underlines the strength of Haskell for concurrent ABS due to its strong guarantees about retry semantics. 

%%%%%%%%%%%%%
% TESTING
\subsection{Property-based testing}
Functional programming in general gives much more control and checking of invariants due to the explicit handling of effects. Together with the strong static type system, testing code is in full, explicit control over the functionality it checks. Property-based testing in particular is a perfect match for testing ABS due to the stochastic nature of both and because it supports convenient expressing of specifications. %Thus we can conclude that in a pure functional setting, testing is very expressive and powerful and supports working towards an implementation which is very likely to be correct.
For further insight into testing FRP we can directly leverage on the work of \cite{perez_testing_2017}. Another unique benefit of pure functional programming, \textit{equational reasoning} was not investigated in this thesis as it was beyond the focus of this Ph.D. We expect that this technique is applicable to parts of our approach as well but leave this for further research.

%\begin{enumerate}
%	\item Run-Time robustness by compile-time guarantees - by expressing stronger guarantees already at compile-time we can restrict the classes of bugs which occur at run-time by a substantial amount due to Haskell's strong and static type system.  This implies the lack of dynamic types and dynamic casts \footnote{Note that there exist casts between different numerical types but they are all safe and can never lead to errors at run-time.} which removes a substantial source of bugs.  Note that we can still have run-time bugs in Haskell when our functions are partial.
%	
%	\item Purity - By being explicit and polymorphic in the types about side-effects and the ability to handle side-effects explicitly in a controlled way allows to rule out non-deterministic side-effects which guarantees reproducibility due to guaranteed same initial conditions and deterministic computation. Also by being explicit about side-effects e.g. Random-Numbers and State makes it easier to verify and test.
%	
%	\item Explicit Data-Flow and Immutable Data - All data must be explicitly passed to functions thus we can rule out implicit data-dependencies because we are excluding IO. This makes reasoning of data-dependencies and data-flow much easier as compared to traditional object-oriented approaches which utilize pointers or references.
%	
%	\item Declarative - describing \textit{what} a system is, instead of \textit{how} (imperative) it works. In this way it should be easier to reason about a system and its (expected) behaviour because it is more natural to reason about the behaviour of a system instead of thinking of abstract operational details.
%	
%	\item Concurrency and parallelism - due to its pure and 'stateless' nature, functional programming is extremely well suited for massively large-scale applications as it allows adding parallelism without any side-effects and provides very powerful and convenient facilities for concurrent programming. The paper of (TODO: cite my own paper on STM) explores the use Haskell for concurrent and parallel ABS in a deeper way.
%\end{enumerate}

\section{Drawbacks}
\label{sec:drawbacks}
The initially hypothesised drawbacks of performance issues and agent interaction were confirmed in our research. We discuss them here more in detail, together with other drawbacks and propose solutions where applicable.

\subsection{Efficiency}
\label{sec:drawback_efficiency}
As mentioned already in the discussions of the respective chapters, currently the performance of our approaches does not come close to imperative implementations. There are two main reasons for it: First, functional programming is known for being slower due to higher levels of abstraction, which are bought by slower code in general and second, updates are the main bottleneck due to immutable data requiring to copying of the whole or subparts of a data structure in cases of a change. The first one is easily addressable through the use of data parallelism and concurrency as shown in Chapter \ref{ch:parallelism_ABS} and \ref{ch:concurrent_abs}. The second reason could potentially be addressed by the use of linear types \cite{bernardy_linear_2017}, which allow to annotate a variable with how often it is used within a function. From this a compiler could derive aggressive optimisations, potentially resulting in imperative-style performance but retaining the declarative nature of the code. We leave this for further research. Also, the use of Monad Transformer stacks has performance implications, which can be quite subtle. A possible optimisation we followed is the careful usage and reordering of lifts, using \texttt{lift (mapM ...)} instead of \texttt{mapM (lift ...)}, which potentially results in increased performance.

However, it was shown by various people \cite{kqr_competing_2017, stewart_haskell_2008, stolarek_haskell_2013} that Haskell does not necessarily have to be slow and that it is indeed possible to reach C speed in Haskell. The direction to do this is using the worker/wrapper transformation \cite{gill_worker/wrapper_2009}, a clever combination of techniques with strict \texttt{foldl'} and data declaration with strictness annotation instead of lazy tuples and Stream Fusion \cite{coutts_stream_2007, mainland_haskell_2013}. The problem is, of course, that to apply these techniques one needs to have deep knowledge of Haskell and its subtle details of lazy evaluation, making this a highly non-trivial task. Another problem is that those techniques seem only applicable in the context of a tight loop, which crunches numbers of a list, thus it is not directly applicable in our case as we are clearly bound by the effectful computations: MSF and Monad Transformers are the limiting factor, not inner loops.

Concluding we can say that the current performance makes our approach not very attractive for real-world use \textit{at the moment}. Also, the fact that the sequential object-oriented implementation of the SIR model seems to outperform the concurrent and parallel implementations as well, seems to question our motivation as to why we are using pure functional programming and parallel computation at all. Still, the bad performance results do not invalidate our research, as this thesis aim is not the development of high-performance pure functional implementations, but rather exploring concepts of ABS in pure functional programming. Thus, this work is seen as a first step which needs to be developed further into something to be used in the real world. We hypothesise that it should be possible for our pure functional approach to come considerably closer to imperative performance, ultimately making it more applicable for real-world usage. We leave a deeper investigation of this problem for further research.

\subsection{Space Leaks}
Haskell is notorious for its memory leaks due to lazy evaluation: data is only evaluated when required. Even for simple programs, one can be hit hard by a serious space leak where thunks - unevaluated code pieces - build up in memory until they are needed, leading to dramatically increased memory usage. It is no surprise that our highly complex Sugarscape implementation initially suffered severely from space leaks, accumulating about 40 MBytes/second. In simulation this is a big issue, threatening the value of the whole implementation despite its other benefits. Due to the fact that simulations might run for a (very) long time or conceptually forever, one must make absolutely sure that the memory usage stays within reasonable bounds. As a remedy, Haskell allows us to add so-called strictness pragmas to code modules, which force strict evaluation of all data even if it is not used. %Carefully adding this conservatively, file-by-file applying other techniques of forcing evaluation removed most of the memory leaks.

Another memory leak was caused by selecting the wrong data structure for the environment, for which we initially used an immutable array. The problem is that in the case of an update the whole array is copied, causing memory leaks \textit{and} a performance problem. We replaced it by an \texttt{IntMap} which uses integers as key and is internally implemented as a radix tree which allows for very fast lookups and inserts because whole sub-trees can be reused.

\subsection{Agent Interactions}
Synchronous, direct agent interactions \textit{do} work in Haskell but they are cumbersome to get right when building from scratch. Furthermore, as we pointed out in Chapter \ref{ch:concurrent_abs}, it seems that our approach to synchronous direct agent interactions is not applicable to concurrency with STM. 

This leads to the fundamental conclusion that in models which require complex agent interactions in a potentially concurrent environment, we are hitting the limits of our pure functional approach. The reason for it is that we have a conceptual mismatch, as in such a setting, agents are more naturally represented using the Actor Model. The Actor Model, a model of concurrency, was initially conceived by Hewitt in 1973 \cite{hewitt_universal_1973} and refined later on \cite{hewitt_what_2007}, \cite{hewitt_actor_2010}. It was a major influence in designing the concept of agents and although there are important differences between actors and agents there are important similarities, thus the idea to use actors to build ABS comes quite natural.
An actor is a uniquely addressable entity which can do the following \textit{in response to a message}:
\begin{itemize}
	\item Send an arbitrary number of messages to other actors.
	\item Create an arbitrary number of actors.
	\item Define its own behaviour upon reception of the next message.
\end{itemize}

When comparing this definition to the one of agents we give in Chapter \ref{sec:method_abs}, it is clear that the Actor Model was quite influential to the development of the concept of agents in ABS, which borrowed it initially from Multi-Agent Systems \cite{wooldridge_introduction_2009}. Technically, it emphasises message-passing concurrency with shared-nothing semantics (no implicitly shared state through side effects between agents), which maps nicely to functional programming concepts.

Indeed, the programming model of actors \cite{agha_actors:_1986} was the inspiration for the functional programming language Erlang, thus we argue that a true concurrent actor approach like Erlang is substantially more natural and much more performant especially in a concurrent setting. Furthermore, we hypothesise that actor based ABS implementations might have a bright future as ABS tends to develop towards larger and larger, distributed, always-online simulations, for which Erlang is arguably perfectly suited. We have prototyped highly promising concurrent event-driven SIR and Sugarscape implementations in Erlang supporting our hypothesis. However, an in-depth discussion is beyond the scope of this thesis and we leave this topic for further research.

%This makes testing easier and also opens the way for property-based testing which is available in Erlang as well where it even allows to detect race conditions \cite{claessen_finding_2009}. 

%erlang processes can implement everything objects can but in a referential transparent and pure functional way: encapsulation, polymorphism, identity, message passing, even inheritance (which you wouldnt want to do)

%difference of erlang to objects is that although it encapsulate state it cannot be accessed at the same time but only through the message passing Interface with one message at a time. this means that state is not really shared and protected against mutation - the process is in full control. it is simply a function which captures the full state of the process in an immutable way: to change the state a recursive call needs to be done. so in the end although it seems conceptually related its technically difference is fundamental importance.

%my mistake was to confuse the concept of objects with their implementation. i was too focused on the drawbacks of e.g. java objects that i forgot that i was critisising its IMPLEMENTATION. the Original idea of alan kays objects IS a deep and strong idea, though it differes substantially from java objects, erlang comes closest. thus the concept is important and valid but different implementations have different benefits and drawbacks. 

\subsection{Productivity and Learning Curve}
A case study in \cite{hanenberg_experiment_2010} hints that simply by switching to a static type system alone does not gain anything and can even be detrimental. To be useful, it needs to have a certain level of abstractions like Haskells' type system. Although such case studies have to be taken with care, there is also some truth in it: working in a statically strong type system prevents the developer from moving quickly and making quick changes. This can be both a benefit and a drawback: in general, it prevents one from breaking changes which show up at compile time. At the same time, the whole program is much more rigid and a proper structure needs to be thought out and designed often up-front, slowing down the process. However, it is a contribution of this thesis that it outlines exactly these structures within ABS, so that implementers who want to use the same approach do not have to reinvent the wheel.

A more severe problem is that pure functional programming, especially Haskell, is regarded as hard to learn with a steep learning curve, putting a high barrier to implementers picking up a pure functional approach to ABS. Thus, the lack of broad availability of Haskell expertise can be enough to pose a serious drawback even if the approach of this thesis seems to be desirable in a project.

\chapter{The Gintis Case}
\label{ch:gintis_case}
TODO UNFINISHED OPTIONAL CHAPTER

In \cite{axelrod_chapter_2006} Axelrod reports the vulnerability of ABS to misunderstanding. Due to informal specifications of models and change-requests among members of a research-team bugs are very likely to be introduced. He also reported how difficult it was to reproduce the work of \cite{axelrod_convergence_1995} which took the team four months which was due to inconsistencies between the original code and the published paper. The consequence is that counter-intuitive simulation results can lead to weeks of checking whether the code matches the model and is bug-free as reported in \cite{axelrod_advancing_1997}.
The same problem was reported in \cite{ionescu_dependently-typed_2012} which tried to reproduce the work of Gintis \cite{gintis_emergence_2006}. In his work Gintis claimed to have found a mechanism in bilateral decentralized exchange which resulted in walrasian general equilibrium without the neo-classical approach of a tatonement process through a central auctioneer. This was a major break-through for economics as the theory of walrasian general equilibrium is non-constructive as it only postulates the properties of the equilibrium \cite{colell_microeconomic_1995} but does not explain the process and dynamics through which this equilibrium can be reached or constructed - Gintis seemed to have found just this process. Ionescu et al. \cite{ionescu_dependently-typed_2012} failed and were only able to solve the problem by directly contacting Gintis which provided the code - the definitive formal reference. It was found that there was a bug in the code which led to the "revolutionary" results which were seriously damaged through this error. They also reported ambiguity between the informal model description in Gintis paper and the actual implementation.
This lead to a research in a functional framework for agent-based models of exchange as described in \cite{botta_functional_2011} which tried to give a very formal functional specification of the model which comes very close to an implementation in Haskell.
This was investigated more in-depth in the thesis by \cite{evensen_extensible_2010} who got access to Gintis code of \cite{gintis_emergence_2006}. They found that the code didn't follow good object-oriented design principles (all was public, code duplication) and - in accordance with \cite{ionescu_dependently-typed_2012} - discovered a number of bugs serious enough to invalidate the results. This reporting seems to confirm the above observations that proper object-oriented programming is hard and if not carefully done introduces bugs.
The author of this text can report the same when implementing \cite{epstein_growing_1996}. Although the work tries to be much more clearer in specifying the rules how the agents behave, when implementing them still some minor inconsistencies and ambiguities show up due to an informal specification.
The fundamental problems of these reports can be subsumed under the term of verification which is the checking whether the implementation matches the specification. Informal specifications in natural language or listings of steps of behaviour will notoriously introduce inconsistencies and ambiguities which result in wrong implementations - wrong in the way that the \textit{intended} specification does not match the \textit{actual} implementation. To find out whether this is the case one needs to verify the model-specification against the code. This is a well established process in the software-industry but has not got as much attention and is not nearly as well established and easy in the field of ABS as will become evident in the literature-review.
As ABS is almost always used for scientific research, producing often break-through scientific results as pointed out in \cite{axelrod_chapter_2006}, these ABS need to be \textit{free of bugs}, \textit{verified against their specification}, \textit{validated against hypotheses} and ultimately be \textit{reproducible}. One of the biggest challenges in ABS is the one of validation. In this process one needs to connect the results and dynamics of the simulation to initial hypotheses e.g. \textit{are the emergent properties the ones anticipated? if it is completely different why?} It is important to understand that we always \textit{must have} a hypothesis regarding the outcome of the simulation, otherwise we leave the path of scientific discovery. We must admit that sometimes it is extremely hard to anticipate \textit{emergent patterns} but still there must be \textit{some} hypothesis regarding the dynamics of the simulation otherwise we drift off into guesswork.

In the concluding remarks of \cite{axelrod_chapter_2006} Axelrod explicitly mentions that the ABS community should converge both on standards for testing the robustness of ABS and on its tools. However as presented above, we can draw the conclusion that there seem to be some problems the way ABS is done so far. We don't say that the current state-of-the-art is flawed, which it is not as proved by influential models which are perfectly sound, but that it always contains some inherent danger of embarrassing failure.

Discuss my developed techniques to the Gintis paper (and its follow ups: the Ionescu paper \cite{botta_functional_2011} and a Masterthesis \cite{evensen_extensible_2010} on it). Answer the following:

\begin{enumerate}
	\item Do the techniques transfer to this problem and model? 
	
	\item Could pure functional programming have prevented the bugs which Gintis made? 
	
	\item Would property-based tests have been of any help to preven the bugs?
	
	\item Could dependent and / or types have prevented the bugs which Gintis made? 
	
	\item How close is our (dependently typed) implementation to Ionescus functional specification? 
	
	\item When having Cezar Ionescu as external examiner, this chapter will be of great influence as it deals heavily with his work.

\end{enumerate}

TODO: my hypothesis is that with a clean and rigorous pure functional implementation it would have been more likely to spot the bug as it would have been stated more explicitly but it would not be guaranteed to be avoided - the same is true for dependent types unless one focuses on getting this bit explicitly right but that would be unfair comparison. however, i hypothesize that with an in-depth property-based testing he could have avoided / found that mistake - and he should have done in-depth property-based testing (verification and validation) due to the fundamental importance of his undertaking and the implications of a positive outcome

Not yet started, need to implement it but there exists code for it already (gintis and java implementations)

%
%after re-reading ionescu paper: too complex and out of scope, but ionescu work more directly applicable in a pure functional implementation than in e.g. c++ (that was what they used).
%
%we base our implementation on the existing gintis code from https://people.umass.edu/gintis/ 
%also we make use of the \cite{evensen_extensible_2010} on gintis work which revealed a few bugs
%
%NOTE: my hypothesis is that just by having used our pure functional approach would NOT have prevented gintis to have made the bugs as reported in the masterthesis \cite{evensen_extensible_2010} because they seemed to be like copy-paste bugs. Only rigorous code-testing (unit- / property-based) would have probably revealed these problems.
%

\subsection{Agent Based Computational Economics}
For many models, our techniques introduced in Part II are too powerful and a much simpler approach would suffice to implement it. In general too much power should always be avoided (at least in programming and software engineering) because with much power comes much responsibility: more power requires to pay more attention to details and thus there is more potential to make mistakes. Thus we should always look for the technique with minimal power, which solves our problem sufficiently.

A very important field, which picked up ABS in recent years is economics. The field of economics is an immensely vast and complex one with many facets to it, ranging from firms, to financial markets to whole economies of a country \cite{bowles_understanding_2005}. Today its very foundations rest on rational expectations, optimization and the efficient market hypothesis. The idea is that the macroeconomics are explained by the micro foundations \cite{colell_microeconomic_1995} defined through behaviour of individual agents. These agents are characterized by rational expectations, optimizing behaviour, having perfect information, equilibrium \cite{focardi_is_2015}.
This approach to economics has come under heavy critizism in the last years for being not realistic, making impossible assumptions like perfect information, not being able to provide a process under which equilibrium is reached \cite{kirman_complex_2010} and failing to predict crashes like the sub-prime mortgage crisis despite all the promises - the science of economics is perceived to be detached from reality \cite{focardi_is_2015}. 
ACE is a promise to repair the empirical deficit which (neo-classic) economics seem to exhibit by allowing to make more realistic, empirical assumptions about the agents which form the micro foundations. The ACE agents are characterized by bounded rationality, local information, restricted interactions over networks and out-of-equilibrium behaviour \cite{farmer_economy_2009}. 
Works which investigate ACE as a discipline and discuss its methodology are \cite{tesfatsion_agent-based_2002}, \cite{richiardi_agent-based_2007}, \cite{ballot_agent-based_2015}, \cite{blume_introduction_2015}.
%look into computable economics book: \url{http://www.e-elgar.com/shop/computable-economics}
Tesfatsion \cite{tesfatsion_agent-based_2017} defines ACE as \textit{[...] computational modelling of economic processes (including whole economies) as open-ended dynamic systems of interacting agents.}. She gives a broad overview \cite{tesfatsion_agent-based_2006} of ACE, discusses advantages and disadvantages and giving the four primary objectives of it which are:

\begin{enumerate}
	\item Empirical understanding: why have particular global regularities evolved and persisted, despite the absence of centralized planning and control?
	\item Normative understanding: how can agent-based models be used as laboratories for the discovery of good economic designs?
	\item Qualitative insight and theory generation: how can economic systems be more fully understood through a systematic examination of their potential dynamical behaviours under alternatively specified initial conditions?
	\item Methodological advancement: how best to provide ACE researchers with the methods and tools they need to undertake the rigorous study of economic systems through controlled computational experiments?
\end{enumerate}

It is important to understand, that ACE utilises ABS different than the social sciences do. The latter one focuses more on agent-interactions, where in ACE the rational and non-rational actions of individual agents are more important. Thus in many ACE models, the full power of the techniques introduced in Part II is not required. More specifically, agents of ACE models tend to have much simpler state, behave often in only one specific way, don't use synchronised agent-interactions and are very rarely located in a spatial environment but focus more on network connections \cite{wilhite_economic_2006, glasserman_contagion_2015} or avoid the notion of connectivity altogether.

\medskip

To investigate this point more in-depth we implemented \footnote{Freely available at \url{https://github.com/thalerjonathan/zerointelligence}} a simulation with so called Zero Intelligence traders \cite{gode_allocative_1993}, inspired by an implementation in Python \footnote{\url{http://people.brandeis.edu/~blebaron/classes/agentfin/GodeSunder.html}}. We don't go into any technical detail here but the implementation drives the main points home:

\begin{itemize}
	\item Even though it is an agent-based model and there is a clear notion of agents in the Python code, where they are represented as objects, the agents are extremely simple. They are characterised by a single floating-point value, identifying how much value they attribute to an asset. Their behaviour is also very simple and does not change over time: they always bid randomly within their profit range. Thus we do \textit{not} implement agents as MSFs in this case but represent them indeed only through a \textit{Double} value, reducing the complexity of the implementation considerably.

	\item There are no direct agent-interactions. Although agents trade with each other, this happens through a central authority (the simulation kernel), which acts like a market with a limit order book. This reduces the complexity of the implementation considerably because there is no need for the full approach of synchronised direct agent-interactions. We could have implemented it in that way but that would have only increased complexity through the use of a quite powerful technique, which is actually not really needed because the same effect can be achieved in much simpler terms.

	\item There is no environment whatsoever and a fully connected network is implicitly assumed because each agent can trade with all other agents. This implies that the full technique of applying an environment is not necessary, which makes the simulation a lot less complex. Still adding an environment e.g. a network would be quite simple and does not require any monadic code as the network information can be made read-only in the way as we do in Chapter \ref{sec:adding_env}.
	
	\item The only side-effect necessary in this simulation is to draw random-numbers. By fixing the seed, repeated runs of initial conditions will always lead to same output, which is guaranteed at compile time. This was already shown in Part II and is a direct consequence of Haskells type-system and explicit way of dealing with effects.	 Further, we focused on keeping as much code \textit{pure} as possible thus splitting code which does not require random numbers into pure functions and only having the basic structure of the implementation running in the Rand Monad. This makes testing and reasoning considerably easier than running everything in the Rand Monad.
\end{itemize}

We are very well aware that this simple example is only one of many ACE models but even though it implements very simple \textit{zero} intelligence agents, it shows that ABS in Haskell does not need to be as complex as the use-cases in Part II - on the contrary, ABS implementations can be very concise and highly performant in Haskell.

\subsection{Gintis Bilateral Bartering}

\chapter{The structure of ABS computation}
\label{ch:structure_abs_computation}

TODO UNFINISHED MANDATORY CHAPTER

The purpose of abstraction is not to be vague, but to create a new semantic level in which one can be absolutely precise. - Dijkstra, EWD340

generalising the structure of agent computation - with our case studies we explore them in a more practical / applied way and in this chapter we extract and distil the general concepts and abstractions behind agent computation: how can ABS, which is pure computation, can be seen structurally? This gives the ABS field for the first time a deeper understanding of the deeper structure of the computations behind agent-based simulation, which has so far always been more ad-hoc without a proper, more rigorous formulation. 

pure functional computation with effects can be seen as computations over some data-structure where the data-structure defines the structure of the computation as well e.g. monoids, applicatives, monads, traversable, foldable

Note that agent-based simulation is almost always entirely pure computation without the need for direct, synchronous user-interaction or impure IO. When IO is really needed we can keep purity by creating IO actions and pass them to the simulation kernel which executes them and communicates the result back if needed - in this case only the simulation kernel needs to run in IO monad but not the agents and the environment computations.

agentout as monoid with writer: solves the Problem of iteratively constructing it the output during an event.

BUT: isnt our approach similar to the early days IO of Haskell with continuations? if this is the case we should be able to get the direct method style by writing an agent monad?

NOTE: "And a closure is just a primitive form of object: the special case of an object with just one method." https://www.tedinski.com/2018/11/20/message-oriented-programming.html

- this is still research which needs to be done by reading the papers below and reflecting  and understanding on co-monads and my implementations in general.

- can we derive an agent-monad?

- what about comonads? read essence dataflow paper \cite{uustalu_essence_2006}: monads not capable of stream-based programming and arrows too general therefor comonads, we are using msfs for abs therefore streambased so maybe applicable to our approach/agents=comonads. comonads structure notions of context-dependent computation or streams, which ABS can be seen as of. this paper says that monads are not capable of doing stream functions, maybe this is the reason why i fail in my attempt of defining an ABS in idris because i always tried to implement a monad family. stopped at comonad section, continue from there. understand comonads: https://www.schoolofhaskell.com/user/edwardk/cellular-automata and https://kukuruku.co/post/cellular-automata-using-comonads/ and https://chshersh.github.io/posts/2019-03-25-comonadic-builders

- Conal Elliott has examined a comonadic formulation of functional reactive programming http://conal.net/blog/posts/functional-interactive-behavior

- comonads https://fmapfixreturn.wordpress.com/2008/07/09/comonads-in-everyday-life/

- comonads are objects very important and closely related http://www.haskellforall.com/2013/02/you-could-have-invented-comonads.html

- if conal elliott can make a comonadic formulatin of FRP and comonads are objects, then i guess i am very close to a pure functional representation of objects? pure functional objects?


independent of time-driven or event-driven, our agents are MSFs.

in fact i am deriving pure functional objects

- i have the feeling that co-algebras might be an underlying structure, which in CS come up in infinite streams - ABS can be seen as this where the agents are such streams with their output and potentially running for an infinite time, depending on the model. Ionescus thesis might reveal more information / might be an additional source on that.

In general it is easy to see why agents can not be represented by pure functions: they change over time. This is precisely what pure functions cannot do: they can't rely on some surrounding context / or on history - everything what they do is determined by their input arguments and their output. In general we have two ways of approaching this: we either have the agents changing data and behaviour internalised as we did in the previous chapters or we externalise it e.g. in the simulation kernel and provide all necessary information through arguments which was the case in the sugarscape environment.

- FREE MONADS % http://www.haskellforall.com/2012/07/purify-code-using-free-monads.html http://comonad.com/reader/2011/free-monads-for-less/, https://stackoverflow.com/questions/13352205/what-are-free-monads

\section{A Functional View}
Due to the fundamentally different approaches of FP, an ABS needs to be implemented fundamentally differently, compared to established OOP approaches. We face the following challenges:

\begin{enumerate}
	\item How can we represent an Agent, its local state and its interface?
	\item How can we implement direct agent-to-agent interactions?
	\item How can we implement an environment and agent-to-environment interactions? 
\end{enumerate}

\subsection{Agent representation}
The fundamental building blocks to solve these problems are \textit{recursion} and \textit{continuations}. In recursion a function is defined in terms of itself: in the process of computing the output it \textit{might} call itself with changed input data. Continuations are functions which allow to encapsulate the execution state of a program by capturing local variables (known as closure) and pick up computation from that point later on by returning a new function. As an illustratory example, we implement a continuation in Haskell which sums up integers and stores the sum locally as well as returning it as return value for the current step:

\begin{HaskellCode}
-- define the type of the continuation: it takes an arbitrary type a 
-- and returns a type a with a new continuation
newtype Cont a = Cont (a -> (a, Cont a))

-- an instance of a continuation with type a fixed to Int
-- takes an initial value x and sums up the values passed to it
-- note that it returns adder with the new sum recursively as 
-- the new continuation
adder :: Int -> Cont Int
adder x = Cont (\x' -> (x + x', adder (x + x')))

-- this function runs the given continuation for a given number of steps
-- and always passes 1 as input and prints the continuations output
runCont :: Int -> Cont Int -> IO ()
runCont 0 _ = return () -- finished
runCont n (Cont cont) = do -- pattern match to extract the function
  -- run the continuation with 1 as input, cont' is the new continuation
  let (x, cont') = cont 1
  print x
  -- recursive call, run next step
  runCont (n-1) cont'

-- main entry point of a Haskell program
-- run the continuation adder with initial value of 0 for 100 steps 
main :: IO ()
main = runCont 100 (adder 0)
\end{HaskellCode}

We implement an agent as a continuation: this lets us encapsulate arbitrary complex agent-state which is only visible and accessible from within the continuation - the agent has exclusive access to it. Further, with a continuation it becomes possible to switch behaviour dynamically e.g. switching from one mode of behaviour to another like in a state-machine, simply by returning new functions which encapsulate the new behaviour. If no change in behaviour should occur, the continuation simply recursively returns itself with the new state captured as seen in the example above.

The fact that we design an agent as a function, raises the question of the interface of it: what are the inputs and the output? Note that the type of the function has to stay the same (type \textit{a} in the example above) although we might switch into different continuations - our interface needs to capture all possible cases of behaviour. The way we define the interface is strongly determined by the direct agent-agent interaction. In case of Sugarscape, agents need to be able to conduct two types of direct agent-agent interaction: 1. one-directional, where agent A sends a message to agent B without requiring agent B to synchronously reply to that message e.g. repaying a loan or inheriting money to children; 2. bi-directional, where two agents negotiate over multiple steps e.g. accepting a trade, mating or lending. Thus it seems reasonable to define as input type an enumeration (algebraic data-type in Haskell, see example below) which defines all possible incoming messages the agent can handle. The agents continuation is then called every time the agent receives a message and can process it, update its local state and might change its behaviour.

As output we define a data-structure which allows the agent to communicate to the simulation kernel 1. whether it wants to be removed from the system, 2. a list of new agents it wants to spawn, 3. a list of messages the agent wants to send to other agents. Further because the agents data is completely local, it also returns a data-structure which holds all \textit{observable} information the agent wants to share with the outside world. Together with the continuation this guarantees that the agent is in full control over its local state, no one can mutate or access from outside. This also implies that information can only get out of the agent by actually running its continuation. It also means that the output type of the function has to cover all possible input cases - it cannot change or depend on the input. 

\begin{HaskellCode}
type AgentId    = Int
data Message    = Tick Int | MatingRequest AgentGender ... 
data AgentState = AgentState { agentAge :: Int, ... }             
data Observable = Observable { agentAgeObs :: Int, ... } 
data AgentOut   = AgentOut
  { kill       :: Bool
  , observable :: Observable
  , messages   :: [(AgentId, Message)] -- list of messages with receiver
  }
-- agent continuation has different types for input and output
newtype AgentCont inp out = AgentCont (in -> (out, AgentCont inp out))
-- taking the initial AgentState as input and returns the continuation
sugarscapeAgent :: AgentState -> AgentCont (AgentId, Message) AgentOut
sugarscapeAgent asInit = AgentCont (\ (sender, msg) -> 
  case msg of
    agentCont (sender, Tick t) = ... handle tick
    agentCont (sender, MatingRequest otherGender) = ... handle mating request)
\end{HaskellCode}

\subsection{Stepping the simulation}
The simulation kernel keeps track of the existing agents and the message-queue and processes the queue one element at a time. The new messages of an agent are inserted \textit{at the front} of the queue, ensuring that synchronous bi-directional messages are possible without violating resources constraints. The Sugarscape model specifies that in each tick all agents run in random order, thus to start the agent-behaviour in a new time-step, the core inserts a \textit{Tick} message to each agent in random order which then results in them being executed and emitting new messages. The current time-step has finished when all messages in the queue have been processed. See algorithm \ref{alg:stepping_simulation} for the pseudo-code for the simulation stepping.

\begin{algorithm}
\SetKwInOut{Input}{input}\SetKwInOut{Output}{output}
\Input{All agents \textit{as}}
\Input{List of agent observables}
shuffle all agents as\;
messageQueue = schedule Tick to all agents\;
agentObservables = empty List\;
\While{messageQueue not empty} {
  msg = pop message from messageQueue\;
  a = lookup receiving agent in as\;
  (out, a') = runAgent a msg\;
  update agent with continuation a' in as\;
  add agent observable from out to agentObservables\;
  add messages of agent at front of messageQueue\;
}
return agentObservables\;
\caption{Stepping the simulation.}
\end{algorithm}
\label{alg:stepping_simulation}

\subsection{Environment and agent-environment interaction}
The agents in the Sugarscape are located in a discrete 2d environment where they move around and harvest resources, which means the need to read and write data of environment. This is conveniently implemented by adding a State side-effect type to the agent continuation function. Further we also add a Random effect type because dynamics in most ABS in general and Sugarscapes in particular are driven by random number streams, so our agent needs to have access to one as well. All of this low level continuation plumbing exists already as a high quality library called Dunai, based on research on Functional Reactive Programming  \cite{hudak_arrows_2003} and Monadic Stream Functions \cite{perez_functional_2016,perez_extensible_2017}.

\section{When not}
%Thus it is not that implementing ABS with OO is wrong - it works reasonably well as a large number of industry strength libraries and frameworks demonstrate. It is more the \textit{missed potential} of a (pure) functional, data-centric approach: strong static type system with explicit controlled side-effects; parallel computation to speed up the simulation with very few changes but retaining static guarantees at compile time; STM to implement concurrent data-flow problems as actual data-flow problems without the need to resort to synchronisation primitives and cluttering the program logic with semantics for synchronisation and concurrency; Property-based testing for verification and validation of a data-centric approach which is central to all simulations; actor model concurrency in the case of Cloud Haskell and Erlang for agent-interaction centric models with a functional, data-centric core. 

We are very well aware that our approach has yet to reach maturity and prove itself over time in real simulation studies. Further, it is missing a mature and stable library yet (see future research) and the fact that there exist a number of the industry strength tool-kits and libraries (Repast, NetLogo, AnyLogic) and the widespread use and knowledge of object-oriented programming making ABS implementers readily available, makes object-oriented programming still the highly compelling approach to implement ABS. This allows for a quick and cheap implementation of low-impact and straightforward models where the need for correctness, reproducibility, verification and validation is not of primary concern. Also as outlined, performance in functional programming is still nowhere near object-oriented programming although that argument might get diminished by further research and the potential of using actor based concurrency like in Erlang to implement ABS. %Another benefit is that object-oriented programming as a modelling tool to a problem is still highly useful in the case of UML. %TODO:  discusses if and how peers object-oriented agent-based modelling framework can be applied to our pure functional approach. TODO: i need to re-read peers framework specifications / paper from the simulation bible book. Although peers framework uses UML and OO techniques to create an agent-based model, we realised from a short case-study with him that most of the framework can be directly applied to our pure functional approach as well, which is not a huge surprise, after all the framework is more a modelling guide than an implementation one. E.g. a class diagram identifies the main datastructures, their operations and relations, which can be expressed equally in our approach - though not that directly as in an oo language but at least the class diagram gives already a good outline and understanding of the required datafields and operations of the respective entities (e.g. agents, environment, actors,...). A state diagram expresses internal states of e.g. an agent, which we discussed how to do in both our time- and even-driven approach. A sequence diagram e.g. expresses the (synchronous) interactions between agents or with their environment, something for which we developed techniques in our event-driven approach and we discuss in depth there. 

\section{Do agents map naturally to objects?}
One of the initial motivation of this thesis was the claim of North et al. \cite{north_managing_2007} that \textit{agents map naturally to objects}. At the very end of this thesis we want to revisit this claim in the new light of our pure functional approach and finally answer the questions whether agents do map naturally to objects or not.

\medskip

To give a satisfactory answer, we first need to reexamine the abstractions used in our pure functional approach, where the fundamental building blocks are \textit{recursion} and \textit{continuations}. In recursion a function is defined in terms of itself: in the process of computing the output it \textit{might} call itself with changed input data. \textit{Continuations} in turn, are functions which allow to encapsulate the execution state of a program by capturing local variables (known as closure) to pick up computation from that point later on by returning a new function.

As an explanatory example, we implement a continuation in Haskell which sums up integers and stores the sum locally as well as returning it as return value of the current step. First, we define the type of a general continuation, which takes a polymorphic type \texttt{i} as input and returns a polymorphic type \texttt{o} as output together with a new continuation

\begin{HaskellCode}
newtype Cont i o = C (i -> (o, Cont i o))
\end{HaskellCode}

Then we implement an actual instance of a continuation with input and output types fixed to \texttt{Int}. It takes an initial value \texttt{x} and sums up the values passed to it. It returns \texttt{adder} with the new sum recursively as the new continuation.

\begin{HaskellCode}
adder :: Int -> Cont Int Int
adder x = C (\x' -> let s = x + x' in
                    (s, adder s))
\end{HaskellCode}

To run a continuation, we implement a function which runs a given continuation for a given number of steps and always passes \texttt{x} as input and prints the continuations output.

\begin{HaskellCode}
runCont :: Int -> Int -> Cont Int Int -> IO ()
runCont 0 x _ = return () 
runCont n x (C cont) = do 
  -- run the continuation with x as input, cont' is the new continuation
  let (x', cont') = cont x
  print x'
  runCont (n-1) x cont' 
\end{HaskellCode}

When actually running the continuation \texttt{adder} with an initial value of -1 for 10 steps and increments of 2, we get the following output:

\begin{HaskellCode}
> runCont 100 2 (adder (-1))
1
3
...
17
19
\end{HaskellCode}

This explanatory example should make it clear that we can encapsulate arbitrary complex state, which is only visible and accessible from within the continuation. Further, with a continuation it becomes possible to switch behaviour dynamically, like switching from one mode of behaviour to another as in a state machine, simply by returning new functions which encapsulate the new behaviour. If no change in behaviour should occur, the continuation simply recursively returns itself with the new state captured as seen in the example above.

In fact, Yampas signal functions (SF) and Dunais Monadic Stream Functions (MSF) are nothing else than such continuations: SF are pure, without a monadic context, as can be seen in the implementation of the supersampling in Chapter \ref{sub:timedriven_results}; MSFs have an additional monadic context, which makes it possible to execute effectful computations within the continuation as can be seen in the implementation of the simulation stepping MSF in Chapter \ref{sub:timedriven_thirdstep_impl}. 

\medskip

When looking closer at the example from above, it becomes clear that the continuation \texttt{adder} is non-terminating and is a potentially infinite structure, possible through lazy evaluation of Haskell, where the function \texttt{runCont} deconstructs / consumes / observes the output of this infinite structure step-by-step. This is related to the concepts of \textit{corecursion} which are an even deeper underlying theory to continuations in general and our approach in particular. Technically speaking, corecursion is the dual to recursion where instead of starting with a data structure and reducing it stepwise until a base case is reached, corecursion starts with an initial value and iterates it ad infinity, producing an infinite data structure as output, enabled through lazy evaluation. 
Indeed, our agents produce infinite streams as output, potentially running for infinite time as it is implemented in the time-driven approach and to a lesser extent in the event-driven SIR. Now it is also easy to see why agents are not represented by pure functions: they have to change over time, which is precisely what pure functions cannot do as they can not rely on a context or history of a system.

The fact that we represented pure functional agents as SF and MSF is thus no coincidence and did not fall from the sky: they are in fact representations of \textit{coalgebras}, which is the way to express dynamical systems in mathematics and in pure functional programming: \textit{"In general, dynamical systems with a hidden, black-box state space, to which a user only has limited access via specified (observer or mutator) operations, are coalgebras of various kinds"} \cite{jacobs_tutorial_1997}. Informally speaking a coalgebra is of the form $S \xrightarrow[\text{}]{c} $ \fbox{... S ...}, with a state space \textit{S}, a function \textit{c} and a structured output (the box), which also contains the original domain \textit{S} \cite{jacobs_introduction_2017}. This is precisely what we see in the recursive type definition of \texttt{Cont} above.

This sounds very much like agents and indeed, coalgebras have been used (amongst others) in Process Theory to model communicating processes, a topic closely related to Actors and ABS; and objects in object-oriented programming \cite{jacobs_coalgebras_2003}. It seems that we have found an underlying theory which connects both object-oriented programming and our pure functional ABS approach. This hints that it might be indeed the case that agents map naturally to objects, as we discuss below. We refer to \cite{jacobs_introduction_2017, jacobs_tutorial_1997} for a proper, formal introduction and discussion of coalgebras as it is beyond the scope of this thesis.

% ionescus thesis \cite{ionescu_vulnerability_2009} somewhere here? 

When following the concepts of continuations and coalgebras from above and the viewpoint that \textit{"... a closure is just a primitive form of object: the special case of an object with just one method."} \footnote{https://www.tedinski.com/2018/11/20/message-oriented-programming.html} then one way to look at an SF and MSF is to see them as very simple immutable objects with a single method - the continuation - following a shared-nothing semantics. 

Like in coalgebras and in continuations we have some internal state which can be altered through specified set of operations (events/inputs) and the effect can be observed through the output but not directly. This is particularly clear in the Sugarscape model, where agents have indeed a complex internal state, which changes only through events and is only observable through the output of data type \textit{SugAgentObservable} as a result of sending an event. Further, we added a notion of agent identity, a clearly specified agent interface, local agent state and synchronous direct agent interactions through tagless final.

\medskip

% 2. it seems that agents map to objects, but what are objects?
This interpretation and the fact that we seem to have achieved all the relevant concepts like encapsulation of local agent state and interactions purely functional, it seems that we indeed have to agree that agents do actually map naturally to objects. However, we argue we have to think objects in a much broader context than the one of existing object-oriented terminology as in the popular family of Java, C++ and Python. The fact that we can represent agents as objects also in a purely functional way, with sound underlying theories like coalgebras, leads us to the question, what actually constitutes objects and we have to be careful not to confuse the \textit{concept} of objects with their \textit{implementation} within a language.

% 3. what are objects
There does not exist a commonly agreed upon definition of objects and object-oriented programming but rather a bunch of ideas and concepts \footnote{\url{http://wiki.c2.com/?DefinitionsForOo}}. It is agreed that the original ideas of objects and object-oriented programming were conceived by Kristen Nygaard, the inventor of Simula 67, the first object-oriented language \cite{dahl_birth_2002} and Alan Kay, the inventor of SmallTalk, another pioneering object-oriented language in the early 70s \cite{kay_early_1993}. %Their ideas about OO where the following: \footnote{\url{http://wiki.c2.com/?AlanKaysDefinitionOfObjectOriented} and \url{http://wiki.c2.com/?NygaardClassification}}:

Kristen Nygaard identified object-oriented programming by \textit{"A program execution is regarded as a physical model, simulating the behaviour of either a real or imaginary part of the world."}, thus he puts the focus on the modelling aspect of the problem. Alan Kay claims to have coined the term \textit{object-oriented} and defines it in more technical terms: everything is an object; every object is an instance of a class; the class holds the shared behaviour for its instances; objects communicate by sending and receiving messages. Alan Kay puts a strong emphasis on sending and receiving messages, with a shared-nothing interpretation. This becomes especially clear in a quote attributed to him: \textit{"The big idea is "messaging" ... "I invented the term Object-Oriented and I can tell you I did not have C++ in mind."}.

\medskip

So we see that the original \textit{concepts} of objects and object-oriented programming vary considerably from how objects and object-oriented programming is \textit{implemented} today in the family of popular object-oriented programming languages like Java, C++ and Python. %The most substantial different to the original definition of Kay is that messages are not pure data - they do not follow a shared nothing semantics.
Our approach is \textit{one} answer to do that in a pure, strong statically typed language - Haskell. It can be seen as an object-centric approach, which \textit{implements} a very simple \textit{concept} of shared-nothing, immutable, pure functional objects.

%It is a fact that simulations are about consuming, processing and producing data. ABS being simulation methodology is no exception to that fact. Unfortunately, due to OO lack of rigour theoretical foundations, OO as it is used today is \textit{not} very good at representing and manipulating pure data and its data flow because of two things: \textit{mutable shared state} and explicitly associate data-types and functions(methods)/code/behaviour.

%FROM https://www.youtube.com/watch?v=QM1iUe6IofM&feature=youtu.be
%Inheritance is not relevant any more: it has come to a widely agreement amongst OO developers that inheritance should be avoided: https://www.javaworld.com/article/2073649/why-extends-is-evil.html . Note that we are speaking about subclassing not implementing an interface, which is something entirely different
%Polymorphism: is not unique to OO and exists in non-OO languages as well and plays a central role in Haskell (and ML languages). Further it is possible to implement polymorphic code in C
%Encapsulation: this is seen as the major strength of OO but unfortunately it does not work at a fine grained level of code in todays OO. The original idea was indeed great and it is no coincidence that my implementation ended up with a variation of that as well as Erlang: encapsulate state behind a public interface and interact with it through messages (TODO: fill in alan kay). The very central point of messages though was that they followed "shared nothing" semantics, meaning that no references or pointers could be contained in that message as this would immediately result in a violation of the public interface and ultimately breaks encapsulation. 
%OO dominates the industry since around mid 90s. There are varying opinions on that but a major influence to popularise OO was Java, which made its first appearance in 1996. Java was a much easier approach to OO than existing ones e.g. in C++ and VB: it abandoned multiple inheritance, introduced interfaces, was cross-platform, provided high quality libraries including a GUI framework (GUI programming was the way to go in the 90s until it got abandoned in 00s with the emergence of Web 2.0), C/C++ syntax made it easy to pick up, avoided header-files, abandoned pointers and memory management and added garbage collection which made applications a lot safer.

% TODO: need to discuss the problem of shared state. state per se is not necessarily a problem and ever program has state in some form. how explicit it is represented is often used as classification between different kind of paradigms e.g. it has been said that functional programming is stateless but that is obviously not true, state is all over the place but it is very very contained, well behaved and explicit. with shared mutable state this is not the case anymore and we get into the troube of data-dependencies and orderings. this is exactly what we encountered when having introduced a global environment in Sugarscape: although our state is referential transparent and pure functional, they way we used it is globally and we run in ordering issues.

% TODO: isnt shared state also a problem in erlang? after all we can send Pids around and interact with those processes as soon as another process has access to a Pid. In which way is it different to reference passing in OO? There seems to be no difference... so maybe the anti OO argument is not that strong after all and my argument is simply weak or wrong? 

%TODO: i REALLY need to find proper literature / research / evidence which shows the problematic nature of modern OO: mutable shared state which is tied to code. Inheritance and open recursion gives the rest. the problem is that deeply linking \textit{shared mutable} state to its code is the path to failure: abstraction breaks, concurrency and parallelism becomes hard and breaks abstraction, data-driven programming becomes difficult (although that got addressed by adding functional features). NOTE: my approach and erlang have state and behaviour as well but in our case the state is shared nothing and immutable (yes in Haskell we update the agents state but that happens ultimately through closures and continuation in a referential transparent way and still no state is shared between agents. the environment is an exception to some extent as agents can access it globally: this works but requires a specific ordering either through sequential access or STM. this is no different than in an erlang implementation of sugarscape: there needs to be some arbitration of concurrent access). TODO: isnt there some fundamental research on that issue out there?
% TODO: maybe these act as a starting point?
% https://www.yegor256.com/2016/08/15/what-is-wrong-object-oriented-programming.html
% https://dl.acm.org/citation.cfm?id=1806847
% https://web.cs.ucdavis.edu/~filkov/papers/lang_github.pdf "Most notably, it does appear that strong typing is modestly better than weak typing, and among functional languages, static typing is also somewhat better than dynamic typing" "We also find that functional languages are somewhat better than procedural languages" but modest effects
% https://www.javaworld.com/article/2073649/why-extends-is-evil.html
% READ extension problem paper
% READ Ted Kaminskis thesis

%This was by no means clear in the early-to-mid 1990s where the OO paradigm was seen as a silver bullet to the problems of programming: a whole software industry had to re-learn best practices, patterns \cite{gamma_design_1994} and how to avoid pitfalls and bad code \cite{fowler_refactoring:_2012}. Thus we cannot blame \cite{epstein_growing_1996} for advertising OO as the ways to implement ABS, at that time it seemed indeed to be the right thing to do. 

%The combination of both was exactly the sales pitch of OO for the last 20+ years. Unfortunately this combination leads to nasty bugs due to shared mutable state, deeply complex object hierarchies due to inheritance overuse which also fix behaviour at compile time, open recursion which in the end costs the potential for higher degree of correctness, ease of parallelism and concurrency and the use of property-based testing. Thus we need to separate both: what we need is immutable, shared-nothing state allowing for a data-centric approach \textit{and} an interaction mechanism which allows agents to communicate with each other.

% 4. implementation of objects: the problems: data-driven programming is difficult, not really encapsulating and shared mutable state makes concurrency and testing a lot harder. this sound as a contradition but it has been shown that despite objects claim they compose and enforce encapsulation, they do not.
% https://dl.acm.org/citation.cfm?doid=242224.242415

%So we see that the original \textit{concepts} of objects and object-oriented programming vary considerably from how objects and object-oriented programming is \textit{implemented} today in the family of popular object-oriented programming languages like Java, C++ and Python. The most substantial different to the original definition of Kay is that messages are not pure data - they do not follow a shared nothing semantics. This leads to the failure of objects to compose behaviour and encapsulate data properly \cite{bill_what_2017}, \cite{erkki_lindpere_why_2013}. Ironically, this has always been the main argument for advertising the use of object-oriented programming. The reason for this is that objects hide both \textit{mutation} and \textit{sharing through pointers or references} of object-internal data. Further, they expose multiple methods on how to operate on this encapsulated data. This makes data-flow mostly implicit due to the side effects on the mutable data which is globally scattered across objects. To deal with the problem of composability and implicit data-flow the seminal work \cite{gamma_design_1994} put forward the use of \textit{patterns} to organize objects and their interaction. Other concepts, trying to address the problems, were the SOLID principles and Dependency Injection. 
%
%% 4a this leads to an inherent difficulty to follow data-flow in an OO program and also makes it very difficult in concurrent settings as semantics of synchronisation "bleed" out of the object, breaking encapsulation. 
%Despite these advances in understanding the object-oriented programming paradigm and how to use it properly, the increased complexity leads to an inherent difficulty to express and follow data flow in an object-oriented program and exploit parallelism and concurrency due to mutable shared state. Even worse, concurrency breaks encapsulation of objects as well and prevents composing them. 
%
%The rise of functional concepts in object-oriented languages in the last years are a strong indication that object-oriented programming is lacking features which have existed in functional programming for decades. Java 8 added lambda expressions and functional style programming using \textit{map, fold, reduce, filter} which together with lambdas allow a data-flow oriented approach to computing. Python, which surges in popularity within the object-oriented family of languages, allows very data-flow centric and functional style of programming through lambda functions, list comprehensions and other functional features as it does not require programmers to stick to the object-oriented programming paradigm. Popularisation of JavaScript frameworks like React, Elm and Purescript, which emphasise a functional, data-flow driven approach of web-programming are another indicator. Thus it seems that functional concepts overcome the weakness of object-oriented programming to model explicit, immutable data flows which can be exploited towards easier parallelisation and concurrency.
%
%% 5. our approach is one of very simple, pure functional, immutable objects and we have shown that they indeed allow us to apply concurrency and property-based testing
%All these properties of explicit data flow and applicability of parallelism and concurrency are highly desirable when implementing simulations: it is a fact that simulations are data-centric, they are all about about consuming, processing and producing data and they have to do it fast and correct. ABS, being a simulation methodology, is no exception to that fact.
%
%The question is then why not use toolkits like Matlab or R - after all they are completely data-centric? This would be the other extreme, just like object-oriented programming is and we would run into difficulties as well. The point is that ABS is not purely data-centric either and is indeed richer: agents can interact with each other and with an environment. So we have a tension here: ABS is data-centric on the one hand, and interaction-centric on the other - can we combine both worlds? 
%
%%TODO: this tension between data and objects has its origins in the expression problem (TODO: cite the paper): https://www.tedinski.com/2018/03/06/more-on-the-expression-problem.html we want to have a general approach and thus abstraction:
%%in the sugarscape implementation we used a tagless final approach to effects, which is extensible in two dimensions: vertically, we can add new interpreters; horizontally, we can add new effectful operations.  
%%TODO: properly study
%%- https://www.tedinski.com/2018/03/06/more-on-the-expression-problem.html
%%- https://serokell.io/blog/2018/12/07/tagless-final
%%- https://jproyo.github.io/posts/2019-03-17-tagless-final-haskell.html
%%- The Expression Problem Revisited - Four new solutions using generics by Mads Torgersen
%
%%\section{Discussion}
%%Building on top of these concepts allowed us to implement pro-activity of agents, encapsulation of local agent state, a pro-active environment with shared mutable state and synchronous agent interactions based on an event-driven approach. With our case studies we have explored the structure of agent computation in a more practical / applied way and we have extracted and distilled the general concepts and abstractions behind agent computation and showed how pure ABS computation can be seen structurally. This should give the ABS field a deeper understanding of the structure of the computations behind ABS, which has so far always been more ad-hoc. %As already mentioned in the introduction, this becomes possible through pure functional programming because it treats computations in a more structural way, where data structures and types defines the structure of various kinds of computation e.g. Monoids, Applicatives, Monads, Comonads, Arrows,...