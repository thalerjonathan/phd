\chapter{Conclusion}
\label{ch:conclusion}
Thus, in the end coming back to the initial point that \textit{"agents map naturally to objects"} \cite{north_managing_2007}, we agree by saying \textit{"Yes they do, but we have to be precise what constitutes objects."}. Whether they are newtypes, tuples, (G)ADTs, pure functions, monads, comonads, arrows, SFs, MSFs, Actors or OO objects. They are all valid ways of representing agents with varying degree of abstraction, flexibility and power and they all come with their benefit and drawbacks which have to be clearly understood together with the problem to solve. This thesis simply added a promising new tool to the family of existing ones and only time will tell whether this tool is indeed as valuable as hinted in this thesis.

\section{Further Research}
In this section we briefly summarise the future research to undertake, other than the one already mentioned in various parts of this thesis, especially in the respective discussion sections.

\subsection{A general purpose library}
For pure functional ABS to ever reach a larger audience and acceptance, it will need a lot of support, especially in the form of a well designed, easy to use, robust, correct, high quality Haskell Library. Designing and developing such a library is research on its own as it needs to combine all the separate concepts introduced in this thesis into one code base. We hope that from this development further insights into ABS in general and pure functional ABS in particular will emerge, which can then be published to the community.

%generalise concepts explored into a pure functional ABS library in Haskell (called chimera)
%stm based concurrency for event-driven ABS using parallel DES. challenge is the time-warp implementation using monads. in general it should be easy to roll-back agents actions but with monads we have to be careful - for some monads rolling back is not neccessary e.g. rand and reader, for others it is, and for some it is impossible e.g. IO

\subsection{Actor based ABS}
One of the thesis fundamental conclusions is that the future of ABS lies in a functional, actor based concurrent approach. It required the full thesis to come to this conclusion thus this thesis only scratched the surface of the potential for actor based ABS as in Erlang or Cloud Haskell. We think that this topic deserves rigorous research as well as it is currently strongly neglected with only very few papers existing, which just scratch the surface. The idea seems compelling: functional programming with an actor language seems the way to do ABS in the future as it gives us almost all properties introduced in this thesis with the exception that agent interactions are a lot easier and naturally expressed.

\subsection{Dependent and linear types}
We see this thesis as an intermediary and necessary step towards dependent types for which we first needed to understand the potential and limitations of a non-dependently typed pure functional approach in Haskell. Dependent types are extremely promising in functional programming as they allow us to express stronger guarantees about the correctness of programs and go as far as allowing to formulate programs and types as constructive proofs, which must be total by definition \cite{thompson_type_1991, mckinna_why_2006, altenkirch_pi_2010}.

So far no research using dependent types in agent-based simulation exists at all. In our next paper we want to explore this for the first time and ask more specifically how we can add dependent types to our pure functional approach, which conceptual implications this has for ABS and what we gain from doing so. We plan on using Idris \cite{brady_idris_2013} as the language of choice as it is very close to Haskell with focus on real-world application and running programs as opposed to other languages with dependent types e.g. Agda and Coq.

We hypothesize that dependent types could help ruling out even more classes of bugs at compile time and encode invariants and model specifications on the type level, which implies that we don't need to test them using e.g. property-testing with QuickCheck. This would allow the ABS community to reason about a model directly in code. We think that a promising approach is to follow the work of \cite{brady_programming_2013, fowler_dependent_2014, brady_state_2016} in which the authors utilize GADTs to implement an indexed monad, which allows to implementation correct-by-construction software.

We have already started to outline a few core principles in the Appendix \ref{ch:equilibrium_totality}, but we conjecture that the true benefit is yet to be revealed.