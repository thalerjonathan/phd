\chapter{Evaluation}
\label{ch:evaluation}
% starting point
This thesis started out by challenging the established views that \textit{"[..] object-oriented programming to be a particularly natural development environment for Sugarscape specifically and artificial societies generally [..]"} \cite{epstein_growing_1996} (p. 179) and that \textit{agents map naturally to objects} \cite{north_managing_2007}. As a highly challenging alternative it proposed a radical different approach to implementing ABS, using the pure functional programming paradigm. As language of choice Haskell was motivated due to its matureness, increasing relevance to real-world applications and cutting edge pure functional programming concepts. 

% motivation
The motivation are the promises pure functional programming in Haskell comes with, which are relevant to ABS as well:
\begin{itemize}
	\item The static strong type system allows to remove substantial number and class of bugs at run-time and if one programs careful one can even guarantee that no bugs as in crashes or exceptions will occur at run-time. This is especially true for purely computational problems (without IO) as ABS almost always are.
	
	\item Explicit handling and control of side-effects delivers even more static guarantees at compile time and allows to deal with deterministic side-effects (random-number streams, read-/write only contexts, state) in a referential transparent way. In combination with strong static type this allows to reduce logical bugs (subject to the domain of the problem) by dramatically reducing available and valid operations on data - after all stateful applications are a fact, the challenge is how to deal with state. As ABS is full of state due to agents and the environment, this seems to be an obvious great thing to have to increase correctness of an implementation. Further, this should allow to produce an implementation which is guaranteed to reproducible (runs with same initial conditions \textit{will} result in same dynamics) at compile time.
	
	\item Parallel and concurrent programming is seen to be lot easier, less painful and less error prone in functional programming in general and in Haskell in particular due to immutable data and the explicit side-effects. The concept of Software Transactional Memory, which offers to formulate a problem as a data-flow problem like there was no concurrency seemed highly promising. Besides, data-parallel programming promises to speed up code without the need for changing any of the logic or types. This seemed to offer a straightforward way of speeding up ABS implementations either through data-parallelism or concurrency. This has always been quite difficult to achieve in traditional ABS due to OO and pure functional programming seems to offer a solution.
	
	\item The data-centric declarative style, referential transparency and immutability of data makes testing substantially easier due to composability: functions can be easily tested in isolation from each other even if they involve side-effects. This opens the door for (randomised) property-based testing which intuitively seemed to be a perfect match to test ABS implementations which are (almost) always stochastic in nature.
\end{itemize}

The relevance of each of these promises to ABS is pointed out in the respective promise and it is quite obvious that these benefits would clearly be of immense value in ABS, with the common baseline that they all support implementations of ABS to be more likely to be correct - something of fundamental value in simulation. The question was then how much of them can we actually reap if we follow a pure functional approach to ABS? To answer this we first needed to actually find out \textit{how} ABS could be done pure functionally, as there didn't exist any research which offered a systematic solution to that problem. This was answered in-depth in part II. Let shortly recap what we have done and put it into perspective:

% the problem
It was unclear how to represent agents: how can we express agent identity, local agents state, changing behaviour and interactions amongst agents and the environment? After all this is straightforward in OO due to mutable shared state encapsulated in objects.

% the answer
The solution was to use arrowized FRP both in the pure implementation of Yampa and the monadic version as in the library Dunai. The central concept behind these approaches are Signal Functions (SFs) (which were generalised in Dunai to Monadic Stream Functions MSFs ) which are implemented using closures and continuations, fundamental building blocks and concepts of pure functional programming. This allows SFs/MSFs to be seen as very simple \textit{immutable} objects with a single method following a \textit{shared nothing} semantics. 

% how: what we have achieved


%we now know how to engineer time- and event-driven ABS with complex state both in the agent and environment, main difficulty is direct agent-interaction (see macal classification into 4 types of ABS), compile-time guaranteed reproducibility, explicit handling of complex state (read only, read/write), concurrency explicit and limited to STM, very promising concurrency but direct agent-interactions main problem (erlang as a rescue?), main drawbacks: everything is explicit, performance

%The case-study strongly hints that our claim that pure FP has indeed its place in ABS is valid but we conclude that it is yet too early to pick up this paradigm for ABS. We think that engineering a proper implementation of a complex ABS model takes substantial effort in pure FP due to different techniques required. We believe that at the moment such an effort pays off only in cases of high-impact and large-scale simulations which results might have far-reaching consequences e.g. influence policy decisions. It is only there where the high requirements for reproducibility, robustness and correctness provided by FP are really needed. Still, we plan on distilling the developed techniques of the case-study into a general purpose ABS library. This should allow implementing models much easier and quicker, making the pure FP approach an attractive alternative for prototyping, opening the direction for a broad use of FP in the field of ABS.
% TODO: emphasise that we pulled of to implement the full sugarscape with a reasonable approach, this is strong evidence that it works

Probably the biggest strength is that we can guarantee reproducibility at compile time: given identical initial conditions, repeated runs of the simulation will lead to same outputs. This is of fundamental importance in simulation and addressed in the Sugarscape model: \textit{"... when the sequence of random numbers is specified ex ante the model is deterministic. Stated yet another way, model output is invariant from run to run when all aspects of the model are kept constant including the stream of random numbers."} (page 28, footnote 16) - we can guarantee that in our pure functional approach already \textit{at compile time}.

Refactoring is very convenient and quickly becomes the norm: guided by types (change / refine them) and relying on the compiler to point out problems, results in very effective and quick changes without danger of bugs showing up at run-time. This is not possible in Python because of its lack of compiler and types, and much less effective in Java due to its dynamic type-system which is only remedied through strong IDE support.

Adding data-parallelism is easy and often requires simply swapping out a data-structure or library function against its parallel version. Concurrency, although still hard, is less painful to address and add in a pure functional setting due to immutable data and explicit side-effects. Further, the benefits of implementing concurrent ABS based on Software Transactional Memory (STM) has been shown \cite{thaler_tale_2018} which underlines the strength of Haskell for concurrent ABS due to its strong guarantees about retry-semantics.

Testing in general allows much more control and checking of invariants due to the explicit handling of effects - together with the strong static type system, the testing-code is in full, explicit control over the functionality it tests. Property-based testing in particular is a perfect match to testing ABS due to the random nature in both and because it supports convenient expressing of specifications. Thus we can conclude that in a pure functional setting, testing is very expressive and powerful and supports working towards an implementation which is very likely to be correct.


\subsection{Verification and Correctness}
General there are the following basic verification \& validation requirements to ABS \cite{robinson_simulation:_2014}, which all can be addressed in our \textit{pure} functional approach as described in the paper in Appendix \ref{app:pfe}:

\begin{itemize}
	%\item Modelling progress of time - achieved using functional reactive programming (FRP)
	%\item Modelling variability - achieved using FRP
	\item Fixing random number streams to allow simulations to be repeated under same conditions - ensured by \textit{pure} functional programming and Random Monads
	\item Rely only on past - guaranteed with \textit{Arrowized} FRP
	\item Bugs due to implicitly mutable state - reduced using pure functional programming
	\item Ruling out external sources of non-determinism / randomness - ensured by \textit{pure} functional programming
	\item Deterministic time-delta - ensured by \textit{pure} functional programming
	\item Repeated runs lead to same dynamics - ensured by \textit{pure} functional programming
\end{itemize}

\begin{enumerate}
	\item Run-Time robustness by compile-time guarantees - by expressing stronger guarantees already at compile-time we can restrict the classes of bugs which occur at run-time by a substantial amount due to Haskell's strong and static type system.  This implies the lack of dynamic types and dynamic casts \footnote{Note that there exist casts between different numerical types but they are all safe and can never lead to errors at run-time.} which removes a substantial source of bugs.  Note that we can still have run-time bugs in Haskell when our functions are partial.
	\item Purity - By being explicit and polymorphic in the types about side-effects and the ability to handle side-effects explicitly in a controlled way allows to rule out non-deterministic side-effects which guarantees reproducibility due to guaranteed same initial conditions and deterministic computation. Also by being explicit about side-effects e.g. Random-Numbers and State makes it easier to verify and test.
	\item Explicit Data-Flow and Immutable Data - All data must be explicitly passed to functions thus we can rule out implicit data-dependencies because we are excluding IO. This makes reasoning of data-dependencies and data-flow much easier as compared to traditional object-oriented approaches which utilize pointers or references.
	\item Declarative - describing \textit{what} a system is, instead of \textit{how} (imperative) it works. In this way it should be easier to reason about a system and its (expected) behaviour because it is more natural to reason about the behaviour of a system instead of thinking of abstract operational details.
	\item Concurrency and parallelism - due to its pure and 'stateless' nature, functional programming is extremely well suited for massively large-scale applications as it allows adding parallelism without any side-effects and provides very powerful and convenient facilities for concurrent programming. The paper of (TODO: cite my own paper on STM) explores the use Haskell for concurrent and parallel ABS in a deeper way.
\end{enumerate}

TODO: haskell-titan
TODO: Testing and Debugging Functional Reactive Programming \cite{perez_testing_2017}

Static type system eliminates a large number run-time bugs.

TODO: can we apply equational reasoning? Can we (informally) reason about various properties e.g. termination?

Follow unit testing of the whole simulation as prototyped for towards paper.

in this we explore something new: property-based testing in ABS

\section{Generalising Research}
We hypothesize that our research can be transferred to other related fields as well, which puts our contributions into a much broader perspective, giving it more impact than restricting it just to the very narrow field of Agent-Based Simulation. Although we don't have the time to back up our claims with in-depth research, we argue that our findings might be applicable to the following fields at least on a conceptual level.

\subsection{Simulation in general}
We already showed in the paper \cite{thaler_pure_2019}, that purity in a simulation leads to repeatability which is of utmost importance in scientific computation. These insights are easily transferable to simulation software in general and might be of huge benefit there. Also my approach to dependent types in ABS might be applicable to simulations in general due to the correspondence between equilibrium \& totality, in use for hypotheses formulation and specifications formulation as pointed out in Chapter \ref{ch:dependent}. 

\subsection{System Dynamics}
\label{sub:generalising_system_dynamics}
we have done that already in the appendix

discuss pure functional system dynamics - correct by construction: benefits: strictly deterministic already at compile time, encode equations directly in code => correct by construction. Can serve as backend implementation of visual SD packages.

\subsection{Discrete Event Simulation}
pure functional DES easily possible with my developed synchronous messaging ABS
DES in FP: we doing part of it in event-driven SIR. this is a potential hint at how to achieve DES in pure FP: arrowized FRP seems to be perfect for it because it allows to express data-flow networks, which is exactly what DES is as well. The problem with pure FP is that it will be more involved to propagate events through the network especially when they don't originate from the source but e.g. after a time-out from a queue. Generally all the techniques are there as discussed in the event-driven chapter but it would be interesting to do research on how to achieve the same in DES. Because data-flow networks of DES generally don't change at runtime, they are fixed already at compile time - it would be interesting to see what dependent types could offer for additional compile-time guarantees.

PDES, should be  conceptually easil possible using STM, optimistic approach should be conceptually easier to implement due to persistent data-structures and controlled side-effects
 
\subsection{Recursive Simulation}
Due to the recursive nature of FP we believe that it is also a natural fit to implement recursive simulations as the one discussed in \cite{gilmer_recursive_2000}. In recursive ABS agents are able to halt time and 'play through' an arbitrary number of actions, compare their outcome and then to resume time and continue with a specifically chosen action e.g. the best performing or the one in which they haven't died. More precisely, an agent has the ability to run the simulation recursively a number of times where the number is not determined initially but can depend on the outcome of the recursive simulation. So recursive ABS gives each Agent the ability to run the simulation locally from its point of view to project its actions into the future and change them in the present. Due to controlled side-effects and referential transparency, combined with the recursive nature of pure FP, we think that implementing a recursive simulation in such a setting should be straight-forward.

Inspired by \cite{gilmer_recursive_2000}, add ideas about recursive simulation described in 1st year report and "paper". functional programming maps naturally here due to its inherently recursive nature and controlled side-effects which makes it easier to construct correct recursive simulations.
recursive simulation should be conceptually easier to implememt and more likely to be correct due to recursive Nature of haskell itself, lack of sideeffeccts and mutable data

\subsection{Multi Agent Systems}
The fields of Multi Agent Systems (MAS) and ABS are closely related where ABS has drawn much inspiration from MAS \cite{wooldridge_introduction_2009}, \cite{weiss_multiagent_2013}. It is important to understand that MAS and ABS are two different fields where in MAS the focus is more on technical details, implementing a system of interacting intelligent agents within a highly complex environment with the focus on solving AI problems.

Because in both fields, the concept of interacting agents is of fundamental importance, we expect our research also to be applicable in parts to the field of MAS. Especially the work on dependent types should be very useful there because MAS is very interested in correctness, verification and formally reasoning about a system and their agents, to show that a system follows a formal specifications.

\section{Drawbacks}
\label{sec:drawbacks}

\subsection{Space-Leaks}
discuss the problem (and potential) of lazy evaluation for ABS: can under some circumstances really increase performance when some stuff is not evaluated (see STM study) but mostly it causes problems by piling up unevaluated thunks leading to crazy memory usage which is a crucial problem in simulation. Using strict pragmas, annotations and data-structures solves the problem but is not trivial and involves carefully studying the code / getting it right from the beginning / and using the haskell profiling tools (which are fucking great at least). TODO: show the stats of memory usage

Haskell is notorious for its memory-leaks due to lazy evaluation: data is only evaluated when required. Even for simple programs one can be hit hard by a serious space-leak where unevaluated code pieces (thunks) build up in memory until they are needed, leading to dramatically increased memory usage. It is no surprise that our highly complex Sugarscape implementation initially suffered severely from space-leaks, piling up about 40 MByte / second. In simulation this is a big issue, threatening the value of the whole implementation despite its other benefits: because simulations might run for a (very) long time or conceptually forever, one must make absolutely sure that the memory usage stays somewhat constant. As a remedy, Haskell allows to add so-called strictness pragmas to code-modules which forces strict evaluation of all data even if it is not used. Carefully adding this conservatively file-by file applying other techniques of forcing evaluation removed most of the memory leaks.

% Another memory leak was caused by selecting the wrong data-structure for our environment, for which we initially used an immutable array. The problem is that in the case of an update the whole array is copied, causing memory leaks AND a performance problem. We replaced it by an IntMap which uses integers as key (mapping 2d coordinates to unique integers is trivial) and is internally implemented as a radix-tree which allows for very fast lookups and inserts because whole sub-trees can be re-used.


\subsection{Efficiency}
ordering of the transformers
when to run a transformer,
lazyness vs. strictness

The main drawback of our approach is performance, which at the moment does not come close to OO implementations. There are two main reasons for it: first, FP is known for being slower due to higher level of abstractions, which are bought by slower code in general and second, updates are the main bottleneck due to immutable data requiring to copy the whole (or subparts) of a data structure in cases of a change. The first one is easily addressable through the use of data-parallelism and concurrency as we have done in our paper on STM \cite{thaler_tale_2018}. The second reason can be addressed by the use of linear types \cite{bernardy_linear_2017}, which allow to annotate a variable with how often it is used within a function. From this a compiler can derive aggressive optimisations, resulting in imperative-style performance but retaining the declarative nature of the code.

\subsection{Productivity and learning curve}
A case study in \cite{hanenberg_experiment_2010} hints that simply by switching to a static typesystem alone does not gain anything and can even be detriment. It also needs to have a certain level of abstractions like Haskell type system does or even dependent types as in Idris. This also mean that there is a substantial learning curve to master when one wants to enter pure functional ABS.

\section{Alternatives}
\subsection{Haskell}
Freer Monads: \url{https://reasonablypolymorphic.com/blog/freer-monads/}. They aim to separate Definition from implementation by writing a domain-specific language using GADTs which are then interpreted. This allows to strictly separate implementation from specification, composes very well and thus is easier to test as parts can be easily mocked. Also Freer Monads free one from the order of effects imposed through Monad Transformers. In general Freer Monads seem to aim for the same abstraction what modern interface-based oop does.
Problem: Yes, freer monads are today somewhere around 30x slower than the equivalent mtl code. because its $O(n^2)$. ABS are not IO bound, so raw computation is all what counts and this is undoubtly worse with Freer monads. Given that we are already having problematic performance, we can't sacrifice even more. There seem to be a better encoding possible, which is about 2x slower than MTL: \url{https://reasonablypolymorphic.com/blog/too-fast-too-free/}. Still it might prove to be useful in other terms like proving correctness and then translating it, but how could we do it?  
ContT is Not an Algebraic Effect so it seems to be difficult to implement continuations in freer monads. Unfortunately this is what we really need as shown in Event driven ABS and generalising structure chapters.
Other criticism of Freer Monads are: \url{https://medium.com/barely-functional/freer-doesnt-come-for-free-c9fade793501} are: boilerplate code which though can be generated automatically by some libraries, performance when not IO based because the program is bascially a data-structure which is interpreted, concurrency seems to be tricky, 
TODO read: \url{https://reasonablypolymorphic.com/blog/freer-yet-too-costly/}

TODO: it seems that Final Tagless is another alternative. Look into it: \url{https://jproyo.github.io/posts/2019-03-17-tagless-final-haskell.html}

\subsection{Languages}
We precisely pointed out in the beginning of this thesis why we chose Haskell as language of choice. Obviously Haskell is not the only (pure) functional language and there exist a number of other alternatives which would be equally worth of systematic investigation of their use for ABS. Shortly we can conclude that the use of Haskell moves the nature of the structure of ABS computation into the light, together with compile-time guarantees, and benefits in testing and parallel implementations. Depending on each language though we get a very different direction:

\paragraph{LISP} Being the oldest functional programming language and the 2nd oldest high-level programming language ever created and still used by many people, LISP had to be considered in the beginning of the thesis. The language has the immense powerful feature of homoiconicity: data is code and code is data at the same time. This allows a LISP program to generate data-structures, which resemble valid LSIP code thus mutating its own code at runtime. This would give immense power to create powerful abstractions in terms of ABS. Unfortunately LISP is fully interpreted and has no types and is also impure, which would probably have led to very imperative, traditional approaches to ABS. Still, there exists research \cite{kawabe_nepi2programming_2000} which implements a MAS in LISP.

\paragraph{Erlang} The programming-model of actors \cite{agha_actors:_1986} was the inspiration for the Erlang programming language \cite{armstrong_erlang_2010}, which was created in the 1980's by Joe Armstrong for Eriksson for developing distributed high reliability software in telecommunications. The implication is that, the focus would shift immediately to the use of the actor model for concurrent interaction of agents through messages. The languages type-system is strong and dynamic and thus lacks type-checking at compile-time. Thus the structure of computation plays naturally no role because we cannot look at it from the abstract perspective as we can in Haskell. Purity can not be guaranteed and due to agents being processes concurrency is everywhere, and even though it is very tamed through shared-nothing messaging semantics, this implies that repeated runs with same initial conditions might lead to different results. Obviously we could avoid implementing agents as processes but then we basically sacrifice the very heart and feature of the language.
	
\paragraph{Scala} Scala is a multi-paradigm language, which also comes with an implementation of the actor-model as a library which enables to do actor-programming in the way of Erlang. It was developed in 2004 and became popular in recent years due to the increased availability of multi-core CPUs which emphasised the distributed, parallel and concurrent programming for which the actor-model is highly suited. There exists research on using Scala for ABS \cite{krzywicki_massively_2015, todd_multi-agent_nodate}. The benefit Scala has over Erlang is that it has type-checking at compile-time and is thus more robust, still it is impure due to side-effects and messages can contain references, thus violating the original shared-nothing semantics of Erlang.

\paragraph{F\#} Widely used in finance TODO 

\section{Actors}
\label{sec:actors}
TODO: this seems not to fit into the narrative here, maybe it fits into discussion part or further research

The Actor-Model, a model of concurrency, was initially conceived by Hewitt in 1973 \cite{hewitt_universal_1973} and refined later on \cite{hewitt_what_2007}, \cite{hewitt_actor_2010}. It was a major influence in designing the concept of agents and although there are important differences between actors and agents there are huge similarities thus the idea to use actors to build agent-based simulations comes quite natural. The theory was put on firm semantic grounds first through Irene Greif by defining its operational semantics \cite{grief_semantics_1975} and then Will Clinger by defining denotational semantics \cite{clinger_foundations_1981}. In the seminal work of Agha \cite{agha_actors:_1986} he developed a semantic mode, he termed \textit{actors} which was then developed further \cite{agha_foundation_1997} into an actor language with operational semantics which made connections to process calculi and functional programming languages (see both below). 

An actor is a uniquely addressable entity which can do the following \textit{in response to a message}
\begin{itemize}
	\item Send an arbitrary number (even infinite) of messages to other actors.
	\item Create an arbitrary number of actors.
	\item Define its own behaviour upon reception of the next message.
\end{itemize}

In the actor model theory there is no restriction on the order of the above actions and so an actor can do all the things above in parallel and concurrently at the same time. This property and that actors are reactive and not pro-active is the fundamental difference between actors and agents, so an agent is \textit{not} an actor but conceptually nearly identical and definitely much closer to an agent in comparison to an object. The actor model can be seen as quite influential to the development of the concept of agents in ABS, which borrowed it from Multi Agent Systems \cite{wooldridge_introduction_2009}. Technically, it emphasises message-passing concurrency with share-nothing semantics (no shared state between agents), which maps nicely to functional programming concepts.

There have been a few attempts on implementing the actor model in real programming languages where the most notable ones are Erlang and Scala. Erlang was created in 1986 by Joe Armstrong for Eriksson for developing distributed high reliability software in telecommunications. It implements light-weight processes, which allows to spawn thousands of them without heavy memory overhead. The language saw some use in implementing ABS with notable papers being \cite{di_stefano_using_2005, di_stefano_exat:_2007, varela_modelling_2004, sher_agent-based_2013, bezirgiannis_improving_2013}

Scala is a modern mixed paradigm programming language, which also allows functional programming and also incorporates a library for the actor model. It also saw the use in the implementation of ABS with a notable paper \cite{krzywicki_massively_2015} and ScalABM \footnote{https://github.com/ScalABM} which is alibrary for ABM in economics.

The paper of \cite{jennings_agent-based_2000} gives an excellent overview over the strengths and weaknesses of agent-based software-engineering, which can be directly applied to both Erlang and Scala.

Due to the very different approach and implications the actor model of concurrency implies, we don't explore it further and leave it for further research as it is beyond the focus of the thesis.

\chapter{The Gintis Case}
\label{ch:gintis_case}
TODO UNFINISHED OPTIONAL CHAPTER

In \cite{axelrod_chapter_2006} Axelrod reports the vulnerability of ABS to misunderstanding. Due to informal specifications of models and change-requests among members of a research-team bugs are very likely to be introduced. He also reported how difficult it was to reproduce the work of \cite{axelrod_convergence_1995} which took the team four months which was due to inconsistencies between the original code and the published paper. The consequence is that counter-intuitive simulation results can lead to weeks of checking whether the code matches the model and is bug-free as reported in \cite{axelrod_advancing_1997}.
The same problem was reported in \cite{ionescu_dependently-typed_2012} which tried to reproduce the work of Gintis \cite{gintis_emergence_2006}. In his work Gintis claimed to have found a mechanism in bilateral decentralized exchange which resulted in walrasian general equilibrium without the neo-classical approach of a tatonement process through a central auctioneer. This was a major break-through for economics as the theory of walrasian general equilibrium is non-constructive as it only postulates the properties of the equilibrium \cite{colell_microeconomic_1995} but does not explain the process and dynamics through which this equilibrium can be reached or constructed - Gintis seemed to have found just this process. Ionescu et al. \cite{ionescu_dependently-typed_2012} failed and were only able to solve the problem by directly contacting Gintis which provided the code - the definitive formal reference. It was found that there was a bug in the code which led to the "revolutionary" results which were seriously damaged through this error. They also reported ambiguity between the informal model description in Gintis paper and the actual implementation.
This lead to a research in a functional framework for agent-based models of exchange as described in \cite{botta_functional_2011} which tried to give a very formal functional specification of the model which comes very close to an implementation in Haskell.
This was investigated more in-depth in the thesis by \cite{evensen_extensible_2010} who got access to Gintis code of \cite{gintis_emergence_2006}. They found that the code didn't follow good object-oriented design principles (all was public, code duplication) and - in accordance with \cite{ionescu_dependently-typed_2012} - discovered a number of bugs serious enough to invalidate the results. This reporting seems to confirm the above observations that proper object-oriented programming is hard and if not carefully done introduces bugs.
The author of this text can report the same when implementing \cite{epstein_growing_1996}. Although the work tries to be much more clearer in specifying the rules how the agents behave, when implementing them still some minor inconsistencies and ambiguities show up due to an informal specification.
The fundamental problems of these reports can be subsumed under the term of verification which is the checking whether the implementation matches the specification. Informal specifications in natural language or listings of steps of behaviour will notoriously introduce inconsistencies and ambiguities which result in wrong implementations - wrong in the way that the \textit{intended} specification does not match the \textit{actual} implementation. To find out whether this is the case one needs to verify the model-specification against the code. This is a well established process in the software-industry but has not got as much attention and is not nearly as well established and easy in the field of ABS as will become evident in the literature-review.
As ABS is almost always used for scientific research, producing often break-through scientific results as pointed out in \cite{axelrod_chapter_2006}, these ABS need to be \textit{free of bugs}, \textit{verified against their specification}, \textit{validated against hypotheses} and ultimately be \textit{reproducible}. One of the biggest challenges in ABS is the one of validation. In this process one needs to connect the results and dynamics of the simulation to initial hypotheses e.g. \textit{are the emergent properties the ones anticipated? if it is completely different why?} It is important to understand that we always \textit{must have} a hypothesis regarding the outcome of the simulation, otherwise we leave the path of scientific discovery. We must admit that sometimes it is extremely hard to anticipate \textit{emergent patterns} but still there must be \textit{some} hypothesis regarding the dynamics of the simulation otherwise we drift off into guesswork.

In the concluding remarks of \cite{axelrod_chapter_2006} Axelrod explicitly mentions that the ABS community should converge both on standards for testing the robustness of ABS and on its tools. However as presented above, we can draw the conclusion that there seem to be some problems the way ABS is done so far. We don't say that the current state-of-the-art is flawed, which it is not as proved by influential models which are perfectly sound, but that it always contains some inherent danger of embarrassing failure.

Discuss my developed techniques to the Gintis paper (and its follow ups: the Ionescu paper \cite{botta_functional_2011} and a Masterthesis \cite{evensen_extensible_2010} on it). Answer the following:

\begin{enumerate}
	\item Do the techniques transfer to this problem and model? 
	
	\item Could pure functional programming have prevented the bugs which Gintis made? 
	
	\item Would property-based tests have been of any help to preven the bugs?
	
	\item Could dependent and / or types have prevented the bugs which Gintis made? 
	
	\item How close is our (dependently typed) implementation to Ionescus functional specification? 
	
	\item When having Cezar Ionescu as external examiner, this chapter will be of great influence as it deals heavily with his work.

\end{enumerate}

TODO: my hypothesis is that with a clean and rigorous pure functional implementation it would have been more likely to spot the bug as it would have been stated more explicitly but it would not be guaranteed to be avoided - the same is true for dependent types unless one focuses on getting this bit explicitly right but that would be unfair comparison. however, i hypothesize that with an in-depth property-based testing he could have avoided / found that mistake - and he should have done in-depth property-based testing (verification and validation) due to the fundamental importance of his undertaking and the implications of a positive outcome

Not yet started, need to implement it but there exists code for it already (gintis and java implementations)

%
%after re-reading ionescu paper: too complex and out of scope, but ionescu work more directly applicable in a pure functional implementation than in e.g. c++ (that was what they used).
%
%we base our implementation on the existing gintis code from https://people.umass.edu/gintis/ 
%also we make use of the \cite{evensen_extensible_2010} on gintis work which revealed a few bugs
%
%NOTE: my hypothesis is that just by having used our pure functional approach would NOT have prevented gintis to have made the bugs as reported in the masterthesis \cite{evensen_extensible_2010} because they seemed to be like copy-paste bugs. Only rigorous code-testing (unit- / property-based) would have probably revealed these problems.
%

\subsection{Agent Based Computational Economics}
For many models, our techniques introduced in Part II are too powerful and a much simpler approach would suffice to implement it. In general too much power should always be avoided (at least in programming and software engineering) because with much power comes much responsibility: more power requires to pay more attention to details and thus there is more potential to make mistakes. Thus we should always look for the technique with minimal power, which solves our problem sufficiently.

A very important field, which picked up ABS in recent years is economics. The field of economics is an immensely vast and complex one with many facets to it, ranging from firms, to financial markets to whole economies of a country \cite{bowles_understanding_2005}. Today its very foundations rest on rational expectations, optimization and the efficient market hypothesis. The idea is that the macroeconomics are explained by the micro foundations \cite{colell_microeconomic_1995} defined through behaviour of individual agents. These agents are characterized by rational expectations, optimizing behaviour, having perfect information, equilibrium \cite{focardi_is_2015}.
This approach to economics has come under heavy critizism in the last years for being not realistic, making impossible assumptions like perfect information, not being able to provide a process under which equilibrium is reached \cite{kirman_complex_2010} and failing to predict crashes like the sub-prime mortgage crisis despite all the promises - the science of economics is perceived to be detached from reality \cite{focardi_is_2015}. 
ACE is a promise to repair the empirical deficit which (neo-classic) economics seem to exhibit by allowing to make more realistic, empirical assumptions about the agents which form the micro foundations. The ACE agents are characterized by bounded rationality, local information, restricted interactions over networks and out-of-equilibrium behaviour \cite{farmer_economy_2009}. 
Works which investigate ACE as a discipline and discuss its methodology are \cite{tesfatsion_agent-based_2002}, \cite{richiardi_agent-based_2007}, \cite{ballot_agent-based_2015}, \cite{blume_introduction_2015}.
%look into computable economics book: \url{http://www.e-elgar.com/shop/computable-economics}
Tesfatsion \cite{tesfatsion_agent-based_2017} defines ACE as \textit{[...] computational modelling of economic processes (including whole economies) as open-ended dynamic systems of interacting agents.}. She gives a broad overview \cite{tesfatsion_agent-based_2006} of ACE, discusses advantages and disadvantages and giving the four primary objectives of it which are:

\begin{enumerate}
	\item Empirical understanding: why have particular global regularities evolved and persisted, despite the absence of centralized planning and control?
	\item Normative understanding: how can agent-based models be used as laboratories for the discovery of good economic designs?
	\item Qualitative insight and theory generation: how can economic systems be more fully understood through a systematic examination of their potential dynamical behaviours under alternatively specified initial conditions?
	\item Methodological advancement: how best to provide ACE researchers with the methods and tools they need to undertake the rigorous study of economic systems through controlled computational experiments?
\end{enumerate}

It is important to understand, that ACE utilises ABS different than the social sciences do. The latter one focuses more on agent-interactions, where in ACE the rational and non-rational actions of individual agents are more important. Thus in many ACE models, the full power of the techniques introduced in Part II is not required. More specifically, agents of ACE models tend to have much simpler state, behave often in only one specific way, don't use synchronised agent-interactions and are very rarely located in a spatial environment but focus more on network connections \cite{wilhite_economic_2006, glasserman_contagion_2015} or avoid the notion of connectivity altogether.

\medskip

To investigate this point more in-depth we implemented \footnote{Freely available at \url{https://github.com/thalerjonathan/zerointelligence}} a simulation with so called Zero Intelligence traders \cite{gode_allocative_1993}, inspired by an implementation in Python \footnote{\url{http://people.brandeis.edu/~blebaron/classes/agentfin/GodeSunder.html}}. We don't go into any technical detail here but the implementation drives the main points home:

\begin{itemize}
	\item Even though it is an agent-based model and there is a clear notion of agents in the Python code, where they are represented as objects, the agents are extremely simple. They are characterised by a single floating-point value, identifying how much value they attribute to an asset. Their behaviour is also very simple and does not change over time: they always bid randomly within their profit range. Thus we do \textit{not} implement agents as MSFs in this case but represent them indeed only through a \textit{Double} value, reducing the complexity of the implementation considerably.

	\item There are no direct agent-interactions. Although agents trade with each other, this happens through a central authority (the simulation kernel), which acts like a market with a limit order book. This reduces the complexity of the implementation considerably because there is no need for the full approach of synchronised direct agent-interactions. We could have implemented it in that way but that would have only increased complexity through the use of a quite powerful technique, which is actually not really needed because the same effect can be achieved in much simpler terms.

	\item There is no environment whatsoever and a fully connected network is implicitly assumed because each agent can trade with all other agents. This implies that the full technique of applying an environment is not necessary, which makes the simulation a lot less complex. Still adding an environment e.g. a network would be quite simple and does not require any monadic code as the network information can be made read-only in the way as we do in Chapter \ref{sec:adding_env}.
	
	\item The only side-effect necessary in this simulation is to draw random-numbers. By fixing the seed, repeated runs of initial conditions will always lead to same output, which is guaranteed at compile time. This was already shown in Part II and is a direct consequence of Haskells type-system and explicit way of dealing with effects.	 Further, we focused on keeping as much code \textit{pure} as possible thus splitting code which does not require random numbers into pure functions and only having the basic structure of the implementation running in the Rand Monad. This makes testing and reasoning considerably easier than running everything in the Rand Monad.
\end{itemize}

We are very well aware that this simple example is only one of many ACE models but even though it implements very simple \textit{zero} intelligence agents, it shows that ABS in Haskell does not need to be as complex as the use-cases in Part II - on the contrary, ABS implementations can be very concise and highly performant in Haskell.

\subsection{Gintis Bilateral Bartering}

\subsection{Agents As Objects}
It seems that we indeed have to agree that agents do actually map naturally to objects. Throughout the course of this thesis though it became clear that we have to think objects in a much broader context than the one of existing OO terminology as in Java, C++, Python. The reason that we have shown that we can represent agents as objects also in a purely functional way, leads us to the next question, what actually constitutes objects? We have to be careful not to confuse the \textit{concept} of objects with the \textit{implementation} of objects. Lets first be clear about the \textit{concept} of an object (Alan kay, Actors, Java/C++/Python family) and then look at \textit{implementations} (Java/C++/Python family, actors, pure functionally)


Although object-oriented programming was invented to give programmers a better way of composing their code, strangely objects ultimately do \textit{not} compose \cite{bill_what_2017}, \cite{erkki_lindpere_why_2013}. The reason for this is that objects hide both \textit{mutation} and \textit{sharing through pointers or references} of object-internal data. This makes data-flow mostly implicit due to the side-effects on the mutable data which is globally scattered across objects. To deal with the problem of composability and implicit data-flow the seminal work \cite{gamma_design_1994} put forward the use of \textit{patterns} to organize objects and their interaction. Other concepts, trying to address the problems, were the SOLID principles and Dependency Injection. Although a huge step in the right direction, these concepts come with a very heavy overhead, are often difficult to understand and to apply and don't solve the fundamental problem \cite{lawrence_krubner_object_2014}. To put it short: even for experienced programmers, proper object-oriented programming \textit{is hard}. The difficulty arises from how to split up a problem into objects and their interactions and controlling the implicit mutation of state which is spread across all objects. Still if one masters the technique of object-oriented program-design and implementation, due to the implicit global mutable state bugs due to side-effects are the daily life of a programmer as shown below. Note that this critique of object-oriented programming addresses the deficits of this paradigm as it is implemented and in use today in languages likes Java and C++. The original idea of object-orientation, invented by Alan Kay \footnote{\url{http://wiki.c2.com/?AlanKaysDefinitionOfObjectOriented}} was very different than today and has much more common with the Actor Model as will be discussed in the literature-review.


\paragraph{Object definition}
identity
internal state
message exchange

\paragraph{Object implementation}
Java/C++/Python family
Actors as in Erlang
pure functionally

% TODO: the history does say something else about objects as alan kay had them in mind.

When we go back to the beginning of this thesis and revisit the viewpoints that object-oriented programming (OO) is a particularly natural development environment for ABS \cite{epstein_growing_1996} and \textit{"agents map naturally to objects"} \cite{north_managing_2007} in the light of the full thesis, the obvious question which comes to mind is whether this thesis implies that we have been doing implementing ABS wrong ever since Epstein advocated OO for it in 1996 - shall Haskell be the new way to got? Obviously that is not the case but as we have shown, with (pure) functional programming comes a lot of potential. Lets elaborate on that.

\medskip

It is a fact that simulations are about consuming, processing and producing data. ABS being simulation methodology is no exception to that fact. Unfortunately, due to OO lack of rigour theoretical foundations, OO as it is used today is \textit{not} very good at representing and manipulating pure data and its data-flow because of two things: \textit{mutable shared state} and explicitly associate data-types and functions(methods)/code/behaviour.

%FROM https://www.youtube.com/watch?v=QM1iUe6IofM&feature=youtu.be
Inheritance is not relevant any more: it has come to a widely agreement amongst OO developers that inheritance should be avoided: https://www.javaworld.com/article/2073649/why-extends-is-evil.html . Note that we are speaking about subclassing not implementing an interface, which is something entirely different
Polymorphism: is not unique to OO and exists in non-OO languages as well and plays a central role in Haskell (and ML languages). Further it is possible to implement polymorphic code in C
Encapsulation: this is seen as the major strength of OO but unfortunately it does not work at a fine grained level of code in todays OO. The original idea was indeed great and it is no coincidence that my implementation ended up with a variation of that as well as Erlang: encapsulate state behind a public interface and interact with it through messages (TODO: fill in alan kay). The very central point of messages though was that they followed "shared nothing" semantics, meaning that no references or pointers could be contained in that message as this would immediately result in a violation of the public interface and ultimately breaks encapsulation. 
OO dominates the industry since around mid 90s. There are varying opinions on that but a major influence to popularise OO was Java, which made its first appearance in 1996. Java was a much easier approach to OO than existing ones e.g. in C++ and VB: it abandoned multiple inheritance, introduced interfaces, was cross-platform, provided high quality libraries including a GUI framework (GUI programming was the way to go in the 90s until it got abandoned in 00s with the emergence of Web 2.0), C/C++ syntax made it easy to pick up, avoided header-files, abandoned pointers and memory management and added garbage collection which made applications a lot safer.

% TODO: need to discuss the problem of shared state. state per se is not necessarily a problem and ever program has state in some form. how explicit it is represented is often used as classification between different kind of paradigms e.g. it has been said that functional programming is stateless but that is obviously not true, state is all over the place but it is very very contained, well behaved and explicit. with shared mutable state this is not the case anymore and we get into the troube of data-dependencies and orderings. this is exactly what we encountered when having introduced a global environment in Sugarscape: although our state is referential transparent and pure functional, they way we used it is globally and we run in ordering issues.

% TODO: isnt shared state also a problem in erlang? after all we can send Pids around and interact with those processes as soon as another process has access to a Pid. In which way is it different to reference passing in OO? There seems to be no difference... so maybe the anti OO argument is not that strong after all and my argument is simply weak or wrong? 

%TODO: i REALLY need to find proper literature / research / evidence which shows the problematic nature of modern OO: mutable shared state which is tied to code. Inheritance and open recursion gives the rest. the problem is that deeply linking \textit{shared mutable} state to its code is the path to failure: abstraction breaks, concurrency and parallelism becomes hard and breaks abstraction, data-driven programming becomes difficult (although that got addressed by adding functional features). NOTE: my approach and erlang have state and behaviour as well but in our case the state is shared nothing and immutable (yes in Haskell we update the agents state but that happens ultimately through closures and continuation in a referential transparent way and still no state is shared between agents. the environment is an exception to some extent as agents can access it globally: this works but requires a specific ordering either through sequential access or STM. this is no different than in an erlang implementation of sugarscape: there needs to be some arbitration of concurrent access). TODO: isnt there some fundamental research on that issue out there?
% TODO: maybe these act as a starting point?
% https://www.yegor256.com/2016/08/15/what-is-wrong-object-oriented-programming.html
% https://dl.acm.org/citation.cfm?id=1806847
% https://web.cs.ucdavis.edu/~filkov/papers/lang_github.pdf "Most notably, it does appear that strong typing is modestly better than weak typing, and among functional languages, static typing is also somewhat better than dynamic typing" "We also find that functional languages are somewhat better than procedural languages" but modest effects
% https://www.javaworld.com/article/2073649/why-extends-is-evil.html
% READ extension problem paper
% READ Ted Kaminskis thesis


%The rise of functional concepts in OO languages in the last years are a strong indication that OO is lacking features which have existed in FP for decades:
%
%\begin{itemize}
%	\item Java 8 added lambda expressions and functional style programming using map, fold, reduce, filter, which together with lambdas allow a data-flow oriented approach to computing.
%	 
%	\item Python, which surges in popularity within the OO family of languages, allows very data-flow centric and functional style of programming through lambda functions, list comprehensions and other functional features as it does not require programmers to stick to the OO paradigm.
%	
%	\item Popularisation of JavaScript frameworks like React, Elm and Purescript, which emphasise a functional, data-flow driven approach of web-programming.
%\end{itemize}

This was by no means clear in the early-to-mid 1990s where the OO paradigm was seen as a silver bullet to the problems of programming: a whole software industry had to re-learn best practices, patterns \cite{gamma_design_1994} and how to avoid pitfalls and bad code \cite{fowler_refactoring:_2012}. Thus we cannot blame \cite{epstein_growing_1996} for advertising OO as the ways to implement ABS, at that time it seemed indeed to be the right thing to do. 

The question is then why not use toolkits like Matlab or R - after all they are completely data-centric? This would be the other extreme, just like OO is and we would run into difficulties as well. The point is that ABS is not purely data-centric either and is indeed richer: agents can interact with each other and with an environment. So we have a tension here: ABS is data-centric on the one hand, and interaction-centric on the other - can we combine both worlds?

The combination of both was exactly the sales pitch of OO for the last 20+ years. Unfortunately this combination leads to nasty bugs due to shared mutable state, deeply complex object hierarchies due to inheritance overuse which also fix behaviour at compile time, open recursion which in the end costs the potential for higher degree of correctness, ease of parallelism and concurrency and the use of property-based testing. Thus we need to separate both: what we need is immutable, shared-nothing state allowing for a data-centric approach \textit{and} an interaction mechanism which allows agents to communicate with each other.

\medskip

This thesis is \textit{one} way of showing how to separate both and reap the benefits. A time-driven ABS like SIR or an ACE with simple agents not interaction with each other like ZI traders is heavily data-centric and very low on agent-interaction. Such data-driven ABS models are quite well expressed in a purely functional approach with the advantage that one can reap the benefits of reproducibility at compile time, using STM for concurrency and property-based testing for verification and validation. An event-driven simulation with complex agent state and agent-interactions like social simulations like Sugarscape or Chemical or Biology simulation with cell interactions are also possible in a pure functional setting as we have shown in the case of the Sugarscape model. Although we were able to give a good solution to complex agent state and synchronous, direct Agent-interactions in our event-driven SIR and Sugarscape and they \textit{do} work in Haskell, they are cumbersome to get right without library support (see further research below) and we cannot reap the benefits of STM in their case. 

This has lead to the fundamental conclusion that simulations which implement complex agent state and agent-communication centric models should rather be implemented in an functional language with actor based concurrency messaging. There are basically two options Cloud Haskell and Erlang \footnote{There is also Elixir, an Erlang dialect. Also there is Scala with the Akka / Actor Library but this is not purely functional and violates the shared nothing semantics, making it less data-centric}.
They are all message orientation will allow you to easily express the complex agent-interactions whilst having the potential to run them in parallel to gain a potentially substantial speed up. Despite its focus on messages, all are (pure) functional languages, which puts you into the data-centric approach: messages are pure data with \textit{shared nothing semantics}. This makes testing easier and also opens the way for property-based testing which is available in Erlang as well where it even allows to detect race conditions \cite{claessen_finding_2009}. Thus we can conclude that agents do \textit{not} map naturally to objects, agents map naturally to actors. 

% TODO: cite armstrongs blog about shared nothing semantics, should be somewhere in my 1st or 2nd year report 

\medskip

Thus it is not that implementing ABS with OO is wrong - it works reasonably well as a large number of industry strength libraries and frameworks demonstrate. It is more the \textit{missed potential} of a (pure) functional, data-centric approach: strong static type system with explicit controlled side-effects; parallel computation to speed up the simulation with very few changes but retaining static guarantees at compile time; STM to implement concurrent data-flow problems as actual data-flow problems without the need to resort to synchronisation primitives and cluttering the program logic with semantics for synchronisation and concurrency; Property-based testing for verification and validation of a data-centric approach which is central to all simulations; actor model concurrency in the case of Cloud Haskell and Erlang for agent-interaction centric models with a functional, data-centric core. 

Still, there are reasons to stick with OO and avoid FP. There exist a bunch the industry strength toolkits and libraries (Repast, NetLogo, AnyLogic) and the widespread use and knowledge of OO which makes abs implementers readily available. This allows for a quick and cheap implementation of low-impact and straightforward models where the need for correctness, reproducibility, verification and validation is not of primary concern. Also, performance in FP is still a far cry from OO although that argument might get diminished by the potential of using actor based concurrency like in Erlang to implement ABS. Another benefit is that OO as a modelling tool to a problem is still highly useful in the case of UML. %TODO:  discusses if and how peers object-oriented agent-based modelling framework can be applied to our pure functional approach. TODO: i need to re-read peers framework specifications / paper from the simulation bible book. Although peers framework uses UML and OO techniques to create an agent-based model, we realised from a short case-study with him that most of the framework can be directly applied to our pure functional approach as well, which is not a huge surprise, after all the framework is more a modelling guide than an implementation one. E.g. a class diagram identifies the main datastructures, their operations and relations, which can be expressed equally in our approach - though not that directly as in an oo language but at least the class diagram gives already a good outline and understanding of the required datafields and operations of the respective entities (e.g. agents, environment, actors,...). A state diagram expresses internal states of e.g. an agent, which we discussed how to do in both our time- and even-driven approach. A sequence diagram e.g. expresses the (synchronous) interactions between agents or with their environment, something for which we developed techniques in our event-driven approach and we discuss in depth there. 

\medskip

After having undertaken this long journey on how to implement ABS pure functionally, what the general computational structures are in ABS and what benefits and drawbacks there are, at the very end of our discussion I want to return to the claim that \textit{agents map naturally to objects} \cite{north_managing_2007}.

My approach of doing ABS and representing agents in pure FP can be interpret as trying to emulate objects in a purely functional way. In this case we have to say: yes agents map naturally to objects. The question is then: are there other, better mechanisms, more in FP I missed / didnt think of ... to implement ABS in FP? I hypothesize that this is probably NOT the case and that every approach in pure FP follows a roughly similar direction with only a few differences. Obviously it is apparent that both OOP and FP are not silverbullets to ABS and both come with their benefits and drawbacks and both have their existence. Thus I hypothesize that we might see the emergence of different computation paradigms in the future which might fit better to ABS than either one.

yes agents map naturally to objects, but what kind of objects? they differ in implementation details and in this thesis proposed a pure functional approach to a notion of objects. objects in Java work different, as well as in Smalltalk and objective c. processes in the functional language Erlang can be seen as objects too as they fullfill all criteria. also cite alan kay

\subsubsection{Actors}
\label{sec:actors}
TODO: this seems not to fit into the narrative here, maybe it fits into discussion part or further research

The Actor-Model, a model of concurrency, was initially conceived by Hewitt in 1973 \cite{hewitt_universal_1973} and refined later on \cite{hewitt_what_2007}, \cite{hewitt_actor_2010}. It was a major influence in designing the concept of agents and although there are important differences between actors and agents there are huge similarities thus the idea to use actors to build agent-based simulations comes quite natural. The theory was put on firm semantic grounds first through Irene Greif by defining its operational semantics \cite{grief_semantics_1975} and then Will Clinger by defining denotational semantics \cite{clinger_foundations_1981}. In the seminal work of Agha \cite{agha_actors:_1986} he developed a semantic mode, he termed \textit{actors} which was then developed further \cite{agha_foundation_1997} into an actor language with operational semantics which made connections to process calculi and functional programming languages (see both below). 

An actor is a uniquely addressable entity which can do the following \textit{in response to a message}
\begin{itemize}
	\item Send an arbitrary number (even infinite) of messages to other actors.
	\item Create an arbitrary number of actors.
	\item Define its own behaviour upon reception of the next message.
\end{itemize}

In the actor model theory there is no restriction on the order of the above actions and so an actor can do all the things above in parallel and concurrently at the same time. This property and that actors are reactive and not pro-active is the fundamental difference between actors and agents, so an agent is \textit{not} an actor but conceptually nearly identical and definitely much closer to an agent in comparison to an object. The actor model can be seen as quite influential to the development of the concept of agents in ABS, which borrowed it from Multi Agent Systems \cite{wooldridge_introduction_2009}. Technically, it emphasises message-passing concurrency with share-nothing semantics (no shared state between agents), which maps nicely to functional programming concepts.

The programming-model of actors \cite{agha_actors:_1986} was the inspiration for the Erlang programming language \cite{armstrong_erlang_2010}, which was created in the 1980's by Joe Armstrong for Eriksson for developing distributed high reliability software in telecommunications. The implication is that, the focus would shift immediately to the use of the actor model for concurrent interaction of agents through messages. The languages type-system is strong and dynamic and thus lacks type-checking at compile-time. Thus the structure of computation plays naturally no role because we cannot look at it from the abstract perspective as we can in Haskell. Purity can not be guaranteed and due to agents being processes concurrency is everywhere, and even though it is very tamed through shared-nothing messaging semantics, this implies that repeated runs with same initial conditions might lead to different results. Obviously we could avoid implementing agents as processes but then we basically sacrifice the very heart and feature of the language.
