\section{Towards Pure Functional ABS}
% the problem
It was unclear how to represent agents: how can we express agent identity, local agents state, changing behaviour and interactions amongst agents and the environment? After all this is straightforward in OO due to mutable shared state encapsulated in objects.

% the answer
The solution was to use arrowized FRP both in the pure implementation of Yampa and the monadic version as in the library Dunai. Building on top of them allowed us to implement pro-activity of agents, encapsulation of local agent state, an environment as shared mutable state and synchronous agent-interactions based on an event-driven approach. The central concept behind these approaches are Signal Functions (SFs) (generalised in Dunai to Monadic Stream Functions MSF) which are implemented using closures and continuations, fundamental building blocks and concepts of pure functional programming. SFs/MSFs can be seen as very simple \textit{immutable} objects with a single method following a \textit{shared nothing} semantics as mentioned in Chapter \ref{ch:structure_abs_computation} already. This interpretation of SFs and MSFs and the fact that we seem to achieve encapsulation of local agent state and interactions obviously raises the question if agents actually \textit{do} map naturally to objects - after all, despite being in a pure functional setting, we are talking about objects again!

%Then on the other hand... my education and thinking was influenced for 15+ years by object oriented thinking, so this could be also the case that i intuitionally arrive at such a solution again.

% discussing the fundamental question of this thesis
\subsection{Agents As Objects}
It seems that we indeed have to agree that agents do actually map naturally to objects. Throughout the course of this thesis though it became clear that we have to think objects in a much broader context than the one of existing OO terminology as in Java, C++, Python. The reason that we have shown that we can represent agents as objects also in a purely functional way, leads us to the next question, what actually constitutes objects? We have to be careful not to confuse the \textit{concept} of objects with the \textit{implementation} of objects. Lets first be clear about the \textit{concept} of an object (Alan kay, Actors, Java/C++/Python family) and then look at \textit{implementations} (Java/C++/Python family, actors, pure functionally)


Although object-oriented programming was invented to give programmers a better way of composing their code, strangely objects ultimately do \textit{not} compose \cite{bill_what_2017}, \cite{erkki_lindpere_why_2013}. The reason for this is that objects hide both \textit{mutation} and \textit{sharing through pointers or references} of object-internal data. This makes data-flow mostly implicit due to the side-effects on the mutable data which is globally scattered across objects. To deal with the problem of composability and implicit data-flow the seminal work \cite{gamma_design_1994} put forward the use of \textit{patterns} to organize objects and their interaction. Other concepts, trying to address the problems, were the SOLID principles and Dependency Injection. Although a huge step in the right direction, these concepts come with a very heavy overhead, are often difficult to understand and to apply and don't solve the fundamental problem \cite{lawrence_krubner_object_2014}. To put it short: even for experienced programmers, proper object-oriented programming \textit{is hard}. The difficulty arises from how to split up a problem into objects and their interactions and controlling the implicit mutation of state which is spread across all objects. Still if one masters the technique of object-oriented program-design and implementation, due to the implicit global mutable state bugs due to side-effects are the daily life of a programmer as shown below. Note that this critique of object-oriented programming addresses the deficits of this paradigm as it is implemented and in use today in languages likes Java and C++. The original idea of object-orientation, invented by Alan Kay \footnote{\url{http://wiki.c2.com/?AlanKaysDefinitionOfObjectOriented}} was very different than today and has much more common with the Actor Model as will be discussed in the literature-review.


\paragraph{Object definition}
identity
internal state
message exchange

\paragraph{Object implementation}
Java/C++/Python family
Actors as in Erlang
pure functionally

% TODO: the history does say something else about objects as alan kay had them in mind.

When we go back to the beginning of this thesis and revisit the viewpoints that object-oriented programming (OO) is a particularly natural development environment for ABS \cite{epstein_growing_1996} and \textit{"agents map naturally to objects"} \cite{north_managing_2007} in the light of the full thesis, the obvious question which comes to mind is whether this thesis implies that we have been doing implementing ABS wrong ever since Epstein advocated OO for it in 1996 - shall Haskell be the new way to got? Obviously that is not the case but as we have shown, with (pure) functional programming comes a lot of potential. Lets elaborate on that.

\medskip

It is a fact that simulations are about consuming, processing and producing data. ABS being simulation methodology is no exception to that fact. Unfortunately, due to OO lack of rigour theoretical foundations, OO as it is used today is \textit{not} very good at representing and manipulating pure data and its data-flow because of two things: \textit{mutable shared state} and explicitly associate data-types and functions(methods)/code/behaviour.

%FROM https://www.youtube.com/watch?v=QM1iUe6IofM&feature=youtu.be
Inheritance is not relevant any more: it has come to a widely agreement amongst OO developers that inheritance should be avoided: https://www.javaworld.com/article/2073649/why-extends-is-evil.html . Note that we are speaking about subclassing not implementing an interface, which is something entirely different
Polymorphism: is not unique to OO and exists in non-OO languages as well and plays a central role in Haskell (and ML languages). Further it is possible to implement polymorphic code in C
Encapsulation: this is seen as the major strength of OO but unfortunately it does not work at a fine grained level of code in todays OO. The original idea was indeed great and it is no coincidence that my implementation ended up with a variation of that as well as Erlang: encapsulate state behind a public interface and interact with it through messages (TODO: fill in alan kay). The very central point of messages though was that they followed "shared nothing" semantics, meaning that no references or pointers could be contained in that message as this would immediately result in a violation of the public interface and ultimately breaks encapsulation. 
OO dominates the industry since around mid 90s. There are varying opinions on that but a major influence to popularise OO was Java, which made its first appearance in 1996. Java was a much easier approach to OO than existing ones e.g. in C++ and VB: it abandoned multiple inheritance, introduced interfaces, was cross-platform, provided high quality libraries including a GUI framework (GUI programming was the way to go in the 90s until it got abandoned in 00s with the emergence of Web 2.0), C/C++ syntax made it easy to pick up, avoided header-files, abandoned pointers and memory management and added garbage collection which made applications a lot safer.

% TODO: need to discuss the problem of shared state. state per se is not necessarily a problem and ever program has state in some form. how explicit it is represented is often used as classification between different kind of paradigms e.g. it has been said that functional programming is stateless but that is obviously not true, state is all over the place but it is very very contained, well behaved and explicit. with shared mutable state this is not the case anymore and we get into the troube of data-dependencies and orderings. this is exactly what we encountered when having introduced a global environment in Sugarscape: although our state is referential transparent and pure functional, they way we used it is globally and we run in ordering issues.

% TODO: isnt shared state also a problem in erlang? after all we can send Pids around and interact with those processes as soon as another process has access to a Pid. In which way is it different to reference passing in OO? There seems to be no difference... so maybe the anti OO argument is not that strong after all and my argument is simply weak or wrong? 

%TODO: i REALLY need to find proper literature / research / evidence which shows the problematic nature of modern OO: mutable shared state which is tied to code. Inheritance and open recursion gives the rest. the problem is that deeply linking \textit{shared mutable} state to its code is the path to failure: abstraction breaks, concurrency and parallelism becomes hard and breaks abstraction, data-driven programming becomes difficult (although that got addressed by adding functional features). NOTE: my approach and erlang have state and behaviour as well but in our case the state is shared nothing and immutable (yes in Haskell we update the agents state but that happens ultimately through closures and continuation in a referential transparent way and still no state is shared between agents. the environment is an exception to some extent as agents can access it globally: this works but requires a specific ordering either through sequential access or STM. this is no different than in an erlang implementation of sugarscape: there needs to be some arbitration of concurrent access). TODO: isnt there some fundamental research on that issue out there?
% TODO: maybe these act as a starting point?
% https://www.yegor256.com/2016/08/15/what-is-wrong-object-oriented-programming.html
% https://dl.acm.org/citation.cfm?id=1806847
% https://web.cs.ucdavis.edu/~filkov/papers/lang_github.pdf "Most notably, it does appear that strong typing is modestly better than weak typing, and among functional languages, static typing is also somewhat better than dynamic typing" "We also find that functional languages are somewhat better than procedural languages" but modest effects
% https://www.javaworld.com/article/2073649/why-extends-is-evil.html
% READ extension problem paper
% READ Ted Kaminskis thesis


%The rise of functional concepts in OO languages in the last years are a strong indication that OO is lacking features which have existed in FP for decades:
%
%\begin{itemize}
%	\item Java 8 added lambda expressions and functional style programming using map, fold, reduce, filter, which together with lambdas allow a data-flow oriented approach to computing.
%	 
%	\item Python, which surges in popularity within the OO family of languages, allows very data-flow centric and functional style of programming through lambda functions, list comprehensions and other functional features as it does not require programmers to stick to the OO paradigm.
%	
%	\item Popularisation of JavaScript frameworks like React, Elm and Purescript, which emphasise a functional, data-flow driven approach of web-programming.
%\end{itemize}

This was by no means clear in the early-to-mid 1990s where the OO paradigm was seen as a silver bullet to the problems of programming: a whole software industry had to re-learn best practices, patterns \cite{gamma_design_1994} and how to avoid pitfalls and bad code \cite{fowler_refactoring:_2012}. Thus we cannot blame \cite{epstein_growing_1996} for advertising OO as the ways to implement ABS, at that time it seemed indeed to be the right thing to do. 

The question is then why not use toolkits like Matlab or R - after all they are completely data-centric? This would be the other extreme, just like OO is and we would run into difficulties as well. The point is that ABS is not purely data-centric either and is indeed richer: agents can interact with each other and with an environment. So we have a tension here: ABS is data-centric on the one hand, and interaction-centric on the other - can we combine both worlds?

The combination of both was exactly the sales pitch of OO for the last 20+ years. Unfortunately this combination leads to nasty bugs due to shared mutable state, deeply complex object hierarchies due to inheritance overuse which also fix behaviour at compile time, open recursion which in the end costs the potential for higher degree of correctness, ease of parallelism and concurrency and the use of property-based testing. Thus we need to separate both: what we need is immutable, shared-nothing state allowing for a data-centric approach \textit{and} an interaction mechanism which allows agents to communicate with each other.

\medskip

This thesis is \textit{one} way of showing how to separate both and reap the benefits. A time-driven ABS like SIR or an ACE with simple agents not interaction with each other like ZI traders is heavily data-centric and very low on agent-interaction. Such data-driven ABS models are quite well expressed in a purely functional approach with the advantage that one can reap the benefits of reproducibility at compile time, using STM for concurrency and property-based testing for verification and validation. An event-driven simulation with complex agent state and agent-interactions like social simulations like Sugarscape or Chemical or Biology simulation with cell interactions are also possible in a pure functional setting as we have shown in the case of the Sugarscape model. Although we were able to give a good solution to complex agent state and synchronous, direct Agent-interactions in our event-driven SIR and Sugarscape and they \textit{do} work in Haskell, they are cumbersome to get right without library support (see further research below) and we cannot reap the benefits of STM in their case. 

This has lead to the fundamental conclusion that simulations which implement complex agent state and agent-communication centric models should rather be implemented in an functional language with actor based concurrency messaging. There are basically two options Cloud Haskell and Erlang \footnote{There is also Elixir, an Erlang dialect. Also there is Scala with the Akka / Actor Library but this is not purely functional and violates the shared nothing semantics, making it less data-centric}.
They are all message orientation will allow you to easily express the complex agent-interactions whilst having the potential to run them in parallel to gain a potentially substantial speed up. Despite its focus on messages, all are (pure) functional languages, which puts you into the data-centric approach: messages are pure data with \textit{shared nothing semantics}. This makes testing easier and also opens the way for property-based testing which is available in Erlang as well where it even allows to detect race conditions \cite{claessen_finding_2009}. Thus we can conclude that agents do \textit{not} map naturally to objects, agents map naturally to actors. 

% TODO: cite armstrongs blog about shared nothing semantics, should be somewhere in my 1st or 2nd year report 

\medskip

Thus it is not that implementing ABS with OO is wrong - it works reasonably well as a large number of industry strength libraries and frameworks demonstrate. It is more the \textit{missed potential} of a (pure) functional, data-centric approach: strong static type system with explicit controlled side-effects; parallel computation to speed up the simulation with very few changes but retaining static guarantees at compile time; STM to implement concurrent data-flow problems as actual data-flow problems without the need to resort to synchronisation primitives and cluttering the program logic with semantics for synchronisation and concurrency; Property-based testing for verification and validation of a data-centric approach which is central to all simulations; actor model concurrency in the case of Cloud Haskell and Erlang for agent-interaction centric models with a functional, data-centric core. 

Still, there are reasons to stick with OO and avoid FP. There exist a bunch the industry strength toolkits and libraries (Repast, NetLogo, AnyLogic) and the widespread use and knowledge of OO which makes abs implementers readily available. This allows for a quick and cheap implementation of low-impact and straightforward models where the need for correctness, reproducibility, verification and validation is not of primary concern. Also, performance in FP is still a far cry from OO although that argument might get diminished by the potential of using actor based concurrency like in Erlang to implement ABS. Another benefit is that OO as a modelling tool to a problem is still highly useful in the case of UML. %TODO:  discusses if and how peers object-oriented agent-based modelling framework can be applied to our pure functional approach. TODO: i need to re-read peers framework specifications / paper from the simulation bible book. Although peers framework uses UML and OO techniques to create an agent-based model, we realised from a short case-study with him that most of the framework can be directly applied to our pure functional approach as well, which is not a huge surprise, after all the framework is more a modelling guide than an implementation one. E.g. a class diagram identifies the main datastructures, their operations and relations, which can be expressed equally in our approach - though not that directly as in an oo language but at least the class diagram gives already a good outline and understanding of the required datafields and operations of the respective entities (e.g. agents, environment, actors,...). A state diagram expresses internal states of e.g. an agent, which we discussed how to do in both our time- and even-driven approach. A sequence diagram e.g. expresses the (synchronous) interactions between agents or with their environment, something for which we developed techniques in our event-driven approach and we discuss in depth there. 

\medskip

After having undertaken this long journey on how to implement ABS pure functionally, what the general computational structures are in ABS and what benefits and drawbacks there are, at the very end of our discussion I want to return to the claim that \textit{agents map naturally to objects} \cite{north_managing_2007}.

My approach of doing ABS and representing agents in pure FP can be interpret as trying to emulate objects in a purely functional way. In this case we have to say: yes agents map naturally to objects. The question is then: are there other, better mechanisms, more in FP I missed / didnt think of ... to implement ABS in FP? I hypothesize that this is probably NOT the case and that every approach in pure FP follows a roughly similar direction with only a few differences. Obviously it is apparent that both OOP and FP are not silverbullets to ABS and both come with their benefits and drawbacks and both have their existence. Thus I hypothesize that we might see the emergence of different computation paradigms in the future which might fit better to ABS than either one.

yes agents map naturally to objects, but what kind of objects? they differ in implementation details and in this thesis proposed a pure functional approach to a notion of objects. objects in Java work different, as well as in Smalltalk and objective c. processes in the functional language Erlang can be seen as objects too as they fullfill all criteria. also cite alan kay

\subsubsection{Actors}
\label{sec:actors}
TODO: this seems not to fit into the narrative here, maybe it fits into discussion part or further research

The Actor-Model, a model of concurrency, was initially conceived by Hewitt in 1973 \cite{hewitt_universal_1973} and refined later on \cite{hewitt_what_2007}, \cite{hewitt_actor_2010}. It was a major influence in designing the concept of agents and although there are important differences between actors and agents there are huge similarities thus the idea to use actors to build agent-based simulations comes quite natural. The theory was put on firm semantic grounds first through Irene Greif by defining its operational semantics \cite{grief_semantics_1975} and then Will Clinger by defining denotational semantics \cite{clinger_foundations_1981}. In the seminal work of Agha \cite{agha_actors:_1986} he developed a semantic mode, he termed \textit{actors} which was then developed further \cite{agha_foundation_1997} into an actor language with operational semantics which made connections to process calculi and functional programming languages (see both below). 

An actor is a uniquely addressable entity which can do the following \textit{in response to a message}
\begin{itemize}
	\item Send an arbitrary number (even infinite) of messages to other actors.
	\item Create an arbitrary number of actors.
	\item Define its own behaviour upon reception of the next message.
\end{itemize}

In the actor model theory there is no restriction on the order of the above actions and so an actor can do all the things above in parallel and concurrently at the same time. This property and that actors are reactive and not pro-active is the fundamental difference between actors and agents, so an agent is \textit{not} an actor but conceptually nearly identical and definitely much closer to an agent in comparison to an object. The actor model can be seen as quite influential to the development of the concept of agents in ABS, which borrowed it from Multi Agent Systems \cite{wooldridge_introduction_2009}. Technically, it emphasises message-passing concurrency with share-nothing semantics (no shared state between agents), which maps nicely to functional programming concepts.

The programming-model of actors \cite{agha_actors:_1986} was the inspiration for the Erlang programming language \cite{armstrong_erlang_2010}, which was created in the 1980's by Joe Armstrong for Eriksson for developing distributed high reliability software in telecommunications. The implication is that, the focus would shift immediately to the use of the actor model for concurrent interaction of agents through messages. The languages type-system is strong and dynamic and thus lacks type-checking at compile-time. Thus the structure of computation plays naturally no role because we cannot look at it from the abstract perspective as we can in Haskell. Purity can not be guaranteed and due to agents being processes concurrency is everywhere, and even though it is very tamed through shared-nothing messaging semantics, this implies that repeated runs with same initial conditions might lead to different results. Obviously we could avoid implementing agents as processes but then we basically sacrifice the very heart and feature of the language.

\subsection{Static guarantees}
TODO: we can enforce update semantics to a certain degree

polymorphism in my approach is really powerful as shown in sugarscape: we can compose effects depending on the model, we can easily swap out Environment and Events\\

Probably the biggest strength is that we can guarantee reproducibility at compile time: given identical initial conditions, repeated runs of the simulation will lead to same outputs. This is of fundamental importance in simulation and addressed in the Sugarscape model: \textit{"... when the sequence of random numbers is specified ex ante the model is deterministic. Stated yet another way, model output is invariant from run to run when all aspects of the model are kept constant including the stream of random numbers."} (page 28, footnote 16) - we can guarantee that in our pure functional approach already \textit{at compile time}.

Refactoring is very convenient and quickly becomes the norm: guided by types (change / refine them) and relying on the compiler to point out problems, results in very effective and quick changes without danger of bugs showing up at run-time. This is not possible in Python because of its lack of compiler and types, and much less effective in Java due to its dynamic type-system which is only remedied through strong IDE support.

Adding data-parallelism is easy and often requires simply swapping out a data-structure or library function against its parallel version. Concurrency, although still hard, is less painful to address and add in a pure functional setting due to immutable data and explicit side-effects. Further, the benefits of implementing concurrent ABS based on Software Transactional Memory (STM) has been shown \cite{thaler_tale_2018} which underlines the strength of Haskell for concurrent ABS due to its strong guarantees about retry-semantics.

Testing in general allows much more control and checking of invariants due to the explicit handling of effects - together with the strong static type system, the testing-code is in full, explicit control over the functionality it tests. Property-based testing in particular is a perfect match to testing ABS due to the random nature in both and because it supports convenient expressing of specifications. Thus we can conclude that in a pure functional setting, testing is very expressive and powerful and supports working towards an implementation which is very likely to be correct.\\

General there are the following basic verification \& validation requirements to ABS \cite{robinson_simulation:_2014}, which all can be addressed in our \textit{pure} functional approach as described in the paper in Appendix \ref{app:pfe}:

\begin{itemize}
	%\item Modelling progress of time - achieved using functional reactive programming (FRP)
	%\item Modelling variability - achieved using FRP
	\item Fixing random number streams to allow simulations to be repeated under same conditions - ensured by \textit{pure} functional programming and Random Monads
	\item Rely only on past - guaranteed with \textit{Arrowized} FRP
	\item Bugs due to implicitly mutable state - reduced using pure functional programming
	\item Ruling out external sources of non-determinism / randomness - ensured by \textit{pure} functional programming
	\item Deterministic time-delta - ensured by \textit{pure} functional programming
	\item Repeated runs lead to same dynamics - ensured by \textit{pure} functional programming
\end{itemize}

\begin{enumerate}
	\item Run-Time robustness by compile-time guarantees - by expressing stronger guarantees already at compile-time we can restrict the classes of bugs which occur at run-time by a substantial amount due to Haskell's strong and static type system.  This implies the lack of dynamic types and dynamic casts \footnote{Note that there exist casts between different numerical types but they are all safe and can never lead to errors at run-time.} which removes a substantial source of bugs.  Note that we can still have run-time bugs in Haskell when our functions are partial.
	\item Purity - By being explicit and polymorphic in the types about side-effects and the ability to handle side-effects explicitly in a controlled way allows to rule out non-deterministic side-effects which guarantees reproducibility due to guaranteed same initial conditions and deterministic computation. Also by being explicit about side-effects e.g. Random-Numbers and State makes it easier to verify and test.
	\item Explicit Data-Flow and Immutable Data - All data must be explicitly passed to functions thus we can rule out implicit data-dependencies because we are excluding IO. This makes reasoning of data-dependencies and data-flow much easier as compared to traditional object-oriented approaches which utilize pointers or references.
	\item Declarative - describing \textit{what} a system is, instead of \textit{how} (imperative) it works. In this way it should be easier to reason about a system and its (expected) behaviour because it is more natural to reason about the behaviour of a system instead of thinking of abstract operational details.
	\item Concurrency and parallelism - due to its pure and 'stateless' nature, functional programming is extremely well suited for massively large-scale applications as it allows adding parallelism without any side-effects and provides very powerful and convenient facilities for concurrent programming. The paper of (TODO: cite my own paper on STM) explores the use Haskell for concurrent and parallel ABS in a deeper way.
\end{enumerate}

TODO: haskell-titan
TODO: Testing and Debugging Functional Reactive Programming \cite{perez_testing_2017}

Static type system eliminates a large number run-time bugs.

TODO: can we apply equational reasoning? Can we (informally) reason about various properties e.g. termination?

Follow unit testing of the whole simulation as prototyped for towards paper.

in this we explore something new: property-based testing in ABS

\section{Drawbacks}
\label{sec:drawbacks}
The initially hypothesised drawbacks of performance issues and agent interaction were confirmed in our research. We discuss them here more in detail, together with other drawbacks and propose solutions where applicable.

\subsection{Efficiency}
\label{sec:drawback_efficiency}
As mentioned already in the discussions of the respective chapters, currently the performance of our approaches does not come close to imperative implementations. There are two main reasons for it: First, functional programming is known for being slower due to higher levels of abstraction, which are bought by slower code in general and second, updates are the main bottleneck due to immutable data requiring to copying of the whole or subparts of a data structure in cases of a change. The first one is easily addressable through the use of data parallelism and concurrency as shown in Chapter \ref{ch:parallelism_ABS} and \ref{ch:concurrent_abs}. The second reason could potentially be addressed by the use of linear types \cite{bernardy_linear_2017}, which allow to annotate a variable with how often it is used within a function. From this a compiler could derive aggressive optimisations, potentially resulting in imperative-style performance but retaining the declarative nature of the code. We leave this for further research. Also, the use of Monad Transformer stacks has performance implications, which can be quite subtle. A possible optimisation we followed is the careful usage and reordering of lifts, using \texttt{lift (mapM ...)} instead of \texttt{mapM (lift ...)}, which potentially results in increased performance.

However, it was shown by various people \cite{kqr_competing_2017, stewart_haskell_2008, stolarek_haskell_2013} that Haskell does not necessarily have to be slow and that it is indeed possible to reach C speed in Haskell. The direction to do this is using the worker/wrapper transformation \cite{gill_worker/wrapper_2009}, a clever combination of techniques with strict \texttt{foldl'} and data declaration with strictness annotation instead of lazy tuples and Stream Fusion \cite{coutts_stream_2007, mainland_haskell_2013}. The problem is, of course, that to apply these techniques one needs to have deep knowledge of Haskell and its subtle details of lazy evaluation, making this a highly non-trivial task. Another problem is that those techniques seem only applicable in the context of a tight loop, which crunches numbers of a list, thus it is not directly applicable in our case as we are clearly bound by the effectful computations: MSF and Monad Transformers are the limiting factor, not inner loops.

Concluding we can say that the current performance makes our approach not very attractive for real-world use \textit{at the moment}. Also, the fact that the sequential object-oriented implementation of the SIR model seems to outperform the concurrent and parallel implementations as well, seems to question our motivation as to why we are using pure functional programming and parallel computation at all. Still, the bad performance results do not invalidate our research, as this thesis aim is not the development of high-performance pure functional implementations, but rather exploring concepts of ABS in pure functional programming. Thus, this work is seen as a first step which needs to be developed further into something to be used in the real world. We hypothesise that it should be possible for our pure functional approach to come considerably closer to imperative performance, ultimately making it more applicable for real-world usage. We leave a deeper investigation of this problem for further research.

\subsection{Space Leaks}
Haskell is notorious for its memory leaks due to lazy evaluation: data is only evaluated when required. Even for simple programs, one can be hit hard by a serious space leak where thunks - unevaluated code pieces - build up in memory until they are needed, leading to dramatically increased memory usage. It is no surprise that our highly complex Sugarscape implementation initially suffered severely from space leaks, accumulating about 40 MBytes/second. In simulation this is a big issue, threatening the value of the whole implementation despite its other benefits. Due to the fact that simulations might run for a (very) long time or conceptually forever, one must make absolutely sure that the memory usage stays within reasonable bounds. As a remedy, Haskell allows us to add so-called strictness pragmas to code modules, which force strict evaluation of all data even if it is not used. %Carefully adding this conservatively, file-by-file applying other techniques of forcing evaluation removed most of the memory leaks.

Another memory leak was caused by selecting the wrong data structure for the environment, for which we initially used an immutable array. The problem is that in the case of an update the whole array is copied, causing memory leaks \textit{and} a performance problem. We replaced it by an \texttt{IntMap} which uses integers as key and is internally implemented as a radix tree which allows for very fast lookups and inserts because whole sub-trees can be reused.

\subsection{Agent Interactions}
Synchronous, direct agent interactions \textit{do} work in Haskell but they are cumbersome to get right when building from scratch. Furthermore, as we pointed out in Chapter \ref{ch:concurrent_abs}, it seems that our approach to synchronous direct agent interactions is not applicable to concurrency with STM. 

This leads to the fundamental conclusion that in models which require complex agent interactions in a potentially concurrent environment, we are hitting the limits of our pure functional approach. The reason for it is that we have a conceptual mismatch, as in such a setting, agents are more naturally represented using the Actor Model. The Actor Model, a model of concurrency, was initially conceived by Hewitt in 1973 \cite{hewitt_universal_1973} and refined later on \cite{hewitt_what_2007}, \cite{hewitt_actor_2010}. It was a major influence in designing the concept of agents and although there are important differences between actors and agents there are important similarities, thus the idea to use actors to build ABS comes quite natural.
An actor is a uniquely addressable entity which can do the following \textit{in response to a message}:
\begin{itemize}
	\item Send an arbitrary number of messages to other actors.
	\item Create an arbitrary number of actors.
	\item Define its own behaviour upon reception of the next message.
\end{itemize}

When comparing this definition to the one of agents we give in Chapter \ref{sec:method_abs}, it is clear that the Actor Model was quite influential to the development of the concept of agents in ABS, which borrowed it initially from Multi-Agent Systems \cite{wooldridge_introduction_2009}. Technically, it emphasises message-passing concurrency with shared-nothing semantics (no implicitly shared state through side effects between agents), which maps nicely to functional programming concepts.

Indeed, the programming model of actors \cite{agha_actors:_1986} was the inspiration for the functional programming language Erlang, thus we argue that a true concurrent actor approach like Erlang is substantially more natural and much more performant especially in a concurrent setting. Furthermore, we hypothesise that actor based ABS implementations might have a bright future as ABS tends to develop towards larger and larger, distributed, always-online simulations, for which Erlang is arguably perfectly suited. We have prototyped highly promising concurrent event-driven SIR and Sugarscape implementations in Erlang supporting our hypothesis. However, an in-depth discussion is beyond the scope of this thesis and we leave this topic for further research.

%This makes testing easier and also opens the way for property-based testing which is available in Erlang as well where it even allows to detect race conditions \cite{claessen_finding_2009}. 

%erlang processes can implement everything objects can but in a referential transparent and pure functional way: encapsulation, polymorphism, identity, message passing, even inheritance (which you wouldnt want to do)

%difference of erlang to objects is that although it encapsulate state it cannot be accessed at the same time but only through the message passing Interface with one message at a time. this means that state is not really shared and protected against mutation - the process is in full control. it is simply a function which captures the full state of the process in an immutable way: to change the state a recursive call needs to be done. so in the end although it seems conceptually related its technically difference is fundamental importance.

%my mistake was to confuse the concept of objects with their implementation. i was too focused on the drawbacks of e.g. java objects that i forgot that i was critisising its IMPLEMENTATION. the Original idea of alan kays objects IS a deep and strong idea, though it differes substantially from java objects, erlang comes closest. thus the concept is important and valid but different implementations have different benefits and drawbacks. 

\subsection{Productivity and Learning Curve}
A case study in \cite{hanenberg_experiment_2010} hints that simply by switching to a static type system alone does not gain anything and can even be detrimental. To be useful, it needs to have a certain level of abstractions like Haskells' type system. Although such case studies have to be taken with care, there is also some truth in it: working in a statically strong type system prevents the developer from moving quickly and making quick changes. This can be both a benefit and a drawback: in general, it prevents one from breaking changes which show up at compile time. At the same time, the whole program is much more rigid and a proper structure needs to be thought out and designed often up-front, slowing down the process. However, it is a contribution of this thesis that it outlines exactly these structures within ABS, so that implementers who want to use the same approach do not have to reinvent the wheel.

A more severe problem is that pure functional programming, especially Haskell, is regarded as hard to learn with a steep learning curve, putting a high barrier to implementers picking up a pure functional approach to ABS. Thus, the lack of broad availability of Haskell expertise can be enough to pose a serious drawback even if the approach of this thesis seems to be desirable in a project.

\subsection{Alternatives}
Freer Monads: \url{https://reasonablypolymorphic.com/blog/freer-monads/}. They aim to separate Definition from implementation by writing a domain-specific language using GADTs which are then interpreted. This allows to strictly separate implementation from specification, composes very well and thus is easier to test as parts can be easily mocked. Also Freer Monads free one from the order of effects imposed through Monad Transformers. In general Freer Monads seem to aim for the same abstraction what modern interface-based oop does.
Problem: Yes, freer monads are today somewhere around 30x slower than the equivalent mtl code. because its $O(n^2)$. ABS are not IO bound, so raw computation is all what counts and this is undoubtly worse with Freer monads. Given that we are already having problematic performance, we can't sacrifice even more. There seem to be a better encoding possible, which is about 2x slower than MTL: \url{https://reasonablypolymorphic.com/blog/too-fast-too-free/}. Still it might prove to be useful in other terms like proving correctness and then translating it, but how could we do it?  
ContT is Not an Algebraic Effect so it seems to be difficult to implement continuations in freer monads. Unfortunately this is what we really need as shown in Event driven ABS and generalising structure chapters.
Other criticism of Freer Monads are: \url{https://medium.com/barely-functional/freer-doesnt-come-for-free-c9fade793501} are: boilerplate code which though can be generated automatically by some libraries, performance when not IO based because the program is bascially a data-structure which is interpreted, concurrency seems to be tricky, 
TODO read: \url{https://reasonablypolymorphic.com/blog/freer-yet-too-costly/}

TODO: it seems that Final Tagless is another alternative. Look into it: \url{https://jproyo.github.io/posts/2019-03-17-tagless-final-haskell.html}

%\subsection{Languages}
%We precisely pointed out in the beginning of this thesis why we chose Haskell as language of choice. Obviously Haskell is not the only (pure) functional language and there exist a number of other alternatives which would be equally worth of systematic investigation of their use for ABS. Shortly we can conclude that the use of Haskell moves the nature of the structure of ABS computation into the light, together with compile-time guarantees, and benefits in testing and parallel implementations. Depending on each language though we get a very different direction:

%\paragraph{LISP} Being the oldest functional programming language and the 2nd oldest high-level programming language ever created and still used by many people, LISP had to be considered in the beginning of the thesis. The language has the immense powerful feature of homoiconicity: data is code and code is data at the same time. This allows a LISP program to generate data-structures, which resemble valid LSIP code thus mutating its own code at runtime. This would give immense power to create powerful abstractions in terms of ABS. Unfortunately LISP is fully interpreted and has no types and is also impure, which would probably have led to very imperative, traditional approaches to ABS. Still, there exists research \cite{kawabe_nepi2programming_2000} which implements a MAS in LISP.
	
%\paragraph{Scala} Scala is a multi-paradigm language, which also comes with an implementation of the actor-model as a library which enables to do actor-programming in the way of Erlang. It was developed in 2004 and became popular in recent years due to the increased availability of multi-core CPUs which emphasised the distributed, parallel and concurrent programming for which the actor-model is highly suited. There exists research on using Scala for ABS \cite{krzywicki_massively_2015, todd_multi-agent_nodate}. The benefit Scala has over Erlang is that it has type-checking at compile-time and is thus more robust, still it is impure due to side-effects and messages can contain references, thus violating the original shared-nothing semantics of Erlang.

\subsection{Actors}
\label{sec:actors}
TODO: this seems not to fit into the narrative here, maybe it fits into discussion part or further research

The Actor-Model, a model of concurrency, was initially conceived by Hewitt in 1973 \cite{hewitt_universal_1973} and refined later on \cite{hewitt_what_2007}, \cite{hewitt_actor_2010}. It was a major influence in designing the concept of agents and although there are important differences between actors and agents there are huge similarities thus the idea to use actors to build agent-based simulations comes quite natural. The theory was put on firm semantic grounds first through Irene Greif by defining its operational semantics \cite{grief_semantics_1975} and then Will Clinger by defining denotational semantics \cite{clinger_foundations_1981}. In the seminal work of Agha \cite{agha_actors:_1986} he developed a semantic mode, he termed \textit{actors} which was then developed further \cite{agha_foundation_1997} into an actor language with operational semantics which made connections to process calculi and functional programming languages (see both below). 

An actor is a uniquely addressable entity which can do the following \textit{in response to a message}
\begin{itemize}
	\item Send an arbitrary number (even infinite) of messages to other actors.
	\item Create an arbitrary number of actors.
	\item Define its own behaviour upon reception of the next message.
\end{itemize}

In the actor model theory there is no restriction on the order of the above actions and so an actor can do all the things above in parallel and concurrently at the same time. This property and that actors are reactive and not pro-active is the fundamental difference between actors and agents, so an agent is \textit{not} an actor but conceptually nearly identical and definitely much closer to an agent in comparison to an object. The actor model can be seen as quite influential to the development of the concept of agents in ABS, which borrowed it from Multi Agent Systems \cite{wooldridge_introduction_2009}. Technically, it emphasises message-passing concurrency with share-nothing semantics (no shared state between agents), which maps nicely to functional programming concepts.

There have been a few attempts on implementing the actor model in real programming languages where the most notable ones are Erlang and Scala. Erlang was created in 1986 by Joe Armstrong for Eriksson for developing distributed high reliability software in telecommunications. It implements light-weight processes, which allows to spawn thousands of them without heavy memory overhead. The language saw some use in implementing ABS with notable papers being \cite{di_stefano_using_2005, di_stefano_exat:_2007, varela_modelling_2004, sher_agent-based_2013, bezirgiannis_improving_2013}

Scala is a modern mixed paradigm programming language, which also allows functional programming and also incorporates a library for the actor model. It also saw the use in the implementation of ABS with a notable paper \cite{krzywicki_massively_2015} and ScalABM \footnote{https://github.com/ScalABM} which is alibrary for ABM in economics.

The paper of \cite{jennings_agent-based_2000} gives an excellent overview over the strengths and weaknesses of agent-based software-engineering, which can be directly applied to both Erlang and Scala.

Due to the very different approach and implications the actor model of concurrency implies, we don't explore it further and leave it for further research as it is beyond the focus of the thesis.

The programming-model of actors \cite{agha_actors:_1986} was the inspiration for the Erlang programming language \cite{armstrong_erlang_2010}, which was created in the 1980's by Joe Armstrong for Eriksson for developing distributed high reliability software in telecommunications. The implication is that, the focus would shift immediately to the use of the actor model for concurrent interaction of agents through messages. The languages type-system is strong and dynamic and thus lacks type-checking at compile-time. Thus the structure of computation plays naturally no role because we cannot look at it from the abstract perspective as we can in Haskell. Purity can not be guaranteed and due to agents being processes concurrency is everywhere, and even though it is very tamed through shared-nothing messaging semantics, this implies that repeated runs with same initial conditions might lead to different results. Obviously we could avoid implementing agents as processes but then we basically sacrifice the very heart and feature of the language.