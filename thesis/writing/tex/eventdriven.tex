\chapter{Pure functional event-driven ABS}
\label{ch:eventdriven}
In this chapter we build on the previous discussion of update strategies in Chapter \ref{ch:impl_abs} and the implementation techniques presented in the time-driven approach of Chapter \ref{ch:timedriven} to develop concepts for event-driven ABS in a pure functional way. 

\medskip

In event-driven ABS \cite{meyer_event-driven_2014}, the simulation is advanced through events. Agents and the environment schedule events into the future and react to incoming events scheduled by themselves, other agents, the environment or the simulation kernel. Time is discrete in this approach and it advances step-wise from event to event, where each event has an associated receiver and $\Delta t$, indicating the delay to the current virtual simulation time when the event should be scheduled. This implies that time could stay constant, for example when an event is scheduled with $\Delta t = 0$ the virtual simulation time does not advance. Further, agents can schedule events to themselves, emulating a recurring behaviour, which in turn emulates pro-active behaviour. Because agents can adopt and change their state and behaviour when processing an event, this means that even if time does not advance, agents can change. This non-signal behaviour is the fundamental difference to the time-driven approach in Chapter \ref{ch:timedriven}. Further, this mechanism is used to implement synchronous agent interactions pure functionally as discussed in the respective sections below.

The event-driven approach makes the simulation kernel technically closely related to a Discrete Event Simulation \cite{zeigler_theory_2000}. Due to the necessity of imposing a correct ordering of events in this type of ABS, it needs to be stepped event by event, with the \textit{sequential} update strategy, as introduced in Chapter \ref{sec:seq_strategy}. It is important to emphasise that only the semantics of the sequential update strategy allow the kind of features  presented in the following sections, as the agents act one after the other, seeing the effects of previous agents in the same time step. This would not make sense in the parallel update strategy as used in time-driven ABS, where agents act conceptually at the same time. This means that event-driven ABS is inherently sequential due to its fundamental reliance on effects as will become clearer in the sections below. There exists also Parallel Discrete Event Simulation \cite{fujimoto_parallel_1990}, which processes events in parallel and deals with inconsistencies by reverting to consistent states. We hypothesise that a pure functional approach could be beneficial in such an approach due to persistent data structures and explicit handling of side effects but we leave this for further research.

\medskip

We start the chapter by introducing the concepts of agent identity and event scheduling using an event-driven agent-based SIR model, inspired by \cite{macal_agent-based_2010}. We then use the highly complex Sugarscape model as introduced in Chapter \ref{sec:sugarscape}, to develop more advanced features of ABS in a pure functional context: dynamic creation and removal of agents during simulation, adding a shared mutable environment, local mutable agent state and synchronous agent interactions. 
%Note that the Sugarscape model is not a real event-driven model like the event-driven SIR one is as in it the agents do schedule events but they don't do this into the future - events in Sugarscape don't have associated time-stamps.

\section{Basics of event-driven ABS}
In this section we derive step-by-step the basics of event-driven ABS using the SIR model, as introduced in Chapter \ref{sec:sir_model}, with an event-driven approach. This is in stark contrast to the time-driven implementation in Chapter \ref{sec:timedriven_firststep}. The solutions are quantitatively equal as they produce the same class of dynamics. Qualitatively they fundamentally differ though in terms of expressivity and performance as we will see in the discussion.

The basics of event-driven ABS are the concept of agent identity, events and event-scheduling. We introduce them step-by-step using various Monads and then generalise to a \textit{tagless final} approach, which has various benefits as pointed out in the respective section. 

\subsection{An event-driven SIR}
Before we can derive implementation concepts, we first need to discuss how an event-driven SIR model works, as inspired by \cite{macal_agent-based_2010}. Fundamentally, what is required is to transform all time-dependent behaviour and agent interactions into the reception and scheduling of events. For the SIR this should be trivial and straight-forward, taking inspiration from the time-driven implementation, where we simply translate the occurrences of events generated by \textit{occasionally} into scheduled events. For agent interactions we also use events, making this more explicit than in the time-driven approach. As already pointed out, assuming that events have a receiver and a scheduling time given as $\Delta t$ relative to the current simulation time, we end up with three event-types:

\begin{enumerate}
	\item \textbf{MakeContact} - is used to let susceptible agents pro-actively make contact with $\beta$ (contact rate) other agents per 1 time-unit.
	\item \textbf{Contact$_{Sender, \ SIRState}$} - is used to make contact between agents where agents reveal their state by sending or replying with their current state.
	\item \textbf{Recover} - is used to let infected agents pro-actively recover after the given $\delta$ (illness duration). 
\end{enumerate}

Now we can give a concise definition of all three agent behaviours:

\paragraph{Susceptible Agent}
\begin{itemize}
	\item A susceptible agent initially schedules a \textit{MakeContact} event with $\Delta t = 1$ to itself.
	\item When receiving \textit{MakeContact}, the agent sends a \textit{Contact} event to $\beta$ (contact rate) random other agents with $\Delta t = 0$ and \textit{SIRState} of \textit{Susceptible}, resulting in these events to be scheduled immediately. Further, the agent schedules \textit{MakeContact} with $\Delta t = 1$ to itself, to keep the pro-active process of making contact with other agents active.
	\item When the agent receives a \textit{Contact} event, it checks if it is from an infected agent (\textit{SIRState} is \textit{Infected}). If the event is not from an infected agent, it ignores it, otherwise it becomes infected with a given probability.
\end{itemize}

\paragraph{Infected Agent}
\begin{itemize}
	\item An infected agent initially schedules a \textit{Recover} event to itself, with an exponentially distributed random $\Delta t$ of $\delta$ (illness duration).
	\item When the agent receives a \textit{Contact} event, it checks if it is from a susceptible agent (\textit{SIRState} is \textit{Susceptible}). If the event is not from a susceptible agent, it ignores it, otherwise it simply replies to this susceptible agent with a \textit{Contact} event with $\Delta t = 0$ and and \textit{SIRState} of \textit{Infected}.
\end{itemize}

\paragraph{Recovered Agent}
The recovered agent does not change any more, reacts to no incoming events and schedules no events - it stays constantly \textit{Recovered} forever.

\medskip

It is easy to see that this behaviour emulates the time-driven one and is qualitatively equivalent and indeed in Figure \ref{fig:sir_eventdriven_dynamics} it is also visually clear that it produces similar dynamics. A striking difference are the small spikes and steps in the dynamics, which stem from the fact that time advances discretely and not continuous as in the time-driven implementation. In Chapter \ref{ch:sir_invariants}, we use property-based testing to show that the different implementations indeed produce similar distributions in their dynamics, thus putting the claim that both implementations are qualitatively equal on a much more robust ground.

\begin{figure}
	\centering
	\includegraphics[width=0.7\textwidth, angle=0]{./fig/eventdriven/sir_eventdriven.png}
	\caption{Dynamics of the event-driven SIR model. Population Size $N$ = 1,000, contact rate $\beta = \frac{1}{5}$, infection probability $\gamma = 0.05$, illness duration $\delta = 15$ with initially 1 infected agent. Simulation run for 150 time-steps.}
	\label{fig:sir_eventdriven_dynamics}
\end{figure}

\subsection{First steps}
We can now start to discuss the concepts from an implementation perspective. First, we need to make the concept of an event explicit: they are of a given type, have a receiver and a time-stamp in \textit{absolute} simulation time when they shall be scheduled. We keep the event-type polymorphic and represent the receiver by an \textit{AgentId} which is a simple \textit{Int}. For efficient scheduling, the events are kept in a priority-queue \footnote{We are using the \textit{Data.PQueue.Min} implementation from the \textit{pqueue} package.}, sorted ascending by the time-stamp. Thus we define the following:

\begin{HaskellCode}
type Time        = Double
type AgentId     = Int
data QueueItem e = QueueItem e AgentId Time

-- the event priority-queue
type EventQueue e = PQ.MinQueue (QueueItem e)

-- implement Ord for QueueItem for acended sorting
instance Ord (QueueItem e) where
  compare (QueueItem _ _ t1) (QueueItem _ _ t2) = compare t1 t2
\end{HaskellCode}

Next, we define a polymorphic type for our agent. As already pointed out, we will switch to the direct use of \textit{MSF} instead of BearRivers \textit{SF}. As input, the polymorphic event-type \textit{e} is used, because in event-driven ABS agents receive events, which drive the simulation. As output, the polymorphic output-type \textit{o} is used, which will be instantiated to a specific monomorphic type in the SIR model below. The question is now, what Monad shall be used. For scheduling purposes (and maybe also some models require it in general), agents should be able to \textit{read} the current simulation time: this is accomplished through a \textit{ReaderT Time}. Further, agents should be able to \textit{read} the identities of the other agents available in the simulation so they can schedule events to them when necessary: this is accomplished through a \textit{ReaderT [AgentId]}. Most importantly, agents have to be able to schedule events, meaning they have to be able to \textit{write} the events into some sink where they are accumulated for scheduling: this is accomplished through a \textit{WriterT [QueueItem e]}. Finally, the transformer stack needs to be extendible by other Monads, specified in concrete models like the SIR below, so we add another polymorphic type \textit{m}, indicating the closing Monad (stack).

\begin{HaskellCode}
type ABSMonad m e   = ReaderT Time (WriterT [QueueItem e] (ReaderT [AgentId] m))
type AgentMSF m e o = MSF (ABSMonad m e) e o
\end{HaskellCode}

Note that the \textit{ReaderT Time} is the same as bearriver SF but the semantics are different, not time-delta but absolute time)

- scheduling

\subsection{Tagless Final}
The main benefit of a \textit{tagless final} approach is that it is a solution to the expression problem \cite{kiselyov_typed_2012}: it is possible to add new interpreters of an embedded language and add new functionality without breaking the existing implementations. Interpretation in our case means that we can use different underlying Monads to run the agents in: if we want to guarantee purity, no IO Monad shall be used in  the interpreter. Otherwise when concurrency with a lock-based approach or a lock-free approach is required IO or STM can be used in the underlying interpreter. Also, for reproducible unit testing testing, one can write custom test-interpreters where methods always return a-priori known results, similar to mocking.
Adding new functionality is less an issue here but might become highly important when designing a more general ABS library in Haskell, building on the \textit{tagless final} approach. It would allow the user of such a library to extend existing agents or default behaviour with new, custom-built methods, without breaking the existing ones.

\section{Advanced Features}
\label{sec:advanced_eventdriven_ABS}

In the previous section we established the basics of event-driven ABS. It is now clear how events are represented, how agent identity is handled, how agents receive and schedule events, how events are scheduled and domain state is sampled. Furthermore, by using the \textit{tagless final} approach, we arrived at an elegant, extensible and robust solution, which separates specification, the agent and its behaviour, from its implementation, a \textit{pure} interpreter. 

In this section we present more advanced concepts of event-driven ABS, necessary in models with much higher complexity than the simple SIR. We developed these concepts using the Sugarscape model as introduced in Chapter \ref{sec:sugarscape}. Consequently we will discuss them from this model's perspective. More specifically, we show how to create and remove agents dynamically during simulation, add a shared mutable environment, model local mutable agent state and finally how synchronous agent interactions can be implemented. Together with the basics of event-driven ABS, with these concepts established it should be possible to implement a wide range of event-driven ABS models. For this we developed a full implementation of the Sugarscape model, in which we explored the concepts presented in this chapter, with the code accessible from the \href{https://github.com/thalerjonathan/haskell-sugarscape}{code repository}~\cite{thaler_sugarscape_repository}.

\section{Case-Study II: Sugarscape}
TODO: 
we can implement everything except synchronous direct agent-interactions atm: if agent-interaction is one-way e.g. paying back a loan then this is no problem. thus the following parts of the Sugarscape are not possible with our current STM approach: mating, trading and lending  because all 3 require direct agent-to-agent interaction over multiple steps. We leave the problem of developing such an algorithm / implementation for further research.

\subsection{Dynamic agent creation and removal}
Some models of ABS in general and Sugarscape in particular require the dynamic creation and removal of agents during simulation. The specific requirements here are that the agents themselves must be able to both remove themselves from the simulation and create new agents with given attributes. To achieve that, in such a simulation the output type of an agent must be richer than the one in the event-driven SIR. First, we define the output of an agent:

\begin{HaskellCode}
data AgentOut m e o = AgentOut
  { aoKill   :: Any              -- True if this agent should be removed 
  , aoCreate :: [AgentDef m e o] -- a list of agents to create
  , aoEvents :: [(AgentId, e)]   -- a list of events (receiver, event)
  }
\end{HaskellCode}

Note that \textit{AgentOut} contains already the list of scheduled events, which makes it clear that scheduling of events in this approach is implemented different than in the event-driven SIR, where the agents Monad stack had a \textit{WriterT} to write events to. The reason for that is that we treat agent-local abstractions different here because of the need to encapsulate local agent state as explained in subsequent sections.

If the agent wants to remove itself from the simulation, it simply sets \textit{aoKill} to True; it it wants to create new agents it adds an agent definition \textit{AgentDef} to the \textit{aoCreate} list. The agent definition \textit{AgentDef} holds the new id of the agent (note that an agent-controlled id makes it possible to re-use ids in case an agent dies and in case ids have no other purposes than identifying event receivers in a model), the MSF of the agent to create and the initial output of the new agent, so it has a representation in the visual or textual output.

\begin{HaskellCode}
data AgentDef m e o = AgentDef
  { adId      :: AgentId         -- unique agent-id
  , adMSF     :: AgentMSF m e o  -- the agent behaviour function
  , adInitOut :: o               -- the value of the initial output
  }
\end{HaskellCode}

Further, the simulation must provide a \textit{global} mechanism to create new, unique \textit{AgentId} for the newly created agents. The generating of ids for the new agents have to occur within the parent agents themselves because in some models they might need this very id to communicate with their children - an indirection through the kernel would only complicate matters. We thus start with a data definition, holding the next agent id - if an agent creates a new agent it simply reads that value and increments it by 1.

\begin{HaskellCode}
data ABSState = ABSState { absNextId :: AgentId }
\end{HaskellCode}

To make it \textit{globally} available to all agents a \textit{StateT ABSState} Monad transformer is used, which is also the innermost Monad of the Monad stack of Sugarscape \footnote{In the Sugarscape implementation, \textit{ABSState} also holds the current simulation time, which is omitted here for clarity reasons.}.

\begin{HaskellCode}
type AgentMonad m = StateT ABSState m
\end{HaskellCode}

Finally, we can define the polymorphic type of the agent MSF, as it is used in Sugarscape, where it is parametrised with model specific types (see next sections). It is similar to the event-driven SIR, where the agent takes the \textit{ABSEvent} as input but the output is now a tuple of \textit{AgentOut} and the polymorphic agent output type \textit{o}. The reason why the output type \textit{o} is not part of \textit{AgentOut} is to keep \textit{AgentOut} a Monoid, which allows accumulative / iterative change to \textit{AgentOut}, which is important for scheduling events, as explained in the agent-local abstractions below.

\begin{HaskellCode}
type AgentMSF m e o = MSF (AgentMonad m) (ABSEvent e) (AgentOut m e o, o)
\end{HaskellCode}

\subsection{Shared mutable pro-active environment}
In many agent-based models, agents are placed on a discrete 2D grid environment and can move around and interact with the environment. Often, there exist specific constraints, for example that each position can only be occupied by one agent at most. This requires specific iteration semantics, which make it impossible that two agents end up at the same time on the same spot. In general, such models solve this problem by using the sequential-strategy as described in Chapter \ref{sec:seq_strategy}, where agents are run in random order, one after another. This allows the agents to access the globally shared, mutable environment exclusively when its their turn and interact and change it without the danger of other agents interfering.

To implement a shared mutable and pro-active environment, first we define a generic discrete 2D grid environment, with polymorphic cells. The selection of the right data structure is crucial. Initially we used an \textit{IArray} from the array package. This data-structure has excellent read performance but in performance tests we experienced serious performance and memory leak issues with updates, leading to allocation of about 40 MByte / second on our machine. Clearly this is unacceptable for simulation purposes, which often requires software to run for hours, and thus needs a constant memory consumption and must prevent even slowly linearly increasing memory usage under all costs. The solution was to switch to \textit{IntMap} from the containers package as an underlying data structure which solved both the performance and memory leak issues.

\begin{HaskellCode}
type Discrete2dCoord  = (Int, Int)
type Discrete2dCell c = (Discrete2dCoord, c)
type Discrete2d c     = Map.IntMap (Discrete2dCell c)
\end{HaskellCode}

Having introduced the \textit{AgentMSF} and fixed the \textit{AgentMonad} with the \textit{StateT ABSState} as innermost Monad, adding a globally shared, mutable environment is straightforward. The solution is to simply add another \textit{StateT} transformer with the given environment as type. Below, we give the parametrised definition as in the Sugarscape implementation. Note that Sugarscape closes the Monad stack with the \textit{Rand} Monad as stochastics play an important role in the Sugarscape model as well. Therefore, a full expansion of the Monad stack used in Sugarscape is  \textit{StateT ABState (StateT SugEnvironment (Rand g))}.

\begin{HaskellCode}
data SugEnvSite = SugEnvSite 
  { sugEnvSiteSugarLevel    :: Double
  , sugEnvSiteSpiceLevel    :: Double
  , sugEnvSitePolutionLevel :: Double
  ...
  }

type SugEnvironment  = Discrete2d SugEnvSite
type SugAgentMonad g = AgentMonad (StateT SugEnvironment (Rand g))
\end{HaskellCode}

When implementing pro-activity of the environment, we must make a clear distinction between the environments data structure, how agents access it and the environments behaviour. In the Sugarscape model, the behaviour of the environment is quite trivial: it simply regrows resources over time and diffuses pollution in case pollution is turned on. This behaviour is achieved by providing a pure function without any monadic context or MSF. This is not necessary because the environment how we implement it, does not encapsulate local state and it does not interact with agents through messages and vice versa. Thus a pure function which maps the environment to the environment is enough: \textit{Time $\rightarrow$ SugEnvironment $\rightarrow$ SugEnvironment}. Further it also takes the current simulation time so it can implement seasons, where the speed of regrowth of resources is different in different regions and swaps after some time. This function is called in the simulation kernel after every \textit{Tick}.

\medskip

Generally, one can distinguish between four different types of environments in ABS:

\begin{enumerate}
	\item \textit{Passive read-only} - implemented in Chapter \ref{sec:adding_env}, where the environment itself is not modelled as an active process and is static information, e.g. a list of neighbours, passed to each agent. The agents cannot change the environment actively - in the case of Chapter \ref{sec:adding_env} this is enforced at compile time by simply making it read only, as input to the agent but not adding it to the output type of an agent. Note the agents change the environment implicitly by changing their state but there is no notion of an active environment process.
	
	\item \textit{Passive read/write} - The environment is just shared data, which can be accessed and manipulated by the agents. Note that this forces some arbitration mechanism to prevent conflicting updates e.g. running the agents sequentially one after the other, to ensure that only one agent has access at a time.
	
	\item \textit{Active read/write} - as implemented above. To make it active a pure function is used where the environment data is owned by the simulation kernel and then made available to the agents through a \textit{State} Monad. Another approach would be to implement the environment process as an agent, which is run together with all the other agents. This allows the environment to send and receive messages but the guarantees about when the environment will be run is lost if agents are run sequentially in random order.
	
	\item \textit{Active read-only} - can be implemented as above but instead of providing the environment data through a \textit{State} Monad, a \textit{Reader} Monad is used. The environment data is owned by the Simulation kernel and the process runs as a pure function as before but the data is provided in a read-only way through the \textit{Reader} Monad. Note that the same can also be achieved by passing it as input only to the agent as was done in Chapter \ref{sec:adding_env}.
\end{enumerate}

\subsection{Agent-Local Abstractions}
After having established Sugarscape's full Monad stack, we can now move on to specify the agent behaviour and develop advanced agent-local concepts and abstractions. Before we can parametrise the \texttt{AgentMSF}, we need to define model-specific data definitions for the event type \texttt{e} and the output type \texttt{o}. Thus, we define the event type \texttt{SugEvent}, which defines all the event types of Sugarscape and the output type \texttt{SugAgentObservable}, which contains all observable properties, an agent wants to communicate to the outside world, for visualisation or exporting purposes. 

\begin{HaskellCode}
data SugEvent = MatingRequest AgentGender
              | MatingReply 
                 (Maybe (Double, Double, Int, Int, CultureTag, ImmuneSystem))
              ...

data SugAgentObservable = SugAgentObservable
  { sugObsSugMetab :: Int     -- metabolism
  , sugObsSugLvl   :: Double  -- sugar wealth
  , sugObsAge      :: Int     -- current age
  ...  
  }
\end{HaskellCode}

We can now parametrise the \texttt{AgentMSF} with the right types for the Sugarscape model.

\begin{HaskellCode}
type SugAgentMSF g = AgentMSF (SugAgentMonad g) SugEvent SugAgentObservable
\end{HaskellCode}

Next, we define the type of the top-level agent behaviour function. We want to make the unique agent id and the model configuration (scenario) explicit, so it will be passed as curried arguments to the function. Furthermore, the initial agent state is passed as curried input as well.

\begin{HaskellCode}
data SugarScapeScenario = SugarScapeScenario 
  { sgScenarioName    :: String
  , sgPollutionActive :: Bool
  ...
  }

data SugAgentState = SugAgentState
  { sugAgSugarMetab :: Int     -- metabolism
  , sugAgVision     :: Int     -- vision in all four directions
  , sugAgSugarLevel :: Double  -- sugar wealth
  , ...
  }
  
type SugarScapeAgent g 
       = SugarScapeScenario -> AgentId -> SugAgentState -> SugAgentMSF g
\end{HaskellCode}

Now we have fully specified types for the Sugarscape agent. The types indicate very clearly the intention and the interface. What is of high importance is that we don't have any impure \texttt{IO} monadic context anywhere in our type definitions and we can also guarantee that it will not somehow sneak in. The transformer stack of the agents \texttt{MSF} is closed through the \texttt{Rand} Monad, consequently it is simply not possible to add other layers. 

An agent is fully represented by a top level \texttt{SugarScapeAgent} function, which encapsulates the whole agent behaviour. Next we will look at how to define agent-local behaviour, which is hidden behind the \texttt{SugarScapeAgent} function type. Whereas the previously defined types are exposed to the whole simulation, the following section deals with types and behaviour which are locally encapsulated and hidden from the simulation kernel. In the next sections we show how to encapsulate the agents' state locally while retaining mutability. Further, we explain how sending of events works in the Sugarscape implementation and how to achieve read-only access to the agents unique id and the model configuration.

\subsubsection{Agent-local state}
To implement the local encapsulation of the agents' state is straightforward with MSFs as they are continuations, allowing them to capture local data using closures. Fortunately we do not need to implement the low-level plumbing, as Dunai provides us with the suitable function \texttt{feedback :: Monad m $\Rightarrow$ c $\rightarrow$ MSF m (a, c) (b, c) $\rightarrow$ MSF m a b}. It takes an initial value of type \texttt{c} and an \texttt{MSF} which takes in addition to its input \texttt{a} also the given type \texttt{c} and outputs in addition to type \texttt{b} also the type \texttt{c}, which clearly indicates the read and write property of type \texttt{c}. The function returns a new \texttt{MSF} which only operates on \texttt{a} as input and returns \texttt{b} as output by running the provided \texttt{MSF} and feeding back the \texttt{c} (with the initial \texttt{c} at the first call).

\begin{HaskellCode}
sugarscapeAgent :: RandomGen g => SugarScapeAgent g
sugarscapeAgent scen aid s0 = feedback s0 (proc (evt, s) -> do ... )
\end{HaskellCode}

Before we can move on to write a function handling incoming events, we need to define the \textit{agent-local} Monad stack. The event handler must be able to manipulate the agent-local state we just encapsulated through \texttt{feedback}, support reading the unique agent id and model scenario and scheduling of events.

Providing the local, mutable agent state is done using a \texttt{State} Monad. Providing the model configuration (scenario) and the unique agent id is done using the \texttt{Reader} Monad. For implementing event scheduling, a \texttt{Writer} Monad is used, which is the same mechanism as in the event-driven SIR. As the Monoid type for \texttt{WriterT}, the \texttt{AgentOut} is used. All fields of its data definition are Monoid instances, making \texttt{AgentOut} a Monoid as well, thus writing a type class instance for it is trivial. This approach allows to easily add new agent definitions and mark an agent for removal throughout the agents behaviour. Further, it is simple to send new events through \texttt{AgentOut} as it contains the list of events, as discussed in the previous section \ref{sec:dynamic_creationremoval}. Having established this, we define the agent local Monad which is only used \textit{within} \texttt{AgentMSF}.

\begin{HaskellCode}
type AgentLocalMonad g = WriterT (SugAgentOut g) 
                           (ReaderT (SugarScapeScenario, AgentId) 
                             (StateT SugAgentState (SugAgentMonad g)))     
-- FULLY EXPANDS TO:
-- WriterT (SugAgentOut g) 
--  (ReaderT (SugarScapeScenario, AgentId) 
--    (StateT SugAgentState 
--      (StateT ABSState 
--        (StateT SugEnvironment 
--          (Rand g)))))
\end{HaskellCode}

Now we can define the \texttt{MSF} which handles an event. It has the \\ \texttt{AgentLocalMonad} monadic context, takes an \texttt{ABSEvent} parametrised over \\ \texttt{SugEvent} (it has also to handle \texttt{Tick}). What might come as a surprise is that it returns unit type, implying that the results of handling an event are only visible as side effects in the Monad stack. This is intended. We could pass all arguments explicitly as input and output but that would complicate the handling code substantially, thus we opted for a monadic, imperative style handling of events.

\begin{HaskellCode}
type EventHandler g = MSF (AgentLocalMonad g) (ABSEvent SugEvent) ()
\end{HaskellCode}

To run the handler within the \texttt{SugarScapeAgent}, we make use of Dunai's functionality which provides functions to run MSFs with additional monadic layers within MSFs. We use \texttt{runStateS}, \texttt{runReaderS} and \texttt{runWriterS} (\texttt{S} indicates the stream character) to run the \texttt{generalEventHandler}, providing the initial values for the respective Monads, \texttt{s} for the \texttt{StateT}, \texttt{(params, aid)} for the \texttt{ReaderT} and the \texttt{evt} as the input to the event handler. \texttt{WriterT} does not need an initial value, it will be provided through the Monoid instance of \texttt{AgentOut}.

\begin{HaskellCode}
sugarscapeAgent :: RandomGen g => SugarScapeAgent g
sugarscapeAgent scen aid s0 = feedback s0 (proc (evt, s) -> do
  (s', (ao', _)) <- runStateS 
                      (runReaderS 
                        (runWriterS generalEventHandler)) -< (s,((scen,aid),evt))
  let obs = sugObservableFromState s
  returnA -< ((ao', obs), s'))

sugObservableFromState :: SugAgentState -> SugAgentObservable
generalEventHandler :: RandomGen g => EventHandler g
\end{HaskellCode}

Now it is also clear why the output of an agent is a tuple of \texttt{AgentOut} and the polymorphic type \texttt{o}: the latter one is parametrised to \texttt{SugAgentObservable}, which is not constructed through the use of \texttt{WriterT} but simply a projection of the agent state through \texttt{sugObservableFromState}. In the next section we explain the details of \texttt{generalEventHandler}, which implements the main event handling mechanisms of an agent.

\subsubsection{Handling and sending of events}
Now we are ready to implement handling of events on an agent-local level: we receive the events from the simulation kernel as input and run within a six-layered Monad Transformer stack which is partly global, controlled by the simulation kernel, and partly local to the agent, controlled by the agent itself. The layers are the following (inner to outer):

\begin{enumerate}
	\item \texttt{WriterT (SugAgentOut g)} - local; provides write-only functionality for constructing the agent output for the simulation kernel communicating whether to kill the agent, a list of new agents to create and a list of events to send to receiving agents.
	
	\item \texttt{ReaderT (SugarScapeScenario, AgentId)} - local; provides the read-only model configuration and unique agent id.

	\item \texttt{StateT SugAgentState} - local; provides the local mutable agent state.

	\item \texttt{StateT ABSState} - global; provides unique agent ids for new agents. %and the current simulation time. The usage of a \textit{StateT} is slightly flawed here because it provides too much power: the current simulation time should be read-only to the agent. Drawing the next agent-id involves reading the current id and writing the incremented value, thus technically it is a \textit{StateT} but ideally we would like to hide the writing operation and only provide a \textit{read-current-and-increment} operation. A possible solution would be to provide the current simulation time through a \textit{ReaderT} and the new agent-id through a new monad which uses the \textit{StateT} under the hood, like the \textit{Rand} monad.

	\item \texttt{StateT SugEnvironment} - global; provides the Sugarscape environment which the agents can read and write.

	\item \texttt{Rand g} - global; provides the random-number stream for all agents.
\end{enumerate}

Finally we can implement the event handler \texttt{generalEventHandler}, which simply matches on the incoming events, extracts data and dispatches to respective handlers. What is crucial here to understand is that only the top level \texttt{sugarscapeAgent} and the \texttt{EventHandler} function are MSFs which simply dispatch to monadic functions, implementing the functionality in an imperative programming style. The main benefit of the MSFs are their continuation character, which allows to encapsulate local state. An additional benefit of MSFs is that the  Dunai library adds a lot of additional functionality of composing MSFs and running different monadic context on top of each other. It even provides exception handling through MSFs with the \texttt{Maybe} type, thus programming with exceptions in ABS models can be done as well. We didn't make use of exceptions, as the Sugarscape model simply does not specify any exception handling on the model level and there was also no opportunity to use exceptions from which to recover on a technical level \footnote{There are exceptions on a technical level but they are non-recoverable and should never occur at runtime, thus the function \texttt{error} is used, which terminates the simulation with an error message.}.

\begin{HaskellCode}              
generalEventHandler :: RandomGen g => EventHandler g
generalEventHandler =
  continueWithAfter -- optionally switching the top event handler 
    (proc evt -> 
      case evt of 
        Tick dt -> do
          mhdl <- arrM handleTick -< dt
          returnA -< ((), mhdl)

        (DomainEvent sender (MatingRequest otherGender)) -> do
          arrM (uncurry handleMatingRequest) -< (sender, otherGender)
          returnA -< ((), Nothing)
        ...)
        
handleTick :: RandomGen g => DTime -> AgentLocalMonad g (Maybe (EventHandler g))
handleMatingRequest :: AgentId -> AgentGender -> AgentLocalMonad g ()
\end{HaskellCode}

The use of \texttt{continueWithAfter} is a customised version of the already known \texttt{switch} combinator. It allows to swap out the event-handler for a different one, which is the foundation for the synchronous agent interactions, as discussed in the next section.

To see how an event handler works, we provide the implementation of \\ \texttt{handleMatingRequest}. It is sent by an agent to its neighbours to request whether they want to mate with this agent. The handler receives the sender and the other agents gender and replies with \texttt{sendEventTo} which sends a \texttt{MatingReply} event back to the sender. The function \texttt{sendEventTo} operates on the \texttt{WriterT} to append (using \texttt{tell}) an event to the list of events this agent sends when handling this event. The use of \texttt{agentProperty} reads the value of a given field of the local agent state. 

\begin{HaskellCode}
handleMatingRequest :: AgentId
                    -> AgentGender
                    -> AgentLocalMonad g ()
handleMatingRequest sender otherGender = do
  -- check if the agent is able to accept the mating request: 
  -- fertile + wealthy enough + different gender
  accept <- acceptMatingRequest otherGender
  -- each parent provides half of its sugar-endowment for the new-born child
  acc <- if not accept
      -- can't mate, simply send Nothing in MatingReply
      then return Nothing
      else do
        sugLvl  <- agentProperty sugAgSugarLevel
        spiLvl  <- agentProperty sugAgSpiceLevel
        metab   <- agentProperty sugAgSugarMetab
        vision  <- agentProperty sugAgVision
        culTag  <- agentProperty sugAgCultureTag
        imSysGe <- agentProperty sugAgImSysGeno -- immune system genotype
        -- able to mate, send Just share in MatingReply
        return Just (sugLvl / 2, spiLvl / 2, metab, vision, culTag, imSysGe)
  -- reply to sender with MatingReply
  sendEventTo sender (MatingReply acc)
\end{HaskellCode}

Finally, we have a look at how to actually run an agents' \texttt{MSF} using the function \texttt{runAgentMSF}. It is a \textit{pure} function as well and thus takes all input as explicit arguments. It might look like an excess to pass in five arguments and get a six-tuple as result but this is the price we have to pay for pure functional programming where everything is explicit, with all its benefits and drawbacks.

\begin{HaskellCode}
runAgentMSF :: RandomGen g        -- RandomGen type class for g
            => SugAgentMSF g      -- agents MSF to run.
            -> ABSEvent SugEvent  -- event it receives.
            -> ABSState           -- ABSState (next agent id and current time)
            -> SugEnvironment     -- environment state
            -> g                  -- random-number generator
            -> (SugAgentOut g, SugAgentObservable, SugAgentMSF g, 
                ABSState, SugEnvironment, g)
runAgentMSF msf evt absState env g = (ao, obs, msf', absState', env', g') 
  where
    -- extract the monadic function to run
    msfAbsState = unMSF msf evt
    -- peel away one State layer: ABSState
    msfEnvState = runStateT msfAbsState absState
    -- peel away the second State layer: SugEnvironment
    msfRand = runStateT msfEnvState env
    -- peel away the 3rd and last layer: Rand Monad
    (((((ao, obs), msf'), absState'), env'), g') = runRand msfRand g
\end{HaskellCode}

We run only the three \textit{global} monadic layers in here, the three \textit{local} layers are indeed completely local to the agent itself as shown above.

\subsection{Synchronous Agent interactions}
With the concepts introduced so far we can achieve already a lot in terms of agent interactions: agents can react to incoming events, which are either the \textit{Tick} event advancing simulation time by one step or a message sent by another agent (or the agent itself). This is enough to implement simple one-directional asynchronous agent interactions where one agent sends a message to another agent but does not await an answer within the same \textit{Tick}. This one-directional asynchronous interactions is used in the model to implement the passing of diseases, the paying back of debt, passing on wealth to children upon death - the agent simply sends a message and forgets about it.

Unfortunately this mechanism is not enough to implement the other agent-interactions in the Sugarscape model, which are structurally richer: they need to be synchronous. In the use-cases of mating, trading and lending two agents need to come to an agreement over multiple interactions steps within the same \textit{Tick} which need to be exclusive and synchronous.  This means that an agent A initiates such a multi-step conversation with another agent B by sending an initial message to which agent B has to react by a reply to agent A who upon reception of the message, will pick up computation from that point and reply with a new message and so on. Both agents must not interact with other agents during this conversation to guarantee resource constraints, otherwise it would become quite difficult and cumbersome to ensure that agents don't spend more than they have when trading with multiple other agents at the same time. Also the initiating agent A must be able to pick up processing of its \textit{Tick} event from the point where it started the conversation with agent B because sending a message always requires the handling of the current event to exit and hand the control back to the simulation kernel. See Figure \ref{fig:syncagentinteractions} for a visualisation of the sequence of actions.

\begin{figure}
	\centering
	\includegraphics[width=1.0\textwidth, angle=0]{./fig/eventdriven/syncagentinteractions.png}
	\caption{Sequence diagram of synchronous agent interaction in the trading use-case. Upon the handling of the \textit{Tick} event, agent A looks for trading partners and finds agent B within its neighbourhood and sends a \textit{TradingOffer} message. Agent B replies to this message with \textit{TradeReply} and agent A continues with the trading algorithm by picking up where it has left the execution when sending the message to agent B. After agent A has finished the trading with agent B, it turns to agent C, where the same procedure follows.}
	\label{fig:syncagentinteractions}
\end{figure}

The way to implement this is to allow an agent to be able to change its internal event-handling state: to switch into different event-handlers, after having sent an event, to be able to react to the incoming reply in a specific way by encapsulating local state for the current synchronous interaction through closures and currying. Further, by making use of continuations the agent can pick up the processing of the \textit{Tick} event after the synchronous agent interaction has finished. Key to this is the function \textit{continueWithAfter} which we already shortly introduced through \textit{generalEventHandler}. This function takes an MSF which returns an output of type \textit{b} and an optional MSF. If this optional \textit{Maybe} MSF is \textit{Just} then the \textit{next} input is handled by this new MSF. In case no new MSF is returned (\textit{Nothing}), the MSF will stay the same. This is a more specialised version of the \textit{switch} combinator introduced in Chapter \ref{sec:back_frp} in the way that it doesn't need an additional function to produce the actual MSF continuation. Note that the semantics are different though: whereas \textit{switch} runs the new MSF immediately, \textit{continueWithAfter} only applies the new MSF in the \textit{next} step. The implementation of the function is as follows:

\begin{HaskellCode}
continueWithAfter :: Monad m => MSF m a (b, Maybe (MSF m a b)) -> MSF m a b
continueWithAfter msf = MSF (\a -> do
  ((b, msfCont), msf') <- unMSF msf a
  let msfNext = fromMaybe (continueWithAfter msf') msfCont
  return (b, msfNext))
\end{HaskellCode}

Finally, we can discuss the \textit{Tick} handling function. It returns a value of type \textit{Maybe (EventHandler g)} which if is \textit{Just} will result in to a change of the top-level event handler through \textit{continueWithAfter} as shown in \textit{generalEventHandler} above. Note the use of continuations in the case of \textit{agentMating, agentTrade} and  \textit{agentLoan}. All these functions return a \textit{Maybe (EventHandler g)} because all of them can potentially result in synchronous agent interactions which require to change the top-level event handler. The function \textit{agentDisease} is the last in the chain of agent behaviour, thus we are passing a default continuation which simply switches back into \textit{generalEventHandler} to finish the processing of a \textit{Tick} in an agent.

\begin{HaskellCode}
handleTick :: RandomGen g => DTime -> AgentLocalMonad g (Maybe (EventHandler g))
handleTick dt = do
  -- perform ageing of agent
  agentAgeing dt
  -- agent move, returns amount it of resources it harvested
  harvestAmount <- agentMove
  -- metabolise resources, depending on agents metabolism rate
  -- returns amount metabolised
  metabAmount <- agentMetabolism
  -- polute environment given harvest and metabolism amount
  agentPolute harvestAmount metabAmount

  -- check if agent has starved to death or died of age
  ifThenElseM
    (starvedToDeath `orM` dieOfAge)
    (do
      -- died of age or starved to deat: remove from simulation
      agentDies agentMsf
      return Nothing) 
    -- still alive, perform the remaining steps of the behaviour
    -- pass agentContAfterMating as continuation to pick up after mating
    -- synchronous conversations have finished
    (agentMating agentMsf agentContAfterMating)

-- after mating continue with cultural process and trading
agentContAfterMating :: RandomGen g => AgentLocalMonad g (Maybe (EventHandler g))
agentContAfterMating = do
    agentCultureProcess
    -- pass agentContAfterTrading as continuation to pick up after trading 
    -- synchronous conversations have finished
    agentTrade agentContAfterTrading 

-- after trading continue with lending and borrowing
agentContAfterTrading :: RandomGen g  => AgentLocalMonad g (Maybe (EventHandler g))
agentContAfterTrading = agentLoan agentContAfterLoan

-- after lending continue with diseases, which is the step in a Tick event
agentContAfterLoan :: RandomGen g => AgentLocalMonad g (Maybe (EventHandler g))
agentContAfterLoan = agentDisease defaultCont

-- safter diseases imply switch back into the general event handler
defaultCont :: RandomGen g => AgentLocalMonad g (Maybe (EventHandler g))
defaultCont = return (Just generalEventHandler)
\end{HaskellCode}

\subsubsection{Synchronous through Tagless Final}
TODO: the indirect, continuation based approach is cumbersome and it is easy to get something wrong, thus synchronous interactions would be desirable. With the tagless final approach as introduced in the TODO section, this becomes possible
TODO: show the syncSend approach of SIR and discuss the problem of not being able to recursively send to itself e.g. agent A initiates syncSend, then it cannot send directly to itself or if it sends to agent B this agent B cannot send another syncSend to agent A. Or A -> B -> C -> A is also not possible, any chain which comes back to the originator leads to wrong semantics where effects might be visible but the changes to the last agent will be overridden.

further ideas for tagless final in sugarscape:
- all creation and removal related stuff create new agent does not require id but kernel returns new id

%\section[Implementation]{Implementation \footnote{Code available under\\ \url{https://github.com/thalerjonathan/phd/tree/master/coding/papers/iteratingABM/}}}
In this section we give a brief overview of comparing implementing the update-strategies in three languages which fundamentally differ among each other. We wanted to cover a wide range of different types of languages but didn't include a language where the memory-management falls in the hands of the developer. This would be the case e.g. in C++. This was looked into partially by \cite{dawson_opening_2014} but the focus of this paper is not on this issue as it would complicated things dramatically. All used languages are garbage-collected / the developer does not need to care how memory is cleared up.

For testing the suitability we selected a variety of simple models we implemented in each language with mostly all strategies. The selected models are \textit{Heroes \& Cowards}, \textit{SIRS}, \textit{Wildfire} and the \textit{Spatial Game} mentioned in \cite{huberman_evolutionary_1993}. We lack the space to explain all models but all are well known and can be easily found, looked up and understood on the Internet. They span different challenges to the ABS implementation: sending messages, accessing the environment, spawning new Agents, killing existing ones, discrete and continuous model. We also can confirm that all the reference-models proposed in \cite{isaac_abm_2011} and the StupidModel 1-16 by Railsback (TODO: cite) can be faithfully capture using our new terminology. Also we could show that the  Parallel-Strategy is the only strategy able to reproduce the pattern of the \textit{Spatial Game} due to the semantics of the model which require that all the Agents play the game at virtually the same time - which is only possible in the Parallel-Strategy.



\subsection{Java}
This language is included as the benchmark of object-oriented (OO) imperative languages as it is extremely popular in the ABS community and widely used in implementing their models and Environments. It comes with a comprehensive programming library, has nice object-oriented features built in and powerful synchronization primitives at built in at language-level. \\

\paragraph{Ease of Use}
Being experienced Java-Programmers we found that implementing all the strategies was straight-forward and easy thanks to the languages features. Especially parallelism and concurrency is quite very easy due to elegant and powerful built-in synchronization primitives so high-performance and large number of agents possible due to aliasing and side-effects.

\paragraph{Benefits}
high-performance and large number of agents possible due to aliasing and side-effects.

\paragraph{Deficits}
We couldn't identify something which absolutely didn't work, that's also why Java can be regarded as a very safe decision when opting which language to use for implementing ABS.
A downside is that one must take care when accessing memory in case of parallel or concurrent strategy. Due to the availability of aliasing and side-effects in the language and the type-system, it can't be guaranteed that access to memory happens only when its safe - something which is possible in Haskell (see below).
Care must be taken when references are sent by messages to other agents in case of parallel or concurrency

although the interfaces encourage it we cannot prevent the agents to use agent-references and directly accessing. a workaround would be to create new agent-instances in every iteration-step which would make old references useless but this doesn't protect us from concurrency issues with a current iteration (the copying must take place within a synchronized block, thus implicitly assuming ordering, something we don't want) and besides, can always work around and update the references.
that's the toll of side-effects: faster execution but less control over abuse
tried to clone agents in each step and let them collect their messages => extremely slow

\paragraph{Natural Strategy}
We think that the Sequential Strategy with Immediate Message-Handling is the most natural Strategy to express in Java due to its heavy reliance on side-effects through references (aliases) and shared thread of execution. Also most of the models work this way and its thus a save decision to start with Java.





 
\subsection{Haskell}
This language is included to put to test whether such a pure functional declarative programming language is suitable for full-blown ABS. What distinguishes it is its complete lack of implicit side-effects, global data, mutable variables and objects. The central concept is the function into which all data has to be passed in and out explicitly through statically typed arguments and return values: data-flow is completely explicit.

\paragraph{Ease of Use}
Being Haskell-Beginners we initially thought that it would be suitable at best for just implementing the Parallel-Strategy but after implementing all the strategies in it we found out that Haskell is extremely well suited to implement all of them. We think this stems from the following facts that it has no implicit side-effects which reduces bugs considerably and reveals the data-flow very explicitly.

dont have objects with methods which can call between each other but we need some way of representing agents. this is done using a struct type with a behaviour function and messaging mechanisms. important: agents are not carried arround but messages are sent to a receiver identified by an id. This is also a big difference to java where don't really need this kind of abstraction due to the use of objects and their 'messaging'. messaging mechanisms have up- and downsides, elaborate on it.

\paragraph{Benefits}
the type-system prevents us from introducing implicit side-effects, global references are only possible in code marked in its types as producing side-effects, something we explicitly avoided and were able to throughout the whole implementation. Thus arriving at a much safer version in the parallel iteration-strategies with separate threads of execution.
Also it is a must-have for STM, although it makes things more difficult in the beginning, in the end it turns out to be a blessing because one can guarantee that side-effects won't occur. We have taken care that the agents all run in side-effect free code.
concurrency and actors extremely elegant possible through: STM which only possible in languages without side-effect	
STM: implementing concurrency is a piece-of-cake
the conc-version and the act-version of the agent-implementations look EXACTLY the same	 BUT we lost the ability to step the simulation!!!

extremely powerful static typesystem: in combination with side-effect free this results in the semantics of an update-strategy to be reflected in the Agent-Transformer function and the messaging-interface. This means a user of this approach can be guided by the types and can't abuse them. Thus the lesson learned here is that \textit{if one tries to abuse the types of the agent-transformer or work around, then this is an indication that the update-strategy one has selected does not match the semantics of the model one wants to implement}. If this happens in Java, it is much more easier to work around by introducing global variables or side-effects but this is not possible in Haskell. \\
Thus our conclusion in using Haskell is that it is an extremely underestimated language in ABM/S which should be explored much more as we have shown that it really shines in this context and we believe that it could be pushed further even more.

\paragraph{Deficits}
still much work to do to capture large and complex models (see further research), performance is a big issue but this has not been about performance (2000 Agents are enough)
but still it is not suitable for big models with heterogeneous agents, there things are lacking: see further research

to implement immediate message-handling in haskell would be very cumbersome: 
drag all agents around, could run into recursion, state-handling becomes cumbersome, => leads to re-building an OO kind-of system in a pure functional language => don't do that! The conclusion is that for now this is left as the single drawback, which is not appropriately implementable using Haskell. Thats really where Java shines with its mutable Objects and Side-Effects.
TODO: could we really implement synchronous message-handling in Haskell? we would need to pass always ALL agents around and return them in every call
it is not possible to send a message directly to an agent
This is the single main drawback of the Haskell implementation and although it only shows up in the Seq Strategy it would be of interest if there is an elegant, pure functional software-architecture which allows messages sent from Agent A to Agent B be immediately - as in: the same execution thread - handled by Agent B. 

\paragraph{Natural Strategy}
note that the difference between SEQ and PAR in Haskell is in the end a 'fold' over the agents in the case of SEQ and a 'map' in the case of PAR




\subsection{Scala with Actors}
This multi-paradigm functional language is included to test the usefulness of the \textit{Act} strategy for implementing ABM/S. The language comes with an Actor-library inspired by \cite{agha_actors:_1986} and resembles the approach of Erlang which allows a very natural implementation of the strategy.

\paragraph{Ease of Use}

\paragraph{Benefits}

\paragraph{Deficits}

\paragraph{Natural Strategy}

%\input{./tex/eventdriven/eventdrivenSIR.tex}

\section{Discussion}

\subsection{Other Models}
TODO: mention that we have also implemented other models, which also work without time-semantics (all agents make a move at discrete time-steps and do not really rely on some notion of time). 

\subsection{Time-Semantics}
The main reason for building our pure functional ABMS approach on top of Yampa was to leverage the powerful time-semantics of Yampa which allows us to implement important concepts of ABMS:

state-chart: agents are at all time of their life-cycle in one state and can switch between multiple states using transitions 
timed transitions: transition to another state/behaviour happens at a discrete time
rate transitions: transition happens with a given rate
message transition: transition upon receiving a given message 

\subsection{Agents as Signals}
Due to the underlying nature and motivation of Functional Reactive Programming (und im speziellen) Yampa, Agents can be seen as Signals which is generated and consumed by a Signal-Function which is the behaviour of an Agent. If an Agent does not change the OUTPUT-signal is constant, if the agent changes e.g. by sending a message, changing its state,... the OUTPUT signal changes. A dead agent has no signal at all.

\subsection{Time-Sampling}
sampling rate depends on the transition times \& rates of the model. when e.g. the contact rate is 5 then the sampling dt should be below 0.2

\subsection{System Dynamics}
can emulate system dynamics due to the parallel update-strategy and continuous time-flow semantics

\subsection{Discrete Event Simulation}
DES in FrABMS? how easily can we implement server/queue systems? do they also look like a specification? potential problem: ordering of messages is not guaranteed by now

\subsection{Advantages}
advantages:
	- no side-effects within agents leads to much safer code
	- edsl for time-semantics
	- declarative style: agent-implementation looks like a model-specification
	- reasoning and verification
	- sequential and parallel
	- powerful time-semantics
	- arrowized programming is optional and only required when utilizing yampas time-semantics. if the model does not rely on time-semantics, it can use monadic-programming by building on the existing monadic functions in the EDSL which allow to run in the State-Monad which simplifies things very much
	- when to use yampas arrowized programing: time-semantics, simple state-chart agents 
	- when not using yampas facilities: in all the other cases e.g. SugarScape is such a case as it proceeds in unit time-steps and all agents act in every time-step
	- can implement System Dynamics building on Yampas facilities with total ease	
	- get replications for free without having to worry about side-effects and can even run them in parallel without headaches
	- cant mess around with time because delta-time is hidden from you (intentional design-decision by Yampa). this would be only very difficult and cumbersome to achieve in an object-oriented approach. TODO: experiment with it in Java - how could we actually implement this? I think it is impossible: may only achieve this through complicated application of patterns and inheritance but then has the problem of how to update the dt and more important how to deal with functions like integral which accumulates a value through closures and continuations. We could do this in OO by having a general base-class e.g. ContinuousTime which provides functions like updateDt and integrate, but we could only accumulate a single integral value.
	- reproducibility statically guaranteed
	- cannot mess around with dt
	- code == specification
	- rule out serious class of bugs
	- different time-sampling leads to different results e.g. in wildfire \& SIR but not in Prisoners Dilemma. why? probabilistic time-sampling?
	- reasoning about equivalence between SD and ABS implementation in the same framework
	- recursive implementations
	
	- we can statically guarantee the reproducibility of the simulation because: no side effects possible within the agents which would result in differences between same runs (e.g. file access, networking, threading), also timedeltas are fixed and do not depend on rendering performance or userinput	
	
\subsection{Disadvantages}
disadvantages:
	- performance is low
	- reasoning about performance is very difficult
	- very steep learning curve for non-functional programmers
	- learning a new EDSL
	- think ABMS different: when to use async messages, when to use sync conversations


[ ] important: increasing sampling freqzency and increasing number of steps so that the same number of simulation steps are executed should lead to same results. but it doesnt. why?
[ ] hypothesis: if time-semantics are involved then event ordering becomes relevant for emergent patterns. there are no tine semantics in heroes and cowards but in the prisoners dilemma
[ ] can we implement different types of agents interacting with each other in the same simulation ? with different behaviour funcs, digferent state? yes, also not possible in NetLogo to my knowledge. but they must have the same messages, emvironment 

[ ] Hypothesis: we can combine with FrABS agent-based simulation and system dynamics (this has been proved by example!)