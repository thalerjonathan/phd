\chapter{Pure Functional Event-Driven ABS}
\label{ch:eventdriven}

In this chapter we build on the previous discussion of update-strategies in Chapter \ref{ch:impl_abs} and the implementation techniques presented in the time-driven approach of Chapter \ref{ch:timedriven} to develop concepts for event-driven ABS in a pure functional way. 

In event-driven ABS \cite{meyer_event-driven_2014}, the simulation is advanced through events: agents and the environment schedule events into the future and react to incoming events scheduled by themselves, other agents, the environment or the simulation kernel. Time is discrete in this approach: it advances step-wise from event to event, where each event has an associated time-stamp, indicating the virtual simulation time when it is scheduled. This implies that time could stay constant e.g. when an event is scheduled with a time-delay of 0 the virtual simulation time does not advance. Because agents can adopt and change their state and behaviour when processing an event, this means that even if time does not advance, agents can change. This non-signal behaviour is the fundamental difference to the time-driven approach in Chapter \ref{ch:timedriven}. Further, we exploit this mechanism to implement direct agent-interactions pure functionally as discussed in the implementation discussion below.

The event-driven approach makes the simulation kernel technically closely related to a Discrete Event Simulation (DES) \cite{zeigler_theory_2000}. Due to the necessity of imposing a correct ordering of events in this type of ABS, we need to step it event by event, with the \textit{sequential} update-strategy, as introduced in Chapter \ref{sec:seq_strategy}, being the only feasible one for this type of ABS. Note that there exists also Parallel DES (PDES) \cite{fujimoto_parallel_1990}, which processes events in parallel and deals with inconsistencies by reverting to consistent states - we hypothesize that a pure functional approach could be beneficial in such an approach due to persistent data-structures and explicit handling of side-effects but we leave this for further research.

\medskip

First we start using an event-driven agent-based SIR model, inspired by \cite{macal_agent-based_2010} to introduce the concepts of agent identity and also event scheduling utilising both \textit{Reader} and \textit{Writer} Monads which we then generalise to a \textit{tagless final} approach. Note that in event-driven ABS, due to the fact that agents are not signals any more, we abandon time-aware signal functions from the previous chapter and focus solely on Monadic Stream Functions (MSFs).

We then use the highly complex Sugarscape model as introduced in Chapter \ref{sec:sugarscape}, to develop more advanced features of ABS in a pure functional context: dynamic creation and removal of agents during simulation, reading/writing from a shared mutable environment, local mutable agent state and synchronised direct agent-interactions. 
Note that the Sugarscape model is not a real event-driven model like the event-driven SIR one is: in it the agents do schedule events but they don't do this into the future - events in Sugarscape don't have associated time-stamps.

\section[Implementation]{Implementation \footnote{Code available under\\ \url{https://github.com/thalerjonathan/phd/tree/master/coding/papers/iteratingABM/}}}
In this section we give a brief overview of comparing implementing the update-strategies in three languages which fundamentally differ among each other. We wanted to cover a wide range of different types of languages but didn't include a language where the memory-management falls in the hands of the developer. This would be the case e.g. in C++. This was looked into partially by \cite{dawson_opening_2014} but the focus of this paper is not on this issue as it would complicated things dramatically. All used languages are garbage-collected / the developer does not need to care how memory is cleared up.

For testing the suitability we selected a variety of simple models we implemented in each language with mostly all strategies. The selected models are \textit{Heroes \& Cowards}, \textit{SIRS}, \textit{Wildfire} and the \textit{Spatial Game} mentioned in \cite{huberman_evolutionary_1993}. We lack the space to explain all models but all are well known and can be easily found, looked up and understood on the Internet. They span different challenges to the ABS implementation: sending messages, accessing the environment, spawning new Agents, killing existing ones, discrete and continuous model. We also can confirm that all the reference-models proposed in \cite{isaac_abm_2011} and the StupidModel 1-16 by Railsback (TODO: cite) can be faithfully capture using our new terminology. Also we could show that the  Parallel-Strategy is the only strategy able to reproduce the pattern of the \textit{Spatial Game} due to the semantics of the model which require that all the Agents play the game at virtually the same time - which is only possible in the Parallel-Strategy.



\subsection{Java}
This language is included as the benchmark of object-oriented (OO) imperative languages as it is extremely popular in the ABS community and widely used in implementing their models and Environments. It comes with a comprehensive programming library, has nice object-oriented features built in and powerful synchronization primitives at built in at language-level. \\

\paragraph{Ease of Use}
Being experienced Java-Programmers we found that implementing all the strategies was straight-forward and easy thanks to the languages features. Especially parallelism and concurrency is quite very easy due to elegant and powerful built-in synchronization primitives so high-performance and large number of agents possible due to aliasing and side-effects.

\paragraph{Benefits}
high-performance and large number of agents possible due to aliasing and side-effects.

\paragraph{Deficits}
We couldn't identify something which absolutely didn't work, that's also why Java can be regarded as a very safe decision when opting which language to use for implementing ABS.
A downside is that one must take care when accessing memory in case of parallel or concurrent strategy. Due to the availability of aliasing and side-effects in the language and the type-system, it can't be guaranteed that access to memory happens only when its safe - something which is possible in Haskell (see below).
Care must be taken when references are sent by messages to other agents in case of parallel or concurrency

although the interfaces encourage it we cannot prevent the agents to use agent-references and directly accessing. a workaround would be to create new agent-instances in every iteration-step which would make old references useless but this doesn't protect us from concurrency issues with a current iteration (the copying must take place within a synchronized block, thus implicitly assuming ordering, something we don't want) and besides, can always work around and update the references.
that's the toll of side-effects: faster execution but less control over abuse
tried to clone agents in each step and let them collect their messages => extremely slow

\paragraph{Natural Strategy}
We think that the Sequential Strategy with Immediate Message-Handling is the most natural Strategy to express in Java due to its heavy reliance on side-effects through references (aliases) and shared thread of execution. Also most of the models work this way and its thus a save decision to start with Java.





 
\subsection{Haskell}
This language is included to put to test whether such a pure functional declarative programming language is suitable for full-blown ABS. What distinguishes it is its complete lack of implicit side-effects, global data, mutable variables and objects. The central concept is the function into which all data has to be passed in and out explicitly through statically typed arguments and return values: data-flow is completely explicit.

\paragraph{Ease of Use}
Being Haskell-Beginners we initially thought that it would be suitable at best for just implementing the Parallel-Strategy but after implementing all the strategies in it we found out that Haskell is extremely well suited to implement all of them. We think this stems from the following facts that it has no implicit side-effects which reduces bugs considerably and reveals the data-flow very explicitly.

dont have objects with methods which can call between each other but we need some way of representing agents. this is done using a struct type with a behaviour function and messaging mechanisms. important: agents are not carried arround but messages are sent to a receiver identified by an id. This is also a big difference to java where don't really need this kind of abstraction due to the use of objects and their 'messaging'. messaging mechanisms have up- and downsides, elaborate on it.

\paragraph{Benefits}
the type-system prevents us from introducing implicit side-effects, global references are only possible in code marked in its types as producing side-effects, something we explicitly avoided and were able to throughout the whole implementation. Thus arriving at a much safer version in the parallel iteration-strategies with separate threads of execution.
Also it is a must-have for STM, although it makes things more difficult in the beginning, in the end it turns out to be a blessing because one can guarantee that side-effects won't occur. We have taken care that the agents all run in side-effect free code.
concurrency and actors extremely elegant possible through: STM which only possible in languages without side-effect	
STM: implementing concurrency is a piece-of-cake
the conc-version and the act-version of the agent-implementations look EXACTLY the same	 BUT we lost the ability to step the simulation!!!

extremely powerful static typesystem: in combination with side-effect free this results in the semantics of an update-strategy to be reflected in the Agent-Transformer function and the messaging-interface. This means a user of this approach can be guided by the types and can't abuse them. Thus the lesson learned here is that \textit{if one tries to abuse the types of the agent-transformer or work around, then this is an indication that the update-strategy one has selected does not match the semantics of the model one wants to implement}. If this happens in Java, it is much more easier to work around by introducing global variables or side-effects but this is not possible in Haskell. \\
Thus our conclusion in using Haskell is that it is an extremely underestimated language in ABM/S which should be explored much more as we have shown that it really shines in this context and we believe that it could be pushed further even more.

\paragraph{Deficits}
still much work to do to capture large and complex models (see further research), performance is a big issue but this has not been about performance (2000 Agents are enough)
but still it is not suitable for big models with heterogeneous agents, there things are lacking: see further research

to implement immediate message-handling in haskell would be very cumbersome: 
drag all agents around, could run into recursion, state-handling becomes cumbersome, => leads to re-building an OO kind-of system in a pure functional language => don't do that! The conclusion is that for now this is left as the single drawback, which is not appropriately implementable using Haskell. Thats really where Java shines with its mutable Objects and Side-Effects.
TODO: could we really implement synchronous message-handling in Haskell? we would need to pass always ALL agents around and return them in every call
it is not possible to send a message directly to an agent
This is the single main drawback of the Haskell implementation and although it only shows up in the Seq Strategy it would be of interest if there is an elegant, pure functional software-architecture which allows messages sent from Agent A to Agent B be immediately - as in: the same execution thread - handled by Agent B. 

\paragraph{Natural Strategy}
note that the difference between SEQ and PAR in Haskell is in the end a 'fold' over the agents in the case of SEQ and a 'map' in the case of PAR




\subsection{Scala with Actors}
This multi-paradigm functional language is included to test the usefulness of the \textit{Act} strategy for implementing ABM/S. The language comes with an Actor-library inspired by \cite{agha_actors:_1986} and resembles the approach of Erlang which allows a very natural implementation of the strategy.

\paragraph{Ease of Use}

\paragraph{Benefits}

\paragraph{Deficits}

\paragraph{Natural Strategy}

\section{Event-Driven SIR}
\label{sec:eventdriven_sir}
TODO REFACTORING
- can we follow a cleaner Tagless Final approach like i prototyped in SIR? this allows easier extensibility in both dimensions: adding new operations (what agents are allowed to do) and new interpreter (pure functional, STM concurrency, IO concurrency). I will learn a great deal about this, it will be a strong selling point and it might event allow to implement synchronous interactions??

This short section shows how to implement the SIR model, as introduced in Chapter \ref{sec:sir_model}, with an event-driven approach. This is in stark contrast to the time-driven implementation in Chapter \ref{sec:timedriven_firststep}. The solutions are quantitatively equal as they produce the same class of dynamics. Qualitatively they fundamentally differ though in terms of expressivity and performance as we will see below.

To keep this section simple, we reduce code-examples as far as possible and focus on the most fundamental differences to the approach used in Sugarscape. An agent in the event-driven SIR has no output (that is: the return-type of the MSF is the empty tuple ()), because the SIR model is much less dynamic than the Sugarscape one: agents don't spawn other agents and agents can't die. Further there is no environment (see Chapter \ref{sec:timedriven_firststep} how to add an environment to the SIR model) and observable dynamics happen not through agent-output but through side-effects.

The very heart of this implementation is the simulation state, which holds (amongst others) a priority queue and a tuple with the number of susceptible, infected and recovered agents. Agents schedule events with a time-stamp and receiver agent id, using this priority queue, which the simulation kernel processes then in order of the time-stamps to run the next agent. This is conceptionally very close to the Sugarscape implementation but events have now an additional time-stamp, which indicates the time when they are about to be scheduled - the priority queue is sorted according to the time-stamps and the simulation kernel simply processes them in order. When agents change their state they also increment / decrement the number of susceptible / infected / recovered agents, depending on which transition they make.

This makes the structure of the whole implementation much smaller than the one of Sugarscape: there is no local monad transformer stack to the agent, only a global one, which holds the simulation state as described above and the random-number monad. To get a feeling on the different approach between the Sugarscape and the SIR we show the initial function of the SIR agent. It is called by the simulation kernel to schedule initial events, adjust the simulation state and get the agents MSF.

\begin{HaskellCode}
-- | A sir agent is in one of three states
sirAgent :: RandomGen g 
         => SIRState    -- ^ the initial state of the agent
         -> SIRAgent g  -- ^ the continuation
sirAgent Susceptible aid = do
    modifyDomainState incSus -- increment number of susceptible agents
    scheduleEvent aid MakeContact makeContactInterval -- schedule make contact event to self
    return (susceptibleAgent aid) -- return susceptible MSF
sirAgent Infected aid = do
    modifyDomainState incInf -- increment number of infected agents
    dt <- lift (randomExpM (1 / illnessDuration)) -- draw random illness duration
    scheduleEvent aid Recover dt -- schedule recovery to self
    return (infectedAgent aid) -- returns infected MSF 
sirAgent Recovered _ = do
    modifyDomainState incRec -- increment number of recovered agents
    return recoveredAgent -- return recovered MSF
\end{HaskellCode}

In the next sections we have a quick look at how we translate the time-driven susceptible, infected and recovered behaviours of into event-driven behaviours.

\subsection{Susceptible Agent}
We use the same switch mechanism of making the transition from a susceptible to an infected agent in case of an infection but how a susceptible agent gets infected works now different:

\begin{itemize}
	\item A susceptible agent initially schedules a \textit{MakeContact} event with $\Delta t = 1$ to itself.
	\item When receiving \textit{MakeContact}, the agent sends a \textit{Contact} event to 5 random other agents with $\Delta t = 0$. This will result in these events to be scheduled immediately. Further the agent schedules \textit{MakeContact} with $\Delta t = 1$ to itself.
	\item When the agent receives a \textit{Contact} event, it checks if it is from an infected agent. If the event is not from an infected agent, it ignores it. Otherwise it becomes infected with a given probability. In case of infection the agent decrements the number of susceptible and increments the number of infected agents.
\end{itemize}

\subsection{Infected Agent}
We use the same switch mechanism of making the transition from an infected to a recovered agent after the illness duration. The main difference is that agents send \textit{Contact} events to each other to indicate that contacts have happened. Thus susceptible and infected agents need to react to incoming \textit{Contact} events.

\begin{itemize}
	\item An infected agent initially schedules a \textit{Recover} event with a random $\Delta t$ (following exponential distribution) to itself.
	\item When the agent receives a \textit{Contact} event, it checks if it is from a susceptible agent. If the event is not from a susceptible agent, it ignores it. Otherwise it simply replies to this susceptible agent with a \textit{Contact} event with $\Delta t = 0$.
\end{itemize}

\subsection{Recovered Agent}
The recovered agent does not change any more, reacts to no incoming events and schedules no events - it stays constant forever and thus outputs the empty tuple forever.

\subsection{Reflections}
Transforming a time-driven into an event-driven approach should always be possible because the ability to schedule events with time-stamps allows to map all features of time-driven ABS to an event-driven one - the discussion above should give a good direction of how this process works. Still for some models one can argue that the time-driven approach is much more expressive than an event-driven one, and we think this is certainly the case for the SIR model. The event-driven approach leads to much more fragmented logical flow and agent behaviour.

The event-driven implementation from this Chapter is around 60 - 70\% faster than the time-driven implementation from Chapter \ref{sec:timedriven_firststep}, which is non-monadic and uses the FRP library Yampa. For the monadic time-driven approach of Chapter \ref{sec:adding_env} the difference is much more dramatic: it is about 700 - 800\% slower. These results dramatically highlight the problem of time-driven ABS: its performance cannot compete with an even-driven approach. This is exaggerated even more so when making use of MSFs as in Chapter \ref{sec:adding_env}. In this case, a time-driven approach becomes extremely expensive in terms of performance and one should consider an event-driven approach. In case the model is specified in a time-driven way, a transformation into an event-driven approach should always be possible as outlined above.

\section{Discussion}

\subsection{Other Models}
TODO: mention that we have also implemented other models, which also work without time-semantics (all agents make a move at discrete time-steps and do not really rely on some notion of time). 

\subsection{Time-Semantics}
The main reason for building our pure functional ABMS approach on top of Yampa was to leverage the powerful time-semantics of Yampa which allows us to implement important concepts of ABMS:

state-chart: agents are at all time of their life-cycle in one state and can switch between multiple states using transitions 
timed transitions: transition to another state/behaviour happens at a discrete time
rate transitions: transition happens with a given rate
message transition: transition upon receiving a given message 

\subsection{Agents as Signals}
Due to the underlying nature and motivation of Functional Reactive Programming (und im speziellen) Yampa, Agents can be seen as Signals which is generated and consumed by a Signal-Function which is the behaviour of an Agent. If an Agent does not change the OUTPUT-signal is constant, if the agent changes e.g. by sending a message, changing its state,... the OUTPUT signal changes. A dead agent has no signal at all.

\subsection{Time-Sampling}
sampling rate depends on the transition times \& rates of the model. when e.g. the contact rate is 5 then the sampling dt should be below 0.2

\subsection{System Dynamics}
can emulate system dynamics due to the parallel update-strategy and continuous time-flow semantics

\subsection{Discrete Event Simulation}
DES in FrABMS? how easily can we implement server/queue systems? do they also look like a specification? potential problem: ordering of messages is not guaranteed by now

\subsection{Advantages}
advantages:
	- no side-effects within agents leads to much safer code
	- edsl for time-semantics
	- declarative style: agent-implementation looks like a model-specification
	- reasoning and verification
	- sequential and parallel
	- powerful time-semantics
	- arrowized programming is optional and only required when utilizing yampas time-semantics. if the model does not rely on time-semantics, it can use monadic-programming by building on the existing monadic functions in the EDSL which allow to run in the State-Monad which simplifies things very much
	- when to use yampas arrowized programing: time-semantics, simple state-chart agents 
	- when not using yampas facilities: in all the other cases e.g. SugarScape is such a case as it proceeds in unit time-steps and all agents act in every time-step
	- can implement System Dynamics building on Yampas facilities with total ease	
	- get replications for free without having to worry about side-effects and can even run them in parallel without headaches
	- cant mess around with time because delta-time is hidden from you (intentional design-decision by Yampa). this would be only very difficult and cumbersome to achieve in an object-oriented approach. TODO: experiment with it in Java - how could we actually implement this? I think it is impossible: may only achieve this through complicated application of patterns and inheritance but then has the problem of how to update the dt and more important how to deal with functions like integral which accumulates a value through closures and continuations. We could do this in OO by having a general base-class e.g. ContinuousTime which provides functions like updateDt and integrate, but we could only accumulate a single integral value.
	- reproducibility statically guaranteed
	- cannot mess around with dt
	- code == specification
	- rule out serious class of bugs
	- different time-sampling leads to different results e.g. in wildfire \& SIR but not in Prisoners Dilemma. why? probabilistic time-sampling?
	- reasoning about equivalence between SD and ABS implementation in the same framework
	- recursive implementations
	
	- we can statically guarantee the reproducibility of the simulation because: no side effects possible within the agents which would result in differences between same runs (e.g. file access, networking, threading), also timedeltas are fixed and do not depend on rendering performance or userinput	
	
\subsection{Disadvantages}
disadvantages:
	- performance is low
	- reasoning about performance is very difficult
	- very steep learning curve for non-functional programmers
	- learning a new EDSL
	- think ABMS different: when to use async messages, when to use sync conversations


[ ] important: increasing sampling freqzency and increasing number of steps so that the same number of simulation steps are executed should lead to same results. but it doesnt. why?
[ ] hypothesis: if time-semantics are involved then event ordering becomes relevant for emergent patterns. there are no tine semantics in heroes and cowards but in the prisoners dilemma
[ ] can we implement different types of agents interacting with each other in the same simulation ? with different behaviour funcs, digferent state? yes, also not possible in NetLogo to my knowledge. but they must have the same messages, emvironment 

[ ] Hypothesis: we can combine with FrABS agent-based simulation and system dynamics (this has been proved by example!)