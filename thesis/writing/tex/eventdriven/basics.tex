\section{Basics of event-driven ABS}
This section we derive step-by-step the basics of event-driven SIR using the SIR model, as introduced in Chapter \ref{sec:sir_model}, with an event-driven approach. This is in stark contrast to the time-driven implementation in Chapter \ref{sec:timedriven_firststep}. The solutions are quantitatively equal as they produce the same class of dynamics. Qualitatively they fundamentally differ though in terms of expressivity and performance as we will see below.

The basics of event-driven ABS are the concept of agent identity, events and event-scheduling. We introduce them step-by-step using various Monads and then generalise the interface to a \textit{tagless final} approach. 

\subsection{An event-driven SIR}
Before we can derive implementation concepts, we first need to discuss how an event-driven SIR model works, as inspired by \cite{macal_agent-based_2010}. Fundamentally, what is required is to transform all time-dependent behaviour and agent interactions into the reception and scheduling of events. In the event-driven SIR only three event-types are needed:

\begin{enumerate}
	\item \textbf{MakeContact} - is used to let susceptible agents pro-actively make contact with $\beta$ (contact rate) other agents per 1 time-unit.
	\item \textbf{Contact$_{Sender \  SIRState}$} - is used to make contact between agents where agents reveal their state by sending or replying with their current state.
	\item \textbf{Recover} - is used to let infected agents pro-actively recover after the given $\delta$ (illness duration). 
\end{enumerate}

Now we can give a concise definition of all three agent behaviours:

\paragraph{Susceptible Agent}
\begin{itemize}
	\item A susceptible agent initially schedules a \textit{MakeContact} event with $\Delta t = 1$ to itself.
	\item When receiving \textit{MakeContact}, the agent sends a \textit{Contact} event to $\beta$ (contact rate) random other agents with $\Delta t = 0$. This will result in these events to be scheduled immediately. Further the agent schedules \textit{MakeContact} with $\Delta t = 1$ to itself, to keep the pro-active process of making contact with other agents active.
	\item When the agent receives a \textit{Contact} event, it checks if it is from an infected agent. If the event is not from an infected agent, it ignores it. Otherwise it becomes infected with a given probability.
\end{itemize}

\paragraph{Infected Agent}
\begin{itemize}
	\item An infected agent initially schedules a \textit{Recover} event to itself, with a random $\Delta t$ of $\delta$ (illness duration), which follows the exponential distribution.
	\item When the agent receives a \textit{Contact} event, it checks if it is from a susceptible agent. If the event is not from a susceptible agent, it ignores it. Otherwise it simply replies to this susceptible agent with a \textit{Contact} event with $\Delta t = 0$.
\end{itemize}

\paragraph{Recovered Agent}
The recovered agent does not change any more, reacts to no incoming events and schedules no events - it stays constant forever.

\subsection{Agent-Identity}
- ReaderT Time for current global time (its the same as bearriver SF but the semantics are different, not time-delta but absolute time)
- ReaderT [AgentId] for all agents

\subsection{Events}
- WriterT [QueueItem e]

\subsection{Scheduling}

\subsection{Tagless Final}
