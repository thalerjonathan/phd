\chapter{Introduction}
% THE PROBLEM
The traditional approach to Agent-Based Simulation (ABS) has so far always been object-oriented techniques, due to the influence of the seminal work of Epstein et al \cite{epstein_growing_1996} in which the authors claim "[..] object-oriented programming to be a particularly natural development environment for Sugarscape specifically and artificial societies generally [..]" (p. 179). This work established the metaphor in the ABS community, that \textit{agents map naturally to objects} \cite{north_managing_2007} which still holds up today.

% MOTIVATION and AIM
This thesis challenges that metaphor and explores ways of approaching ABS with the \textit{pure} functional programming paradigm using the languages Haskell and Idris. It is the first one to do so on a \textit{systematical} level and develops a foundation by presenting fundamental concepts and advanced features to show how to leverage the benefits of both languages \cite{hudak_history_2007, brady_idris_2013} to become available when implementing ABS functionally. By doing this, the thesis both shows \textit{how} to implement ABS purely functional and \textit{why} it is of benefit of doing so, what the drawbacks are and also when a pure functional approach should \textit{not} be used. 

% SCOPE
This thesis claims that the agent-based simulation community needs functional programming because of its \textit{scientific computing} nature, where results need to be reproducible and correct while simulations should be able to massively scale-up as well. %The established object-oriented approaches need considerably high effort and might even fail to deliver these objectives due to its conceptually different approach to computing.

% RESEARCH QUESTION and HYPOTHESEIS
Thus this thesis' general research question is \textit{how to implement ABS purely functional and what the benefits and drawbacks are of doing so.} Further, it hypothesises that by using pure functional programming for implementing ABS makes it is easy to add parallelism and concurrency, the resulting simulations are easy to test and verify, applicable to property-based testing, guaranteed to be reproducible already at compile-time, have fewer potential sources of bugs and thus can raise the level of confidence in the correctness of an implementation to a new level.

\newpage

\section{Publications}
Throughout the course of the Ph.D. four (4) papers were published:

\begin{enumerate}
	\item The Art Of Iterating - Update Strategies in Agent-Based Simulation \cite{thaler_art_2017} - This paper derives the 4 different update-strategies and their properties possible in time-driven ABS and discusses them from a programming-paradigm agnostic point of view. It is the first paper which makes the very basics of update-semantics clear on a conceptual level and is necessary to understand the options one has when implementing time-driven ABS purely functional.
	
	\item Pure Functional Epidemics \cite{thaler_pure_2019} - Using an agent-based SIR model, this paper establishes in technical detail \textit{how} to implement time-driven ABS in Haskell using non-monadic FRP with Yampa and monadic FRP with Dunai. It outlines benefits and drawbacks and also touches on important points which were out of scope and lack of space in this paper but which will be addressed in the Methodology chapter of this thesis.
	
	\item A Tale Of Lock-Free Agents (TODO cite) - This paper is the first to discuss the use of Software Transactional Memory (STM) for implementing concurrent ABS both on a conceptual and on a technical level. It presents two case-studies, with the agent-based SIR model as the first and the famous SugarScape being the second one. In both case-studies it compares performance of STM and lock-based implementations in Haskell and object-oriented implementations of established languages. Although STM is now not unique to Haskell any more, this paper shows why Haskell is particularly well suited for the use of STM and is the only language which can overcome the central problem of how to prevent persistent side-effects in retry-semantics. It does not go into technical details of functional programming as it is written for a simulation Journal.

	\item Towards Pure Functional Agent-Based Simulation (TODO cite) - This paper summarizes the main benefits of using pure functional programming as in Haskell to implement ABS and discusses on a conceptual level how to implement it and also what potential drawbacks are and where the use of a functional approach is not encouraged. It is written as a conceptual / review paper, which tries to "sell" pure functional programming to the agent-based community without too much technical detail and parlance where it refers to the important technical literature from where an interested reader can start.
\end{enumerate}

\newpage

\section{Contributions}
\begin{enumerate}
	\item This thesis is the first to \textit{systematically} investigate the use of the functional programming paradigm, as in Haskell, to ABS, laying out in-depth technical foundations and identifying its benefits and drawbacks. Due to the increased interested in functional concepts which were added to object-oriented languages in recent years, because of its established benefits in concurrent programming, testing and software-development in general, presenting such foundational research gives this thesis significant impact. Also it opens the way for the benefits of FP to incorporate into scientific computing, which are explored in the contributions below.
	
	\item This thesis is the first to show the use of Software Transactional Memory (STM) to implement concurrent ABS and its potential benefit over lock-based approaches. STM is particularly strong in pure FP because of retry-semantics can be guaranteed to exclude non-repeatable persistent side-effects already at compile time. By showing how to employ STM it is possible to implement a simulation which allows massively large-scale ABS but without the low level difficulties of concurrent programming, making it easier and quicker to develop working and correct concurrent ABS models. Due to the increasing need for massively large-scale ABS in recent years \cite{lysenko_framework_2008}, making this possible within a purely functional approach as well, gives this thesis substantial impact.
	
	\item This thesis is the first to present the use of property-based testing in ABS which allows a declarative specification- testing of the implemented ABS directly in code with \textit{automated} test-case generation. This is an addition to the established Test Driven Development process and a complementary approach to unit-testing, ultimately giving the developers an additional, powerful tool to test the implementation on a more conceptual level. This should lead to simulation software which is more likely to be correct, thus making this a significant contribution with valuable impact.

	\item This thesis is the first to outline the potential use of \textit{dependent types} to Agent-Based Simulation on a \textit{conceptual level} to investigate its usefulness for increasing the correctness of a simulation. Dependent types can help to narrow the gap between the model specification and its implementation, reducing the potential for conceptual errors in model-to-code translation. This immediately leads to fewer number of tests required due to guarantees being expressed already at compile time. Ultimately dependent types lead to higher confidence in correctness due to formal guarantees in code, making this a unique contribution with high impact.
\end{enumerate}

\newpage

\section{Thesis structure}
This thesis focuses on a strong narrative which tells the story of \textit{how} to do ABS with pure functional programming, \textit{why} one would do so and when one should \textit{avoid} this paradigm in ABS.

TODO: write when all is finished