\chapter{Concurrent ABS}
\label{ch:concurrent_abs}

%- no comparison to io or repast: the story is "concurrency with compile time guarantees", only mention that an io based single lock performs much worse in sir and slightly worse in sugarscape. leave Array i IORef for further research

In an ideal world, we would like to solve all our problems using parallelism but unfortunately, it can't be applied to all parallel problems and ABS is no exception. As soon as there are data-dependencies, like we have them in the Sugarscape model in the form of the read/write environment and synchronous agent interactions, and to a lesser extent in the monadic SIR with the \textit{Rand} Monad, we cannot avoid concurrency. More general, this is due to the fact that agents are executed within a monadic context, from which the  sequencing of effectful computations immediately follows - this is the very meaning of the Monad abstraction. Indeed, we have shown both by argument and measurement in the previous chapter the very fact that parallelism is simply not applicable to monadic execution of agents due to sequencing of effects, which renders all attempts of running monadic agents in parallel void. In this chapter we discuss the use of concurrency to run agents which have a monadic context in parallel - which is the only way we can execute monadic agents at the same time.

\medskip

Traditional approaches to concurrency follow a lock-based approach, where sections which access shared data are synchronised through synchronisation primitives like mutexes, semaphores, monitors,... The lock-based path is a well trodden one, with all problems and benefits well established. In this chapter we follow a different path and look into using Software Transactional Memory (STM) for implementing concurrent ABS, which promises to overcome the problems of lock-based approaches. Although STM exists in other languages as well, Haskell was one of the first to natively build it into its core, thus it is a natural choice to follow that direction when already investigating pure functional ABS.

Unfortunately, as soon as we employ concurrency, we lose all static guarantees about reproducibility and the use of STM is no exception. Still, STM has the unique benefit that it can guarantee the lack of persistent side effects at compile time, allowing unproblematic retries of transactions, something of fundamental importance in STM as will be described below. This implies also another \textit{very} compelling advantage of STM over unrestricted lock-based approaches: by using STM, we can reduce the side effects allowed substantially and guarantee at compile time, that the differences between runs of same initial conditions will only stem from the fact that we run the simulation concurrently - \textit{and from nothing else}. All this makes the use of STM very compelling and to our best knowledge we are the very first to investigate the use of STM for implementing concurrent ABS in a systematic way.

\medskip

The paper \cite{discolo_lock_2006} gives a good indication how difficult and complex constructing a correct concurrent program is and shows how much easier, concise and less error-prone an STM implementation is over traditional locking with mutexes and semaphores. More important, it shows that STM consistently outperforms the lock-based implementations. We follow this work and compare the performance of lock-based and STM implementations and hypothesise that the reduced complexity and increased performance will be directly applicable to ABS as well.

We present two case studies using the already introduced SIR (Chapter \ref{sec:sir_model}) and Sugarscape (Chapter \ref{sec:sugarscape}) models. We compare the performance of lock-based and STM implementations in each case where we investigate both the scaling performance under increasing number of CPUs and agents. We show that the STM implementations consistently outperform the lock-based ones and scale much better to increasing number of CPUs both on local machines and on Amazon Cloud Services.

%Note that there exists also the actor model of concurrency, which is especially well suited to implement concurrent applications in functional languages. We give a short overview over it, existing research and its use in ABS in the section \ref{sec:actors} but leave it for further research as it has very different implications, which are beyond the focus of this thesis.

\section{Software Transactional Memory}
Software Transactional Memory (STM) was introduced by \cite{shavit_software_1995} in 1995 as an alternative to lock-based synchronisation in concurrent programming which, in general, is notoriously difficult to get right. This is because reasoning about the interactions of multiple concurrently running threads and low level operational details of synchronisation primitives is \textit{very hard}. The main problems are:

\begin{itemize}
	\item Race conditions due to forgotten locks;
	\item Deadlocks resulting from inconsistent lock ordering;
	\item Corruption caused by uncaught exceptions;
	\item Lost wake-ups induced by omitted notifications.
\end{itemize}

Worse, concurrency does not compose. It is very difficult to write two functions (or methods in an object) acting on concurrent data which can be composed into a larger concurrent behaviour. The reason for it is that one has to know about internal details of locking, which breaks encapsulation and makes composition dependent on knowledge about their implementation. Therefore, it is impossible to compose two  functions e.g. where one withdraws some amount of money from an account and the other deposits this amount of money into a different account: one ends up with a temporary state where the money is in none of either accounts, creating an inconsistency - a potential source for errors because threads can be rescheduled at any time.

STM promises to solve all these problems for a low cost by executing actions \textit{atomically}, where modifications made in such an action are invisible to other threads and changes by other threads are invisible as well until actions are committed - STM actions are atomic and isolated. When an STM action exits, either one of two outcomes happen: if no other thread has modified the same data as the thread running the STM action, then the modifications performed by the action will be committed and become visible to the other threads. If other threads have modified the data then the modifications will be discarded, the action block rolled-back and automatically restarted.

STM in Haskell is implemented using optimistic synchronisation, which means that instead of locking access to shared data, each thread keeps a transaction log for each read and write to shared data it makes. When the transaction exits, the thread checks whether it has a consistent view to the shared data or not: whether other threads have written to memory it has read. % This might look like a serious overhead but the implementations are very mature by now, being very performant and the benefits outweigh its costs by far.

In the paper \cite{heindl_modeling_2009} the authors use a model of STM to simulate optimistic and pessimistic STM behaviour under various scenarios using the AnyLogic simulation package. They conclude that optimistic STM may lead to 25\% less retries of transactions. The authors of \cite{perfumo_limits_2008} analyse several Haskell STM programs with respect to their transactional behaviour. They identified the roll-back rate as one of the key metric which determines the scalability of an application. Although STM might promise better performance, they also warn of the overhead it introduces which could be quite substantial in particular for programs which do not perform much work inside transactions as their commit overhead appears to be high.

\subsection{STM in Haskell}
The work of \cite{harris_composable_2005, harris_transactional_2006} added STM to Haskell, which was one of the first programming languages to incorporate STM into its main core and added the ability to composable operations. There exist various implementations of STM in other languages as well (Python, Java, C\#, C/C++, etc) but we argue, that it is in Haskell with its type-system and the way how side-effects are treated where it truly shines.

In the Haskell implementation, STM actions run within the \textit{STM} context. This restricts the operations to only STM primitives as shown below, which allows to enforce that STM actions are always repeatable without persistent side-effects because such persistent side-effects (e.g. writing to a file, launching a missile) are not possible in an \textit{STM} context. This is also the fundamental difference to  \textit{IO}, where all bets are off because \textit{everything} is possible as there are basically no restrictions because \textit{IO} can run everything.

Thus the ability to \textit{restart} a block of actions without any visible effects is only possible due to the nature of Haskells type-system: by restricting the effects to STM only, ensures that no uncontrolled effects, which cannot be rolled-back, occur.

STM comes with a number of primitives to share transactional data. Amongst others the most important ones are:

\begin{itemize}
	\item \textit{TVar} - A transactional variable which can be read and written arbitrarily;
	\item \textit{TArray} - A transactional array where each cell is an individual shared data, allowing much finer-grained transactions instead of e.g. having the whole array in a \textit{TVar};
	\item \textit{TChan} - A transactional channel, representing an unbounded FIFO channel;
	\item \textit{TMVar} - A transactional \textit{synchronising} variable which is either empty of full. To read from an empty or write to a full \textit{TMVar} will cause the current thread to retry its transaction.
\end{itemize}

% NOTE: too technical
%To run an \textit{STM} action the function \textit{atomically :: STM a $\to$ IO a} is provided, which can be seen as the STM effect-runner as it performs a series of \textit{STM} actions atomically within an \textit{IO} context. It takes the STM action which returns a value of type \textit{a} and returns an \textit{IO} action which returns a value of type \textit{a}. This \textit{IO} action can only be executed within an \textit{IO} context.

\section{STM in ABS}
\label{sec:stm_abs}
In this section we give a short overview of how we apply STM in our ABS. In both case-studies we fundamentally follow a time-driven, parallel approach as introduced in Chapter \ref{sub:par_strategy}, where the simulation is advanced by a given $\Delta t$ and in each step all agents are executed. To employ parallelism, each agent runs within its own thread and agents are executed in lock-step, synchronising between each $\Delta t$, which is controlled by the main thread. See Figure \ref{fig:stm_abs_structure} for a visualisation of our concurrent, time-driven lock-step approach.

By running each agent in a thread will guarantee the execution in parallel even if the agent has a monadic context. This is forces us to evaluate each agents monadic context separately instead of running them all in a common context. Note that ultimately we are ending up in the \textit{IO} context because \textit{STM} can be only transacted from within an \textit{IO} context due to non-deterministic side-effects. This is no contradiction to our original claim: yes we are running in IO but not the agent behaviour itself, which is a fundamental difference.

An agent thread will block until the main-thread sends the next $\Delta t$ and runs the \textit{STM} action atomically with the given $\Delta t$. When the \textit{STM} action has been committed, the thread will send the output of the agent action to the main-thread to signal it has finished. The main thread awaits the results of all agents to collect them for output of the current step e.g. visualisation or writing to a file.

As will be described in subsequent sections, central to both case-studies is an environment which is shared between the agents using a \textit{TVar} or \textit{TArray} primitive through which the agents communicate concurrently with each other. To get the environment in each step for visualisation purposes, the main thread can access the \textit{TVar} and \textit{TArray} as well. 

\begin{figure}
	\centering
	\includegraphics[width=1.0\textwidth, angle=0]{./fig/concurrentabs/stm_abs.png}
	\caption{Diagram of the parallel time-driven lock-step approach.}
	\label{fig:stm_abs_structure}
\end{figure}

\subsection{Adding and running the STM Monad}
We briefly show how to add STM to agents and run them within their own threads. We use the SIR implementation as example - applying it to the Sugarscape implementation works exactly the same way and is left as a trivial exercise to the reader.

The first step is to simply add the \textit{STM} to the existing transformer stack as the \textit{innermost} monad. The reason why we make it the innermost is to guarantee that in case of a retry \textit{all} outer monadic effects are retried as well - if the STM would be placed on a higher stack level, the levels below would not be subject to a retry. For monads like the \textit{ReaderT} this would not matter because they are read-only but for a StateT this fact would matter a lot. Note that STM does not provide a transformer instance, so this is not an option anyway. If STM would provide a transformer then we could make \textit{IO} the innermost monad and do \textit{IO} within STM, which should be prevented under all circumstances because then rolling back a transaction cannot guarantee to undo the effects. To better understand the semantics of retries consider the following example:

\begin{HaskellCode}

\end{HaskellCode}



\begin{HaskellCode}
innerSTMAction :: RandomGen g => StateT SomeState (RandT g STM) SomeResult

let randAction = runStateT innerSTMAction initState
let stmAction  = runRandT randAction (mkStdGen 42)
let ioAction   = atomically stmAction
((someResult, someState), g) <- ioAction
\end{HaskellCode}

In this case the STM is the \textit{innermost} monad thus it will be run last. This means that all the outer monads are subject to re-computation due to retries.

\begin{HaskellCode}
outerSTMAction :: STMT (StateT Environment (Rand g)) SomeResult

let ioAction = runSTMT outerSTMAction
stateAction <- ioAction
let randAction = runStateT stateAction initState
let ((someResult, someState), g) = runRandT randAction (mkStdGen 42)
\end{HaskellCode}

In this case, the STM is the \textit{outermost} monad, thus it will be run first. This means that it will return a StateT computation which will be computed \textit{after} the STM has transacted. The computation construction is subject to the retries but the computation itself won't be repeated in case of retries.

TODO: add STM

\begin{HaskellCode}
agentThread :: RandomGen g 
            => Int
            -> SIRAgent g
            -> g
            -> MVar SIRState
            -> MVar DTime
            -> IO ()
agentThread 0 _ _ _ _ = return () -- all steps computed, terminate thread
agentThread n sf rng retVar dtVar = do
  -- wait for dt to compute current step
  dt <- takeMVar dtVar

  -- compute output of current step
  let sfReader = unMSF sf ()
      sfRand   = runReaderT sfReader dt
      sfSTM    = runRandT sfRand rng
  ((ret, sf'), rng') <- atomically sfSTM -- run the STM action atomically within IO

  -- post result to main thread
  putMVar retVar ret
  
  -- to next step
  agentThread (n - 1) sf' rng retVar dtVar
\end{HaskellCode}

\begin{HaskellCode}
simulationStep :: TVar SIREnv
               -> [MVar DTime]
               -> [MVar SIRState]
               -> DTime
               -> IO SIREnv
simulationStep env dtVars retVars dt = do
  -- tell all threads to continue with the corresponding DTime
  mapM_ (`putMVar` dt) dtVars
  -- wait for results but ignore them, SIREnv contains all states
  mapM_ takeMVar retVars
  -- return state of environment when step has finished
  readTVarIO env
\end{HaskellCode}

\subsection{The SIR model}
\label{sec:sir_model}

The explanatory SIR model is a thoroughly studied and well understood compartment model from epidemiology \cite{kermack_contribution_1927}, which allows simulation of the dynamics of an infectious disease like influenza, tuberculosis, chicken pox, rubella and measles spreading through a population. The reason for choosing this model is its simplicity. It is easy to understand fully but complex enough to develop basic concepts of pure functional ABS, which are then extended and deepened in the much more complex Sugarscape model explained in the next section.

In this model, people in a population of size $N$ can be in either one of three states: \textit{Susceptible}, \textit{Infected} or \textit{Recovered}, at any particular time. It is assumed that initially there is at least one infected person in the population. People interact \textit{on average} with a given rate of $\beta$ other people per time unit, and become infected with a given probability $\gamma$ when interacting with an infected person. When infected, a person recovers \textit{on average} after $\delta$ time units and is then immune to further infections. An interaction between infected persons does not lead to reinfection, thus these interactions are ignored in this model. This definition gives rise to three compartments with the transitions seen in Figure \ref{fig:sir_transitions}.

\begin{figure}
	\centering
	\includegraphics[width=.7\textwidth, angle=0]{./fig/timedriven/SIR_transitions.png}
	\caption[States and transitions in the SIR compartment model]{States and transitions in the SIR compartment model.}
	\label{fig:sir_transitions}
\end{figure}

This model was also formalized using System Dynamics \cite{porter_industrial_1962}. In System Dynamics a system is modelled through differential equations, which allow expressing continuous systems, changing over time. They are solved by numerically integrating over time, which gives rise to the respective dynamics. The SIR model is modelled using the following equation, with the dynamics shown in Figure \ref{fig:sir_sd_dynamics} .

\begin{equation}
\begin{aligned}
\frac{\mathrm d S}{\mathrm d t} = -infectionRate \\
\frac{\mathrm d I}{\mathrm d t} = infectionRate - recoveryRate \\
\frac{\mathrm d R}{\mathrm d t} = recoveryRate 
\end{aligned}
\end{equation}

\begin{equation}
\begin{aligned}
infectionRate = \frac{I \beta S \gamma}{N} \\
recoveryRate = \frac{I}{\delta} 
\end{aligned}
\end{equation}

\begin{figure}
	\centering
	\includegraphics[width=0.5\textwidth, angle=0]{./fig/timedriven/SIR_SD_1000agents_150t_001dt.png}
	\caption[Dynamics of the SIR compartment model using the System Dynamics approach]{Dynamics of the SIR compartment model using the System Dynamics approach. Population Size $N$ = 1,000, contact rate $\beta =  \frac{1}{5}$, infection probability $\gamma = 0.05$, illness duration $\delta = 15$ with initially 1 infected agent. Simulation run for 150 time steps. Generated using our pure functional System Dynamics approach (see Appendix \ref{app:sd_simulation}).}
	\label{fig:sir_sd_dynamics}
\end{figure}

The approach of mapping the SIR model to an ABS is to discretise the population and model each person in the population as an individual agent. The transitions between the states are happening due to discrete events caused both by interactions amongst the agents and timeouts. The major advantage of ABS over System Dynamics is that it allows for the incorporation of spatiality and heterogeneity of a population, for example accounting for different sexes and ages. This is not directly possible with other simulation methods of System Dynamics or Discrete Event Simulation \cite{zeigler_theory_2000}.

This is directly related to a networked SIR model, where the interactions between agents are restricted by either a statically fixed or dynamically evolving network. Various network types exist, allowing for simulation of various scenarios. Very small communities where all agents are in contact with each other are modelled by a fully connected network. Real world scenarios where a few agents act as hubs are modelled by complex networks \cite{BarabasiAlbert_EmergenceScaling, Jackson2008, Newman_ComplexNetworks, WattsStrogatz_DynamicsSmallWorld}. In this thesis we do not impose restrictions on the connections among agents and always assume a fully connected network. Adding various network types to our thesis would unnecessarily complicate things in the beginning but would not constitute anything fundamentally new in terms of research. However, the use of complex networks, which in general are generated randomly, constitute an interesting direction for further research especially in the context of randomised property-based testing in ABS, which we discuss in Chapters \ref{ch:agentspec} and \ref{ch:sir_invariants}.

In the ABS classification of \cite{macal_everything_2016}, this model can be seen as an \textit{Interactive ABMS}: agents are individual heterogeneous agents with diverse set characteristics; they have autonomic, dynamic, endogenously defined behaviour; interactions happen between other agents and the environment through observed states, behaviours of other agents and the state of the environment.

\section{Case-Study II: Sugarscape}
TODO: 
we can implement everything except synchronous direct agent-interactions atm: if agent-interaction is one-way e.g. paying back a loan then this is no problem. thus the following parts of the Sugarscape are not possible with our current STM approach: mating, trading and lending  because all 3 require direct agent-to-agent interaction over multiple steps. We leave the problem of developing such an algorithm / implementation for further research.

\section{Discussion}

\subsection{Other Models}
TODO: mention that we have also implemented other models, which also work without time-semantics (all agents make a move at discrete time-steps and do not really rely on some notion of time). 

\subsection{Time-Semantics}
The main reason for building our pure functional ABMS approach on top of Yampa was to leverage the powerful time-semantics of Yampa which allows us to implement important concepts of ABMS:

state-chart: agents are at all time of their life-cycle in one state and can switch between multiple states using transitions 
timed transitions: transition to another state/behaviour happens at a discrete time
rate transitions: transition happens with a given rate
message transition: transition upon receiving a given message 

\subsection{Agents as Signals}
Due to the underlying nature and motivation of Functional Reactive Programming (und im speziellen) Yampa, Agents can be seen as Signals which is generated and consumed by a Signal-Function which is the behaviour of an Agent. If an Agent does not change the OUTPUT-signal is constant, if the agent changes e.g. by sending a message, changing its state,... the OUTPUT signal changes. A dead agent has no signal at all.

\subsection{Time-Sampling}
sampling rate depends on the transition times \& rates of the model. when e.g. the contact rate is 5 then the sampling dt should be below 0.2

\subsection{System Dynamics}
can emulate system dynamics due to the parallel update-strategy and continuous time-flow semantics

\subsection{Discrete Event Simulation}
DES in FrABMS? how easily can we implement server/queue systems? do they also look like a specification? potential problem: ordering of messages is not guaranteed by now

\subsection{Advantages}
advantages:
	- no side-effects within agents leads to much safer code
	- edsl for time-semantics
	- declarative style: agent-implementation looks like a model-specification
	- reasoning and verification
	- sequential and parallel
	- powerful time-semantics
	- arrowized programming is optional and only required when utilizing yampas time-semantics. if the model does not rely on time-semantics, it can use monadic-programming by building on the existing monadic functions in the EDSL which allow to run in the State-Monad which simplifies things very much
	- when to use yampas arrowized programing: time-semantics, simple state-chart agents 
	- when not using yampas facilities: in all the other cases e.g. SugarScape is such a case as it proceeds in unit time-steps and all agents act in every time-step
	- can implement System Dynamics building on Yampas facilities with total ease	
	- get replications for free without having to worry about side-effects and can even run them in parallel without headaches
	- cant mess around with time because delta-time is hidden from you (intentional design-decision by Yampa). this would be only very difficult and cumbersome to achieve in an object-oriented approach. TODO: experiment with it in Java - how could we actually implement this? I think it is impossible: may only achieve this through complicated application of patterns and inheritance but then has the problem of how to update the dt and more important how to deal with functions like integral which accumulates a value through closures and continuations. We could do this in OO by having a general base-class e.g. ContinuousTime which provides functions like updateDt and integrate, but we could only accumulate a single integral value.
	- reproducibility statically guaranteed
	- cannot mess around with dt
	- code == specification
	- rule out serious class of bugs
	- different time-sampling leads to different results e.g. in wildfire \& SIR but not in Prisoners Dilemma. why? probabilistic time-sampling?
	- reasoning about equivalence between SD and ABS implementation in the same framework
	- recursive implementations
	
	- we can statically guarantee the reproducibility of the simulation because: no side effects possible within the agents which would result in differences between same runs (e.g. file access, networking, threading), also timedeltas are fixed and do not depend on rendering performance or userinput	
	
\subsection{Disadvantages}
disadvantages:
	- performance is low
	- reasoning about performance is very difficult
	- very steep learning curve for non-functional programmers
	- learning a new EDSL
	- think ABMS different: when to use async messages, when to use sync conversations


[ ] important: increasing sampling freqzency and increasing number of steps so that the same number of simulation steps are executed should lead to same results. but it doesnt. why?
[ ] hypothesis: if time-semantics are involved then event ordering becomes relevant for emergent patterns. there are no tine semantics in heroes and cowards but in the prisoners dilemma
[ ] can we implement different types of agents interacting with each other in the same simulation ? with different behaviour funcs, digferent state? yes, also not possible in NetLogo to my knowledge. but they must have the same messages, emvironment 

[ ] Hypothesis: we can combine with FrABS agent-based simulation and system dynamics (this has been proved by example!)