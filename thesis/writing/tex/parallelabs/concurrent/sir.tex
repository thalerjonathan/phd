\section{Case Study I: SIR}
\label{sec:concurrent_sir}
Our first case study is the SIR model as introduced in Chapter \ref{sec:sir_model}. The aim of this case study is to investigate the potential speed up a concurrent \textit{STM} implementation gains over a sequential one under varying number of CPU cores and agents. The behaviour of the agents is quite simple and the interactions are happening indirectly through the environment, where reads from the environment outnumber the writes to it by far. Further, a comparison to a lock-based implementation with the \textit{IO} Monad is done to understand that \textit{STM} is also able to outperform traditional concurrency, \textit{in a pure functional ABS setting} while still retaining its greater static guarantees than \textit{IO} \footnote{The code of all three implementations is available at \url{https://github.com/thalerjonathan/phd/tree/master/public/stmabs/code/SIR}}.

\begin{enumerate}
	\item Sequential - this is the original implementation as discussed in Chapter \ref{sec:adding_env}, where the discrete 2D environment is shared amongst all agents as read-only data and the agents are executed sequentially within the main thread without any concurrency.
	\item STM - this is the same implementation as the \textit{Sequential} one but agents run now in the \textit{STM} Monad and have access to the discrete 2D environment through a transactional variable \textit{TVar}. This means that the agents now communicate indirectly by reads and writes through the \textit{TVar}.
	\item Lock-Based - this follows the \textit{STM} implementation, with the agents running in \textit{IO}. They share the discrete 2D environment using an \textit{IORef} and have access to an \textit{MVar} lock to synchronise access to it.
\end{enumerate}

Each experiment was run until $t = 100$ and stepped using $\Delta t = 0.1$. For each experiment we conducted 8 runs on our machine (see Table \ref{tab:machine_specs}) under no additional work-load and report the mean. %Further, we checked the visual outputs and the dynamics and they look qualitatively the same as the reference \textit{Sequential}. We could have used more rigour and properly validated the implementations against the formal specification using tests as we do in Chapter Property-based testing but we leave this for further res.
In the experiments we varied the number of agents (grid size) as well as the number of cores when running concurrently - the numbers are always indicated clearly.

\begin{table}
	\centering
	\begin{tabular}{ c || c }
		OS & Fedora 28, 64-bit \\ \hline
		RAM & 16 GByte \\ \hline
		CPU & Intel i5-4670K @ 3.4GHz \\ \hline
		HD & 250Gbyte SSD \\ \hline
		Haskell & GHC 8.2.2
	\end{tabular}
	
	\caption{Machine and Software specs for all experiments}
	\label{tab:machine_specs}
\end{table}

\subsection{Constant Grid Size, Varying Cores}
In this experiment we held the grid size constant to 51 x 51 (2,601 agents) and varied the cores. The results are reported in Table \ref{tab:constgrid_varyingcores}.

\begin{table}
	\centering
	\begin{tabular}{cc|c}
		\multicolumn{1}{ c||  }{\multirow{2}{*}{} } &
		\multicolumn{1}{ |c| }{Cores} & Duration      \\ \hline \hline 
		
		\multicolumn{1}{ c||  }{\multirow{1}{*}{Sequential} } &
		\multicolumn{1}{ |c| }{1} & 72.5      \\ \hline \hline 
		
		\multicolumn{1}{ c||  }{\multirow{4}{*}{Lock-Based} } &
		\multicolumn{1}{ |c| }{1} & 60.6       \\ \cline{2-3}
		\multicolumn{1}{ c||  }{}                       &
		\multicolumn{1}{ |c| }{2} & 42.8    \\ \cline{2-3}
		\multicolumn{1}{ c||  }{}                       &
		\multicolumn{1}{ |c| }{3} & 38.6    \\ \cline{2-3}
		\multicolumn{1}{ c||  }{}                       &
		\multicolumn{1}{ |c| }{4} & 41.6    \\ \hline \hline 
		
		\multicolumn{1}{ c||  }{\multirow{4}{*}{STM} } &
		\multicolumn{1}{ |c| }{1} & 53.2       \\ \cline{2-3}
		\multicolumn{1}{ c||  }{}                       &
		\multicolumn{1}{ |c| }{2} & 27.8    \\ \cline{2-3}
		\multicolumn{1}{ c||  }{}                       &
		\multicolumn{1}{ |c| }{3} & 21.8    \\ \cline{2-3}
		\multicolumn{1}{ c||  }{}                       &
		\multicolumn{1}{ |c| }{4} & \textbf{20.8}    \\ \hline \hline 
	\end{tabular}
  	
  	\caption{Experiments on 51x51 (2,601 agents) grid with varying number of cores. Timings in seconds (lower is better).}
	\label{tab:constgrid_varyingcores}
\end{table}

The \textit{STM} implementation running on 4 cores shows a speed up factor of 3.6 over \textit{Sequential}, which is a quite impressive number when considering that we can achieve at most a factor of 4 when running on 4 cores. It seems that \textit{STM} allow us to push the practical limit very close to the theoretical one, whereas the \textit{Lock-Based} approach just arrives at a factor of 1.74 on 4 cores.

Comparing the performance and scaling to multiple cores shows that the \textit{STM} implementation significantly outperforms the \textit{Lock-Based} one and scales better to multiple cores. The \textit{Lock-Based} implementation performs best with 3 cores and shows slightly worse performance on 4 cores as can be seen in Figure \ref{fig:core_duration_stm_io}. This is no surprise because the more cores are running at the same time, the more contention for the lock, thus the more likely synchronisation happening, resulting in higher potential for reduced performance. This is not an issue in \textit{STM} because no locks are taken in advance. 

\begin{figure}
	\centering
	\includegraphics[width=0.8\textwidth, angle=0]{./fig/concurrentabs/sir/core_duration_stm_io.png}
	\caption{Comparison of performance and scaling on multiple cores of STM and Lock-Based. Note that the Lock-Based implementation seems to perform slightly worse on 4 than on 3 cores probably due to lock-contention.}
	\label{fig:core_duration_stm_io}
\end{figure}

\subsection{Varying Grid Size, Constant Cores}
In this experiment we varied the grid size and used always 4 cores. The results are reported in Table \ref{tab:varyinggrid_constcores} and plotted in Figure \ref{fig:varyinggrid_constcores}.

\begin{table}
	\centering
  	\begin{tabular}{ c || c | c | c }
        Grid-Size          & STM              & Lock-Based   & Ratio \\ \hline \hline 
   		51 x 51 (2,601)    & \textbf{20.2}    & 41.9         & 2.1 \\ \hline
   		101 x 101 (10,201) & \textbf{74.5}    & 170.5        & 2.3 \\ \hline
   		151 x 151 (22,801) & \textbf{168.5}   & 376.9        & 2.2 \\ \hline
   		201 x 201 (40,401) & \textbf{302.4}   & 672.0        & 2.2 \\ \hline
   		251 x 251 (63,001) & \textbf{495.7}   & 1,027.3      & 2.1 \\ \hline \hline
  	\end{tabular}

  	\caption{Performance on varying grid sizes. Timings in seconds (lower is better). Ratio compares STM to Lock-Based.}
	\label{tab:varyinggrid_constcores}
\end{table}

It is clear that the \textit{STM} implementation outperforms the \textit{Lock-Based} implementation by a substantial factor.

\begin{figure}
	\centering
	\includegraphics[width=1\textwidth, angle=0]{./fig/concurrentabs/sir/stm_io_varyinggrid_performance.png}
	\caption{Performance on varying grid sizes.}
	\label{fig:varyinggrid_constcores}
\end{figure}

\subsection{Retries}
Of very much interest when using STM is the retry-ratio, which obviously depends highly on the read-write patterns of the respective model. We used the \textit{stm-stats} library to record statistics of commits, retries and the ratio. The results are reported in Table \ref{tab:retries_stm}.

\begin{table}
	\centering
  	\begin{tabular}{ c || c | c | c }
        Grid-Size 		   & Commits    & Retries & Ratio \\ \hline \hline 
   		51 x 51 (2,601)    & 2,601,000  & 1306.5  & 0.0 \\ \hline
   		101 x 101 (10,201) & 10,201,000 & 3712.5  & 0.0 \\ \hline
   		151 x 151 (22,801) & 22,801,000 & 8189.5  & 0.0 \\ \hline
   		201 x 201 (40,401) & 40,401,000 & 13285.0 & 0.0 \\ \hline 
   		251 x 251 (63,001) & 63,001,000 & 21217.0 & 0.0 \\ \hline \hline
  	\end{tabular}
  	
  	\caption{Retry ratios on varying grid sizes on 4 cores.}
	\label{tab:retries_stm}
\end{table}

Independent of the number of agents we always have a retry-ratio of 0.0. This indicates that this model is \textit{very} well suited to STM, which is also directly reflected in the much better performance over the \textit{Lock-Based} implementation. Obviously this ratio stems from the fact that in our implementation we have \textit{very} few writes, which happen only in case when an agent changes from Susceptible to Infected or from Infected to Recovered. On the other hand, there are a very large number of reads due to indirect agent interaction. For \textit{STM} this is no problem because no lock is taken but the \textit{Lock-Based} approach is forced to conservatively take the lock to ensure mutual exclusive access to the critical section across all agents.

\subsection{Going Large-Scale}
To test how far we can scale up the number of cores in both the \textit{Lock-Based} and \textit{STM} cases, we ran two experiments, 51x51 and 251x251, on Amazon EC2 instances with a larger number of cores than our local machinery, starting with 16 and 32 to see if we are running into decreasing returns. The results are reported in Table \ref{tab:sir_varying_cores_amazon}.

\begin{table}
	\centering
  	\begin{tabular}{cc|c|c}
		\multicolumn{1}{ c||  }{\multirow{2}{*}{} } &
		\multicolumn{1}{ |c| }{Cores} & 51x51    & 251x251       \\ \hline \hline 
		
		\multicolumn{1}{ c||  }{\multirow{2}{*}{Lock-Based} } &
		\multicolumn{1}{ |c| }{16} & 72.5    & 1830.5       \\ \cline{2-4}
		\multicolumn{1}{ c||  }{}                       &
		\multicolumn{1}{ |c| }{32} & 73.1    & 1882.2      \\ \hline \hline 
		
		\multicolumn{1}{ c||  }{\multirow{2}{*}{STM} } &
		\multicolumn{1}{ |c| }{16} & \textbf{8.6}     & \textbf{237.0}       \\ \cline{2-4}
		\multicolumn{1}{ c||  }{}                       &
		\multicolumn{1}{ |c| }{32} & 12.0    & 248.7      \\ \hline \hline 
	\end{tabular}

  	\caption{Performance on varying cores on Amazon S2 Services. Timings in seconds (lower is better).}
	\label{tab:sir_varying_cores_amazon}
\end{table}

As expected, the \textit{Lock-Based} approach doesn't scale up to many cores because each additional core brings more contention to the lock, resulting in an even more decreased performance, even worse than the \textit{Sequential} implementation. This is particularly obvious in the 251x251 experiment because of the much larger number of concurrent agents. The \textit{STM} approach returns better performance on 16 cores but fails to scale further up to 32 where the performance drops below the one with 16 cores. In both STM cases we measured a retry-ratio of 0, thus we assume that with 32 cores we become limited by the overhead of STM transactions \cite{perfumo_limits_2008} because the workload of an STM action in our SIR implementation is quite small.

Compared to the \textit{Sequential} implementation, \textit{STM} reaches a speed up factor of 8.4 on 16 cores, which is still impressive but is much further away from the theoretical limit than in the case of only 4 cores -  a further indication that this model in particular and our approach in general does not scale up arbitrarily.

% NOTE: 0 retries in both cases means that the STM transactions themselves are becoming the bottleneck. this makes sens because the STM trasnactions in our SIR implementation are very small (especially recovered and infected agent) and could therefore really cause substantial overhead as pointed out by \cite{perfumo_limits_2008}
%16 cores 251x251: 0.0 retry-ratio
%32 cores 251x251: 0.0 retry ratio
%
%16 cores 51x51: 0.0 retry-ratio
%32 cores 51x51: 0.0 retry ratio

\subsection{Discussion}
The timing measurements speak a clear language: running in \textit{STM} and sharing state using a transactional variable \textit{TVar} is much more time-efficient than both the \textit{Sequential} and \textit{Lock-Based} approach. On 4 cores \textit{STM} achieves a speed up factor of 3.6, nearly reaching the theoretical limit.
Obviously both \textit{STM} and \textit{Lock-Based} sacrifices determinism: repeated runs might not lead to same dynamics despite same initial conditions. Still, by sticking to \textit{STM}, we get the guarantee that the source of this non-determinism is concurrency within the \textit{STM} monad but \textit{nothing else}. This we can not guarantee in the case of the \textit{Lock-Based} approach as all bets are off when running within \textit{IO}. The fact to have \textit{both} the better performance \textit{and} the stronger static guarantees in the \textit{STM} approach makes it \textit{very} compelling.