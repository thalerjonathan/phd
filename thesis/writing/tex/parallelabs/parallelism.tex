\section{Parallelism in ABS}
The promise of parallelism is compelling: speeding up the execution but retaining all static compile-time guarantees about determinism. In other words, using parallelism could give us a substantial performance improvement without sacrificing the static guarantees of reproducible outputs from repeated runs.

Generally, parallelism can be applied whenever the execution of code is order-independent, that is, referential transparent and has no implicit or explicit side-effects. We follow the book \cite{marlow_parallel_2013}, which can be seen as the main authority for parallelism and concurrency in Haskell and refer to it for an in-depth discussions of parallel Haskell. From this reference, we can derive the following potential directions for parallelism in ABS:

\begin{itemize}
	\item \textit{Parallel map} - can be used without any second thoughts wherever \textit{map} is encountered. This can be on a very local scale within an agent, like mapping a simple function, or it can be global like executing the agents through map where much more work will be done in parallel due to the higher complexity of an agent function. Note that the latter one is only possible in non-monadic code.
	
	\item \textit{Data-parallel data-structures} - an environment could be organised and accessed through such a data-structure, which could potentially lead to big speed ups. Agents could locally read the data-structure data-parallel and the simulation kernel could feed the output of the agents data-parallel back into this structure.
	
	\item Par Monad: what for? not possible in yampa, but in dunai?
\end{itemize}

We keep this section rather short and exemplary and only try to apply the outlined parallelism techniques directly to our existing implementations and measure the speed-up. 

\subsection{Parallelism in non-monadic time-driven ABS}
here is obvious potential for adding (data-)parallelism to ABS e.g. u, in time-driven ABS agents can be updated in parallel using parMap because they all act conceptually at the same time as shown already in Yampa \cite{perez_60_2014}.

% http://keera.co.uk/blog/2014/10/15/from-60-fps-to-500/
% https://www.reddit.com/r/haskell/comments/2jbl78/from_60_frames_per_second_to_500_in_haskell/

par monad not possible in yampa


\subsection{Monadic SIR}
We can try to apply the same techniques of parallelising the agents as we did in the previous section in the non-monadic version of the SIR model. There is but a fundamental problem in this case, as we have already outlined in the section on data-flow parallelism: we are running the simulation in the monadic context of a \textit{ReaderT} and \textit{Rand} Monad stack. In monadic execution, depending on the monad (stack), we deal with side effects, which immediately necessitates the ordering of execution: whether an effectful expression is evaluated before another one can have indeed very fundamental differences and in general we have to assume that it does.
Indeed: the way the agents are evaluated is through the \textit{mapM} function, which evaluates them sequentially applying their side effects in sequence. It does not matter that the agents behave as if they are run in parallel without the possibility to interfere with each other, the simple fact that they are run within the \textit{ReaderT (Rand g)} transformer stack requires sequencing. It is not the \textit{ReaderT} which causes the delicate issue, it is rather the \textit{Rand} Monad, which basically behaves like a \textit{State} Monad with the random-number generator as internal state, which gets updated with each draw.
Due to this sequential evaluation, we can hypothesise that our approach is doomed from the beginning and that we will not see any speedup  when we apply parallelism - on the contrary, we can expect the performance to be worse with it due to the overhead caused by it.

Indeed, when we put our hypotheses to a test \footnote{We used the same experiment setup as in the non-monadic implementation.} we see exactly that behaviour: the sequential implementation, which does not use any parallelism and is not compiled with the -threaded option takes on average 41.76 seconds to finish. When adding parallelism with evaluation strategies in the same way as we did in non-monadic SIR, we end up with 49.63 seconds on average to finish - a clear performance \textit{decrease}! For the \textit{Par} Monad approach its even worse, which averages at 52.98 seconds to finish. These timings clearly show that 1) agents which are run in a monadic context with \textit{mapM} are not applicable to parallelism, 2) the parallelism mechanisms add a substantial overhead which is in accordance with the reports in \cite{marlow_parallel_2013}.

Still we don't give up completely and want to see if running the agents sequentially but some \textit{Par} monadic code \textit{within} them could gain us some speedup. The function we target is the neighbourhood querying function, which looks up the 8 (Moore) surrounding neighbours of an agent. It is a pure function and uses \textit{map} and is thus perfectly suitable to parallelism. We simply extend the transformer stack by putting the \textit{Par} Monad innermost and then run the \textit{neighbours} function within the \textit{Par} Monad:

\begin{HaskellCode}
-- type simplified for explanatory reasons
neighbours :: Disc2dCoord -> SIREnv -> Par [SIRState]
neighbours (x, y) e = do
    ivs <- mapM (\c -> spawn (return (e ! c))) nCoords
    mapM get ivs
  where
    nCoords = ... -- create neighbours coordinates
\end{HaskellCode}

Unfortunately the performance is even worse than without it, averaging at 66.68 seconds to finish. The workload seems to be too low for parallelism to pay off. Further, when keeping the \textit{Par} Monad as outermost Monad but using the original pure \textit{neighbours} function without \textit{Par} we arrive at an average of 55.9 seconds to finish when running multi-threaded on 8 cores and 45.56 seconds when compiled with threading enabled but running on a single core. These measurements demonstrate that using the \textit{Par} Monad and parallelism in general can lead to a substantially \textit{reduced} performance, due do massive overhead and too fine-grained parallelism.

This leaves us basically without any options of parallelism for the monadic SIR model. Still, we will come back to this use case in the chapter on concurrency, where we will show that by using concurrency it is possible to achieve a substantial speedup even in monadic computations.

\chapter{Pure Functional Event-Driven ABS}
\label{ch:eventdriven}

In this chapter we build on the previous discussion of update-strategies in Chapter \ref{ch:impl_abs} and the implementation techniques presented in the time-driven approach of Chapter \ref{ch:timedriven} to develop concepts for event-driven ABS in a pure functional way. 

In event-driven ABS \cite{meyer_event-driven_2014}, the simulation is advanced through events: agents and the environment schedule events into the future and react to incoming events scheduled by themselves, other agents, the environment or the simulation kernel. Time is discrete in this approach: it advances step-wise from event to event where each event has an accociated time-stamp which indicates the virtual simulation time when it is scheduled. This implies that time could stay constant e.g. when an event is scheduled with a time-delay of 0 the virtual simulation time does not advance. Because agents can adopt and change their state and behaviour when processing an event this means that even if time does not advance, agents can change. This non-signal behaviour is the fundamental difference to the time-driven approach in Chapter \ref{ch:timedriven}. Further, we exploit this mechanism to implement direct agent-interactions in pure functional ABS as discussed in our use-case of the Sugarscape model.

The event-driven approach makes the simulation kernel technically quite close to a Discrete Event Simulation (DES) \cite{zeigler_theory_2000}. Due to the necessity of imposing a correct ordering of events in this type of ABS, we are forced to step it event by event, with the \textit{sequential} update-strategy. Note that there exists also Parallel DES (PDES) \cite{fujimoto_parallel_1990}, which processes events in parallel and deals with inconsistencies by reverting to consistent states - we hypothesize that a pure functional approach could be beneficial due to persistent data-structures and explicit handling of side-effects but we leave this for further research.

We use the Sugarscape model to develop pure functional concepts for event-driven ABS \footnote{The code of all steps can be accessed freely through the following URL: \url{https://github.com/thalerjonathan/phd/tree/master/public/towards/SugarScape/sequential}}. We chose this model for the following reasons: it is quite well known in the ABS community; it was highly influential in sparking the interest in ABS; it is quite complex with non-trivial agent-interactions; the original implementation used object-oriented techniques (Objective C) and explicitly advocates them as a good fit to ABS which begged the question whether and how well a pure functional implementation is possible. The Sugarscape model is not a classic event-driven model as in it the agents do schedule events they don't do this into the future - events in Sugarscape don't have associated time-stamps. Still the underlying concepts are the same as in event-driven ABS and it is trivial to add time-stamps, moving towards a real event-driven ABS with DES character.

\section{Case-Study II: Sugarscape}
TODO: 
we can implement everything except synchronous direct agent-interactions atm: if agent-interaction is one-way e.g. paying back a loan then this is no problem. thus the following parts of the Sugarscape are not possible with our current STM approach: mating, trading and lending  because all 3 require direct agent-to-agent interaction over multiple steps. We leave the problem of developing such an algorithm / implementation for further research.

\section{Synchronised Agent-Interactions}
Following towards papers SugarScape implementation

% direct MSF call, Problem is recursive nature. maybe try it with gintis Implementation. I just learned that what i want to achieve is actually: https://en.wikipedia.org/wiki/This_(computer_programming)#Open_recursion AND OPEN RECURSION IS PRETTY BAD


\section{Discussion}

\subsection{Other Models}
TODO: mention that we have also implemented other models, which also work without time-semantics (all agents make a move at discrete time-steps and do not really rely on some notion of time). 

\subsection{Time-Semantics}
The main reason for building our pure functional ABMS approach on top of Yampa was to leverage the powerful time-semantics of Yampa which allows us to implement important concepts of ABMS:

state-chart: agents are at all time of their life-cycle in one state and can switch between multiple states using transitions 
timed transitions: transition to another state/behaviour happens at a discrete time
rate transitions: transition happens with a given rate
message transition: transition upon receiving a given message 

\subsection{Agents as Signals}
Due to the underlying nature and motivation of Functional Reactive Programming (und im speziellen) Yampa, Agents can be seen as Signals which is generated and consumed by a Signal-Function which is the behaviour of an Agent. If an Agent does not change the OUTPUT-signal is constant, if the agent changes e.g. by sending a message, changing its state,... the OUTPUT signal changes. A dead agent has no signal at all.

\subsection{Time-Sampling}
sampling rate depends on the transition times \& rates of the model. when e.g. the contact rate is 5 then the sampling dt should be below 0.2

\subsection{System Dynamics}
can emulate system dynamics due to the parallel update-strategy and continuous time-flow semantics

\subsection{Discrete Event Simulation}
DES in FrABMS? how easily can we implement server/queue systems? do they also look like a specification? potential problem: ordering of messages is not guaranteed by now

\subsection{Advantages}
advantages:
	- no side-effects within agents leads to much safer code
	- edsl for time-semantics
	- declarative style: agent-implementation looks like a model-specification
	- reasoning and verification
	- sequential and parallel
	- powerful time-semantics
	- arrowized programming is optional and only required when utilizing yampas time-semantics. if the model does not rely on time-semantics, it can use monadic-programming by building on the existing monadic functions in the EDSL which allow to run in the State-Monad which simplifies things very much
	- when to use yampas arrowized programing: time-semantics, simple state-chart agents 
	- when not using yampas facilities: in all the other cases e.g. SugarScape is such a case as it proceeds in unit time-steps and all agents act in every time-step
	- can implement System Dynamics building on Yampas facilities with total ease	
	- get replications for free without having to worry about side-effects and can even run them in parallel without headaches
	- cant mess around with time because delta-time is hidden from you (intentional design-decision by Yampa). this would be only very difficult and cumbersome to achieve in an object-oriented approach. TODO: experiment with it in Java - how could we actually implement this? I think it is impossible: may only achieve this through complicated application of patterns and inheritance but then has the problem of how to update the dt and more important how to deal with functions like integral which accumulates a value through closures and continuations. We could do this in OO by having a general base-class e.g. ContinuousTime which provides functions like updateDt and integrate, but we could only accumulate a single integral value.
	- reproducibility statically guaranteed
	- cannot mess around with dt
	- code == specification
	- rule out serious class of bugs
	- different time-sampling leads to different results e.g. in wildfire \& SIR but not in Prisoners Dilemma. why? probabilistic time-sampling?
	- reasoning about equivalence between SD and ABS implementation in the same framework
	- recursive implementations
	
	- we can statically guarantee the reproducibility of the simulation because: no side effects possible within the agents which would result in differences between same runs (e.g. file access, networking, threading), also timedeltas are fixed and do not depend on rendering performance or userinput	
	
\subsection{Disadvantages}
disadvantages:
	- performance is low
	- reasoning about performance is very difficult
	- very steep learning curve for non-functional programmers
	- learning a new EDSL
	- think ABMS different: when to use async messages, when to use sync conversations


[ ] important: increasing sampling freqzency and increasing number of steps so that the same number of simulation steps are executed should lead to same results. but it doesnt. why?
[ ] hypothesis: if time-semantics are involved then event ordering becomes relevant for emergent patterns. there are no tine semantics in heroes and cowards but in the prisoners dilemma
[ ] can we implement different types of agents interacting with each other in the same simulation ? with different behaviour funcs, digferent state? yes, also not possible in NetLogo to my knowledge. but they must have the same messages, emvironment 

[ ] Hypothesis: we can combine with FrABS agent-based simulation and system dynamics (this has been proved by example!)

\subsection{Parallel Runs}
Often one needs to perform a large number of runs of the same simulation. The most prominent use-cases for this are:

\begin{itemize}
	\item Parameter Sweeps / Variations - To explore the parameter space and the dynamics under varying parameter configurations, the same simulation is run with varying parameters and the results recorded for statistical analysis.
	
	\item Stochastic replications - Due to ABS stochastic nature, running a simulation only once does not allow to generalise or predict overall behaviour - one might have just hit an (un)fortunate special case. To counter this problem, in ABS multiple replications of the  simulation are run with same initial model parameters but with different random-number streams. All the results are collected and analysed stochastically (averaged, median,...) from which then more general properties can be derived.
\end{itemize}

In each case thousands of runs of the same simulation with different model parameters and / or varying random-number streams are needed, requiring a considerable amount of computing power.

Parallelism is a remedy to this problem because in each of these cases individual runs do not interfere with each other and thus can be seen as isolated from each other, like referential, pure computations. Our approaches shown in the Part II make this very explicit: the top level functions can always be made pure computations because we are ruling out IO (so far) and thus even though Monads are employed in many cases, they are still pure. A benefit of our approach is that it is guaranteed at compile time, that individual runs do not interfere with each other and thus there is no danger that parallel runs influence each other. 

All this allows to implement parameter sweeps and stochastic replications both through evaluation and data-flow parallelism making another very compelling use-case - probably the most striking one - for the use of parallelism in ABS. We hypothesize that data-flow parallelism is better suited for this task because it makes parallelism more explicit as it is indeed a data-flow problem: we pass parameters to single replications which are run and return their results. To apply this we simply run the top level replication logic in the Par Monad where replications are run in parallel by forking tasks and results are handed back through IVars. If we want the convenience of having a monadic random-number generator within the Par monad as well, one can use the combined ParRand monad which provides both.

\subsection{Reflection}
In general we aimed at running agents in parallel using the various techniques. Because of the quite sequential nature of the agent behaviours themselves, there is much less potential for parallelism \textit{within} an agent, thus the obvious idea was to run them all in parallel because they are an obvious unit of partitioning, have considerable workload and can indeed be run in parallel under given circumstances.
Unfortunately it is not possible applying parallelism in case the agents run within a monadic context: we have side-effects which imposes ordering e.g. in the case of a

We see a direct consequence of this that types also reflect the semantics of our model: when our agents are pure they can be run indeed in parallel and independent from each other, if they are monadic, then this is not applicable to parallelism. In the next section, we show how to approach this problem and come up with a solution where we can run monadic agents in parallel. This is obviously only possible within a concurrent setting which means we have to sacrifice determinism in our solution. Still we reach considerable speed ups using Software Transactional Memory.