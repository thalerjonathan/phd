\section{Parallelism in ABS}
The promise of parallelism in Haskell is compelling: speeding up the execution but retaining all static compile-time guarantees about determinism. In other words, using parallelism could give us a substantial performance improvement without sacrificing the static guarantees of reproducible outputs from repeated runs with initial conditions.

Generally, parallelism can be applied whenever the execution of code is order-independent, that is referential transparent, and has no implicit or explicit side-effects. In this section we introduce the two most important parallelism concepts of Haskell, \textit{evaluation} and \textit{data-flow} parallelism, and discuss their potential use in pure functional ABS in general. We follow \cite{marlow_parallel_2013} and refer to it for an in-depth discussion. Further, we show how these concepts can be added to our previously discussed use-cases of Chapters \ref{sec:timedriven_firststep}, \ref{sec:adding_env} and Sugarscape \ref{sec:eventdriven_implementation} and compare their performance over the original sequential approaches.

\section{Evaluation Parallelism}
Evaluation parallelism introduces so called strategies to evaluate lazy data structures in parallel. Example strategies would be to evaluate a list, or tuples in parallel where a spark is created for each element. The fundamental concept Haskell uses to achieve evaluation parallelism is its own non-strictness nature. Non-strictness means that expressions are not eagerly evaluated when defined, like in imperative programming languages but only evaluated when their result is actually needed. This is implemented internally using thunks, which are pointers to expressions. When the value of an expression is needed, this thunk is accessed and the expression is reduced until the next data constructor or lambda expression is encountered. This is called Weak Head Normal Form (WHNF) evaluation because it only reduces the 'head' of the expression, which could consist of sub expressions. This indirection, the separation of data creation from consumption and evaluation, enables evaluation parallelism and Haskell provides two additional functions to support this:

\begin{itemize}
	\item \texttt{par :: a $\rightarrow$ b $\rightarrow$ b} returns the second argument \texttt{b} but evaluates the first argument \texttt{a} in parallel. It is used when the result of evaluating \texttt{a} is required later.
	
	\item \texttt{seq :: a $\rightarrow$ b $\rightarrow$ b} returns the second argument \texttt{b} but is strict in its first argument, which means it forces its evaluation to WHNF. It is used when the result of evaluating \texttt{a} is required now.
\end{itemize}

Internally, evaluation parallelism is handled through so-called \textit{sparks}, which are thunks evaluated in parallel. The Haskell runtime system manages sparks and distributes them to threads where they get executed. Due to their extremely light-weight nature, it is easy to create tens of thousands of sparks. One has to bear in mind that, although evaluating in parallel through sparks is extremely cheap, it still has some overhead. Thus, if the work load of each element in a list might be too low for a spark, then one can split a list into chunks and distribute them onto a single spark.
All this works without side effects and the strategy combinators are all pure functions building on \texttt{par} and \texttt{seq}. This allows us to add parallelism to an algorithm by applying a parallel evaluation strategy to its result which could be a lazy list. This is made possible by the non-strictness nature of Haskell, which separates the construction of data from its consumption.

\subsection{Evaluation Parallelism in ABS}
Using \textit{compositional} parallelism is exactly what we used to aim at adding evaluation parallelism for agent execution in the non-monadic SIR implementation in Chapter \ref{sec:timedriven_firststep}. We know that the whole simulation is a completely pure computation because Yampa is non-monadic. Consequently it is guaranteed that there are no side effects. Moreover, agents are then run conceptually in parallel using \texttt{map}, which should enable us to add parallelism without needing to reimplement \texttt{dpSwitch} (the function running the agents in parallel). %(also re-implementing switch functions would not get us very far because of WHNF evaluation it is the wrong end to start parallel evaluation: probably only the arguments would be evaluated but not the agent behaviour.)

The solution is to add evaluation parallelism in the agent-output collection phase, where the recursive switch into the \texttt{stepSimulation} function happens. It is there where we use an evaluation strategy to evaluate the outputs of all agents in parallel. The agents will then be evaluated in parallel due to compositional parallelism, when we force the output of each in parallel. We provide more details on the topic in a short case study in section \ref{parallel_nonmonadic_sir}.

\section{Data-flow parallelism}
When relying on a lazy data structure to apply parallelism is not an option, evaluation strategies as presented before are not applicable. Further, although lazy evaluation brings compositional parallelism, it makes it hard to reason about performance. Data-flow parallelism offers an alternative over evaluation strategies, where the programmer can give more details but gains more control: data dependencies are made explicit and reliance on lazy evaluation is avoided.
Data-flow parallelism is implemented through the \textit{Par} Monad, which provides combinators for expressing data-flows: in this monad it is possible to \textit{fork} parallel tasks which communicate with each other through shared locations, so called \textit{IVar}s. Internally these tasks are scheduled by a work-stealing scheduler which distributes the work evenly on available processors at runtime. \textit{IVars} behave like futures or promises: they are initially empty and can be written once. Reading from an empty \textit{IVar} will cause the calling task (or main thread) to wait until it is filled. An example is a parallel evaluation of two fibonacci numbers:

\begin{HaskellCode}
runPar (do
  i <- new             -- create new IVar
  j <- new             -- create new IVar
  fork (put i (fib n)) -- fork new task compute fib n and put result into IVar i
  fork (put j (fib m)) -- fork new task compute fib m and put result into IVar j
  a <- get i           -- wait for the result from IVar i and collect it
  b <- get j           -- wait for the result from IVar j and collect it
  return (a,b)         -- return the sum
\end{HaskellCode}

Note that with this it is also possible to express parallel evaluation of a list or a tuple as with evaluation strategies. The difference though is, that it does avoid lazy evaluation. More importantly, putting a value into an \textit{IVar} requires the type of the value to have an instance of the \textit{NFData} typeclass. This simply means that a value of this type can be fully evaluated, not just to WHNF but to evaluate the full expression the value represents.

\subsection{Data-flow parallelism in ABS}
NOTE: running the agents in parallel with par doesn't work because we use mapM and are thus monadic, which involves sequencing. so this is really out of the window here. Also we cannot put a Par in a transformer stack because the library doesn't support it, what actually makes sense. But we can do the following: we can run an agents MSF only within the Par monad which gives agents the ability to spawn data-flow parallel computations - random-number streams are handled like in the non-monadic version. Note that this is only possible with the MSFs of dunai and not the SF because the latter one adds already the a ReaderT DTime which makes it impossible already. 
What is actually possible would be to write a combined monad for Par and ReaderT because the latter one is a read-only value and could thus potentially run in parallel - we leave this for further research. There exists also a combination of the Par with the Rand monad, so if the time-driven approach is not needed then this could be used to give the agents the ability to both draw random numbers AND do deterministic data-parallel computations. The agents can then be run in parallel through the par monad.


% IGNORE FOR NOW
%\subsection{Data-structure parallelism}
%An environment could be organised and accessed through such a data-structure, which could potentially lead to big speed ups. Agents could locally read the data-structure data-parallel and the simulation kernel could feed the output of the agents data-parallel back into this structure.
%
%%https://learning.oreilly.com/library/view/parallel-and-concurrent/9781449335939/ch05.html
%
%general solution we opt for is  to run agents in parallel in our approaches. in other abs models we could apply data-structure parallelism and/or data-flow parallelism with huge Performance potential but thats always highly model dependent thus we dont go in depth here
\section{Case studies}
In this section we go a little bit more into detail how to apply the parallelism concepts as already outline above to our use-cases from Chapters \ref{sec:timedriven_firststep}, \ref{sec:adding_env} and Sugarscape \ref{sec:advanced_eventdriven_ABS}. We briefly demonstrate the technical details and refer to the full code in footnotes. Note that all timings are rough averages over multiple runs and not precise measurements because that is not the point here. We are only interested in showing what rough potential there is for speeding up computation through deterministic parallelism - we are not interested in high performance computation here but rather in conceptual comparisons between sequential and parallel implementations.

\subsection{Parallelism in non-monadic time-driven ABS}
here is obvious potential for adding (data-)parallelism to ABS e.g. u, in time-driven ABS agents can be updated in parallel using parMap because they all act conceptually at the same time as shown already in Yampa \cite{perez_60_2014}.

% http://keera.co.uk/blog/2014/10/15/from-60-fps-to-500/
% https://www.reddit.com/r/haskell/comments/2jbl78/from_60_frames_per_second_to_500_in_haskell/

par monad not possible in yampa


\subsection{Monadic SIR}
We can try to apply the same techniques of parallelising the agents as we did in the previous section in the non-monadic version of the SIR model. There is but a fundamental problem in this case, as we have already outlined in the section on data-flow parallelism: we are running the simulation in the monadic context of a \textit{ReaderT} and \textit{Rand} Monad stack. In monadic execution, depending on the monad (stack), we deal with side effects, which immediately necessitates the ordering of execution: whether an effectful expression is evaluated before another one can have indeed very fundamental differences and in general we have to assume that it does.
Indeed: the way the agents are evaluated is through the \textit{mapM} function, which evaluates them sequentially applying their side effects in sequence. It does not matter that the agents behave as if they are run in parallel without the possibility to interfere with each other, the simple fact that they are run within the \textit{ReaderT (Rand g)} transformer stack requires sequencing. It is not the \textit{ReaderT} which causes the delicate issue, it is rather the \textit{Rand} Monad, which basically behaves like a \textit{State} Monad with the random-number generator as internal state, which gets updated with each draw.
Due to this sequential evaluation, we can hypothesise that our approach is doomed from the beginning and that we will not see any speedup  when we apply parallelism - on the contrary, we can expect the performance to be worse with it due to the overhead caused by it.

Indeed, when we put our hypotheses to a test \footnote{We used the same experiment setup as in the non-monadic implementation.} we see exactly that behaviour: the sequential implementation, which does not use any parallelism and is not compiled with the -threaded option takes on average 41.76 seconds to finish. When adding parallelism with evaluation strategies in the same way as we did in non-monadic SIR, we end up with 49.63 seconds on average to finish - a clear performance \textit{decrease}! For the \textit{Par} Monad approach its even worse, which averages at 52.98 seconds to finish. These timings clearly show that 1) agents which are run in a monadic context with \textit{mapM} are not applicable to parallelism, 2) the parallelism mechanisms add a substantial overhead which is in accordance with the reports in \cite{marlow_parallel_2013}.

Still we don't give up completely and want to see if running the agents sequentially but some \textit{Par} monadic code \textit{within} them could gain us some speedup. The function we target is the neighbourhood querying function, which looks up the 8 (Moore) surrounding neighbours of an agent. It is a pure function and uses \textit{map} and is thus perfectly suitable to parallelism. We simply extend the transformer stack by putting the \textit{Par} Monad innermost and then run the \textit{neighbours} function within the \textit{Par} Monad:

\begin{HaskellCode}
-- type simplified for explanatory reasons
neighbours :: Disc2dCoord -> SIREnv -> Par [SIRState]
neighbours (x, y) e = do
    ivs <- mapM (\c -> spawn (return (e ! c))) nCoords
    mapM get ivs
  where
    nCoords = ... -- create neighbours coordinates
\end{HaskellCode}

Unfortunately the performance is even worse than without it, averaging at 66.68 seconds to finish. The workload seems to be too low for parallelism to pay off. Further, when keeping the \textit{Par} Monad as outermost Monad but using the original pure \textit{neighbours} function without \textit{Par} we arrive at an average of 55.9 seconds to finish when running multi-threaded on 8 cores and 45.56 seconds when compiled with threading enabled but running on a single core. These measurements demonstrate that using the \textit{Par} Monad and parallelism in general can lead to a substantially \textit{reduced} performance, due do massive overhead and too fine-grained parallelism.

This leaves us basically without any options of parallelism for the monadic SIR model. Still, we will come back to this use case in the chapter on concurrency, where we will show that by using concurrency it is possible to achieve a substantial speedup even in monadic computations.
\\
\section{Case-Study II: Sugarscape}
TODO: 
we can implement everything except synchronous direct agent-interactions atm: if agent-interaction is one-way e.g. paying back a loan then this is no problem. thus the following parts of the Sugarscape are not possible with our current STM approach: mating, trading and lending  because all 3 require direct agent-to-agent interaction over multiple steps. We leave the problem of developing such an algorithm / implementation for further research.

\subsection{Parallel Runs}
Often one needs to perform a large number of runs of the same simulation. The most prominent use-cases for this are:

\begin{itemize}
	\item Parameter Sweeps / Variations - To explore the parameter space and the dynamics under varying parameter configurations, the same simulation is run with varying parameters and the results recorded for statistical analysis.
	
	\item Stochastic replications - Due to ABS stochastic nature, running a simulation only once does not allow to generalise or predict overall behaviour - one might have just hit an (un)fortunate special case. To counter this problem, in ABS multiple replications of the  simulation are run with same initial model parameters but with different random-number streams. All the results are collected and analysed stochastically (averaged, median,...) from which then more general properties can be derived.
\end{itemize}

In each case thousands of runs of the same simulation with different model parameters and / or varying random-number streams are needed, requiring a considerable amount of computing power.

Parallelism is a remedy to this problem because in each of these cases individual runs do not interfere with each other and thus can be seen as isolated from each other, like referential, pure computations. Our approaches shown in the Part II make this very explicit: the top level functions can always be made pure computations because we are ruling out IO (so far) and thus even though Monads are employed in many cases, they are still pure. A benefit of our approach is that it is guaranteed at compile time, that individual runs do not interfere with each other and thus there is no danger that parallel runs influence each other. 

All this allows to implement parameter sweeps and stochastic replications both through evaluation and data-flow parallelism making another very compelling use-case - probably the most striking one - for the use of parallelism in ABS. We hypothesize that data-flow parallelism is better suited for this task because it makes parallelism more explicit as it is indeed a data-flow problem: we pass parameters to single replications which are run and return their results. To apply this we simply run the top level replication logic in the Par Monad where replications are run in parallel by forking tasks and results are handed back through IVars. If we want the convenience of having a monadic random-number generator within the Par monad as well, one can use the combined ParRand monad which provides both.

\subsection{Reflection}
In general we aimed at running agents in parallel using the various techniques. Because of the quite sequential nature of the agent behaviours themselves, there is much less potential for parallelism \textit{within} an agent, thus the obvious idea was to run them all in parallel because they are an obvious unit of partitioning, have considerable workload and can indeed be run in parallel under given circumstances.
Unfortunately it is not possible applying parallelism in case the agents run within a monadic context: we have side-effects which imposes ordering e.g. in the case of a

We see a direct consequence of this that types also reflect the semantics of our model: when our agents are pure they can be run indeed in parallel and independent from each other, if they are monadic, then this is not applicable to parallelism. In the next section, we show how to approach this problem and come up with a solution where we can run monadic agents in parallel. This is obviously only possible within a concurrent setting which means we have to sacrifice determinism in our solution. Still we reach considerable speed ups using Software Transactional Memory.