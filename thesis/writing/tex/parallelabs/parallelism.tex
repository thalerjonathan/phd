\section{Parallelism in ABS}
The promise of parallelism in Haskell is compelling: speeding up the execution but retaining all static compile-time guarantees about determinism. In other words, using parallelism could give us a substantial performance improvement without sacrificing the static guarantees of reproducible outputs from repeated runs with initial conditions.

Generally, parallelism can be applied whenever the execution of code is order-independent, that is referential transparent, and has no implicit or explicit side-effects. Without going into too much technical detail, in this section we outline the parallelism techniques available in Haskell and briefly discuss how they can be used in ABS in general. We also discuss if and how parallelism can be added to our previously discussed use-cases of Chapters \ref{sec:timedriven_firststep}, \ref{sec:adding_env} and Sugarscape TODO and report the performance improvements where applicable.

\subsection{Parallelism in Haskell}
We follow the book \cite{marlow_parallel_2013}, which can be seen as the main source for parallelism and concurrency in Haskell and refer to it for an in-depth discussions of parallel Haskell.

\paragraph{Evaluation parallelism}
Put short, evaluation parallelism allows to build functions which run in parallel e.g. a parallel version of \textit{map}, which is called \textit{parMap}. This is achieved using the \textit{Eval Monad} which is run using a parallel evaluation strategy, arriving at a pure value - the evaluation of the \textit{Eval Monad} itself is pure and does not require the IO (this is exactly what we expect from parallelism: to be deterministic). Obviously this gives huge potential for speeding up programs because maps are omnipresent in a lot of functional code. Not only parmap! explain a little bit more in detail without going into too much technical stuff.
%https://www.oreilly.com/library/view/parallel-and-concurrent/9781449335939/ch02.html
%https://www.oreilly.com/library/view/parallel-and-concurrent/9781449335939/ch03.html

\paragraph{Data-flow parallelism}
Par Monad: how does it work?
%https://www.oreilly.com/library/view/parallel-and-concurrent/9781449335939/ch04.html

\paragraph{Data-structure parallelism}
An environment could be organised and accessed through such a data-structure, which could potentially lead to big speed ups. Agents could locally read the data-structure data-parallel and the simulation kernel could feed the output of the agents data-parallel back into this structure.

%https://learning.oreilly.com/library/view/parallel-and-concurrent/9781449335939/ch05.html

general solution we opt for is  to run agents in parallel in our approaches. in other abs models we could apply data-structure parallelism and/or data-flow parallelism with huge Performance potential but thats always highly model dependent thus we dont go in depth here

\subsection{Use-Cases}

\subsubsection{Non-Monadic SIR}
Although \textit{parMap} can be applied in all cases where a map us used, we are particularly interested in running agents in parallel. With \textit{parMap} this should become possible in our non-monadic SIR implementation built on Yampa from Chapter \ref{sec:timedriven_firststep}. Even thought the Eval Monad is used under the hood and Yampa is non-monadic, it is still applicable because running the monad is pure, resulting in a pure result - \textit{parMap} is a pure function. TODO: how can we apply this?

Inspired by the work of \cite{perez_60_2014}, which shows the potential of speeding up real-world Haskell programs using Yampa We conducted a comparison of an implementation which makes use of evaluation parallelism to run agents in parallel.


% https://www.reddit.com/r/haskell/comments/2jbl78/from_60_frames_per_second_to_500_in_haskell/\\

\subsubsection{Monadic SIR}
Unfortunately \textit{parMap} is not applicable to the monadic SIR version of Chapter \ref{sec:adding_env} because of the use of mapM, which cannot be replaced due to its inherent sequential nature: mapM runs monadic actions which have side-effects thus ordering matters. Even if the implementation in that chapter behaves as if the agents are run in parallel, technically they are run sequentially because of the need for the Random Monad effect. This leaves us basically without any options of parallelism for the monadic SIR model, we will come back to this use-case in the concurrency section, where we will show that by using concurrency it is possible to achieve a substantial speed up of orders of magnitude.

\subsubsection{Sugarscape}
use parmap for updating Pollution/regrow resources. still agents can't be run in parallel because of monadic effects, we show in the concurrent section how we can use concurrency to achieve a substantial speed up using STM.

compare Environment parallelism between sequential and concurrent sugarscape: should see alarger speedup in conc bcs the sequential percentage is larger there

\subsection{Parallel Runs}
Often one needs to perform a large number of runs of the same simulation. The most prominent use-cases for this are:

\begin{itemize}
	\item Parameter Sweeps / Variations - To explore the parameter space and the dynamics under varying parameter configurations, the same simulation is run with varying parameters and the results recorded for statistical analysis.
	
	\item Stochastic replications - Due to ABS stochastic nature, running a simulation only once does not allow to generalise or predict overall behaviour - one might have just hit an (un)fortunate special case. To counter this problem, in ABS multiple replications of the  simulation are run with same initial model parameters but with different random-number streams. All the results are collected and analysed stochastically (averaged, median,...) from which then more general properties can be derived.
\end{itemize}

In each case thousands of runs of the same simulation with different model parameters and / or varying random-number streams are needed, requiring a considerable amount of computing power.

Parallelism is a remedy to this problem because in each of these cases individual runs do not interfere with each other and thus can be seen as isolated from each other, like referential, pure computations. Our approaches shown in the Part II make this very explicit: the top level functions can always be made pure computations because we are ruling out IO (so far) and thus even though Monads are employed in many cases, they are still pure. A benefit of our approach is that it is guaranteed at compile time, that individual runs do not interfere with each other and thus there is no danger that parallel runs influence each other. 

All this allows to implement parameter sweeps and stochastic replications both through evaluation and data-flow parallelism making another very compelling use-case - probably the most striking one - for the use of parallelism in ABS. We hypothesize that data-flow parallelism is better suited for this task because it makes parallelism more explicit as it is indeed a data-flow problem: we pass parameters to single replications which are run and return their results. To apply this we simply run the top level replication logic in the Par Monad where replications are run in parallel by forking tasks and results are handed back through IVars. If we want the convenience of having a monadic random-number generator within the Par monad as well, one can use the combined ParRand monad which provides both.

\subsection{Reflection}
In general we aimed at running agents in parallel using the various techniques. Because of the quite sequential nature of the agent behaviours themselves, there is much less potential for parallelism \textit{within} an agent, thus the obvious idea was to run them all in parallel because they are an obvious unit of partitioning, have considerable workload and can indeed be run in parallel under given circumstances.
Unfortunately it is not possible applying parallelism in case the agents run within a monadic context: we have side-effects which imposes ordering e.g. in the case of a

We see a direct consequence of this that types also reflect the semantics of our model: when our agents are pure they can be run indeed in parallel and independent from each other, if they are monadic, then this is not applicable to parallelism. In the next section, we show how to approach this problem and come up with a solution where we can run monadic agents in parallel. This is obviously only possible within a concurrent setting which means we have to sacrifice determinism in our solution. Still we reach considerable speed ups using Software Transactional Memory.