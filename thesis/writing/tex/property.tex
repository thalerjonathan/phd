\chapter*{} %Property-Based Testing in pure functional ABS
\label{ch:property}

When implementing an Agent-Based Simulation (ABS) it is of fundamental importance that the implementation is correct up to some specification and that this specification matches the real world in some way. This process is called verification and validation (V\&V), where \textit{validation} is the process of ensuring that a model or specification is sufficiently accurate for the purpose at hand whereas \textit{verification} is the process of ensuring that the model design has been transformed into a computer model with sufficient accuracy \cite{robinson_simulation:_2014}. In other words, validation determines if we are we building the \textit{right model}, and verification if we are building the \textit{model right} up to some specification \cite{balci_verification_1998}.

% there is no general validity, an approach is TDD: V&V particularly difficult in ABS
One can argue that ABS should require more rigorous programming standards than other computer simulations \cite{polhill_ghost_2005}. Because researchers in ABS look for an emergent behaviour in the dynamics of the simulation, they are always tempted to look for some surprising behaviour and expect something unexpected from their simulation. 
Also, due to ABS' \textit{constructive / exploratory} nature \cite{epstein_chapter_2006, epstein_generative_2012}, there exists some uncertainty about the dynamics the simulation will produce before running it. The authors \cite{ormerod_validation_2006} see the current process of building ABS as a discovery process where models of an ABS often lack an analytical solution in general, which makes verification much harder if there is no such solution. Thus it is often very difficult to judge whether an unexpected outcome can be attributed to the model or has in fact its roots in a subtle programming error \cite{galan_errors_2009}.

In general this implies that it is not possible to prove that a model is valid in general but that the best we can do is to \textit{raise the confidence} in the correctness of the simulation. Therefore, the process of V\&V is not the proof that a model is correct but the \textit{process} of trying to show that the model is \textit{not incorrect}. The more checks one carries out which show that it is not incorrect, the more confidence we can place on the models validity. To tackle such a problem in software, software engineers have developed the concept of test-driven development (TDD).

Test-Driven Development (TDD) was popularised in the early 00s by Kent Beck \cite{beck_test_2002} as a way to a more agile approach to software-engineering, where instead of doing each step (requirements, implementation, testing,...) as separated from each other, all of them are combined in shorter cycles. Put shortly, in TDD tests are written for each feature before actually implementing it, then the feature is fully implemented and the tests for it should pass. This cycle is repeated until the implementation of all requirements has finished. Traditionally TDD relies on so called unit-tests which can be understood as a piece of code which when run isolated, tests some functionality of an implementation. Thus we can say that test-driven development in general and unit testing together with code-coverage in particular, guarantee the correctness of an implementation to some informal degree, which has been proven to be sufficiently enough through years of practice in the software industry all over the world. 

\medskip

The work \cite{collier_test-driven_2013} was the first to discusses how to apply TDD to ABS, using unit testing to verify the correctness of the implementation up to a certain level. They show how to implement unit-tests within the RePast Framework \cite{north_complex_2013} and make the important point that such a software needs to be designed to be sufficiently modular otherwise testing becomes too cumbersome and involves too many parts. The paper \cite{asta_investigation_2014} discusses a similar approach to DES in the AnyLogic software toolkit. 

The paper \cite{onggo_test-driven_2016} proposes Test Driven Simulation Modelling (TDSM) which combines techniques from TDD to simulation modelling. The authors present a case study for maritime search-operations where they employ ABS. They emphasise that simulation modelling is an iterative process, where changes are made to existing parts, making a TDD approach to simulation modelling a good match. They present how to validate their model against analytical solutions from theory using unit-tests by running the whole simulation within a unit-test and then perform a statistical comparison against a formal specification. %This approach will become of importance later on in our SIR and Sugarscape case studies.

The paper \cite{gurcan_generic_2013} gives an in-depth and detailed overview over verification, validation and testing of agent-based models and simulations and proposes a generic framework for it. The authors present a generic UML class model for their framework which they then implement in the two ABS frameworks RePast and MASON. Both of them are implemented in Java and the authors provide a detailed description how their generic testing framework architecture works and how it utilises unit testing with JUnit to run automated tests. To demonstrate their framework they provide also a case study of an agent-base simulation of synaptic connectivity where they provide an in-depth explanation of their levels of test together with code.

\medskip

% the gap
Although it would be interesting to see how we can apply unit testing to our approach, it is straight forward, nothing new and does not constitute unique research. Thus, in this chapter we introduce an additional technique for TDD: \textit{property-based testing}, which can be seen as complementary to unit testing. Property-based testing has its origins \cite{claessen_quickcheck_2000,claessen_testing_2002,runciman_smallcheck_2008} in Haskell, where it was first conceived and implemented. It has been successfully used for testing Haskell code for years and also been proven to be useful in the industry \cite{hughes_quickcheck_2007}. We show and discuss how this technique can be applied to test pure functional ABS implementations. To our best knowledge property-based testing has never been looked at in the context of ABS and this thesis is the first one to do so.

\medskip

The main idea of property-based testing is to express model-specifications and laws directly in code and test them through \textit{automated} and \textit{randomised} test-data generation. Thus one hypothesis of this thesis is that due to ABS \textit{stochastic} and \textit{exploratory / generative / constructive } nature, property-based testing is a natural fit for testing ABS in general and pure functional ABS implementations in particular. It thus should pose a valuable addition to the already existing testing methods in this field, worth exploring. To substantiate and test our hypothesis, we present two case-studies. First, the agent-based SIR model as introduced in Chapter \ref{sec:sir_model}, which is of explanatory nature, where we show how to express formal model-specifications in property-tests. Second, the SugarScape model as introduced in Chapter \ref{sec:sugarscape}, which is of exploratory nature, where we show how to express hypotheses in property-tests and how to property-test agent functionality. 
%Again, we emphasise that we see it as an addition to TDD, where it works in combination with unit testing to verify and validate a simulation to increase the confidence in its correctness. % and is a useful tool for expressing regression tests.

\medskip

Note that property-based testing has a close connection to model-checking \cite{mcmillan_symbolic_1993}, where properties of a system are proved in a formal way. The important difference is that the checking happens directly on code and not on the abstract, formal model, thus one can say that it combines model-checking and unit testing, embedding it directly in the software-development and TDD process without an intermediary step. We hypothesise that adding it to the already existing testing methods in the field of ABS is of substantial value as it allows to cover a much wider range of test-cases due to automatic data generation. This can be used in two ways: to verify an implementation against a formal specification and to test hypotheses about an implemented simulation. This puts property-based testing on the same level as agent- and system testing, where not technical implementation details of e.g. agents are checked like in unit-tests but their individual complete behaviour and the system behaviour as a whole.

The work \cite{onggo_test-driven_2016} explicitly mentions the problem of test coverage which would often require to write a large number of tests manually to cover the parameter ranges sufficiently enough - property-based testing addresses exactly this problem by \textit{automating} the test-data generation. Note that this is closely related to data-generators \cite{gurcan_generic_2013}, load generators and random testing \cite{burnstein_practical_2010}. Property-based testing though goes one step further by integrating this into a specification language directly into code, emphasising a declarative approach and pushing the generators behind the scenes, making them transparent and focusing on the specification rather than on the data-generation. 

\section{Property-based testing}
\label{sec:proptesting}
Property-based testing allows to formulate \textit{functional specifications} in code which then a property-based testing library tries to falsify by \textit{automatically} generating test data, covering as much cases as possible. When a case is found for which the property fails, the library then reduces the test data to its simplest form for which the test still fails, for example shrinking a list to a smaller size. It is clear to see that this kind of testing is especially suited to ABS, because we can formulate specifications, meaning we describe \textit{what} to test instead of \textit{how} to test. Also the deductive nature of falsification in property-based testing suits very well the constructive and exploratory nature of ABS. Further, the automatic test generation can make testing of large scenarios in ABS feasible because it does not require the programmer to specify all test cases by hand, as is required in traditional unit tests.

Property-based testing was introduced in \cite{claessen_quickcheck_2000,claessen_testing_2002} where the authors present the QuickCheck library in Haskell, which tries to falsify the specifications by \textit{randomly} sampling the test space. %We argue, that the stochastic sampling nature of this approach is particularly well suited to ABS, because it is itself almost always driven by stochastic events and randomness in the agents behaviour, thus this correlation should make it straightforward to map ABS to property-testing.
%The main challenge when using QuickCheck, as will be shown later, is to write \textit{custom} test data generators for agents and the environment which cover the space sufficiently enough to not miss out on important test cases.
According to the authors of QuickCheck \textit{"The major limitation is that there is no measurement of test coverage."} \cite{claessen_quickcheck_2000}. Although QuickCheck provides help to report the distribution of test cases it is not able to measure the coverage of tests in general. This could lead to the case that test cases which would fail are never tested because of the stochastic nature of QuickCheck. Fortunately, the library provides mechanisms for the developer to measure coverage in specific test cases where the data and its expected distribution is known to the developer. This is a powerful tool for testing randomness in ABS as will be shown in the next chapters.

\medskip

As a remedy for the potential coverage problems of QuickCheck, there exists also a deterministic property-testing library called SmallCheck \cite{runciman_smallcheck_2008}, which instead of randomly sampling the test space, enumerates test cases exhaustively up to some depth. It is based on two observations, derived from model-checking, that (1) \textit{"If a program fails to meet its specification in some cases, it almost always fails in some simple case"} and (2) \textit{"If a program does not fail in any simple case, it hardly ever fails in any case} \cite{runciman_smallcheck_2008}. This non-stochastic approach to property-based testing might be a complementary addition in some cases where the tests are of non-stochastic nature with a search space too large to test manually by unit testing but small enough to enumerate exhaustively. The main difficulty and weakness of using SmallCheck is to reduce the dimensionality of the test case depth search to prevent combinatorial explosion, which would lead to exponential number of cases. Thus one can see QuickCheck and SmallCheck as complementary instead of in opposition to each other.

\subsection{A brief overview of QuickCheck}
To give a good understanding of how property-based testing works with \\ QuickCheck, we give a few examples of property tests on lists, which are directly expressed as functions in Haskell. Such a function has to return a \texttt{Bool} which indicates \texttt{True} in case the test succeeds or \texttt{False} if not and can take input arguments which data is automatically generated by QuickCheck.

\begin{HaskellCode}
-- append operator (++) is associative
append_associative :: [Int] -> [Int] -> [Int] -> Bool
append_associative xs ys zs = (xs ++ ys) ++ zs == xs ++ (ys ++ zs)

-- The reverse of a reversed list is the original list
reverse_reverse :: [Int] -> Bool
reverse_reverse xs = reverse (reverse xs) == xs

-- reverse is distributive over append (++)
-- This test fails for explanatory reasons, for a correct 
-- property xs and ys need to be swapped on the right-hand side!
reverse_distributive :: [Int] -> [Int] -> Bool
reverse_distributive xs ys = reverse (xs ++ ys) == reverse xs ++ reverse ys

-- running the tests
main :: IO ()
main = do
  quickCheck append_associative
  quickCheck reverse_reverse
  quickCheck reverse_distributive
\end{HaskellCode}

When we run the tests using \textit{main}, we get the following output:

\begin{verbatim}
+++ OK, passed 100 tests.
+++ OK, passed 100 tests.
*** Failed! Falsifiable (after 5 tests and 6 shrinks):    
[0]
[1]
\end{verbatim}

We see that QuickCheck generates 100 test cases for each property test and it does this by generating random data for the input arguments. We have not specified any data for our input arguments because QuickCheck is able to provide a suitable data generator through type inference. For lists and all the existing Haskell types there exist custom data generators already. We have to use a monomorphic list, in our case \texttt{Int}, and cannot use polymorphic lists because QuickCheck would not know how to generate data for a polymorphic type. Still, by appealing to genericity and polymorphism, we get the guarantee that the test case is the same for all types of a lists.

QuickCheck generates 100 test cases by default and requires all of them to pass. If there is a test case which fails, the overall property test fails and QuickCheck shrinks the input to a minimal size, which still fails and reports it as a counter example. This is the case in the last property test \texttt{reverse\_distributive} which is wrong as \textit{xs} and \textit{ys} need to be swapped on the right-hand side. In this run, QuickCheck found a counter example to the property after 5 tests and applied 6 shrinks to find the minimal failing example of \texttt{xs = [0]} and \texttt{ys = [1]}. If we swap \texttt{xs} and \texttt{ys}, the property test passes 100 test cases just like the other two did. It is possible to configure QuickCheck to generate more or less random test cases, which can be used to increase the coverage if the sampling space is quite large - this will become useful later.

\subsubsection{Generators}
QuickCheck comes with a lot of data generators for existing types like \texttt{String, Int, Double, []}, but in case one wants to randomize custom data types one has to write custom data generators. There are two ways to do this. Either fix them at compile time by writing an \texttt{Arbitrary} instance or write a run-time generator running in the \texttt{Gen} Monad. The advantage of having an \texttt{Arbitrary} instance is that the custom data type can then be used as random argument to a function as in the examples above.

Lets implement a custom data generator for the \texttt{SIRState} for both cases. We start with the run-time option, running in the \texttt{Gen} Monad:

\begin{HaskellCode}
genSIRState :: Gen SIRState
genSIRState = elements [Susceptible, Infected, Recovered]
\end{HaskellCode}

This implementation makes use of the \texttt{elements :: [a] $\rightarrow$ Gen a} functions, which picks a random element from a non-empty list with uniform probability. If a skewed distribution is needed, one can use the \texttt{frequency :: [(Int, Gen a)] $\rightarrow$ Gen a} function, where a frequency can be specified for each element. For example generating on average 80\% \texttt{Susceptible}, 15\% \texttt{Infected} and 5\% \texttt{Recovered} can be achieved using this function:

\begin{HaskellCode}
genSIRState :: Gen SIRState
genSIRState = frequency [(80, Susceptible), (15, Infected), (5, Recovered)]
\end{HaskellCode}

Implementing an \texttt{Arbitrary} instance is straightforward, one only needs to implement the \texttt{arbitrary :: Gen a} method:

\begin{HaskellCode}
instance Arbitrary SIRState where
  arbitrary = genSIRState
\end{HaskellCode}

When we have a random \texttt{Double} as input to a function but want to restrict its random range to (0,1) because it reflects a probability, we can do this easily with \texttt{newtype} and implementing an \texttt{Arbitrary} instance. The same can be done for limiting the simulation duration to a lower range than the full \texttt{Double} range.

\begin{HaskellCode}
newtype Probability = P Double
newtype TimeRange   = T Double

instance Arbitrary Probability where
  arbitrary = P <$> choose (0, 1)
  
instance Arbitrary TimeRange where
  arbitrary = T <$> choose (0, 50)
\end{HaskellCode}

The simulations we run all rely on a random-number generator, thus we need a randomly initialised random-number generator each time we run a simulation. This can be easily achieved by drawing a seed from the full \texttt{Int} range and creating an \texttt{StdGen} from it:

\begin{HaskellCode}
genStdGen :: Gen StdGen
-- min/maxBound are defined in the Haskell Prelude and
-- define the smallest and largest value of a Bounded type 
genStdGen = mkStdGen <$> choose (minBound, maxBound)

instance Arbitrary StdGen where
  arbitrary = genStdGen
\end{HaskellCode}
%$

This generator then can be used to write another custom data generator which generates simulation runs. Here we give an example for the time-driven SIR:

\begin{HaskellCode}
genTimeSIR :: [SIRState]  -- ^ Population
           -> Double      -- ^ Contact rate (beta)
           -> Double      -- ^ Infectivity (gamma)
           -> Double      -- ^ Illness duration (delta)
           -> Double      -- ^ Time Delta
           -> Double      -- ^ Time Limit
           -> Gen [(Double, (Int, Int, Int))]
genTimeSIR as beta gamma delta dt tMax 
  = runTimeSIR as beta gamma delta dt tMax <$> genStdGen
\end{HaskellCode}
%$

\subsubsection{Distributions}
As already mentioned, QuickCheck provides functions to measure the coverage of test cases. This can be done using the 
\texttt{label :: Testable prop $\Rightarrow$ String $\rightarrow$ prop $\rightarrow$ Property} function. It takes a \texttt{String} as first argument and a testable property and constructs a \texttt{Property}. QuickCheck collects all generated labels, counts their occurrences and reports their distribution. For example it could be used to get a rough idea about the length of the random lists created in the \texttt{reverse\_reverse} property shown above:

\begin{HaskellCode}
reverse_reverse_label :: [Int] -> Property
reverse_reverse_label xs  
  = label ("length of random-list is " ++ show (length xs)) 
          (reverse (reverse xs) == xs)
\end{HaskellCode}
%$
When running the test, we see the following output:

\begin{verbatim}
+++ OK, passed 100 tests:
 5% length of random-list is 27
 5% length of random-list is 0
 4% length of random-list is 19
 ...
\end{verbatim}

\subsubsection{Coverage}
The most powerful functions to work with test-case distributions though are \texttt{cover} and \texttt{checkCoverage}. The function \texttt{cover :: Testable prop $\Rightarrow$ Double $\rightarrow$ Bool $\rightarrow$ String $\rightarrow$ prop $\rightarrow$ Property} allows to explicitly specify that a given percentage of successful test cases belong to a given class. The first argument is the expected percentage; the second argument is a \texttt{Bool} indicating whether the current test case belongs to the class or not; the third argument is a label for the coverage; the fourth argument is the property which needs to hold for the test case to succeed. 

Lets look at an example where we use \texttt{cover} to express that we expect 15\% of all test cases to have a random list with at least 50 elements.

\begin{HaskellCode}
reverse_reverse_cover :: [Int] -> Property
reverse_reverse_cover xs  
  = cover 15 (length xs >= 50) "Length of random list at least 50"
             (reverse (reverse xs) == xs)
\end{HaskellCode}

When repeatedly running the test, we see the following output:

\begin{verbatim}
+++ OK, passed 100 tests (10% length of random list at least 50).
Only 10% Length of random-list at least 50, but expected 15%.
+++ OK, passed 100 tests (21% length of random list at least 50).
\end{verbatim}

As can be seen, QuickCheck runs the default 100 test cases and prints a warning if the expected coverage is not reached. This is a useful feature but it is up to us to decide whether 100 test cases are suitable and whether we can really claim that the given coverage will be reached or not. Fortunately, QuickCheck provides the powerful function \texttt{checkCoverage :: Testable prop $\Rightarrow$ prop $\rightarrow$ Property} which does this for us. When \texttt{checkCoverage} is used, QuickCheck will run an increasing number of test cases until it can decide whether the percentage in \texttt{cover} was reached or cannot be reached at all. The way QuickCheck does it, is by using sequential statistical hypothesis testing \cite{wald_sequential_1992}, thus if QuickCheck comes to the conclusion that the given percentage can or cannot be reached, it is based on a robust statistical test giving strong confidence in the result.

When we run the example from above but now with \texttt{checkCoverage} we get the following output:

\begin{verbatim}
+++ OK, passed 12800 tests 
    (15.445% length of random-list at least 50).
\end{verbatim}

We see that after QuickCheck has run 12,800 tests it came to the statistically robust conclusion that indeed at least 15\% of the test cases have a random list with at least 50 elements. 

\subsubsection{Emulating failure}
As already mentioned, \textit{all} test cases have to pass for the whole property test to succeed. If just a single test case fails, the whole property test fails. This requirement is sometimes too strong, especially when we are dealing with stochastic systems like ABS.

The function \texttt{cover} can be used to emulate failure of test cases and get a measure of failure. Instead of computing the \texttt{True/False} property in the last \texttt{prop} argument, we set the last argument always to \texttt{True} and compute the \texttt{True/False} property in the second \texttt{Bool} argument, indicating whether the test case belongs to the class of passed tests or not. This has the effect that \textit{all} test cases are successful but that we get a distribution of failed and successful ones. In combination with \texttt{checkCoverage}, this is a particularly powerful pattern for testing ABS, which allows us to test hypotheses and statistical tests on distributions as will be shown in the following chapters.

\input{./tex/property/abstesting.tex}

\section{Explanatory Model Testing: SIR}
\paragraph{Finding optimal $\Delta t$}
The selection of the right $\Delta t$ can be quite difficult in FRP because we have to make assumptions about the system a priori. One could just play it safe with a very conservative, small $\Delta t < 0.1$ but the smaller $\Delta t$, the lower the performance as it multiplies the number of steps to calculate. Obviously one wants to select the \textit{optimal} $\Delta t$, which in the case of ABS is the largest possible $\Delta t$ for which we still get the correct simulation dynamics.
To find out the \textit{optimal} $\Delta t$ one can make direct use of the black-box tests: start with a large $\Delta t = 1.0$ and reduce it by half every time the tests fail until no more tests fail - if for $\Delta t = 1.0$ tests already pass, increasing it may be an option. It is important to note that although isolated agent behaviour tests might result in larger $\Delta t$, in the end when they are run in the aggregate system, one needs to sample the whole system with the smallest $\Delta t$ found amongst all tests. Another option would be to apply super-sampling to just the parts which need a very small $\Delta t$ but this is out of scope of this paper.

\paragraph{Agents as signals}
Agents \textit{might} behave as signals in FRP which means that their behaviour is completely determined by the passing of time: they only change when time changes thus if they are a signal they should stay constant if time stays constant. This means that they should not change in case one is sampling the system with $\Delta t = 0$. Of course to prove whether this will \textit{always} be the case is strictly speaking impossible with a black-box verification but we can gain a good level of confidence with them also because we are staying pure. It is only through white-box verification that we can really guarantee and prove this property.

\paragraph{Black-Box Verification}
The interface of the agent behaviours are defined below. When running the SF with a given $\Delta t$ one has to feed in the state of all the other agents as input and the agent outputs its state it is after this $\Delta t$.

\begin{HaskellCode}
data SIRState 
  = Susceptible 
  | Infected 
  | Recovered
  
type SIRAgent = SF [SIRState] SIRState

susceptibleAgent :: RandomGen g => g -> SIRAgent
infectedAgent :: RandomGen g => g -> SIRAgent
recoveredAgent :: SIRAgent
\end{HaskellCode}

\paragraph{Finding optimal $\Delta t$}
Obviously the \textit{optimal} $\Delta t$ of the SIR model depends heavily on the model parameters: contact rate $\beta$ and illness duration $\delta$. We fixed them in our tests to be $\beta = 5$ and $\delta = 15$. By using the isolated behaviour tests we found an optimal $\Delta t = 0.125$ for the susceptible behaviour and $\Delta t = 0.25$ for the infected behaviour. %TODO: dynamics comparison?

\paragraph{Agents as signals}
Our SIR agents \textit{are} signals due to the underlying continuous nature of the analytical SIR model and to some extent we can guarantee this through black-box testing. For this we write tests for each individual behaviour as previously but instead of checking whether agents got infected or have recovered we assume that they stay constant: they will output always the same state when sampling the system with $\Delta t = 0$. The tests are conceptual the complementary tests of the previous behaviour tests so in conjunction with them we can assume to some extent that agents are signals. To prove it, we need to look into white-box verification as we cannot make guarantees about properties which should hold \textit{forever} in a computational setting.

\paragraph{Recovered Behaviour}
The implementation of the recovered behaviour is as follows:

\begin{HaskellCode}
recoveredAgent :: SIRAgent
recoveredAgent = arr (const Recovered)
\end{HaskellCode}

Just by looking at the type we can guarantee the following:
\begin{itemize}
	\item it is pure, no side-effects of any kind can occur
	\item no stochasticity possible because no RNG is fed in / we don't run in the random monad
\end{itemize}

The implementation is as concise as it can get and we can reason that it is indeed a correct implementation of the recovered specification: we lift the constant function which returns the Recovered state into an arrow. Per definition and by looking at the implementation, the constant function ignores its input and returns always the same value. This is exactly the behaviour which we need for the recovered agent. Thus we can reason that the recovered agent will return Recovered \textit{forever} which means our implementation is indeed correct.

Because we use multiple replications in combination with QuickCheck obviously results in longer test-runs (about 5 minutes on my machine) In our implementation we utilized the FRP paradigm. It seems that functional programming and FRP allow extremely easy testing of individual agent behaviour because FP and FRP compose extremely well which in turn means that there are no global dependencies as e.g. in OOP where we have to be very careful to clean up the system after each test - this is not an issue at all in our \textit{pure} approach to ABS.

\paragraph{Simulation Dynamics}
We won't go into the details of comparing the dynamics of an ABS to an analytical solution, that has been done already by \cite{macal_agent-based_2010}. What is important is to note that population-size matters: different population-size results in slightly different dynamics in SD => need same population size in ABS (probably...?). Note that it is utterly difficult to compare the dynamics of an ABS to the one of a SD approach as ABS dynamics are stochastic which explore a much wider spectrum of dynamics e.g. it could be the case, that the infected agent recovers without having infected any other agent, which would lead to an extreme mismatch to the SD approach but is absolutely a valid dynamic in the case of an ABS. The question is then rather if and how far those two are \textit{really} comparable as it seems that the ABS is a more powerful system which presents many more paths through the dynamics.
%TODO: i really want to solve this for the SIR approach
%	-> confidence intervals?
%	-> NMSE?
%	-> does it even make sense?

\paragraph{White-Box Verification}
%TODO: the implementation below has a SEVERE bug, all stochastic functions are correlated because they use the same RNG. this leads to different distributions of the dynamics, which can be shown using the test-code which generates the dynamics. The random monad version seems to perform much better where the mean is very close to the SD solution.

In the case of the SIR model we have the following invariants: 
\begin{itemize}
	\item A susceptible agent will \textit{never} make the transition to recovered.
	\item An infected agent will \textit{never} make the transition to susceptible.
	\item A recovered agent will \textit{forever} stay recovered.
\end{itemize}

All these invariants can be guaranteed when reasoning about the code. An additional help will be then coverage testing with which we can show that an infected agent never returns susceptible, and a susceptible agent never returned infected given all of their functionality was covered which has to imply that it can never occur!

%Lets start with looking at the recovered behaviour as it is the simplest one. We then continue with the infected behaviour and end with the susceptible behaviour as it is the most complex one.

We will only look at the recovered behaviour as it is the simplest one. We leave the susceptible and infected behaviours for further research / the final thesis because the conceptual idea becomes clear from looking at the recovered agent.


\chapter{Verifying an exploratory model: \\ Hypotheses in Sugarscape}
\label{ch:prop_exploratory}
In this chapter we look at how property-based testing can be made of use to verify the \textit{exploratory} Sugarscape model \cite{epstein_growing_1996} as already introduced in Chapter \ref{sec:sugarscape}. Whereas in the previous chapter on testing the explanatory SIR case-study we had an analytical solution, the fundamental difference in the exploratory Sugarscape model is that none such analytical solutions exist. This raises the question, which properties we can actually test in such a mode.

The answer lies in the very nature of exploratory models: they exist to explore and understand phenomena of the real world. Researchers come up with a model to explain the phenomena and then (hopefully) come up with a few questions and  \textit{hypotheses} about the emergent properties. The actual simulation is then used to test and refine the hypotheses. Indeed, descriptions, assumptions and hypotheses of varying formal degree are abound in the Sugarscape model. Examples are: \textit{the carrying capacity becomes stable after 100 steps; when agents trade with each other, after 1000 steps the standard deviation of trading prices is less than 0.05; when there are cultures, after 2700 steps either one culture dominates the other or both are equally present}. 

We show how to use property-testing to formalise and check such hypotheses. For this purpose we undertook a full \textit{verification} of our implementation \footnote{The code can be accessed freely from \url{https://github.com/thalerjonathan/phd/tree/master/public/towards/SugarScape/sequential}} from Chapter \ref{sec:sugarscape}. We validated it against the book \cite{epstein_growing_1996} and a NetLogo implementation \cite{weaver_replicating_2009} \footnote{\url{https://www2.le.ac.uk/departments/interdisciplinary-science/research/replicating-sugarscape}}. A longer report on the details of this validation process is attached as Appendix \ref{app:validating_sugarscape}.

The property we test for is whether \textit{the emergent property / hypothesis under test is stable under varying random-number seeds} or not. Put another way, we let QuickCheck generate random number streams and require that the tests all pass with arbitrary random number streams. We simply implement an \textit{Arbitrary} instance for \textit{StdGen}:

TODO: make stdgen arbitrary and use forAll and make the sugarscape simulation a generator

\begin{HaskellCode}
instance Arbitrary StdGen where
  -- arbitrary :: Gen StdGen
  arbitrary = do
  	-- draw a seed from the whole Int range
    seed <- choose (minBound, maxBound) 
    return (mkStdGen seed)
\end{HaskellCode}

%It is very likely that we run into the same problem as in the previous chapter on SIR testing: most of the tests might pass but some might fail for same reasons. By using the \textit{maxFailPercentage} argument, we can formally specify how strict we want the hypothesis to hold and still succeed in cases where there is a low percentage of failure. Unfortunately this revealed that this property doesn't hold for all emergent properties. The problem is that QuickCheck generates by default 100 test-cases for each property-test where all need to pass for the whole property-test to pass - this wasn't the case, where most of the 100 test-cases passed but unfortunately not all. Thus in this case a different approach is required: instead of requiring \textit{every} test to pass we require that \textit{most} tests pass, which can be achieved using a t-test with a confidence interval of e.g. 95\%. This means we won't use QuickCheck anymore and resort to a normal unit-test where we run the simulation 100 times with different random number streams each time and then performing a t-test with a 95\% confidence interval. Note that we are now technically speaking of a unit-test but conceptually it is still a property-test.

We have implemented property-tests for the following hypotheses

\begin{enumerate}
	\item Disease Dynamics - Depending on the number of initial diseases the population is either able to rid itself of all diseases or not.
	
	\item Cultural Dynamics - When having two cultures, red and green, after a given number of steps, either the red or the blue culture dominates or both are equally strong.
	
	\item Trading Dynamics - When trading is enable, the trading prices should stabilise with the standard deviation of the prices having dropped below 0.05 after 1,000 ticks.
	
	\item Terracing -
		
	\item Carrying Capacity - When agents don't mate nor can die from age (chapter II), due to the environment, there is a maximum carrying capacity of agents the environment can sustain. The capacity should be reached after 100 ticks and should be stable from then on.
			
	\item Inheritance Gini - TODO but maybe ignore it because not very interesting

	\item Wealth Distribution - TODO but maybe ignore it because not very interesting
\end{enumerate}

\section{Disease Dynamics}
We were able to exactly replicate the behaviour of Animation V-1 and Animation V-2: in the first case the population rids itself of all diseases (maximum 10) which happens pretty quickly, in less than 100 ticks. In the second case the population fails to do so because of the much larger number of diseases (25) in circulation. We used the same parameters as in the book. 
The authors of \cite{weaver_replicating_2009} could only replicate the first animation exactly and the second was only deemed "good". Their implementation differs slightly from ours: In their case a disease can be passed to an agent who is immune to it - this is not possible in ours. In their case if an agent has already the disease, the transmitting agent selects a new disease, the other agent has not yet - this is not the case in our implementation and we think this is unreasonable to follow: it would require too much information and is also unrealistic.

We give the code for the property-test where the population rids itself from all diseases within 100 steps.

\begin{HaskellCode}
prop_disease_dynamics_allrecover :: StdGen -> Bool
prop_disease_dynamics_allrecover g = infected == 0 -- all agents must be free of diseases after 100 ticks
  where
  	-- run for 100 steps
    ticks  = 100 
    -- use the AnimationV-1 parameter configuration
    params = mkParamsAnimationV_1 

	-- initialise the simulation
    (simState, _, _) = initSimulationRng g params
    -- run the simulation until 100 ticks and get the last output
    (_, _, _, aos) = simulateUntilLast ticks simState  
    -- count the number of infected
    infected = length $ filter (==False) $ map (null . sugObsDiseases . snd) aos
\end{HaskellCode}

The implementations of the property-tests for the other hypotheses follow a similar pattern and won't be repeated subsequently.

\section{Cultural Dynamics}
We could replicate the cultural dynamics of AnimationIII-6 / Figure III-8 (TODO page): after 2,700 steps either one culture (red / blue) dominates both hills or each hill is dominated by a the other culture. We wrote a test for it in which we run the simulation for 2,700 steps and then check if either culture dominates with a ratio of 95\% or if they are equal dominant with 45\%. Because always a few agents stay stationary on sugarlevel 1 (they have a metabolism of 1 and cant see far enough to move towards the hills, thus stay always on same spot because no improvement and grow back to 1 after 1 step), there are a few agents which never participate in the cultural process and thus no complete convergence can happen. This is accordance with \cite{weaver_replicating_2009}.

Running one test takes about 3 minutes, which makes it quite hard to run 100. Therefore we generally only run 10 in each case

\section{Trading Dynamics}

We still managed to exactly replicate the trading-dynamics as shown in the book in Figure IV-3, Figure IV-4 and Figure IV-5. The book is also pretty specific on the dynamics of the trading-prices standard-deviation: on page 109 the authors specify that at t=1000 the standard deviation will have always fallen below 0.05 (Figure IV-5), thus we implemented a property-test which tests for exactly that property.

Running a single test takes about 26 seconds.

\section{Terracing}
Our implementation reproduces the terracing phenomenon as described on page TODO in Animation and as can be seen in the NetLogo implementation as well. We implemented a property-test in which we measure the closeness of agents to the ridge: counting the number of same-level sugars cells around them and if there is at least one lower then they are at the edge. If a certain percentage is at the edge then we accept terracing. The question is just how much, which we estimated from tests and resulted in 45\%. Also, in the terracing animation the agents actually never move which is because sugar immediately grows back thus there is no incentive for an agent to actually move after it has moved to the nearest largest cite in can see. Therefore we test that the coordinates of the agents after 50 steps are the same for the remaining steps.

TODO: implement 
TODO: if we talk about averages then do t-test over multiple

\section{Carrying Capacity}
For regression-tests we implemented a property-test which tests that the carrying capacity of 100 simulation runs lies within a 95\% confidence interval of a 210 mean. These values are quite reasonable to assume, when looking at the NetLogo implementation - again we deem the reported Carrying Capacity of 224 in the Book to be an outlier / part of other details we don't know. After 100 ticks dropped and then stable from then on - check till 400 ticks.

TODO: implement

TODO: if we talk about averages then do t-test over multiple

\section{Discussion}
These tests are less of the property-based testing character as QuickCheck has intended it but we argue that they are still property-tests: they formalise hypotheses about the model and test them in the simulation. Due to ABS' stochastic nature, hypotheses have to hold for varying runs, single observations are not statistically significant. The use of QuickCheck is the instantiation of random-number generators, being the test-driver and reporting the results. 

% note: we never use shrinks in those tests because shrinking has no meaning in these tests. It will be more important in agent testing

\section{Case-Study II: Sugarscape}
TODO: 
we can implement everything except synchronous direct agent-interactions atm: if agent-interaction is one-way e.g. paying back a loan then this is no problem. thus the following parts of the Sugarscape are not possible with our current STM approach: mating, trading and lending  because all 3 require direct agent-to-agent interaction over multiple steps. We leave the problem of developing such an algorithm / implementation for further research.

\section{Discussion}

\subsection{Other Models}
TODO: mention that we have also implemented other models, which also work without time-semantics (all agents make a move at discrete time-steps and do not really rely on some notion of time). 

\subsection{Time-Semantics}
The main reason for building our pure functional ABMS approach on top of Yampa was to leverage the powerful time-semantics of Yampa which allows us to implement important concepts of ABMS:

state-chart: agents are at all time of their life-cycle in one state and can switch between multiple states using transitions 
timed transitions: transition to another state/behaviour happens at a discrete time
rate transitions: transition happens with a given rate
message transition: transition upon receiving a given message 

\subsection{Agents as Signals}
Due to the underlying nature and motivation of Functional Reactive Programming (und im speziellen) Yampa, Agents can be seen as Signals which is generated and consumed by a Signal-Function which is the behaviour of an Agent. If an Agent does not change the OUTPUT-signal is constant, if the agent changes e.g. by sending a message, changing its state,... the OUTPUT signal changes. A dead agent has no signal at all.

\subsection{Time-Sampling}
sampling rate depends on the transition times \& rates of the model. when e.g. the contact rate is 5 then the sampling dt should be below 0.2

\subsection{System Dynamics}
can emulate system dynamics due to the parallel update-strategy and continuous time-flow semantics

\subsection{Discrete Event Simulation}
DES in FrABMS? how easily can we implement server/queue systems? do they also look like a specification? potential problem: ordering of messages is not guaranteed by now

\subsection{Advantages}
advantages:
	- no side-effects within agents leads to much safer code
	- edsl for time-semantics
	- declarative style: agent-implementation looks like a model-specification
	- reasoning and verification
	- sequential and parallel
	- powerful time-semantics
	- arrowized programming is optional and only required when utilizing yampas time-semantics. if the model does not rely on time-semantics, it can use monadic-programming by building on the existing monadic functions in the EDSL which allow to run in the State-Monad which simplifies things very much
	- when to use yampas arrowized programing: time-semantics, simple state-chart agents 
	- when not using yampas facilities: in all the other cases e.g. SugarScape is such a case as it proceeds in unit time-steps and all agents act in every time-step
	- can implement System Dynamics building on Yampas facilities with total ease	
	- get replications for free without having to worry about side-effects and can even run them in parallel without headaches
	- cant mess around with time because delta-time is hidden from you (intentional design-decision by Yampa). this would be only very difficult and cumbersome to achieve in an object-oriented approach. TODO: experiment with it in Java - how could we actually implement this? I think it is impossible: may only achieve this through complicated application of patterns and inheritance but then has the problem of how to update the dt and more important how to deal with functions like integral which accumulates a value through closures and continuations. We could do this in OO by having a general base-class e.g. ContinuousTime which provides functions like updateDt and integrate, but we could only accumulate a single integral value.
	- reproducibility statically guaranteed
	- cannot mess around with dt
	- code == specification
	- rule out serious class of bugs
	- different time-sampling leads to different results e.g. in wildfire \& SIR but not in Prisoners Dilemma. why? probabilistic time-sampling?
	- reasoning about equivalence between SD and ABS implementation in the same framework
	- recursive implementations
	
	- we can statically guarantee the reproducibility of the simulation because: no side effects possible within the agents which would result in differences between same runs (e.g. file access, networking, threading), also timedeltas are fixed and do not depend on rendering performance or userinput	
	
\subsection{Disadvantages}
disadvantages:
	- performance is low
	- reasoning about performance is very difficult
	- very steep learning curve for non-functional programmers
	- learning a new EDSL
	- think ABMS different: when to use async messages, when to use sync conversations


[ ] important: increasing sampling freqzency and increasing number of steps so that the same number of simulation steps are executed should lead to same results. but it doesnt. why?
[ ] hypothesis: if time-semantics are involved then event ordering becomes relevant for emergent patterns. there are no tine semantics in heroes and cowards but in the prisoners dilemma
[ ] can we implement different types of agents interacting with each other in the same simulation ? with different behaviour funcs, digferent state? yes, also not possible in NetLogo to my knowledge. but they must have the same messages, emvironment 

[ ] Hypothesis: we can combine with FrABS agent-based simulation and system dynamics (this has been proved by example!)