\chapter*{Discussion}
TODO: wrap-up, comparison of time- and event-driven implementations.

we can see quickcheck not only as testing but also as verification and validation allowing to test hypotheses and properties during the development process, also has connections to Monte-Carlo simulation

further research: inclusion of Environment: all  properties should still hold under different Environments which can be generated randomly as Well e.g. random Networks.

further research: does rng correlation as discussed in time-driven chapter implementation show up with property-tests?

prop test with cover but without checkcoverage helps quick overview

We found property-based testing particularly well suited for ABS firstly due to ABS stochastic nature and second because we can formulate specifications, meaning we describe \textit{what} to test instead of \textit{how} to test. Also the deductive nature of falsification in property-based testing suits very well the constructive and often exploratory nature of ABS. 

Although property-based testing has its origins in Haskell, similar libraries have been developed for other languages e.g. Java, Pyhton, C++ as well and we hope that our research has sparked an interest in applying property-based testing to the established object-oriented languages in ABS as well.
