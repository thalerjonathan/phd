\section*{Publications}
Throughout the course of the Ph.D. six (6) papers with me as main author and contributor were written and submitted. Three (3) were accepted, two (2) remain work in progress and one (1) is pending:

\begin{enumerate}
	\item \textit{The Art Of Iterating - Update Strategies in Agent-Based Simulation} \cite{thaler_art_2017, payne_social_2019}, submitted and accepted at the Social Simulation Conference 2017. This paper derives four different update strategies and their properties possible in time- and event-driven ABS and discusses them from a programming paradigm agnostic point of view. It is the first paper which makes the very basics of update semantics clear on a conceptual level and is necessary to understand the options one has when implementing time- and event-driven ABS purely functional.
	
	\item \textit{Pure Functional Epidemics} \cite{thaler_pure_2018}, submitted and accepted at the IFL Conference 2018. Using an agent-based SIR model, this paper establishes in technical detail \textit{how} to implement time-driven ABS in Haskell using non-monadic Functional Reactive Programming with Yampa and monadic Functional Reactive Programming with Dunai. It outlines benefits and drawbacks and also touches on important points which were out of scope and lack of space in this paper but which will be addressed in more detail in this thesis.
	
	\item \textit{A Tale Of Lock-Free Agents} \cite{thaler_tale_2018}, submitted to the TOMACS Journal. This paper is the first to discuss the use of Software Transactional Memory for implementing concurrent ABS both on a conceptual and on a technical level. It presents two case studies, with the agent-based SIR model as the first and the Sugarscape model being the second one. In both case studies it compares performance of Software Transactional Memory and lock-based implementations in Haskell and object-oriented implementations of established languages. Although Software Transactional Memory is now not unique to Haskell any more, this paper shows why Haskell is particularly well suited for the use of Software Transactional Memory and is the only language which can overcome the central problem of how to prevent persistent side effects in retry semantics.

	\item \textit{The Agents' New Cloths? Towards Pure Functional Agent-Based Simulation} \cite{thaler_agents_2019}, submitted to the Summer Simulation Conference 2019. This paper summarizes the main benefits of using pure functional programming with Haskell to implement ABS and discusses on a conceptual level how to implement it and also what potential drawbacks are and where the use of a functional approach is not encouraged. It is written as a conceptual review paper, which tries to 'sell' pure functional programming to the agent-based community without too much technical detail and parlance where it refers to the important technical literature from where an interested reader can start.
	
	\item \textit{Show Me Your Properties! The Potential Of Property-Based Testing In Agent-Based Simulation} \cite{thaler_show_2019}, submitted and accepted at the Summer Simulation Conference 2019. This paper introduces property-based testing on a conceptual level to ABS using the agent-based SIR model and the Sugarscape model as two case studies. It shows how to encode specifications of explanatory models into properties and test individual agent behaviour and emergent properties of exploratory models.
	
	\item \textit{Specification Testing of Agent-Based Simulation using Property-Based Testing.} \cite{thaler_specification_2019}, submitted to the JAAMAS Journal in August 2019, currently under review.
This paper introduces property-based testing on a technical level to ABS using the agent-based SIR model model as case study. It builds on the previous, conceptual paper and discusses the topic in much more technical detail and also shows the use of statistically robust hypothesis testing with property-based testing.
\end{enumerate}