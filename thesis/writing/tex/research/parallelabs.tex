\chapter{Parallel ABS}
\label{ch:parallel_abs}
In the introduction in Chapter \ref{ch:motivation}, this thesis hypothesised that functional programming should allow to easily apply parallel computation to ABS. The subsequent two chapters tests this hypothesis by performing a deeper investigation of the potential of parallel programming offered by pure functional programming to apply to ABS. An additional motivation for this undertaking are the notorious performance problems of our time- and event-driven implementations, and the work in these chapters can be seen as an attempt to at least mitigate the notorious performance problems of functional programming.

\medskip

It is reasonable to say that pure functional programming as in Haskell is well known and accepted as a remedy against the difficulties and problems of parallel computation \cite{hudak_history_2007}. The reason for it is that immutable data and explicit control of side effects removes a large class of bugs due to data conflicts and data races. A fundamental benefit and strength of Haskell is that it clearly distinguishes between parallelism and concurrency \textit{in its types} \cite{jones_tackling_2002}. It is very important for us to do so as well:

\begin{itemize}
	\item \textbf{Parallelism} - in parallelism, code runs in parallel solely for the purpose of doing more work within the same time, without interfering with other code through shared data (references, mutexes, semaphores,...). An example is the function \texttt{map :: (a $\rightarrow$ b) $\rightarrow$ [a] $\rightarrow$ [b]}, which maps each element of type \texttt{a} to \texttt{b} using the function \texttt{(a $\rightarrow$ b)}. It is a pure function and thus no sharing of data either through some monadic context or through the function \texttt{(a $\rightarrow$ b)} is possible. This opens the potential to run it in parallel as each function evaluation \texttt{(a $\rightarrow$ b)} could theoretically be executed at the same time, if we had enough CPU cores. Whether it runs actually in parallel or not has no influence on the outcome as it is not subject to any non-deterministic influences. Thus we identify parallelism with pure and deterministic execution of data transformations in parallel (data parallelism).

	\item \textbf{Concurrency} - concurrency refers to the decomposability property of a program, algorithm, or problem into order-independent or partially-ordered components or units \cite{lamport_time_1978}. Those parts \textit{can} be run in parallel which as a consequence \textit{might} give rise to asynchronous, non-deterministic events \footnote{The functional \textit{concurrent} programming language Erlang \cite{armstrong_erlang_2010}, which uses the actor model for its concurrency model, was single threaded from its conception in 1986 until around 2008. This might sound surprising but underlines the fact that concurrency per se has nothing to do with parallel execution.}.

	An example are two threads, running in parallel, which share data through a reference. Depending on the scheduling and the code which is run in each thread, this gives rise to very different access patterns - the events - to the shared data, with the potential for race conditions and dirty reads. In concurrency, per definition ordering is important and the challenge of implementing parallel, concurrent programs, is to write the program in a way that despite of these non-deterministic events it is still a correctly working program. Thus, we identify concurrency with parallel, impure, non-deterministic execution of imperative-style and ordered monadic evaluation.
\end{itemize}

In the next two chapters we investigate the application of both parallelism and concurrency to our pure functional ABS approach. In general, we want to see if and how parallel and concurrent programming in Haskell is transferable to pure functional ABS and what the benefits are. In particular we are interested in speeding up the existing implementations by generally developing techniques that allow us to  \textit{run agents in parallel \footnote{We use the term \textit{parallel} to identify both \textit{parallelism} and \textit{concurrency} and we distinguish between them whenever necessary using their respective terms.}}. 

The focus here is primarily on the conceptual nature of how to apply parallelism and concurrency to pure functional ABS, thus we refrain from doing in-depth performance analysis up-front as it is beyond the scope of this work. Still, we are very well aware that mindlessly trying to apply parallel computation can actually result in loss of performance as a problem can only be sped up in so far as we can partition it and run those partitions in parallel. Further, parallel computation comes with an overhead and if the partitioning is too fine-grained, this overhead might eat up the speedup or make it even worse. Thus, in real-world problems, performance measurements have to come first, then one can investigate where and why the performance is lost. Only if this is understood properly one can decide whether parallelism or concurrency is applicable - or none at all because the problem is actually completely sequential. %As D. Knuth famously put it: \textit{"Premature optimisation is the root of all evil"}, thus, when we see adding parallel computation as one way of optimising a problem, we need hard facts instead of wild guesses.

Besides performance improvement, we are generally interested in the implications of the way Haskell deals with parallelism and concurrency in its types. In particular we ask about the ability of keeping deterministic guarantees about the reproducibility of our simulations. We hypothesise that parallelism will allow us to retain \textit{all} static guarantees about reproducibility \textit{and} gives us a noticeable speedup. Further, we hypothesise, that in concurrency we might see a bigger speedup but sacrifice the very guarantee about reproducibility. However, we assume that by using Haskell's unique approach to Software Transactional Memory, we don't lose this guarantee completely, but it will get weakened by guaranteeing that the non-deterministic influence is through concurrency only \textit{and nothing else}.

\section{Parallelism in ABS}
The promise of parallelism in Haskell is compelling: speeding up the execution but retaining all static compile-time guarantees about determinism. In other words, using parallelism could give us a substantial performance improvement without sacrificing the static guarantees of reproducible outputs from repeated runs with initial conditions.

Generally, parallelism can be applied whenever the execution of code is order-independent, that is referential transparent, and has no implicit or explicit side-effects. Without going into too much technical detail, in this section we outline the parallelism techniques available in Haskell and briefly discuss how they can be used in ABS in general. We also discuss if and how parallelism can be added to our previously discussed use-cases of Chapters \ref{sec:timedriven_firststep}, \ref{sec:adding_env} and Sugarscape TODO and report the performance improvements where applicable.

\subsection{Parallelism in Haskell}
parallelism in haskell builds on laziness

We follow the book \cite{marlow_parallel_2013}, which can be seen as the main source for parallelism and concurrency in Haskell and refer to it for an in-depth discussions of parallel Haskell.

\paragraph{Evaluation parallelism}
The basis are the following functions: \textit{rpar :: a -> Eval a} and \textit{rseq :: a -> Eval a}, where Eval is a Monad which can be run with \textit{runEval :: Eval a -> a}. rpar runs the computation in parallel and immediately returns without waiting for the evaluation of the thunk - this will happen behind the sences. rseq runs the computation in parallel as well but waits for the evaluation to WFNF. Using this we can start evaluating multiple expressions in parallel with rpar and then wait for their result with rseq. Note that both cases evaluate their argument to weak head normal form (WHNF), thus if the argument is already in WHNF, then the computation does nothing. This becomes important when to understand how far we can to in evaluation of parallelism. TODO: need to say a bit about haskell as a lazy language. TODO: isnt this all the very basics of haskells parallelism together with lazy evaluation?

Put short, evaluation parallelism allows to build functions which run in parallel e.g. a parallel version of \textit{map}, which is called \textit{parMap}. This is achieved using the \textit{Eval Monad} which is run using a parallel evaluation strategy, arriving at a pure value - the evaluation of the \textit{Eval Monad} itself is pure and does not require the IO (this is exactly what we expect from parallelism: to be deterministic). Obviously this gives huge potential for speeding up programs because maps are omnipresent in a lot of functional code. Not only parmap! explain a little bit more in detail without going into too much technical stuff.


Very important: "In the previous two chapters, we looked at the Eval monad and Strategies, which work in conjunction with lazy evaluation to express parallelism. A Strategy consumes a lazy data structure and evaluates parts of it in parallel. This model has some advantages: it allows the decoupling of the algorithm from the parallelism, and it allows parallel evaluation strategies to be built compositionally. But Strategies and Eval are not always the most convenient or effective way to express parallelism. We might not want to build a lazy data structure, for example. Lazy evaluation brings the nice modularity properties that we get with Strategies, but on the flip side, lazy evaluation can make it tricky to understand and diagnose performance."
Its laziness which allows that.


%https://www.oreilly.com/library/view/parallel-and-concurrent/9781449335939/ch02.html
%https://www.oreilly.com/library/view/parallel-and-concurrent/9781449335939/ch03.html

\paragraph{Data-flow parallelism}
Par Monad: how does it work? can express data-flow networks where tasks are forked and then results are synchronised. all this happens deterministically by building on the same mechanics the Eval monad is using thus technically speaking they are equivalent. 

can use both par and eval monad but which is applicable? par seems to require strict data, eval works on lazy data-structure. can we use both inside an msf? what is the Advantage over simple rpar or rseq?\\

%https://www.oreilly.com/library/view/parallel-and-concurrent/9781449335939/ch04.html

\paragraph{Data-structure parallelism}
An environment could be organised and accessed through such a data-structure, which could potentially lead to big speed ups. Agents could locally read the data-structure data-parallel and the simulation kernel could feed the output of the agents data-parallel back into this structure.

%https://learning.oreilly.com/library/view/parallel-and-concurrent/9781449335939/ch05.html

general solution we opt for is  to run agents in parallel in our approaches. in other abs models we could apply data-structure parallelism and/or data-flow parallelism with huge Performance potential but thats always highly model dependent thus we dont go in depth here

\subsection{Use-Cases}

\subsubsection{Non-Monadic SIR}
Although \textit{parMap} can be applied in all cases where a map us used, we are particularly interested in running agents in parallel. With \textit{parMap} this should become possible in our non-monadic SIR implementation built on Yampa from Chapter \ref{sec:timedriven_firststep}. Even thought the Eval Monad is used under the hood and Yampa is non-monadic, it is still applicable because running the monad is pure, resulting in a pure result - \textit{parMap} is a pure function. TODO: how can we apply this?

Inspired by the work of \cite{perez_60_2014}, which shows the potential of speeding up real-world Haskell programs using Yampa We conducted a comparison of an implementation which makes use of evaluation parallelism to run agents in parallel.

OK, rephrase: compare performance of non-parallel implementation WITH threaded an -N option to non-parallel implementation without threaded and / or N1 to make sure that no performance improvement happens automatically by using threaded e.g. GCs or something else...
I observed the behaviour in the following code: https://github.com/thalerjonathan/phd/tree/master/public/purefunctionalepidemics/code/SIR_Yampa

I analysed a bit more using the threadscope tool. I ran the same program twice with different ghc-options:
1. -O2 -Wall -Werror -eventlog 
2. -O2 -Wall -Werror -eventlog -threaded -with-rtsopts=-N

When looking at the event logs with threadscope it becomes appartent, that parallel garbage collection is the cause of the CPU usage above 100%:
-  In the single-threaded case 0 sparks are created and everything runs indeed only on one core. There are two Garbage Collectors (Gen0 and Gen1) but nothing runs in parallel (Par collections are 0 for both).
- In the multi-threaded case also 0 sparks are created but now 8 cores are used: all 'running' activity happens on only 1 core as expected but garbage collection happens on all 8 cores: the diagrams and the number of Par collections clearly indicates that. The time spent on parallel GC work is 10.76% (0 is completely serial and 100% is completely parallel).

Now when we compare the timing between both runs we see the following: 
- single-threaded: 11.68s total, 7.35s mutator, 4.34s GC,
- multi-threaded: 10.70s total, 7.03s mutator, 3.68s GC

This adds up: the ~ 10\% of parallel GC work done in multi-threaded are also the ~ 10\% it is faster over the single-threaded one. Of course I only did a single run in each case but I think the analysis is still valid and the point was made: when running a Haskell program which does not use any parallel features, running it with the -threaded option can lead to an increase in performance due to parallel GC.

% https://www.reddit.com/r/haskell/comments/2jbl78/from_60_frames_per_second_to_500_in_haskell/\\

The idea:
Using compositional parallelism we can add evaluation parallelism for agent execution, without needing to re-implement dpSwitch. Also re-implementing switch functions would not get us very far because of WHNF evaluation it is the wrong end to start parallel evaluation: probably only the arguments would be evaluated but not the agent behaviour. The solution is to add evaluation parallelism in the agent-output collection phase: where the recursive switch into the step Simulation function happens. There we use a parMap to evaulate the outputs of all agents in parallel simply using a parMap with id, which due to compositional parallelism because of lazy evaluation, should then run the whole agent when it is evaluated in parallel because the output is forced in parallel evaluation.

Our use case:
Unfortunately in our non-monadic Yampa implementation we see a negligible speedup of less than 10\% between running it on 1 or 8 cores and this difference is probably due to garbage collection. When analysing the problem more in-depth it becomes clear that 50\% of the parallel evaluation sparks (todo explain) are duplications and get never evaluated, which is due to the thunk being already evaluated before thus no need to run it actually in parallel. Unfortunately this seems reasonable in this example: the way the agent-behaviour is implemented forces the values, including the output, due to lots of comparisons, which results basically in a strict behaviour with the output already evaluated for many agents. It seems that it depends on the current state the agent is in otherwise we could not explain why some sparks are duplications and others not. Further it seems, that although work happens in parallel, the overhead eats up the benefit and thus we arrive at roughly the same performance of the non-parallel version. This might be completely different for much more computational intensive agent behaviour with a more complex agent-output data-structure - but we leave this for further research.

\subsubsection{Monadic SIR}
Unfortunately \textit{parMap} is not applicable to the monadic SIR version of Chapter \ref{sec:adding_env} because of the use of mapM, which cannot be replaced due to its inherent sequential nature: mapM runs monadic actions which have side-effects thus ordering matters. Even if the implementation in that chapter behaves as if the agents are run in parallel, technically they are run sequentially because of the need for the Random Monad effect. This leaves us basically without any options of parallelism for the monadic SIR model, we will come back to this use-case in the concurrency section, where we will show that by using concurrency it is possible to achieve a substantial speed up of orders of magnitude.

TODO: the marlow book says: don't do repeated calls to runPar, so although the agent can in fact do that it should avoid it and if there is heavy parallel work in each agent then one should consider running the agent in the par monad with a single runPar outside

NOTE: running the agents in parallel with par doesn't work because we use mapM and are thus monadic, which involves sequencing. so this is really out of the window here. Also we cannot put a Par in a transformer stack because the library doesn't support it, what actually makes sense. But we can do the following: we can run an agents MSF only within the Par monad which gives agents the ability to spawn data-flow parallel computations - random-number streams are handled like in the non-monadic version. Note that this is only possible with the MSFs of dunai and not the SF because the latter one adds already the a ReaderT DTime which makes it impossible already. 
What is actually possible would be to write a combined monad for Par and ReaderT because the latter one is a read-only value and could thus potentially run in parallel - we leave this for further research. There exists also a combination of the Par with the Rand monad, so if the time-driven approach is not needed then this could be used to give the agents the ability to both draw random numbers AND do deterministic data-parallel computations. The agents can then be run in parallel through the par monad.

TODO: try the same thing as in the non-monadic SIR: parMap rpar id evaluating the output. Hypothesis is that it hould not show any parallelism because of monadic code.

\subsubsection{Sugarscape}
The same case as in the monadic SIR: running the agents with evaluation or data-flow parallelism is not possible in a monadic context  \textit{in general}. We have shortly discussed how it could be achieved in specific circumstances where then agents are running in the Par monad only, but this is highly model specific and for the Sugarscape this approach does not work. 

There is though one tiny thing we could optimise.

use parmap for updating Pollution/regrow resources. still agents can't be run in parallel because of monadic effects, we show in the concurrent section how we can use concurrency to achieve a substantial speed up using STM.

compare Environment parallelism between sequential and concurrent sugarscape: should see alarger speedup in conc bcs the sequential percentage is larger there

unfortunately its a Map datastructure, so we cannot operate in parallel e.g. map. but we can compute pollution because it uses map

\subsection{Parallel Runs}
Often one needs to perform a large number of runs of the same simulation. The most prominent use-cases for this are:

\begin{itemize}
	\item Parameter Sweeps / Variations - To explore the parameter space and the dynamics under varying parameter configurations, the same simulation is run with varying parameters and the results recorded for statistical analysis.
	
	\item Stochastic replications - Due to ABS stochastic nature, running a simulation only once does not allow to generalise or predict overall behaviour - one might have just hit an (un)fortunate special case. To counter this problem, in ABS multiple replications of the  simulation are run with same initial model parameters but with different random-number streams. All the results are collected and analysed stochastically (averaged, median,...) from which then more general properties can be derived.
\end{itemize}

In each case thousands of runs of the same simulation with different model parameters and / or varying random-number streams are needed, requiring a considerable amount of computing power.

Parallelism is a remedy to this problem because in each of these cases individual runs do not interfere with each other and thus can be seen as isolated from each other, like referential, pure computations. Our approaches shown in the Part II make this very explicit: the top level functions can always be made pure computations because we are ruling out IO (so far) and thus even though Monads are employed in many cases, they are still pure. A benefit of our approach is that it is guaranteed at compile time, that individual runs do not interfere with each other and thus there is no danger that parallel runs influence each other. 

All this allows to implement parameter sweeps and stochastic replications both through evaluation and data-flow parallelism making another very compelling use-case - probably the most striking one - for the use of parallelism in ABS. We hypothesize that data-flow parallelism is better suited for this task because it makes parallelism more explicit as it is indeed a data-flow problem: we pass parameters to single replications which are run and return their results. To apply this we simply run the top level replication logic in the Par Monad where replications are run in parallel by forking tasks and results are handed back through IVars. If we want the convenience of having a monadic random-number generator within the Par monad as well, one can use the combined ParRand monad which provides both.

\subsection{Reflection}
In general we aimed at running agents in parallel using the various techniques. Because of the quite sequential nature of the agent behaviours themselves, there is much less potential for parallelism \textit{within} an agent, thus the obvious idea was to run them all in parallel because they are an obvious unit of partitioning, have considerable workload and can indeed be run in parallel under given circumstances.
Unfortunately it is not possible applying parallelism in case the agents run within a monadic context: we have side-effects which imposes ordering e.g. in the case of a

We see a direct consequence of this that types also reflect the semantics of our model: when our agents are pure they can be run indeed in parallel and independent from each other, if they are monadic, then this is not applicable to parallelism. In the next section, we show how to approach this problem and come up with a solution where we can run monadic agents in parallel. This is obviously only possible within a concurrent setting which means we have to sacrifice determinism in our solution. Still we reach considerable speed ups using Software Transactional Memory.

\section{Concurrent ABS}
In an ideal world, we would like to solve all our problems using parallelism but unfortunately, it can't be applied to all parallel problems and ABS is no exception. As soon as there are data-dependencies, like we have them in the Sugarscape model in the form of the read/write environment and synchronous agent-interactions, we cannot avoid concurrency.

Traditional approaches to concurrency follow a lock-based approach, where sections which access shared data are synchronised through synchronisation primitives like mutexes, semaphores, monitors,... The lock-based path is a well trodden one, with all problems and benefits well established. In this chapter we want to follow a different path and look into using Software Transactional Memory (STM) for implementing concurrent ABS, which promises to overcome the problems of lock-based approaches. Although STM exists in other languages as well, Haskell was one of the first to natively build it into its core, thus it is a natural choice to follow that direction when already investigating pure functional ABS.

Unfortunately, as soon as we employ concurrency, we lose all static guarantees about reproducibility and the use of STM is no exception. Still, STM has the unique benefit that it can guarantee the lack of persistent side-effects at compile time, allowing unproblematic retries of transactions, something of fundamental importance in STM as will be described below. This has also another \textit{very} compelling advantage of STM over unrestricted lock-based approaches: by using STM, we can reduce the side-effects allowed substantially and guarantee at compile time, that the differences between runs of same initial conditions will only stem from the fact that we run the simulation concurrently - \textit{and from nothing else}. All this makes the use of STM together with Haskell very compelling and to our best knowledge we are the very first to investigate the use of STM for implementing concurrent ABS in a systematic way.

The paper \cite{discolo_lock_2006} gives a good indication how difficult and complex constructing a correct concurrent program is and shows how much easier, concise and less error-prone an STM implementation is over traditional locking with mutexes and semaphores. More important, it shows that STM consistently outperforms the lock-based implementations. We follow this work and compare the performance of lock-based and STM implementations and hypothesise that the reduced complexity and increased performance will be directly applicable to ABS as well.

We present two case studies using the already introduced SIR (Chapter \ref{sec:sir_model}) and Sugarscape (Chapter \ref{sec:sugarscape}) models. We compare the performance of lock-based and STM implementations in each case where we investigate both the scaling performance under increasing number of CPUs and under increasing number of agents. We show that the STM implementations consistently outperform the lock-based ones and scale much better to increasing number of CPUs both on local machines and on Amazon Cloud Services.

\section{Software Transactional Memory}
Software Transactional Memory (STM) was introduced by \cite{shavit_software_1995} in 1995 as an alternative to lock-based synchronisation in concurrent programming which, in general, is notoriously difficult to get right. This is because reasoning about the interactions of multiple concurrently running threads and low level operational details of synchronisation primitives is \textit{very hard}. The main problems are:

\begin{itemize}
	\item Race conditions due to forgotten locks;
	\item Deadlocks resulting from inconsistent lock ordering;
	\item Corruption caused by uncaught exceptions;
	\item Lost wake-ups induced by omitted notifications.
\end{itemize}

Worse, concurrency does not compose. It is very difficult to write two functions (or methods in an object) acting on concurrent data which can be composed into a larger concurrent behaviour. The reason for it is that one has to know about internal details of locking, which breaks encapsulation and makes composition dependent on knowledge about their implementation. Therefore, it is impossible to compose two  functions e.g. where one withdraws some amount of money from an account and the other deposits this amount of money into a different account: one ends up with a temporary state where the money is in none of either accounts, creating an inconsistency - a potential source for errors because threads can be rescheduled at any time.

STM promises to solve all these problems for a low cost by executing actions \textit{atomically}, where modifications made in such an action are invisible to other threads and changes by other threads are invisible as well until actions are committed - STM actions are atomic and isolated. When an STM action exits, either one of two outcomes happen: if no other thread has modified the same data as the thread running the STM action, then the modifications performed by the action will be committed and become visible to the other threads. If other threads have modified the data then the modifications will be discarded, the action block rolled-back and automatically restarted.

STM in Haskell is implemented using optimistic synchronisation, which means that instead of locking access to shared data, each thread keeps a transaction log for each read and write to shared data it makes. When the transaction exits, the thread checks whether it has a consistent view to the shared data or not: whether other threads have written to memory it has read. % This might look like a serious overhead but the implementations are very mature by now, being very performant and the benefits outweigh its costs by far.

In the paper \cite{heindl_modeling_2009} the authors use a model of STM to simulate optimistic and pessimistic STM behaviour under various scenarios using the AnyLogic simulation package. They conclude that optimistic STM may lead to 25\% less retries of transactions. The authors of \cite{perfumo_limits_2008} analyse several Haskell STM programs with respect to their transactional behaviour. They identified the roll-back rate as one of the key metric which determines the scalability of an application. Although STM might promise better performance, they also warn of the overhead it introduces which could be quite substantial in particular for programs which do not perform much work inside transactions as their commit overhead appears to be high.

\subsection{STM in Haskell}
The work of \cite{harris_composable_2005, harris_transactional_2006} added STM to Haskell, which was one of the first programming languages to incorporate STM into its main core and added the ability to composable operations. There exist various implementations of STM in other languages as well (Python, Java, C\#, C/C++, etc) but we argue, that it is in Haskell with its type-system and the way how side-effects are treated where it truly shines.

In the Haskell implementation, STM actions run within the \textit{STM} context. This restricts the operations to only STM primitives as shown below, which allows to enforce that STM actions are always repeatable without persistent side-effects because such persistent side-effects (e.g. writing to a file, launching a missile) are not possible in an \textit{STM} context. This is also the fundamental difference to  \textit{IO}, where all bets are off because \textit{everything} is possible as there are basically no restrictions because \textit{IO} can run everything.

Thus the ability to \textit{restart} a block of actions without any visible effects is only possible due to the nature of Haskells type-system: by restricting the effects to STM only, ensures that no uncontrolled effects, which cannot be rolled-back, occur.

STM comes with a number of primitives to share transactional data. Amongst others the most important ones are:

\begin{itemize}
	\item \textit{TVar} - A transactional variable which can be read and written arbitrarily;
	\item \textit{TArray} - A transactional array where each cell is an individual shared data, allowing much finer-grained transactions instead of e.g. having the whole array in a \textit{TVar};
	\item \textit{TChan} - A transactional channel, representing an unbounded FIFO channel;
	\item \textit{TMVar} - A transactional \textit{synchronising} variable which is either empty of full. To read from an empty or write to a full \textit{TMVar} will cause the current thread to retry its transaction.
\end{itemize}

% NOTE: too technical
%To run an \textit{STM} action the function \textit{atomically :: STM a $\to$ IO a} is provided, which can be seen as the STM effect-runner as it performs a series of \textit{STM} actions atomically within an \textit{IO} context. It takes the STM action which returns a value of type \textit{a} and returns an \textit{IO} action which returns a value of type \textit{a}. This \textit{IO} action can only be executed within an \textit{IO} context.

\section{STM in ABS}
\label{sec:stm_abs}
In this section we give a short overview of how we apply STM in our ABS. In both case-studies we fundamentally follow a time-driven, parallel approach as introduced in Chapter \ref{sub:par_strategy}, where the simulation is advanced by a given $\Delta t$ and in each step all agents are executed. To employ parallelism, each agent runs within its own thread and agents are executed in lock-step, synchronising between each $\Delta t$, which is controlled by the main thread. See Figure \ref{fig:stm_abs_structure} for a visualisation of our concurrent, time-driven lock-step approach.

By running each agent in a thread will guarantee the execution in parallel even if the agent has a monadic context. This is forces us to evaluate each agents monadic context separately instead of running them all in a common context. Note that ultimately we are ending up in the \textit{IO} context because \textit{STM} can be only transacted from within an \textit{IO} context due to non-deterministic side-effects. This is no contradiction to our original claim: yes we are running in IO but not the agent behaviour itself, which is a fundamental difference.

An agent thread will block until the main-thread sends the next $\Delta t$ and runs the \textit{STM} action atomically with the given $\Delta t$. When the \textit{STM} action has been committed, the thread will send the output of the agent action to the main-thread to signal it has finished. The main thread awaits the results of all agents to collect them for output of the current step e.g. visualisation or writing to a file.

As will be described in subsequent sections, central to both case-studies is an environment which is shared between the agents using a \textit{TVar} or \textit{TArray} primitive through which the agents communicate concurrently with each other. To get the environment in each step for visualisation purposes, the main thread can access the \textit{TVar} and \textit{TArray} as well. 

\begin{figure}
	\centering
	\includegraphics[width=1.0\textwidth, angle=0]{./fig/concurrentabs/stm_abs.png}
	\caption{Diagram of the parallel time-driven lock-step approach.}
	\label{fig:stm_abs_structure}
\end{figure}

\subsection{Adding and running the STM Monad}
We briefly show how to add STM to agents and run them within their own threads. We use the SIR implementation as example - applying it to the Sugarscape implementation works exactly the same way and is left as a trivial exercise to the reader.

The first step is to simply add the \textit{STM} to the existing transformer stack as the \textit{innermost} monad. The reason why we make it the innermost is to guarantee that in case of a retry \textit{all} outer monadic effects are retried as well - if the STM would be placed on a higher stack level, the levels below would not be subject to a retry. For monads like the \textit{ReaderT} this would not matter because they are read-only but for a StateT this fact would matter a lot. Note that STM does not provide a transformer instance, so this is not an option anyway. If STM would provide a transformer then we could make \textit{IO} the innermost monad and do \textit{IO} within STM, which should be prevented under all circumstances because then rolling back a transaction cannot guarantee to undo the effects. To better understand the semantics of retries consider the following example:

\begin{HaskellCode}

\end{HaskellCode}



\begin{HaskellCode}
innerSTMAction :: RandomGen g => StateT SomeState (RandT g STM) SomeResult

let randAction = runStateT innerSTMAction initState
let stmAction  = runRandT randAction (mkStdGen 42)
let ioAction   = atomically stmAction
((someResult, someState), g) <- ioAction
\end{HaskellCode}

In this case the STM is the \textit{innermost} monad thus it will be run last. This means that all the outer monads are subject to re-computation due to retries.

\begin{HaskellCode}
outerSTMAction :: STMT (StateT Environment (Rand g)) SomeResult

let ioAction = runSTMT outerSTMAction
stateAction <- ioAction
let randAction = runStateT stateAction initState
let ((someResult, someState), g) = runRandT randAction (mkStdGen 42)
\end{HaskellCode}

In this case, the STM is the \textit{outermost} monad, thus it will be run first. This means that it will return a StateT computation which will be computed \textit{after} the STM has transacted. The computation construction is subject to the retries but the computation itself won't be repeated in case of retries.

TODO: add STM

\begin{HaskellCode}
agentThread :: RandomGen g 
            => Int
            -> SIRAgent g
            -> g
            -> MVar SIRState
            -> MVar DTime
            -> IO ()
agentThread 0 _ _ _ _ = return () -- all steps computed, terminate thread
agentThread n sf rng retVar dtVar = do
  -- wait for dt to compute current step
  dt <- takeMVar dtVar

  -- compute output of current step
  let sfReader = unMSF sf ()
      sfRand   = runReaderT sfReader dt
      sfSTM    = runRandT sfRand rng
  ((ret, sf'), rng') <- atomically sfSTM -- run the STM action atomically within IO

  -- post result to main thread
  putMVar retVar ret
  
  -- to next step
  agentThread (n - 1) sf' rng retVar dtVar
\end{HaskellCode}

\begin{HaskellCode}
simulationStep :: TVar SIREnv
               -> [MVar DTime]
               -> [MVar SIRState]
               -> DTime
               -> IO SIREnv
simulationStep env dtVars retVars dt = do
  -- tell all threads to continue with the corresponding DTime
  mapM_ (`putMVar` dt) dtVars
  -- wait for results but ignore them, SIREnv contains all states
  mapM_ takeMVar retVars
  -- return state of environment when step has finished
  readTVarIO env
\end{HaskellCode}