\chapter{Pure Functional Time-Driven ABS}
\label{ch:timedriven}

In the following, we derive a pure functional approach for an agent-based SIR model in which we pose solutions to the previously mentioned problems. We start out with a first approach in Yampa and show its limitations. Then we generalise it to a more powerful approach, which utilises Monadic Stream Functions, a generalisation of FRP. Finally we add a structured environment, making the example more interesting and showing the real strength of ABS over other simulation methodologies like System Dynamics and Discrete Event Simulation \footnote{The code of all steps can be accessed freely through the following URL: \url{https://github.com/thalerjonathan/phd/tree/master/public/purefunctionalepidemics/code}}.

\section{The SIR model}
The \textit{explanatory} SIR model is a very well studied and understood compartment model from epidemiology \cite{kermack_contribution_1927}, which allows to simulate the dynamics of an infectious disease like influenza, tuberculosis, chicken pox, rubella and measles spreading through a population.

In this model, people in a population of size $N$ can be in either one of three states \textit{Susceptible}, \textit{Infected} or \textit{Recovered} at a particular time, where it is assumed that initially there is at least one infected person in the population. People interact \textit{on average} with a given rate of $\beta$ other people per time-unit and become infected with a given probability $\gamma$ when interacting with an infected person. When infected, a person recovers \textit{on average} after $\delta$ time-units and is then immune to further infections. An interaction between infected persons does not lead to re-infection, thus these interactions are ignored in this model. This definition gives rise to three compartments with the transitions seen in Figure \ref{fig:sir_transitions}.

\begin{figure}
	\centering
	\includegraphics[width=.4\textwidth, angle=0]{./fig/timedriven/SIR_transitions.png}
	\caption{States and transitions in the SIR compartment model.}
	\label{fig:sir_transitions}
\end{figure}

This model was also formalized using System Dynamics (SD) \cite{porter_industrial_1962}. In SD one models a system through differential equations, allowing to conveniently express continuous systems, which change over time, solving them by numerically integrating over time, which gives then rise to the dynamics. The SIR model is modelled using the following equation, with the dynamics shown in Figure \ref{fig:sir_sd_dynamics} .

\begin{equation}
\begin{aligned}
\frac{\mathrm d S}{\mathrm d t} = -infectionRate \\
\frac{\mathrm d I}{\mathrm d t} = infectionRate - recoveryRate \\
\frac{\mathrm d R}{\mathrm d t} = recoveryRate 
\end{aligned}
\end{equation}

\begin{equation}
\begin{aligned}
infectionRate = \frac{I \beta S \gamma}{N} \\
recoveryRate = \frac{I}{\delta} 
\end{aligned}
\end{equation}

\begin{figure}
	\centering
	\includegraphics[width=0.5\textwidth, angle=0]{./fig/timedriven/SIR_SD_1000agents_150t_001dt.png}
	\caption{Dynamics of the SIR compartment model using the System Dynamics approach. Population Size $N$ = 1,000, contact rate $\beta =  \frac{1}{5}$, infection probability $\gamma = 0.05$, illness duration $\delta = 15$ with initially 1 infected agent. Simulation run for 150 time-steps. Generated using our pure functional SD approach (see Chapter \ref{sub:generalising_system_dynamics}).}
	\label{fig:sir_sd_dynamics}
\end{figure}

The approach of mapping the SIR model to an ABS is to discretize the population and model each person in the population as an individual agent. The transitions between the states are happening due to discrete events caused both by interactions amongst the agents and time-outs. The major advantage of ABS is that it allows to incorporate spatiality as shown in Section \ref{sec:adding_env} and simulate heterogenity of population e.g. different sex, age. This is not possible with other simulation methods e.g. SD or Discrete Event Simulation \cite{zeigler_theory_2000}.

According to the model, every agent makes \textit{on average} contact with $\beta$ random other agents per time unit. In ABS we can only contact discrete agents thus we model this by generating a random event on average every $\frac{1}{\beta}$ time units. We need to sample from an exponential distribution because the rate is proportional to the size of the population \cite{borshchev_system_2004}. Note that an agent does not know the other agents' state when making contact with it, thus we need a mechanism in which agents reveal their state in which they are in \textit{at the moment of making contact}. This mechanism is an implementation detail, which we will derive in our implementation steps. For now we only assume that agents can make contact with each other somehow.

The \textit{parallel} strategy matches the semantics of the agent-based SIR model due to the underlying roots in the System Dynamics approach. As discussed already in Chapter \ref{sub:par_strategy}, in the parallel update-strategy, the agents act conceptually all at the same time in lock-step. This implies that they observe the same environment state during a time-step and actions of an agent
are only visible in the next time-step - they are isolated from each other. As will become apparent, FP can be used to enforce the correct application of this strategy already on the compile-time level.

TODO: follow the PFE paper but add a few references back to the implementing ABS chapter so it is clear what problems we are currently solving. also might explain some concepts bit more in depth e.g. continuations?

\section{First step: pure computation}
\label{sec:timedriven_firststep}
As described in Chapter \ref{sec:back_frp}, Arrowized FRP \cite{hughes_generalising_2000} is a way to implement systems  with continuous and discrete time-semantics where the central concept is the signal function, which can be understood as a process over time, mapping an input- to an output-signal. Technically speaking, a signal function is a continuation which allows to capture state using closures and hides away the $\Delta t$, which means that it is never exposed explicitly to the programmer, meaning it cannot be manipulated. As already pointed out, agents need to perceive time, which means that the concept of processes over time is an ideal match for our agents and our system as a whole, thus we will implement them and the whole system as signal functions.

We start by defining the SIR states as ADT and our agents as signal functions (SF) which receive the SIR states of all agents form the previous step as input and outputs the current SIR state of the agent. This definition, and the fact that Yampa is not monadic, guarantees already at compile, that the agents are isolated from each other, enforcing the \textit{parallel} lock-step semantics of the model.

\begin{HaskellCode}
data SIRState = Susceptible | Infected | Recovered

type SIRAgent = SF [SIRState] SIRState 

sirAgent :: RandomGen g => g -> SIRState -> SIRAgent
sirAgent g Susceptible = susceptibleAgent g
sirAgent g Infected    = infectedAgent g
sirAgent _ Recovered   = recoveredAgent
\end{HaskellCode}

Depending on the initial state we return the corresponding behaviour. Note that we are passing a random-number generator instead of running in the Random Monad because signal functions as implemented in Yampa are not capable of being monadic. 

We see that the recovered agent ignores the random-number generator because a recovered agent does nothing, stays immune forever and can not get infected again in this model. Thus a recovered agent is a consuming state from which there is no escape, it simply acts as a sink which returns constantly \textit{Recovered}:

\begin{HaskellCode}
recoveredAgent :: SIRAgent
recoveredAgent = arr (const Recovered)
\end{HaskellCode}

Next, we implement the behaviour of a susceptible agent. It makes contact \textit{on average} with $\beta$ other random agents. For every \textit{infected} agent it gets into contact with, it becomes infected with a probability of $\gamma$. If an infection happens, it makes the transition to the \textit{Infected} state. To make contact, it gets fed the states of all agents in the system from the previous time-step, so it can draw random contacts - this is one, very naive way of implementing the interactions between agents.

Thus a susceptible agent behaves as susceptible until it becomes infected. Upon infection an \textit{Event} is returned, which results in switching into the \textit{infectedAgent} SF, which causes the agent to behave as an infected agent from that moment on. When an infection event occurs we change the behaviour of an agent using the Yampa combinator \textit{switch}, which is quite elegant and expressive as it makes the change of behaviour at the occurrence of an event explicit. Note that to make contact \textit{on average}, we use Yampas \textit{occasionally} function which requires us to carefully select the right $\Delta t$ for sampling the system as will be shown in results. 

Note the use of \textit{iPre :: a $\rightarrow$ SF a a}, which delays the input signal by one sample, taking an initial value for the output at time zero. The reason for it is that we need to delay the transition from susceptible to infected by one step due to the semantics of the \textit{switch} combinator: whenever the switching event occurs, the signal function into which is switched will be run at the time of the event occurrence. This means that a susceptible agent could make a transition to recovered within one time-step, which we want to prevent, because the semantics should be that only one state-transition can happen per time-step.

\begin{HaskellCode}
susceptibleAgent :: RandomGen g => g -> SIRAgent
susceptibleAgent g 
    = switch 
      -- delay switching by 1 step to prevent against transition
      -- from Susceptible to Recovered within one time-step
      (susceptible g >>> iPre (Susceptible, NoEvent)) 
      (const (infectedAgent g))
  where
    susceptible :: RandomGen g => g -> SF [SIRState] (SIRState, Event ())
    susceptible g = proc as -> do
      makeContact <- occasionally g (1 / contactRate) () -< ()
      if isEvent makeContact
        then (do
          -- draw random element from the list
          a <- drawRandomElemSF g -< as
          case a of
            Infected -> do
              -- returns True with given probability
              i <- randomBoolSF g infectivity -< ()
              if i
                then returnA -< (Infected, Event ())
                else returnA -< (Susceptible, NoEvent)
             _       -> returnA -< (Susceptible, NoEvent))
        else returnA -< (Susceptible, NoEvent)
\end{HaskellCode}

To deal with randomness in an FRP way, we implemented additional signal functions built on the \textit{noiseR} function provided by Yampa. This is an example for the stream character and statefulness of a signal function as it allows to keep track of the changed random-number generator internally through the use of continuations and closures. Here we provide the implementation of \textit{randomBoolSF}. \textit{drawRandomElemSF} works similar but takes a list as input and returns a randomly chosen element from it:

\begin{HaskellCode}
randomBoolSF :: RandomGen g => g -> Double -> SF () Bool
randomBoolSF g p = proc _ -> do
  r <- noiseR ((0, 1) :: (Double, Double)) g -< ()
  returnA -< (r <= p)
\end{HaskellCode}

An infected agent recovers \textit{on average} after $\delta$ time units. This is implemented by drawing the duration from an exponential distribution \cite{borshchev_system_2004} with $\lambda = \frac{1}{\delta}$ and making the transition to the \textit{Recovered} state after this duration. Thus the infected agent behaves as infected until it recovers, on average after the illness duration, after which it behaves as a recovered agent by switching into \textit{recoveredAgent}. As in the case of the susceptible agent, we use the \textit{occasionally} function to generate the event when the agent recovers. Note that the infected agent ignores the states of the other agents as its behaviour is completely independent of them.

\begin{HaskellCode}
infectedAgent :: RandomGen g => g -> SIRAgent
infectedAgent g 
    = switch 
      -- delay switching by 1 step 
      (infected >>> iPre (Infected, NoEvent))
      (const recoveredAgent)
  where
    infected :: SF [SIRState] (SIRState, Event ())
    infected = proc _ -> do
      recEvt <- occasionally g illnessDuration () -< ()
      let a = event Infected (const Recovered) recEvt
      returnA -< (a, recEvt)
\end{HaskellCode}

For running the simulation we use Yampas function \textit{embed}:

\begin{HaskellCode}
runSimulation :: RandomGen g => g -> Time -> DTime -> [SIRState] -> [[SIRState]]
runSimulation g t dt as 
    = embed (stepSimulation sfs as) ((), dts)
  where
    steps     = floor (t / dt)
    dts       = replicate steps (dt, Nothing)
    n         = length as
    (rngs, _) = rngSplits g n [] -- unique rngs for each agent
    sfs       = zipWith sirAgent rngs as
\end{HaskellCode}

What we need to implement next is a closed feedback-loop - the heart of every agent-based simulation. Fortunately, \cite{nilsson_functional_2002, courtney_yampa_2003} discusses implementing this in Yampa. The function \textit{stepSimulation} is an implementation of such a closed feedback-loop. It takes the current signal functions and states of all agents, runs them all in parallel and returns this step's new agent states. Note the use of \textit{notYet}, which is required because in Yampa switching occurs immediately at $t = 0$. If we don't delay the switching at $t = 0$ until the next step, we would enter an infinite switching loop - \textit{notYet} simply delays the first switching until the next time-step.

\begin{HaskellCode}
stepSimulation :: [SIRAgent] -> [SIRState] -> SF () [SIRState]
stepSimulation sfs as =
    dpSwitch
      -- feeding the agent states to each SF
      (\_ sfs' -> (map (\sf -> (as, sf)) sfs'))
      -- the signal functions
      sfs
      -- switching event, ignored at t = 0         
      (switchingEvt >>> notYet)
      -- recursively switch back into stepSimulation         
      stepSimulation                            
  where
    switchingEvt :: SF ((), [SIRState]) (Event [SIRState])
    switchingEvt = arr (\ (_, newAs) -> Event newAs)
\end{HaskellCode}

Yampa provides the \textit{dpSwitch} combinator for running signal functions in parallel, which has the following type-signature:

\begin{HaskellCode}
dpSwitch :: Functor col
         -- routing function
         => (forall sf. a -> col sf -> col (b, sf))
         -- SF collection
         -> col (SF b c)
         -- SF generating switching event     
         -> SF (a, col c) (Event d)
         -- continuation to invoke upon event           
         -> (col (SF b c) -> d -> SF a (col c))
         -> SF a (col c)
\end{HaskellCode}

Its first argument is the pairing-function, which pairs up the input to the signal functions - it has to preserve the structure of the signal function collection. The second argument is the collection of signal functions to run. The third argument is a signal function generating the switching event. The last argument is a function, which generates the continuation after the switching event has occurred. \textit{dpSwitch} returns a new signal function, which runs all the signal functions in parallel and switches into the continuation when the switching event occurs. The d in \textit{dpSwitch} stands for decoupled which guarantees that it delays the switching until the next time-step: the function into which we switch is only applied in the next step, which prevents an infinite loop if we switch into a recursive continuation.

Conceptually, \textit{dpSwitch} allows us to recursively switch back into the \textit{stepSimulation} with the continuations and new states of all the agents after they were run in parallel. 

\subsection{Results}
The dynamics generated by this step can be seen in Figure \ref{fig:sir_abs_dynamics_frp}. 

\begin{figure}
\begin{center}
	\begin{tabular}{c c}
		\begin{subfigure}[b]{0.4\textwidth}
			\centering
			\includegraphics[width=1\textwidth, angle=0]{./fig/timedriven/SIR_Yampa/SIR_Yampa_dt01.png}
			\caption{$\Delta t = 0.1$}
			\label{fig:sir_abs_approximating_01dt_1000agents}
		\end{subfigure}
		
		&
    	
		\begin{subfigure}[b]{0.4\textwidth}
			\centering
			\includegraphics[width=1\textwidth, angle=0]{./fig/timedriven/SIR_Yampa/SIR_Yampa_dt001.png}
			\caption{$\Delta t = 0.01$}
			\label{fig:sir_abs_approximating_001dt_1000agents}
		\end{subfigure}
	\end{tabular}
	
	\caption{FRP simulation of agent-based SIR showing the influence of different $\Delta t$. Population size of 1,000 with contact rate $\beta = \frac{1}{5}$, infection probability $\gamma = 0.05$, illness duration $\delta = 15$ with initially 1 infected agent. Simulation run for 150 time-steps with respective $\Delta t$.} 
	\label{fig:sir_abs_dynamics_frp}
\end{center}
\end{figure}

By following the FRP approach we assume a continuous flow of time, which means that we need to select a \textit{correct} $\Delta t$, otherwise we would end up with wrong dynamics. The selection of a correct $\Delta t$ depends in our case on \textit{occasionally} in the \textit{susceptible} behaviour, which randomly generates an event on average with \textit{contact rate} following the exponential distribution. To arrive at the correct dynamics, this requires us to sample \textit{occasionally}, and thus the whole system, with small enough $\Delta t$ which matches the frequency of events generated by \textit{contact rate}. If we choose a too large $\Delta t$, we loose events, which will result in wrong dynamics as can be seen in Figure \ref{fig:sir_abs_approximating_01dt_1000agents}. This issue is known as under-sampling and is described in Figure \ref{fig:sampling_issue}.

\begin{figure}
\begin{center}
	\begin{tabular}{c}
		\begin{subfigure}[b]{0.5\textwidth}
			\centering
			\includegraphics[width=1\textwidth, angle=0]{./fig/timedriven/undersampling.png}
			\caption{Under-sampling}
			\label{fig:undersampling}
		\end{subfigure}
		
		\\
		
		\begin{subfigure}[b]{0.5\textwidth}
			\centering
			\includegraphics[width=1\textwidth, angle=0]{./fig/timedriven/supersampling.png}
			\caption{Super-sampling}
			\label{fig:supersampling}
		\end{subfigure}
	\end{tabular}
	
	\caption{A visual explanation of under-sampling and super-sampling. The black dots represent the time-steps of the simulation. The red dots represent virtual events which occur at specific points in continuous time. In the case of under-sampling, 3 events occur in between the two time steps but \textit{occasionally} only captures the first one. By increasing the sampling frequency either through a smaller $\Delta t$ or super-sampling all 3 events can be captured.} 
	\label{fig:sampling_issue}
\end{center}
\end{figure}

For tackling this issue we have three options. The first one is to use a smaller $\Delta t$ as can be seen in \ref{fig:sir_abs_approximating_001dt_1000agents}, which results in the whole system being sampled more often, thus reducing performance. The second option is to step the simulation with $\Delta t = 1$ and in each step, instead of using \textit{occasionally}, to make a number of contacts drawn from the exponential distribution. Note that if we follow this option, we abandon the time-driven approach altogether because we don't abstract away from $\Delta t$ and violate the fundamental abstraction of FRP which assumes that time is continuous and signal functions are running conceptually infinitely fast and infinitely often \cite{winograd-cort_wormholes:_2012}. We will come back to this approach in the even-driven approach to ABS in Chapter \ref{sec:eventdriven_sir}. This leaves us with the third option to implement super-sampling and apply it to \textit{occasionally}, which allows us then to run the whole simulation with $\Delta t = 1.0$ and only sample the \textit{occasionally} function with a much higher frequency.

In Yampa there exists a function \textit{embed} which allows to run a given signal-function with provided $\Delta t$ but the problem is that this function does not really help because it does not return a signal-function. What we need is a signal-function which takes the number of super-samples \textit{n}, the signal-function \textit{sf} to sample and returns a new signal-function which performs super-sampling on it. We provide a full implementation of such a function, which also gives an insight into how signal functions are implemented in Yampa:

\begin{HaskellCode}
import FRP.Yampa.InternalCore

-- SF is the signal-function defined for time t = 0 and returns
-- a continuation of type SF' which is the signal-function 
-- defined for t > 0: it receives an additional time-delta
-- data SF a b  = SF { sfTF :: a -> (SF' a b, b) }
-- data SF' a b = DTime -> a -> (SF' a b, b)

superSampling :: Int -> SF a b -> SF a [b]
superSampling n sf0 = SF { sfTF = tf0 }
  where
    -- no supersampling at time 0
    tf0 :: a -> (SF' a b, [b])
    tf0 a0 = (tfCont, [b0])
      where
        (sf', b0) = sfTF sf0 a0 -- running a SF
        tfCont    = superSamplingAux sf'

    superSamplingAux :: SF' a [b]
    superSamplingAux sf' = SF' tf
      where
      	tf0 :: DTime -> a -> (SF' a b, [b])
        tf dt a = (tf', bs)
          where
            (sf'', bs) = superSampleRun n dt sf' a
            tf'        = superSamplingAux sf''

    superSampleRun :: Int -> DTime -> SF' a b -> a -> (SF' a b, [b])
    superSampleRun n dt sf a 
        | n <= 1    = superSampleMulti 1 dt sf a []
        | otherwise = (sf', reverse bs)  -- reverse due to accumulator
      where
        superDt = dt / fromIntegral n
        (sf', bs) = superSampleMulti n superDt sf a []

    superSampleMulti :: Int -> DTime -> SF' a b -> a -> [b] -> (SF' a b, [b])
    superSampleMulti 0 _ sf _ acc  = (sf, acc)
    superSampleMulti n dt sf a acc = superSampleMulti (n-1) dt sf' a (b:acc) 
      where
        (sf', b) = sfTF' sf dt a -- running a SF'
\end{HaskellCode}

It evaluates the \textit{SF} argument for \textit{n} times, each with $\Delta t = \frac{\Delta t}{n}$ and the same input argument \textit{a} for all \textit{n} evaluations. At time 0 no super-sampling is performed and just a single output of the \textit{SF} argument is calculated. A list of \textit{b} is returned with length of \textit{n} containing the result of the \textit{n} evaluations of the \textit{SF} argument. If 0 or less super samples are requested exactly one is calculated. We could then wrap the occasionally function which would then generate a list of events. 

\subsection{Discussion}
We can conclude that our first step already introduced most of the fundamental concepts of ABS:
\begin{itemize}
	\item Time - the simulation occurs over virtual time which is modelled explicitly, divided into \textit{fixed} $\Delta t$, where at each step all agents are executed.
	\item Agents - we implement each agent as an individual, with the behaviour depending on its state. It is clear to see that agents behave as signals: when the system is sampled with $\Delta t = 0$ then their behaviour will stay constant and won't change because it is completely determined by the flow of time. 
	\item Feedback - the output state of the agent in the current time-step $t$ is the input state for the next time-step $t + \Delta t$.
	\item Environment - as environment we implicitly assume a fully-connected network (complete graph) where every agent 'knows' every other agent, including itself and thus can make contact with all of them.
	\item Stochasticity - it is an inherently stochastic simulation, which is indicated by the random-number generator and the usage of \textit{occasionally}, \textit{randomBoolSF} and \textit{drawRandomElemSF}.
	\item Deterministic - repeated runs with the same initial random-number generator result in same dynamics. This may not come as a surprise but in Haskell we can guarantee that property statically already at compile time because our simulation runs \textit{not} in the IO Monad. This guarantees that no external, uncontrollable sources of non-determinism can interfere with the simulation.
	\item Parallel, lock-step semantics - the simulation implements a \textit{parallel} update-strategy where in each step the agents are run isolated in parallel and don't see the actions of the others until the next step.
\end{itemize}

Using FRP in the instance of Yampa results in a clear, expressive and robust implementation. State is implicitly encoded, depending on which signal function is active. By using explicit time-semantics with \textit{occasionally} we can achieve extremely fine grained stochastics by sampling the system with small $\Delta t$: we are treating it as a truly continuous time-driven agent-based system.

A very severe problem, hard to find with testing but detectable with in-depth validation analysis, is the fact that in the \textit{susceptible} agent the same random-number generator is used in \textit{occasionally}, \textit{drawRandomElemSF} and \textit{randomBoolSF}. This means that all three stochastic functions, which should be independent from each other, are inherently correlated. This is something one wants to prevent under all circumstances in a simulation, as it can invalidate the dynamics on a very subtle level, and indeed we have tested the influence of the correlation in this example and it has an impact. We left this severe bug in for explanatory reasons, as it shows an example where functional programming actually encourages very subtle bugs if one is not careful. A possible but not very elegant solution would be to simply split the initial random-number generator in \textit{sirAgent} three times (using one of the splited generators for the next split) and pass three random-number generators to \textit{susceptible}. A much more elegant solution would be to use the Random Monad which is not possible because Yampa is not monadic.

So far we have an acceptable implementation of an agent-based SIR approach. What we are lacking at the moment is a general treatment of an environment and an elegant solution to the random number correlation. In the next step we make the transition to Monadic Stream Functions as introduced in Dunai \cite{perez_functional_2016}, which allows FRP within a monadic context and gives us a way for an elegant solution to the random number correlation.

\section{Second step: going monadic}
\label{sec:timedriven_monadic}

A part of the library Dunai is BearRiver, a wrapper re-implementing Yampa on top of Dunai, which allows us to easily replace Yampa with MSFs. This will enable us to run arbitrary monadic computations in a signal function, solving the problem of correlated random numbers through the use of the \textit{Rand} Monad.

\subsection{Identity Monad}
We start by making the transition to BearRiver by simply replacing Yampas signal function by BearRivers', which is the same but takes an additional type parameter \textit{m}, indicating the monadic context. If we replace this type parameter with the \textit{Identity} Monad, we should be able to keep the code exactly the same, because BearRiver re-implements all necessary functions we are using from Yampa. We simply re-define the agent signal function, introducing the monad stack our SIR implementation runs in:

\begin{HaskellCode}
type SIRMonad = Identity
type SIRAgent = SF SIRMonad [SIRState] SIRState
\end{HaskellCode}

\subsection{Rand Monad}
Using the \textit{Identity} Monad does not gain us anything but it is a first step towards a more general solution. Our next step is to replace the \textit{Identity} Monad by the \textit{Rand} Monad, which will allow us to run the whole simulation within the \textit{Rand} Monad with the full features of FRP, finally solving the problem of correlated random numbers in an elegant way. We start by re-defining the SIRMonad and SIRAgent:

\begin{HaskellCode}
type SIRMonad g = Rand g
type SIRAgent g = SF (SIRMonad g) [SIRState] SIRState
\end{HaskellCode}

To access the \textit{Rand} Monad functionality within the MSF context, overloaded functions are used. For the function \textit{occasionally}, there exists a monadic pendant \textit{occasionallyM} which requires a \textit{MonadRandom} type class. Because we are now running within a \textit{MonadRandom} instance we simply replace \textit{occasionally} with \textit{occasionallyM}. 

\begin{HaskellCode}
occasionallyM :: MonadRandom m => Time -> b -> SF m a (Event b)
-- can be used through the use of arrM and lift
randomBoolM :: RandomGen g => Double -> Rand g Bool
-- this can be used directly as a SF with the arrow notation
drawRandomElemSF :: MonadRandom m => SF m [a] a
\end{HaskellCode}

\subsection{Discussion} 
Running in the Random Monad elegantly solves the problem of correlated random numbers and guarantees that we will not have correlated stochastics as discussed in the previous section. In the next step we introduce the concept of an explicit discrete 2D environment.

\section{Third Step: Adding an environment}
\label{sec:adding_env}
So far we have implicitly assumed a fully connected network amongst agents, where each agent can see and knows every other agent. This is a valid environment and in accordance with the System Dynamics inspired implementation of the SIR model but does not show the real advantage of ABS to situate agents within arbitrary environments. Often, agents are situated within a discrete 2D environment \cite{epstein_growing_1996} which is simply a finite $N x M$ grid with either a Moore or von Neumann neighbourhood (Figure \ref{fig:abs_neighbourhoods}). Agents are either static or can move freely around with cells allowing either single or multiple occupants.

We can directly map the SIR model to a discrete 2D environment by placing the agents on a corresponding 2D grid with an unrestricted neighbourhood. The behaviour of the agents is the same but they select their interactions directly from the shared read-only environment, which will be passed to the agents as input. This allows agents to read the states of all their neighbours, which tells them if a neighbour is infected or not. To show the benefit over the System Dynamics approach  and for purposes of a more interesting approach, we restrict the neighbourhood to Moore (Figure \ref{fig:moore_neighbourhood}).

\begin{figure}
\begin{center}
	\begin{tabular}{c c}
		\begin{subfigure}[b]{0.3\textwidth}
			\centering
			\includegraphics[width=0.5\textwidth, angle=0]{./fig/timedriven/neumann.png}
			\caption{von Neumann}
			\label{fig:neumann_neighbourhood}
		\end{subfigure}
    	&
		\begin{subfigure}[b]{0.3\textwidth}
			\centering
			\includegraphics[width=0.5\textwidth, angle=0]{./fig/timedriven/moore.png}
			\caption{Moore}
			\label{fig:moore_neighbourhood}
		\end{subfigure}
    \end{tabular}
	\caption{Common neighbourhoods in discrete 2D environments of Agent-Based Simulation.}
	\label{fig:abs_neighbourhoods}
\end{center}
\end{figure}

We also implemented this spatial approach in Java using the well known ABS library RePast \cite{north_complex_2013}, to have a comparison with a state of the art approach and came to the same results as shown in Figure \ref{fig:sir_dunai}. This supports, that our pure functional approach can produce such results as well and compares positively to the state of the art in the ABS field.

\subsection{Implementation}
We start by defining the discrete 2D environment for which we use an indexed two dimensional array. Each cell stores the agent state of the last time-step, thus we use the \textit{SIRState} as type for our array data. Also, we re-define the agent signal function to take the structured environment \textit{SIREnv} as input instead of the list of all agents as in our previous approach. As output we keep the \textit{SIRState}, which is the state the agent is currently in. Also we run in the \textit{Rand} Monad as introduced before to avoid the random number correlation. 

\begin{HaskellCode}
type Disc2dCoord = (Int, Int)
type SIREnv      = Array Disc2dCoord SIRState

type SIRAgent g  = SF (Rand g) SIREnv SIRState
\end{HaskellCode}

Note that the environment is not returned as output because the agents do not directly manipulate the environment but only read from it. Again, this enforces the semantics of the \textit{parallel} update-strategy through the types where the agents can only see the previous state of the environment and see the actions of other agents reflected in the environment only in the next step.

Note that we could have chosen to use a \textit{StateT} transformer with the \textit{SIREnv} as state, instead of passing it as input, with the agents then able to arbitrarily read/write, but this would have violated the semantics of our model because actions of agents would have become visible within the same time-step.

The implementation of the susceptible, infected and recovered agents are almost the same with only the neighbour querying now slightly different. 

Stepping the simulation needs a new approach because in each step we need to collect the agent outputs and update the environment for the next step. For this we implemented a separate MSF, which receives the coordinates for every agent to be able to update the state in the environment after the agent was run. Note that we need use \textit{mapM} to run the agents because we are running now in the context of the \textit{Rand} Monad. This has the consequence that the agents are in fact run sequentially one after the other but because they cannot see the other agents actions nor observe changes in the shared read-only environment, it is \textit{conceptually} a \textit{parallel} update-strategy where agents run in lock-step, isolated from each other at conceptually the same time.
  
\begin{HaskellCode}
simulationStep :: RandomGen g => [(SIRAgent g, Disc2dCoord)]
               -> SIREnv -> SF (Rand g) () SIREnv
simulationStep sfsCoords env = MSF (\_ -> do
  let (sfs, coords) = unzip sfsCoords 
  -- run agents sequentially but with shared, read-only environment
  ret <- mapM (`unMSF` env) sfs
  -- construct new environment from all agent outputs for next step
  let (as, sfs') = unzip ret
      env' = foldr (\ (a, coord) envAcc -> updateCell coord a envAcc) 
               env (zip as coords)

      sfsCoords' = zip sfs' coords
      cont       = simulationStep sfsCoords' env'
  return (env', cont))
 
updateCell :: Disc2dCoord -> SIRState -> SIREnv -> SIREnv
\end{HaskellCode}

\subsection{Results}
We implemented rendering of the environments using the gloss library which enabled us to cycle arbitrarily through the steps and inspect the spreading of the disease over time visually as seen in Figure \ref{fig:sir_dunai}.

\begin{figure}
\begin{center}
	\begin{tabular}{c c}
		\begin{subfigure}[b]{0.4\textwidth}
			\centering
			\includegraphics[width=1\textwidth, angle=0]{./fig/timedriven/SIR_Dunai/SIR_Dunai_dt001_environment.png}
			\caption{Environment at $t = 50$}
			\label{fig:sir_dunai_env}
		\end{subfigure}
    	
    	&
  
		\begin{subfigure}[b]{0.43\textwidth}
			\centering
			\includegraphics[width=1\textwidth, angle=0]{./fig/timedriven/SIR_Dunai/SIR_Dunai_dt001.png}
			\caption{Dynamics over time}
			\label{fig:sir_dunai_env_dynamics}
		\end{subfigure}
	\end{tabular}
	
	\caption{Simulating the agent-based SIR model on a 21x21 2D grid with Moore neighbourhood (Figure \ref{fig:moore_neighbourhood}), a single infected agent at the center and same SIR parameters as in Figure \ref{fig:sir_sd_dynamics}. Simulation run until $t = 200$ with fixed $\Delta t = 0.01$. Last infected agent recovers around $t = 194$. The susceptible agents are rendered as blue hollow circles for better contrast.}
	\label{fig:sir_dunai}
\end{center}
\end{figure}

Note that the dynamics of the spatial SIR simulation, which are seen in Figure \ref{fig:sir_dunai_env_dynamics}, look quite different from the reference dynamics of Figure \ref{fig:sir_sd_dynamics}. This is due to a much more restricted neighbourhood, resulting in far fewer infected agents at a time and a lower number of recovered agents at the end of the epidemic, meaning that fewer agents got infected overall.

\subsection{Discussion}
Introducing a structured environment with a Moore neighbourhood, shows the ability of ABS to place the heterogeneous agents in a generic environment, which is the fundamental advantage of an agent-based approach over other simulation methodologies and allows us to simulate much more realistic scenarios.

Note, that an environment is not restricted to be a discrete 2D grid and can be anything from a continuous N-dimensional space to a complex network - one only needs to change the type of the environment and agent input and provide corresponding neighbourhood querying functions. 

\section{Discussion}
Our FRP based approach is different from traditional approaches in the ABS community. First it builds on the already quite powerful FRP paradigm. Second, due to our continuous time approach, it forces one to think properly of time-semantics of the model and how small $\Delta t$ should be. Third it requires one to think about agent interactions in a new way instead of being just method-calls.

Because no part of the simulation runs in the IO Monad and we do not use \textit{unsafePerformIO} we can rule out a serious class of bugs caused by implicit data-dependencies and side-effects, which can occur in traditional imperative implementations.

Also we can statically guarantee the reproducibility of the simulation, which means that repeated runs with the same initial conditions are guaranteed to result in the same dynamics. Although we allow side-effects within agents, we restrict them to only the Random Monad in a controlled, deterministic way and never use the IO Monad, which guarantees the absence of non-deterministic side effects within the agents and other parts of the simulation.

Determinism is also ensured by fixing the $\Delta t$ and not making it dependent on the performance of e.g. a rendering-loop or other system-dependent sources of non-determinism as described by \cite{perez_testing_2017}. Also by using FRP we gain all the benefits from it and can use research on testing, debugging and exploring FRP systems \cite{perez_testing_2017, perez_back_2017}.

Also we showed how to implement the \textit{parallel} update-strategy \cite{thaler_art_2017} in a way that the correct semantics are enforced and guaranteed already at compile time through the types. This is not possible in traditional imperative implementations and poses another unique benefit over the use of functional programming in ABS.

The result of using FRP allows expressing continuous time-semantics in a very clear, compositional and declarative way, abstracting away the low-level details of time-stepping and progress of time within an agent.
	
Our approach can guarantee reproducibility already at compile time, which means that repeated runs of the simulation with the same initial conditions will always result in the same dynamics, something highly desirable in simulation in general. This can only be achieved through purity, which guarantees the absence of implicit side-effects, which allows to rule out non-deterministic influences at compile time through the strong static type system, something not possible with traditional object-oriented approaches. Further, through purity and the strong static type system, we can rule out important classes of run-time bugs e.g. related to dynamic typing, and the lack of implicit data-dependencies which are common in traditional imperative object-oriented approaches.
	
Using pure functional programming, we can enforce the correct semantics of agent execution through types where we demonstrate that this allows us to have both, sequential monadic behaviour, and agents acting \textit{conceptually} at the same time in lock-step, something not possible using traditional object-oriented approaches.

Currently, the performance of the system does not come close to imperative implementations. We compared the performance of our pure functional approach as presented in Section \ref{sec:adding_env} to an implementation in Java using the ABS library RePast \cite{north_complex_2013}. We ran the simulation until $t = 100$ on a 51x51 (2,601 agents) with $\Delta t = 0.1$ (unknown in RePast) and averaged 8 runs. The performance results make the lack of speed of our approach quite clear: the pure functional approach needs around 72.5 seconds whereas the Java RePast version just 10.8 seconds on our machine to arrive at $t = 100$. It must be mentioned, that RePast does implement an event-driven approach to ABS, which can be much more performant \cite{meyer_event-driven_2014} than a time-driven one as ours, so the comparison is not completely valid. Still, we have already started investigating speeding up performance through the use of Software Transactional Memory \cite{harris_composable_2005, harris_transactional_2006}, which is quite straight forward when using MSFs. It shows very good results but we have to leave the investigation and optimization of the performance aspect of our approach for further research as it is beyond the scope of this paper.

Despite the strengths and benefits we get by leveraging on FRP, there are errors that are not raised at compile time, e.g. we can still have infinite loops and run-time errors. This was for example investigated in \cite{sculthorpe_safe_2009} where the authors use dependent types to avoid some run-time errors in FRP. We suggest that one could go further and develop a domain specific type system for FRP that makes the FRP based ABS more predictable and that would support further mathematical analysis of its properties. Furthermore, moving to dependent types would pose a unique benefit over the traditional object-oriented approach and should allow us to express and guarantee even more properties at compile time. We leave this for further research.

In our pure functional approach, agent identity is not as clear as in traditional object-oriented programming, where there is a quite clear concept of object-identity through the encapsulation of data and methods. Signal functions don't offer this strong identity and one needs to build additional identity mechanisms on top e.g. when sending messages to specific agents.

We can conclude that the main difficulty of a pure functional approach evolves around the communication and interaction between agents, which is a direct consequence of the issue with agent identity. Agent interaction is straight-forward in object-oriented programming, where it is achieved using method-calls mutating the internal state of the agent, but that comes at the cost of a new class of bugs due to implicit data flow. In pure functional programming these data flows are explicit but our current approach of feeding back the states of all agents as inputs is not very general. We have added further mechanisms of agent interaction which we had to omit due to lack of space.

\subsection{Other approaches}
Sequential: see event-driven
Concurrent: see STM chapter
Actor: not discussed in this thesis but briefly 

\subsection{Super-Sampling}
