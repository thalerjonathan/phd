\section{Papers}
This is the list of papers I have in my mind and includes both work mandatory for the PhD and optional ones. The latter ones are just fun/philosophical-papers, not directly related to my PhD but somehow tangential with the very basic direction - they are intended to be worked on in my free-time and to free my head when wrestling too hard with my PhDs main work.

\subsection{Influence of Simulation-Semantics on Dynamics of Agent-Based Simulations.}
\textbf{Type:} Groundwork \\
\textbf{Target:} Conference/Journal \\
\textbf{Requirement:} Mandatory \\

The first paper which describes how one can implement ABM/S in Haskell and compares the implementation and results to Java and Akka. A major focus are update-strategies, parallelism, reproducibility, reasoning and comparability between the various implementations. Actors: The Future in Agent-Based Simulation \& Modelling? Although the actor-model is quite old (beginning of the 70s) it seems to have a revival both in Erlang in the 90s and now in the Framework Akka (based on Scala). It is one way of organizing highly parallel (and optionally distributed) applications. Also the actor-model is very close to the agent-metaphor where the latter one was strongly inspired by the former one. Thus It would be very interesting to look closer into how the Actor-Model can be utilized to ABM/S as it seems that this has not been properly done yet. This paper will establish my methodology in using Haskell / pure functional programming in the 2nd year main work.

\subsection{Pure Functional ACE (Catchy title yet to be defined)}
\textbf{Type:} Main PhD Work \\
\textbf{Target:} Journal \\
\textbf{Requirement:} Mandatory \\

Is the main work of the PhD and targeted at publication in a Journal. The exact topic and content will be clarified at the beginning of the 2nd year. Mainly it will describe how to implement Ionescus Framework of Gintis trading model and extend it to a more general Market-Model. It will also give an outlook on implementing it using dependent types.

\subsection{Pure by Nature: A Library for pure Agent-Based Simulation \& Modelling in Haskell}
\textbf{Type:} Extension \\
\textbf{Target:} Conference \\
\textbf{Requirement:} Optional \\

This paper describes the ideas and theory behind the implementation of my ABM/S library "PureAgents" in Haskell.

\subsection{Time in Games: a Tron Light-Cycle Game in Dunai}
\textbf{Type:} Fun \\
\textbf{Target:} Conference \\
\textbf{Requirement:} Optional \\

This paper describes the 2D light-cycle game inspired by the movie Tron implemented in Dunai. It allows to turn back time.

\subsection{Pure Functional Islamic Design}
\textbf{Type:} Fun \\
\textbf{Target:} Conference \\
\textbf{Requirement:} Optional \\

Inspired by the paper "Functional Geometry" by Peter Henderson I had the idea to come up with a  EDSL for declaratively describing pictures of islamic design which are then rendered using the gloss-library. From its focus totally unrelated to the PhD topic but still a great opportunity to learn Haskell, to learn to think functional, to learn to design my own EDSL - thus it may be a great paper to pursue even if I won't finish or produce something publishable.

\subsection{The Genesis According to Computer-Science: Reality as Simulation of Free Will}
\textbf{Type:} Philosophy \\
\textbf{Target:} ? \\
\textbf{Requirement:} Optional \\

I've always been interested in a deeper meaning behind things so I want to look into the philosophy and future of simulation: why do we simulate, what can we derive from simulations, what does it say that we humans simulate, what will the future of simulation be? \\
I claim that our ability to "simulate" in our mind separates our intelligence from those of the animals and that this is a unique property of humans. Also i think the future of simulation will be that humankind will do its own creation/live (artifical life, conciousness) which allows to accurately simulate a given setting - this of course could have ethical implications.