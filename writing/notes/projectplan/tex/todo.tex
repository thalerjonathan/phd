\section{TODO-List as of \today}

\begin{enumerate}
\item prepare presentation: 15 slides max excluding chapters. up to 30 min

\item implement wildfire dynamic in java for creating new agents and deleting old ones

\item make act run ok in java

\item implement sirs and spatialgame in scala with actors

\item act versions should send state updates to a visualizer agent: in haskell very difficult due to static typing of messages and not running in IO. can we do that in java?

\item look into statistics of retries in STM \url{https://hackage.haskell.org/package/stm-stats-0.2.0.0/docs/Control-Concurrent-STM-Stats.html}

\item find interesting and large, complex ACE model simulating an economy

\item Bring PureAgents to Yampa
	\begin{itemize}
		\item embed PureAgentsPar and Seq in Yampa: PureAgentsYampa so we can leverage the power of the EDSL, SFs, continuations,... of Yampa/Dunai.
		\item implement agent monad: PureAgentsMonadic. but what is an Agent-Monad? build a monad to chain actions of the agent and always run inside an agent-monad
		\item embed PureAgentsMonadic in Dunai
		\item implement wait blocking for a message so far. utilize yampas event mechanism?
	\end{itemize}
	
\item Push PureAgentsYampa by considering the next step: implementing an economics example at different levels of complexity e.g. Ionescus implementation of Gintis

\item try algebraic reasoning using my framework

\item Start Writing the paper
the main idea of this paper is:...
in this section we present the main contributions of the paper:...

write a story:
	1. the problem
	2. its an interesting problem
	3. its an unsolved problem
	4. my idea
	5. my idea works (see implementations)
	6. my idea compares to other peoples approaches
	
Structure (Conf paper)
	Title
	Abstract (4 sentences)
		descibe the problem & state your contributions
	Introduction 1 page
		write list of contributions first,
		this list drives the entire paper,
		the paper substantiates the claims you have made
		
		Dont tell what happens in which section,
		instead use forward-references from the
		narrative in the introduction
	The problem 1 page
	My idea 2 pages
	The details 5 pages
		convey the intuitino is primary, not secondary
		once your reader has the intuitive, she can follow the details
		
		give examples and how your idea solves it
		choose the most direct route to the idea
			dont explain all paths which didn't work
			but mention if an obvious route does not work
	related work 1-2 pages
		do at the very end because only expert readers will understand it upfront
		giving credit to other papers doesnt diminish your paper
	conclusion and further work


\item FINALIZE AND FREEZE the literature I use for my paper (see next items READ): its all on my desk plus some prints to do (organize papers i need to read: ACE \& Economics, complex systems \& simulation, functional programming, abm \& actor model)
	\begin{itemize}
		\item READ: Gleichzeitige Ungleichzeitigkeiten
		\item READ: Complex Systems 'Cellular Automata' \& 'Turing Machines'
		\item READ: Functional programming \& Simulation papers (on desk)
		\item READ: Semantics of Actors (Hewitt, Greif \& Clinger)
		\item READ: philosophical papers of simulation
		\item READ: functional programming papers
		\item SEARCH: for further papers looking into my direction (when the difference between SEQ and PAR matters / matters not)
	\end{itemize}

\item Always ask the question: what is the difference to the OO approach? 
\end{enumerate} 

\section{Future-List as of \today}
\begin{enumerate}
\item Topics \& Issues of Haskell Implementation
	\begin{itemize}
		\item performance unacceptable: 1000 in haskell vs 100.000 in java is a shame on haskell, more should be possible. investigate using profiling both of CPU and memory: \url{http://keera.co.uk/blog/2014/10/15/from-60-fps-to-500/}. strictness \& tail-recursion!	
		\item look into QuickCheck and HPC
		\item problem: so far only agents with same static messagetypes, environment and states, can communicate: the agents are homogenous. how can we implement hetereogenous agents in this library?
		\item implement a general-purpose rendering-frontend with gloss but let the simulation be driven by Yampa/SimulationBackend instead of frontend (not real-time)
		\item implement PheroTrails: agents move on a 2d grid in a 8-neighbourhood and leave pheromone-trails which decay over time. the agents select the neighbourhood cell they move in the next step randomly according to the amount of pheromones present: the more pheromones are present the more likely they will move to that cell - note that this is relative: the pheromones of all neighbour cells are added up and normalized!

	\end{itemize}
	
\item Simulation Model Ideas
	\begin{itemize}
		\item IDEA: abm/s of a go game
		\item IDEA: what about an ABM/S of karma and rebirth? add to genesis paper
		\item IDEA: what about ABM/S generating sound? could be a perfect example for Yampa due to its signal functions. the sound is the result of interactions of agents which try to generate harmonies and agents trying to create dissonance
		\item IDEA: what about abm/s creating drawings/art? 2d continuous and each agents path is drawn

	\end{itemize}
\end{enumerate} 
