%%%%%%%%%%%%%%%%%%%%%%%%%%%%%%%%%%%%%%%%%
% University/School Laboratory Report
% LaTeX Template
% Version 3.1 (25/3/14)
%
% This template has been downloaded from:
% http://www.LaTeXTemplates.com
%
% Original author:
% Linux and Unix Users Group at Virginia Tech Wiki 
% (https://vtluug.org/wiki/Example_LaTeX_chem_lab_report)
%
% License:
% CC BY-NC-SA 3.0 (http://creativecommons.org/licenses/by-nc-sa/3.0/)
%
%%%%%%%%%%%%%%%%%%%%%%%%%%%%%%%%%%%%%%%%%

%----------------------------------------------------------------------------------------
%	PACKAGES AND DOCUMENT CONFIGURATIONS
%----------------------------------------------------------------------------------------

\documentclass{article}

\usepackage[utf8]{inputenc}
\usepackage{graphicx} % Required for the inclusion of images
%\usepackage{natbib} % Required to change bibliography style to APA
\usepackage{amsmath} % Required for some math elements 
\usepackage{glossaries}
\usepackage[toc,page]{appendix}
\usepackage[autostyle=true]{csquotes}
\usepackage{hyperref}
\usepackage{amssymb}
\usepackage{caption} 
\usepackage{hhline}
\usepackage{float}
\usepackage{listings}

\setlength\parindent{0pt} % Removes all indentation from paragraphs

%\usepackage{times} % Uncomment to use the Times New Roman font

%----------------------------------------------------------------------------------------
%	DOCUMENT INFORMATION
%----------------------------------------------------------------------------------------

\title{Influence of Simulation-Semantics on \\ Dynamics of an Agent-Based Simulation} % Title
 
\author{Jonathan \textsc{Thaler}} % Author name

\date{\today} % Date for the report

\begin{document}

\maketitle % Insert the title, author and date

% If you wish to include an abstract, uncomment the lines below
\begin{abstract}
TODO: too long, make an abstract of the abstract!

In this paper we look at the very simple social-simulation of \textit{Heroes \& Cowards} invented by \cite{wilensky_introduction_2015} to study the impact of different simulation-semantics on the dynamics of the simulation. By simulation-semantics we understand the different approaches of how to iterate a simulation and we ask whether the dynamics are somewhat stable, change or completely break down under different semantics. We draw parallels to complex dynamic systems for which it is well known that slightly different starting-conditions can lead to completely different dynamics after a few steps. We develop a classification of all potential simulation-semantics and discuss general implementation-considerations independent of the programming language. We then give implementations of each simulation-semantic in varying programming languages and compare the resulting dynamics. Depending on the given semantics we choose out of three languages: Java, Haskell and Scala with Actors where each of them has their strengths and limitations implementing simulation-semantics. We implement only the semantics for which the given language is suited for without abusing it. Thus in Java we focus on object-oriented programming, side-effects and global data, where in Haskell we focus on pure functional programming with local-only immutable data, explicit dataflow, higher order functions and recursion, and in Scala with Actors we put emphasis on a mixed-paradigm approach and the usage of Actors as defined by \cite{agha_actors:_1986}. \\
It is important to note that the implementations in Haskell present a novel, pure functional approach to ABM/S which strengths are the declarative style of programming and the strong static type-system. This allows to reason about a program and implement embedded domain-specific languages (EDSL) where ideally the distinction between a formal specification and the implementation disappears. We present the EDSL implemented for this Model and investigate what and how we can reason about properties and dynamics of our simulation and of ABM/S in general having the EDSL, the types and our program at hand. \\
Thus the main contributions of this paper are fourfold. First it establishes a terminology for speaking about and classifying simulation-semantics, second it gives a general framework to discuss dynamics of an ABM/S under different simulation-semantics, third it discusses the suitability of three very different programming-languages to implement the various simulation-semantics with a special emphasis on pure functional approach with comparison to state-of-the-art OO and fourth it looks into the possibility and power of reasoning about properties and dynamics of an ABM/S with specific simulation-semantics implemented in the pure functional language Haskell.
\end{abstract}

\section{Overview}
In this paper we look at the very simple social-simulation of \textit{Heroes \& Cowards} invented by \cite{wilensky_introduction_2015} to study the impact of different simulation-semantics on the dynamics of the simulation. By simulation-semantics we understand the different approaches of how to iterate a simulation and we ask whether the dynamics are somewhat stable, change or completely break down under different semantics. We draw parallels to complex dynamic systems for which it is well known that slightly different starting-conditions can lead to completely different dynamics after a few steps. We develop a classification of all potential simulation-semantics and discuss general implementation-considerations independent of the programming language. We then give implementations of each simulation-semantic in varying programming languages and compare the resulting dynamics. Depending on the given semantics we choose out of three languages: Java, Haskell and Scala with Actors where each of them has their strengths and limitations implementing simulation-semantics. We implement only the semantics for which the given language is suited for without abusing it. Thus in Java we focus on object-oriented programming, side-effects and global data, where in Haskell we focus on pure functional programming with local-only immutable data, explicit dataflow, higher order functions and recursion, and in Scala with Actors we put emphasis on a mixed-paradigm approach and the usage of Actors as defined by \cite{agha_actors:_1986}. \\
It is important to note that the implementations in Haskell present a novel, pure functional approach to ABM/S which strengths are the declarative style of programming and the strong static type-system. This allows to reason about a program and implement embedded domain-specific languages (EDSL) where ideally the distinction between a formal specification and the implementation disappears. We present the EDSL implemented for this Model and investigate what and how we can reason about properties and dynamics of our simulation and of ABM/S in general having the EDSL, the types and our program at hand. \\
Thus the main contributions of this paper are fourfold. First it establishes a terminology for speaking about and classifying simulation-semantics, second it gives a general framework to discuss dynamics of an ABM/S under different simulation-semantics, third it discusses the suitability of three very different programming-languages to implement the various simulation-semantics with a special emphasis on pure functional approach with comparison to state-of-the-art OO and fourth it looks into the possibility and power of reasoning about properties and dynamics of an ABM/S with specific simulation-semantics implemented in the pure functional language Haskell.

\subsection{What we try to do in this paper}
approach complex dynamic systems from a computational perspective

\subsection{What we don't want to do in this paper}
a general theory of complex dynamic systems
solve the presented complex dynamic system analytically
go into any complex analytical mathematical stuff


%*******************************************************************************
%*********************************** First Chapter *****************************
%*******************************************************************************

\chapter{Introduction}  %Title of the First Chapter
I noticed that it is pretty hard to convince an agent-based economics specialist who is not a computer scientist about a pure functional approach. My conjecture is that the implementation technique and method does not matter much to them because they have very little knowledge about programming and are almost always self-taught - they don't know about software-engineering, nothing about proper software-design and architecture, nothing about software-maintenance, nothing about unit-testing,... In the end they just "hack" the simulation in whatever language they are able to: C++, Visual Basic, Java or toolboxes like Netlogo. For them it is all about to \textit{get things done somehow} and not to get things done the right way or in a beautiful way - the way and the method doesn't matter, its just a necessary evil which needs to be done. Thus if functional programming could make their lives easier, then they will definitely welcome it. But functional programming is, i think, harder to learn and harder to understand - so one needs to provide an abstraction through EDSL. So I REALLY need to come up with convincing arguments why to use pure functional approaches in ACE THEY can understand, otherwise I will be lost and not heard (not published,...). \\

What ACE economists care for:

\begin{itemize}
\item Very: Qualitative modelling with quantitative results
\item Yes: Easy reproducibility
\item Likely: Reasoning about convergence?
\item Likely: EDSL
\end{itemize}

My contributions are: pure functional framework, functional agent-model for market-simulations, EDSL for market-simulations, qualitative / implicit modelling with quanitative results, reasoning in my framework about convergence \\

IDEA: could I develop non-causal modelling (models are expressed in terms of non-directed equations, modelled in signal-relations) to allow for qualitative modelling for the agent-based economists? See hybrid modelling paper of Yampa. \textbf{THIS WOULD BE A HUGE NOVEL CONTRIBUTION TO ACE ESPECIALLY WHEN COMBINED WITH AN EDSL AND PROVIDING FULL REFERENTIAL TRANSPARENCY TO KEEP THE ABILITY TO REASON ABOUT CONVERGENCE}. This should be covered in the "EDSL"-paper.

TODO: maybe i should really focus only on market models? otherwise too much? \\

central novelty of my PhD: model specification = runnable code. possible through EDSL. but only in specific subfield of ACE: market-models. need a functional description of the model, then translate it to model specification in EDSL and then run it to see dynamics. But: model specification moves closer to functional programming languages. \\

another novelty approach: model specification through qualitative instead of quantiative approaches. is this possible? \\

WHY FUNCTIONAL? "because its the ultimate approach to scientific computing": fewer bugs due to mutable state (why? is thos shown obkectively by someone?), shorter (again as above, productivity), more expressive and closer to math, EDSL, EDSL=model=simulation, better parallelising due to referental transparency, reasoning \\

scientific results need to be reproduced, especially when they have high impact. a more formal approach of specifying the model and the simulation (model=simulation) could lead to easier sharing and easier reporduction without ambigouites \\

pure functional agent-model \& theory, EDSL framework in Haskell for ACE

\begin{enumerate}
\item Which kind of problem do we have?
\item What aim is there? Solving the problem? 
\item How the aim is achieved by enumerating VERY CLEAR objectives.
\item What the impact one expects (hypothesis) and what it is (after results).
\end{enumerate}

Note: It is not in the interest of the researcher to develop new economic theories but to research the use of functional methods (programming and specification) in agent-based computational economics (ACE).

NOTE: Get the reader’s attention early in the introduction: motivation, significance, originality and novelty.

\section{Methods}
Methods need to be selected to implement the simulations. Special emphasis will be put on functional ones which will then be compared to established methods in the field of ABM/S and ACE. \\

Claim: non-programming environments are considered to be not powerful enough to capture the complexity of ACE implementations thus a programming approach to ACE will be always required.

\section{Scenarios}
To apply and test functional methods in ACE, four scenarios of ACE are selected and then the methods applied and compared with each other to see how each of them perform in comparison. The 4 selected scenarios represent a selection of the challenges posed in ACE: from very abstract ones to very operational ones.

\section{Comparison}
Each of the selected scenarios is then implemented using the selected methods where each solution is then compared against the following criteria: 

\begin{enumerate}
\item suitability for scientific computation
\item robustness
\item error-sources
\item testability
\item stability
\item extendability
\item size of code
\item maintainability
\item time taken for development
\item verification \& correctness
\item replications \& parallelism
\item EDSL
\end{enumerate}

This will then allow to compare the different methods against each other and to show under which circumstances functional methods shine and when they should not be used.

\section{Agent-Based Modelling and Simulation (ABM/S)}
ABM/S is a method of modelling and simulating a system where the global behaviour may be unknown but the behaviour and interactions of the parts making up the system is of knowledge (Wooldrige, M. (2009). An Introduction to MultiAgent Systems. John Wiley & Sons). Those parts, called agents, are modelled and simulated out of which then the aggregate global behaviour of the whole system emerges. Thus the central aspect of ABM/S is the concept of an Agent which can be understood as a metaphor for a pro-active unit, able to spawn new Agents, and interacting with other Agents in a network of neighbours by exchange of messages. The implementation of Agents can vary and strongly depends on the programming language and the kind of domain the simulation and model is situated in.

\section{Agent-Based Economics (ACE)}
According to Leigh Tesfatsion (Tesfatsion, L. (2006). Agent-based computational economics: A constructive approach to economic theory. In Tesfatsion, L. and Judd, K. L., editors, Handbook of Computational Economics, volume 2, chapter 16, pages 831–880. Elsevier, 1 edition.), one of the leading figures, ACE is "[...] computational modelling of economic processes (including whole economies) as open-ended dynamic systems of interacting agents." - thus lending perfectly to the use of ABM/S as already the name suggests. Whereas classical economic models fall short by only looking at the average, pure rational, individual interacting in anonymous markets, the ACE approach looks at heterogeneous, non-rational individuals interacting with each other in networks (Kirman, A. (2010). Complex Economics: Individual and Collective Rationality. Routledge, London ; New York, NY.). Thus ACE can be understood as a combination of computer-science, cognitive/social science and evolutionary economics.

\section{Functional programming}
TODO: read \cite{Backus1978}

The state-of-the-art approach to implementing Agents are object-oriented methods and programming as the metaphor of an Agent as presented above lends itself very naturally to object-orientation (OO). The author of this thesis claims that OO in the hands of inexperienced or ignorant programmers is dangerous, leading to bugs and hardly maintainable and extensible code. The reason for this is that OO provides very powerful techniques of organising and structuring programs through Classes, Type Hierarchies and Objects, which, when misused, lead to the above mentioned problems. Also major problems, which experts face as well as beginners are 1. state is highly scattered across the program which disguises the flow of data in complex simulations and 2. objects don’t compose as well as functions. The reason for this is that objects always carry around some internal state which makes it obviously much more complicated as complex dependencies can be introduced according to the internal state.
All this is tackled by (pure) functional programming which abandons the concept of global state, Objects and Classes and makes data-flow explicit. This then allows to reason about correctness, termination and other properties of the program e.g. if a given function exhibits side-effects or not. Other benefits are fewer lines of code, easier maintainability and ultimately fewer bugs thus making functional programming the ideal choice for scientific computing and simulation and thus also for ACE. A very powerful feature of functional programming is Lazy evaluation. It allows to describe infinite data-structures and functions producing an infinite stream of output but which are only computed as currently needed. Thus the decision of how many is decoupled from how to (Hughes, J. (1989). Why functional programming matters. Comput. J., 32(2):98–107.).
The most powerful aspect using pure functional programming however is that it allows the design of embedded domain specific languages (EDSL). In this case one develops and programs primitives e.g. types and functions in a host language (embed) in a way that they can be combined. The combination of these primitives then looks like a language specific to a given domain, in the case of this thesis ACE. The ease of development of EDSLs in pure functional programming is also a proof of the superior extensibility and composability of pure functional languages over OO (Henderson P. (1982). Functional Geometry. Proceedings of the 1982 ACM Symposium on LISP and Functional Programming.).
One of the most compelling example to utilize pure functional programming is the reporting of Hudak (Hudak P., Jones M. (1994). Haskell vs. Ada vs. C++ vs. Awk vs. ... An Experiment in Software Prototyping Productivity. Department of Computer Science, Yale University.)  where in a prototyping contest of DARPA the Haskell prototype was by far the shortest with 85 lines of code. Also the Jury mistook the code as specification because the prototype did actually implement a small EDSL which is a perfect proof how close EDSL can get to and look like a specification.

Functional languages can best be characterized by their way computation works: instead of \textit{how} something is computed, \textit{what} is computed is described. Thus functional programming follows a declarative instead of an imperative style of programming. The key points are:
\begin{itemize}
\item No assignment statements - variables values can never change once given a value.
\item Function calls have no side-effect and will only compute the results - this makes order of execution irrelevant, as due to the lack of side-effects the logical point in \textit{time} when the function is calculated within the program-execution does not matter.
\item higher-order functions
\item lazy evaluation
\item Looping is achieved using recursion, mostly through the use of the general fold or the more specific map.
\item Pattern-matching
\end{itemize}

This alone does not really explain the \textit{real} advantages of functional programming and one must look for better motivations using functional programming languages. One motivation is given in \cite{Hughes1989} which is a great paper explaining to non-functional programmers what the significance of functional programming is and helping functional programmers putting functional languages to maximum use by showing the real power and advantages of functional languages. The main conclusion is that \textit{modularity}, which is the key to successful programming, can be achieved best using higher-order functions and lazy evaluation provided in functional languages like Haskell. \cite{Hughes1989} argues that the ability to divide problems into sub-problems depends on the ability to glue the sub-problems together which depends strongly on the programming-language and \cite{Hughes1989} argues that in this ability functional languages are superior to structured programming.

TODO: comparison of functional and object-oriented programming. My points are:
\begin{itemize}
\item The way state can be changed and treated - distributed over multiple objects - is often very difficult to understand.
\item Inheritance is a dangerous thing if not used with care because inheritance introduces very strong dependencies which cannot be changed during runtime anymore.
\item Objects don't compose very well: \url{http://zeroturnaround.com/rebellabs/why-the-debate-on-object-oriented-vs-functional-programming-is-all-about-composition/}
\item (Nearly) impossible to reason about programs
\end{itemize}

In conclusion the upsides of functional programming as opposed to OO are:
\begin{itemize}
\item Much more explicit flow of data \& control
\item Much better compose-able
\item Much better parallelism
\end{itemize}

\section{Related Research}
Tim Sweeney, CTO of Epic Games gave an invited talk about how "future programming languages could help us write better code" by "supplying stronger typing, reduce run-time failures;  and the need for pervasive concurrency support, both implicit and explicit, to effectively exploit the several forms of parallelism present in games and graphics." \cite{Sweeney2006}. Although the fields of games and agent-based simulations seem to be very different in the end, they have also very important similarities: both are simulations which perform numerical computations and update objects - in games they are called "game-objects" and in abm they are called agents but they are in fact the same thing - in a loop either concurrently or sequential. His key-points were:

\begin{itemize}
\item Dependent types as the remedy of most of the run-time failures.
\item Parallelism for numerical computation: these are pure functional algorithms, operate locally on mutable state. Haskell ST, STRef solution enables encapsulating local heaps and mutability within referentially transparent code.
\item Updating game-objects (agents) concurrently using STM: update all objects concurrently in arbitrary order, with each update wrapped in atomic block - depends on collisions if performance goes up.
\end{itemize}

\section{Related Research}

\cite{schneider_towards_2012} and \cite{vendrov_frabjous:_2014} present a domain-specific language for developing functional reactive agent-based simulations. This language called FRABJOUS is very human readable and easily understandable by domain-experts. It is not directly implemented in FRP/Haskell/Yampa but is compiled to Haskell/Yampa code which they claim is also readable. This is the direction we want to head but we don't want this intermediate step but look for how a most simple domain-specific language embedded in Haskell would look like. In this paper we explicitly dive deep into FRP And Yampa and see how we can combine the best of both.

\section{Implementation: General Considerations}
All implementations of ABM/S models must solve two problems:

\begin{enumerate}
\item Agent-Implementation: how can the Agent in the model-specification be implemented?
\item Simulation-Stepping: which kind of stepping is required or best suited for the given model?
\end{enumerate}

Of course both problems influence each other and cannot be considered separated from each other.

-> Java: supports global data => suitable to implement global decisions: implementing global-time, sequential iteration with global decisions
	-> Haskell: has no global data => local decisions (has support for global data through STM/IO but then looses very power?) => implementing global-time, parallel iteration with local-decisions. 
		-> Haskell STM solution => implementing concurrent version using STM? but this is very complicated in its own right but utilizing STM it will be much more easier than in java
	-> Scala: mixed, can do both => implementing local time with random iteration and local decisions
	
\subsubsection{Agent-Implementation}
This is the process of implementing the behaviour of the Agent as specified in the model. Although there are various kinds of Agent-Models like BDI but the basic principle is always the same: sense the environment, process messages, execute actions: change environment, send messages. According to \cite{wooldridge_introduction_2009} and also influenced by Actors from \cite{agha_actors:_1986} one can abstract the abilities in each step of an Agent to be the following:

\begin{enumerate}
\item Process received messages
\item Create new Agents
\item Send messages to other Agents
\item Sense (read) the environment
\item Influence (write) the environment
\end{enumerate}

The difference between communicating with the environment and other agents is that the communication with the former one is synchronized, persists and is visible immediately (at least by the agent performing the action) whereas the communication with other agents is asynchronous.

\subsubsection{Semantics of a Simulation}
When one has implemented the model of behaviour of an Agent one needs to bring the whole simulation to life by enabling the Agents to execute their behaviour in a recurring fashion. This allows an Agent to change the environment by actions and react to changes in the environment, either by other Agents or the environment itself thus resulting in a feedback-loop. There are two ways of looking and implementing such feedback-loops. 

\paragraph{Global Stepping}
In this case the simulation is iterated in global steps where in each step each Agent is updated by running its behaviour.

\begin{enumerate}
\item \textbf{Sequential} - update one Agent after another. We assume that, given the updates are done in order of the index $i 1 to n$, then Agents $a_{n>i}$ see the updated agent-state / influence on the environment of agent $a_i$. Note that if this is not the case we would end up in the parallel-case (see next) \textit{independent} whether it is in fact running in parallel or not. For breaking deterministic ordering which could result in giving an Agent an advantage (e.g. having more information towards the end of the step) one could implement a random-walk in each step but this does not fundamentally change this approach. Also if one thinks the simulation continuously, where each step is just a very small update like in Heroes \& Cowards, then the random ordering should not change anything fundamental as no agent has real information-benefit over others as there is continuous iteration thus the agent once ahead is then behind. TODO: maybe need to make more formal

\item \textbf{Parallel} - update all Agents in parallel. This case is obviously only possible if the agents cannot interfere with each other or the environment through shared state. In this case it will make no difference how we iterate over the agents, the outcome \textit{has to be} the same - it is event-ordering invariant as all events/updates happen \textit{virtually} at the \textit{same time}. Haskell is a strong proponent of this implementation-technique.

\item \textbf{Concurrent} - update all Agents concurrently. In this case the agents run in parallel but share some state which access has to be synchronized thus introducing real random event-orderings which may or may not be desirable in the given simulation model. Can be implemented in both Java and Haskell.
\end{enumerate}

\paragraph{Local Stepping}
In this case there is no global iteration over steps but all the Agents run in parallel, doing local stepping and communicate with each other either through shared state or messages. Note that this does not impose any specific ordering of the update and can thus regarded to be real random due to its concurrent nature. It is possible to simulate the global-stepping methods from above by introducing some global locking forcing the agents into lock-step. This is the approach chosen for Scala \& Actors.

\bigskip 

The following table gives an overview of the methods presented above. Real Randomness identifies methods which produce a random ordering of their events due to their implicit workings (e.g.  concurrency) as opposed to explicit implementation (e.g random-walk of agents using a random-number generator).

TODO: add SEQ with time-advance
TODO: what about then PAR with time-advance for all? this is not logical

TODO: add actors LT (which does the time-update internal instead of being distributed by the global simulation): then again we can argue that actors ST is the same as concurrency!

\begin{table}[H]
	\center
	\begin{tabular}{ c | c | c | c | c | c }
		\textbf{Name} & \textit{Time} & \textit{Order} & \textit{Decisions} & \textit{Non-Deterministic} & \textit{Type}\\
		\hhline{=|=|=|=|=|=}
	    SEQ & Global & Sequential & Global & No & Continuous \\ 
	    \hline
	    PAR & Global & Parallel & Local & No & Discrete \\ 
	    \hline
	    CONC & Global & Concurrent & Global & Yes & Continuous \\ 
	    \hline
	    Actors ST & Global & Random & Local & Yes & Continuous \\ 
	    \hline
	    Actors RT & Local & Random & Local & Yes & Continuous \\ 
	\end{tabular}
	\caption{Summary of simulation-stepping methods.}
\end{table}



note that different types of update-strategies amount to different types of simulation . all can be used in continuous but discrete is in parallel only

semantics in interface: if the simultion is discrete, use queueMsg, if it is continuous, use sendMsg

central question is: should we hide the semantics from the sgent by providing a single interface e.g. sendMsg and make the semantic explicit only when executing the simulation or should we have different interfaces for explicit semantics e.g. sendMsg and queueMsg? the point is: when we want parallel semantics where all updates happen at once we can also run them technically in parallel. i feel we should make this explicit thus providing a queueMsg which will deliver it at the end of the iteration

global, concurrent = parallel with iterative updates where the following agents see updates. but random ordering introduced due to sync and scheduling

Each of the above presented methods imposes a different kind of event-ordering and thus all will obviously result in different \textit{absolute} simulation results. The point here is that when using ABM/S to study a system one is not interested in individual runs but in replications due to randomness and whether the system shows some emergent behaviour or not. Thus one can ask the question whether the emergent behaviour of a simulation is stable under event-ordering or not. TODO: I have no clue how to show that other than by simulations, this is also a limitation of simulations: just because it does not show up in a run it does not mean that it isn't there, just that it is unlikely - also the reverse is true: just because the emergent behaviour was there in the last n runs, does not mean it is ALWAYS there. For this we need different, more formal methods. But then again, if the level of complexity is too high we cannot solve such systems in closed form and must again fall back to simulation.

\section{Implementation in Haskell}
It is important to prevent oneself to implement an object-oriented approach in Haskell because then we would loose the power of pure functional programming. Thus the main design-considerations were:

\begin{enumerate}
\item Implement all 4 approaches mentioned in General Implementation for maximum flexibility for users of the library so they can choose which semantics of the simulation they want.
\item Pure simulation to keep up the ability to reason about it. This means to avoid running in IO Monad at all costs (except for the main-loop).
\item Utilizing STM which allows very flexible handling of state and allows running.
\item Run inside Yampa/Dunai to leverage the EDSL and continuous/discrete time-implementation.
\end{enumerate}

take haskell, add yampa and dunai and implement ActorModel on top using STM => have an ABM library in Haskell, put on hackage. NO IO!

\subsection{Utilizing STM}
with stm one can determine when everything is visible: at the end of the step or after each agent update. but this applies only to parallelism! if running agents in parallel they must execute atomically, otherwise the probability for concurrent read/writes goes to 100\% also as the number of agents goes up. if no parallelism then one final atomically at the end of each step enough. but in any case: using STM leads to the effect that agents see later updates. so really 2 cases where STM is of use in haskell: global, sequential and global concurrent

how can we implement true parallelism? can we use STM somehow or do we need local mailboxes?
\subsection{Yampa}

\subsection{Dunai}

\subsection{Putting it all together}

\section{Prototyping}

\graphicspath{{./fig/}}	%specifying the folder for the figures

The coordinates calculated by the agents are \textit{virtual} ones ranging between 0.0 and 1.0. This prevents us from knowing the rendering-resolution and polluting code which has nothing to do with rendering with these implementation-details. Also this simulation could run without rendering-output or any rendering-frontend thus sticking to virtual coordinates is also very useful regarding this (but then again: what is the use of this simulation without any visual output=


\subsection{Reasoning}
Allowing to reason about a program is one of the most interesting and powerful features of a Haskell-program. Just by looking at the types one can show that there is no randomness in the simulation \textit{after} the random initialization, which is not slightest possible in the case of a Java, Scala, ReLogo or NetLogo solution. Things we can reason about just by looking at types:

\begin{itemize}
\item Concurrency involved?
\item Randomness involved?
\item IO with the system (e.g. user-input, read/write to file(s),...) involved?
\item Termination?
\end{itemize}

This all boils down to the question of whether there are \textit{side-effects} included in the simulation or not.

What about reasoning about the termination? Is this possible in Haskell? Is it possible by types alone? My hypothesis is that the types are an important hint but are not able to give a clear hint about termination and thus we we need a closer look at the implementation. In dependently-typed programming languages like Agda this should be then possible and the program is then also a 
proof that the program itself terminates.

reasoning about Heros \& Cowards: what can we deduce from the types? what can we deduce from the implementation?\\

Compare the pure-version (both Yampa and classic) with the IO-version of haskell: we loose basically all power to reason by just looking at the types as all kind of side-effects are possible when running in the IO-Monad.

in haskell pure version i can guarantee by reasoning and looking at the types that the update strategy will be simultaneous deterministic. i cant do that in java

TODO: implement haskell-version with shared-state (STM primitives) without using IO
TODO: implement haskell-version which defines abstract types for the simulation


\subsubsection{The type of a Simulation}
the type of a simulation: try to define the most general types of a simulation and then do reasoning about it

simulation :: Model -> Double -> Int -> [Model]

step :: Model -> Double -> Model

TODO: can we say something about the methods Model can/must/should support?

\subsection{Debugging \& Testing}
Because functions compose easier than classes \& objects (TODO: we need hard claims here, look for literature supporting this thesis or proof it by myself) it is also much easier to debug \textit{parts} of the implementation e.g. the rendering of the agents without any changes to the system as a whole - just the main-loop has do be adopted. Then it is very easy to calculate e.g. only one iteration and to freeze the result or to manually create agents instead of randomly create initial ones.

TODO: quickcheck \cite{claessen_quickcheck:_2000}


\subsection{Lazy Evaluation}
can specify to run the simulation for an unlimited number of steps but only the ones which are required so far are calculated.

\subsection{Performance}
Java outperforms Haskell implementation easily with 100.000 Agents - at first not surprising because of in-place updates of friend and enemies and no massive copy-overhead as in haskell. But look WHERE exactly we loose / where the hotspots are in both solutions. 1000.000 seems to be too much even for the Java-implementation.

\subsection{Numerical Stability}
The agents in the Java-implementation collapsed after a given number of iterations into a single point as during normalization of the direction-vector the length was calculated to be 0. This could be possible if agents come close enough to each other e.g. in the border-worldtype it was highly probable after some iterations when enough agents have assembled at the borders whereas in the Wrapping-WorldType it didn't occur in any run done so far. \\
In the case of a 0-length vector a division by 0  resulting in NaN which \textit{spread} through the network of neighbourhood as every agent calculated its new position it got \textit{infected} by the NaN of a neighbour at some point. The solution was to simply return a 0-vector instead of the normalized which resulted in no movement at all for the current iteration step of the agent. 


\subsection{Update-Strategies}
\begin{enumerate}
\item All states are copied/frozen which has the effect that all agents update their positions \textit{simultaneously}
\item Updating one agent after another utilizing aliasing (sharing of references) to allow agents updated \textit{after} agents before to see the agents updated before them. Here we have also two strategies: deterministic- and random-traversal.
\item Local observations: Akka
\end{enumerate}

\subsection{Different results with different Update-Strategies?}
Problem: the following properties have to be the same to reproduce the same results in different implementations: \\

Same initial data: Random-Number-Generators
Same numerical-computation: floating-point arithmetic
Same ordering of events: update-strategy, traversal, parallelism, concurrency

\begin{itemize}
\item Same Random-Number Generator (RNG) algorithm which must produce the same sequence given the same initial seed.
\item Same Floating-Point arithmetic
\item Same ordering of events: in Scala \& Actors this is impossible to achieve because actors run in parallel thus relying on os-specific non-deterministic scheduling. Note that although the scheduling algorithm is of course deterministic in all os (i guess) the time when a thread is scheduled depends on the current state of the system which can change all the time due to \textit{very} high number of variables outside of influence (some of the non-deterministic): user-input, network-input, .... which in effect make the system appear as non-deterministic due to highly complex dependencies and feedback.
\item Same dt sequence => dt MUST NOT come from GUI/rendering-loop because gui/rendering is, as all parallelism/concurency subject to performance variations depending on scheduling and load of OS.
\end{itemize}

It is possible to compare the influences of update-strategies in the Java implementation by running two exact simulations (agentcount, speed, dt, herodistribution, random-seed, world-type) in lock-step and comparing the positions of the agent-pairs with same ids after each iteration. If either the x or y coordinate is no equal then the positions are defined to be \textit{not} equal and thus we assume the simulations have then diverged from each other. \\
It is clear that we cannot compare two floating-point numbers by trivial == operator as floating-point numbers always suffer rounding errors thus introducing imprecision. What may seem to be a straight-forward solution would be to introduce some epsilon, measuring the absolute error: abs(x1 - x2) > epsilon, but this still has its pitfalls. The problem with this is that, when number being compared are very small as well then epsilon could be far too big thus returning to be true despite the small numbers are compared to each other quite different. Also if the numbers are very large the epsilon could end up being smaller than the smallest rounding error, so that this comparison will always return false. The solution would be to look at the \textit{relative error}: abs((a-b)/b) < epsilon. \\
The problem of introducing a relative error is that in our case although the relative error can be very small the comparison could be determined to be different but looking in fact exactly the same without being able to be distinguished with the eye. Thus we make use of the fact that our coordinates are virtual ones, always being in the range of [0..1] and are falling back to the measure of absolute error with an epsilon of 0.1. Why this big epsilon? Because this will then definitely show us that the simulation is \textit{different}. \\

The question is then which update-strategies lead to diverging results. The hypothesis is that when doing simultaneous updates it should make no difference when doing random-traversal or deterministic traversal => when comparing two simulations with simultaneous updates and all the same except first random- and the other deterministic traversal then they should never diverge. Why? Because in the simultaneous updates there is no ordering introduce, all states are frozen and thus the ordering of the updates should have no influence, \textit{both simulations should never diverge, \textbf{independent how dt and epsilon are selected}}. \\
Do the simulation-results support the hypothesis? Yes they support the hypothesis - even in the worst cast with very large dt compared to epsilon (e.g. dt = 1.0, epsilon = 1.0-12)

The 2nd hypothesis is then of course that when doing consecutive updates the simulations will \textit{always} diverge independent when having different traversal-strategies. \\
Simulations show that the selection of \textit{dt} is crucial in how fast the simulations diverge when using different traversal-strategies. The observation is that \textit{The larger dt the faster they diverge and the more substantial and earlier the divergence.}. Of course it is not possible to proof using simulations alone that they will always diverge when having different traversal-strategies. Maybe looking at the dynamics of the error (the maximum of the difference of the x and y pairs) would reveal some insight? \\

The 3rd hypothesis is that the number of agents should also lead to increased speed of divergence when having different traversal-strategies. This could be shown when going from 60 agents with a dt of 0.01 which never exceeded a global error of 0.02 to 6000 agents which after 3239 steps exceeded the absolute error of 0.1.

\subsection{Reproducing Results in different Implementations}
actors: time is always local and thus information as well. if we fall back to a global time like system time we would also fall back to real-time. anyway in distributed systems clock sync is a very non-trivial problem and inherently not possible (really?). thus using some global clock on a metalevel above/outside the simulation will only buy us more problems than it would solve us. real-time does not help either as it is never hard real time and thus also unpredictable: if one tells the actor to send itself a message after 100ms then one relies on the capability of the OS-timer and scheduler to schedule exactly after 100ms: something which is always possible thus 100ms are never hard 100ms but soft with variations.

qualitative comparison: print pucture with patterns. all implementations are able to reproduce these patterns independent from the update strategy

no need to compare individual runs and waste time in implementing RNGs, what is more interesting is whether the qualitative results are the same: does the system show the same emergent behaviour? Of course if we can show that the system will behave exactly the same then it will also exhibit the same emergent behaviour but that is not possible under some circumstances e.g. the simulation-runs of Akka are always unique and never comparable due to random event-ordering produced by concurrency \& scheduling. Also we don't have to proof the obvious: given the same algorithm, the same random-data, the same treatment of numbers and the same ordering of events, the outcome \textit{must} be the same, otherwise there are bugs in the program. Thus when comparing results given all the above mentioned properties are the same one in effect tests only if the programs contain no bugs - or the same bugs, if they \textit{are the same}. \\

Thus we can say: the systems behave qualitatively the same under different event-orderings.

Thus the essence of this boils down to the question: "Is the emergent behaviour of the system is stable under random/different/varying event-ordering?". In this case it seems to be so as proofed by the Akka implementation. In fact this is a very desirable property of a system showing emergent behaviour but we need to get much more precise here: what is an event? what is an emergent behaviour of a system? what is random-ordering of events? (Note: obviously we are speaking about repeated runs of a system where the initial conditions may be the same but due to implementation details like concurrency we get a different event-ordering in each simulation-run, thus the event-orderings vary between runs, they can be in fact be regarded as random).

\begin{figure}[H]
	\centering
  \includegraphics[width=1.0\textwidth, angle=0]{EMERGENT_PATTERN.png}
  	\caption{The emergent pattern used as criteria for qualitative comparison of implementations. Note the big green cross in the center and the smaller red crosses in each sub-sector. World-type is \textit{border} with 100.000 Agents where 25\% are Heroes.}
	\label{fig:EMERGENT_PATTERN}
\end{figure}


\subsection{Problem of RNG}
Have to behave EXACTLY The same: VERY difficult because of differing interfaces e.g. compare java to haskell RNGs.
Solution: create a deterministic RNG generating a number-stream starting from 1 and just counting up. The program should work also in this case, if not, something should be flawed!

Peer told me to implement a RNG-Trace: generate a list of 1000.0000 pre-calculated random-numbers in range of [0..1], store them in a file and read the trace in all implementations. Needs lots of implementation.

\subsection{Run-Time Complexity}
what if the number of agents grows? how does the run-time complexity of the simulation increases? Does it differ from implementation to implementation? The model is O(n) but is this true for the implementation?

\subsection{Simulation-Loops}
There are at least 2 parts to implementing a simulation: 1. implementing the logic of an agent and 2. implementing the iteration/recursion which drives the whole simulation

Classic \\
Yampa \\ TODO: use par to parallelize
Gloss \\
gloss provides means for simple simulation using simulate method. But: are all ABM systems like that?

\subsection{Agent-Representation}
Java: (immutable) Object
Haskell Classic: a struct
Haskell Yampa: a Signal-Function
Gloss: same as haskell classic
Akka: Actors

\subsection{EDSL}
simplify simulation into concise EDSL: distinguish between different kind if sims: continuous/discrete iteration on: fixed set, growing set, shrinking set, dynamic set. 

\section{Agent-Based Dynamics}
We can now run simulations of our agent-based approach and see whether they reach the SD dynamics of Figure \ref{fig:sir_sd_dynamics}. In Figure \ref{fig:sir_abs_approximating_1dt} the dynamics of a first naive attempt using 1,000 agents with $\Delta t= 1.0$ can be seen. 

\begin{figure}
\begin{center}
	\begin{tabular}{c c}
		\begin{subfigure}[b]{0.3\textwidth}
			\centering
			\includegraphics[width=1\textwidth, angle=0]{./../shared/fig/frabs/SIR_1000agents_150t_1dt_NOSS_parallel.png}
			\caption{$\Delta t = 1.0$}
			\label{fig:sir_abs_approximating_1dt}
		\end{subfigure}
    	&
		\begin{subfigure}[b]{0.3\textwidth}
			\centering
			\includegraphics[width=1\textwidth, angle=0]{./../shared/fig/frabs/SIR_1000agents_150t_05dt_NOSS_parallel.png}
			\caption{$\Delta t = 0.5$}
			\label{fig:sir_abs_approximating_05dt}
		\end{subfigure}
    	
    	\\
    	
		\begin{subfigure}[b]{0.3\textwidth}
			\centering
			\includegraphics[width=1\textwidth, angle=0]{./../shared/fig/frabs/SIR_1000agents_150t_02dt_NOSS_parallel.png}
			\caption{$\Delta t = 0.2$}
			\label{fig:sir_abs_approximating_02dt}
		\end{subfigure}
		& 
		\begin{subfigure}[b]{0.3\textwidth}
			\centering
			\includegraphics[width=1\textwidth, angle=0]{./../shared/fig/frabs/SIR_1000agents_150t_01dt_NOSS_parallel.png}
			\caption{$\Delta t = 0.1$}
			\label{fig:sir_abs_approximating_01dt}
		\end{subfigure}
	\end{tabular}
	
	\caption{Naive simulation of SIR using agent-based approach. Population Size $N$ = 1,000, contact rate $\beta = \frac{1}{5}$, infection probability $\gamma = 0.05$, illness duration $\delta = 15$ with initially 1 infected agent. Simulation run for 150 time-steps with various $\Delta t$.} 
	\label{fig:sir_abs_dynamics_naive}
\end{center}
\end{figure}

%TODO: reproducing about the same dynamics of the SD-solution (1.0 dt)
%	- super-sampling: 	contact-rate ss high, illness time-out low 
%	- agent number:		1000 vs. 10.000 agents
%	- Susceptibles making contact and infected response VS. only Infected make contact
%	- update-strat:		Sequential vs. Parallel
%	- making contact: susceptible only vs. susceptible AND infected
%	- do conversations make a difference?
%	- does a delayed switch (dSwitch) in transitions makes a difference?

Clearly something is going wrong as the dynamics do not resemble the ones of SD in any way with only 10 agents making the transition to infected to recovered. The problem is that we are running into sampling issues. TODO: explain deeper and better

\subsection{Sampling the System}
When sampling the system, the correct $\Delta t$ must be selected which depends on the highest frequency which occurs in a time-reactive function in the whole system. For example in the SIR model we want infected agents to make on average contact with $\beta = 5$ other agents per time-unit, which means with a frequency of $\frac{1}{5}$. This functionality is built on Yampas function \textit{occasionally} which behaviour we investigated under differing $\Delta t$ with the above frequency. In this investigation we simply sampled occasionally with different $\Delta t$ for a duration of $t = 1,000$ and the event-frequency of $\frac{1}{5}$. The result can be seen in Figure \ref{fig:sampling_occasionally_5evts} and is quite striking. The plot clearly shows that occasionally needs a quite high sampling frequency even for a comparatively low event-frequency, which becomes of course worse for higher event-frequencies.

The other time-reactive function which occurs in the SIR model is the timed transition from infected to recovered which occurs on average with an exponential random-distribution after $\delta = 15$. This functionality is built on a custom implementation of Yampas \textit{after} which creates an event after a time-out of the passed in time-duration drawn from an exponential random-distribution. Clearly this function has different semantics as although it also continuously emit events over time - \textit{NoEvent} before the time was hit, and \textit{Event b} after the time hit - the relevant point is that it switches to Event at some discrete point in time. This is implemented as simply adding up the $\Delta t$ until the accumulator is greater of equal than the previously drawn exponential time-out. We also investigated the behaviour of this function under varying $\Delta t$ using a time-out of $\delta = 15$. Our approach was to sample the \textit{afterExp} until an event occurs and then see when it has occurred. We run this with 10,000 replications with different random-number seeds and average the resulting times. The results can be seen in Figure \ref{fig:sampling_afterExp_5time}. The result is striking in another way: this function seems to be pretty invariant to the time-deltas, for obvious reasons: we are basically just interested in the "after"-condition of the whole semantics whereas in occasionally we are interested in the "repeatedly"-conditions. If we under-sample the \textit{afterExp} then we can be off by one $\Delta t$. If we under-sample \textit{occasionally} we keep loosing events - the less difference between $\Delta t$ and event-frequency, the more events we lose. Of course \textit{afterExp} can also be used for very short time-outs e.g. $\frac{1}{5}$. We have investigated the behaviour of this function for various $\Delta t$ as well as seen in Figure \ref{fig:sampling_afterExp_02time}. Here the result is much more striking and shows that \textit{afterExp} is vulnerable to small time-outs as well as \textit{occasionally}.  
To show that \textit{occasionally} is not vulnerable to very low frequencies of e.g. one event every 5 time-steps we plotted the behaviour of this under varying time-steps in Figure \ref{fig:sampling_occasionally_02evts}. The result shows that for low frequencies occasionally works fine with larger $\Delta t$.

\begin{figure}
\begin{center}
	\begin{tabular}{c c}
	\begin{subfigure}[b]{0.5\textwidth}
			\centering
			\includegraphics[width=.6\textwidth, angle=0]{./../shared/fig/sampling/samplingTest_occasionally_5evts.png}
			\caption{Sampling \textit{occasional} with a frequency of $\frac{1}{5}$ (average of 5 events per time-unit). The theoretical average is 5000 events within this time-frame.}
			\label{fig:sampling_occasionally_5evts}
		\end{subfigure}
		& 
		\begin{subfigure}[b]{0.5\textwidth}
			\centering
			\includegraphics[width=.6\textwidth, angle=0]{./../shared/fig/sampling/samplingTest_occasionally_02evts.png}
			\caption{Sampling \textit{occasional} with a frequency of 5 (average of 0.2 events per time-unit). The theoretical average is 200 events within this time-frame.}
			\label{fig:sampling_occasionally_02evts}
		\end{subfigure}
		
		\\
		
		\begin{subfigure}[b]{0.5\textwidth}
			\centering
			\includegraphics[width=.6\textwidth, angle=0]{./../shared/fig/sampling/samplingTest_afterExp_5time.png}
			\caption{Sampling \textit{afterExp} with an average time-out of 5.}
			\label{fig:sampling_afterExp_5time}
		\end{subfigure}
		& 
		\begin{subfigure}[b]{0.5\textwidth}
			\centering
			\includegraphics[width=.6\textwidth, angle=0]{./../shared/fig/sampling/samplingTest_afterExp_02time.png}
			\caption{Sampling \textit{afterExp} with an average time-out of 0.2.}
			\label{fig:sampling_afterExp_02time}
		\end{subfigure}
	\end{tabular}
	
	\caption{Sampling the \textit{afterExp} and \textit{occasional} functions to visualise the influence of sampling frequencies on the occurrence of the respective events. $\Delta t$ are [ 5, 2, 1, $\frac{1}{2}$, $\frac{1}{5}$, $\frac{1}{10}$, $\frac{1}{20}$, $\frac{1}{50}$, $\frac{1}{100}$ ]. The experiments for \textit{afterExp} used 10,000 replications. The experiments for \textit{occasional} ran for $t= 1,000$ with 100 replications.} 
	\label{fig:sampling_tests}
\end{center}
\end{figure}

Using these observation we run simulations with varying $\Delta t$ with $\Delta = 0.5$, $\Delta = 0.2$ and $\Delta = 0.1$ with the results visible in Figures \ref{fig:sir_abs_approximating_05dt}, \ref{fig:sir_abs_approximating_02dt} and \ref{fig:sir_abs_approximating_01dt} but still when decreasing $\Delta t$ we don't approach the SD dynamics. As previously mentioned the agent-based approach is a discrete one which means that with increasing number of agents, the discrete dynamics approximate the continuous dynamics of the SD simulation. We run further simulations with $\Delta = 0.1$ but with varying agent numbers to see the influence with the results seen in Figure \ref{fig:sir_abs_approximating}.

\begin{figure}
\begin{center}
	\begin{tabular}{c c}
		\begin{subfigure}[b]{0.3\textwidth}
			\centering
			\includegraphics[width=1\textwidth, angle=0]{./../shared/fig/frabs/SIR_100agents_150t_01dt_NOSS_parallel.png}
			\caption{100 Agents}
			\label{fig:sir_abs_approximating_100}
		\end{subfigure}
    	&
		\begin{subfigure}[b]{0.3\textwidth}
			\centering
			\includegraphics[width=1\textwidth, angle=0]{./../shared/fig/frabs/SIR_1000agents_150t_01dt_NOSS_parallel.png}
			\caption{1,000 Agents}
			\label{fig:sir_abs_approximating_1000}
		\end{subfigure}
    	
    	\\
    	
		\begin{subfigure}[b]{0.3\textwidth}
			\centering
			\includegraphics[width=1\textwidth, angle=0]{./../shared/fig/frabs/SIR_5000agents_150t_01dt_NOSS_parallel.png}
			\caption{5,000 Agents}
			\label{fig:sir_abs_approximating_5000}
		\end{subfigure}
		& 
		\begin{subfigure}[b]{0.3\textwidth}
			\centering
			\includegraphics[width=1\textwidth, angle=0]{./../shared/fig/frabs/SIR_10000agents_150t_01dt_NOSS_parallel.png}
			\caption{10,000 Agents}
			\label{fig:sir_abs_approximating_10000}
		\end{subfigure}
	\end{tabular}
	
	\caption{Varying agent numbers with same model-parameters except population size. All simulations run for 150 time-steps with $\Delta t = 0.1$}
	\label{fig:sir_abs_approximating}
\end{center}
\end{figure}

Still the dynamics of 10,000 Agents do not match the dynamics of the SD simulation perfectly. This is because as opposed to the SD simulation, which is deterministic, the agent-based approach is inherently a stochastic one as we continuously draw from random-distributions which drive our state-transitions. What we see in Figure \ref{fig:sir_abs_approximating} is then just a single run where the dynamics would result in slightly different shapes when run with a different random-number generator seed. The agent-based approach thus generates a distribution of dynamics over which ones needs to average to arrive at the correct solution. This can be done using replications in which the simulation is run with the exact same parameters multiple times but each with a different random-number generator see. The resulting dynamics are then averaged and the result is then regarded as the correct dynamics.
We have done this as can be seen in Figure \ref{fig:sir_abs_agents_repls}, using 10 replications, which matches the SD dynamics to a very satisfactory level. Note that in the replications we are using 10 initially infected agents to ensure that no simulation run will terminate too early (meaning that the disease gets extinct after a few time steps) which would offset the dynamics completely. This happens due to "unlucky" random distributions which can be repaired by introducing more initially infected agents which increases the probability of spreading the disease in the very early stage of the simulation drastically. We found that when using 10 initially infected agents in a population of 5,000 (which amounts to 0.2\%) is enough to never result in an early terminating simulation. In the case of 100 agents 10 initially infected ones might be too much and distorts the dynamics but this is irrelevant in this case. This is also a fundamental difference between SD and ABS: the dynamics of the agent-based approach can result in a wide range of scenarios which includes also the one in which the disease gets extinct in the early stages (a lucky coincidence for mankind) - this is simply not possible in the SD approach. So we can argue that ABS is much closer to reality than SD as it allows to explore alternate futures in the dynamics.

\begin{figure}
\begin{center}
	\begin{tabular}{c c}
		\begin{subfigure}[b]{0.3\textwidth}
			\centering
			\includegraphics[width=1\textwidth, angle=0]{./../shared/fig/frabs/SIR_100agents_150t_01dt_NOSS_parallel_10replications.png}
			\caption{100 Agents}
			\label{fig:sir_abs_agents_repls_100}
		\end{subfigure}
    	&
		\begin{subfigure}[b]{0.3\textwidth}
			\centering
			\includegraphics[width=1\textwidth, angle=0]{./../shared/fig/frabs/SIR_1000agents_150t_01dt_NOSS_parallel_10replications.png}
			\caption{1,000 Agents}
			\label{fig:sir_abs_agents_repls_1000}
		\end{subfigure}
    	
    	\\
    	
		\begin{subfigure}[b]{0.3\textwidth}
			\centering
			\includegraphics[width=1\textwidth, angle=0]{./../shared/fig/frabs/SIR_5000agents_150t_01dt_NOSS_parallel_10replications.png}
			\caption{5,000 Agents}
			\label{fig:sir_abs_agents_repls_5000}
		\end{subfigure}
		&
		\begin{subfigure}[b]{0.3\textwidth}
			\centering
			\includegraphics[width=1\textwidth, angle=0]{./../shared/fig/frabs/SIR_10000agents_150t_01dt_NOSS_parallel_10replications.png}
			\caption{10,000 Agents}
			\label{fig:sir_abs_agents_repls_10000}
		\end{subfigure}
	\end{tabular}
	
	\caption{Dynamics of Figure \ref{fig:sir_abs_approximating} averaged over 10 replications with initially 10 infected agents.} 
	\label{fig:sir_abs_agents_repls}
\end{center}
\end{figure}

When comparing the results of the dynamics of the agent-based approach from Figure \ref{fig:sir_abs_approximating} and Figure \ref{fig:sir_abs_agents_repls} to the SD dynamics of Figure \ref{fig:sir_sd_dynamics} it becomes apparent that by increasing the number of agents the dynamics approximate the SD dynamics with increasing accuracy. Still although using 5,000 agents and replications seem to be not enough yet, we need to increase our number of agents to 10,000

Still although using a quite small $\Delta t = 0.1$ and using replications we are nowhere close to the SD dynamics. The only option we have is to further decrease $\Delta t$. Of course performance is a big issue and it decreases as $\Delta t$ get smaller and smaller. This is because when running a simulation for a duration of $t$ and sampling it with $\Delta t$ when the steps to calculate is $\frac{t}{\Delta t}$. In each step all agents are run, messages delivered and environments folded and updated which implies that the more steps the lower the performance. If we could perform super-sampling just for the given high-frequency functions with the whole system running in lower frequency then we could achieve a substantial performance boost.

\subsection{Super-Sampling}
In Yampa there exists a function \textit{embed} which allows to run a given signal-function with provided $\Delta t$ but the problem is that this function does not really help because it does not return a signal-function. What we need is a signal-function which takes the number of super-samples \textit{n}, the signal-function \textit{sf} to sample and returns a new signal-function which performs super-sampling on it:

\begin{minted}[fontsize=\footnotesize]{haskell}
superSampling :: Int -> SF a b -> SF a [b]
\end{minted}

It does this by evaluating \textit{sf} for \textit{n} times, each with $\Delta t = \frac{\Delta t}{n}$ and the same input argument \textit{a} for all \textit{n} evaluations. At time 0 no super-sampling is done and just a single output of \textit{sf} is calculated. A list of \textit{b} is returned with length of \textit{n} containing the result of the \textit{n} evaluations of \textit{sf}. If 0 or less super samples are requested exactly one is calculated.

We ran tests super-sampling both \textit{occasionally} Figure \ref{fig:sampling_occasionally_ss_02evts}, Figure \ref{fig:sampling_occasionally_ss_5evts} and \textit{afterExp} Figure \ref{fig:sampling_afterExp_ss_5time}, Figure \ref{fig:sampling_afterExp_ss_02time}. They work the same way as above except that now $\Delta t = 1.0$ but using increasing numbers of super-samples. The results are as expected: as the number of super-samples increase, so increases the accuracy.

\begin{figure*}
\begin{center}
	\begin{tabular}{c c}
		\begin{subfigure}[b]{0.5\textwidth}
			\centering
			\includegraphics[width=.6\textwidth, angle=0]{./../shared/fig/sampling/samplingTest_occasionally_ss_02evts.png}
			\caption{Super-Sampling the \textit{occasional} function with event-frequency of 5 (average of 0.2 events per time-unit). The theoretical average is 20 event within this time-frame.}
			\label{fig:sampling_occasionally_ss_02evts}
		\end{subfigure}
	
		&
		
		\begin{subfigure}[b]{0.5\textwidth}
			\centering
			\includegraphics[width=.6\textwidth, angle=0]{./../shared/fig/sampling/samplingTest_occasionally_ss_5evts.png}
			\caption{Super-Sampling the \textit{occasional} function with event-frequency of $\frac{1}{5}$ (average of 5 events per time-unit). The theoretical average is 500 event within this time-frame.}
			\label{fig:sampling_occasionally_ss_5evts}
		\end{subfigure}

		\\
		
		\begin{subfigure}[b]{0.5\textwidth}
			\centering
			\includegraphics[width=.6\textwidth, angle=0]{./../shared/fig/sampling/samplingTest_afterExp_SS_5time.png}
			\caption{Super-Sampling the \textit{afterExp} function with average time-out of 5.}
			\label{fig:sampling_afterExp_ss_5time}
		\end{subfigure}

		&
		
		\begin{subfigure}[b]{0.5\textwidth}
			\centering
			\includegraphics[width=.6\textwidth, angle=0]{./../shared/fig/sampling/samplingTest_afterExp_SS_02time.png}
			\caption{Super-Sampling the \textit{afterExp} function with average time-out of 0.2.}
			\label{fig:sampling_afterExp_ss_02time}
		\end{subfigure}
	\end{tabular}
	
	\caption{Super-Sampling the \textit{afterExp} and \textit{occasional} functions to visualize the influence of increasing number of super-samples on the average occurrence of the respective events. The $\Delta t = 1.0$ in both cases with super-samples of [1, 2, 5, 10, 100, 1000]. The experiments for \textit{afterExp} used 10,000 replications. The experiments for \textit{occasional} ran for $t = 100$ with 100 replications.} 
	\label{fig:supersampling_tests}
\end{center}
\end{figure*}

At first this might not seem to be a real win as we still need to calculate a big number of samples every time. The big win comes though when these super-sampled signal-functions are embedded in a larger system which could run on a comparatively low frequency of $\Delta t = 1.0$. So we are then increasing the sampling-frequency just where we need it and keep the frequency low where it is not required.

We are using super-sampling in our SIR implementation to increase performance. We do this by setting $\Delta t = 1.0$ and super-sampling the relevant functions with time-semantics which are \textit{transitionAfterExp} and \textit{sendMessageOccationallySrc}. For both we provide in our EDSL versions which support super-sampling:

\begin{minted}[fontsize=\footnotesize]{haskell}
sendMessageOccasionallySrcSS :: RandomGen g => g -> Double -> Int -> MessageSource 
                                -> SF (AgentOut, e) AgentOut
                                
transitionAfterExpSS :: RandomGen g => g -> Double -> Int 
                        -> AgentBehaviour -> AgentBehaviour -> AgentBehaviour
\end{minted}

Both now take an additional parameter which determines the number of super-samples to be calculated. According to the above observations of the \textit{occasionally} and \textit{afterExp} functions which are the foundations of both of the functions we sample \textit{sendMessageOccasionallySrcSS} with 20 super-samples and \textit{transitionAfterExpSS} with 2. This will ensure that by using $\Delta t = 1.0$ we only calculate $t$ steps when running a simulation for $t$ time but that we sample our relevant functions with enough resolution to capture its frequencies. Optimally we should increase the number of super-samples for \textit{sendMessageOccasionallySrcSS} to about 100. This will result in lower performance as \textit{every} agent will perform this super-sampling. So in the end it is a struggle of performance vs. sufficiently close approximation. We define the number of super-samples in lines 29 and 32 and use the functions in line 96 and 106 of Appendix \ref{app:abs_code}.

TODO: 10.000 with SS and dt = 1.0 with ss

Unfortunately when setting $\Delta t = 1.0$ the dynamics of the agent-based approach won't approach the dynamics of the SD, despite using super-sampling as can be seen in Figure \ref{fig:sir_10000_1dt}.

\begin{figure}
\begin{center}
	\begin{tabular}{c c}
		\begin{subfigure}[b]{0.5\textwidth}
			\centering
			\includegraphics[width=.8\textwidth, angle=0]{./../shared/fig/frabs/SIR_10000agents_150t_1dt_parallel.png}
			\caption{$\Delta t = 1.0$}
			\label{fig:sir_10000_1dt}
		\end{subfigure}
	
		&
		
		\begin{subfigure}[b]{0.5\textwidth}
			\centering
			\includegraphics[width=.8\textwidth, angle=0]{./../shared/fig/frabs/SIR_10000agents_150t_01dt_parallel.png}
			\caption{$\Delta t = 0.1$}
			\label{fig:sir_10000_01dt}
		\end{subfigure}
	\end{tabular}
	
	\caption{Comparing the influence of different $\Delta t$. Both dynamics were generated with the same configuration of 10,000 agents, super-sampling enabled as described and the same model-parameters. When using $\Delta t = 1.0$, the dynamics do not match the ones of the SD approach, whereas in the case of $\Delta t = 0.1$, they can be seen as matching completely.} 
	\label{fig:sir_10000_dt_comparisons}
\end{center}
\end{figure}

When reflecting on the messaging mechanism it becomes apparent that a round-trip from sender to receiver and back takes $2 \Delta t$. A round-trip happens in our agent-based SIR approach to implement the transition from infected to susceptible - susceptible agents send \textit{Contact Susceptible} messages to random agents (except itself) where only infected agents reply with a \textit{Contact Infected} message. This means that it takes $2 \Delta t$ until a susceptible agent might get infected. This becomes an issue if we want to match the dynamics of our agent-based approach to the one of SD in which no time-delay happens - the agents act instantaneous with each other during one time-step. 
We have two solutions for this problem: either we resort to \textit{conversations} or we increase the global sampling frequency of the system which matches the \textit{message frequency} of messages which are subject to round-trips. Implementing conversations is only available in the \textit{sequential} update-strategy and is much more involved, so we followed the approach of increasing the frequency. As can be seen in Figure \ref{fig:sir_10000_01dt} when setting $t\Delta = 0.1$ the resulting dynamics are a sufficiently good approximation to the SD solution.

\section{Conclusions}
\label{sec:conclusions}

Our approach is radically different from traditional approaches in the ABS community. First it builds on the already quite powerful FRP paradigm. Second, due to our continuous time approach, it forces one to think properly of time-semantics of the model and how small $\Delta t$ should be. Third it requires to think about agent interactions in a new way instead of being just method-calls.

Because no part of the simulation runs in the IO Monad and we do not use unsafePerformIO we can rule out a serious class of bugs caused by implicit data-dependencies and side-effects which can occur in traditional imperative implementations.

Also we can statically guarantee the reproducibility of the simulation, which means that repeated runs with the same initial conditions are guaranteed to result in the same dynamics. Although we allow side-effects within agents, we restrict them to only the Random and State Monad in a controlled, deterministic way and never use the IO Monad which guarantees the absence of non-deterministic side effects within the agents and other parts of the simulation.

Determinism is also ensured by fixing the $\Delta t$ and not making it dependent on the performance of e.g. a rendering-loop or other system-dependent sources of non-determinism as described by \cite{perez_testing_2017}. Also by using FRP we gain all the benefits from it and can use research on testing, debugging and exploring FRP systems \cite{perez_testing_2017, perez_back_2017}.

\subsection*{Issues}
Currently, the performance of the system is not comparable to imperative implementations but our research was not focusing on this aspect. We leave the investigation and optimization of the performance aspect of our approach for further research.

Despite the strengths and benefits we get by leveraging on FRP, there are errors that are not raised at compile time, e.g. we can still have infinite loops and run-time errors. This was for example investigated in \cite{sculthorpe_safe_2009} where the authors use dependent types to avoid some run-time errors in FRP. We suggest that one could go further and develop a domain specific type system for FRP that makes the FRP based ABS more predictable and that would support further mathematical analysis of its properties. Furthermore, moving to dependent types would pose a unique benefit over the traditional object-oriented approach and should allow us to express and guarantee even more properties at compile time. We leave this for further research.

In our pure functional approach, agent identity is not as clear as in traditional object-oriented programming, where an agent can be hidden behind a polymorphic interface which is much more abstract than in our approach. Also the identity of an agent is much clearer in object-oriented programming due to the concept of object-identity and the encapsulation of data and methods.

We can conclude that the main difficulty of a pure functional approach evolves around the communication and interaction between agents, which is a direct consequence of the issue with agent identity. Agent interaction is straight-forward in object-oriented programming, where it is achieved using method-calls mutating the internal state of the agent, but that comes at the cost of a new class of bugs due to implicit data flow. In pure functional programming these data flows are explicit but our current approach of feeding back the states of all agents as inputs is not very general and we have added further mechanisms of agent interaction which we had to omit due to lack of space.

\section{Further Research}

\subsection{Multi-Step Conversations}
The communication in this simulation is single-step unidirectional: in each step of the simulation an agent looks at the position of the enemy and friend and updates its position, there is no conversation going on between the agent and its friend and enemy thus making it single-step and unidirectional because the whole information flow is initiated from one agent and no response is given. This is becomes kind of relaxed in the Scala implementation but is still basically unidirectional and single-step - the agents don't engage in a conversation. In many ABM/S models this is perfectly reasonable because many of the models work this way but when having e.g. bartering processes like in agent-based computational economics (ACE) where agents have a conversation with multiple asks and bids to find a price they are happy with, this method becomes obviously too restricted. \\ The author investigates exactly this problem in an additional paper, where he looks at how to implement bartering-processes in ACE using Akka and Haskell (bidirectional multi-step conversations)

\subsection{LISP}
LISP is the oldest functional programming language and the second-oldest high-level programming language, only one year younger than Fortran. It would have been very interesting to research how we can do ABM/S in LISP utilizing its \textit{homoiconicity} but that would have opened up too much complexity also because LISP, despite being a functional programming language, is too far away from both Haskell and Scala. Thus the topic of applying LISP to ABM/S is left for further research in another paper. 

\subsection{Process-Calculi}
There is a strong connection of the ideas between the Actor-Model and Process-Calculi like the Pi-Calculus (TODO: cite Milner) and research has been done on connecting both worlds (TODO: cite Agha Gul). Also because the Actor-Model is so close to Agents because it was a major inspiration for the development of Agents and thus be regarded as one way of implementing Agents, one can argue due to transitivity that Agents can be connected to Pi-Calculus as well. This would allow to formalize Agents using the algebraic power and tools developed in Pi-Calculus.

\subsection{Dependent Types}
Holy Grail: would solve a specific class of problems with types, but typing and programing becomes more complicated then - also poses problems because everything programmed in this way has to be constructive. For this paper out of focus but will look into this in the main work.

\newpage

\bibliographystyle{acm}
\bibliography{../../references/phdReferences.bib}


\end{document}