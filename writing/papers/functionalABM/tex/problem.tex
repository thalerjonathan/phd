\section{The Problem}
In the paper of \cite{huberman_evolutionary_1993} the authors showed that the results of a simulation reported in \cite{nowak_evolutionary_1992} depend on a a very specific way of iterating this simulation and show that the beautiful patterns as reported by \cite{nowak_evolutionary_1992} will not form when selecting a different iteration-strategy. Thus the implication is that when one builds simulation-models it is important to select a correct iteration-strategy \textit{which reflects and supports the corresponding semantics of the model}. We find that this awareness is yet still under-represented in the literature about simulation, more precisely the Agent-Based Modelling \& Simulation and is lacking a systematic treatment. Thus it is our aim in this paper to provide such a systematic treatment by
\begin{itemize}
	\item Presenting all the strategies which are possible to iterate a simulation 
	\item Developing a systematic terminology of talking about them
	\item Give the philosophical meaning of each strategy
	\item Comparing the 3 programming languages Java, Haskell and Scala in regard of their suitability to implement each of these strategies
	\item Show how reasoning about the strategies can be done using Haskell (maybe this is gonna be a separate paper)
\end{itemize}
