%%%%%%%%%%%%%%%%%%%%%%%%%%%%%%%%%%%%%%%%%
% University/School Laboratory Report
% LaTeX Template
% Version 3.1 (25/3/14)
%
% This template has been downloaded from:
% http://www.LaTeXTemplates.com
%
% Original author:
% Linux and Unix Users Group at Virginia Tech Wiki 
% (https://vtluug.org/wiki/Example_LaTeX_chem_lab_report)
%
% License:
% CC BY-NC-SA 3.0 (http://creativecommons.org/licenses/by-nc-sa/3.0/)
%
%%%%%%%%%%%%%%%%%%%%%%%%%%%%%%%%%%%%%%%%%

%----------------------------------------------------------------------------------------
%	PACKAGES AND DOCUMENT CONFIGURATIONS
%----------------------------------------------------------------------------------------

\documentclass[twocolumn]{article}

\usepackage[utf8]{inputenc}
\usepackage{graphicx} % Required for the inclusion of images
%\usepackage{natbib} % Required to change bibliography style to APA
\usepackage{amsmath} % Required for some math elements 
\usepackage{glossaries}
\usepackage[toc,page]{appendix}
\usepackage[autostyle=true]{csquotes}
\usepackage{hyperref}
\usepackage{amssymb}
\usepackage{caption} 
\usepackage{hhline}
\usepackage{float}
\usepackage{listings}

\setlength\parindent{0pt} % Removes all indentation from paragraphs

%\usepackage{times} % Uncomment to use the Times New Roman font

%----------------------------------------------------------------------------------------
%	DOCUMENT INFORMATION
%----------------------------------------------------------------------------------------

\title{The Art of Iterating:\\Update-Strategies in Agent-Based Simulations} % Title
 
\author{Jonathan \textsc{Thaler}} % Author name

\date{\today} % Date for the report

\begin{document}
%
\maketitle % Insert the title, author and date

% If you wish to include an abstract, uncomment the lines below
\begin{abstract}
When developing a model for an Agent-Based Simulation (ABS) it is of very importance to select the right update-strategy for the agents to produce the desired results. In this paper we develop a systematic treatment of all general update-strategies in ABS and discuss their philosophical interpretation and meaning. Further we discuss the suitability of the very three different programming languages Java, Scala with Actors and Haskell to implement each of the update-strategies. Thus this papers contribution is the development of a general terminology of update-strategies and their implementation issues in various kinds of programming languages.
\end{abstract}

%*******************************************************************************
%*********************************** First Chapter *****************************
%*******************************************************************************

\chapter{Introduction}  %Title of the First Chapter
I noticed that it is pretty hard to convince an agent-based economics specialist who is not a computer scientist about a pure functional approach. My conjecture is that the implementation technique and method does not matter much to them because they have very little knowledge about programming and are almost always self-taught - they don't know about software-engineering, nothing about proper software-design and architecture, nothing about software-maintenance, nothing about unit-testing,... In the end they just "hack" the simulation in whatever language they are able to: C++, Visual Basic, Java or toolboxes like Netlogo. For them it is all about to \textit{get things done somehow} and not to get things done the right way or in a beautiful way - the way and the method doesn't matter, its just a necessary evil which needs to be done. Thus if functional programming could make their lives easier, then they will definitely welcome it. But functional programming is, i think, harder to learn and harder to understand - so one needs to provide an abstraction through EDSL. So I REALLY need to come up with convincing arguments why to use pure functional approaches in ACE THEY can understand, otherwise I will be lost and not heard (not published,...). \\

What ACE economists care for:

\begin{itemize}
\item Very: Qualitative modelling with quantitative results
\item Yes: Easy reproducibility
\item Likely: Reasoning about convergence?
\item Likely: EDSL
\end{itemize}

My contributions are: pure functional framework, functional agent-model for market-simulations, EDSL for market-simulations, qualitative / implicit modelling with quanitative results, reasoning in my framework about convergence \\

IDEA: could I develop non-causal modelling (models are expressed in terms of non-directed equations, modelled in signal-relations) to allow for qualitative modelling for the agent-based economists? See hybrid modelling paper of Yampa. \textbf{THIS WOULD BE A HUGE NOVEL CONTRIBUTION TO ACE ESPECIALLY WHEN COMBINED WITH AN EDSL AND PROVIDING FULL REFERENTIAL TRANSPARENCY TO KEEP THE ABILITY TO REASON ABOUT CONVERGENCE}. This should be covered in the "EDSL"-paper.

TODO: maybe i should really focus only on market models? otherwise too much? \\

central novelty of my PhD: model specification = runnable code. possible through EDSL. but only in specific subfield of ACE: market-models. need a functional description of the model, then translate it to model specification in EDSL and then run it to see dynamics. But: model specification moves closer to functional programming languages. \\

another novelty approach: model specification through qualitative instead of quantiative approaches. is this possible? \\

WHY FUNCTIONAL? "because its the ultimate approach to scientific computing": fewer bugs due to mutable state (why? is thos shown obkectively by someone?), shorter (again as above, productivity), more expressive and closer to math, EDSL, EDSL=model=simulation, better parallelising due to referental transparency, reasoning \\

scientific results need to be reproduced, especially when they have high impact. a more formal approach of specifying the model and the simulation (model=simulation) could lead to easier sharing and easier reporduction without ambigouites \\

pure functional agent-model \& theory, EDSL framework in Haskell for ACE

\begin{enumerate}
\item Which kind of problem do we have?
\item What aim is there? Solving the problem? 
\item How the aim is achieved by enumerating VERY CLEAR objectives.
\item What the impact one expects (hypothesis) and what it is (after results).
\end{enumerate}

Note: It is not in the interest of the researcher to develop new economic theories but to research the use of functional methods (programming and specification) in agent-based computational economics (ACE).

NOTE: Get the reader’s attention early in the introduction: motivation, significance, originality and novelty.

\section{Methods}
Methods need to be selected to implement the simulations. Special emphasis will be put on functional ones which will then be compared to established methods in the field of ABM/S and ACE. \\

Claim: non-programming environments are considered to be not powerful enough to capture the complexity of ACE implementations thus a programming approach to ACE will be always required.

\section{Scenarios}
To apply and test functional methods in ACE, four scenarios of ACE are selected and then the methods applied and compared with each other to see how each of them perform in comparison. The 4 selected scenarios represent a selection of the challenges posed in ACE: from very abstract ones to very operational ones.

\section{Comparison}
Each of the selected scenarios is then implemented using the selected methods where each solution is then compared against the following criteria: 

\begin{enumerate}
\item suitability for scientific computation
\item robustness
\item error-sources
\item testability
\item stability
\item extendability
\item size of code
\item maintainability
\item time taken for development
\item verification \& correctness
\item replications \& parallelism
\item EDSL
\end{enumerate}

This will then allow to compare the different methods against each other and to show under which circumstances functional methods shine and when they should not be used.

\section{Agent-Based Modelling and Simulation (ABM/S)}
ABM/S is a method of modelling and simulating a system where the global behaviour may be unknown but the behaviour and interactions of the parts making up the system is of knowledge (Wooldrige, M. (2009). An Introduction to MultiAgent Systems. John Wiley & Sons). Those parts, called agents, are modelled and simulated out of which then the aggregate global behaviour of the whole system emerges. Thus the central aspect of ABM/S is the concept of an Agent which can be understood as a metaphor for a pro-active unit, able to spawn new Agents, and interacting with other Agents in a network of neighbours by exchange of messages. The implementation of Agents can vary and strongly depends on the programming language and the kind of domain the simulation and model is situated in.

\section{Agent-Based Economics (ACE)}
According to Leigh Tesfatsion (Tesfatsion, L. (2006). Agent-based computational economics: A constructive approach to economic theory. In Tesfatsion, L. and Judd, K. L., editors, Handbook of Computational Economics, volume 2, chapter 16, pages 831–880. Elsevier, 1 edition.), one of the leading figures, ACE is "[...] computational modelling of economic processes (including whole economies) as open-ended dynamic systems of interacting agents." - thus lending perfectly to the use of ABM/S as already the name suggests. Whereas classical economic models fall short by only looking at the average, pure rational, individual interacting in anonymous markets, the ACE approach looks at heterogeneous, non-rational individuals interacting with each other in networks (Kirman, A. (2010). Complex Economics: Individual and Collective Rationality. Routledge, London ; New York, NY.). Thus ACE can be understood as a combination of computer-science, cognitive/social science and evolutionary economics.

\section{Functional programming}
TODO: read \cite{Backus1978}

The state-of-the-art approach to implementing Agents are object-oriented methods and programming as the metaphor of an Agent as presented above lends itself very naturally to object-orientation (OO). The author of this thesis claims that OO in the hands of inexperienced or ignorant programmers is dangerous, leading to bugs and hardly maintainable and extensible code. The reason for this is that OO provides very powerful techniques of organising and structuring programs through Classes, Type Hierarchies and Objects, which, when misused, lead to the above mentioned problems. Also major problems, which experts face as well as beginners are 1. state is highly scattered across the program which disguises the flow of data in complex simulations and 2. objects don’t compose as well as functions. The reason for this is that objects always carry around some internal state which makes it obviously much more complicated as complex dependencies can be introduced according to the internal state.
All this is tackled by (pure) functional programming which abandons the concept of global state, Objects and Classes and makes data-flow explicit. This then allows to reason about correctness, termination and other properties of the program e.g. if a given function exhibits side-effects or not. Other benefits are fewer lines of code, easier maintainability and ultimately fewer bugs thus making functional programming the ideal choice for scientific computing and simulation and thus also for ACE. A very powerful feature of functional programming is Lazy evaluation. It allows to describe infinite data-structures and functions producing an infinite stream of output but which are only computed as currently needed. Thus the decision of how many is decoupled from how to (Hughes, J. (1989). Why functional programming matters. Comput. J., 32(2):98–107.).
The most powerful aspect using pure functional programming however is that it allows the design of embedded domain specific languages (EDSL). In this case one develops and programs primitives e.g. types and functions in a host language (embed) in a way that they can be combined. The combination of these primitives then looks like a language specific to a given domain, in the case of this thesis ACE. The ease of development of EDSLs in pure functional programming is also a proof of the superior extensibility and composability of pure functional languages over OO (Henderson P. (1982). Functional Geometry. Proceedings of the 1982 ACM Symposium on LISP and Functional Programming.).
One of the most compelling example to utilize pure functional programming is the reporting of Hudak (Hudak P., Jones M. (1994). Haskell vs. Ada vs. C++ vs. Awk vs. ... An Experiment in Software Prototyping Productivity. Department of Computer Science, Yale University.)  where in a prototyping contest of DARPA the Haskell prototype was by far the shortest with 85 lines of code. Also the Jury mistook the code as specification because the prototype did actually implement a small EDSL which is a perfect proof how close EDSL can get to and look like a specification.

Functional languages can best be characterized by their way computation works: instead of \textit{how} something is computed, \textit{what} is computed is described. Thus functional programming follows a declarative instead of an imperative style of programming. The key points are:
\begin{itemize}
\item No assignment statements - variables values can never change once given a value.
\item Function calls have no side-effect and will only compute the results - this makes order of execution irrelevant, as due to the lack of side-effects the logical point in \textit{time} when the function is calculated within the program-execution does not matter.
\item higher-order functions
\item lazy evaluation
\item Looping is achieved using recursion, mostly through the use of the general fold or the more specific map.
\item Pattern-matching
\end{itemize}

This alone does not really explain the \textit{real} advantages of functional programming and one must look for better motivations using functional programming languages. One motivation is given in \cite{Hughes1989} which is a great paper explaining to non-functional programmers what the significance of functional programming is and helping functional programmers putting functional languages to maximum use by showing the real power and advantages of functional languages. The main conclusion is that \textit{modularity}, which is the key to successful programming, can be achieved best using higher-order functions and lazy evaluation provided in functional languages like Haskell. \cite{Hughes1989} argues that the ability to divide problems into sub-problems depends on the ability to glue the sub-problems together which depends strongly on the programming-language and \cite{Hughes1989} argues that in this ability functional languages are superior to structured programming.

TODO: comparison of functional and object-oriented programming. My points are:
\begin{itemize}
\item The way state can be changed and treated - distributed over multiple objects - is often very difficult to understand.
\item Inheritance is a dangerous thing if not used with care because inheritance introduces very strong dependencies which cannot be changed during runtime anymore.
\item Objects don't compose very well: \url{http://zeroturnaround.com/rebellabs/why-the-debate-on-object-oriented-vs-functional-programming-is-all-about-composition/}
\item (Nearly) impossible to reason about programs
\end{itemize}

In conclusion the upsides of functional programming as opposed to OO are:
\begin{itemize}
\item Much more explicit flow of data \& control
\item Much better compose-able
\item Much better parallelism
\end{itemize}

\section{Related Research}
Tim Sweeney, CTO of Epic Games gave an invited talk about how "future programming languages could help us write better code" by "supplying stronger typing, reduce run-time failures;  and the need for pervasive concurrency support, both implicit and explicit, to effectively exploit the several forms of parallelism present in games and graphics." \cite{Sweeney2006}. Although the fields of games and agent-based simulations seem to be very different in the end, they have also very important similarities: both are simulations which perform numerical computations and update objects - in games they are called "game-objects" and in abm they are called agents but they are in fact the same thing - in a loop either concurrently or sequential. His key-points were:

\begin{itemize}
\item Dependent types as the remedy of most of the run-time failures.
\item Parallelism for numerical computation: these are pure functional algorithms, operate locally on mutable state. Haskell ST, STRef solution enables encapsulating local heaps and mutability within referentially transparent code.
\item Updating game-objects (agents) concurrently using STM: update all objects concurrently in arbitrary order, with each update wrapped in atomic block - depends on collisions if performance goes up.
\end{itemize}

\section{Related Research}

\cite{schneider_towards_2012} and \cite{vendrov_frabjous:_2014} present a domain-specific language for developing functional reactive agent-based simulations. This language called FRABJOUS is very human readable and easily understandable by domain-experts. It is not directly implemented in FRP/Haskell/Yampa but is compiled to Haskell/Yampa code which they claim is also readable. This is the direction we want to head but we don't want this intermediate step but look for how a most simple domain-specific language embedded in Haskell would look like. In this paper we explicitly dive deep into FRP And Yampa and see how we can combine the best of both.

\section{Problem}
1 Page

\begin{itemize}
	\item describe in more technical detail what the introduction tells.
	\item \cite{huberman_evolutionary_1993} and \url{https://www.openabm.org/book/33102/54-importance-sequence-updating} only mentions synchronous and asynchronous updates but this is not precise enough and lacks subcategories
\end{itemize}

\section{Update-Strategies}
2 Pages

In this section we will present all the update-strategies which are available in ABM/S in a general form, discuss abstract details independent from programming languages, give philosophical meaning and interpretation of them and advices for selecting them for which model.

\begin{table}[H]
	\center
	\begin{tabular}{ c | c | c | c  }
		\textit{Name} & \textit{Time-Flow} & \textit{Iteration-Order} & \textit{Deterministic} \\
		\hhline{=|=|=|=}
	    \textbf{Seq} & Global & Sequential & Yes \\
	    \hline
	    \textbf{Par} & Global & Parallel & Yes \\
	    \hline
	    \textbf{Con} & Global & Concurrent & No \\
	    \hline
	    \textbf{Act} & Local & Actor & No \\
	\end{tabular}
	\caption{List of all general update-strategies in ABM/S.}
\end{table}

\subsection{Seq}
\textbf{Description:} This strategy has a global time-flow and in each time-step iterates through all the agents and updates one Agent after another. Messages sent and changes to the environment made by Agents are visible immediately. More formally, we assume that, given the updates are done in order of the index $i = [1..n]$, then Agents $a_{n>i}$ see the changes to environment and messages sent to them by Agent $a_i$. Note that there is no source of randomness and non-determinism thus rendering this strategy to be completely deterministic in each step. 

\textbf{Time-Updates:} The most straight-forward approach is to keep the time constant for each agent in one iteration. This though may seem non-logic as the actions of preceding Agents are visible to later ones but time has not changed since then. Thus if the model semantics require that time changes with every Agent we could advance for every agent by fraction of dt: agent-time = t + (ai * dt/n) where t is the current simulation time, ai the agents index, dt the amount of time the simulation will be advanced by and n the number of agents. In the end the new current time will be then tnext = tcurr + dt. Other possibilities of advancing is agent-time = t + ai * dt. in the end the new current time will be then tnext = tcurr + n * dt

\textbf{Extensions}: If the sequential iteration from 1..n imposes an advantage over the Agents further ahead or behind in the queue (e.g. if it is of benefit when making choices earlier than others in auctions or later when more information is available) then one could use random-walk iteration where in each time-step the agents are shuffled before iterated. Note that although this would introduce randomness in the model the source is a random-number generator thus reproduce-able.


\subsection{Par}
\textbf{Description:} This strategy has a global time-flow and in each time-step iterates through all the agents and updates all Agents in parallel. Messages sent and changes to the environment made by Agents are visible in the next global step. We can think about this strategy that all Agents make their moves at the same time. 

\textbf{Environment:} If one wants to change the environment in a way that it would be visible to other Agents this is regarded as a systematic error in this strategy. First it is not logical because all actions are meant to happen at the same time and also it would implicitly induce an ordering thus violating the \textit{happens at the same time} idea. Thus we require different semantics for accessing the environment in this strategy. We introduce thus a \textit{global} environment which is made up of the set of \textit{local} environments. Each local environment is owned by an Agent thus there are as many local environments as there are Agents. The semantics are then as follows: in each step all Agents can \textit{read} the global environment and \textit{read/write} their local environment. The changes to a local environment are only visible \textit{after} the local step and can be fed back into the global environment after the parallel processing of the Agents.

\textbf{Time-Updates:} Time really stays constant in this case for all Agents in one step as all updates happen really at the same \textit{virtual} time. It would make no sense to advance the time between Agents as all actions are meant to happen at the same time, without imposing any ordering amongst them.

\textbf{Semantics:} It does not make a difference if the Agents are really computed in parallel or just sequentially, due to the semantics of changes, this has the same effect. In this case it will make no difference how we iterate over the agents (sequentially, randomly), the outcome \textit{has to be} the same - it is event-ordering invariant as all events/updates happen \textit{virtually} at the \textit{same time}. Thus if one needs to have the semantics of writes on the whole (global) environment in ones model, then this strategy is not the right one and one should resort to one of the other strategies.

		
\subsection{Con}
\textbf{Description:} This strategy has a global time-flow and in each time-step iterates through all the agents and updates all Agents in parallel but all messages sent and changes to the environment are immediately visible. Thus this strategy can be understood as a mix of Seq and Par: all Agents run at the same time with actions becoming immediately visible.

\textbf{Semantics:} It is important to realize that, when running Agents in parallel which are able to see actions by others immediately, this is the very definition of concurrency: parallel execution with mutual read/write access to shared data. Of course this shared data-access needs to be synchronized which in turn will introduce event-orderings in the execution of the Agents. Thus at this point we have a source of inherent non-determinism: although we would ignore any hardware-model of concurrency at some point we need arbitration to decide which Agent gets access first to a shared resource thus arriving at non-deterministic solutions - this will become much clearer in the results-section. This has the very important influence that repeated runs with the same configuration of the Agents and the Model may lead to different results each time.


\subsection{Act}
TODO: discuss how local-time can be handled: real-time or simulation-time - its always local and not synchronized globally because then we would end up in Concurrent Strategy

in the Act-version we need to observe the agents: we need to sample them regularly => we have all the issues with sampling 

In this case there is no global iteration over steps but all the Agents run in parallel, doing local stepping and communicate with each other either through shared state or messages. Note that this does not impose any specific ordering of the update and can thus regarded to be real random due to its concurrent nature. It is possible to simulate the global-stepping methods from above by introducing some global locking forcing the agents into lock-step. This is the approach chosen for Scala \& Actors.


\cite{clinger_foundations_1981}
\cite{grief_semantics_1975}

\textbf{Semantics:} This is the most general one of all the strategies as it can emulate all the others by introducing the necessary synchronization mechanisms and Agents.


\subsection{Update-Strategies}
\begin{enumerate}
\item All states are copied/frozen which has the effect that all agents update their positions \textit{simultaneously}
\item Updating one agent after another utilizing aliasing (sharing of references) to allow agents updated \textit{after} agents before to see the agents updated before them. Here we have also two strategies: deterministic- and random-traversal.
\item Local observations: Akka
\end{enumerate}



\subsection{Different results with different Update-Strategies?}
Problem: the following properties have to be the same to reproduce the same results in different implementations: \\

Same initial data: Random-Number-Generators
Same numerical-computation: floating-point arithmetic
Same ordering of events: update-strategy, traversal, parallelism, concurrency

\begin{itemize}
\item Same Random-Number Generator (RNG) algorithm which must produce the same sequence given the same initial seed.
\item Same Floating-Point arithmetic
\item Same ordering of events: in Scala \& Actors this is impossible to achieve because actors run in parallel thus relying on os-specific non-deterministic scheduling. Note that although the scheduling algorithm is of course deterministic in all os (i guess) the time when a thread is scheduled depends on the current state of the system which can change all the time due to \textit{very} high number of variables outside of influence (some of the non-deterministic): user-input, network-input, .... which in effect make the system appear as non-deterministic due to highly complex dependencies and feedback.
\item Same dt sequence => dt MUST NOT come from GUI/rendering-loop because gui/rendering is, as all parallelism/concurency subject to performance variations depending on scheduling and load of OS.
\end{itemize}

It is possible to compare the influences of update-strategies in the Java implementation by running two exact simulations (agentcount, speed, dt, herodistribution, random-seed, world-type) in lock-step and comparing the positions of the agent-pairs with same ids after each iteration. If either the x or y coordinate is no equal then the positions are defined to be \textit{not} equal and thus we assume the simulations have then diverged from each other. \\
It is clear that we cannot compare two floating-point numbers by trivial == operator as floating-point numbers always suffer rounding errors thus introducing imprecision. What may seem to be a straight-forward solution would be to introduce some epsilon, measuring the absolute error: abs(x1 - x2) > epsilon, but this still has its pitfalls. The problem with this is that, when number being compared are very small as well then epsilon could be far too big thus returning to be true despite the small numbers are compared to each other quite different. Also if the numbers are very large the epsilon could end up being smaller than the smallest rounding error, so that this comparison will always return false. The solution would be to look at the \textit{relative error}: abs((a-b)/b) < epsilon. \\
The problem of introducing a relative error is that in our case although the relative error can be very small the comparison could be determined to be different but looking in fact exactly the same without being able to be distinguished with the eye. Thus we make use of the fact that our coordinates are virtual ones, always being in the range of [0..1] and are falling back to the measure of absolute error with an epsilon of 0.1. Why this big epsilon? Because this will then definitely show us that the simulation is \textit{different}. \\

The question is then which update-strategies lead to diverging results. The hypothesis is that when doing simultaneous updates it should make no difference when doing random-traversal or deterministic traversal => when comparing two simulations with simultaneous updates and all the same except first random- and the other deterministic traversal then they should never diverge. Why? Because in the simultaneous updates there is no ordering introduce, all states are frozen and thus the ordering of the updates should have no influence, \textit{both simulations should never diverge, \textbf{independent how dt and epsilon are selected}}. \\
Do the simulation-results support the hypothesis? Yes they support the hypothesis - even in the worst cast with very large dt compared to epsilon (e.g. dt = 1.0, epsilon = 1.0-12)

The 2nd hypothesis is then of course that when doing consecutive updates the simulations will \textit{always} diverge independent when having different traversal-strategies. \\
Simulations show that the selection of \textit{dt} is crucial in how fast the simulations diverge when using different traversal-strategies. The observation is that \textit{The larger dt the faster they diverge and the more substantial and earlier the divergence.}. Of course it is not possible to proof using simulations alone that they will always diverge when having different traversal-strategies. Maybe looking at the dynamics of the error (the maximum of the difference of the x and y pairs) would reveal some insight? \\

The 3rd hypothesis is that the number of agents should also lead to increased speed of divergence when having different traversal-strategies. This could be shown when going from 60 agents with a dt of 0.01 which never exceeded a global error of 0.02 to 6000 agents which after 3239 steps exceeded the absolute error of 0.1.

\subsection{Reproducing Results in different Implementations}
actors: time is always local and thus information as well. if we fall back to a global time like system time we would also fall back to real-time. anyway in distributed systems clock sync is a very non-trivial problem and inherently not possible (really?). thus using some global clock on a metalevel above/outside the simulation will only buy us more problems than it would solve us. real-time does not help either as it is never hard real time and thus also unpredictable: if one tells the actor to send itself a message after 100ms then one relies on the capability of the OS-timer and scheduler to schedule exactly after 100ms: something which is always possible thus 100ms are never hard 100ms but soft with variations.

qualitative comparison: print pucture with patterns. all implementations are able to reproduce these patterns independent from the update strategy

no need to compare individual runs and waste time in implementing RNGs, what is more interesting is whether the qualitative results are the same: does the system show the same emergent behaviour? Of course if we can show that the system will behave exactly the same then it will also exhibit the same emergent behaviour but that is not possible under some circumstances e.g. the simulation-runs of Akka are always unique and never comparable due to random event-ordering produced by concurrency \& scheduling. Also we don't have to proof the obvious: given the same algorithm, the same random-data, the same treatment of numbers and the same ordering of events, the outcome \textit{must} be the same, otherwise there are bugs in the program. Thus when comparing results given all the above mentioned properties are the same one in effect tests only if the programs contain no bugs - or the same bugs, if they \textit{are the same}. \\

Thus we can say: the systems behave qualitatively the same under different event-orderings.

Thus the essence of this boils down to the question: "Is the emergent behaviour of the system is stable under random/different/varying event-ordering?". In this case it seems to be so as proofed by the Akka implementation. In fact this is a very desirable property of a system showing emergent behaviour but we need to get much more precise here: what is an event? what is an emergent behaviour of a system? what is random-ordering of events? (Note: obviously we are speaking about repeated runs of a system where the initial conditions may be the same but due to implementation details like concurrency we get a different event-ordering in each simulation-run, thus the event-orderings vary between runs, they can be in fact be regarded as random).


\section{A general terminology}

\subsection{Mapping to our terminology}
Although all definitions mentioned in the related research section have different semantics and implications, our framework faithfully captures all of them:

TODO: rework: don't cite the works but recap what they mean with sync and async

When following \cite{yuxuan_agent-based_2016} we use \textit{Seq}, \textit{Par} or \textit{Con} and the synchronous updates will happen in the pro-active internal stimulus and the asynchronous by reacting to and sending messages.

when following \cite{dawson_opening_2014} we use \textit{Seq}, \textit{Par} or \textit{Con} and the synchronous updates happen by providing the same global time-delta to all agents, which means all agents have global time and advance at the same speed. The asynchronous version would be to provide each agent with a different, random time-delta, rendering the time of the agent local instead of globally synchronized. When taking this into account we can argue that \textit{Seq}, \textit{Par} and \textit{Con} are, as described above, synchronous update-strategies and only \textit{Act} is truly asynchronous due to the locality of time and information. We can though make \textit{Seq}, \textit{Par} and \textit{Con} asynchronous too if we iterate globally as specified but instead of feeding each Agent the global time which advanced by some constant delta since the previous step, we feed each Agent a local time by introducing random time-deltas drawn individually in a given range for each Agent. This would resemble the locality of the Act strategy without introducing the non-determinism.

When following \cite{huberman_evolutionary_1993} we use the \textit{Seq} for an asynchronous-time model and \textit{Par} for the synchronous one. This was demonstrated by our implementations: when running this model with \textit{Par} the beautiful patterns emerge as reported but when following the \textit{Seq} approach they wont and all Agents defect after a given number of generations.



\section{Agent-Based Dynamics}
We can now run simulations of our agent-based approach and see whether they reach the SD dynamics of Figure \ref{fig:sir_sd_dynamics}. In Figure \ref{fig:sir_abs_approximating_1dt} the dynamics of a first naive attempt using 1,000 agents with $\Delta t= 1.0$ can be seen. 

\begin{figure}
\begin{center}
	\begin{tabular}{c c}
		\begin{subfigure}[b]{0.3\textwidth}
			\centering
			\includegraphics[width=1\textwidth, angle=0]{./../shared/fig/frabs/SIR_1000agents_150t_1dt_NOSS_parallel.png}
			\caption{$\Delta t = 1.0$}
			\label{fig:sir_abs_approximating_1dt}
		\end{subfigure}
    	&
		\begin{subfigure}[b]{0.3\textwidth}
			\centering
			\includegraphics[width=1\textwidth, angle=0]{./../shared/fig/frabs/SIR_1000agents_150t_05dt_NOSS_parallel.png}
			\caption{$\Delta t = 0.5$}
			\label{fig:sir_abs_approximating_05dt}
		\end{subfigure}
    	
    	\\
    	
		\begin{subfigure}[b]{0.3\textwidth}
			\centering
			\includegraphics[width=1\textwidth, angle=0]{./../shared/fig/frabs/SIR_1000agents_150t_02dt_NOSS_parallel.png}
			\caption{$\Delta t = 0.2$}
			\label{fig:sir_abs_approximating_02dt}
		\end{subfigure}
		& 
		\begin{subfigure}[b]{0.3\textwidth}
			\centering
			\includegraphics[width=1\textwidth, angle=0]{./../shared/fig/frabs/SIR_1000agents_150t_01dt_NOSS_parallel.png}
			\caption{$\Delta t = 0.1$}
			\label{fig:sir_abs_approximating_01dt}
		\end{subfigure}
	\end{tabular}
	
	\caption{Naive simulation of SIR using agent-based approach. Population Size $N$ = 1,000, contact rate $\beta = \frac{1}{5}$, infection probability $\gamma = 0.05$, illness duration $\delta = 15$ with initially 1 infected agent. Simulation run for 150 time-steps with various $\Delta t$.} 
	\label{fig:sir_abs_dynamics_naive}
\end{center}
\end{figure}

%TODO: reproducing about the same dynamics of the SD-solution (1.0 dt)
%	- super-sampling: 	contact-rate ss high, illness time-out low 
%	- agent number:		1000 vs. 10.000 agents
%	- Susceptibles making contact and infected response VS. only Infected make contact
%	- update-strat:		Sequential vs. Parallel
%	- making contact: susceptible only vs. susceptible AND infected
%	- do conversations make a difference?
%	- does a delayed switch (dSwitch) in transitions makes a difference?

Clearly something is going wrong as the dynamics do not resemble the ones of SD in any way with only 10 agents making the transition to infected to recovered. The problem is that we are running into sampling issues. TODO: explain deeper and better

\subsection{Sampling the System}
When sampling the system, the correct $\Delta t$ must be selected which depends on the highest frequency which occurs in a time-reactive function in the whole system. For example in the SIR model we want infected agents to make on average contact with $\beta = 5$ other agents per time-unit, which means with a frequency of $\frac{1}{5}$. This functionality is built on Yampas function \textit{occasionally} which behaviour we investigated under differing $\Delta t$ with the above frequency. In this investigation we simply sampled occasionally with different $\Delta t$ for a duration of $t = 1,000$ and the event-frequency of $\frac{1}{5}$. The result can be seen in Figure \ref{fig:sampling_occasionally_5evts} and is quite striking. The plot clearly shows that occasionally needs a quite high sampling frequency even for a comparatively low event-frequency, which becomes of course worse for higher event-frequencies.

The other time-reactive function which occurs in the SIR model is the timed transition from infected to recovered which occurs on average with an exponential random-distribution after $\delta = 15$. This functionality is built on a custom implementation of Yampas \textit{after} which creates an event after a time-out of the passed in time-duration drawn from an exponential random-distribution. Clearly this function has different semantics as although it also continuously emit events over time - \textit{NoEvent} before the time was hit, and \textit{Event b} after the time hit - the relevant point is that it switches to Event at some discrete point in time. This is implemented as simply adding up the $\Delta t$ until the accumulator is greater of equal than the previously drawn exponential time-out. We also investigated the behaviour of this function under varying $\Delta t$ using a time-out of $\delta = 15$. Our approach was to sample the \textit{afterExp} until an event occurs and then see when it has occurred. We run this with 10,000 replications with different random-number seeds and average the resulting times. The results can be seen in Figure \ref{fig:sampling_afterExp_5time}. The result is striking in another way: this function seems to be pretty invariant to the time-deltas, for obvious reasons: we are basically just interested in the "after"-condition of the whole semantics whereas in occasionally we are interested in the "repeatedly"-conditions. If we under-sample the \textit{afterExp} then we can be off by one $\Delta t$. If we under-sample \textit{occasionally} we keep loosing events - the less difference between $\Delta t$ and event-frequency, the more events we lose. Of course \textit{afterExp} can also be used for very short time-outs e.g. $\frac{1}{5}$. We have investigated the behaviour of this function for various $\Delta t$ as well as seen in Figure \ref{fig:sampling_afterExp_02time}. Here the result is much more striking and shows that \textit{afterExp} is vulnerable to small time-outs as well as \textit{occasionally}.  
To show that \textit{occasionally} is not vulnerable to very low frequencies of e.g. one event every 5 time-steps we plotted the behaviour of this under varying time-steps in Figure \ref{fig:sampling_occasionally_02evts}. The result shows that for low frequencies occasionally works fine with larger $\Delta t$.

\begin{figure}
\begin{center}
	\begin{tabular}{c c}
	\begin{subfigure}[b]{0.5\textwidth}
			\centering
			\includegraphics[width=.6\textwidth, angle=0]{./../shared/fig/sampling/samplingTest_occasionally_5evts.png}
			\caption{Sampling \textit{occasional} with a frequency of $\frac{1}{5}$ (average of 5 events per time-unit). The theoretical average is 5000 events within this time-frame.}
			\label{fig:sampling_occasionally_5evts}
		\end{subfigure}
		& 
		\begin{subfigure}[b]{0.5\textwidth}
			\centering
			\includegraphics[width=.6\textwidth, angle=0]{./../shared/fig/sampling/samplingTest_occasionally_02evts.png}
			\caption{Sampling \textit{occasional} with a frequency of 5 (average of 0.2 events per time-unit). The theoretical average is 200 events within this time-frame.}
			\label{fig:sampling_occasionally_02evts}
		\end{subfigure}
		
		\\
		
		\begin{subfigure}[b]{0.5\textwidth}
			\centering
			\includegraphics[width=.6\textwidth, angle=0]{./../shared/fig/sampling/samplingTest_afterExp_5time.png}
			\caption{Sampling \textit{afterExp} with an average time-out of 5.}
			\label{fig:sampling_afterExp_5time}
		\end{subfigure}
		& 
		\begin{subfigure}[b]{0.5\textwidth}
			\centering
			\includegraphics[width=.6\textwidth, angle=0]{./../shared/fig/sampling/samplingTest_afterExp_02time.png}
			\caption{Sampling \textit{afterExp} with an average time-out of 0.2.}
			\label{fig:sampling_afterExp_02time}
		\end{subfigure}
	\end{tabular}
	
	\caption{Sampling the \textit{afterExp} and \textit{occasional} functions to visualise the influence of sampling frequencies on the occurrence of the respective events. $\Delta t$ are [ 5, 2, 1, $\frac{1}{2}$, $\frac{1}{5}$, $\frac{1}{10}$, $\frac{1}{20}$, $\frac{1}{50}$, $\frac{1}{100}$ ]. The experiments for \textit{afterExp} used 10,000 replications. The experiments for \textit{occasional} ran for $t= 1,000$ with 100 replications.} 
	\label{fig:sampling_tests}
\end{center}
\end{figure}

Using these observation we run simulations with varying $\Delta t$ with $\Delta = 0.5$, $\Delta = 0.2$ and $\Delta = 0.1$ with the results visible in Figures \ref{fig:sir_abs_approximating_05dt}, \ref{fig:sir_abs_approximating_02dt} and \ref{fig:sir_abs_approximating_01dt} but still when decreasing $\Delta t$ we don't approach the SD dynamics. As previously mentioned the agent-based approach is a discrete one which means that with increasing number of agents, the discrete dynamics approximate the continuous dynamics of the SD simulation. We run further simulations with $\Delta = 0.1$ but with varying agent numbers to see the influence with the results seen in Figure \ref{fig:sir_abs_approximating}.

\begin{figure}
\begin{center}
	\begin{tabular}{c c}
		\begin{subfigure}[b]{0.3\textwidth}
			\centering
			\includegraphics[width=1\textwidth, angle=0]{./../shared/fig/frabs/SIR_100agents_150t_01dt_NOSS_parallel.png}
			\caption{100 Agents}
			\label{fig:sir_abs_approximating_100}
		\end{subfigure}
    	&
		\begin{subfigure}[b]{0.3\textwidth}
			\centering
			\includegraphics[width=1\textwidth, angle=0]{./../shared/fig/frabs/SIR_1000agents_150t_01dt_NOSS_parallel.png}
			\caption{1,000 Agents}
			\label{fig:sir_abs_approximating_1000}
		\end{subfigure}
    	
    	\\
    	
		\begin{subfigure}[b]{0.3\textwidth}
			\centering
			\includegraphics[width=1\textwidth, angle=0]{./../shared/fig/frabs/SIR_5000agents_150t_01dt_NOSS_parallel.png}
			\caption{5,000 Agents}
			\label{fig:sir_abs_approximating_5000}
		\end{subfigure}
		& 
		\begin{subfigure}[b]{0.3\textwidth}
			\centering
			\includegraphics[width=1\textwidth, angle=0]{./../shared/fig/frabs/SIR_10000agents_150t_01dt_NOSS_parallel.png}
			\caption{10,000 Agents}
			\label{fig:sir_abs_approximating_10000}
		\end{subfigure}
	\end{tabular}
	
	\caption{Varying agent numbers with same model-parameters except population size. All simulations run for 150 time-steps with $\Delta t = 0.1$}
	\label{fig:sir_abs_approximating}
\end{center}
\end{figure}

Still the dynamics of 10,000 Agents do not match the dynamics of the SD simulation perfectly. This is because as opposed to the SD simulation, which is deterministic, the agent-based approach is inherently a stochastic one as we continuously draw from random-distributions which drive our state-transitions. What we see in Figure \ref{fig:sir_abs_approximating} is then just a single run where the dynamics would result in slightly different shapes when run with a different random-number generator seed. The agent-based approach thus generates a distribution of dynamics over which ones needs to average to arrive at the correct solution. This can be done using replications in which the simulation is run with the exact same parameters multiple times but each with a different random-number generator see. The resulting dynamics are then averaged and the result is then regarded as the correct dynamics.
We have done this as can be seen in Figure \ref{fig:sir_abs_agents_repls}, using 10 replications, which matches the SD dynamics to a very satisfactory level. Note that in the replications we are using 10 initially infected agents to ensure that no simulation run will terminate too early (meaning that the disease gets extinct after a few time steps) which would offset the dynamics completely. This happens due to "unlucky" random distributions which can be repaired by introducing more initially infected agents which increases the probability of spreading the disease in the very early stage of the simulation drastically. We found that when using 10 initially infected agents in a population of 5,000 (which amounts to 0.2\%) is enough to never result in an early terminating simulation. In the case of 100 agents 10 initially infected ones might be too much and distorts the dynamics but this is irrelevant in this case. This is also a fundamental difference between SD and ABS: the dynamics of the agent-based approach can result in a wide range of scenarios which includes also the one in which the disease gets extinct in the early stages (a lucky coincidence for mankind) - this is simply not possible in the SD approach. So we can argue that ABS is much closer to reality than SD as it allows to explore alternate futures in the dynamics.

\begin{figure}
\begin{center}
	\begin{tabular}{c c}
		\begin{subfigure}[b]{0.3\textwidth}
			\centering
			\includegraphics[width=1\textwidth, angle=0]{./../shared/fig/frabs/SIR_100agents_150t_01dt_NOSS_parallel_10replications.png}
			\caption{100 Agents}
			\label{fig:sir_abs_agents_repls_100}
		\end{subfigure}
    	&
		\begin{subfigure}[b]{0.3\textwidth}
			\centering
			\includegraphics[width=1\textwidth, angle=0]{./../shared/fig/frabs/SIR_1000agents_150t_01dt_NOSS_parallel_10replications.png}
			\caption{1,000 Agents}
			\label{fig:sir_abs_agents_repls_1000}
		\end{subfigure}
    	
    	\\
    	
		\begin{subfigure}[b]{0.3\textwidth}
			\centering
			\includegraphics[width=1\textwidth, angle=0]{./../shared/fig/frabs/SIR_5000agents_150t_01dt_NOSS_parallel_10replications.png}
			\caption{5,000 Agents}
			\label{fig:sir_abs_agents_repls_5000}
		\end{subfigure}
		&
		\begin{subfigure}[b]{0.3\textwidth}
			\centering
			\includegraphics[width=1\textwidth, angle=0]{./../shared/fig/frabs/SIR_10000agents_150t_01dt_NOSS_parallel_10replications.png}
			\caption{10,000 Agents}
			\label{fig:sir_abs_agents_repls_10000}
		\end{subfigure}
	\end{tabular}
	
	\caption{Dynamics of Figure \ref{fig:sir_abs_approximating} averaged over 10 replications with initially 10 infected agents.} 
	\label{fig:sir_abs_agents_repls}
\end{center}
\end{figure}

When comparing the results of the dynamics of the agent-based approach from Figure \ref{fig:sir_abs_approximating} and Figure \ref{fig:sir_abs_agents_repls} to the SD dynamics of Figure \ref{fig:sir_sd_dynamics} it becomes apparent that by increasing the number of agents the dynamics approximate the SD dynamics with increasing accuracy. Still although using 5,000 agents and replications seem to be not enough yet, we need to increase our number of agents to 10,000

Still although using a quite small $\Delta t = 0.1$ and using replications we are nowhere close to the SD dynamics. The only option we have is to further decrease $\Delta t$. Of course performance is a big issue and it decreases as $\Delta t$ get smaller and smaller. This is because when running a simulation for a duration of $t$ and sampling it with $\Delta t$ when the steps to calculate is $\frac{t}{\Delta t}$. In each step all agents are run, messages delivered and environments folded and updated which implies that the more steps the lower the performance. If we could perform super-sampling just for the given high-frequency functions with the whole system running in lower frequency then we could achieve a substantial performance boost.

\subsection{Super-Sampling}
In Yampa there exists a function \textit{embed} which allows to run a given signal-function with provided $\Delta t$ but the problem is that this function does not really help because it does not return a signal-function. What we need is a signal-function which takes the number of super-samples \textit{n}, the signal-function \textit{sf} to sample and returns a new signal-function which performs super-sampling on it:

\begin{minted}[fontsize=\footnotesize]{haskell}
superSampling :: Int -> SF a b -> SF a [b]
\end{minted}

It does this by evaluating \textit{sf} for \textit{n} times, each with $\Delta t = \frac{\Delta t}{n}$ and the same input argument \textit{a} for all \textit{n} evaluations. At time 0 no super-sampling is done and just a single output of \textit{sf} is calculated. A list of \textit{b} is returned with length of \textit{n} containing the result of the \textit{n} evaluations of \textit{sf}. If 0 or less super samples are requested exactly one is calculated.

We ran tests super-sampling both \textit{occasionally} Figure \ref{fig:sampling_occasionally_ss_02evts}, Figure \ref{fig:sampling_occasionally_ss_5evts} and \textit{afterExp} Figure \ref{fig:sampling_afterExp_ss_5time}, Figure \ref{fig:sampling_afterExp_ss_02time}. They work the same way as above except that now $\Delta t = 1.0$ but using increasing numbers of super-samples. The results are as expected: as the number of super-samples increase, so increases the accuracy.

\begin{figure*}
\begin{center}
	\begin{tabular}{c c}
		\begin{subfigure}[b]{0.5\textwidth}
			\centering
			\includegraphics[width=.6\textwidth, angle=0]{./../shared/fig/sampling/samplingTest_occasionally_ss_02evts.png}
			\caption{Super-Sampling the \textit{occasional} function with event-frequency of 5 (average of 0.2 events per time-unit). The theoretical average is 20 event within this time-frame.}
			\label{fig:sampling_occasionally_ss_02evts}
		\end{subfigure}
	
		&
		
		\begin{subfigure}[b]{0.5\textwidth}
			\centering
			\includegraphics[width=.6\textwidth, angle=0]{./../shared/fig/sampling/samplingTest_occasionally_ss_5evts.png}
			\caption{Super-Sampling the \textit{occasional} function with event-frequency of $\frac{1}{5}$ (average of 5 events per time-unit). The theoretical average is 500 event within this time-frame.}
			\label{fig:sampling_occasionally_ss_5evts}
		\end{subfigure}

		\\
		
		\begin{subfigure}[b]{0.5\textwidth}
			\centering
			\includegraphics[width=.6\textwidth, angle=0]{./../shared/fig/sampling/samplingTest_afterExp_SS_5time.png}
			\caption{Super-Sampling the \textit{afterExp} function with average time-out of 5.}
			\label{fig:sampling_afterExp_ss_5time}
		\end{subfigure}

		&
		
		\begin{subfigure}[b]{0.5\textwidth}
			\centering
			\includegraphics[width=.6\textwidth, angle=0]{./../shared/fig/sampling/samplingTest_afterExp_SS_02time.png}
			\caption{Super-Sampling the \textit{afterExp} function with average time-out of 0.2.}
			\label{fig:sampling_afterExp_ss_02time}
		\end{subfigure}
	\end{tabular}
	
	\caption{Super-Sampling the \textit{afterExp} and \textit{occasional} functions to visualize the influence of increasing number of super-samples on the average occurrence of the respective events. The $\Delta t = 1.0$ in both cases with super-samples of [1, 2, 5, 10, 100, 1000]. The experiments for \textit{afterExp} used 10,000 replications. The experiments for \textit{occasional} ran for $t = 100$ with 100 replications.} 
	\label{fig:supersampling_tests}
\end{center}
\end{figure*}

At first this might not seem to be a real win as we still need to calculate a big number of samples every time. The big win comes though when these super-sampled signal-functions are embedded in a larger system which could run on a comparatively low frequency of $\Delta t = 1.0$. So we are then increasing the sampling-frequency just where we need it and keep the frequency low where it is not required.

We are using super-sampling in our SIR implementation to increase performance. We do this by setting $\Delta t = 1.0$ and super-sampling the relevant functions with time-semantics which are \textit{transitionAfterExp} and \textit{sendMessageOccationallySrc}. For both we provide in our EDSL versions which support super-sampling:

\begin{minted}[fontsize=\footnotesize]{haskell}
sendMessageOccasionallySrcSS :: RandomGen g => g -> Double -> Int -> MessageSource 
                                -> SF (AgentOut, e) AgentOut
                                
transitionAfterExpSS :: RandomGen g => g -> Double -> Int 
                        -> AgentBehaviour -> AgentBehaviour -> AgentBehaviour
\end{minted}

Both now take an additional parameter which determines the number of super-samples to be calculated. According to the above observations of the \textit{occasionally} and \textit{afterExp} functions which are the foundations of both of the functions we sample \textit{sendMessageOccasionallySrcSS} with 20 super-samples and \textit{transitionAfterExpSS} with 2. This will ensure that by using $\Delta t = 1.0$ we only calculate $t$ steps when running a simulation for $t$ time but that we sample our relevant functions with enough resolution to capture its frequencies. Optimally we should increase the number of super-samples for \textit{sendMessageOccasionallySrcSS} to about 100. This will result in lower performance as \textit{every} agent will perform this super-sampling. So in the end it is a struggle of performance vs. sufficiently close approximation. We define the number of super-samples in lines 29 and 32 and use the functions in line 96 and 106 of Appendix \ref{app:abs_code}.

TODO: 10.000 with SS and dt = 1.0 with ss

Unfortunately when setting $\Delta t = 1.0$ the dynamics of the agent-based approach won't approach the dynamics of the SD, despite using super-sampling as can be seen in Figure \ref{fig:sir_10000_1dt}.

\begin{figure}
\begin{center}
	\begin{tabular}{c c}
		\begin{subfigure}[b]{0.5\textwidth}
			\centering
			\includegraphics[width=.8\textwidth, angle=0]{./../shared/fig/frabs/SIR_10000agents_150t_1dt_parallel.png}
			\caption{$\Delta t = 1.0$}
			\label{fig:sir_10000_1dt}
		\end{subfigure}
	
		&
		
		\begin{subfigure}[b]{0.5\textwidth}
			\centering
			\includegraphics[width=.8\textwidth, angle=0]{./../shared/fig/frabs/SIR_10000agents_150t_01dt_parallel.png}
			\caption{$\Delta t = 0.1$}
			\label{fig:sir_10000_01dt}
		\end{subfigure}
	\end{tabular}
	
	\caption{Comparing the influence of different $\Delta t$. Both dynamics were generated with the same configuration of 10,000 agents, super-sampling enabled as described and the same model-parameters. When using $\Delta t = 1.0$, the dynamics do not match the ones of the SD approach, whereas in the case of $\Delta t = 0.1$, they can be seen as matching completely.} 
	\label{fig:sir_10000_dt_comparisons}
\end{center}
\end{figure}

When reflecting on the messaging mechanism it becomes apparent that a round-trip from sender to receiver and back takes $2 \Delta t$. A round-trip happens in our agent-based SIR approach to implement the transition from infected to susceptible - susceptible agents send \textit{Contact Susceptible} messages to random agents (except itself) where only infected agents reply with a \textit{Contact Infected} message. This means that it takes $2 \Delta t$ until a susceptible agent might get infected. This becomes an issue if we want to match the dynamics of our agent-based approach to the one of SD in which no time-delay happens - the agents act instantaneous with each other during one time-step. 
We have two solutions for this problem: either we resort to \textit{conversations} or we increase the global sampling frequency of the system which matches the \textit{message frequency} of messages which are subject to round-trips. Implementing conversations is only available in the \textit{sequential} update-strategy and is much more involved, so we followed the approach of increasing the frequency. As can be seen in Figure \ref{fig:sir_10000_01dt} when setting $t\Delta = 0.1$ the resulting dynamics are a sufficiently good approximation to the SD solution.

\section{Conclusions}
\label{sec:conclusions}

Our approach is radically different from traditional approaches in the ABS community. First it builds on the already quite powerful FRP paradigm. Second, due to our continuous time approach, it forces one to think properly of time-semantics of the model and how small $\Delta t$ should be. Third it requires to think about agent interactions in a new way instead of being just method-calls.

Because no part of the simulation runs in the IO Monad and we do not use unsafePerformIO we can rule out a serious class of bugs caused by implicit data-dependencies and side-effects which can occur in traditional imperative implementations.

Also we can statically guarantee the reproducibility of the simulation, which means that repeated runs with the same initial conditions are guaranteed to result in the same dynamics. Although we allow side-effects within agents, we restrict them to only the Random and State Monad in a controlled, deterministic way and never use the IO Monad which guarantees the absence of non-deterministic side effects within the agents and other parts of the simulation.

Determinism is also ensured by fixing the $\Delta t$ and not making it dependent on the performance of e.g. a rendering-loop or other system-dependent sources of non-determinism as described by \cite{perez_testing_2017}. Also by using FRP we gain all the benefits from it and can use research on testing, debugging and exploring FRP systems \cite{perez_testing_2017, perez_back_2017}.

\subsection*{Issues}
Currently, the performance of the system is not comparable to imperative implementations but our research was not focusing on this aspect. We leave the investigation and optimization of the performance aspect of our approach for further research.

Despite the strengths and benefits we get by leveraging on FRP, there are errors that are not raised at compile time, e.g. we can still have infinite loops and run-time errors. This was for example investigated in \cite{sculthorpe_safe_2009} where the authors use dependent types to avoid some run-time errors in FRP. We suggest that one could go further and develop a domain specific type system for FRP that makes the FRP based ABS more predictable and that would support further mathematical analysis of its properties. Furthermore, moving to dependent types would pose a unique benefit over the traditional object-oriented approach and should allow us to express and guarantee even more properties at compile time. We leave this for further research.

In our pure functional approach, agent identity is not as clear as in traditional object-oriented programming, where an agent can be hidden behind a polymorphic interface which is much more abstract than in our approach. Also the identity of an agent is much clearer in object-oriented programming due to the concept of object-identity and the encapsulation of data and methods.

We can conclude that the main difficulty of a pure functional approach evolves around the communication and interaction between agents, which is a direct consequence of the issue with agent identity. Agent interaction is straight-forward in object-oriented programming, where it is achieved using method-calls mutating the internal state of the agent, but that comes at the cost of a new class of bugs due to implicit data flow. In pure functional programming these data flows are explicit but our current approach of feeding back the states of all agents as inputs is not very general and we have added further mechanisms of agent interaction which we had to omit due to lack of space.

\section{Further Research}
\label{sec:further_research}

We see this paper as an intermediary and necessary step towards dependent types for which we first needed to understand the potentials and limitations of a non-dependently typed pure functional approach in Haskell. Dependent types are extremely promising in functional programming as they allow us to express stronger guarantees about the correctness of programs and go as far as allowing to formulate programs and types as constructive proofs \cite{wadler_propositions_2015} which must be total by definition \cite{thompson_type_1991}, \cite{altenkirch_why_2005}, \cite{altenkirch_pi_2010}, \cite{program_homotopy_2013}. So far no research using dependent types in agent-based simulation exists at all and it is not clear whether dependent types make sense in this context. In our next paper we want to explore this for the first time and ask more specifically how we can add dependent types to our pure functional approach, which conceptual implications this has for ABS and what we gain from doing so. We plan on using Idris \cite{brady_idris_2013}, \cite{brady_type-driven_2017} as the language of choice as it is very close to Haskell with focus on real-world application and running programs as opposed to other languages with dependent types e.g. Agda and Coq which serve primarily as proof assistants.
It would be of immense interest whether we could apply dependent types to the model meta-level or not - this boils down to the question if we can encode our model specification in a dependent type way. This would allow the ABS community for the first time to reason about a proper formalisation of a model. We plan to implement a total and terminating implementation of our approach which would be a formal proof-by-construction that the agent-based approach of the SIR model terminates after a finite number of steps.

\newpage

\bibliographystyle{acm}
\bibliography{../../references/phdReferences}

\end{document}