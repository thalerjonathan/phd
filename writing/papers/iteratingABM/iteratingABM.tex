%%%%%%%%%%%%%%%%%%%%%%%%%%%%%%%%%%%%%%%%%
% University/School Laboratory Report
% LaTeX Template
% Version 3.1 (25/3/14)
%
% This template has been downloaded from:
% http://www.LaTeXTemplates.com
%
% Original author:
% Linux and Unix Users Group at Virginia Tech Wiki 
% (https://vtluug.org/wiki/Example_LaTeX_chem_lab_report)
%
% License:
% CC BY-NC-SA 3.0 (http://creativecommons.org/licenses/by-nc-sa/3.0/)
%
%%%%%%%%%%%%%%%%%%%%%%%%%%%%%%%%%%%%%%%%%

%----------------------------------------------------------------------------------------
%	PACKAGES AND DOCUMENT CONFIGURATIONS
%----------------------------------------------------------------------------------------

\documentclass[twocolumn]{article}
% \documentclass{llncs}
% \usepackage{makeidx}


\usepackage[utf8]{inputenc}
\usepackage{graphicx} % Required for the inclusion of images
%\usepackage{natbib} % Required to change bibliography style to APA
\usepackage{amsmath} % Required for some math elements 
\usepackage{glossaries}
\usepackage[toc,page]{appendix}
\usepackage[autostyle=true]{csquotes}
\usepackage{hyperref}
\usepackage{amssymb}
\usepackage{caption} 
\usepackage{hhline}
\usepackage{float}
\usepackage{listings}
\usepackage{graphicx}
\usepackage{subcaption}
\usepackage{rotating}

\setlength\parindent{0pt} % Removes all indentation from paragraphs

%\usepackage{times} % Uncomment to use the Times New Roman font

%----------------------------------------------------------------------------------------
%	DOCUMENT INFORMATION
%----------------------------------------------------------------------------------------

\title{The Art of Iterating:\\Update-Strategies in Agent-Based Simulations} % Title
 
\author{Jonathan \textsc{Thaler}} % Author name

\date{\today} % Date for the report

\begin{document}
%
\maketitle % Insert the title, author and date

% If you wish to include an abstract, uncomment the lines below
\begin{abstract}
When developing a model for an Agent-Based Simulation (ABS) it is of very importance to select the update-strategy which reflects the semantics of the model to produce the correct results. This awareness, we claim, is still lacking in the field of ABS. In this paper we present a novel terminology by systematic treatment of all general properties of ABS and deriving the general update-strategies. This will allow implementers and researchers in this field to use a general terminology, removing ambiguities when discussing ABS and their models. Further we present a novel investigation into the suitability of three very different programming languages Java, Haskell and Scala with Actors to implement ABS in generality by implementing each of the update-strategies we derived.
\end{abstract}

\subsection*{Keywords}
Agent-Based Simulation, Parallelism, Concurrency, Haskell, Actors, Prisoners Dilemma, Heroes and Cowards

%*******************************************************************************
%*********************************** First Chapter *****************************
%*******************************************************************************

\chapter{Introduction}  %Title of the First Chapter
I noticed that it is pretty hard to convince an agent-based economics specialist who is not a computer scientist about a pure functional approach. My conjecture is that the implementation technique and method does not matter much to them because they have very little knowledge about programming and are almost always self-taught - they don't know about software-engineering, nothing about proper software-design and architecture, nothing about software-maintenance, nothing about unit-testing,... In the end they just "hack" the simulation in whatever language they are able to: C++, Visual Basic, Java or toolboxes like Netlogo. For them it is all about to \textit{get things done somehow} and not to get things done the right way or in a beautiful way - the way and the method doesn't matter, its just a necessary evil which needs to be done. Thus if functional programming could make their lives easier, then they will definitely welcome it. But functional programming is, i think, harder to learn and harder to understand - so one needs to provide an abstraction through EDSL. So I REALLY need to come up with convincing arguments why to use pure functional approaches in ACE THEY can understand, otherwise I will be lost and not heard (not published,...). \\

What ACE economists care for:

\begin{itemize}
\item Very: Qualitative modelling with quantitative results
\item Yes: Easy reproducibility
\item Likely: Reasoning about convergence?
\item Likely: EDSL
\end{itemize}

My contributions are: pure functional framework, functional agent-model for market-simulations, EDSL for market-simulations, qualitative / implicit modelling with quanitative results, reasoning in my framework about convergence \\

IDEA: could I develop non-causal modelling (models are expressed in terms of non-directed equations, modelled in signal-relations) to allow for qualitative modelling for the agent-based economists? See hybrid modelling paper of Yampa. \textbf{THIS WOULD BE A HUGE NOVEL CONTRIBUTION TO ACE ESPECIALLY WHEN COMBINED WITH AN EDSL AND PROVIDING FULL REFERENTIAL TRANSPARENCY TO KEEP THE ABILITY TO REASON ABOUT CONVERGENCE}. This should be covered in the "EDSL"-paper.

TODO: maybe i should really focus only on market models? otherwise too much? \\

central novelty of my PhD: model specification = runnable code. possible through EDSL. but only in specific subfield of ACE: market-models. need a functional description of the model, then translate it to model specification in EDSL and then run it to see dynamics. But: model specification moves closer to functional programming languages. \\

another novelty approach: model specification through qualitative instead of quantiative approaches. is this possible? \\

WHY FUNCTIONAL? "because its the ultimate approach to scientific computing": fewer bugs due to mutable state (why? is thos shown obkectively by someone?), shorter (again as above, productivity), more expressive and closer to math, EDSL, EDSL=model=simulation, better parallelising due to referental transparency, reasoning \\

scientific results need to be reproduced, especially when they have high impact. a more formal approach of specifying the model and the simulation (model=simulation) could lead to easier sharing and easier reporduction without ambigouites \\

pure functional agent-model \& theory, EDSL framework in Haskell for ACE

\begin{enumerate}
\item Which kind of problem do we have?
\item What aim is there? Solving the problem? 
\item How the aim is achieved by enumerating VERY CLEAR objectives.
\item What the impact one expects (hypothesis) and what it is (after results).
\end{enumerate}

Note: It is not in the interest of the researcher to develop new economic theories but to research the use of functional methods (programming and specification) in agent-based computational economics (ACE).

NOTE: Get the reader’s attention early in the introduction: motivation, significance, originality and novelty.

\section{Methods}
Methods need to be selected to implement the simulations. Special emphasis will be put on functional ones which will then be compared to established methods in the field of ABM/S and ACE. \\

Claim: non-programming environments are considered to be not powerful enough to capture the complexity of ACE implementations thus a programming approach to ACE will be always required.

\section{Scenarios}
To apply and test functional methods in ACE, four scenarios of ACE are selected and then the methods applied and compared with each other to see how each of them perform in comparison. The 4 selected scenarios represent a selection of the challenges posed in ACE: from very abstract ones to very operational ones.

\section{Comparison}
Each of the selected scenarios is then implemented using the selected methods where each solution is then compared against the following criteria: 

\begin{enumerate}
\item suitability for scientific computation
\item robustness
\item error-sources
\item testability
\item stability
\item extendability
\item size of code
\item maintainability
\item time taken for development
\item verification \& correctness
\item replications \& parallelism
\item EDSL
\end{enumerate}

This will then allow to compare the different methods against each other and to show under which circumstances functional methods shine and when they should not be used.

\section{Agent-Based Modelling and Simulation (ABM/S)}
ABM/S is a method of modelling and simulating a system where the global behaviour may be unknown but the behaviour and interactions of the parts making up the system is of knowledge (Wooldrige, M. (2009). An Introduction to MultiAgent Systems. John Wiley & Sons). Those parts, called agents, are modelled and simulated out of which then the aggregate global behaviour of the whole system emerges. Thus the central aspect of ABM/S is the concept of an Agent which can be understood as a metaphor for a pro-active unit, able to spawn new Agents, and interacting with other Agents in a network of neighbours by exchange of messages. The implementation of Agents can vary and strongly depends on the programming language and the kind of domain the simulation and model is situated in.

\section{Agent-Based Economics (ACE)}
According to Leigh Tesfatsion (Tesfatsion, L. (2006). Agent-based computational economics: A constructive approach to economic theory. In Tesfatsion, L. and Judd, K. L., editors, Handbook of Computational Economics, volume 2, chapter 16, pages 831–880. Elsevier, 1 edition.), one of the leading figures, ACE is "[...] computational modelling of economic processes (including whole economies) as open-ended dynamic systems of interacting agents." - thus lending perfectly to the use of ABM/S as already the name suggests. Whereas classical economic models fall short by only looking at the average, pure rational, individual interacting in anonymous markets, the ACE approach looks at heterogeneous, non-rational individuals interacting with each other in networks (Kirman, A. (2010). Complex Economics: Individual and Collective Rationality. Routledge, London ; New York, NY.). Thus ACE can be understood as a combination of computer-science, cognitive/social science and evolutionary economics.

\section{Functional programming}
TODO: read \cite{Backus1978}

The state-of-the-art approach to implementing Agents are object-oriented methods and programming as the metaphor of an Agent as presented above lends itself very naturally to object-orientation (OO). The author of this thesis claims that OO in the hands of inexperienced or ignorant programmers is dangerous, leading to bugs and hardly maintainable and extensible code. The reason for this is that OO provides very powerful techniques of organising and structuring programs through Classes, Type Hierarchies and Objects, which, when misused, lead to the above mentioned problems. Also major problems, which experts face as well as beginners are 1. state is highly scattered across the program which disguises the flow of data in complex simulations and 2. objects don’t compose as well as functions. The reason for this is that objects always carry around some internal state which makes it obviously much more complicated as complex dependencies can be introduced according to the internal state.
All this is tackled by (pure) functional programming which abandons the concept of global state, Objects and Classes and makes data-flow explicit. This then allows to reason about correctness, termination and other properties of the program e.g. if a given function exhibits side-effects or not. Other benefits are fewer lines of code, easier maintainability and ultimately fewer bugs thus making functional programming the ideal choice for scientific computing and simulation and thus also for ACE. A very powerful feature of functional programming is Lazy evaluation. It allows to describe infinite data-structures and functions producing an infinite stream of output but which are only computed as currently needed. Thus the decision of how many is decoupled from how to (Hughes, J. (1989). Why functional programming matters. Comput. J., 32(2):98–107.).
The most powerful aspect using pure functional programming however is that it allows the design of embedded domain specific languages (EDSL). In this case one develops and programs primitives e.g. types and functions in a host language (embed) in a way that they can be combined. The combination of these primitives then looks like a language specific to a given domain, in the case of this thesis ACE. The ease of development of EDSLs in pure functional programming is also a proof of the superior extensibility and composability of pure functional languages over OO (Henderson P. (1982). Functional Geometry. Proceedings of the 1982 ACM Symposium on LISP and Functional Programming.).
One of the most compelling example to utilize pure functional programming is the reporting of Hudak (Hudak P., Jones M. (1994). Haskell vs. Ada vs. C++ vs. Awk vs. ... An Experiment in Software Prototyping Productivity. Department of Computer Science, Yale University.)  where in a prototyping contest of DARPA the Haskell prototype was by far the shortest with 85 lines of code. Also the Jury mistook the code as specification because the prototype did actually implement a small EDSL which is a perfect proof how close EDSL can get to and look like a specification.

Functional languages can best be characterized by their way computation works: instead of \textit{how} something is computed, \textit{what} is computed is described. Thus functional programming follows a declarative instead of an imperative style of programming. The key points are:
\begin{itemize}
\item No assignment statements - variables values can never change once given a value.
\item Function calls have no side-effect and will only compute the results - this makes order of execution irrelevant, as due to the lack of side-effects the logical point in \textit{time} when the function is calculated within the program-execution does not matter.
\item higher-order functions
\item lazy evaluation
\item Looping is achieved using recursion, mostly through the use of the general fold or the more specific map.
\item Pattern-matching
\end{itemize}

This alone does not really explain the \textit{real} advantages of functional programming and one must look for better motivations using functional programming languages. One motivation is given in \cite{Hughes1989} which is a great paper explaining to non-functional programmers what the significance of functional programming is and helping functional programmers putting functional languages to maximum use by showing the real power and advantages of functional languages. The main conclusion is that \textit{modularity}, which is the key to successful programming, can be achieved best using higher-order functions and lazy evaluation provided in functional languages like Haskell. \cite{Hughes1989} argues that the ability to divide problems into sub-problems depends on the ability to glue the sub-problems together which depends strongly on the programming-language and \cite{Hughes1989} argues that in this ability functional languages are superior to structured programming.

TODO: comparison of functional and object-oriented programming. My points are:
\begin{itemize}
\item The way state can be changed and treated - distributed over multiple objects - is often very difficult to understand.
\item Inheritance is a dangerous thing if not used with care because inheritance introduces very strong dependencies which cannot be changed during runtime anymore.
\item Objects don't compose very well: \url{http://zeroturnaround.com/rebellabs/why-the-debate-on-object-oriented-vs-functional-programming-is-all-about-composition/}
\item (Nearly) impossible to reason about programs
\end{itemize}

In conclusion the upsides of functional programming as opposed to OO are:
\begin{itemize}
\item Much more explicit flow of data \& control
\item Much better compose-able
\item Much better parallelism
\end{itemize}

\section{Related Research}
Tim Sweeney, CTO of Epic Games gave an invited talk about how "future programming languages could help us write better code" by "supplying stronger typing, reduce run-time failures;  and the need for pervasive concurrency support, both implicit and explicit, to effectively exploit the several forms of parallelism present in games and graphics." \cite{Sweeney2006}. Although the fields of games and agent-based simulations seem to be very different in the end, they have also very important similarities: both are simulations which perform numerical computations and update objects - in games they are called "game-objects" and in abm they are called agents but they are in fact the same thing - in a loop either concurrently or sequential. His key-points were:

\begin{itemize}
\item Dependent types as the remedy of most of the run-time failures.
\item Parallelism for numerical computation: these are pure functional algorithms, operate locally on mutable state. Haskell ST, STRef solution enables encapsulating local heaps and mutability within referentially transparent code.
\item Updating game-objects (agents) concurrently using STM: update all objects concurrently in arbitrary order, with each update wrapped in atomic block - depends on collisions if performance goes up.
\end{itemize}

\section{Related Research}

\cite{schneider_towards_2012} and \cite{vendrov_frabjous:_2014} present a domain-specific language for developing functional reactive agent-based simulations. This language called FRABJOUS is very human readable and easily understandable by domain-experts. It is not directly implemented in FRP/Haskell/Yampa but is compiled to Haskell/Yampa code which they claim is also readable. This is the direction we want to head but we don't want this intermediate step but look for how a most simple domain-specific language embedded in Haskell would look like. In this paper we explicitly dive deep into FRP And Yampa and see how we can combine the best of both.

\section{Background}

\subsection{Schelling Segregation}
We follow in our implementation the original paper of Schelling as in \cite{schelling_dynamic_1971} where we focus on the \textit{Area Distribution} section (Schelling starts with movement in a linear, 1-dimensional world where agents are able to move to the nearest point which meets the agents satisfaction but this is not what we follow here). One assumes a discrete 2-dimensional lattice-world with NxM fields. Each field is either occupied by an agent of a given color (e.g. Red or Green) or is free. Each field has 8 neighbours, which denotes a Moore-Neighbourhood. In Schellings original work the lattice-world is limited at its borders but we assume a torus world which is wrapped around in both the x- and y-dimensions resulting in 8 neighbours also for fields at the border. The occupation density was set by Schelling to be about 70\%-75\% which he identifies as being a setting which allows the agents to move around freely without making the lattice-world too sparse.
Now the agents make their move sequentially one after another. In each move an agent calculates the number of neighbours which are of equal color. If the number satisfies the agents needs about the neighbourhood then the agent is regarded as being 'happy' and will stay on this field. On the other hand the agent moves to the nearest unoccupied field which satisfies its needs. An agent which moves selects an unoccupied place randomly relative from its current place within a rectangle of side-length 2r where its current place is at the center. The interpretation for that behaviour is that agents won't move too far as it could be costly. Also children might attend a school in this area or the family has friends in this area, so they don't want to break that.



Agents just move depending on their movement-strategy to another place if they are not happy on the current one - they don't care how the target place is in the present or in the future, they will decide again in the next time-step. The interpretation for that behaviour is: agents want to 'just get out' at any cost, not caring what the future place will look like - it might be better or worse but they will see then.

\subsubsection{Optimizing behaviour}
TODO: define utility

The original schelling model didn't have a move-optimizing behaviour, meaning agents are just binary: if it is happy it will not move, if it is unhappy it will move but they won't care where they move. We introduce local move-optimizing behaviours which can be interpreted as being realistic in the real-world. It is important to note that we focus on \textit{local} instead of \textit{global} move-optimization: the agents are limited in their reasoning-capabilities and have limited information available: they cannot check out \textit{every} place and pick the globally best one.\\

\subsubsection{Anticipating behaviour}
Schelling explicitly mentions in \cite{schelling_dynamic_1971} that nobody anticipates moves of others. This is what we introduce using the recursive simulation.

TODO: is this optimizing behaviour in the spirit of schellings original work? 

\paragraph{Optimizing future} Agents pick an unoccupied random place and move to it if it increases their utility in the future. The interpretation for that behaviour is: agents heard about a place which will be cool in the future.

\paragraph{Optimizing present \& future} Agents pick an unoccupied random place and move to it if it increases their utility in the now and in the future. The interpretation for that behaviour is: agents heard about a cool spot in town, check it out and move to it if they like it but they also anticipate the coolness of the place in the future and if it seems that the place is going down then they won't move there.

\subsection{Related Research}
TODO: \cite{kirman_complex_2010} mention kirman complex economics where he investigates the model more in depth


\section{Update-Strategies}
2 Pages

In this section we will present all the update-strategies which are available in ABM/S in a general form, discuss abstract details independent from programming languages, give philosophical meaning and interpretation of them and advices for selecting them for which model.

\begin{table}[H]
	\center
	\begin{tabular}{ c | c | c | c  }
		\textit{Name} & \textit{Time-Flow} & \textit{Iteration-Order} & \textit{Deterministic} \\
		\hhline{=|=|=|=}
	    \textbf{Seq} & Global & Sequential & Yes \\
	    \hline
	    \textbf{Par} & Global & Parallel & Yes \\
	    \hline
	    \textbf{Con} & Global & Concurrent & No \\
	    \hline
	    \textbf{Act} & Local & Actor & No \\
	\end{tabular}
	\caption{List of all general update-strategies in ABM/S.}
\end{table}

\subsection{Seq}
\textbf{Description:} This strategy has a global time-flow and in each time-step iterates through all the agents and updates one Agent after another. Messages sent and changes to the environment made by Agents are visible immediately. More formally, we assume that, given the updates are done in order of the index $i = [1..n]$, then Agents $a_{n>i}$ see the changes to environment and messages sent to them by Agent $a_i$. Note that there is no source of randomness and non-determinism thus rendering this strategy to be completely deterministic in each step. 

\textbf{Time-Updates:} The most straight-forward approach is to keep the time constant for each agent in one iteration. This though may seem non-logic as the actions of preceding Agents are visible to later ones but time has not changed since then. Thus if the model semantics require that time changes with every Agent we could advance for every agent by fraction of dt: agent-time = t + (ai * dt/n) where t is the current simulation time, ai the agents index, dt the amount of time the simulation will be advanced by and n the number of agents. In the end the new current time will be then tnext = tcurr + dt. Other possibilities of advancing is agent-time = t + ai * dt. in the end the new current time will be then tnext = tcurr + n * dt

\textbf{Extensions}: If the sequential iteration from 1..n imposes an advantage over the Agents further ahead or behind in the queue (e.g. if it is of benefit when making choices earlier than others in auctions or later when more information is available) then one could use random-walk iteration where in each time-step the agents are shuffled before iterated. Note that although this would introduce randomness in the model the source is a random-number generator thus reproduce-able.


\subsection{Par}
\textbf{Description:} This strategy has a global time-flow and in each time-step iterates through all the agents and updates all Agents in parallel. Messages sent and changes to the environment made by Agents are visible in the next global step. We can think about this strategy that all Agents make their moves at the same time. 

\textbf{Environment:} If one wants to change the environment in a way that it would be visible to other Agents this is regarded as a systematic error in this strategy. First it is not logical because all actions are meant to happen at the same time and also it would implicitly induce an ordering thus violating the \textit{happens at the same time} idea. Thus we require different semantics for accessing the environment in this strategy. We introduce thus a \textit{global} environment which is made up of the set of \textit{local} environments. Each local environment is owned by an Agent thus there are as many local environments as there are Agents. The semantics are then as follows: in each step all Agents can \textit{read} the global environment and \textit{read/write} their local environment. The changes to a local environment are only visible \textit{after} the local step and can be fed back into the global environment after the parallel processing of the Agents.

\textbf{Time-Updates:} Time really stays constant in this case for all Agents in one step as all updates happen really at the same \textit{virtual} time. It would make no sense to advance the time between Agents as all actions are meant to happen at the same time, without imposing any ordering amongst them.

\textbf{Semantics:} It does not make a difference if the Agents are really computed in parallel or just sequentially, due to the semantics of changes, this has the same effect. In this case it will make no difference how we iterate over the agents (sequentially, randomly), the outcome \textit{has to be} the same - it is event-ordering invariant as all events/updates happen \textit{virtually} at the \textit{same time}. Thus if one needs to have the semantics of writes on the whole (global) environment in ones model, then this strategy is not the right one and one should resort to one of the other strategies.

		
\subsection{Con}
\textbf{Description:} This strategy has a global time-flow and in each time-step iterates through all the agents and updates all Agents in parallel but all messages sent and changes to the environment are immediately visible. Thus this strategy can be understood as a mix of Seq and Par: all Agents run at the same time with actions becoming immediately visible.

\textbf{Semantics:} It is important to realize that, when running Agents in parallel which are able to see actions by others immediately, this is the very definition of concurrency: parallel execution with mutual read/write access to shared data. Of course this shared data-access needs to be synchronized which in turn will introduce event-orderings in the execution of the Agents. Thus at this point we have a source of inherent non-determinism: although we would ignore any hardware-model of concurrency at some point we need arbitration to decide which Agent gets access first to a shared resource thus arriving at non-deterministic solutions - this will become much clearer in the results-section. This has the very important influence that repeated runs with the same configuration of the Agents and the Model may lead to different results each time.


\subsection{Act}
TODO: discuss how local-time can be handled: real-time or simulation-time - its always local and not synchronized globally because then we would end up in Concurrent Strategy

in the Act-version we need to observe the agents: we need to sample them regularly => we have all the issues with sampling 

In this case there is no global iteration over steps but all the Agents run in parallel, doing local stepping and communicate with each other either through shared state or messages. Note that this does not impose any specific ordering of the update and can thus regarded to be real random due to its concurrent nature. It is possible to simulate the global-stepping methods from above by introducing some global locking forcing the agents into lock-step. This is the approach chosen for Scala \& Actors.


\cite{clinger_foundations_1981}
\cite{grief_semantics_1975}

\textbf{Semantics:} This is the most general one of all the strategies as it can emulate all the others by introducing the necessary synchronization mechanisms and Agents.


\subsection{Update-Strategies}
\begin{enumerate}
\item All states are copied/frozen which has the effect that all agents update their positions \textit{simultaneously}
\item Updating one agent after another utilizing aliasing (sharing of references) to allow agents updated \textit{after} agents before to see the agents updated before them. Here we have also two strategies: deterministic- and random-traversal.
\item Local observations: Akka
\end{enumerate}



\subsection{Different results with different Update-Strategies?}
Problem: the following properties have to be the same to reproduce the same results in different implementations: \\

Same initial data: Random-Number-Generators
Same numerical-computation: floating-point arithmetic
Same ordering of events: update-strategy, traversal, parallelism, concurrency

\begin{itemize}
\item Same Random-Number Generator (RNG) algorithm which must produce the same sequence given the same initial seed.
\item Same Floating-Point arithmetic
\item Same ordering of events: in Scala \& Actors this is impossible to achieve because actors run in parallel thus relying on os-specific non-deterministic scheduling. Note that although the scheduling algorithm is of course deterministic in all os (i guess) the time when a thread is scheduled depends on the current state of the system which can change all the time due to \textit{very} high number of variables outside of influence (some of the non-deterministic): user-input, network-input, .... which in effect make the system appear as non-deterministic due to highly complex dependencies and feedback.
\item Same dt sequence => dt MUST NOT come from GUI/rendering-loop because gui/rendering is, as all parallelism/concurency subject to performance variations depending on scheduling and load of OS.
\end{itemize}

It is possible to compare the influences of update-strategies in the Java implementation by running two exact simulations (agentcount, speed, dt, herodistribution, random-seed, world-type) in lock-step and comparing the positions of the agent-pairs with same ids after each iteration. If either the x or y coordinate is no equal then the positions are defined to be \textit{not} equal and thus we assume the simulations have then diverged from each other. \\
It is clear that we cannot compare two floating-point numbers by trivial == operator as floating-point numbers always suffer rounding errors thus introducing imprecision. What may seem to be a straight-forward solution would be to introduce some epsilon, measuring the absolute error: abs(x1 - x2) > epsilon, but this still has its pitfalls. The problem with this is that, when number being compared are very small as well then epsilon could be far too big thus returning to be true despite the small numbers are compared to each other quite different. Also if the numbers are very large the epsilon could end up being smaller than the smallest rounding error, so that this comparison will always return false. The solution would be to look at the \textit{relative error}: abs((a-b)/b) < epsilon. \\
The problem of introducing a relative error is that in our case although the relative error can be very small the comparison could be determined to be different but looking in fact exactly the same without being able to be distinguished with the eye. Thus we make use of the fact that our coordinates are virtual ones, always being in the range of [0..1] and are falling back to the measure of absolute error with an epsilon of 0.1. Why this big epsilon? Because this will then definitely show us that the simulation is \textit{different}. \\

The question is then which update-strategies lead to diverging results. The hypothesis is that when doing simultaneous updates it should make no difference when doing random-traversal or deterministic traversal => when comparing two simulations with simultaneous updates and all the same except first random- and the other deterministic traversal then they should never diverge. Why? Because in the simultaneous updates there is no ordering introduce, all states are frozen and thus the ordering of the updates should have no influence, \textit{both simulations should never diverge, \textbf{independent how dt and epsilon are selected}}. \\
Do the simulation-results support the hypothesis? Yes they support the hypothesis - even in the worst cast with very large dt compared to epsilon (e.g. dt = 1.0, epsilon = 1.0-12)

The 2nd hypothesis is then of course that when doing consecutive updates the simulations will \textit{always} diverge independent when having different traversal-strategies. \\
Simulations show that the selection of \textit{dt} is crucial in how fast the simulations diverge when using different traversal-strategies. The observation is that \textit{The larger dt the faster they diverge and the more substantial and earlier the divergence.}. Of course it is not possible to proof using simulations alone that they will always diverge when having different traversal-strategies. Maybe looking at the dynamics of the error (the maximum of the difference of the x and y pairs) would reveal some insight? \\

The 3rd hypothesis is that the number of agents should also lead to increased speed of divergence when having different traversal-strategies. This could be shown when going from 60 agents with a dt of 0.01 which never exceeded a global error of 0.02 to 6000 agents which after 3239 steps exceeded the absolute error of 0.1.

\subsection{Reproducing Results in different Implementations}
actors: time is always local and thus information as well. if we fall back to a global time like system time we would also fall back to real-time. anyway in distributed systems clock sync is a very non-trivial problem and inherently not possible (really?). thus using some global clock on a metalevel above/outside the simulation will only buy us more problems than it would solve us. real-time does not help either as it is never hard real time and thus also unpredictable: if one tells the actor to send itself a message after 100ms then one relies on the capability of the OS-timer and scheduler to schedule exactly after 100ms: something which is always possible thus 100ms are never hard 100ms but soft with variations.

qualitative comparison: print pucture with patterns. all implementations are able to reproduce these patterns independent from the update strategy

no need to compare individual runs and waste time in implementing RNGs, what is more interesting is whether the qualitative results are the same: does the system show the same emergent behaviour? Of course if we can show that the system will behave exactly the same then it will also exhibit the same emergent behaviour but that is not possible under some circumstances e.g. the simulation-runs of Akka are always unique and never comparable due to random event-ordering produced by concurrency \& scheduling. Also we don't have to proof the obvious: given the same algorithm, the same random-data, the same treatment of numbers and the same ordering of events, the outcome \textit{must} be the same, otherwise there are bugs in the program. Thus when comparing results given all the above mentioned properties are the same one in effect tests only if the programs contain no bugs - or the same bugs, if they \textit{are the same}. \\

Thus we can say: the systems behave qualitatively the same under different event-orderings.

Thus the essence of this boils down to the question: "Is the emergent behaviour of the system is stable under random/different/varying event-ordering?". In this case it seems to be so as proofed by the Akka implementation. In fact this is a very desirable property of a system showing emergent behaviour but we need to get much more precise here: what is an event? what is an emergent behaviour of a system? what is random-ordering of events? (Note: obviously we are speaking about repeated runs of a system where the initial conditions may be the same but due to implementation details like concurrency we get a different event-ordering in each simulation-run, thus the event-orderings vary between runs, they can be in fact be regarded as random).


\section[Language Comparison]{Language Comparison \footnote{Code available under\\ \url{https://github.com/thalerjonathan/phd/tree/master/coding/papers/iteratingABM/}}}
In this section we give a brief overview of comparing the suitability of three fundamentally different languages to implement the update-strategies. We wanted to cover a wide range of different types of languages but didn't include a language where the memory-management falls in the hands of the developer. This would be the case e.g. in C++. This was looked into partially by \cite{dawson_opening_2014} but the focus of this paper is not on this issue as it would complicated things dramatically. All used languages are garbage-collected / the developer does not need to care how memory is cleared up.

For testing the suitability we selected a variety of simple models we implemented in each language with mostly all strategies. The selected models are \textit{Heroes \& Cowards}, \textit{SIRS}, \textit{Wildfire} and the \textit{Spatial Game} mentioned in \cite{huberman_evolutionary_1993}. We lack the space to explain all models but all are well known and can be easily found, looked up and understood on the Internet. They span different challenges to the ABS implementation: sending messages, accessing the environment, spawning new Agents, killing existing ones, discrete and continuous model. We also can confirm that all the reference-models proposed in \cite{isaac_abm_2011} and the StupidModel 1-16 by Railsback (TODO: cite) can be faithfully capture using our new terminology. Also we could show that the  Parallel Strategy is the only strategy able to reproduce the pattern of the prisoners dilemma due to the semantics of the model which require that all the Agents play the game at virtually the same time - which is only possible in the Parallel Strategy.

TODO: mention, that we didn't want to abuse the language and focus on its strengths e.g. don't rebuild OO in Haskell.


\subsection{Java}
This language is included as the benchmark of object-oriented (OO) imperative languages as it is extremely popular in the ABS community and widely used in implementing their models and Environments. It comes with a comprehensive programming library, has nice object-oriented features and powerful synchronization primitives built in at language-level.

\paragraph{Ease of Use}
Being experienced Java-Programmers we found that implementing all the strategies was straight-forward and easy thanks to the languages features. Especially parallelism and concurrency is quite very easy due to elegant and powerful built-in synchronization primitives.

\paragraph{Benefits}
We experienced quite high-performance even for a large number of agents which we attributed to aliasing using references and side-effects. This prevents massive copying like Haskell but comes at the cost of explicit data-flow.

\paragraph{Deficits}
We couldn't identify something which absolutely didn't work. That's also why Java can be regarded as a very safe decision when looking for an appropriate language to use for implementing ABS.A downside is that one must take care when accessing memory in case of Parallel or Concurrent strategy. Due to the availability of aliasing and side-effects in the language and the type-system, it can't be guaranteed that access to memory happens only when its safe. Thus care must be taken when accessing references sent by messages to other Agents, accessing references to other Agents or the infrastructure of an Agent itself e.g. the message-box. TODO: actor not possible if high number of agents because java can't handle very large number of threads. in parallel and concurrent one uses executorservice with the number of cores and submits for each agent a task which runs the update and the tasks are then evenly executed on the available threads. This approach does not work for the actor strategy where we don't have tasks for a single update but we constantly run the agent in a thread thus the task would not return until the agent shuts itself down. 

\paragraph{Natural Strategy}
We found that the Sequential Strategy with immediate message-handling is the most natural strategy to express in Java due to its heavy reliance on side-effects through references (aliases) and shared thread of execution. Also most of the models work this way and its thus a save decision to use Java.





 
\subsection{Haskell}
This language is included to put to test whether such a pure functional, declarative programming language is suitable for full-blown ABS. What distinguishes it is its complete lack of implicit side-effects, global data, mutable variables and objects. The central concept is the function into which all data has to be passed in and out explicitly through statically typed arguments and return values: data-flow is completely explicit.

\paragraph{Ease of Use}
Being beginners in Haskell we initially thought that it would be suitable at best for just implementing the Parallel Strategy due the inherent data-parallel nature of pure functional languages. After having implementing all strategies we had to admit that Haskell is very well suited to implement all of them faithfully. We think this stems from the facts that it has no implicit side-effects which reduces bugs considerably and results in very explicit data-flow. \\

Not having objects with data and methods which can call between each other meant, that we needed some different way of representing Agents. This was done using a struct-like type to carry data and a transformer function which would receive and process messages. This may seem to look like OO but it is not: Agents are not carried around but messages are sent to a receiver identified by an id.

\paragraph{Benefits}
We really enjoyed working in the extremely powerful static type-system. Although it seems to be restrictive in the beginning, when one gets used to it and knows how to use it for ones help, then it becomes rewarding. Our major point was to let the type-system prevent us from introducing side-effects. In Haskell this is only possible in code marked in its types as producing side-effects, so this was something we explicitly avoided and were able to do so throughout the whole implementation. This means a user of this approach can be guided by the types and is prevented from abusing them. Thus the lesson learned here is that \textit{if one tries to abuse the types or work around, then this is an indication that the update-strategy one has selected does not match the semantics of the model one wants to implement}. If this happens in Java, it is much more easier to work around by introducing global variables or side-effects but this is not possible in Haskell. Also we claim that when using Haskell one arrives at a much safer version in the case of Parallel or Concurrent Strategies than in Java.\\

Parallelism and Concurrency is a breeze in Haskell due to its complete lack of implicit side-effects. Adding hardware-parallel execution in the Parallel-Strategy required the adoption of only 5 lines of code and no change to the existing Agent-Code at all (e.g. no synchronization, as there are no implicit side-effects). For implementing the Concurrent Strategy we utilized the programming model of Software-Transactional-Memory (STM). The approach is that one optimistically runs Agents which introduce explicit side-effects in parallel where each Agent executes in a transaction and then to simply retry the transaction if another Agent has made concurrent side-effect modifications. This frees one from thinking in terms of synchronization and leaves the code of the Agent nearly the same as in the Sequential Strategy.
TODO: spawning thousands of parallel threads and keeping them running is no problem due to the leightweight handling, something which is missing in java.

\paragraph{Deficits}
Performance is an issue. Our Haskell solution could run only about 2000 Agents in real-time with 25 updates per second as opposed to 50.000 in our Java solution, which is not very fast. It is important though to note, that being beginners in Haskell, we are largely unaware of the subtle performance-details of the language thus we expect to achieve a massive speed-up in the hands of an experienced programmer. \\

Another thing is that currently only homogeneous agents are possible and still much work needs do be done to capture large and complex models with heterogeneous agents. For this we need a more robust and comprehensive surrounding framework, which is already existent in the form of functional reactive programming (FRP). Our next paper is targeted on combining our Haskell solution with an FRP framework like Yampa (see further research). \\ 

Our solution so far is unable to implement the Sequential Strategy with immediate message-handling. This is where object-orientation really shines and pure functional programming seems to be lacking in convenience. A solution would need to drag the collection of all Agents around which would make state-handling and manipulation very cumbersome. In the end it would have meant to rebuild OO concepts in a pure functional language, something we didn't wanted to do. For now this is left as an open, unsolved issue and we hope that it could be solved in our approach with FRP (see future research).

\paragraph{Natural Strategy}
The most natural strategy is the Parallel-Strategy as it lends itself so well to the concepts of pure functional programming where things are evaluated virtually in parallel without side-effects on each other - something which resembles exactly the semantics of the Parallel Strategy. We argue that the Concurrent Strategy is also very natural formulated in Haskell due to the availability of STM, something only possible in a language without implicit side-effects as otherwise retries of transactions would not be possible.



\subsection{Scala with Actors}
This multi-paradigm functional language is included to test the usefulness of the Actor Strategy for implementing ABS. The language comes with an Actor-library inspired by \cite{agha_actors:_1986} and resembles the approach of Erlang which allows a very natural implementation of the strategy.

\paragraph{Ease of Use}
We were completely new to Scala with Actors although we have some experience using Erlang. We found that the language has some very nice mixed-paradigm features which allow to program in a very flexible way without inducing too much restrictions on one.

\paragraph{Benefits}
Implementing Agent-behaviour is extremely convenient, especially for simple state-chart Agents. The Actor-language has a built-in feature which allows to change the behaviour of an Agent on message reception where the Agent then simply switches to a different message-handler, allowing elegant implementation of state-charts. \\

Performance is very high. We could run simulations in real-time with about 200.000 Agents concurrently, something the run-time system easily manages. Also it is very important to note that one can use the framework Akka to build real distributed systems using Scala with Actors so there are potentially no limits on the size and complexity of the models and number of agents one wants to run with it.

\paragraph{Deficits}
Care must be taken not to send references and mutable data, which is still possible in this mixed-paradigm language.

\paragraph{Natural Strategy}
The most natural strategy would be of course the Actor Strategy and we only used this Strategy in this language to implement our Models. Note that the Actor Strategy is the most general one and would allow to capture all the other strategies using the appropriate synchronization mechanisms.

\section{Case-Studies}
TODO: emphasise different types of games/simulations: need to be very clear of the difference between the 2 games
TODO: experimentation description: which language, with which model, with which configuration, number of agents, number of replications,...
TODO: what about language-differences, which language did we use?

In this section we revisit the Prisoners Dilemma model of \cite{nowak_evolutionary_1992} and present the Heroes \& Cowards model of \cite{wilensky_introduction_2015} and show results simulating both with the four update-strategies. 

TODO: why 25\% heroes, why this big number of agents 100.000?

TODO: what do I want to achieve with this test? it is basically to show 1st: that only the parallel-strategy produces the results which match the ones of the original paper in the prisoners dilemma, 2nd: that the other update-strategies create completely different results, 3rd: the heroes \& cowards game seems to be unaffected by different update-strategies.

\subsection{Effect of using different Update-Strategies}

\begin{table*}[t]
	\begin{tabular}{c c c}
		& Prisoners Dilemma & Heroes \& Cowards \\ 

		\textit{\rotatebox{90}{sequential strategy}}
		&
		\begin{subfigure}[b]{0.4\textwidth}
			\centering
			\includegraphics[width=.7\textwidth, angle=0]{./fig/seq_99x99_436steps_MSG_haskell.png}
			\caption{}
			\label{fig:pd_seq}
		\end{subfigure}
    	&
		\begin{subfigure}[b]{0.4\textwidth}
			\centering
			\includegraphics[width=.7\textwidth, angle=0]{./fig/seq_HAC_100_000_500steps_java.png}
			\caption{}
			\label{fig:hac_seq}
		\end{subfigure}
    	\\
    	
    	\textit{\rotatebox{90}{parallel strategy}}
		&
		\begin{subfigure}[b]{0.4\textwidth}
			\centering
			\includegraphics[width=.7\textwidth, angle=0]{./fig/par_99x99_436steps_MSG_haskell.png}
			\caption{}
			\label{fig:pd_par}
		\end{subfigure}
    	&
		\begin{subfigure}[b]{0.4\textwidth}
			\centering
			\includegraphics[width=.7\textwidth, angle=0]{./fig/par_HAC_100_000_500steps_java.png}
			\caption{}
			\label{fig:hac_par}
		\end{subfigure}
    	\\
    	
    	\textit{\rotatebox{90}{concurrent strategy}}
		&
		\begin{subfigure}[b]{0.4\textwidth}
			\centering
			\includegraphics[width=.7\textwidth, angle=0]{./fig/con_99x99_436steps_MSG_haskell.png}
			\caption{}
			\label{fig:pd_con}
		\end{subfigure}
    	&
		\begin{subfigure}[b]{0.4\textwidth}
			\centering
			\includegraphics[width=.7\textwidth, angle=0]{./fig/con_HAC_100_000_500steps_java.png}
			\caption{}
			\label{fig:hac_con}
		\end{subfigure}
    	\\ 
    	
    	\textit{\rotatebox{90}{actor strategy}}
		&
		\begin{subfigure}[b]{0.4\textwidth}
			\centering
			\includegraphics[width=.7\textwidth, angle=0]{./fig/act_99x99_436steps_MSG_haskell.png}
			\caption{}
			\label{fig:pd_act}
		\end{subfigure}
    	&
    	% TODO: this the same picture as in concurrent version from java, put in actor-version generated by scala    
		\begin{subfigure}[b]{0.4\textwidth}
			\centering
			\includegraphics[width=.7\textwidth, angle=0]{./fig/act_HAC_100_000_500steps_scala.png}
			\caption{}
			\label{fig:hac_act}
		\end{subfigure}
    	\\ \hline
	\end{tabular}
	
	\caption{\small Results of Prisoners Dilemma and Heroes \& Cowards with all four update-strategies.} 
	\label{fig:results}
\end{table*}

%\begin{figure*}
%
%	 \centering
	
%    \begin{subfigure}[b]{0.4\textwidth}
%			\centering
%       	\includegraphics[width=.7\textwidth, angle=0]{./fig/seq_99x99_436steps_MSG_haskell.png}
%        \caption{\textit{sequential} Prisoners Dilemma}
%        \label{fig:pd_seq}
%    \end{subfigure}
%    \begin{subfigure}[b]{0.4\textwidth}
%		\centering
%        \includegraphics[width=.7\textwidth, angle=0]{./fig/seq_HAC_100_000_500steps_java.png}
%        \caption{\textit{sequential} Heroes \& Cowards}
%        \label{fig:hac_seq}
%    \end{subfigure}
%       

%    \begin{subfigure}[b]{0.4\textwidth}
%		\centering
%       	\includegraphics[width=.7\textwidth, angle=0]{./fig/par_99x99_436steps_MSG_haskell.png}
%        \caption{\textit{parallel} Prisoners Dilemma}
%        \label{fig:pd_par}
%    \end{subfigure}
%    \begin{subfigure}[b]{0.4\textwidth}
%    	\centering
%        \includegraphics[width=.7\textwidth, angle=0]{./fig/par_HAC_100_000_500steps_java.png}
%        \caption{\textit{parallel} Heroes \& Cowards}
%        \label{fig:hac_par}
%    \end{subfigure}
%        
%
%    \begin{subfigure}[b]{0.4\textwidth}
%		\centering
%       	\includegraphics[width=.7\textwidth, angle=0]{./fig/con_99x99_436steps_MSG_haskell.png}
%        \caption{\textit{concurrent} Prisoners Dilemma}
%        \label{fig:pd_con}
%    \end{subfigure}
%    \begin{subfigure}[b]{0.4\textwidth}
%    	\centering
%        \includegraphics[width=.7\textwidth, angle=0]{./fig/con_HAC_100_000_500steps_java.png}
%        \caption{\textit{concurrent} Heroes \& Cowards}
%        \label{fig:hac_con}
%    \end{subfigure}
%
%
%    \begin{subfigure}[b]{0.4\textwidth}
%		\centering
%       	\includegraphics[width=.7\textwidth, angle=0]{./fig/act_99x99_436steps_MSG_haskell.png}
%        \caption{\textit{actor} Prisoners Dilemma}
%        \label{fig:pd_act}
%    \end{subfigure}  
%    \begin{subfigure}[b]{0.4\textwidth}
%    	\centering
%        \includegraphics[width=.7\textwidth, angle=0]{./fig/act_HAC_100_000_500steps_scala.png}
%        \caption{\textit{actor} Heroes \& Cowards}
%        \label{fig:hac_act}
%    \end{subfigure}
% TODO: this the same picture as in concurrent version from java, put in actor-version generated by scala    
% TODO: this the same picture as in concurrent version from java, put in actor-version generated by scala    
% TODO: this the same picture as in concurrent version from java, put in actor-version generated by scala    
% TODO: this the same picture as in concurrent version from java, put in actor-version generated by scala    
% TODO: this the same picture as in concurrent version from java, put in actor-version generated by scala    
% TODO: this the same picture as in concurrent version from java, put in actor-version generated by scala  

%	\caption{\small Results of Prisoners Dilemma and Heroes \& Cowards with all four update-strategies.} 
%	\label{fig:results}
%\end{figure*}

When looking at figure \ref{fig:results} the update-strategy which reflects the semantics of the model is the Parallel Strategy as all others clearly fail to reproduce the pattern as shown by the results in the original paper TODO: in figure \ref{fig:sync_patterns}. We can imply that only the Parallel Strategy is suitable to simulate this model because only that strategy is the one which renders the results of the original paper, meaning it is the 'correct' strategy. \\
The reason why the others fail to reproduce the pattern is due to the non-parallel and unsynchronized way that information spreads through the grid. In the Sequential Strategy the agents further ahead in the queue play the game earlier and influence the neighbourhood so agents in the neighbourhood which play the game later experience an already changed environment and  messages in their queue and act differently based upon these informations. This is not the case in the Parallel version where all agents play the game on the frozen state of the previous step and the outcome of each agents game will only be visible in the next step. In the Concurrent and Actor Strategy the agents run in parallel but changes are visible immediately and concurrently, leading to the same non-structural patterns as in the Sequential Strategy. \\
Note that the Concurrent and Actor Strategy produce different results on every run due to the inherent non-deterministic event-ordering introduce by concurrency. Also note that it is not possible to calculate 45 steps for the Actor Strategy as it lacks the Global Synchronization property. To arrive at a relative comparative result we just waited until the first agent arrives at a local time of 45 and then rendered the result. 

\subsection{Heroes \& Cowards}
Although the individual agent-positions of runs with the same configuration differ between update-strategies we experienced the forming of the emergent cross-pattern as seen in figure \ref{fig:results} in all four update-strategies. We can conclude that the Heroes \& Cowards model seems to be more robust to the selection of its update-strategy and that its emergent property - the formation of the cross - is stable under differing update-strategies. One would not see a difference between the different strategies so only one picture was included. Note that to test the Actor Strategy with a this high number of agents we used our implementation in Scala with Actors as Java is not able to have this high number of threads and our Haskell implementation suffers from performance issues, resorting to Scala with Actors. The results were nearly the same there, showing the big green emergent cross-pattern in the center but lacking the smaller red crosses in each section, something we attribute to the local-time of each agent and the relativity of observing the simulation.

\section{Conclusions}
\label{sec:conclusions}

Our approach is radically different from traditional approaches in the ABS community. First it builds on the already quite powerful FRP paradigm. Second, due to our continuous time approach, it forces one to think properly of time-semantics of the model and how small $\Delta t$ should be. Third it requires to think about agent interactions in a new way instead of being just method-calls.

Because no part of the simulation runs in the IO Monad and we do not use unsafePerformIO we can rule out a serious class of bugs caused by implicit data-dependencies and side-effects which can occur in traditional imperative implementations.

Also we can statically guarantee the reproducibility of the simulation, which means that repeated runs with the same initial conditions are guaranteed to result in the same dynamics. Although we allow side-effects within agents, we restrict them to only the Random and State Monad in a controlled, deterministic way and never use the IO Monad which guarantees the absence of non-deterministic side effects within the agents and other parts of the simulation.

Determinism is also ensured by fixing the $\Delta t$ and not making it dependent on the performance of e.g. a rendering-loop or other system-dependent sources of non-determinism as described by \cite{perez_testing_2017}. Also by using FRP we gain all the benefits from it and can use research on testing, debugging and exploring FRP systems \cite{perez_testing_2017, perez_back_2017}.

\subsection*{Issues}
Currently, the performance of the system is not comparable to imperative implementations but our research was not focusing on this aspect. We leave the investigation and optimization of the performance aspect of our approach for further research.

Despite the strengths and benefits we get by leveraging on FRP, there are errors that are not raised at compile time, e.g. we can still have infinite loops and run-time errors. This was for example investigated in \cite{sculthorpe_safe_2009} where the authors use dependent types to avoid some run-time errors in FRP. We suggest that one could go further and develop a domain specific type system for FRP that makes the FRP based ABS more predictable and that would support further mathematical analysis of its properties. Furthermore, moving to dependent types would pose a unique benefit over the traditional object-oriented approach and should allow us to express and guarantee even more properties at compile time. We leave this for further research.

In our pure functional approach, agent identity is not as clear as in traditional object-oriented programming, where an agent can be hidden behind a polymorphic interface which is much more abstract than in our approach. Also the identity of an agent is much clearer in object-oriented programming due to the concept of object-identity and the encapsulation of data and methods.

We can conclude that the main difficulty of a pure functional approach evolves around the communication and interaction between agents, which is a direct consequence of the issue with agent identity. Agent interaction is straight-forward in object-oriented programming, where it is achieved using method-calls mutating the internal state of the agent, but that comes at the cost of a new class of bugs due to implicit data flow. In pure functional programming these data flows are explicit but our current approach of feeding back the states of all agents as inputs is not very general and we have added further mechanisms of agent interaction which we had to omit due to lack of space.

\newpage

\bibliographystyle{acm}
\bibliography{../../references/phdReferences}

\end{document}