\section{A general terminology}

\subsection{Mapping to our terminology}
Although all definitions mentioned in the related research section have different semantics and implications, our framework faithfully captures all of them:

TODO: rework: don't cite the works but recap what they mean with sync and async

When following \cite{yuxuan_agent-based_2016} we use \textit{Seq}, \textit{Par} or \textit{Con} and the synchronous updates will happen in the pro-active internal stimulus and the asynchronous by reacting to and sending messages.

when following \cite{dawson_opening_2014} we use \textit{Seq}, \textit{Par} or \textit{Con} and the synchronous updates happen by providing the same global time-delta to all agents, which means all agents have global time and advance at the same speed. The asynchronous version would be to provide each agent with a different, random time-delta, rendering the time of the agent local instead of globally synchronized. When taking this into account we can argue that \textit{Seq}, \textit{Par} and \textit{Con} are, as described above, synchronous update-strategies and only \textit{Act} is truly asynchronous due to the locality of time and information. We can though make \textit{Seq}, \textit{Par} and \textit{Con} asynchronous too if we iterate globally as specified but instead of feeding each Agent the global time which advanced by some constant delta since the previous step, we feed each Agent a local time by introducing random time-deltas drawn individually in a given range for each Agent. This would resemble the locality of the Act strategy without introducing the non-determinism.

When following \cite{huberman_evolutionary_1993} we use the \textit{Seq} for an asynchronous-time model and \textit{Par} for the synchronous one. This was demonstrated by our implementations: when running this model with \textit{Par} the beautiful patterns emerge as reported but when following the \textit{Seq} approach they wont and all Agents defect after a given number of generations.

