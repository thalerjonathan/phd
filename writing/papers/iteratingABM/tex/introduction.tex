\section{Introduction}
In this paper we are looking at two different kind of games to 

TODO: emphasise different types of games/simulations: need to be very clear of the difference between the 2 games

TODO: here use less references

TODO: main-message: "distinguish between update-strategies", by-product is a framework, language-section needs to be incorporated, link it to the initial main-message in introduction.

TODO: describe games in a more common way, not only focus on prisoners dilemma but also incorporate heroes \& cowards. prisoners dilemma is a discrete game whereas the HAC is a continuous game.

TODO: don't write in past tense but write in present-tense: authors SHOW that...

TODO: introduction is about what you are going to discuss and how that fits into the bigger picture. not too detailed


We generalize their observation in our main message of this paper that \textit{when developing a model for an ABS it is of most importance to select the right update-strategy which reflects and supports the corresponding semantics of the model}. As we will show in the section on Related Research, we find that due to conflicting ideas about update-strategies this awareness is yet still under-represented in the field of ABS and is lacking a systematic treatment. Our contribution in this paper is to
\begin{itemize}
	\item Present general properties of ABS.
	\item Derive update-strategies from these properties.
	\item Establish a general terminology of talking about these update-strategies.
	\item Compare the three programming languages Java, Haskell and Scala with Actors in regard of their suitability to implement each of these strategies.
\end{itemize}

