\section{Problem}
1 Page

\begin{itemize}
	\item describe in more technical detail what the introduction tells.
	\item \cite{huberman_evolutionary_1993} and \url{https://www.openabm.org/book/33102/54-importance-sequence-updating} only mentions synchronous and asynchronous updates but this is not precise enough and lacks subcategories
\end{itemize}

question: do emergent patterns break down / global dynamics change completely in some ABM/S when changing sim-semantic? which kind of ABM could show this behaviour? which properties are responsible for it? Answer: Yes they do but only under given circumstances: discrete simulations with dependence on each other. continuous not so easy. 
		
do we find a continuous simulation in which it breaks down under given circumstances? Yes: Heroes \& Cowards can lead to specific patterns as shown by the creators

\subsection{Simulations}
Today simulations are at the very heart of many sciences. They allow to put hypotheses to test by building a model which abstracts from reality, keeping only the important and relevant details, and then bringing this model to life in simulation. Based on the results shown by the dynamics, previously formulated hypotheses can be verified or falsified resulting in a formulate-simulate-refine cycle. \\
The meaning of simulating a model can be understood as calculating the dynamics of a (model of a) system over time thus the state of the system at time t depends on the state of the system at time t - epsilon. Here we only consider simulations in a computer-system (TODO: are there simulations NOT in a computer?), which is an inherently discrete system which poses us with the question of how to represent time which seems linear and continuously flowing to us in reality (NOTE: this may not be physically the case but for our considerations this should be a good approximation). Being in a discrete system, of course implies that time has to be discretised as well and there are two ways of doing it: discrete and continuous where in discrete case time advances in steps of the natural numbers and where in the continuous case time advances in steps of real-numbers. Note that in both cases the system is iterated in steps where only the \textit{numerical type} of the input to the time-dependent functions differs. Thus a simulation in a computer can be understood as an iteration over a model for a given number of steps where each step advances time by dt (either discrete or continuous) and, based on the previous model-state, producing an updated model-state which again becomes the input for the next step. Thus in each simulation we have three inputs: 1. the model, 2. number of steps, 3. time dt. There are of course different models and types of simulations and in this paper we will focus on one particular one: agent-based, which will be described next.

\subsection{Agent-Based Modelling and Simulation (ABM/S)}
ABM/S is a method of modelling and simulating a system where the global behaviour may be unknown but the behaviour and interactions of the parts making up the system is of knowledge \cite{wooldridge_introduction_2009}. Those parts, called Agents, are modelled and simulated out of which then the aggregate global behaviour of the whole system emerges. Thus the central aspect of ABM/S is the concept of an Agent which can be understood as a metaphor for a pro-active unit, able to spawn new Agents, and interacting with other Agents in a network of neighbours by exchange of messages. The implementation of Agents can vary and strongly depends on the programming language and the kind of domain the simulation and model is situated in. Whereas the majority of ABM/S are implemented in object-oriented (OO) languages e.g. Java, C++, this paper focuses on functional ones.