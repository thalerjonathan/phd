\section{Related Research}
Already noted in the Introduction, \cite{huberman_evolutionary_1993} where the first to discuss the differences update-strategies can make and introduced the terms of synchronous and asynchronous updates. They define to be synchronous as agents being updated in unison and asynchronous where one agent is updated and the others are held constant.

\medskip

\cite{a_framework_2008} give an approach for ABS on GPUs which is a very different approach to updating and iterating agents in ABS. They discuss execution order at length, highlight the problem of inducing a specific execution-order in a model which is problematic for parallel execution and give solutions how to circumvent these shortcomings. Although we haven't mapped our ideas to GPUs we explicitly include an approach for data-parallelism which, we hypothesize, can be utilized to roughly map their approach onto our terminology. 
	
\medskip
	
\cite{botta_time_2010} sketch a minimal ABS implementation in Haskell which is very similar in the basic structure of ours. This proves that our approach seems to be a very natural one to apply to Haskell. Their focus is primarily on economic simulations and instead of iterating a simulation with a global time, their focus is on how to synchronize agents which have internal, local transition times. Although their work uses Haskell as well, our focus is very different from theirs and approaches ABS in a more general and comprehensive way.

\medskip

\cite{dawson_opening_2014} describe basic inner workings of ABS environments and compare their implementation in C++ to the existing ABS environment AnyLogic which is programmed in Java. They explicitly mention asynchronous and synchronous time-models and compare them in theory but unfortunately couldn't report the results of asynchronous updates due to limited space. They interpret asynchronous time-models to be the ones in which an agent acts at random time intervals and synchronous time-models where agents are updated all in same time intervals.

\medskip

\cite{yuxuan_agent-based_2016} presents in his Master-Thesis a comprehensive discussion on how to implement an ABS for state-charts in Java and also mentions synchronous and asynchronous time-models. He identifies the asynchronous time-model to be one in which updates are triggered by the exchange of messages and the synchronous ones which trigger changes immediately without the indirection of messages.

\medskip

We observe that there seems to be a variety of meanings attributed to the terminology of asynchronous and synchronous updates but the very semantic and technical details are unclear and not described very precisely. In the next section we will address this issue by presenting the basic background and propose properties for a new terminology from which we can derive common update-strategies.



\cite{jankovic_functional_2007} discuss using functional programming for discrete event simulation (DES) and mention the paradigm of Functional Reactive Programming (FRP) to be very suitable to DES. We were aware of the existence of this paradigm and have experimented with it using the library Yampa, but decided to leave that topic to a side and really keep our implementation clear and very basic.

It is important to note that the amount of research on using Haskell in the field of ABS has so far been moderate. Though there exist a few papers which look into Haskell and ABS \cite{de_jong_suitability_2014}, \cite{sulzmann_specifying_2007}, \cite{jankovic_functional_2007} all treat Haskell in this context very generally and focus primarily on how to specify Agents. An interesting library for Discrete Event Simulation (DES) and System Dynamics (SD) in Haskell called \textit{Aivika 3} is described in \cite{sorokin_aivika_2015}. It also comes with very basic features for ABS but only allows to specify simple state-based Agents with timed transitions. Thus this papers is investigating Haskell in a more fundamental way by looking into its suitability in implementing update-strategies in ABS, something not looked at in the ABS community thus presenting an original novelty as well.

There already exists research using the Actor Model for ABS in the context of Erlang \cite{varela_modelling_2004}, \cite{di_stefano_using_2005}, \cite{di_stefano_exat:_2007}, \cite{sher_agent-based_2013}