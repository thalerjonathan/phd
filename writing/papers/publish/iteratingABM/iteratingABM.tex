%% ----------------------------------------------------------------------
%% 
%% apa6 - A LaTeX class for formatting documents in compliance with the
%% American Psychological Association's Publication Manual, 6th edition
%% 
%% Copyright (C) 2011-2016 by Brian D. Beitzel <brian at beitzel.com>
%% 
%% This work may be distributed and/or modified under the
%% conditions of the LaTeX Project Public License (LPPL), either
%% version 1.3c of this license or (at your option) any later
%% version.  The latest version of this license is in the file:
%% 
%% http://www.latex-project.org/lppl.txt
%% 
%% Users may freely modify these files without permission, as long as the
%% copyright line and this statement are maintained intact.
%% 
%% This work is not endorsed by, affiliated with, or probably even known
%% by, the American Psychological Association.
%% 
%% ----------------------------------------------------------------------

\documentclass[a4paper, 10pt, conference]{IEEEconf}

\usepackage[utf8]{inputenc}
\usepackage{graphicx} % Required for the inclusion of images
\usepackage{amsmath} % Required for some math elements 
\usepackage[autostyle=true]{csquotes}
\usepackage{hyperref}
\usepackage{amssymb}
\usepackage{caption} 
\usepackage{hhline}
\usepackage{float}
\usepackage{listings}
\usepackage{subcaption}
\usepackage{apacite}

\title{The Art of Iterating:\\Update-Strategies in Agent-Based Simulation}

\author{
	Jonathan Thaler \\
	\email{jonathan.thaler@nottingham.ac.uk} \\
	\begin{affiliation}
		School of Computer Science, University of Nottingham
	\end{affiliation} \\
	\and 
	Peer-Olaf Siebers \\
	\email{peer-olaf.siebers@nottingham.ac.uk} \\
	\begin{affiliation}
		School of Computer Science, University of Nottingham
	\end{affiliation} \\
}

\begin{document}


\maketitle

\begin{abstract}
When developing a model for an Agent-Based Simulation (ABS) it is very important to select the update-strategy which reflects the semantics of the model because simulation results can vary vastly across different update-strategies. This awareness, we claim, is still lacking in the field of ABS. In this paper our contribution is to derive general properties of ABS and use them to classify all update-strategies possible in ABS. This will allow implementers and researchers in this field to use a general terminology, removing ambiguities when discussing ABS and their models. We will give results of simulating a discrete and a continuous game using our update-strategies and show that in the case of the discrete game only one specific strategy seems to be able to produce its emergent patterns whereas the pattern of the continuous game seems to be robust under varying update-strategies.
\end{abstract}

%*******************************************************************************
%*********************************** First Chapter *****************************
%*******************************************************************************

\chapter{Introduction}  %Title of the First Chapter
I noticed that it is pretty hard to convince an agent-based economics specialist who is not a computer scientist about a pure functional approach. My conjecture is that the implementation technique and method does not matter much to them because they have very little knowledge about programming and are almost always self-taught - they don't know about software-engineering, nothing about proper software-design and architecture, nothing about software-maintenance, nothing about unit-testing,... In the end they just "hack" the simulation in whatever language they are able to: C++, Visual Basic, Java or toolboxes like Netlogo. For them it is all about to \textit{get things done somehow} and not to get things done the right way or in a beautiful way - the way and the method doesn't matter, its just a necessary evil which needs to be done. Thus if functional programming could make their lives easier, then they will definitely welcome it. But functional programming is, i think, harder to learn and harder to understand - so one needs to provide an abstraction through EDSL. So I REALLY need to come up with convincing arguments why to use pure functional approaches in ACE THEY can understand, otherwise I will be lost and not heard (not published,...). \\

What ACE economists care for:

\begin{itemize}
\item Very: Qualitative modelling with quantitative results
\item Yes: Easy reproducibility
\item Likely: Reasoning about convergence?
\item Likely: EDSL
\end{itemize}

My contributions are: pure functional framework, functional agent-model for market-simulations, EDSL for market-simulations, qualitative / implicit modelling with quanitative results, reasoning in my framework about convergence \\

IDEA: could I develop non-causal modelling (models are expressed in terms of non-directed equations, modelled in signal-relations) to allow for qualitative modelling for the agent-based economists? See hybrid modelling paper of Yampa. \textbf{THIS WOULD BE A HUGE NOVEL CONTRIBUTION TO ACE ESPECIALLY WHEN COMBINED WITH AN EDSL AND PROVIDING FULL REFERENTIAL TRANSPARENCY TO KEEP THE ABILITY TO REASON ABOUT CONVERGENCE}. This should be covered in the "EDSL"-paper.

TODO: maybe i should really focus only on market models? otherwise too much? \\

central novelty of my PhD: model specification = runnable code. possible through EDSL. but only in specific subfield of ACE: market-models. need a functional description of the model, then translate it to model specification in EDSL and then run it to see dynamics. But: model specification moves closer to functional programming languages. \\

another novelty approach: model specification through qualitative instead of quantiative approaches. is this possible? \\

WHY FUNCTIONAL? "because its the ultimate approach to scientific computing": fewer bugs due to mutable state (why? is thos shown obkectively by someone?), shorter (again as above, productivity), more expressive and closer to math, EDSL, EDSL=model=simulation, better parallelising due to referental transparency, reasoning \\

scientific results need to be reproduced, especially when they have high impact. a more formal approach of specifying the model and the simulation (model=simulation) could lead to easier sharing and easier reporduction without ambigouites \\

pure functional agent-model \& theory, EDSL framework in Haskell for ACE

\begin{enumerate}
\item Which kind of problem do we have?
\item What aim is there? Solving the problem? 
\item How the aim is achieved by enumerating VERY CLEAR objectives.
\item What the impact one expects (hypothesis) and what it is (after results).
\end{enumerate}

Note: It is not in the interest of the researcher to develop new economic theories but to research the use of functional methods (programming and specification) in agent-based computational economics (ACE).

NOTE: Get the reader’s attention early in the introduction: motivation, significance, originality and novelty.

\section{Methods}
Methods need to be selected to implement the simulations. Special emphasis will be put on functional ones which will then be compared to established methods in the field of ABM/S and ACE. \\

Claim: non-programming environments are considered to be not powerful enough to capture the complexity of ACE implementations thus a programming approach to ACE will be always required.

\section{Scenarios}
To apply and test functional methods in ACE, four scenarios of ACE are selected and then the methods applied and compared with each other to see how each of them perform in comparison. The 4 selected scenarios represent a selection of the challenges posed in ACE: from very abstract ones to very operational ones.

\section{Comparison}
Each of the selected scenarios is then implemented using the selected methods where each solution is then compared against the following criteria: 

\begin{enumerate}
\item suitability for scientific computation
\item robustness
\item error-sources
\item testability
\item stability
\item extendability
\item size of code
\item maintainability
\item time taken for development
\item verification \& correctness
\item replications \& parallelism
\item EDSL
\end{enumerate}

This will then allow to compare the different methods against each other and to show under which circumstances functional methods shine and when they should not be used.

\section{Agent-Based Modelling and Simulation (ABM/S)}
ABM/S is a method of modelling and simulating a system where the global behaviour may be unknown but the behaviour and interactions of the parts making up the system is of knowledge (Wooldrige, M. (2009). An Introduction to MultiAgent Systems. John Wiley & Sons). Those parts, called agents, are modelled and simulated out of which then the aggregate global behaviour of the whole system emerges. Thus the central aspect of ABM/S is the concept of an Agent which can be understood as a metaphor for a pro-active unit, able to spawn new Agents, and interacting with other Agents in a network of neighbours by exchange of messages. The implementation of Agents can vary and strongly depends on the programming language and the kind of domain the simulation and model is situated in.

\section{Agent-Based Economics (ACE)}
According to Leigh Tesfatsion (Tesfatsion, L. (2006). Agent-based computational economics: A constructive approach to economic theory. In Tesfatsion, L. and Judd, K. L., editors, Handbook of Computational Economics, volume 2, chapter 16, pages 831–880. Elsevier, 1 edition.), one of the leading figures, ACE is "[...] computational modelling of economic processes (including whole economies) as open-ended dynamic systems of interacting agents." - thus lending perfectly to the use of ABM/S as already the name suggests. Whereas classical economic models fall short by only looking at the average, pure rational, individual interacting in anonymous markets, the ACE approach looks at heterogeneous, non-rational individuals interacting with each other in networks (Kirman, A. (2010). Complex Economics: Individual and Collective Rationality. Routledge, London ; New York, NY.). Thus ACE can be understood as a combination of computer-science, cognitive/social science and evolutionary economics.

\section{Functional programming}
TODO: read \cite{Backus1978}

The state-of-the-art approach to implementing Agents are object-oriented methods and programming as the metaphor of an Agent as presented above lends itself very naturally to object-orientation (OO). The author of this thesis claims that OO in the hands of inexperienced or ignorant programmers is dangerous, leading to bugs and hardly maintainable and extensible code. The reason for this is that OO provides very powerful techniques of organising and structuring programs through Classes, Type Hierarchies and Objects, which, when misused, lead to the above mentioned problems. Also major problems, which experts face as well as beginners are 1. state is highly scattered across the program which disguises the flow of data in complex simulations and 2. objects don’t compose as well as functions. The reason for this is that objects always carry around some internal state which makes it obviously much more complicated as complex dependencies can be introduced according to the internal state.
All this is tackled by (pure) functional programming which abandons the concept of global state, Objects and Classes and makes data-flow explicit. This then allows to reason about correctness, termination and other properties of the program e.g. if a given function exhibits side-effects or not. Other benefits are fewer lines of code, easier maintainability and ultimately fewer bugs thus making functional programming the ideal choice for scientific computing and simulation and thus also for ACE. A very powerful feature of functional programming is Lazy evaluation. It allows to describe infinite data-structures and functions producing an infinite stream of output but which are only computed as currently needed. Thus the decision of how many is decoupled from how to (Hughes, J. (1989). Why functional programming matters. Comput. J., 32(2):98–107.).
The most powerful aspect using pure functional programming however is that it allows the design of embedded domain specific languages (EDSL). In this case one develops and programs primitives e.g. types and functions in a host language (embed) in a way that they can be combined. The combination of these primitives then looks like a language specific to a given domain, in the case of this thesis ACE. The ease of development of EDSLs in pure functional programming is also a proof of the superior extensibility and composability of pure functional languages over OO (Henderson P. (1982). Functional Geometry. Proceedings of the 1982 ACM Symposium on LISP and Functional Programming.).
One of the most compelling example to utilize pure functional programming is the reporting of Hudak (Hudak P., Jones M. (1994). Haskell vs. Ada vs. C++ vs. Awk vs. ... An Experiment in Software Prototyping Productivity. Department of Computer Science, Yale University.)  where in a prototyping contest of DARPA the Haskell prototype was by far the shortest with 85 lines of code. Also the Jury mistook the code as specification because the prototype did actually implement a small EDSL which is a perfect proof how close EDSL can get to and look like a specification.

Functional languages can best be characterized by their way computation works: instead of \textit{how} something is computed, \textit{what} is computed is described. Thus functional programming follows a declarative instead of an imperative style of programming. The key points are:
\begin{itemize}
\item No assignment statements - variables values can never change once given a value.
\item Function calls have no side-effect and will only compute the results - this makes order of execution irrelevant, as due to the lack of side-effects the logical point in \textit{time} when the function is calculated within the program-execution does not matter.
\item higher-order functions
\item lazy evaluation
\item Looping is achieved using recursion, mostly through the use of the general fold or the more specific map.
\item Pattern-matching
\end{itemize}

This alone does not really explain the \textit{real} advantages of functional programming and one must look for better motivations using functional programming languages. One motivation is given in \cite{Hughes1989} which is a great paper explaining to non-functional programmers what the significance of functional programming is and helping functional programmers putting functional languages to maximum use by showing the real power and advantages of functional languages. The main conclusion is that \textit{modularity}, which is the key to successful programming, can be achieved best using higher-order functions and lazy evaluation provided in functional languages like Haskell. \cite{Hughes1989} argues that the ability to divide problems into sub-problems depends on the ability to glue the sub-problems together which depends strongly on the programming-language and \cite{Hughes1989} argues that in this ability functional languages are superior to structured programming.

TODO: comparison of functional and object-oriented programming. My points are:
\begin{itemize}
\item The way state can be changed and treated - distributed over multiple objects - is often very difficult to understand.
\item Inheritance is a dangerous thing if not used with care because inheritance introduces very strong dependencies which cannot be changed during runtime anymore.
\item Objects don't compose very well: \url{http://zeroturnaround.com/rebellabs/why-the-debate-on-object-oriented-vs-functional-programming-is-all-about-composition/}
\item (Nearly) impossible to reason about programs
\end{itemize}

In conclusion the upsides of functional programming as opposed to OO are:
\begin{itemize}
\item Much more explicit flow of data \& control
\item Much better compose-able
\item Much better parallelism
\end{itemize}

\section{Related Research}
Tim Sweeney, CTO of Epic Games gave an invited talk about how "future programming languages could help us write better code" by "supplying stronger typing, reduce run-time failures;  and the need for pervasive concurrency support, both implicit and explicit, to effectively exploit the several forms of parallelism present in games and graphics." \cite{Sweeney2006}. Although the fields of games and agent-based simulations seem to be very different in the end, they have also very important similarities: both are simulations which perform numerical computations and update objects - in games they are called "game-objects" and in abm they are called agents but they are in fact the same thing - in a loop either concurrently or sequential. His key-points were:

\begin{itemize}
\item Dependent types as the remedy of most of the run-time failures.
\item Parallelism for numerical computation: these are pure functional algorithms, operate locally on mutable state. Haskell ST, STRef solution enables encapsulating local heaps and mutability within referentially transparent code.
\item Updating game-objects (agents) concurrently using STM: update all objects concurrently in arbitrary order, with each update wrapped in atomic block - depends on collisions if performance goes up.
\end{itemize}

\section{Background}

\subsection{Schelling Segregation}
We follow in our implementation the original paper of Schelling as in \cite{schelling_dynamic_1971} where we focus on the \textit{Area Distribution} section (Schelling starts with movement in a linear, 1-dimensional world where agents are able to move to the nearest point which meets the agents satisfaction but this is not what we follow here). One assumes a discrete 2-dimensional lattice-world with NxM fields. Each field is either occupied by an agent of a given color (e.g. Red or Green) or is free. Each field has 8 neighbours, which denotes a Moore-Neighbourhood. In Schellings original work the lattice-world is limited at its borders but we assume a torus world which is wrapped around in both the x- and y-dimensions resulting in 8 neighbours also for fields at the border. The occupation density was set by Schelling to be about 70\%-75\% which he identifies as being a setting which allows the agents to move around freely without making the lattice-world too sparse.
Now the agents make their move sequentially one after another. In each move an agent calculates the number of neighbours which are of equal color. If the number satisfies the agents needs about the neighbourhood then the agent is regarded as being 'happy' and will stay on this field. On the other hand the agent moves to the nearest unoccupied field which satisfies its needs. An agent which moves selects an unoccupied place randomly relative from its current place within a rectangle of side-length 2r where its current place is at the center. The interpretation for that behaviour is that agents won't move too far as it could be costly. Also children might attend a school in this area or the family has friends in this area, so they don't want to break that.



Agents just move depending on their movement-strategy to another place if they are not happy on the current one - they don't care how the target place is in the present or in the future, they will decide again in the next time-step. The interpretation for that behaviour is: agents want to 'just get out' at any cost, not caring what the future place will look like - it might be better or worse but they will see then.

\subsubsection{Optimizing behaviour}
TODO: define utility

The original schelling model didn't have a move-optimizing behaviour, meaning agents are just binary: if it is happy it will not move, if it is unhappy it will move but they won't care where they move. We introduce local move-optimizing behaviours which can be interpreted as being realistic in the real-world. It is important to note that we focus on \textit{local} instead of \textit{global} move-optimization: the agents are limited in their reasoning-capabilities and have limited information available: they cannot check out \textit{every} place and pick the globally best one.\\

\subsubsection{Anticipating behaviour}
Schelling explicitly mentions in \cite{schelling_dynamic_1971} that nobody anticipates moves of others. This is what we introduce using the recursive simulation.

TODO: is this optimizing behaviour in the spirit of schellings original work? 

\paragraph{Optimizing future} Agents pick an unoccupied random place and move to it if it increases their utility in the future. The interpretation for that behaviour is: agents heard about a place which will be cool in the future.

\paragraph{Optimizing present \& future} Agents pick an unoccupied random place and move to it if it increases their utility in the now and in the future. The interpretation for that behaviour is: agents heard about a cool spot in town, check it out and move to it if they like it but they also anticipate the coolness of the place in the future and if it seems that the place is going down then they won't move there.

\subsection{Related Research}
TODO: \cite{kirman_complex_2010} mention kirman complex economics where he investigates the model more in depth


\section{Update-Strategies}
In this section we present the four general update-strategies which are possible in ABS. We give the list of all properties presented in the previous section, give a short description of the strategy and discuss their semantics and variations. We will discuss all details programming-language agnostic, give semantic meanings and interpretations of them and the implications selecting update-strategies for a model.

\subsection{Sequential Strategy}
\textbf{Iteration-Order:} Sequential \\
\textbf{Global Synchronization:} Yes \\
\textbf{Thread of Execution:} Shared \\
\textbf{Message-Handling:} Immediate or Queued \\
\textbf{Visibility of Changes:}	In-Iteration \\
\textbf{Repeatability:}	Deterministic 
	
\paragraph{Description:} This strategy has a globally synchronized time-flow and in each time-step iterates through all the agents and updates one Agent after another. Messages sent and changes to the environment made by Agents are visible immediately. 

\paragraph{Semantics:} There is no source of randomness and non-determinism thus rendering this strategy to be completely deterministic in each step. Messages can be processed either immediately or queued depending on the semantics of the model. If the model requires to process the messages immediately the model must be free of potential recursions.

\paragraph{Variation:} If the sequential iteration from Agent [1..n] imposes an advantage over the Agents further ahead or behind in the queue (e.g. if it is of benefit when making choices earlier than others in auctions or later when more information is available) then one could use random-walk iteration where in each time-step the agents are shuffled before iterated. Note that although this would introduce randomness in the model the source is a random-number generator thus still deterministic. \\
Using this strategy it is very easy to create the illusion of a local-time for each Agent by adding a random-offset to the global time for every Agent. \\
If one wants to have a very specific ordering, e.g. 'better performing' Agents first, then this can be easily implemented too by exposing some sorting-criterion and sorting the list of Agents after each Iteration.

\subsection{Parallel Strategy}
\textbf{Iteration-Order:} Parallel \\
\textbf{Global Synchronization:} Yes \\
\textbf{Thread of Execution:} Separate (or Shared) \\
\textbf{Message-Handling:} Queued \\
\textbf{Visibility of Changes:}	Post-Iteration \\
\textbf{Repeatability:}	Deterministic 

\paragraph{Description:} This strategy has a globally synchronized time-flow and in each time-step iterates through all the agents and updates all Agents in parallel. Messages sent and changes to the environment made by Agents are visible in the next global step. We can think about this strategy that all Agents make their moves at the same time. 

\paragraph{Semantics:} If one wants to change the environment in a way that it would be visible to other Agents this is regarded as a systematic error in this strategy. First it is not logical because all actions are meant to happen at the same time and also it would implicitly induce an ordering thus violating the \textit{happens at the same time} idea. Thus we require different semantics for accessing the environment in this strategy. We introduce thus a \textit{global} environment which is made up of the set of \textit{local} environments. Each local environment is owned by an Agent thus there are as many local environments as there are Agents. The semantics are then as follows: in each step all Agents can \textit{read} the global environment and \textit{read/write} their local environment. The changes to a local environment are only visible \textit{after} the local step and can be fed back into the global environment after the parallel processing of the Agents. \\
It does not make a difference if the Agents are really computed in parallel or just sequentially, due to the isolation of actions, this has the same effect. Also it will make no difference if we iterate over the agents sequentially or randomly, the outcome \textit{has to be} the same: the strategy is event-ordering invariant as all events/updates happen \textit{virtually} at the \textit{same time}. Thus if one needs to have the semantics of writes on the whole (global) environment in ones model, then this strategy is not the right one and one should resort to one of the other strategies. A workaround would be to implement the global environment as an Agent with which the non-environment Agents can communicate via messages thus we introduce an ordering but which is then sorted in a controlled order by an Agent, something which is not possible in the case of a passive, non-agent environment.

\paragraph{Variation:} Using this strategy it is very easy to create the illusion of a local-time for each agent by adding a random-offset to the global time for every Agent.

\subsection{Concurrent Strategy}
\textbf{Iteration-Order:} Parallel \\
\textbf{Global Synchronization:} Yes \\
\textbf{Thread of Execution:} Separate \\
\textbf{Message-Handling:} Queued \\
\textbf{Visibility of Changes:}	In-Iteration \\
\textbf{Repeatability:}	Non-Deterministic 

\paragraph{Description:} This strategy has a globally synchronized time-flow and in each time-step iterates through all the agents and updates all Agents in parallel but all messages sent and changes to the environment are immediately visible. Thus this strategy can be understood as a more general form of the Parallel Strategy: all Agents run at the same time but with actions becoming visible immediately.

\paragraph{Semantics:} It is important to realize that, when running Agents in parallel which are able to see actions by others immediately, this is the very definition of concurrency: parallel execution with mutual read/write access to shared data. Of course this shared data-access needs to be synchronized which in turn will introduce event-orderings in the execution of the Agents. Thus at this point we have a source of inherent non-determinism: although when one ignores any hardware-model of concurrency, at some point we need arbitration to decide which Agent gets access first to a shared resource thus arriving at non-deterministic solutions. This has the very important influence that repeated runs with the same configuration of the Agents and the Model may lead to different results.

\paragraph{Variation:} Using this strategy it is very easy to create the illusion of a local-time for each agent by adding a random-offset to the global time for every Agent.



\subsection{Actor Strategy}
\textbf{Iteration-Order:} Parallel \\
\textbf{Global Synchronization:} No \\
\textbf{Thread of Execution:} Separate \\
\textbf{Message-Handling:} Queued \\
\textbf{Visibility of Changes:}	In-Iteration \\
\textbf{Repeatability:}	Non-Deterministic 

\paragraph{Description:} This strategy has no globally synchronized time-flow but all the Agents run concurrently in parallel, with their own local time-flow. The messages and changes to the environment are visible as soon as the data arrive at the local Agents - this can be immediately when running locally on a multi-processor or with a significant delay when running in a cluster over a network. Obviously this is also a non-deterministic strategy and repeated runs with the same Agent and Model-configuration may (and will) lead to different results.

\paragraph{Semantics:} It is of most importance to note that information and thus also time in this strategy is always local to an Agent as each Agent progresses in its own speed through the simulation. Thus in this case one needs to explicitly \textit{observe} an Agent when one wants to e.g. visualize it. This observation is then only valid for this current point in time, local to the observer but not to the Agent itself, which may have changed immediately after the observation. This implies that we need to sample our Agents with observations when wanting to visualize them, which would inherently lead to well known sampling issues. A solution would be to invert the problem and create an Observer-Agent which is known to all Agents where each Agent sends a \textit{'I have changed'} message with the necessary information to the observer if it has changed its internal state. This also does not guarantee that the observations will really reflect the actual state the Agent is in but is a remedy against the notorious sampling. \\
This is the most general one of all the strategies as it can emulate all the others by introducing the necessary synchronization mechanisms and Agents. Also this concept was proposed by C. Hewitt in 1973 in his work \cite{hewitt_universal_1973} for which I. Grief in \cite{grief_semantics_1975} and W. Clinger in \cite{clinger_foundations_1981} developed semantics of different kinds. These works were very influential in the development of the concepts of Agents and and can be regarded as foundational basics for ABS.

\paragraph{Variation:} It is important to understand that this strategy is the most general one as it allows to simulate all other strategies using synchronization.

\section{Case-Studies}
TODO: emphasise different types of games/simulations: need to be very clear of the difference between the 2 games
TODO: experimentation description: which language, with which model, with which configuration, number of agents, number of replications,...
TODO: what about language-differences, which language did we use?

In this section we revisit the Prisoners Dilemma model of \cite{nowak_evolutionary_1992} and present the Heroes \& Cowards model of \cite{wilensky_introduction_2015} and show results simulating both with the four update-strategies. 

TODO: why 25\% heroes, why this big number of agents 100.000?

TODO: what do I want to achieve with this test? it is basically to show 1st: that only the parallel-strategy produces the results which match the ones of the original paper in the prisoners dilemma, 2nd: that the other update-strategies create completely different results, 3rd: the heroes \& cowards game seems to be unaffected by different update-strategies.

\subsection{Effect of using different Update-Strategies}

\begin{table*}[t]
	\begin{tabular}{c c c}
		& Prisoners Dilemma & Heroes \& Cowards \\ 

		\textit{\rotatebox{90}{sequential strategy}}
		&
		\begin{subfigure}[b]{0.4\textwidth}
			\centering
			\includegraphics[width=.7\textwidth, angle=0]{./fig/seq_99x99_436steps_MSG_haskell.png}
			\caption{}
			\label{fig:pd_seq}
		\end{subfigure}
    	&
		\begin{subfigure}[b]{0.4\textwidth}
			\centering
			\includegraphics[width=.7\textwidth, angle=0]{./fig/seq_HAC_100_000_500steps_java.png}
			\caption{}
			\label{fig:hac_seq}
		\end{subfigure}
    	\\
    	
    	\textit{\rotatebox{90}{parallel strategy}}
		&
		\begin{subfigure}[b]{0.4\textwidth}
			\centering
			\includegraphics[width=.7\textwidth, angle=0]{./fig/par_99x99_436steps_MSG_haskell.png}
			\caption{}
			\label{fig:pd_par}
		\end{subfigure}
    	&
		\begin{subfigure}[b]{0.4\textwidth}
			\centering
			\includegraphics[width=.7\textwidth, angle=0]{./fig/par_HAC_100_000_500steps_java.png}
			\caption{}
			\label{fig:hac_par}
		\end{subfigure}
    	\\
    	
    	\textit{\rotatebox{90}{concurrent strategy}}
		&
		\begin{subfigure}[b]{0.4\textwidth}
			\centering
			\includegraphics[width=.7\textwidth, angle=0]{./fig/con_99x99_436steps_MSG_haskell.png}
			\caption{}
			\label{fig:pd_con}
		\end{subfigure}
    	&
		\begin{subfigure}[b]{0.4\textwidth}
			\centering
			\includegraphics[width=.7\textwidth, angle=0]{./fig/con_HAC_100_000_500steps_java.png}
			\caption{}
			\label{fig:hac_con}
		\end{subfigure}
    	\\ 
    	
    	\textit{\rotatebox{90}{actor strategy}}
		&
		\begin{subfigure}[b]{0.4\textwidth}
			\centering
			\includegraphics[width=.7\textwidth, angle=0]{./fig/act_99x99_436steps_MSG_haskell.png}
			\caption{}
			\label{fig:pd_act}
		\end{subfigure}
    	&
    	% TODO: this the same picture as in concurrent version from java, put in actor-version generated by scala    
		\begin{subfigure}[b]{0.4\textwidth}
			\centering
			\includegraphics[width=.7\textwidth, angle=0]{./fig/act_HAC_100_000_500steps_scala.png}
			\caption{}
			\label{fig:hac_act}
		\end{subfigure}
    	\\ \hline
	\end{tabular}
	
	\caption{\small Results of Prisoners Dilemma and Heroes \& Cowards with all four update-strategies.} 
	\label{fig:results}
\end{table*}

%\begin{figure*}
%
%	 \centering
	
%    \begin{subfigure}[b]{0.4\textwidth}
%			\centering
%       	\includegraphics[width=.7\textwidth, angle=0]{./fig/seq_99x99_436steps_MSG_haskell.png}
%        \caption{\textit{sequential} Prisoners Dilemma}
%        \label{fig:pd_seq}
%    \end{subfigure}
%    \begin{subfigure}[b]{0.4\textwidth}
%		\centering
%        \includegraphics[width=.7\textwidth, angle=0]{./fig/seq_HAC_100_000_500steps_java.png}
%        \caption{\textit{sequential} Heroes \& Cowards}
%        \label{fig:hac_seq}
%    \end{subfigure}
%       

%    \begin{subfigure}[b]{0.4\textwidth}
%		\centering
%       	\includegraphics[width=.7\textwidth, angle=0]{./fig/par_99x99_436steps_MSG_haskell.png}
%        \caption{\textit{parallel} Prisoners Dilemma}
%        \label{fig:pd_par}
%    \end{subfigure}
%    \begin{subfigure}[b]{0.4\textwidth}
%    	\centering
%        \includegraphics[width=.7\textwidth, angle=0]{./fig/par_HAC_100_000_500steps_java.png}
%        \caption{\textit{parallel} Heroes \& Cowards}
%        \label{fig:hac_par}
%    \end{subfigure}
%        
%
%    \begin{subfigure}[b]{0.4\textwidth}
%		\centering
%       	\includegraphics[width=.7\textwidth, angle=0]{./fig/con_99x99_436steps_MSG_haskell.png}
%        \caption{\textit{concurrent} Prisoners Dilemma}
%        \label{fig:pd_con}
%    \end{subfigure}
%    \begin{subfigure}[b]{0.4\textwidth}
%    	\centering
%        \includegraphics[width=.7\textwidth, angle=0]{./fig/con_HAC_100_000_500steps_java.png}
%        \caption{\textit{concurrent} Heroes \& Cowards}
%        \label{fig:hac_con}
%    \end{subfigure}
%
%
%    \begin{subfigure}[b]{0.4\textwidth}
%		\centering
%       	\includegraphics[width=.7\textwidth, angle=0]{./fig/act_99x99_436steps_MSG_haskell.png}
%        \caption{\textit{actor} Prisoners Dilemma}
%        \label{fig:pd_act}
%    \end{subfigure}  
%    \begin{subfigure}[b]{0.4\textwidth}
%    	\centering
%        \includegraphics[width=.7\textwidth, angle=0]{./fig/act_HAC_100_000_500steps_scala.png}
%        \caption{\textit{actor} Heroes \& Cowards}
%        \label{fig:hac_act}
%    \end{subfigure}
% TODO: this the same picture as in concurrent version from java, put in actor-version generated by scala    
% TODO: this the same picture as in concurrent version from java, put in actor-version generated by scala    
% TODO: this the same picture as in concurrent version from java, put in actor-version generated by scala    
% TODO: this the same picture as in concurrent version from java, put in actor-version generated by scala    
% TODO: this the same picture as in concurrent version from java, put in actor-version generated by scala    
% TODO: this the same picture as in concurrent version from java, put in actor-version generated by scala  

%	\caption{\small Results of Prisoners Dilemma and Heroes \& Cowards with all four update-strategies.} 
%	\label{fig:results}
%\end{figure*}

When looking at figure \ref{fig:results} the update-strategy which reflects the semantics of the model is the Parallel Strategy as all others clearly fail to reproduce the pattern as shown by the results in the original paper TODO: in figure \ref{fig:sync_patterns}. We can imply that only the Parallel Strategy is suitable to simulate this model because only that strategy is the one which renders the results of the original paper, meaning it is the 'correct' strategy. \\
The reason why the others fail to reproduce the pattern is due to the non-parallel and unsynchronized way that information spreads through the grid. In the Sequential Strategy the agents further ahead in the queue play the game earlier and influence the neighbourhood so agents in the neighbourhood which play the game later experience an already changed environment and  messages in their queue and act differently based upon these informations. This is not the case in the Parallel version where all agents play the game on the frozen state of the previous step and the outcome of each agents game will only be visible in the next step. In the Concurrent and Actor Strategy the agents run in parallel but changes are visible immediately and concurrently, leading to the same non-structural patterns as in the Sequential Strategy. \\
Note that the Concurrent and Actor Strategy produce different results on every run due to the inherent non-deterministic event-ordering introduce by concurrency. Also note that it is not possible to calculate 45 steps for the Actor Strategy as it lacks the Global Synchronization property. To arrive at a relative comparative result we just waited until the first agent arrives at a local time of 45 and then rendered the result. 

\subsection{Heroes \& Cowards}
Although the individual agent-positions of runs with the same configuration differ between update-strategies we experienced the forming of the emergent cross-pattern as seen in figure \ref{fig:results} in all four update-strategies. We can conclude that the Heroes \& Cowards model seems to be more robust to the selection of its update-strategy and that its emergent property - the formation of the cross - is stable under differing update-strategies. One would not see a difference between the different strategies so only one picture was included. Note that to test the Actor Strategy with a this high number of agents we used our implementation in Scala with Actors as Java is not able to have this high number of threads and our Haskell implementation suffers from performance issues, resorting to Scala with Actors. The results were nearly the same there, showing the big green emergent cross-pattern in the center but lacking the smaller red crosses in each section, something we attribute to the local-time of each agent and the relativity of observing the simulation.

\section{Conclusions}
\label{sec:conclusions}

Our approach is radically different from traditional approaches in the ABS community. First it builds on the already quite powerful FRP paradigm. Second, due to our continuous time approach, it forces one to think properly of time-semantics of the model and how small $\Delta t$ should be. Third it requires to think about agent interactions in a new way instead of being just method-calls.

Because no part of the simulation runs in the IO Monad and we do not use unsafePerformIO we can rule out a serious class of bugs caused by implicit data-dependencies and side-effects which can occur in traditional imperative implementations.

Also we can statically guarantee the reproducibility of the simulation, which means that repeated runs with the same initial conditions are guaranteed to result in the same dynamics. Although we allow side-effects within agents, we restrict them to only the Random and State Monad in a controlled, deterministic way and never use the IO Monad which guarantees the absence of non-deterministic side effects within the agents and other parts of the simulation.

Determinism is also ensured by fixing the $\Delta t$ and not making it dependent on the performance of e.g. a rendering-loop or other system-dependent sources of non-determinism as described by \cite{perez_testing_2017}. Also by using FRP we gain all the benefits from it and can use research on testing, debugging and exploring FRP systems \cite{perez_testing_2017, perez_back_2017}.

\subsection*{Issues}
Currently, the performance of the system is not comparable to imperative implementations but our research was not focusing on this aspect. We leave the investigation and optimization of the performance aspect of our approach for further research.

Despite the strengths and benefits we get by leveraging on FRP, there are errors that are not raised at compile time, e.g. we can still have infinite loops and run-time errors. This was for example investigated in \cite{sculthorpe_safe_2009} where the authors use dependent types to avoid some run-time errors in FRP. We suggest that one could go further and develop a domain specific type system for FRP that makes the FRP based ABS more predictable and that would support further mathematical analysis of its properties. Furthermore, moving to dependent types would pose a unique benefit over the traditional object-oriented approach and should allow us to express and guarantee even more properties at compile time. We leave this for further research.

In our pure functional approach, agent identity is not as clear as in traditional object-oriented programming, where an agent can be hidden behind a polymorphic interface which is much more abstract than in our approach. Also the identity of an agent is much clearer in object-oriented programming due to the concept of object-identity and the encapsulation of data and methods.

We can conclude that the main difficulty of a pure functional approach evolves around the communication and interaction between agents, which is a direct consequence of the issue with agent identity. Agent interaction is straight-forward in object-oriented programming, where it is achieved using method-calls mutating the internal state of the agent, but that comes at the cost of a new class of bugs due to implicit data flow. In pure functional programming these data flows are explicit but our current approach of feeding back the states of all agents as inputs is not very general and we have added further mechanisms of agent interaction which we had to omit due to lack of space.

\bibliographystyle{apacite}
\bibliography{../../../references/phdReferences_apa6}

%\bibliographystyle{apalike}
%\bibliography{../../../references/phdReferences}

\end{document}