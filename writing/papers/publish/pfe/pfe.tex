%\documentclass[format=acmsmall, review=false, screen=true]{acmart}		% ICFP
%\documentclass[format=sigplan, review=true]{acmart}		% HASKELL SYMPOSIUM 
\documentclass[format=sigconf, review=true]{acmart}		% IFL

\usepackage{float}
\usepackage{graphicx}
\usepackage{subcaption}
\usepackage{ifthen}
\usepackage{minted}
\usepackage{verbatim}

% Metadata Information
%% use defaults for review submission.
%\acmConference[HS18]{Haskell Symposium}{2018}{09}
%\acmYear{2018}
%\copyrightyear{2018}
\acmConference[IFL'18]{International Symposium on Implementation and Application of Functional Languages}{August 2019}{Lowell, MA, USA}
\acmYear{2019}
\copyrightyear{2019}
%\acmDOI{} % \acmDOI{10.1145/nnnnnnn.nnnnnnn}

% Copyright
%% use 'none' for review submission.
\setcopyright{none}
%\setcopyright{acmcopyright}	% = copyright transfer to ACM
%\setcopyright{acmlicensed} 		% = retaining copyright but granting ACM exclusive publication rights
%\setcopyright{rightsretained}  % = open access on payment of a fee
%\setcopyright{usgov}
%\setcopyright{usgovmixed}
%\setcopyright{cagov}
%\setcopyright{cagovmixed}

% TODO : get the data
% DOI
% \acmDOI{0000001.0000001}

% TODO: fill in
% Paper history
\received{May 2018}
%\received[revised]{March 2018}
%\received[accepted]{March 2018}

% Document starts
\begin{document}

\newminted[HaskellCode]{haskell}{fontsize=\footnotesize}


% Title portion. Note the short title for running heads
\title[Pure Functional Epidemics]{Pure Functional Epidemics}
\subtitle{An Agent-Based Approach}

\author{Jonathan Thaler}
%\orcid{TODO}
\email{jonathan.thaler@nottingham.ac.uk}
\author{Thorsten Altenkirch}
\email{thorsten.altenkirch@nottingham.ac.uk}
\author{Peer-Olaf Siebers}
\email{peer-olaf.siebers@nottingham.ac.uk}
\affiliation{%
  \institution{University of Nottingham}
  \streetaddress{7301 Wollaton Rd}
  \city{Nottingham}
  \postcode{NG8 1BB}
  \country{United Kingdom}}

\begin{abstract}
%TODOs
%	TODO signposting: introduce things more clearly - we do this because...
%	OK we have much more space now: can we bring in sections from the first, too long version?
%	OK do we need all the pictures?
%	OK fix RNG correlation bug and compare dynamics
%		-> it seems that fixing this results in quite different dynamics. Compare dynamics of Random Monad as well: if it comes closer then maybe there is really something wrong with the correlation between the RNGs
%
%	FAIR fix referee comment: "Some entries in the references lack details for finding them without a search engine, e.g. 26, 27, 28. 
%	OK correlation in Yampa version: A) fix it or B) mention that this is actually a problem and simply wrong in Simulation (but not in a game), the next step solves that problem, and leave the bug in as an educational example where functional programming can encourage bugs if not carefully thinking about it	
%	OK dependent types: klarer machen was wir dort wollen (agent-interaction, internal state transitions, safe environment access, totality checking)	
%	OK fix referee comment: In an English text citations should not be nouns (the authors of [x]), they are parenthesised additions. Removing them should leave a readable text. 
%	OK fix referee comment "You often say "The authors of [XY] discuss/present ...". This is bad for two reasons. First of all, numeric references should
%	not be used as a part of the sentence, because the "[XY]" form typographically is an annotation, and also because there's no info conveyed by a number and flipping back and forth is not very nice. Furthermore, it is semantically not what I think you mean. The authors of a given paper may have done a lot of things in their lives, but I think you actually mean that the paper you refer to discusses/presents these things. If you simply name the authors and say "A and B discuss/present ... [XY]", then you avoid both problems.
%	OK Is reference 2 still in preparation, since 2005?"
%	OK performance: ignore as far as possible and only mention in further work
%	OK we started with high hopes: remove completely
%	OK section 4: put into background
%	OK related work vorne lassen
%	OK fix referee comment: Also citations such as [6], [9], [12] should be combined into [6, 9, 12].

%SPACE: reduce supersampling: dont provide yampa function
%SPACE: compress the step to msfs into adding Environment
%SPACE: shorten MSF example in Background
%SPACE: reduce code in paper as well in last step because hardly necessary anymore

Agent-Based Simulation (ABS) is a methodology in which a system is simulated in a bottom-up approach by modelling the micro interactions of its constituting parts, called agents, out of which the global system behaviour emerges.

So far mainly object-oriented techniques and languages have been used in ABS. Using the SIR model of epidemiology, which simulates the spreading of an infectious disease through a population, we demonstrate how to use pure Functional Reactive Programming to implement ABS. With our approach we can guarantee the reproducibility of the simulation at compile time and rule out specific classes of run-time bugs, something that is not possible with traditional object-oriented languages. Also, we found that the representation in a purely functional format is conceptually quite elegant and opens the way to formally reason about ABS.
\end{abstract}

%
% The code below should be generated by the tool at
% http://dl.acm.org/ccs.cfm
% Please copy and paste the code instead of the example below.
%
% TODO needs to be generated
%\begin{CCSXML}
%<ccs2012>
% <concept>
%  <concept_id>10010520.10010553.10010562</concept_id>
%  <concept_desc>Computer systems organization~Embedded systems</concept_desc>
%  <concept_significance>500</concept_significance>
% </concept>
% <concept>
%  <concept_id>10010520.10010575.10010755</concept_id>
%  <concept_desc>Computer systems organization~Redundancy</concept_desc>
%  <concept_significance>300</concept_significance>
% </concept>
% <concept>
%  <concept_id>10010520.10010553.10010554</concept_id>
%  <concept_desc>Computer systems organization~Robotics</concept_desc>
%  <concept_significance>100</concept_significance>
% </concept>
% <concept>
%  <concept_id>10003033.10003083.10003095</concept_id>
%  <concept_desc>Networks~Network reliability</concept_desc>
%  <concept_significance>100</concept_significance>
% </concept>
%</ccs2012>
%\end{CCSXML}
%
%\ccsdesc[500]{Computer systems organization~Embedded systems}
%\ccsdesc[300]{Computer systems organization~Redundancy}
%\ccsdesc{Computer systems organization~Robotics}
%\ccsdesc[100]{Networks~Network reliability}

%
% End generated code
%

\keywords{Functional Reactive Programming, Monadic Stream Functions, Agent-Based Simulation}

\maketitle

%*******************************************************************************
%*********************************** First Chapter *****************************
%*******************************************************************************

\chapter{Introduction}  %Title of the First Chapter
I noticed that it is pretty hard to convince an agent-based economics specialist who is not a computer scientist about a pure functional approach. My conjecture is that the implementation technique and method does not matter much to them because they have very little knowledge about programming and are almost always self-taught - they don't know about software-engineering, nothing about proper software-design and architecture, nothing about software-maintenance, nothing about unit-testing,... In the end they just "hack" the simulation in whatever language they are able to: C++, Visual Basic, Java or toolboxes like Netlogo. For them it is all about to \textit{get things done somehow} and not to get things done the right way or in a beautiful way - the way and the method doesn't matter, its just a necessary evil which needs to be done. Thus if functional programming could make their lives easier, then they will definitely welcome it. But functional programming is, i think, harder to learn and harder to understand - so one needs to provide an abstraction through EDSL. So I REALLY need to come up with convincing arguments why to use pure functional approaches in ACE THEY can understand, otherwise I will be lost and not heard (not published,...). \\

What ACE economists care for:

\begin{itemize}
\item Very: Qualitative modelling with quantitative results
\item Yes: Easy reproducibility
\item Likely: Reasoning about convergence?
\item Likely: EDSL
\end{itemize}

My contributions are: pure functional framework, functional agent-model for market-simulations, EDSL for market-simulations, qualitative / implicit modelling with quanitative results, reasoning in my framework about convergence \\

IDEA: could I develop non-causal modelling (models are expressed in terms of non-directed equations, modelled in signal-relations) to allow for qualitative modelling for the agent-based economists? See hybrid modelling paper of Yampa. \textbf{THIS WOULD BE A HUGE NOVEL CONTRIBUTION TO ACE ESPECIALLY WHEN COMBINED WITH AN EDSL AND PROVIDING FULL REFERENTIAL TRANSPARENCY TO KEEP THE ABILITY TO REASON ABOUT CONVERGENCE}. This should be covered in the "EDSL"-paper.

TODO: maybe i should really focus only on market models? otherwise too much? \\

central novelty of my PhD: model specification = runnable code. possible through EDSL. but only in specific subfield of ACE: market-models. need a functional description of the model, then translate it to model specification in EDSL and then run it to see dynamics. But: model specification moves closer to functional programming languages. \\

another novelty approach: model specification through qualitative instead of quantiative approaches. is this possible? \\

WHY FUNCTIONAL? "because its the ultimate approach to scientific computing": fewer bugs due to mutable state (why? is thos shown obkectively by someone?), shorter (again as above, productivity), more expressive and closer to math, EDSL, EDSL=model=simulation, better parallelising due to referental transparency, reasoning \\

scientific results need to be reproduced, especially when they have high impact. a more formal approach of specifying the model and the simulation (model=simulation) could lead to easier sharing and easier reporduction without ambigouites \\

pure functional agent-model \& theory, EDSL framework in Haskell for ACE

\begin{enumerate}
\item Which kind of problem do we have?
\item What aim is there? Solving the problem? 
\item How the aim is achieved by enumerating VERY CLEAR objectives.
\item What the impact one expects (hypothesis) and what it is (after results).
\end{enumerate}

Note: It is not in the interest of the researcher to develop new economic theories but to research the use of functional methods (programming and specification) in agent-based computational economics (ACE).

NOTE: Get the reader’s attention early in the introduction: motivation, significance, originality and novelty.

\section{Methods}
Methods need to be selected to implement the simulations. Special emphasis will be put on functional ones which will then be compared to established methods in the field of ABM/S and ACE. \\

Claim: non-programming environments are considered to be not powerful enough to capture the complexity of ACE implementations thus a programming approach to ACE will be always required.

\section{Scenarios}
To apply and test functional methods in ACE, four scenarios of ACE are selected and then the methods applied and compared with each other to see how each of them perform in comparison. The 4 selected scenarios represent a selection of the challenges posed in ACE: from very abstract ones to very operational ones.

\section{Comparison}
Each of the selected scenarios is then implemented using the selected methods where each solution is then compared against the following criteria: 

\begin{enumerate}
\item suitability for scientific computation
\item robustness
\item error-sources
\item testability
\item stability
\item extendability
\item size of code
\item maintainability
\item time taken for development
\item verification \& correctness
\item replications \& parallelism
\item EDSL
\end{enumerate}

This will then allow to compare the different methods against each other and to show under which circumstances functional methods shine and when they should not be used.

\section{Agent-Based Modelling and Simulation (ABM/S)}
ABM/S is a method of modelling and simulating a system where the global behaviour may be unknown but the behaviour and interactions of the parts making up the system is of knowledge (Wooldrige, M. (2009). An Introduction to MultiAgent Systems. John Wiley & Sons). Those parts, called agents, are modelled and simulated out of which then the aggregate global behaviour of the whole system emerges. Thus the central aspect of ABM/S is the concept of an Agent which can be understood as a metaphor for a pro-active unit, able to spawn new Agents, and interacting with other Agents in a network of neighbours by exchange of messages. The implementation of Agents can vary and strongly depends on the programming language and the kind of domain the simulation and model is situated in.

\section{Agent-Based Economics (ACE)}
According to Leigh Tesfatsion (Tesfatsion, L. (2006). Agent-based computational economics: A constructive approach to economic theory. In Tesfatsion, L. and Judd, K. L., editors, Handbook of Computational Economics, volume 2, chapter 16, pages 831–880. Elsevier, 1 edition.), one of the leading figures, ACE is "[...] computational modelling of economic processes (including whole economies) as open-ended dynamic systems of interacting agents." - thus lending perfectly to the use of ABM/S as already the name suggests. Whereas classical economic models fall short by only looking at the average, pure rational, individual interacting in anonymous markets, the ACE approach looks at heterogeneous, non-rational individuals interacting with each other in networks (Kirman, A. (2010). Complex Economics: Individual and Collective Rationality. Routledge, London ; New York, NY.). Thus ACE can be understood as a combination of computer-science, cognitive/social science and evolutionary economics.

\section{Functional programming}
TODO: read \cite{Backus1978}

The state-of-the-art approach to implementing Agents are object-oriented methods and programming as the metaphor of an Agent as presented above lends itself very naturally to object-orientation (OO). The author of this thesis claims that OO in the hands of inexperienced or ignorant programmers is dangerous, leading to bugs and hardly maintainable and extensible code. The reason for this is that OO provides very powerful techniques of organising and structuring programs through Classes, Type Hierarchies and Objects, which, when misused, lead to the above mentioned problems. Also major problems, which experts face as well as beginners are 1. state is highly scattered across the program which disguises the flow of data in complex simulations and 2. objects don’t compose as well as functions. The reason for this is that objects always carry around some internal state which makes it obviously much more complicated as complex dependencies can be introduced according to the internal state.
All this is tackled by (pure) functional programming which abandons the concept of global state, Objects and Classes and makes data-flow explicit. This then allows to reason about correctness, termination and other properties of the program e.g. if a given function exhibits side-effects or not. Other benefits are fewer lines of code, easier maintainability and ultimately fewer bugs thus making functional programming the ideal choice for scientific computing and simulation and thus also for ACE. A very powerful feature of functional programming is Lazy evaluation. It allows to describe infinite data-structures and functions producing an infinite stream of output but which are only computed as currently needed. Thus the decision of how many is decoupled from how to (Hughes, J. (1989). Why functional programming matters. Comput. J., 32(2):98–107.).
The most powerful aspect using pure functional programming however is that it allows the design of embedded domain specific languages (EDSL). In this case one develops and programs primitives e.g. types and functions in a host language (embed) in a way that they can be combined. The combination of these primitives then looks like a language specific to a given domain, in the case of this thesis ACE. The ease of development of EDSLs in pure functional programming is also a proof of the superior extensibility and composability of pure functional languages over OO (Henderson P. (1982). Functional Geometry. Proceedings of the 1982 ACM Symposium on LISP and Functional Programming.).
One of the most compelling example to utilize pure functional programming is the reporting of Hudak (Hudak P., Jones M. (1994). Haskell vs. Ada vs. C++ vs. Awk vs. ... An Experiment in Software Prototyping Productivity. Department of Computer Science, Yale University.)  where in a prototyping contest of DARPA the Haskell prototype was by far the shortest with 85 lines of code. Also the Jury mistook the code as specification because the prototype did actually implement a small EDSL which is a perfect proof how close EDSL can get to and look like a specification.

Functional languages can best be characterized by their way computation works: instead of \textit{how} something is computed, \textit{what} is computed is described. Thus functional programming follows a declarative instead of an imperative style of programming. The key points are:
\begin{itemize}
\item No assignment statements - variables values can never change once given a value.
\item Function calls have no side-effect and will only compute the results - this makes order of execution irrelevant, as due to the lack of side-effects the logical point in \textit{time} when the function is calculated within the program-execution does not matter.
\item higher-order functions
\item lazy evaluation
\item Looping is achieved using recursion, mostly through the use of the general fold or the more specific map.
\item Pattern-matching
\end{itemize}

This alone does not really explain the \textit{real} advantages of functional programming and one must look for better motivations using functional programming languages. One motivation is given in \cite{Hughes1989} which is a great paper explaining to non-functional programmers what the significance of functional programming is and helping functional programmers putting functional languages to maximum use by showing the real power and advantages of functional languages. The main conclusion is that \textit{modularity}, which is the key to successful programming, can be achieved best using higher-order functions and lazy evaluation provided in functional languages like Haskell. \cite{Hughes1989} argues that the ability to divide problems into sub-problems depends on the ability to glue the sub-problems together which depends strongly on the programming-language and \cite{Hughes1989} argues that in this ability functional languages are superior to structured programming.

TODO: comparison of functional and object-oriented programming. My points are:
\begin{itemize}
\item The way state can be changed and treated - distributed over multiple objects - is often very difficult to understand.
\item Inheritance is a dangerous thing if not used with care because inheritance introduces very strong dependencies which cannot be changed during runtime anymore.
\item Objects don't compose very well: \url{http://zeroturnaround.com/rebellabs/why-the-debate-on-object-oriented-vs-functional-programming-is-all-about-composition/}
\item (Nearly) impossible to reason about programs
\end{itemize}

In conclusion the upsides of functional programming as opposed to OO are:
\begin{itemize}
\item Much more explicit flow of data \& control
\item Much better compose-able
\item Much better parallelism
\end{itemize}

\section{Related Research}
Tim Sweeney, CTO of Epic Games gave an invited talk about how "future programming languages could help us write better code" by "supplying stronger typing, reduce run-time failures;  and the need for pervasive concurrency support, both implicit and explicit, to effectively exploit the several forms of parallelism present in games and graphics." \cite{Sweeney2006}. Although the fields of games and agent-based simulations seem to be very different in the end, they have also very important similarities: both are simulations which perform numerical computations and update objects - in games they are called "game-objects" and in abm they are called agents but they are in fact the same thing - in a loop either concurrently or sequential. His key-points were:

\begin{itemize}
\item Dependent types as the remedy of most of the run-time failures.
\item Parallelism for numerical computation: these are pure functional algorithms, operate locally on mutable state. Haskell ST, STRef solution enables encapsulating local heaps and mutability within referentially transparent code.
\item Updating game-objects (agents) concurrently using STM: update all objects concurrently in arbitrary order, with each update wrapped in atomic block - depends on collisions if performance goes up.
\end{itemize}

\section{Background}

\subsection{Schelling Segregation}
We follow in our implementation the original paper of Schelling as in \cite{schelling_dynamic_1971} where we focus on the \textit{Area Distribution} section (Schelling starts with movement in a linear, 1-dimensional world where agents are able to move to the nearest point which meets the agents satisfaction but this is not what we follow here). One assumes a discrete 2-dimensional lattice-world with NxM fields. Each field is either occupied by an agent of a given color (e.g. Red or Green) or is free. Each field has 8 neighbours, which denotes a Moore-Neighbourhood. In Schellings original work the lattice-world is limited at its borders but we assume a torus world which is wrapped around in both the x- and y-dimensions resulting in 8 neighbours also for fields at the border. The occupation density was set by Schelling to be about 70\%-75\% which he identifies as being a setting which allows the agents to move around freely without making the lattice-world too sparse.
Now the agents make their move sequentially one after another. In each move an agent calculates the number of neighbours which are of equal color. If the number satisfies the agents needs about the neighbourhood then the agent is regarded as being 'happy' and will stay on this field. On the other hand the agent moves to the nearest unoccupied field which satisfies its needs. An agent which moves selects an unoccupied place randomly relative from its current place within a rectangle of side-length 2r where its current place is at the center. The interpretation for that behaviour is that agents won't move too far as it could be costly. Also children might attend a school in this area or the family has friends in this area, so they don't want to break that.



Agents just move depending on their movement-strategy to another place if they are not happy on the current one - they don't care how the target place is in the present or in the future, they will decide again in the next time-step. The interpretation for that behaviour is: agents want to 'just get out' at any cost, not caring what the future place will look like - it might be better or worse but they will see then.

\subsubsection{Optimizing behaviour}
TODO: define utility

The original schelling model didn't have a move-optimizing behaviour, meaning agents are just binary: if it is happy it will not move, if it is unhappy it will move but they won't care where they move. We introduce local move-optimizing behaviours which can be interpreted as being realistic in the real-world. It is important to note that we focus on \textit{local} instead of \textit{global} move-optimization: the agents are limited in their reasoning-capabilities and have limited information available: they cannot check out \textit{every} place and pick the globally best one.\\

\subsubsection{Anticipating behaviour}
Schelling explicitly mentions in \cite{schelling_dynamic_1971} that nobody anticipates moves of others. This is what we introduce using the recursive simulation.

TODO: is this optimizing behaviour in the spirit of schellings original work? 

\paragraph{Optimizing future} Agents pick an unoccupied random place and move to it if it increases their utility in the future. The interpretation for that behaviour is: agents heard about a place which will be cool in the future.

\paragraph{Optimizing present \& future} Agents pick an unoccupied random place and move to it if it increases their utility in the now and in the future. The interpretation for that behaviour is: agents heard about a cool spot in town, check it out and move to it if they like it but they also anticipate the coolness of the place in the future and if it seems that the place is going down then they won't move there.

\subsection{Related Research}
TODO: \cite{kirman_complex_2010} mention kirman complex economics where he investigates the model more in depth


\section{The SIR Model}
To explain the concepts of ABS and of our functional reactive approach to it, we introduce the SIR model as a motivating example. It is a very well studied and understood compartment model from epidemiology which allows to simulate the dynamics of an infectious disease spreading through a population. In this model, people in a population of size $N$ can be in either one of three states \textit{Susceptible}, \textit{Infected} or \textit{Recovered} at a particular time, where it is assumed that initially there is at least one infected person in the population. People interact with each other \textit{on average} with a given rate $\beta$ per time-unit and get infected with a given probability $\gamma$ when interacting with an infected person. When infected, a person recovers \textit{on average} after $\delta$ time-units and is then immune to further infections. An interaction between infected persons does not lead to re-infection, thus these interactions are ignored in this model. This definition gives rise to three compartments with the transitions as seen in Figure \ref{fig:sir_transitions}.

\begin{figure}
	\centering
	\includegraphics[width=.4\textwidth, angle=0]{./../shared/fig/diagrams/SIR_transitions.png}
	\caption{Transitions in the SIR compartment model.}
	\label{fig:sir_transitions}
\end{figure}

The dynamics of this model over time can be formalized using the System Dynamics (SD) approach which models a system through differential equations. For the SIR model we get the following equations:

\begin{align*}
\frac{\mathrm d S}{\mathrm d t} &= -infectionRate \\ 
\frac{\mathrm d I}{\mathrm d t} &= infectionRate - recoveryRate \\ 
\frac{\mathrm d R}{\mathrm d t} &= recoveryRate 
\end{align*}

\begin{align*}
infectionRate &= \frac{I \beta S \gamma}{N} \\
recoveryRate &= \frac{I}{\delta} 
\end{align*}

Solving these equations is then done by integrating over time. In the SD terminology, the integrals are called \textit{Stocks} and the values over which is integrated over time are called \textit{Flows} \footnote{The $1+$ in $I(t)$ amounts to the initially infected agent - if there wouldn't be a single infected one, the system would immediately reach equilibrium.}.

\begin{align*}
S(t) &= N + \int_0^t -infectionRate\, \mathrm{d}t \\
I(t) &= 1 + \int_0^t infectionRate - recoveryRate\, \mathrm{d}t \\
R(t) &= \int_0^t recoveryRate\, \mathrm{d}t
\end{align*}

There exist a huge number of software packages which allow to conveniently express SD models using a visual approach like in Figure \ref{fig:sir_sd_stockflow_diagramm}.

\begin{figure}
	\centering
	\includegraphics[width=.4\textwidth, angle=0]{./../shared/fig/diagrams/SIR_SD_STOCKFLOW_DIAGRAMM.png}
	\caption{A visual representation of the SD stocks and flows of the SIR compartment model. Picture taken using AnyLogic Personal Learning Edition 8.1.0.}
	\label{fig:sir_sd_stockflow_diagramm}
\end{figure}

Running the SD simulation over time results in the dynamics as shown in Figure \ref{fig:sir_sd_dynamics} with the given variables.

\begin{figure}
	\centering
	\includegraphics[width=.4\textwidth, angle=0]{./../shared/fig/frsd/SIR_SD_1000agents_150t_001dt.png}
	\caption{Dynamics of the SIR compartment model using the System Dynamics approach. Population Size $N$ = 1000, contact rate $\beta =  \frac{1}{5}$, infection probability $\gamma = 0.05$, illness duration $\delta = 15$ with initially 1 infected agent. Simulation run for 150 time-steps.}
	\label{fig:sir_sd_dynamics}
\end{figure}

\subsection*{An Agent-Based approach}
The SD approach is inherently top-down because the emergent property of the system is formalized in differential equations. The question is if such a top-down behaviour can be emulated using ABS, which is inherently bottom-up. Also the question is if there are fundamental drawbacks and benefits when doing so using ABS. Such questions were asked before and modelling the SIR model using an agent-based approach is indeed possible. It is important to note that SD treats the population completely continuous which results in non-discrete values of stocks e.g. 3.1415 infected persons. Thus the fundamental approach to map the SIR model to an ABS is to discretize the population and model each person in the population as an individual agent. The transition  between the states are no longer happening according to continuous differential equations but due to discrete events caused both by interactions amongst the agents and time-outs.

\begin{itemize}
	\item Every agent makes \textit{on average} contact with $\beta$ random other agents per time unit. In ABS we can only contact discrete agents thus we model this by generating a random event on average every $\beta$ time units. Note that we need to sample from an exponential CDF because the rate is proportional to the size of the population as \cite{borshchev_system_2004} pointed out.
	
	\item An agent does not know the other agents' state when making contact with it, thus we need a mechanism in which agents reveal their state in which they are in \textit{at the moment of making contact}. Obviously the already mentioned messaging which allows agents to interact is perfectly suited to do this.
	\begin{itemize}
		\item \textit{Susceptibles}: These agents make contact with other random agents (excluding themselves) with a "Susceptible" message. They can be seen to be the drivers of the dynamics.
		\item \textit{Infected}: These agents only reply to incoming "Susceptible" messages with an "Infected" message to the sender. Note that they themselves do \textit{not} make contact pro-actively but only react to incoming one. 
		\item \textit{Recovered}: These agents do not need to send messages because contacting it or being contacted by it has no influence on the state.
	\end{itemize}
	
	\item Transition of susceptible to infected state - a susceptible agent needs to have made contact with an infected agent which happens when it receives an "Infected" message. If this happens an infection occurs with a probability of $\gamma$. The infection can be calculated by drawing $p$ from a uniform random-distribution between 0 and 1 - infection occurs in case of $\gamma >= p $. Note that this needs to be done for \textit{every} received "Infected" message.
	
	\item Transition of infected to recovered - a person recovers \textit{on average} after $\delta$ time unites. This is implemented by drawing the duration from an exponential distribution \cite{borshchev_system_2004} with $\lambda = \frac{1}{\delta}$ and making the transition after this duration.
\end{itemize}

In Figure \ref{fig:sir_abs_anylogic_agents} we give the dynamics simulating the SIR model with the agent-based approach.

%TODO: replace these pictures with ones generated by FrABS
%
%TODO: reproducing about the same dynamics of the SD-solution (1.0 dt)
%	- super-sampling: 	contact-rate ss high, illness time-out low 
%	- agent number:		1000 vs. 10.000 agents
%	- Susceptibles making contact and infected response VS. only Infected make contact
%	- update-strat:		Sequential vs. Parallel
%	- making contact: susceptible only vs. susceptible AND infected
%	- do conversations make a difference?
%	- does a delayed switch (dSwitch) in transitions makes a difference?
	
\begin{figure*}
\begin{center}

	\begin{tabular}{c c c}
		\begin{subfigure}[b]{0.3\textwidth}
			\centering
			\includegraphics[width=1\textwidth, angle=0]{./../shared/fig/frabs/SIR_100agents_150t_01dt_parallel.png}
			\caption{100 Agents}
			\label{fig:pd_seq}
		\end{subfigure}
    	&
		\begin{subfigure}[b]{0.3\textwidth}
			\centering
			\includegraphics[width=1\textwidth, angle=0]{./../shared/fig/frabs/SIR_1000agents_150t_01dt_parallel.png}
			\caption{1,000 Agents}
			\label{fig:pd_seq}
		\end{subfigure}
    	&
		\begin{subfigure}[b]{0.3\textwidth}
			\centering
			\includegraphics[width=1\textwidth, angle=0]{./../shared/fig/frabs/SIR_10000agents_150t_01dt_parallel.png}
			\caption{10,000 Agents}
			\label{fig:hac_seq}
		\end{subfigure}
	\end{tabular}
	
	\caption{Approximating the continuous dynamics of the system dynamics simulation using the agent-based approach. Model-parameters are the same ($\beta = \frac{1}{5}$, $\gamma = 0.05$, $\delta = 15$ with initially 1 infected agent) except population size. All simulations run for 150 time-steps.} 
	\label{fig:sir_abs_anylogic_agents}
\end{center}
\end{figure*}

As previously mentioned the agent-based approach is a discrete one which means that with increasing number of agents, the discrete dynamics approximate the continuous dynamics of the SD simulation. Still the dynamics of 10,000 Agents do not match the dynamics of the SD simulation perfectly. This is because as opposed to the SD simulation the agent-based approach is inherently a stochastic one as we continuously draw from random-distributions which drive our state-transitions. What we see in Figure \ref{fig:sir_abs_anylogic_agents} is then just a single run where the dynamics would result in slightly different shapes when run with a different random-number generator seed. The agent-based approach thus generates a distribution of dynamics over which ones needs to average to arrive at the correct solution. This can be done using replications in which the simulation is run with the exact same parameters multiple times but each with a different random-number generator see. The resulting dynamics are then averaged and the result is then regarded as the correct dynamics.
We have done this as can be seen in Figure \ref{fig:sir_abs_agents_repls}, using 10 replications, which matches the SD dynamics to a very satisfactory level \footnote{Note that in the replications we are using 10 initially infected agents to ensure that no simulation run will terminate too early (meaning that the disease gets extinct after a few time steps) which would offset the dynamics completely. This happens due to "unlucky" random distributions which can be repaired by introducing more initially infected agents which increases the probability of spreading the disease in the very early stage of the simulation drastically. We found that when using 10 initially infected agents in a population of 10,000 (which amounts to 0.1\%) is enough to never result in an early terminating simulation. This is also a fundamental difference between SD and ABS: the dynamics of the agent-based approach can result in a wide range of scenarios which includes also the one in which the disease gets extinct in the early stages (a lucky coincidence for mankind) - this is simply not possible in the SD approach. So we can argue that ABS is much closer to reality than SD as it allows to explore alternate futures in the dynamics.}.

\begin{figure*}
\begin{center}

	\begin{tabular}{c c c}
		\begin{subfigure}[b]{0.3\textwidth}
			\centering
			\includegraphics[width=1\textwidth, angle=0]{./../shared/fig/frabs/SIR_100agents_150t_01dt_parallel_10replications.png}
			\caption{100 Agents}
			\label{fig:sir_abs_agents_repls_100}
		\end{subfigure}
    	&
		\begin{subfigure}[b]{0.3\textwidth}
			\centering
			\includegraphics[width=1\textwidth, angle=0]{./../shared/fig/frabs/SIR_1000agents_150t_01dt_parallel_10replications.png}
			\caption{1,000 Agents}
			\label{fig:sir_abs_agents_repls_1000}
		\end{subfigure}
    	&
		\begin{subfigure}[b]{0.3\textwidth}
			\centering
			\includegraphics[width=1\textwidth, angle=0]{./../shared/fig/frabs/SIR_1000agents_150t_01dt_parallel_10replications.png}
			\caption{TODO: create run with 10 replications of 10,000 Agents}
			\label{fig:sir_abs_agents_repls_10000}
		\end{subfigure}
	\end{tabular}
	
	\caption{Dynamics of Figure \ref{fig:sir_abs_anylogic_agents} averaged over 10 replications with initially 10 infected agents.} 
	\label{fig:sir_abs_agents_repls}
\end{center}
\end{figure*}

For a more in-depth introduction of how to approximate an SD model by ABS see \cite{macal_agent-based_2010} who discusses a general approach and how to compare dynamics and \cite{borshchev_system_2004} which explain the need to draw the illness-duration from an exponential-distribution.

We will derive the implementation of this approach in the next section, with the complete code provided in Appendix \ref{app:abs_code}. As will be seen our approach allows us express this behaviour very explicitly and is looking very much like a formal ABS specification of the problem. 

\section{Deriving a pure functional approach}
\label{sec:functional_approach}

In \cite{thaler_art_2017} two fundamental problems of implementing an ABS from a programming-language agnostic point of view is discussed. The first problem is how agents can be pro-active and the second how interactions and communication between agents can happen. For agents to be pro-active, they must be able to perceive the passing of time, which means there must be a concept of an agent-process which executes over time. Interactions between agents can be reduced to the problem of how an agent can expose information about its internal state which can be perceived by other agents. 

Both problems are strongly related to the semantics of a model and the authors show that it is of fundamental importance to match the update-strategy with the semantics of the model - the order in which agents are updated and actions of agents are visible can make a big difference and need to match the model semantics. The authors identify four different update-strategies, of which the \textit{parallel} update-strategy matches the semantics of the agent-based SIR model due to the underlying roots in the System Dynamics approach. In the parallel update-strategy, the agents act \textit{conceptually} all at the same time in lock-step. This implies that they observe the same environment state during a time-step and actions of an agent are only visible in the next time-step - they are isolated from each other, see Figure \ref{fig:parallel_strategy}.

\begin{figure}
	\centering
	\includegraphics[width=.4\textwidth, angle=0]{./fig/diagrams/parallel_strategy.png}
	\caption{Parallel, lock-step execution of the agents.}
	\label{fig:parallel_strategy}
\end{figure}


Also, the authors \cite{thaler_art_2017} have shown the influence of different deterministic and non-deterministic elements in ABS on the dynamics and how the influence of non-determinism can completely break them down or result in different dynamics despite same initial conditions. This means that we want to rule out any potential source of non-determinism, which we achieve by keeping our implementation pure. This rules out the use of the IO Monad and thus any potential source of non-determinism under all circumstances because we would lose all compile time guarantees about reproducibility. Still we will make use of the Random Monad, which indeed allows side-effects but the crucial point here is that we restrict side-effects only to this type in a controlled way without allowing general unrestricted effects like in traditional object-oriented approaches in the field.

In the following, we derive a pure functional approach for an ABS of the SIR model in which we pose solutions to the previously mentioned problems. We start out with a first approach in Yampa and show its limitations. Then we generalise it to a more powerful approach, which utilises Monadic Stream Functions, a generalisation of FRP. Finally we add a structured environment, making the example more interesting and shows the real strength of ABS over other simulation methodologies like System Dynamics and Discrete Event Simulation \footnote{The code of all steps can be accessed freely through the following URL: \url{https://github.com/thalerjonathan/phd/tree/master/public/purefunctionalepidemics/code}}.

\subsection{Functional Reactive Programming}
\label{sec:implement_frp}
As described in the Section \ref{sec:back_frp}, Arrowized FRP \cite{hughes_generalising_2000} is a way to implement systems  with continuous and discrete time-semantics where the central concept is the signal function, which can be understood as a process over time, mapping an input- to an output-signal. Technically speaking, a signal function is a continuation which allows to capture state using closures and hides away the $\Delta t$, which means that it is never exposed explicitly to the programmer, meaning it cannot be messed with.

As already pointed out, agents need to perceive time, which means that the concept of processes over time is an ideal match for our agents and our system as a whole, thus we will implement them and the whole system as signal functions.

\subsubsection{Implementation}
We start by defining the SIR states as ADT and our agents as signal functions (SF) which receive the SIR states of all agents form the previous step as input and outputs the current SIR state of the agent. This definition, and the fact that Yampa is not monadic, guarantees already at compile, that the agents are isolated from each other, enforcing the \textit{parallel} lock-step semantics of the model.

\begin{HaskellCode}
data SIRState = Susceptible | Infected | Recovered

type SIRAgent = SF [SIRState] SIRState 

sirAgent :: RandomGen g => g -> SIRState -> SIRAgent
sirAgent g Susceptible = susceptibleAgent g
sirAgent g Infected    = infectedAgent g
sirAgent _ Recovered   = recoveredAgent
\end{HaskellCode}

Depending on the initial state we return the corresponding behaviour. Note that we are passing a random-number generator instead of running in the Random Monad because signal functions as implemented in Yampa are not capable of being monadic. 

We see that the recovered agent ignores the random-number generator because a recovered agent does nothing, stays immune forever and can not get infected again in this model. Thus a recovered agent is a consuming state from which there is no escape, it simply acts as a sink which returns constantly \textit{Recovered}:

\begin{HaskellCode}
recoveredAgent :: SIRAgent
recoveredAgent = arr (const Recovered)
\end{HaskellCode}

Next, we implement the behaviour of a susceptible agent. It makes contact \textit{on average} with $\beta$ other random agents. For every \textit{infected} agent it gets into contact with, it becomes infected with a probability of $\gamma$. If an infection happens, it makes the transition to the \textit{Infected} state. To make contact, it gets fed the states of all agents in the system from the previous time-step, so it can draw random contacts - this is one, very naive way of implementing the interactions between agents.

Thus a susceptible agent behaves as susceptible until it becomes infected. Upon infection an \textit{Event} is returned, which results in switching into the \textit{infectedAgent} SF, which causes the agent to behave as an infected agent from that moment on. When an infection event occurs we change the behaviour of an agent using the Yampa combinator \textit{switch}, which is quite elegant and expressive as it makes the change of behaviour at the occurrence of an event explicit. Note that to make contact \textit{on average}, we use Yampas \textit{occasionally} function which requires us to carefully select the right $\Delta t$ for sampling the system as will be shown in results. 

Note the use of \textit{iPre :: a $\rightarrow$ SF a a}, which delays the input signal by one sample, taking an initial value for the output at time zero. The reason for it is that we need to delay the transition from susceptible to infected by one step due to the semantics of the \textit{switch} combinator: whenever the switching event occurs, the signal function into which is switched will be run at the time of the event occurrence. This means that a susceptible agent could make a transition to recovered within one time-step, which we want to prevent, because the semantics should be that only one state-transition can happen per time-step.

\begin{HaskellCode}
susceptibleAgent :: RandomGen g => g -> SIRAgent
susceptibleAgent g 
    = switch 
      -- delay switching by 1 step to prevent against transition
      -- from Susceptible to Recovered within one time-step
      (susceptible g >>> iPre (Susceptible, NoEvent)) 
      (const (infectedAgent g))
  where
    susceptible :: RandomGen g 
      => g -> SF [SIRState] (SIRState, Event ())
    susceptible g = proc as -> do
      makeContact <- occasionally g (1 / contactRate) () -< ()
      if isEvent makeContact
        then (do
          -- draw random element from the list
          a <- drawRandomElemSF g -< as
          case a of
            Infected -> do
              -- returns True with given probability
              i <- randomBoolSF g infectivity -< ()
              if i
                then returnA -< (Infected, Event ())
                else returnA -< (Susceptible, NoEvent)
             _       -> returnA -< (Susceptible, NoEvent))
        else returnA -< (Susceptible, NoEvent)
\end{HaskellCode}

To deal with randomness in an FRP way, we implemented additional signal functions built on the \textit{noiseR} function provided by Yampa. This is an example for the stream character and statefulness of a signal function as it allows to keep track of the changed random-number generator internally through the use of continuations and closures. Here we provide the implementation of \textit{randomBoolSF}. \textit{drawRandomElemSF} works similar but takes a list as input and returns a randomly chosen element from it:

\begin{HaskellCode}
randomBoolSF :: RandomGen g => g -> Double -> SF () Bool
randomBoolSF g p = proc _ -> do
  r <- noiseR ((0, 1) :: (Double, Double)) g -< ()
  returnA -< (r <= p)
\end{HaskellCode}

An infected agent recovers \textit{on average} after $\delta$ time units. This is implemented by drawing the duration from an exponential distribution \cite{borshchev_system_2004} with $\lambda = \frac{1}{\delta}$ and making the transition to the \textit{Recovered} state after this duration. Thus the infected agent behaves as infected until it recovers, on average after the illness duration, after which it behaves as a recovered agent by switching into \textit{recoveredAgent}. As in the case of the susceptible agent, we use the \textit{occasionally} function to generate the event when the agent recovers. Note that the infected agent ignores the states of the other agents as its behaviour is completely independent of them.

\begin{HaskellCode}
infectedAgent :: RandomGen g => g -> SIRAgent
infectedAgent g 
    = switch 
      -- delay switching by 1 step 
      (infected >>> iPre (Infected, NoEvent))
      (const recoveredAgent)
  where
    infected :: SF [SIRState] (SIRState, Event ())
    infected = proc _ -> do
      recEvt <- occasionally g illnessDuration () -< ()
      let a = event Infected (const Recovered) recEvt
      returnA -< (a, recEvt)
\end{HaskellCode}

For running the simulation we use Yampas function \textit{embed}:

\begin{HaskellCode}
runSimulation :: RandomGen g => g -> Time -> DTime 
              -> [SIRState] -> [[SIRState]]
runSimulation g t dt as 
    = embed (stepSimulation sfs as) ((), dts)
  where
    steps     = floor (t / dt)
    dts       = replicate steps (dt, Nothing)
    n         = length as
    (rngs, _) = rngSplits g n [] -- unique rngs for each agent
    sfs       = zipWith sirAgent rngs as
\end{HaskellCode}

What we need to implement next is a closed feedback-loop - the heart of every agent-based simulation. Fortunately, \cite{nilsson_functional_2002, courtney_yampa_2003} discusses implementing this in Yampa. The function \textit{stepSimulation} is an implementation of such a closed feedback-loop. It takes the current signal functions and states of all agents, runs them all in parallel and returns this step's new agent states. Note the use of \textit{notYet}, which is required because in Yampa switching occurs immediately at $t = 0$. If we don't delay the switching at $t = 0$ until the next step, we would enter an infinite switching loop - \textit{notYet} simply delays the first switching until the next time-step.

\begin{HaskellCode}
stepSimulation :: [SIRAgent] -> [SIRState] -> SF () [SIRState]
stepSimulation sfs as =
    dpSwitch
      -- feeding the agent states to each SF
      (\_ sfs' -> (map (\sf -> (as, sf)) sfs'))
      -- the signal functions
      sfs
      -- switching event, ignored at t = 0         
      (switchingEvt >>> notYet)
      -- recursively switch back into stepSimulation         
      stepSimulation                            
  where
    switchingEvt :: SF ((), [SIRState]) (Event [SIRState])
    switchingEvt = arr (\ (_, newAs) -> Event newAs)
\end{HaskellCode}

Yampa provides the \textit{dpSwitch} combinator for running signal functions in parallel, which has the following type-signature:

\begin{HaskellCode}
dpSwitch :: Functor col
         -- routing function
         => (forall sf. a -> col sf -> col (b, sf))
         -- SF collection
         -> col (SF b c)
         -- SF generating switching event     
         -> SF (a, col c) (Event d)
         -- continuation to invoke upon event           
         -> (col (SF b c) -> d -> SF a (col c))
         -> SF a (col c)
\end{HaskellCode}

Its first argument is the pairing-function, which pairs up the input to the signal functions - it has to preserve the structure of the signal function collection. The second argument is the collection of signal functions to run. The third argument is a signal function generating the switching event. The last argument is a function, which generates the continuation after the switching event has occurred. \textit{dpSwitch} returns a new signal function, which runs all the signal functions in parallel and switches into the continuation when the switching event occurs. The d in \textit{dpSwitch} stands for decoupled which guarantees that it delays the switching until the next time-step: the function into which we switch is only applied in the next step, which prevents an infinite loop if we switch into a recursive continuation.

Conceptually, \textit{dpSwitch} allows us to recursively switch back into the \textit{stepSimulation} with the continuations and new states of all the agents after they were run in parallel. 

\subsubsection{Results}
The dynamics generated by this step can be seen in Figure \ref{fig:sir_abs_dynamics_frp}. 

\begin{figure}
\begin{center}
	\begin{tabular}{c c}
		\begin{subfigure}[b]{0.22\textwidth}
			\centering
			\includegraphics[width=1\textwidth, angle=0]{./fig/SIR_Yampa/SIR_Yampa_dt01.png}
			\caption{$\Delta t = 0.1$}
			\label{fig:sir_abs_approximating_01dt_1000agents}
		\end{subfigure}
		
		&
    	
		\begin{subfigure}[b]{0.22\textwidth}
			\centering
			\includegraphics[width=1\textwidth, angle=0]{./fig/SIR_Yampa/SIR_Yampa_dt001.png}
			\caption{$\Delta t = 0.01$}
			\label{fig:sir_abs_approximating_001dt_1000agents}
		\end{subfigure}
	\end{tabular}
	
	\caption{FRP simulation of agent-based SIR showing the influence of different $\Delta t$. Population size of 1,000 with contact rate $\beta = \frac{1}{5}$, infection probability $\gamma = 0.05$, illness duration $\delta = 15$ with initially 1 infected agent. Simulation run for 150 time-steps with respective $\Delta t$.} 
	\label{fig:sir_abs_dynamics_frp}
\end{center}
\end{figure}

By following the FRP approach we assume a continuous flow of time, which means that we need to select a \textit{correct} $\Delta t$, otherwise we would end up with wrong dynamics. The selection of a correct $\Delta t$ depends in our case on \textit{occasionally} in the \textit{susceptible} behaviour, which randomly generates an event on average with \textit{contact rate} following the exponential distribution. To arrive at the correct dynamics, this requires us to sample \textit{occasionally}, and thus the whole system, with small enough $\Delta t$ which matches the frequency of events generated by \textit{contact rate}. If we choose a too large $\Delta t$, we loose events, which will result in wrong dynamics as can be seen in Figure \ref{fig:sir_abs_approximating_01dt_1000agents}. This issue is known as under-sampling and is described in Figure \ref{fig:sampling_issue}.

\begin{figure}
\begin{center}
	\begin{tabular}{c}
		\begin{subfigure}[b]{0.3\textwidth}
			\centering
			\includegraphics[width=1\textwidth, angle=0]{./fig/diagrams/Undersampling.png}
			\caption{Under-sampling}
			\label{fig:undersampling}
		\end{subfigure}
		
		\\
		
		\begin{subfigure}[b]{0.3\textwidth}
			\centering
			\includegraphics[width=1\textwidth, angle=0]{./fig/diagrams/Supersampling.png}
			\caption{Super-sampling}
			\label{fig:supersampling}
		\end{subfigure}
	\end{tabular}
	
	\caption{A visual explanation of under-sampling and super-sampling. The black dots represent the time-steps of the simulation. The red dots represent virtual events which occur at specific points in continuous time. In the case of under-sampling, 3 events occur in between the two time steps but \textit{occasionally} only captures the first one. By increasing the sampling frequency either through a smaller $\Delta t$ or super-sampling all 3 events can be captured.} 
	\label{fig:sampling_issue}
\end{center}
\end{figure}

For tackling this issue we have two options. The first one is to use a smaller $\Delta t$ as can be seen in \ref{fig:sir_abs_approximating_001dt_1000agents}, which results in the whole system being sampled more often, thus reducing performance. The other option is to implement super-sampling and apply it to \textit{occasionally}, which would allow us to run the whole simulation with $\Delta t = 1.0$ and only sample the \textit{occasionally} function with a much higher frequency.

\subsubsection{Discussion}
We can conclude that our first step already introduced most of the fundamental concepts of ABS:
\begin{itemize}
	\item Time - the simulation occurs over virtual time which is modelled explicitly, divided into \textit{fixed} $\Delta t$, where at each step all agents are executed.
	\item Agents - we implement each agent as an individual, with the behaviour depending on its state.
	\item Feedback - the output state of the agent in the current time-step $t$ is the input state for the next time-step $t + \Delta t$.
	\item Environment - as environment we implicitly assume a fully-connected network (complete graph) where every agent 'knows' every other agent, including itself and thus can make contact with all of them.
	\item Stochasticity - it is an inherently stochastic simulation, which is indicated by the random-number generator and the usage of \textit{occasionally}, \textit{randomBoolSF} and \textit{drawRandomElemSF}.
	\item Deterministic - repeated runs with the same initial random-number generator result in same dynamics. This may not come as a surprise but in Haskell we can guarantee that property statically already at compile time because our simulation runs \textit{not} in the IO Monad. This guarantees that no external, uncontrollable sources of non-determinism can interfere with the simulation.
	\item Parallel, lock-step semantics - the simulation implements a \textit{parallel} update-strategy where in each step the agents are run isolated in parallel and don't see the actions of the others until the next step.
\end{itemize}

Using FRP in the instance of Yampa results in a clear, expressive and robust implementation. State is implicitly encoded, depending on which signal function is active. By using explicit time-semantics with \textit{occasionally} we can achieve extremely fine grained stochastics by sampling the system with small $\Delta t$: we are treating it as a truly continuous time-driven agent-based system.

A very severe problem, hard to find with testing but detectable with in-depth validation analysis, is the fact that in the \textit{susceptible} agent the same random-number generator is used in \textit{occasionally}, \textit{drawRandomElemSF} and \textit{randomBoolSF}. This means that all three stochastic functions, which should be independent from each other, are inherently correlated. This is something one wants to prevent under all circumstances in a simulation, as it can invalidate the dynamics on a very subtle level, and indeed we have tested the influence of the correlation in this example and it has an impact. We left this severe bug in for explanatory reasons, as it shows an example where functional programming actually encourages very subtle bugs if one is not careful. A possible but not very elegant solution would be to simply split the initial random-number generator in \textit{sirAgent} three times (using one of the splited generators for the next split) and pass three random-number generators to \textit{susceptible}. A much more elegant solution would be to use the Random Monad which is not possible because Yampa is not monadic.

So far we have an acceptable implementation of an agent-based SIR approach. What we are lacking at the moment is a general treatment of an environment and an elegant solution to the random number correlation. In the next step we make the transition to Monadic Stream Functions as introduced in Dunai \cite{perez_functional_2016}, which allows FRP within a monadic context and gives us a way for an elegant solution to the random number correlation.

\subsection{Generalising to Monadic Stream Functions}
\label{sec:generalising_msfs}
A part of the library Dunai is BearRiver, a wrapper which re-implements Yampa on top of Dunai, which should allow us to easily replace Yampa with MSFs. This will enable us to run arbitrary monadic computations in a signal function, solving our problem of correlated random numbers through the use of the Random Monad.

\subsubsection{Identity Monad}
We start by making the transition to BearRiver by simply replacing Yampas signal function by BearRivers' which is the same but takes an additional type parameter \textit{m}, indicating the monadic context. If we replace this type-parameter with the Identity Monad, we should be able to keep the code exactly the same, because BearRiver re-implements all necessary functions we are using from Yampa. We simply re-define our agent signal function, introducing the monad stack our SIR implementation runs in:

\begin{HaskellCode}
type SIRMonad    = Identity
type SIRAgent    = SF SIRMonad [SIRState] SIRState
\end{HaskellCode}

\subsubsection{Random Monad}
Using the Identity Monad does not gain us anything but it is a first step towards a more general solution. Our next step is to replace the Identity Monad by the Random Monad, which will allow us to run the whole simulation within the Random Monad with the full features of FRP, finally solving the problem of correlated random numbers in an elegant way. We start by re-defining the SIRMonad and SIRAgent:

\begin{HaskellCode}
type SIRMonad g = Rand g
type SIRAgent g = SF (SIRMonad g) [SIRState] SIRState
\end{HaskellCode}

The question is now how to access this Random Monad functionality within the MSF context. For the function \textit{occasionally}, there exists a monadic pendant \textit{occasionallyM} which requires a MonadRandom type-class. Because we are now running within a MonadRandom instance we simply replace \textit{occasionally} with \textit{occasionallyM}. 

\begin{HaskellCode}
occasionallyM :: MonadRandom m => Time -> b -> SF m a (Event b)
-- can be used through the use of arrM and lift
randomBoolM :: RandomGen g => Double -> Rand g Bool
-- this can be used directly as a SF with the arrow notation
drawRandomElemSF :: MonadRandom m => SF m [a] a
\end{HaskellCode}

\subsubsection{Discussion} 
Running in the Random Monad solved the problem of correlated random numbers and elegantly guarantees us that we won't have correlated stochastics as discussed in the previous section. In the next step we introduce the concept of an explicit discrete 2D environment.

\subsection{Adding an environment}
\label{sec:step5_environment}
In this step we will add an environment in which the agents exist and through which they interact with each other. This is a fundamental different approach to agent interaction but is as valid as the approach in the previous steps.

In ABS agents are often situated within a discrete 2D environment \cite{epstein_growing_1996} which is simply a finite $N x M$ grid with either a Moore or von Neumann neighbourhood (Figure \ref{fig:abs_neighbourhoods}). Agents are either static or can move freely around with cells allowing either single or multiple occupants.

We can directly map the SIR model to a discrete 2D environment by placing the agents on a corresponding 2D grid with an unrestricted neighbourhood. The behaviour of the agents is the same but they select their interactions directly from the environment. Also instead of feeding back the states of all agents as inputs, agents now communicate through the environment by revealing their current state to their neighbours by placing it on their cell. Agents can read the states of all their neighbours which tells them if a neighbour is infected or not. This allows us to implement the infection mechanism as in the beginning. For purposes of a more interesting approach, we restrict the neighbourhood to Moore (Figure \ref{fig:moore_neighbourhood}).

\begin{figure}
\begin{center}
	\begin{tabular}{c c}
		\begin{subfigure}[b]{0.2\textwidth}
			\centering
			\includegraphics[width=0.5\textwidth, angle=0]{./fig/diagrams/neumann.png}
			\caption{von Neumann}
			\label{fig:neumann_neighbourhood}
		\end{subfigure}
    	&
		\begin{subfigure}[b]{0.2\textwidth}
			\centering
			\includegraphics[width=0.5\textwidth, angle=0]{./fig/diagrams/moore.png}
			\caption{Moore}
			\label{fig:moore_neighbourhood}
		\end{subfigure}
    \end{tabular}
	\caption{Common neighbourhoods in discrete 2D environments of Agent-Based Simulation.}
	\label{fig:abs_neighbourhoods}
\end{center}
\end{figure}

\subsubsection{Implementation}
We start by defining our discrete 2D environment for which we use an indexed two dimensional array. In each cell the agents will store their current state, thus we use the \textit{SIRState} as type for our array data:

\begin{HaskellCode}
type Disc2dCoord = (Int, Int)
type SIREnv      = Array Disc2dCoord SIRState
\end{HaskellCode}

Next we redefine our monad stack and agent signal function. We use a StateT transformer on top of our Random Monad from the previous step with \textit{SIREnv} as type for the state. Our agent signal function now has unit input and output type, which indicates that the actions of the agents are only visible through side-effects in the monad stack they are running in.

\begin{HaskellCode}
type SIRMonad g = StateT SIREnv (Rand g)
type SIRAgent g = SF (SIRMonad g) () ()
\end{HaskellCode}

The implementation of a susceptible agent is now a bit different and a mix between previous steps. The agent directly queries the environment for its neighbours and randomly selects one of them. The remaining behaviour is similar:

\begin{HaskellCode}
susceptibleAgent :: RandomGen g => Disc2dCoord -> SIRAgent g
susceptibleAgent coord
    = switch susceptible (const (infectedAgent coord))
  where
    susceptible :: RandomGen g 
      => SF (SIRMonad g) () ((), Event ())
    susceptible = proc _ -> do
      makeContact <- occasionallyM (1 / contactRate) () -< ()
      if not (isEvent makeContact)
        then returnA -< ((), NoEvent)
        else (do
          env <- arrM_ (lift get) -< ()
          let ns = neighbours env coord agentGridSize moore
          s <- drawRandomElemS -< ns
          case s of
            Infected -> do
              infected <- arrM_ 
                (lift $ lift $ randomBoolM infectivity) -< ()
              if infected 
                then (do
                  arrM (put . changeCell coord Infected) -< env
                  returnA -< ((), Event ()))
                else returnA -< ((), NoEvent)
            _        -> returnA -< ((), NoEvent))

\end{HaskellCode}
Querying the neighbourhood is done using the \textit{neighbours :: SIREnv -> Disc2dCoord -> Disc2dCoord -> [Disc2dCoord] -> [SIRState]} function. It takes the environment, the coordinate for which to query the neighbours for, the dimensions of the 2D grid and the neighbourhood information and returns the data of all neighbours it could find. Note that on the edge of the environment, it could be the case that fewer neighbours than provided in the neighbourhood information will be found due to clipping.

The behaviour of an infected agent is nearly the same as in the previous step, with the difference that upon recovery the infected agent updates its state in the environment from Infected to Recovered.

Running the simulation with MSFs works slightly different. The function \textit{embed} we used before is not provided by BearRiver but by Dunai which has important implications. Dunai does not know about time in MSFs, which is exactly what BearRiver builds on top of MSFs. It does so by adding a ReaderT Double which carries the $\Delta t$. This is the reason why we need lifts e.g. in case of getting the environment. Thus \textit{embed} returns a computation in the ReaderT Double Monad which we need to peel away using \textit{runReaderT}. This then results in a StateT computation which we evaluate by using \textit{evalStateT} and an initial environment as initial state. This then results in another monadic computation of the Random Monad type which we evaluate using \textit{evalRand} which delivers the final result. Note that instead of returning agent states we simply return a list of environments, one for each step. The agent states can then be extracted from each environment.

\begin{HaskellCode}
runSimulation :: RandomGen g => g -> Time -> DTime 
  -> SIREnv -> [(Disc2dCoord, SIRState)] -> [SIREnv]
runSimulation g t dt env as = evalRand esRand g
  where
    steps    = floor (t / dt)
    dts      = replicate steps ()
    -- initial SFs of all agents
    sfs      = map (uncurry sirAgent) as   
    -- running the simulation   
    esReader = embed (stepSimulation sfs) dts 
    esState  = runReaderT esReader dt 
    esRand   = evalStateT esState env     
\end{HaskellCode}

Due to the different approach of returning the SIREnv in every step, we implemented our own MSF:
\begin{HaskellCode}
stepSimulation :: RandomGen g 
  => [SIRAgent g] -> SF (SIRMonad g) () SIREnv
stepSimulation sfs = MSF (\_ -> do
  -- running all SFs with unit input
  res <- mapM (`unMSF` ()) sfs
  -- extracting continuations, ignore output
  let sfs' = fmap snd res
  -- getting environment of current step   
  env <- get
  -- recursive continuation    
  let ct = stepSimulation sfs'  
  return (env, ct))
\end{HaskellCode}

\subsubsection{Results}
We implemented rendering of the environments using the gloss library which allows us to cycle arbitrarily through the steps and inspect the spreading of the disease over time visually as seen in Figure \ref{fig:sir_env}.

\begin{figure}
\begin{center}
	\begin{tabular}{c c}
		\begin{subfigure}[b]{0.2\textwidth}
			\centering
			\includegraphics[width=1\textwidth, angle=0]{./fig/step5_environment/SIR_environment_30x30agents_t100_01dt.png}
			\caption{$t = 100$}
			\label{fig:sir_env_t100}
		\end{subfigure}
    	
    	&
  
		\begin{subfigure}[b]{0.25\textwidth}
			\centering
			\includegraphics[width=1\textwidth, angle=0]{./fig/step5_environment/SIR_dynamics_30x30agents_300t_01dt.png}
			\caption{Dynamics over time}
			\label{fig:sir_dynamics_30x30agents_300t_01dt}
		\end{subfigure}
	\end{tabular}
	
	\caption{Simulating the agent-based SIR model on a 21x21 2D grid with Moore neighbourhood (Figure \ref{fig:moore_neighbourhood}), a single infected agent at the center and same SIR parameters as in Figure \ref{fig:sir_sd_dynamics}. Simulation run until $t = 200$ with fixed $\Delta t = 0.1$. Last infected agent recovers shortly after $t = 160$. The susceptible agents are rendered as blue hollow circles for better contrast.}
	\label{fig:sir_env}
\end{center}
\end{figure}

Note that the dynamics of the spatial SIR simulation which are seen in Figure \ref{fig:sir_dynamics_30x30agents_300t_01dt} look quite different from the SD dynamics of Figure \ref{fig:sir_sd_dynamics}. This is due to a much more restricted neighbourhood which results in far fewer infected agents at a time and a lower number of recovered agents at the end of the epidemic, meaning that fewer agents got infected overall.

\subsubsection{Discussion}
At first the environment approach might seem a bit overcomplicated and one might ask what we have gained by using an unrestricted neighbourhood where all agents can contact all others. The real win is that we can introduce arbitrary restrictions on the neighbourhood as shown with the Moore neighbourhood.

Of course an environment is not restricted to be a discrete 2D grid and can be anything from a continuous N-dimensional space to a complex network - one only needs to change the type of the StateT monad and provide corresponding neighbourhood querying functions. The ability to place the heterogeneous agents in a generic environment is also the fundamental advantage of an agent-based over the SD approach and allows to simulate much more realistic scenarios. 

\subsubsection{Discussion}
At first the environment approach might seem a bit overcomplicated and one might ask what we have gained by using an unrestricted neighbourhood where all agents can contact all others. The real win is that we can introduce arbitrary restrictions on the neighbourhood as shown using the Moore neighbourhood. Of course the environment is not restricted to a discrete 2D grid and can be anything from a continuous N-dimensional space to a complex network - one only needs to change the type of the StateT monad and provide corresponding neighbourhood querying functions. The ability to place the heterogeneous agents in a generic environment is also the fundamental advantage of an agent-based over the SD approach and allows to simulate much more realistic scenarios. Note that for reasons of clarity we have removed the data-flow approach from this implementation which results in the unit-types of input and output. In a full blown agent-based simulation library we would combine both approaches.

Generally, there exist four different types of environments in agent-based simulation:
\begin{enumerate}
	\item Passive read-only - implemented in the previous steps, where the environment never changes and is passed as static information, e.g. list of neighbours, to each agent.
	\item Passive read/write - implemented in this step. The environment itself is not modelled as an active process but just as shared data which can be accessed and manipulated by the agents.
	\item Active  read-only - can be implemented by adding an environment agent which broadcasts changes in the environment to all agents using the data-flow mechanism.
	\item Active read/write - can be implemented as in this step plus adding an environment agent which reads/writes the environment e.g. regrowing some resources.
\end{enumerate}

Attempting to introduce an active/passive read/write environment to the Yampa implementation would be quite cumbersome. A possible solution could be to add a type-parameter \textit{e} which captures the type of the environment and then pass it in through the input and allow it to be returned in the output of an agent signal function. We would then end up with $n$ copies of the environment - one for each agent - which we need to fold back into a single environment. Having an active environment complicates things even further. All these problems are not an issue when using MSFs with a StateT which is a compelling example for making the transition to the more general MSFs. The convenient thing is that although conceptually all agents act at the same time, technically by using \textit{mapM} in stepSimulation they are run after another, serialising the environment access which gives every agent exclusive read/write access while it is active.

\subsection{Additional Steps}
ABS involves a few more advanced concepts which we don't fully explore in this paper due to lack of space. Instead we give a short overview and discuss them without presenting code or going into technical details.

\subsubsection{Synchronous Agent Interactions}
Synchronous agent interactions are necessary when an arbitrary number of interactions between two agents need to happen instantaneously within the same time-step. The use-case for this are price negotiations between multiple agents where each pair of agents needs to come to an agreement in the same time-step \cite{epstein_growing_1996}. In object-oriented programming, the concept of synchronous communication between agents is implemented directly with method calls.

We have implemented synchronous interactions in an additional step. We solved it pure functionally by running the signal functions of the transacting agent pair as often as their protocol requires but with $\Delta t=0$, which indicates the instantaneous character of these interactions.

\subsubsection{Event Scheduling}
Our approach is inherently time-driven where the system is sampled with fixed $\Delta t$. The other fundamental way to implement an ABS in general, is to follow an event-driven approach \cite{meyer_event-driven_2014}, which is based on the theory of Discrete Event Simulation \cite{zeigler_theory_2000}. In such an approach the system is not sampled in fixed $\Delta t$ but advanced as events occur, where the system stays constant in between. Depending on the model, in an event-driven approach it may be more natural to express the requirements of the model.

In an additional step we have implemented a rudimentary event-driven approach, which allows the scheduling of events. Using the flexibility of MSFs we added a State transformer to the monad stack, which allows queuing of events into a priority queue. The simulation is advanced by processing the next event at the top of the queue, which means running the MSF of the agent which receives the event. The simulation terminates if there are either no more events in the queue or after a given number of events, or if the simulation time has advanced to some limit. Having made the transition to MSFs, implementing this feature was quite straight forward, which shows the power and strength of the generalised approach to FRP using MSFs.

% NOTE: ran out of space
%\subsubsection{Dynamic Agent creation}
%In the SIR model, the agent population stays constant - agents don't die and no agents are created during simulation - but some simulations \cite{epstein_growing_1996} require dynamic agent creation and destruction. We can easily add and remove agents signal functions in the recursive switch after each time-step. The only problem is that creating new agents requires unique agent ids but with the transition to MSFs we can add a monadic context, which allows agents to draw the next unique agent id when they create a new agent. 

\subsubsection{Conflicts in Environment}
The semantics of the agent-based SIR model allowed a straight-forward implementation of the parallel update-strategy. This is not easily possible when there could be conflicts in the environment e.g. moving agents where only a single one can occupy a cell. Most models in ABS \cite{epstein_growing_1996} solve this by implementing a sequential update-strategy \cite{thaler_art_2017}, where agents are run after another but can already observe the changes by agents run before them in the same time-step. To prevent the introduction of artefacts due to a specific ordering, these models shuffle the agents before running them in each step to average the probability for a specific agent to be run at a fixed position.

A \textit{sequential} update-strategy does not map naturally do functional programming because in it the focus is primarily on a data-parallel approach, which is enforced by the fundamental different nature of side-effects and the immutability of data. Thus, in such a model, different conflict resolving mechanisms need to be implemented. One approach is to randomly select a winner and re-run other conflicting agents signal functions as long as the underlying monadic context is robust to re-runs - in case of the Random Monad this would be no problem.

\section{Related Research}

\cite{schneider_towards_2012} and \cite{vendrov_frabjous:_2014} present a domain-specific language for developing functional reactive agent-based simulations. This language called FRABJOUS is very human readable and easily understandable by domain-experts. It is not directly implemented in FRP/Haskell/Yampa but is compiled to Haskell/Yampa code which they claim is also readable. This is the direction we want to head but we don't want this intermediate step but look for how a most simple domain-specific language embedded in Haskell would look like. In this paper we explicitly dive deep into FRP And Yampa and see how we can combine the best of both.

\section{Conclusions}
\label{sec:conclusions}

Our approach is radically different from traditional approaches in the ABS community. First it builds on the already quite powerful FRP paradigm. Second, due to our continuous time approach, it forces one to think properly of time-semantics of the model and how small $\Delta t$ should be. Third it requires to think about agent interactions in a new way instead of being just method-calls.

Because no part of the simulation runs in the IO Monad and we do not use unsafePerformIO we can rule out a serious class of bugs caused by implicit data-dependencies and side-effects which can occur in traditional imperative implementations.

Also we can statically guarantee the reproducibility of the simulation, which means that repeated runs with the same initial conditions are guaranteed to result in the same dynamics. Although we allow side-effects within agents, we restrict them to only the Random and State Monad in a controlled, deterministic way and never use the IO Monad which guarantees the absence of non-deterministic side effects within the agents and other parts of the simulation.

Determinism is also ensured by fixing the $\Delta t$ and not making it dependent on the performance of e.g. a rendering-loop or other system-dependent sources of non-determinism as described by \cite{perez_testing_2017}. Also by using FRP we gain all the benefits from it and can use research on testing, debugging and exploring FRP systems \cite{perez_testing_2017, perez_back_2017}.

\subsection*{Issues}
Currently, the performance of the system is not comparable to imperative implementations but our research was not focusing on this aspect. We leave the investigation and optimization of the performance aspect of our approach for further research.

Despite the strengths and benefits we get by leveraging on FRP, there are errors that are not raised at compile time, e.g. we can still have infinite loops and run-time errors. This was for example investigated in \cite{sculthorpe_safe_2009} where the authors use dependent types to avoid some run-time errors in FRP. We suggest that one could go further and develop a domain specific type system for FRP that makes the FRP based ABS more predictable and that would support further mathematical analysis of its properties. Furthermore, moving to dependent types would pose a unique benefit over the traditional object-oriented approach and should allow us to express and guarantee even more properties at compile time. We leave this for further research.

In our pure functional approach, agent identity is not as clear as in traditional object-oriented programming, where an agent can be hidden behind a polymorphic interface which is much more abstract than in our approach. Also the identity of an agent is much clearer in object-oriented programming due to the concept of object-identity and the encapsulation of data and methods.

We can conclude that the main difficulty of a pure functional approach evolves around the communication and interaction between agents, which is a direct consequence of the issue with agent identity. Agent interaction is straight-forward in object-oriented programming, where it is achieved using method-calls mutating the internal state of the agent, but that comes at the cost of a new class of bugs due to implicit data flow. In pure functional programming these data flows are explicit but our current approach of feeding back the states of all agents as inputs is not very general and we have added further mechanisms of agent interaction which we had to omit due to lack of space.

\section{Further Research}
\label{sec:further_research}

We see this paper as an intermediary and necessary step towards dependent types for which we first needed to understand the potentials and limitations of a non-dependently typed pure functional approach in Haskell. Dependent types are extremely promising in functional programming as they allow us to express stronger guarantees about the correctness of programs and go as far as allowing to formulate programs and types as constructive proofs \cite{wadler_propositions_2015} which must be total by definition \cite{thompson_type_1991}, \cite{altenkirch_why_2005}, \cite{altenkirch_pi_2010}, \cite{program_homotopy_2013}. So far no research using dependent types in agent-based simulation exists at all and it is not clear whether dependent types make sense in this context. In our next paper we want to explore this for the first time and ask more specifically how we can add dependent types to our pure functional approach, which conceptual implications this has for ABS and what we gain from doing so. We plan on using Idris \cite{brady_idris_2013}, \cite{brady_type-driven_2017} as the language of choice as it is very close to Haskell with focus on real-world application and running programs as opposed to other languages with dependent types e.g. Agda and Coq which serve primarily as proof assistants.
It would be of immense interest whether we could apply dependent types to the model meta-level or not - this boils down to the question if we can encode our model specification in a dependent type way. This would allow the ABS community for the first time to reason about a proper formalisation of a model. We plan to implement a total and terminating implementation of our approach which would be a formal proof-by-construction that the agent-based approach of the SIR model terminates after a finite number of steps.

\begin{acks}
The authors would like to thank I. Perez, H. Nilsson, J. Greensmith, M. Baerenz, H. Vollbrecht, S. Venkatesan and J. Hey for constructive feedback, comments and valuable discussions. TODO: add thanks to the anonymous referees from IFL 2018?
\end{acks}

% Bibliography
\bibliographystyle{ACM-Reference-Format}
%% Citation style
%% Note: author/year citations are required for papers published as an
%% issue of PACMPL.
%%\citestyle{acmauthoryear}   %% For author/year citations
\bibliography{../../../references/phdReferences.bib}

\end{document}
