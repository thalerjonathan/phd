\subsection{Functional Reactive Programming}
\label{sec:implement_frp}
As described in the Section \ref{sec:back_frp}, Arrowized FRP \cite{hughes_generalising_2000} is a way to implement systems  with continuous and discrete time-semantics where the central concept is the signal function, which can be understood as a process over time, mapping an input- to an output-signal. Technically speaking, a signal function is a continuation which allows to capture state using closures and hides away the $\Delta t$, which means that it is never exposed explicitly to the programmer, meaning it cannot be messed with.

As already pointed out, agents need to perceive time, which means that the concept of processes over time is an ideal match for our agents and our system as a whole, thus we will implement them and the whole system as signal functions.

\subsubsection{Implementation}
We start by defining the SIR states as ADT and our agents as signal functions (SF) which receive the SIR states of all agents form the previous step as input and outputs the current SIR state of the agent. This definition, and the fact that Yampa is not monadic, guarantees already at compile, that the agents are isolated from each other, enforcing the \textit{parallel} lock-step semantics of the model.

\begin{HaskellCode}
data SIRState = Susceptible | Infected | Recovered

type SIRAgent = SF [SIRState] SIRState 

sirAgent :: RandomGen g => g -> SIRState -> SIRAgent
sirAgent g Susceptible = susceptibleAgent g
sirAgent g Infected    = infectedAgent g
sirAgent _ Recovered   = recoveredAgent
\end{HaskellCode}

Depending on the initial state we return the corresponding behaviour. Note that we are passing a random-number generator instead of running in the Random Monad because signal functions as implemented in Yampa are not capable of being monadic. 

We see that the recovered agent ignores the random-number generator because a recovered agent does nothing, stays immune forever and can not get infected again in this model. Thus a recovered agent is a consuming state from which there is no escape, it simply acts as a sink which returns constantly \textit{Recovered}:

\begin{HaskellCode}
recoveredAgent :: SIRAgent
recoveredAgent = arr (const Recovered)
\end{HaskellCode}

Next, we implement the behaviour of a susceptible agent. It makes contact \textit{on average} with $\beta$ other random agents. For every \textit{infected} agent it gets into contact with, it becomes infected with a probability of $\gamma$. If an infection happens, it makes the transition to the \textit{Infected} state. To make contact, it gets fed the states of all agents in the system from the previous time-step, so it can draw random contacts - this is one, very naive way of implementing the interactions between agents.

Thus a susceptible agent behaves as susceptible until it becomes infected. Upon infection an \textit{Event} is returned, which results in switching into the \textit{infectedAgent} SF, which causes the agent to behave as an infected agent from that moment on. When an infection event occurs we change the behaviour of an agent using the Yampa combinator \textit{switch}, which is quite elegant and expressive as it makes the change of behaviour at the occurrence of an event explicit. Note that to make contact \textit{on average}, we use Yampas \textit{occasionally} function which requires us to carefully select the right $\Delta t$ for sampling the system as will be shown in results. 

Note the use of \textit{iPre :: a $\rightarrow$ SF a a}, which delays the input signal by one sample, taking an initial value for the output at time zero. The reason for it is that we need to delay the transition from susceptible to infected by one step due to the semantics of the \textit{switch} combinator: whenever the switching event occurs, the signal function into which is switched will be run at the time of the event occurrence. This means that a susceptible agent could make a transition to recovered within one time-step, which we want to prevent, because the semantics should be that only one state-transition can happen per time-step.

\begin{HaskellCode}
susceptibleAgent :: RandomGen g => g -> SIRAgent
susceptibleAgent g 
    = switch 
      -- delay switching by 1 step to prevent against transition
      -- from Susceptible to Recovered within one time-step
      (susceptible g >>> iPre (Susceptible, NoEvent)) 
      (const (infectedAgent g))
  where
    susceptible :: RandomGen g 
      => g -> SF [SIRState] (SIRState, Event ())
    susceptible g = proc as -> do
      makeContact <- occasionally g (1 / contactRate) () -< ()
      if isEvent makeContact
        then (do
          -- draw random element from the list
          a <- drawRandomElemSF g -< as
          case a of
            Infected -> do
              -- returns True with given probability
              i <- randomBoolSF g infectivity -< ()
              if i
                then returnA -< (Infected, Event ())
                else returnA -< (Susceptible, NoEvent)
             _       -> returnA -< (Susceptible, NoEvent))
        else returnA -< (Susceptible, NoEvent)
\end{HaskellCode}

To deal with randomness in an FRP way, we implemented additional signal functions built on the \textit{noiseR} function provided by Yampa. This is an example for the stream character and statefulness of a signal function as it allows to keep track of the changed random-number generator internally through the use of continuations and closures. Here we provide the implementation of \textit{randomBoolSF}. \textit{drawRandomElemSF} works similar but takes a list as input and returns a randomly chosen element from it:

\begin{HaskellCode}
randomBoolSF :: RandomGen g => g -> Double -> SF () Bool
randomBoolSF g p = proc _ -> do
  r <- noiseR ((0, 1) :: (Double, Double)) g -< ()
  returnA -< (r <= p)
\end{HaskellCode}

An infected agent recovers \textit{on average} after $\delta$ time units. This is implemented by drawing the duration from an exponential distribution \cite{borshchev_system_2004} with $\lambda = \frac{1}{\delta}$ and making the transition to the \textit{Recovered} state after this duration. Thus the infected agent behaves as infected until it recovers, on average after the illness duration, after which it behaves as a recovered agent by switching into \textit{recoveredAgent}. As in the case of the susceptible agent, we use the \textit{occasionally} function to generate the event when the agent recovers. Note that the infected agent ignores the states of the other agents as its behaviour is completely independent of them.

\begin{HaskellCode}
infectedAgent :: RandomGen g => g -> SIRAgent
infectedAgent g 
    = switch 
      -- delay switching by 1 step 
      (infected >>> iPre (Infected, NoEvent))
      (const recoveredAgent)
  where
    infected :: SF [SIRState] (SIRState, Event ())
    infected = proc _ -> do
      recEvt <- occasionally g illnessDuration () -< ()
      let a = event Infected (const Recovered) recEvt
      returnA -< (a, recEvt)
\end{HaskellCode}

For running the simulation we use Yampas function \textit{embed}:

\begin{HaskellCode}
runSimulation :: RandomGen g => g -> Time -> DTime 
              -> [SIRState] -> [[SIRState]]
runSimulation g t dt as 
    = embed (stepSimulation sfs as) ((), dts)
  where
    steps     = floor (t / dt)
    dts       = replicate steps (dt, Nothing)
    n         = length as
    (rngs, _) = rngSplits g n [] -- unique rngs for each agent
    sfs       = zipWith sirAgent rngs as
\end{HaskellCode}

What we need to implement next is a closed feedback-loop - the heart of every agent-based simulation. Fortunately, \cite{nilsson_functional_2002, courtney_yampa_2003} discusses implementing this in Yampa. The function \textit{stepSimulation} is an implementation of such a closed feedback-loop. It takes the current signal functions and states of all agents, runs them all in parallel and returns this step's new agent states. Note the use of \textit{notYet}, which is required because in Yampa switching occurs immediately at $t = 0$. If we don't delay the switching at $t = 0$ until the next step, we would enter an infinite switching loop - \textit{notYet} simply delays the first switching until the next time-step.

\begin{HaskellCode}
stepSimulation :: [SIRAgent] -> [SIRState] -> SF () [SIRState]
stepSimulation sfs as =
    dpSwitch
      -- feeding the agent states to each SF
      (\_ sfs' -> (map (\sf -> (as, sf)) sfs'))
      -- the signal functions
      sfs
      -- switching event, ignored at t = 0         
      (switchingEvt >>> notYet)
      -- recursively switch back into stepSimulation         
      stepSimulation                            
  where
    switchingEvt :: SF ((), [SIRState]) (Event [SIRState])
    switchingEvt = arr (\ (_, newAs) -> Event newAs)
\end{HaskellCode}

Yampa provides the \textit{dpSwitch} combinator for running signal functions in parallel, which has the following type-signature:

\begin{HaskellCode}
dpSwitch :: Functor col
         -- routing function
         => (forall sf. a -> col sf -> col (b, sf))
         -- SF collection
         -> col (SF b c)
         -- SF generating switching event     
         -> SF (a, col c) (Event d)
         -- continuation to invoke upon event           
         -> (col (SF b c) -> d -> SF a (col c))
         -> SF a (col c)
\end{HaskellCode}

Its first argument is the pairing-function, which pairs up the input to the signal functions - it has to preserve the structure of the signal function collection. The second argument is the collection of signal functions to run. The third argument is a signal function generating the switching event. The last argument is a function, which generates the continuation after the switching event has occurred. \textit{dpSwitch} returns a new signal function, which runs all the signal functions in parallel and switches into the continuation when the switching event occurs. The d in \textit{dpSwitch} stands for decoupled which guarantees that it delays the switching until the next time-step: the function into which we switch is only applied in the next step, which prevents an infinite loop if we switch into a recursive continuation.

Conceptually, \textit{dpSwitch} allows us to recursively switch back into the \textit{stepSimulation} with the continuations and new states of all the agents after they were run in parallel. 

\subsubsection{Results}
The dynamics generated by this step can be seen in Figure \ref{fig:sir_abs_dynamics_frp}. 

\begin{figure}
\begin{center}
	\begin{tabular}{c c}
		\begin{subfigure}[b]{0.22\textwidth}
			\centering
			\includegraphics[width=1\textwidth, angle=0]{./fig/SIR_Yampa/SIR_Yampa_dt01.png}
			\caption{$\Delta t = 0.1$}
			\label{fig:sir_abs_approximating_01dt_1000agents}
		\end{subfigure}
		
		&
    	
		\begin{subfigure}[b]{0.22\textwidth}
			\centering
			\includegraphics[width=1\textwidth, angle=0]{./fig/SIR_Yampa/SIR_Yampa_dt001.png}
			\caption{$\Delta t = 0.01$}
			\label{fig:sir_abs_approximating_001dt_1000agents}
		\end{subfigure}
	\end{tabular}
	
	\caption{FRP simulation of agent-based SIR showing the influence of different $\Delta t$. Population size of 1,000 with contact rate $\beta = \frac{1}{5}$, infection probability $\gamma = 0.05$, illness duration $\delta = 15$ with initially 1 infected agent. Simulation run for 150 time-steps with respective $\Delta t$.} 
	\label{fig:sir_abs_dynamics_frp}
\end{center}
\end{figure}

By following the FRP approach we assume a continuous flow of time, which means that we need to select a \textit{correct} $\Delta t$, otherwise we would end up with wrong dynamics. The selection of a correct $\Delta t$ depends in our case on \textit{occasionally} in the \textit{susceptible} behaviour, which randomly generates an event on average with \textit{contact rate} following the exponential distribution. To arrive at the correct dynamics, this requires us to sample \textit{occasionally}, and thus the whole system, with small enough $\Delta t$ which matches the frequency of events generated by \textit{contact rate}. If we choose a too large $\Delta t$, we loose events, which will result in wrong dynamics as can be seen in Figure \ref{fig:sir_abs_approximating_01dt_1000agents}. This issue is known as under-sampling and is described in Figure \ref{fig:sampling_issue}.

\begin{figure}
\begin{center}
	\begin{tabular}{c}
		\begin{subfigure}[b]{0.3\textwidth}
			\centering
			\includegraphics[width=1\textwidth, angle=0]{./fig/diagrams/Undersampling.png}
			\caption{Under-sampling}
			\label{fig:undersampling}
		\end{subfigure}
		
		\\
		
		\begin{subfigure}[b]{0.3\textwidth}
			\centering
			\includegraphics[width=1\textwidth, angle=0]{./fig/diagrams/Supersampling.png}
			\caption{Super-sampling}
			\label{fig:supersampling}
		\end{subfigure}
	\end{tabular}
	
	\caption{A visual explanation of under-sampling and super-sampling. The black dots represent the time-steps of the simulation. The red dots represent virtual events which occur at specific points in continuous time. In the case of under-sampling, 3 events occur in between the two time steps but \textit{occasionally} only captures the first one. By increasing the sampling frequency either through a smaller $\Delta t$ or super-sampling all 3 events can be captured.} 
	\label{fig:sampling_issue}
\end{center}
\end{figure}

For tackling this issue we have two options. The first one is to use a smaller $\Delta t$ as can be seen in \ref{fig:sir_abs_approximating_001dt_1000agents}, which results in the whole system being sampled more often, thus reducing performance. The other option is to implement super-sampling and apply it to \textit{occasionally}, which would allow us to run the whole simulation with $\Delta t = 1.0$ and only sample the \textit{occasionally} function with a much higher frequency.

\subsubsection{Discussion}
We can conclude that our first step already introduced most of the fundamental concepts of ABS:
\begin{itemize}
	\item Time - the simulation occurs over virtual time which is modelled explicitly, divided into \textit{fixed} $\Delta t$, where at each step all agents are executed.
	\item Agents - we implement each agent as an individual, with the behaviour depending on its state.
	\item Feedback - the output state of the agent in the current time-step $t$ is the input state for the next time-step $t + \Delta t$.
	\item Environment - as environment we implicitly assume a fully-connected network (complete graph) where every agent 'knows' every other agent, including itself and thus can make contact with all of them.
	\item Stochasticity - it is an inherently stochastic simulation, which is indicated by the random-number generator and the usage of \textit{occasionally}, \textit{randomBoolSF} and \textit{drawRandomElemSF}.
	\item Deterministic - repeated runs with the same initial random-number generator result in same dynamics. This may not come as a surprise but in Haskell we can guarantee that property statically already at compile time because our simulation runs \textit{not} in the IO Monad. This guarantees that no external, uncontrollable sources of non-determinism can interfere with the simulation.
	\item Parallel, lock-step semantics - the simulation implements a \textit{parallel} update-strategy where in each step the agents are run isolated in parallel and don't see the actions of the others until the next step.
\end{itemize}

Using FRP in the instance of Yampa results in a clear, expressive and robust implementation. State is implicitly encoded, depending on which signal function is active. By using explicit time-semantics with \textit{occasionally} we can achieve extremely fine grained stochastics by sampling the system with small $\Delta t$: we are treating it as a truly continuous time-driven agent-based system.

A very severe problem, hard to find with testing but detectable with in-depth validation analysis, is the fact that in the \textit{susceptible} agent the same random-number generator is used in \textit{occasionally}, \textit{drawRandomElemSF} and \textit{randomBoolSF}. This means that all three stochastic functions, which should be independent from each other, are inherently correlated. This is something one wants to prevent under all circumstances in a simulation, as it can invalidate the dynamics on a very subtle level, and indeed we have tested the influence of the correlation in this example and it has an impact. We left this severe bug in for explanatory reasons, as it shows an example where functional programming actually encourages very subtle bugs if one is not careful. A possible but not very elegant solution would be to simply split the initial random-number generator in \textit{sirAgent} three times (using one of the splited generators for the next split) and pass three random-number generators to \textit{susceptible}. A much more elegant solution would be to use the Random Monad which is not possible because Yampa is not monadic.

So far we have an acceptable implementation of an agent-based SIR approach. What we are lacking at the moment is a general treatment of an environment and an elegant solution to the random number correlation. In the next step we make the transition to Monadic Stream Functions as introduced in Dunai \cite{perez_functional_2016}, which allows FRP within a monadic context and gives us a way for an elegant solution to the random number correlation.