\section{Case Study I: SIR}
\label{sec:case_SIR}
As first use-case we discuss property-based testing for the \textit{explanatory} agent-based SIR model. It is a very well studied and understood compartment model from epidemiology \cite{kermack_contribution_1927} which allows to simulate the dynamics of an infectious disease like influenza, tuberculosis, chicken pox, rubella and measles spreading through a population. We implemented an agent-based version of this model \footnote{The code is freely accessible from \url{https://github.com/thalerjonathan/phd/tree/master/public/propabs/sir}}, inspired by \cite{macal_agent-based_2010}.

In this model, people in a population of size $N$ can be in either one of three states \textit{Susceptible}, \textit{Infected} or \textit{Recovered} at a particular time, where it is assumed that initially there is at least one infected person in the population. People interact \textit{on average} with a given rate of $\beta$ other people per time-unit and become infected with a given probability $\gamma$ when interacting with an infected person. When infected, a person recovers \textit{on average} after $\delta$ time-units and is then immune to further infections. An interaction between infected persons does not lead to re-infection, thus these interactions are ignored in this model. Due to the models' origin in System Dynamics (SD) \cite{porter_industrial_1962}, there exists a top-down formalisation in SD with the following equations:

\begin{equation}
\begin{split}
\frac{\mathrm d S}{\mathrm d t} = -infectionRate \\
\frac{\mathrm d I}{\mathrm d t} = infectionRate - recoveryRate \\
\frac{\mathrm d R}{\mathrm d t} = recoveryRate 
\end{split}
\quad
\begin{split}
infectionRate = \frac{I \beta S \gamma}{N} \\
recoveryRate = \frac{I}{\delta} 
\end{split}
\end{equation}

\subsection{Deriving a property}
Our goal is to derive a property which connects the agent-based implementation to the SD equations. The foundation are both the infection- and recovery-rate where the infection-rate determines how many \textit{Susceptible} agents per time-unit become \textit{Infected} and the recovery-rate determines how many \textit{Infected} agents per time-unit become \textit{Recovered}. Let's look at the algorithm of the susceptible agent behaviour, which is key for the infection-rate:

\begin{algorithm}
generate on average $\beta$ make-contact events per time-unit\; 
\If{make-contact event}{
  select random agent \textit{randA} from population\; 
  \If{agent randA infected}{
    become infected with probability $\gamma$\; 
  }  
}
\caption{Susceptible behaviour}
\end{algorithm}

Per time-unit, a susceptible agent makes \textit{on average} contact with $\beta$ other agents, where in the case of a contact with an infected agent, the susceptible agent becomes infected with a given probability $\gamma$. In this description there is another probability hidden, the probability of making contact with an infected agent, which is simply the ratio of number of infected agents to number non-infected agents. We can now derive the formula for the probability of a \textit{Susceptible} agent to become infected: $\frac{\beta * \gamma * \text{number of infected (I)}}{\text{number of non-infected (N)}}$. When we look at the formula we can see that it is conceptually the same representation of the \textit{infection-rate} of the SD specification as shown above - except that it only considers a single \textit{Susceptible} agent instead of the aggregate of \textit{S} susceptible agents. We have now a property we can check using a property-test.

\subsection{Constructing the property-based test}
Having a property (law), we want now to construct a property-test for it. The formula is invariant under random population mixes and thus should hold for varying agent populations where the mix of \textit{Susceptible, Infected and Recovered} agents is random - thus we use QuickCheck to generate the population randomly, the property must still hold.

Obviously we need to pay attention to the fact that we are dealing with a stochastic system thus we can only talk about averages and thus it does not suffice to only run a single agent but we are repeating this for 1.000 \textit{Susceptible} agents (all with different random-number seeds). We thus compute the simulated infection-rate simply by counting the agents which got infected and divide it by the number of total replications N = 1.000. To check whether this test has passed we run this 100 times and use a 2-sided t-test to check if the sample infection-rate is statistically significant equal to the hypothetical infection-rate. When executing the tests, QuickCheck generates 100 test-cases by randomly generating 100 different \textit{randAs} inputs to the test. All have to pass for the whole property-test to pass. See Algorithm \ref{alg:prop_test_infectionrate} for the pseudo-code of this property-based test.

\begin{algorithm}
\SetKwInOut{Input}{input}\SetKwInOut{Output}{output}
\Input{List \textit{randAs} of random agent-population generated by QuickCheck}
populationCount = length \textit{randAs}\;
infectedCount   = count \textit{Infected} in \textit{randAs}\;
hypInfRate      = infectivity * contactRate * (infectedCount / populationCount)\;
sampleRateList  = empty list\;

\For{$i\leftarrow 1$ \KwTo 100}{
susceptibles = create 1000 \textit{Susceptible} agents\;
countInfected = 0\;
\For{each agent sa in susceptibles}{
  run agent sa for 1.0 time-unit, with list \textit{randAs} as input\;
  \If{agent sa became \textit{Infected} }{
	countInfected = countInfected + 1\;
  }
}
avgInfRate = countInfected / (length susceptibles)\;
insert avgInfRate into sampleRateList\;
}

tTestPass = perform 2-sided t-test comparing hypInfRate with sampleRateList on a 0.95 interval\;
\eIf{tTestPass}{
  PASS\;
} {
  FAIL\;
}
\caption{Property-based test for infection-rate.}
\end{algorithm}
\label{alg:prop_test_infectionrate}

This is the very power which property-based testing is offering us: we directly express the specification of the original SD model in a test of our agent-based implementation and let QuickCheck generate random test cases for us. This closely ties our implementation to the original specification and raises the confidence to a very high level that it is actually a valid and correct implementation. Also using this test we can determine the \textit{optimal} $\Delta t$ for running our simulation: because the SIR model is a time-driven one, we need to select a sufficiently small $\Delta t$ to avoid sampling issues \cite{thaler_pure_2019}. Using this property-test one can start out with an initial $\Delta t$ and divides it by 2 it until the tests pass.
Further, by using property-tests we found out about a special case we haven't covered in the implementation of the Susceptible agent behaviour. This show that property-based testing is not only useful for encoding specifications for regression tests but that is indeed also valuable tool in finding real bugs e.g. due to missed edge-cases. 