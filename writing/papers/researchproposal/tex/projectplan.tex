\section{Project-Plan}

\subsection{Years}
The whole PhD lasts for 3 years and thus I will structure it according to 3 years where each year will be a major milestone - which is also intended by the Computer School.

\subsubsection{1st Year: Basics}
In this year I will learn basics and develop and research the methodology I will use for the main work in the 2nd year. Also I want to write 2 papers, see below.

\begin{itemize}
\item Prototyping in Haskell, Scala and Java
\item Study Actor-Model theory
\item Learn how to do reasoning about programs
\item Basics of Type- and Category-Theory
\item Basics of Economics \cite{bowles_understanding_2005}, \cite{LehalleLaruelle2013}, \cite{baker_market_2013}
\item Basics of ACE: \cite{KirmanComplex2010}, \cite{Darley2007}
\item Write the 2 papers
\item Write 1st year report
\end{itemize}

\subsubsection{2nd Year: Main Work}
Applying 1st year results, methods and experiences to develop and write main paper to be published in a journal in 3rd year thus in 2nd year the main  work and implementation will be done. The idea is to start from Ionescus Framework \cite{Botta20114025} and build on his paper.

\begin{itemize}
\item Implement Ionescous framework using the methodology developed in 1st year
\item Generalize implementation to market models
\item Learn Agda and dependent types
\item Category Theory
\item Type Theory
\end{itemize}

\subsubsection{3rd Year: Finalizing, Publishing \& Writing}
I plan to be finished - or nearly finished - at the end of the 3rd year. In this year I will finalize the work of the 2nd year, publish the my main journal paper (and optional fun-papers if possible) and will write down the thesis. \\ To have a bit of distraction and to prevent myself to become too locked in in writing on the thesis I will also work on my optional fun-papers (see below) and hope to at least finish them and maybe publish them - at least I want to present them to 2-3 audiences (e.g. FP Lunch) to test the reaction (especially the Genesis-Paper).

\begin{itemize}
\item Finalize research of 2nd year
\item Publish journal paper
\item Write thesis
\item Work on fun-papers
\end{itemize}

\subsection{Papers}
This is the list of papers I want to work on. The goals of publishing I set for myself are given beside the paper.

\begin{enumerate}
\item \textit{Pure by Nature: Agent-Based Simulation \& Modelling in Haskell} - Conference paper
\item \textit{Actors: The Future in Agent-Based Simulation \& Modelling?} - Conference paper
\item \textit{Pure Functional ACE (Catchy title yet to be defined)} - Journal paper
\item \textit{The Genesis According to Computer-Science: Reality as Simulation of Free Will} - Optional publishing
\item \textit{Pure Functional Islamic Design} - Optional publishing
\end{enumerate}

\subsubsection{Pure by Nature: Agent-Based Simulation \& Modelling in Haskell}
The first paper which describes how one can implement ABM/S in Haskell and compares the implementation and results to Java and Akka. A major focus are update-strategies, parallelism, reproducibility, reasoning and comparability between the various implementations. Multiple approaches are shown in Haskell: no framework, Yampa-based, gloss-based, ?-based. This paper will establish my methodology in using Haskell / pure functional programming in the 2nd year main work.

\subsubsection{Actors: The Future in Agent-Based Simulation \& Modelling?}
Although the actor-model is quite old (beginning of the 70s) it seems to have a revival both in Erlang in the 90s and now in the Framework Akka (based on Scala). It is one way of organizing highly parallel (and optionally distributed) applications. Also the actor-model is very close to the agent-metaphor where the latter one was strongly inspired by the former one. Thus It would be very interesting to look closer into how the Actor-Model can be utilized to ABM/S as it seems that this has not been properly done yet.

\subsubsection{Pure Functional ACE (Catchy title yet to be defined)}
Is the main work of the PhD and targeted at publication in a Journal. The exact topic and content will be clarified at the beginning of the 2nd year. Mainly it will describe how to implement Ionescus Framework of Gintis trading model and extend it to a more general Market-Model. It will also give an outlook on implementing it using dependent types.

\subsubsection{The Genesis According to Computer-Science: Reality as Simulation of Free Will}
I've always been interested in a deeper meaning behind things so I want to look into the philosophy and future of simulation: why do we simulate, what can we derive from simulations, what does it say that we humans simulate, what will the future of simulation be? \\
I claim that our ability to "simulate" in our mind separates our intelligence from those of the animals and that this is a unique property of humans. Also i think the future of simulation will be that humankind will do its own creation/live (artifical life, conciousness) which allows to accurately simulate a given setting - this of course could have ethical implications. \\
This is fun paper 1.

\subsubsection{Pure Functional Islamic Design}
Inspired by the paper "Functional Geometry" by Peter Henderson I had the idea to come up with a  EDSL for declaratively describing pictures of islamic design which are then rendered using the gloss-library. \\
This is fun paper 2 - from its focus totally unrelated to the PhD topic but still a great opportunity to learn Haskell, to learn to think functional, to learn to design my own EDSL - thus it may be a great paper to pursue even if I won't finish or produce something publishable.