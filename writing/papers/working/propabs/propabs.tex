%**************************************************************************
%* SpringSim 2019 Author Kit
%*
%* Word Processing System: TeXnicCenter and MiKTeX
%*
%**************************************************************************
\documentclass{scspaperproc}

\usepackage{latexsym}
\usepackage{graphicx}
\usepackage{mathptmx}

%
%****************************************************************************
% AUTHOR: You may want to use some of these packages. (Optional)
%\usepackage{amsmath}
%\usepackage{amsfonts}
%\usepackage{amssymb}
%\usepackage{amsbsy}
%\usepackage{amsthm}
%****************************************************************************

%
%****************************************************************************
% AUTHOR: If you do not wish to use hyperlinks, then just comment
% out the hyperref usepackage commands below.

%% This version of the command is used if you use pdflatex. In this case you
%% cannot use ps or eps files for graphics, but pdf, jpeg, png etc are fine.

\usepackage[pdftex,colorlinks=true,urlcolor=blue,citecolor=black,anchorcolor=black,linkcolor=black]{hyperref}

%% The next versions of the hyperref command are used if you adopt the
%% outdated latex-dvips-ps2pdf route in generating your pdf file. In
%% this case you can use ps or eps files for graphics, but not pdf, jpeg, png etc.
%% However, the final pdf file should embed all fonts required which means that you have to use file
%% formats which can embed fonts. Please note that the final PDF file will not be generated on your computer!
%% If you are using WinEdt or PCTeX, then use the following. If you are using
%% Y&Y TeX then replace "dvips" with "dvipsone"

%% \usepackage[dvips,colorlinks=true,urlcolor=blue,citecolor=black,%
%% anchorcolor=black,linkcolor=black]{hyperref}

%% The use of the long citation format (e.g. "Brown and Edwards (1993)" rather than "[5]") and at the same
%% time using the hyperref package can lead to hard to trace bugs in case the citation is broken accross the
%% line (usually this will mark the entire paragraph as a hyperlink (clickable) which is easily noticeable and fixed
%% if using colorlinks, but not if the color is black -- as it is now). Worse yet, if a citation spans page boundary,
%% LaTeX compilation can fail, with an obscure error message. Since this depends a lot on the flow of the text
%% and wording, these bugs come and go and can be extremely hard for a beginner to trace. The error
%% message can look like this:
%%
%%    ! pdfTeX error (ext4): \pdfendlink ended up in different nesting level than \pdfstartlink.
%%    \AtBegShi@Output ...ipout \box \AtBeginShipoutBox 
%%    \fi \fi 
%%    l.174 
%%    ! ==> Fatal error occurred, no output PDF file produced!
%%
%% and can be universally fixed by putting an \mbox{} around the citation in question (in this case, at line 174)
%% and maybe adapting the wording a little bit to improve the paragraph typesetting, which is perhaps not
%% immediately obvious.
%****************************************************************************

%
%****************************************************************************
%*
%* AUTHOR: YOUR CALL!  Document-specific macros can come here.
%*
%****************************************************************************

\usepackage{float}
\usepackage{subcaption}
\usepackage{minted}
\usepackage{verbatim}
\usepackage[]{algorithm2e}

%#########################################################
%*
%*  The Document.
%*
\begin{document}

%***************************************************************************
% AUTHOR: AUTHOR NAMES GO HERE
% FORMAT AUTHORS NAMES Like: Author1, Author2 and Author3 (last names)
%
%		You need to change the author listing below!
%               Please list ALL authors using last name only, separate by a comma except
%               for the last author, separate with "and"
%
\SCSpagesetup{Thaler, Siebers and Altenkirch}

% AUTHOR: Uncomment ONE of these correct conference names.
%\def\SCSconferenceacro{SpringSim}
\def\SCSconferenceacro{SummerSim}
%\def\SCSconferenceacro{AutumnSim}
%\def\SCSconferenceacro{PowerPlantSim}

% AUTHOR: Set the correct year of the conference.
\def\SCSpublicationyear{2019}

% AUTHOR: Set the correct month and dates; the dates are separated by a single minus sign
% with no spaces and no leading zeros, the month is a full name (e.g. April) with the first letter
% capitalized. For example, "April 8-13".
\def\SCSconferencedates{July 22-July 24}

% AUTHOR: Set the correct venue in the form "City, State, Country", for example "Los Angeles, CA, USA".
\def\SCSconferencevenue{Berlin, Germany}

% AUTHOR: Uncomment ONE of the track/symposium names where you are going to submit. Please, do NOT change.
% In case your symposium is not on this list, please DO contact your symposium chair.
\def\SCSsymposiumacro{ANSS} % Annual Simulation Symposium
%\def\SCSsymposiumacro{CNS} % Communications and Networking Simulation Symposium
%\def\SCSsymposiumacro{HPC} % High Performance Computing Symposium
%\def\SCSsymposiumacro{TMS/DEVS} % Symposium on Theory of Modeling and Simulation
%\def\SCSsymposiumacro{ADS} % Agent-Directed Simulation
%\def\SCSsymposiumacro{MSCIAAS} % Modeling and Simulation of Complexity in Intelligent, Adaptive and Autonomous Systems
%\def\SCSsymposiumacro{MSM} % Modeling and Simulation in Medicine
%\def\SCSsymposiumacro{Mod4Sim} % Model-driven Approaches for Simulation Engineering Symposium
%\def\SCSsymposiumacro{Tutorial} % Tutorial Track
%\def\SCSsymposiumacro{WIP} % WIP Track
%\def\SCSsymposiumacro{Poster/Colloquium} % Poster Session and Student Colloquium
%\def\SCSsymposiumacro{MobileApp} % Student M\&S Mobile App Competition
%\def\SCSsymposiumacro{SPECTS} % Symposium on Performance Evaluation of Computer and Telecommunication Systems
%\def\SCSsymposiumacro{SCSC} % Summer Computer Simulation Conference
%\def\SCSsymposiumacro{ICBGM} % International Conference on Bond-Graph Modeling
%\def\SCSsymposiumacro{Fossil} % Fossil Power Track
%\def\SCSsymposiumacro{Nuclear} % Nuclear Agent Power Track

% AUTHOR: Enter the title, all letters in upper case

\newminted[HaskellCode]{haskell}{fontsize=\footnotesize}

% Title portion. Note the short title for running heads
\title{Show me your properties! \\ \small{The Potential Of Property-Based Testing In Agent-Based Simulation}}
%\subtitle{The Potential Of Property-Based Testing In Agent-Based Simulation}

% AUTHOR: Enter the authors of the article, see end of the example document for further examples
\author{
\\%To level with the author block on the right.
Jonathan Thaler \\ 
Peer Olaf Siebers \\ [12pt] 
School Of Computer Science \\
University of Nottingham \\
7301 Wollaton Rd \\
Nottingham, United Kingdom \\
\{jonathan.thaler,peer-olaf.siebers\}@nottingham.ac.uk\\
}

\maketitle

\section*{Abstract}
%TODO: target Summer Simulation Conference: http://scs.org/summersim/
%- there are two possible sub-tracks for it: Agent-based Modeling and Simulation (ABMS) or Verification and Validation of Computer Simulation Models (V&V).

This paper presents property-based testing, a novel approach of testing implementations of agent-based simulations (ABS). It is a complementary technique to unit-testing and allows to test specifications and laws of an implementation directly in code which is then checked using \textit{automated} test-data generation. As case-studies, we present two different models, an agent-based SIR model and the SugarScape model, in which we will show how to apply property-based testing to explanatory and exploratory agent-based models and what its limits are.
%We conducted our research in the pure functional programming language Haskell, which is the origin of property-based testing. Besides being especially suited for property-based testing, it supports strong isolation in unit-tests and controlled side-effects which further increases isolation of tests and allows extremely convenient checking of invariants.

\textbf{Keywords:} Agent-Based Simulation, Validation \& Verification, Property-Based Testing, Haskell.

\maketitle

%*******************************************************************************
%*********************************** First Chapter *****************************
%*******************************************************************************

\chapter{Introduction}  %Title of the First Chapter
I noticed that it is pretty hard to convince an agent-based economics specialist who is not a computer scientist about a pure functional approach. My conjecture is that the implementation technique and method does not matter much to them because they have very little knowledge about programming and are almost always self-taught - they don't know about software-engineering, nothing about proper software-design and architecture, nothing about software-maintenance, nothing about unit-testing,... In the end they just "hack" the simulation in whatever language they are able to: C++, Visual Basic, Java or toolboxes like Netlogo. For them it is all about to \textit{get things done somehow} and not to get things done the right way or in a beautiful way - the way and the method doesn't matter, its just a necessary evil which needs to be done. Thus if functional programming could make their lives easier, then they will definitely welcome it. But functional programming is, i think, harder to learn and harder to understand - so one needs to provide an abstraction through EDSL. So I REALLY need to come up with convincing arguments why to use pure functional approaches in ACE THEY can understand, otherwise I will be lost and not heard (not published,...). \\

What ACE economists care for:

\begin{itemize}
\item Very: Qualitative modelling with quantitative results
\item Yes: Easy reproducibility
\item Likely: Reasoning about convergence?
\item Likely: EDSL
\end{itemize}

My contributions are: pure functional framework, functional agent-model for market-simulations, EDSL for market-simulations, qualitative / implicit modelling with quanitative results, reasoning in my framework about convergence \\

IDEA: could I develop non-causal modelling (models are expressed in terms of non-directed equations, modelled in signal-relations) to allow for qualitative modelling for the agent-based economists? See hybrid modelling paper of Yampa. \textbf{THIS WOULD BE A HUGE NOVEL CONTRIBUTION TO ACE ESPECIALLY WHEN COMBINED WITH AN EDSL AND PROVIDING FULL REFERENTIAL TRANSPARENCY TO KEEP THE ABILITY TO REASON ABOUT CONVERGENCE}. This should be covered in the "EDSL"-paper.

TODO: maybe i should really focus only on market models? otherwise too much? \\

central novelty of my PhD: model specification = runnable code. possible through EDSL. but only in specific subfield of ACE: market-models. need a functional description of the model, then translate it to model specification in EDSL and then run it to see dynamics. But: model specification moves closer to functional programming languages. \\

another novelty approach: model specification through qualitative instead of quantiative approaches. is this possible? \\

WHY FUNCTIONAL? "because its the ultimate approach to scientific computing": fewer bugs due to mutable state (why? is thos shown obkectively by someone?), shorter (again as above, productivity), more expressive and closer to math, EDSL, EDSL=model=simulation, better parallelising due to referental transparency, reasoning \\

scientific results need to be reproduced, especially when they have high impact. a more formal approach of specifying the model and the simulation (model=simulation) could lead to easier sharing and easier reporduction without ambigouites \\

pure functional agent-model \& theory, EDSL framework in Haskell for ACE

\begin{enumerate}
\item Which kind of problem do we have?
\item What aim is there? Solving the problem? 
\item How the aim is achieved by enumerating VERY CLEAR objectives.
\item What the impact one expects (hypothesis) and what it is (after results).
\end{enumerate}

Note: It is not in the interest of the researcher to develop new economic theories but to research the use of functional methods (programming and specification) in agent-based computational economics (ACE).

NOTE: Get the reader’s attention early in the introduction: motivation, significance, originality and novelty.

\section{Methods}
Methods need to be selected to implement the simulations. Special emphasis will be put on functional ones which will then be compared to established methods in the field of ABM/S and ACE. \\

Claim: non-programming environments are considered to be not powerful enough to capture the complexity of ACE implementations thus a programming approach to ACE will be always required.

\section{Scenarios}
To apply and test functional methods in ACE, four scenarios of ACE are selected and then the methods applied and compared with each other to see how each of them perform in comparison. The 4 selected scenarios represent a selection of the challenges posed in ACE: from very abstract ones to very operational ones.

\section{Comparison}
Each of the selected scenarios is then implemented using the selected methods where each solution is then compared against the following criteria: 

\begin{enumerate}
\item suitability for scientific computation
\item robustness
\item error-sources
\item testability
\item stability
\item extendability
\item size of code
\item maintainability
\item time taken for development
\item verification \& correctness
\item replications \& parallelism
\item EDSL
\end{enumerate}

This will then allow to compare the different methods against each other and to show under which circumstances functional methods shine and when they should not be used.

\section{Agent-Based Modelling and Simulation (ABM/S)}
ABM/S is a method of modelling and simulating a system where the global behaviour may be unknown but the behaviour and interactions of the parts making up the system is of knowledge (Wooldrige, M. (2009). An Introduction to MultiAgent Systems. John Wiley & Sons). Those parts, called agents, are modelled and simulated out of which then the aggregate global behaviour of the whole system emerges. Thus the central aspect of ABM/S is the concept of an Agent which can be understood as a metaphor for a pro-active unit, able to spawn new Agents, and interacting with other Agents in a network of neighbours by exchange of messages. The implementation of Agents can vary and strongly depends on the programming language and the kind of domain the simulation and model is situated in.

\section{Agent-Based Economics (ACE)}
According to Leigh Tesfatsion (Tesfatsion, L. (2006). Agent-based computational economics: A constructive approach to economic theory. In Tesfatsion, L. and Judd, K. L., editors, Handbook of Computational Economics, volume 2, chapter 16, pages 831–880. Elsevier, 1 edition.), one of the leading figures, ACE is "[...] computational modelling of economic processes (including whole economies) as open-ended dynamic systems of interacting agents." - thus lending perfectly to the use of ABM/S as already the name suggests. Whereas classical economic models fall short by only looking at the average, pure rational, individual interacting in anonymous markets, the ACE approach looks at heterogeneous, non-rational individuals interacting with each other in networks (Kirman, A. (2010). Complex Economics: Individual and Collective Rationality. Routledge, London ; New York, NY.). Thus ACE can be understood as a combination of computer-science, cognitive/social science and evolutionary economics.

\section{Functional programming}
TODO: read \cite{Backus1978}

The state-of-the-art approach to implementing Agents are object-oriented methods and programming as the metaphor of an Agent as presented above lends itself very naturally to object-orientation (OO). The author of this thesis claims that OO in the hands of inexperienced or ignorant programmers is dangerous, leading to bugs and hardly maintainable and extensible code. The reason for this is that OO provides very powerful techniques of organising and structuring programs through Classes, Type Hierarchies and Objects, which, when misused, lead to the above mentioned problems. Also major problems, which experts face as well as beginners are 1. state is highly scattered across the program which disguises the flow of data in complex simulations and 2. objects don’t compose as well as functions. The reason for this is that objects always carry around some internal state which makes it obviously much more complicated as complex dependencies can be introduced according to the internal state.
All this is tackled by (pure) functional programming which abandons the concept of global state, Objects and Classes and makes data-flow explicit. This then allows to reason about correctness, termination and other properties of the program e.g. if a given function exhibits side-effects or not. Other benefits are fewer lines of code, easier maintainability and ultimately fewer bugs thus making functional programming the ideal choice for scientific computing and simulation and thus also for ACE. A very powerful feature of functional programming is Lazy evaluation. It allows to describe infinite data-structures and functions producing an infinite stream of output but which are only computed as currently needed. Thus the decision of how many is decoupled from how to (Hughes, J. (1989). Why functional programming matters. Comput. J., 32(2):98–107.).
The most powerful aspect using pure functional programming however is that it allows the design of embedded domain specific languages (EDSL). In this case one develops and programs primitives e.g. types and functions in a host language (embed) in a way that they can be combined. The combination of these primitives then looks like a language specific to a given domain, in the case of this thesis ACE. The ease of development of EDSLs in pure functional programming is also a proof of the superior extensibility and composability of pure functional languages over OO (Henderson P. (1982). Functional Geometry. Proceedings of the 1982 ACM Symposium on LISP and Functional Programming.).
One of the most compelling example to utilize pure functional programming is the reporting of Hudak (Hudak P., Jones M. (1994). Haskell vs. Ada vs. C++ vs. Awk vs. ... An Experiment in Software Prototyping Productivity. Department of Computer Science, Yale University.)  where in a prototyping contest of DARPA the Haskell prototype was by far the shortest with 85 lines of code. Also the Jury mistook the code as specification because the prototype did actually implement a small EDSL which is a perfect proof how close EDSL can get to and look like a specification.

Functional languages can best be characterized by their way computation works: instead of \textit{how} something is computed, \textit{what} is computed is described. Thus functional programming follows a declarative instead of an imperative style of programming. The key points are:
\begin{itemize}
\item No assignment statements - variables values can never change once given a value.
\item Function calls have no side-effect and will only compute the results - this makes order of execution irrelevant, as due to the lack of side-effects the logical point in \textit{time} when the function is calculated within the program-execution does not matter.
\item higher-order functions
\item lazy evaluation
\item Looping is achieved using recursion, mostly through the use of the general fold or the more specific map.
\item Pattern-matching
\end{itemize}

This alone does not really explain the \textit{real} advantages of functional programming and one must look for better motivations using functional programming languages. One motivation is given in \cite{Hughes1989} which is a great paper explaining to non-functional programmers what the significance of functional programming is and helping functional programmers putting functional languages to maximum use by showing the real power and advantages of functional languages. The main conclusion is that \textit{modularity}, which is the key to successful programming, can be achieved best using higher-order functions and lazy evaluation provided in functional languages like Haskell. \cite{Hughes1989} argues that the ability to divide problems into sub-problems depends on the ability to glue the sub-problems together which depends strongly on the programming-language and \cite{Hughes1989} argues that in this ability functional languages are superior to structured programming.

TODO: comparison of functional and object-oriented programming. My points are:
\begin{itemize}
\item The way state can be changed and treated - distributed over multiple objects - is often very difficult to understand.
\item Inheritance is a dangerous thing if not used with care because inheritance introduces very strong dependencies which cannot be changed during runtime anymore.
\item Objects don't compose very well: \url{http://zeroturnaround.com/rebellabs/why-the-debate-on-object-oriented-vs-functional-programming-is-all-about-composition/}
\item (Nearly) impossible to reason about programs
\end{itemize}

In conclusion the upsides of functional programming as opposed to OO are:
\begin{itemize}
\item Much more explicit flow of data \& control
\item Much better compose-able
\item Much better parallelism
\end{itemize}

\section{Related Research}
Tim Sweeney, CTO of Epic Games gave an invited talk about how "future programming languages could help us write better code" by "supplying stronger typing, reduce run-time failures;  and the need for pervasive concurrency support, both implicit and explicit, to effectively exploit the several forms of parallelism present in games and graphics." \cite{Sweeney2006}. Although the fields of games and agent-based simulations seem to be very different in the end, they have also very important similarities: both are simulations which perform numerical computations and update objects - in games they are called "game-objects" and in abm they are called agents but they are in fact the same thing - in a loop either concurrently or sequential. His key-points were:

\begin{itemize}
\item Dependent types as the remedy of most of the run-time failures.
\item Parallelism for numerical computation: these are pure functional algorithms, operate locally on mutable state. Haskell ST, STRef solution enables encapsulating local heaps and mutability within referentially transparent code.
\item Updating game-objects (agents) concurrently using STM: update all objects concurrently in arbitrary order, with each update wrapped in atomic block - depends on collisions if performance goes up.
\end{itemize}

\section{Related Work}
TODO: wormholes thesis and paper

\section{Property-Based Testing}
Property-based testing allows to formulate \textit{functional specifications} in code which then a property-based testing library tries to falsify by \textit{automatically} generating test-data with some user-defined coverage. When a case is found for which the property fails, the library then reduces it to the most simple one. It is clear to see that this kind of testing is especially suited to ABS, because we can formulate specifications, meaning we describe \textit{what} to test instead of \textit{how} to test. Also the deductive nature of falsification in property-based testing suits very well the constructive and exploratory nature of ABS. Further, the automatic test-generation can make testing of large scenarios in ABS feasible as it does not require the programmer to specify all test-cases by hand, as is required in unit-tests.

Property-based testing was invented by the authors of \cite{claessen_quickcheck:_2000, claessen_testing_2002} in which they present the QuickCheck library, which tries to falsify the specifications by \textit{randomly} sampling the space. We argue, that the stochastic sampling nature of this approach is particularly well suited to ABS, because it is itself almost always driven by stochastic events and randomness in the agents behaviour, thus this correlation should make it straight-forward to map ABS to property-testing. The main challenge when using QuickCheck, as will be shown later, is to write \textit{custom} test-data generators for agents and the environment which cover the space sufficiently enough to not miss out on important test-cases. According to the authors of QuickCheck \textit{"The major limitation is that there is no measurement of test coverage."} \cite{claessen_quickcheck:_2000}. QuickCheck provides help to report the distribution of test-cases but still it could be the case that simple test-cases which would fail are never tested.

As a remedy for the potential sampling difficulties of QuickCheck, there exists also a deterministic property-testing library called SmallCheck \cite{runciman_smallcheck_2008} which instead of randomly sampling the test-space, enumerates test-cases exhaustively up to some depth. It is based on two observations, derived from model-checking, that (1) \textit{"If a program fails to meet its specification in some cases, it almost always fails in some simple case"} and (2) \textit{"If a program does not fail in any simple case, it hardly ever fails in any case} \cite{runciman_smallcheck_2008}. This non-stochastic approach to property-based testing might be a complementary addition in some cases where the tests are of non-stochastic nature with a search-space which is too large to implement manually by unit-tests but is relatively easy and small enough to enumerate exhaustively. The main difficulty and weakness of using SmallCheck is to reduce the dimensionality of the test-case depth search to prevent combinatorial explosion, which would lead to exponential number of cases. Thus one can see QuickCheck and SmallCheck as complementary instead of in opposition to each other.

Note that in this paper we primarily focus on the use of QuickCheck due to the match of ABS stochastic nature and the random test generation. We refer to SmallCheck in cases where appropriate. Also note that we regard property-based testing as \textit{complementary} to unit-tests and not in opposition - we see it as an addition in the TDD process of developing an ABS.

\section{Testing ABS implementations}
\label{sec:testingABS}

Generally we need to distinguish between two types of testing / verification in ABS.

\begin{enumerate}
	\item Testing / verification of models for which we have real-world data or an analytical solution which can act as a ground-truth - examples for such models are the SIR model, stock-market simulations, social simulations of all kind.
	\item Testing / verification of models which are of exploratory nature, inspired by real-world phenomena but for which no ground-truth per se exists - examples for such models is the Sugarscape \cite{epstein_growing_1996} or Agent\_Zero model \cite{epstein_agent_zero:_2014}.
\end{enumerate}

The baseline is that either one has an analytical model as the foundation of an agent-based model or one does not. In the former case, e.g. the SIR model, one can very easily validate the dynamics generated by the simulation to the one generated by the analytical solution through System Dynamics. In the latter case one has basically no idea or description of the emergent behaviour of the system prior to its execution e.g. SugarScape. In this case it is important to have some hypothesis about the emergent property / dynamics. The question is how verification / validation works in this setting as there is no formal description of the expected behaviour: we don't have a ground-truth against which we can compare our simulation dynamics.

One distinguishes between black-box and white-box verification where in white-box verification one looks directly at code and reasons about it whereas in black-box verification one generally feeds input to the software / functions / methods and compares it to expected output. Black-box verification is our primary concern in this paper as property-based testing is an instance of black-box verification. In the case of ABS we have the following levels of black-box tests:
\begin{enumerate}
	\item Isolated and interacting agent behaviour parts - test the individual parts which make up the agent behaviour under given inputs and test if interaction between agents are correct. For this we can use traditional unit-tests as shown by \cite{collier_test-driven_2013} and also property-based testing as we will show in the use-cases.
	\item Simulation dynamics - compare emergent dynamics of the ABS as a whole under given inputs to an analytical solution or real-world dynamics in case there exists some, using statistical tests. We see this type of tests conceptually as property-tests as well because we are testing properties of the model / simulation as we will see in the use-cases. Technically speaking we can both use traditional unit-tests and also property-based tests to implement them - conceptually they are property-tests.
	\item Hypotheses - test whether hypotheses about the model are valid or invalid. This is very similar to the previous point but without comparing it to analytical solutions or real-world dynamics but only to some hypothetical values.
\end{enumerate}

\section{Case Study I: SIR}
\label{sec:case_SIR}
As first use-case we discuss property-based testing for the \textit{explanatory} agent-based SIR model. It is a very well studied and understood compartment model from epidemiology \cite{kermack_contribution_1927} which allows to simulate the dynamics of an infectious disease like influenza, tuberculosis, chicken pox, rubella and measles spreading through a population. We implemented an agent-based version of this model \footnote{The code is freely accessible from \url{https://github.com/thalerjonathan/phd/tree/master/public/propabs/sir}}, inspired by \cite{macal_agent-based_2010}.

In this model, people in a population of size $N$ can be in either one of three states \textit{Susceptible}, \textit{Infected} or \textit{Recovered} at a particular time, where it is assumed that initially there is at least one infected person in the population. People interact \textit{on average} with a given rate of $\beta$ other people per time-unit and become infected with a given probability $\gamma$ when interacting with an infected person. When infected, a person recovers \textit{on average} after $\delta$ time-units and is then immune to further infections. An interaction between infected persons does not lead to re-infection, thus these interactions are ignored in this model. Due to the models' origin in System Dynamics (SD) \cite{porter_industrial_1962}, there exists a top-down formalisation in SD with the following equations:

\begin{equation}
\begin{aligned}
\frac{\mathrm d S}{\mathrm d t} = -infectionRate \\
\frac{\mathrm d I}{\mathrm d t} = infectionRate - recoveryRate \\
\frac{\mathrm d R}{\mathrm d t} = recoveryRate 
\end{aligned}
\end{equation}

\begin{equation}
\begin{aligned}
infectionRate = \frac{I \beta S \gamma}{N} \\
recoveryRate = \frac{I}{\delta} 
\end{aligned}
\end{equation}

\subsection{Deriving a property}
Our goal is to derive a property which connects the agent-based implementation to the SD equations. The foundation are both the infection- and recovery-rate where the infection-rate determines how many \textit{Susceptible} agents per time-unit become \textit{Infected} and the recovery-rate determines how many \textit{Infected} agents per time-unit become \textit{Recovered}. Lets look at the algorithm of the susceptible agent behaviour, which is key for the infection-rate:

\begin{algorithm}
generate on average $\beta$ make-contact events per time-unit\; 
\If{make-contact event}{
  select random agent \textit{randA} from population\; 
  \If{agent randA infected}{
    become infected with probability $\gamma$\; 
  }  
}
\caption{Susceptible behaviour}
\end{algorithm}

Per time-unit, a susceptible agent makes \textit{on average} contact with $\beta$ other agents, where in the case of a contact with an infected agent, the susceptible agent becomes infected with a given probability $\gamma$. In this description there is another probability hidden, the probability of making contact with an infected agent, which is simply the ratio of number of infected agents to number non-infected agents. We can now derive the formula for the probability of a \textit{Susceptible} agent to become infected: $\frac{\beta * \gamma * \text{number of infected (I)}}{\text{number of non-infected (N)}}$. When we look at the formula we can see that it is conceptually the same representation of the \textit{infection-rate} of the SD specification as shown above - except that it only considers a single \textit{Susceptible} agent instead of the aggregate of \textit{S} susceptible agents. We have now a property we can check using a property-test.

\subsection{Constructing the property-based test}
Having a property (law), we want now to construct a property-test for it. The formula is invariant under random population mixes and thus should hold for varying agent populations where the mix of \textit{Susceptible, Infected and Recovered} agents is random - thus we use QuickCheck to generate the population randomly, the property must still hold.

Obviously we need to pay attention to the fact that we are dealing with a stochastic system thus we can only talk about averages and thus it does not suffice to only run a single agent but we are repeating this for e.g. 10.000 \textit{Susceptible} agents (all with different random-number seeds). 

To check whether this test has passed we compare the required amount of agents which on average should become infected using the above formula to the one from our tests (simply count the agents which got infected and divide by N) and if the value lies within some small $\epsilon$ then we accept the test as passed. Now we can construct the following property-based test as shown in Algorithm \ref{alg:prop_test_infectionrate}.

\begin{algorithm}
\SetKwInOut{Input}{input}\SetKwInOut{Output}{output}
\Input{List \textit{randAs} of random agent-population generated by QuickCheck}
populationCount     = length \textit{randAs}\;
infectedCount       = count \textit{Infected} in \textit{randAs}\;
infectionRate       = infectivity * contactRate * (infectedCount / populationCount)\;

susceptibles = create 10000 \textit{Susceptible} agents\;
countInfected = 0\;
\For{each agent sa in susceptibles}{
  run agent sa for 1.0 time-unit, with list \textit{randAs} as input\;
  \If{agent sa became \textit{Infected} }{
	countInfected = countInfected + 1\;
  }
}

averageInfectionRate = countInfected / (length susceptibles)\;
$\epsilon$ = 0.1\;
\eIf{abs (averageInfectionRate - infectionRate) $\leq \epsilon$}{
  PASS\;
} {
  FAIL\;
}
\caption{Property-based test for infection-rate.}
\end{algorithm}
\label{alg:prop_test_infectionrate}

When running, QuickCheck generates 100 test-cases by randomly generating 100 different \textit{randAs} inputs to the test. All have to pass for the whole property-test to pass, which should be the case with an $\epsilon = 0.1$. 

This is the very power which property-based testing is offering us: we directly express the specification of the original SD model in a test of our agent-based implementation and let QuickCheck generate random test cases for us. This closely ties our implementation to the original specification and raises the confidence to a very high level that it is actually a valid and correct implementation.

%\subsection{Infected Behaviour}
%An infected agent will \textit{always} recover after some finite time, which is \textit{on average} after $\delta$ time-units. Note that this property involves stochastics too, so to test this property we run a large number of infected agents e.g. $N = 10.000$ (all with different random-number seeds) until they recover, record the time of each agents recovery and then average over all recovery times. To check whether this test has passed we compare the average recovery times to $\delta$ and if they lie within some small $\epsilon$ then we accept the test as passed (note again that we could use a t-test for better stochastic robustness but this is not the point of this paper).
%
%TODO: clearly state the property we test
%
%TODO: produce some pseudo-code of how the property-test conceptually works
%
%in the infected agent test we check if the average duration is as specified. does this resemble the recovery rate? or in other words: can we somehow test the recovery rate?
%durationsAvg = sum durations / fromIntegral (length durations)
%
%We use property-testing with QuickCheck in this case as well to generate the set of other agents as input for the infected agents. Strictly speaking this would not be necessary as an infected agent never makes contact with other agents and simply ignores them - we could as well just feed in an empty list. We opted for using QuickCheck for the following reasons:
%
%\begin{itemize}
%	\item We wanted to stick to the interface specification of the agent-implementation as close as possible which asks to pass the states of all agents as input.
%	\item We shouldn't make any assumptions about the actual implementation and if it REALLY ignores the other agents, so we strictly stick to the interface which requires us to input the states of all the other agents.
%	\item The set of other agents is ignored when determining whether the test has failed or not which indicates by construction that the behaviour of an infected agent does not depend on other agents.
%	\item We are not just running a single replication over 10.000 agents but 100 of them which should give black-box verification more strength.
%\end{itemize}
%
%\subsection{Recovered Behaviour}
%A recovered agent will stay recovered \textit{forever}. Obviously we cannot write a property-based test that truly verifies that because it had to run in fact \textit{forever}. In this case we need to resort to white-box verification and look directly at the code and reason whether this property holds true.

\section{Case Study II: SugarScape}
\label{sec:case_sug}
We now look at how property-based testing can be made of use in the \textit{exploratory} Sugarscape model \cite{epstein_growing_1996}. It was one of the first models in ABS, with the aim to \textit{grow} an artificial society by simulation and connect observations in their simulation to phenomenon observed in real-world societies. In this model a population of agents move around in a discrete 2D environment, where sugar grows, and interact with each other and the environment in many different ways. The main features of this model are (amongst others): searching, harvesting and consuming of resources, wealth and age distributions, population dynamics under sexual reproduction, cultural processes and transmission, combat and assimilation, bilateral decentralized trading (bartering) between agents with endogenous demand and supply, disease processes transmission and immunology. For our research we undertook a \textit{full and validated} implementation of the Sugarscape model \footnote{The code can be accessed freely from \url{https://github.com/thalerjonathan/phd/tree/master/public/towards/SugarScape/sequential}}. We undertook a full validation of our implementation against the book \cite{epstein_growing_1996} and a NetLogo implementation \cite{weaver_replicating_nodate} during which we also implemented property tests. Due to lack of space we added a discussion of the validation process as an Appendix \ref{app:validation}.

Whereas in the explanatory SIR case-study we had an analytical solution, inspired by the SD origins of the model, the fundamental difference in the exploratory Sugarscape model is that none such analytical solutions exist. This raises the question, which properties we can actually test in such a model: 
\begin{itemize}
	\item Environment behaviour parts.
	\item Agent behaviour parts.
	\item Hypotheses about emergent properties which when proved to be valid can be seen as regression tests.
\end{itemize}

\subsection{Environment behaviour}
The environment in the Sugarscape model has some very simple behaviour: each site has a sugar level and when harvested by an agent, it regrows back to the full level over time. Depending on the configuration of the model it either grows back immediately within 1 tick or over multiple ticks. We can construct simple property-based tests for these behaviours. In the case the sugar grows back immediately we let QuickCheck generate a random environment and then run the environment behaviour for 1 tick and then check the property that all sites have to be back to their maximum sugar level. In the case of regrow over multiple ticks, we also use QuickCheck to generate a random environment but additionally a random \textit{positive} rate (which is a floating point number) which we then use to calculate the number of steps until full regrowth. After running the random environment for the given number of steps all sites have to be back to full sugar level - we provided pseudo code for this case, see \ref{alg:prop_test_rateregwroth}.

Note that QuickCheck initially doesn't know how to generate a random environment because each site consists of a custom data-structure for which QuickCheck is not able to generate random instances by default. This problem is solved by writing a custom data-generator, for which existing QuickCheck functions can be used e.g. picking the current sugar level of a site from a random range.

%\begin{algorithm}
%\SetKwInOut{Input}{input}\SetKwInOut{Output}{output}
%\Input{Random environment \textit{env} generated by QuickCheck}
%env' = runEnvironmentTicks 1 env\;
%sites = getEnvironmentSites env'\;
%
%\eIf{all sites maxSugarLevel}{
%  PASS\;
%} {
%  FAIL\;
%}
%\caption{Property-based test for immediate regrow of sugar on all sites.}
%\end{algorithm}
%\label{alg:prop_test_fullregrowth}

\begin{algorithm}
\SetKwInOut{Input}{input}\SetKwInOut{Output}{output}
\Input{Random environment \textit{env} generated by QuickCheck}
\Input{Regrowth rate \textit{randRate} (positive floating point) generated by QuickCheck}
maxTicks = maxSugarCapacityOnSites / randRate\;
env' = runEnvironmentTicks maxTicks env\;
sites = getEnvironmentSites env'\;

\eIf{all sites maxSugarLevel}{
  PASS\;
} {
  FAIL\;
}
\caption{Property-based test for rate-based regrow of sugar on all sites.}
\end{algorithm}
\label{alg:prop_test_rateregwroth}

The Sugarscape environment is a torus where the coordinates wrap around in both dimensions. To check whether the implementation of the wrapping-calculation is correct we used both unit- and property-tests. With the unit-tests we carefully constructed all possible cases we could think of and came up with 13 test-cases. With the property-based test we simply defined a single test-case where we expressed the property that after wrapping \textit{any} coordinates, supplied by QuickCheck, the wrapped coordinates have to be within bounds. See pseudo code \ref{alg:prop_test_wrapcoords}.

\begin{algorithm}
\SetKwInOut{Input}{input}\SetKwInOut{Output}{output}
\Input{Random 2d discrete coordinate \textit{randCoord} generated by QuickCheck}
(x, y) = wrapCoordinates randCoord\;

\eIf{(x $\geq$ 0 and x $\leq$ environmentDimX) and (y $\geq$ 0 and y $\leq$ environmentDimY)}{
  PASS\;
} {
  FAIL\;
}
\caption{Property-based test for wrap-coordinates functionality.}
\end{algorithm}
\label{alg:prop_test_wrapcoords}

\subsection{Agent behaviour parts}
We implemented a number of tests for agent functions which just cover the part of an agents behaviour: checks whether an agent has died of age or starved to death, the metabolism, immunisation step, check if an agent is a potential borrower or fertile, lookout, trading transaction. What all these functions have in common is that they are not pure computations like utility functions but require an agent-continuation which means they have access to the agent state, environment and random-number stream. This allows testing to capture the \textit{complete} system state in one location, which allows the checking of much more invariants than in approaches which have implicit side-effects.

We implement custom data-generators for our agent state and environment and its cells and then let QuickCheck generate the random data and us running the agent with the provided data, checking for the properties. An example for such a property is that an agent has starved to death in case its sugar (or spice) level has dropped to 0. The corresponding property-test generates a random agent state and also a random sugar level which we set in the agent state. We then run the function which returns True in case the agent has starved to death. We can then check that this flag is true only if the initial random sugar level was less then or equal 0.

%What is particularly powerful is that one has complete control and insight over the changed state before and after e.g. a function was called on an agent: thus it is very easy to check if the function just tested has changed the agent-state itself or the environment: the new environment is returned after running the agent and can be checked for equality of the initial one - if the environments are not the same, one simply lets the test fail. This behaviour is very hard to emulate in OOP because one can not exclude side-effect at compile time, which means that some implicit data-change might slip away unnoticed. In FP we get this for free.

\subsection{Emergent Properties}
In our validation and verification process of our Sugarscape implementation we put informal descriptions and hypotheses about emergent properties from the Sugarscape book into formal property-tests. Examples for such hypotheses / informal descriptions of emergent properties are e.g. the carrying capacity becomes stable after 100 steps; when agents trade with each other after 1000 steps the standard deviation of trading prices is less than 0.05; when there are cultures after 2700 steps either one culture dominates the other or both are equally present.

The property we test for is whether \textit{the emergent property under test is stable under varying random-number seeds} or not. Put another way, we let QuickCheck generate random number generators and require that the tests all pass with arbitrary random number streams. Unfortunately this revealed that this property didn't hold for all emergent properties. The problem is that QuickCheck generates by default 100 test-cases in for each property-test where all need to pass for the whole property-test to pass - this wasn't the case, where most of the 100 test-cases passed but unfortunately not all. Thus in this case a different approach is required: instead of requiring \textit{every} test to pass we require that \textit{most} tests pass, which can be achieved using a t-test with a confidence interval of e.g. 95\%. This means we won't use QuickCheck anymore and resort to a normal unit-test where we run the simulation 100 times with different random number streams each time and then performing a t-test with a 95\% confidence interval. Note that we are now technically speaking of a unit-test but conceptually it is still a property-test.

In listing \ref{alg:prop_test_trading} we show the pseudo code of a property-test for checking whether after 1000 steps the standard deviation of trading prices is less than 0.05. The test passes if out of 100 runs a 95\% confidence interval is reached using a t-test.

\begin{algorithm}
maxTicks = 1000\;
replications = 100\;
stdAverage = 0.05\;
tradingPriceStdsList = empty list\;

\For{$i\leftarrow 1$ \KwTo replications}{
rng = new random number generator\;
simContext = initSimulation rng\;
out = runSimulation maxTicks simContext\;
tps = extractTradingPrices out\;
tpsStd = calculate standard deviation of tps\;
insert tpsStd into tradingPriceStdsList\;
}

tTestPass = perform 1-sided t-test comparing stdAverage with tradingPriceStdsList on a 0.95 interval\;

\eIf{tTestPass}{
  PASS\;
} {
  FAIL\;
}
\caption{Property-based test for trading prices.}
\end{algorithm}
\label{alg:prop_test_trading}

\chapter{Conclusions}
\label{ch:conclusions}

This chapter concludes the whole thesis and outlines future research. Roughly 20\% exists already.

%we now know how to engineer time- and event-driven ABS with complex state both in the agent and environment, main difficulty is direct agent-interaction (see macal classification into 4 types of ABS), compile-time guaranteed reproducibility, explicit handling of complex state (read only, read/write), concurrency explicit and limited to STM, very promising concurrency but direct agent-interactions main problem (erlang as a rescue?), main drawbacks: everything is explicit, performance

\section{Further Research}
clearly outline the ideas for further research

\subsection{A general purpose library}
generalise concepts explored into a pure functional ABS library in Haskell (called chimera)

\subsection{Dependent and linear types}
dependent types and linear types are the next big step, towards a stronger formalisation of agents and ABS,
focus on the equilibrium - totality correspondence

\subsection{Concurrent event-driven ABS}
stm based concurrency for event-driven ABS using parallel DES. challenge is the time-warp implementation using monads. in general it should be easy to roll-back agents actions but with monads we have to be careful - for some monads rolling back is not neccessary e.g. rand and reader, for others it is, and for some it is impossible e.g. IO

\section*{Acknowledgments}
The authors would like to thank J. Hey and M. Handley for valuable feedback and discussions.

% Please don't change the bibliographystyle style
\bibliographystyle{scsproc}
% AUTHOR: Include your bib file here
\bibliography{../../../references/phdReferences.bib}


\appendix

\newpage

\chapter{Validating Sugarscape in Haskell}
\label{app:validating_sugarscape}

In this chapter we look at how property-based testing can be made of use to verify the \textit{exploratory} Sugarscape model \cite{epstein_growing_1996} as introduced in Chapter \ref{sec:sugarscape}. Whereas in the chapters on testing the explanatory SIR model we had an analytical solution, the fundamental difference in the exploratory Sugarscape model is that none such analytical solutions exist. This raises the question, which properties we can actually test in such a model.

The answer lies in the very nature of exploratory models, they exist to explore and understand phenomena of the real world. Researchers come up with a model to explain the phenomena and (hopefully) with a few questions and \textit{hypotheses} about the emergent properties. The actual simulation is then used to test and refine the hypotheses. Indeed, descriptions, assumptions and hypotheses of varying formal degree abound in the Sugarscape model. Examples are: \textit{the carrying capacity becomes stable after 100 steps; when agents trade with each other, after 1000 steps the standard deviation of trading prices is less than 0.05; when there are cultures, after 2700 steps either one culture dominates the other or both are equally present}. 

We show how to use property-based testing to formalise and check such hypotheses. For this purpose we undertook a full \textit{verification} of our \href{https://github.com/thalerjonathan/haskell-sugarscape}{implementation}~\cite{thaler_sugarscape_repository} from Chapter \ref{sec:sugarscape}. We validated it against the book \cite{epstein_growing_1996} and a NetLogo implementation \cite{weaver_replicating_2009}  \footnote{Lending didn't properly work in their NetLogo code and that they didn't implement Combat.}.

\section{Property-based hypothesis testing}
The property we test for is whether \textit{the emergent property / hypothesis under test is stable under replicated runs} or not. To put it more technical, we use QuickCheck to run multiple replications with the same configuration but with different random-number streams and require that all tests pass. During the verification process we have derived and implemented property tests for the following hypotheses:

\begin{enumerate}
	\item Disease dynamics where all agents recover - when disease are turned on, if the number of initial diseases is 10, then the population is  able to rid itself completely from all disease within 100 ticks. 
	
	\item Disease dynamics where a minority recovers - when disease are turned on, if the number of initial diseases is 25, the population is not able to rid itself completely from all diseases within 1,000 ticks.
	
	\item Trading dynamics - when trading is enabled, the trading prices stabilise after 1,000 ticks with the standard deviation of the prices having dropped below 0.05.
	
	\item Cultural dynamics - when having two cultures, red and blue, after 2,700 ticks, either the red or the blue culture dominates or both are equally strong. If they dominate they make up 95\% of all agents, if they are equally strong they are both within 45\% - 55\%.
	
	\item Inheritance Gini coefficient - when agents reproduce and can die of age then inheritance of their wealth leads to an unequal wealth distribution measured using the Gini Coefficient \textit{averaging} at 0.7.
	
	\item Carrying capacity - when agents don't mate nor can die from age, due to the environment, there is an \textit{average} maximum carrying capacity of agents the environment can sustain. The capacity should be reached after 100 ticks and should be stable from then on.
		
	\item Terracing - when resources regrow immediately, after a few steps the simulation becomes static. Agents will stay on their terraces and will not move any more because they have found the best spot due to their behaviour. About 45\% will be on terraces and 95\% - 100\% are static, not moving any more.
\end{enumerate}

The hypotheses and their validation is described more in-depth in the section \ref{sec:hypotheses_testcases} below.

\subsection{Implementation}
To start with, we implement a custom data generator to produce output from a Sugarscape simulation. The generator takes the number of ticks and the scenario with which to run the simulation and returns a list of outputs, one for each tick.

\begin{HaskellCode}
sugarscapeUntil :: Int                -- ^ Number of ticks to run
                -> SugarScapeScenario -- ^ Scenario to run
                -> Gen [SimStepOut]   -- ^ Output of each step
sugarscapeUntil ticks params = do
  -- create a random-number generator
  g <- genStdGen
  -- initialise the simulation state with the given random-number generator
  -- and the scenario
  let simState = initSimulationRng g params
  -- run the simulation with the given state for number of ticks
  return (simulateUntil ticks simState)
\end{HaskellCode}

Using this generator, we can very conveniently produce Sugarscape data within a QuickCheck \texttt{Property}. Depending on the problem, we can generate only a single run or multiple replications, in case the hypothesis is assuming \textit{averages}. To see its use, we show the implementation of the \textit{Disease Dynamics (1)} hypothesis.

\begin{HaskellCode}
prop_disease_allrecover :: Property
prop_disease_allrecover = property (do
  -- after 100 ticks...
  let ticks = 100
  -- ... given Animation V-1 parameter configuration ...
  let params = mkParamsAnimationV_1
  -- ... from 1 sugarscape simulation ...
  aos <- last <*> (sugarscapeUntil ticks params)
  -- ... counting all infected agents ...
  let infected = length (filter (==False)) map (null . sugObsDiseases . snd) aos
  -- ... should result in all agents to be recovered
  return (cover 100 (infected == 0) "Diseases all recover" True)
\end{HaskellCode}

From the implementation it becomes clear, that this hypothesis states that the property has to hold \textit{for all} replications. The \textit{Inheritance Gini Coefficient (5)} hypothesis on the other hand assumes that the Gini Coefficient \textit{averages} at 0.7. We cannot average over replicated runs of the same property thus we generate multiple replications of the Sugarscape data within the property and employ a two-sided t-test with a 95\% confidence to test the hypothesis:

\begin{HaskellCode}
prop_gini :: Int      -- ^ Number of replications
          -> Double   -- ^ Confidence of the t-test
          -> Property
prop_gini repls confidence = property (do
  -- after 1000 ticks...
  let ticks = 1000
  -- ... the gini coefficient should average at 0.7 ...
  let expGini = 0.7
  -- ... given the Figure III-7 parameter configuration ...
  let params = mkParamsFigureIII_7
  -- ... from 100 replications ... 
  gini <- vectorOf repls (genGiniCoeff ticks params)
  -- on a two-tailed t-test with given confidence
  return (tTestSamples TwoTail expGini (1 - confidence) gini)
\end{HaskellCode}

%genGiniCoeff :: Int -> SugarScapeScenario -> Gen Double
%genGiniCoeff ticks params = do
%  -- generate sugarscape data
%  aos <- sugarscapeUntil ticks params
%  -- extract wealth of the agents in the last step
%  let agentWealths = map (sugObsSugLvl . snd) (last aos)
%  -- compute gini coefficient and return it
%  return (giniCoeff agentWealths)

\subsection{Running the tests}
As already pointed out in Part \ref{ch:property}, QuickCheck by default runs up to 100 test cases of a property and if all evaluate to \texttt{True} the property test succeeds. On the other hand, QuickCheck will stop at the first test case which evaluates to \texttt{False} and marks the whole property test as failed, no matter how many test cases got through already. For this reason we have used \texttt{cover} with an expected percentage of 100, meaning that we expect all tests to fall into the coverage class. This allows us to emulate failure with QuickCheck reporting the actual percentage of passed test cases.

Due to the duration even 1,000 ticks can take to compute, to get a first estimate of our hypotheses tests within reasonable time, we reduce the number of maximum successful replications required to 10 and when doing t-tests 10 replications are run there as well. 

\begin{verbatim}
SugarScape Tests
  Disease Dynamics All Recover:      OK (29.25s)
    +++ OK, passed 10 tests (100% Diseases all recover).
    
  Disease Dynamics Minority Recover: OK (536.00s)
    +++ OK, passed 10 tests (100% Diseases no recover).
    
  Trading Dynamics:                  OK (149.33s)
    +++ OK, passed 10 tests (70% Prices std less than 5.0e-2).
    Only 70% Prices std less than 5.0e-2, but expected 100%
    
  Cultural Dynamics:                 OK (996.84s)
    +++ OK, passed 10 tests (50% Cultures dominate or equal).
    Only 50% Cultures dominate or equal, but expected 100%
    
  Carrying Capacity:   OK (988.20s)
    +++ OK, passed 10 tests (90% Carrying capacity averages at 204.0).    
    Only 90% Carrying capacity averages at 204.0, but expected 100%
    
  Terracing:           OK (280.59s)
    +++ OK, passed 10 tests (80% Terracing is happening).
    Only 80% Terracing is happening, but expected 100%
    
  Inheritance Gini:    OK (7232.59s)
    +++ OK, passed 0 tests (0% Gini coefficient averages at 0.7).
    Only 0% Gini coefficient averages at 0.7, but expected 100%
\end{verbatim}

%\begin{enumerate}
%	\item Disease Dynamics all recover: \textit{+++ OK, passed 10 tests.}
%
%	\item Disease Dynamics minority recover: \textit{+++ OK, passed 10 tests.}
%		
%	\item Trading Dynamics: \textit{+++ OK, passed 10 tests; 2 failed (16\%).} (In total 12 tests (replications) were run, out of which 2 failed, which is a 16\% failure rate.)
%	
%	\item Cultural Dynamics: \textit{+++ OK, passed 10 tests; 3 failed (23\%).}
%
%	\item Inheritance Gini Coefficient: \textit{*** Failed! Passed only 0 tests; 10 failed (100\%) tests.}
%
%	\item Carrying Capacity: \textit{+++ OK, passed 10 tests; 2 failed (16\%).}
%
%	\item Terracing: \textit{+++ OK, passed 10 tests; 2 failed (16\%).}
%\end{enumerate}

How to deal with the failure of hypotheses is obviously highly model specific. A first approach is to increase the number of replications to run to 100 to get a more robust estimate of the failure rate. If the failure rate stays within reasonable ranges then one can arguably assume that the hypothesis is valid for sufficiently enough cases. On the other hand, if the failure rate escalates, then it is reasonable to deem the hypothesis invalid and refine it or even abandon it altogether.

With the exception of the Gini coefficient, we accept the failure rate of the hypotheses we presented here and deem them sufficiently valid for the task at hand. In case of the Gini coefficient, none of the replication was successful, which makes it obvious that it does \textit{not} average at 0.7. Thus the hypothesis as stated in the book does not hold and is invalid. One way to deal with it would be to simply delete it. Another, more constructive approach, is to keep it but require all replications to fail by marking it with \texttt{expectFailure} instead of \texttt{property}. In this way an invalid hypothesis is marked explicitly and acts as documentation and also as regression test.

\section{Hypotheses and test cases}
\label{sec:hypotheses_testcases}

In this section we briefly describe the process of validating our Sugarscape implementation against the specification of the Sugarscape book \cite{epstein_growing_1996} and the work of \cite{weaver_replicating_2009}.

\subsection{Terracing}
Our implementation reproduces the terracing phenomenon as described in the book and as can be seen in the NetLogo implementation as well. We implemented a property test in which we measure the closeness of agents to the ridge: counting the number of same-level sugars cells around them and if there is at least one lower then they are at the edge. If a certain percentage is at the edge then we accept terracing. The question is just how much, which we estimated from tests and resulted in 45\%. Also, in the terracing animation the agents actually never move which is because sugar immediately grows back thus there is no incentive for an agent to actually move after it has moved to the nearest largest cite in can see. Therefore we test that the coordinates of the agents after 50 steps are the same for the remaining steps.

\subsection{Carrying capacity}
Our simulation reached a steady state (variance $<$ 4 after 100 steps) with a mean around ~182. Epstein reported a carrying capacity of 224 (page 30) and the NetLogo implementations' \cite{weaver_replicating_2009} carrying capacity fluctuates around 205 which both are significantly higher than ours. Something was definitely wrong - the carrying capacity has to be around 200 (we trust in this case the NetLogo implementation and deem 224 an outlier).

After inspection of the NetLogo model we realised that we implicitly assumed that the metabolism range is \textit{continuously} uniformly randomized between 1 and 4 but this seemed not what the original authors intended: in the NetLogo model there were a few agents surviving on sugar level 1 which was never the case in ours as the probability of drawing a metabolism of exactly 1 is practically zero when drawing from a continuous range. We thus changed our implementation to draw a discrete value as the metabolism. %Note that this actually makes sense as massive floating-point number calculations were quite expensive in the mid 90s (e.g. computer games ran still on CPU only and exploited various  clever tricks to avoid the need of floating point calculations whenever possible) when SugarScape was implemented which might have been a reason for the authors to assume it implicitly.

This partly solved the problem, the carrying capacity was now around 204 which is much better than 182 but still a far cry from 210 or even 224. After adjusting the order in which agents apply the Sugarscape rules, by looking at the code of the NetLogo implementation, we arrived at a comparable carrying capacity of the NetLogo implementation: agents first make their move and harvest sugar and only after this the agents metabolism is applied (and ageing in subsequent experiments).

For regression tests we implemented a property test which tests that the carrying capacity of 100 simulation runs lies within a 95\% confidence interval of a 210 mean. These values are quite reasonable to assume, when looking at the NetLogo implementation - again we deem the reported carrying capacity of 224 in the book to be an outlier / part of other details we don't know.

One lesson learned is that even such seemingly minor things like continuous vs. discrete or order of actions an agent makes, can have substantial impact on the dynamics of a simulation.

\subsection{Wealth distribution}
By visual comparison we validated that the wealth distribution (page 32-37) becomes strongly skewed with a histogram showing a fat tail, power-law distribution where very few agents are very rich and most of the agents are quite poor. We compute the skewness and kurtosis of the distribution which is around a skewness of 1.5, clearly indicating a right skewed distribution and a kurtosis which is around 2.0 which clearly indicates the 1st histogram of Animation II-3 on page 34. Also we compute the Gini coefficient and it varies between 0.47 and 0.5 - this is accordance with Animation II-4 on page 38 which shows a gini-coefficient which stabilises around 0.5 after. 
We implemented a regression-test testing skewness, kurtosis and gini coefficients of 100 runs to be within a 95\% confidence interval of a two-sided t-test using an expected skewness of 1.5, kurtosis of 2.0 and gini coefficient of 0.48.

\subsection{Migration}
With the information provided by \cite{weaver_replicating_2009} we could replicate the waves as visible in the NetLogo implementation as well. Also we propose that a vision of 10 is not enough yet and shall be increased to 15 which makes the waves very prominent and keeps them up for much longer - agent waves are travelling back and forth between both Sugarscape peaks. We have not implemented a regression test for this property as we couldn't come up with a reasonable straightforward approach to implement it.

\subsection{Pollution and diffusion}
With the information provided by \cite{weaver_replicating_2009} we could replicate the pollution behaviour as visible in the NetLogo implementation as well. We have not implemented a regression test for this property as we couldn't come up with a reasonable straightforward approach to implement it.

%Note that we spent quite a lot of time of getting this and the terracing properties right because they form the very basics of the other ones which follow so we had to be sure that those were correct otherwise validating would have been much more difficult.

%\section{Order of Rules}
%order in which rules are applied is not specified and might have an impact on dynamics e.g. when does the agent mate with others: is it after it has harvested but before metabolism kicks in?

\subsection{Mating}
We could not replicate Figure III-1 - our dynamics first raised and then plunged to about 100 agents and go then on to recover and fluctuate around 300. This findings are in accordance with \cite{weaver_replicating_2009}, where they report similar findings - also when running their NetLogo code we find the dynamics to be qualitatively the same.

Also at first we weren't able to reproduce the cycles of population sizes. Then we realised that our agent behaviour was not correct: agents which died from age or metabolism could still engage in mating before actually dying - fixing this to the behaviour, that agents which died from age or metabolism will not engage in mating solved that and produces the same swings as in \cite{weaver_replicating_2009}. Although our bug might be obvious, the lack of specification of the order of the application of the rules is an issue in the SugarScape book.

\subsection{Inheritance}
We couldn't replicate the findings of the Sugarscape book regarding the Gini coefficient with inheritance. The authors report that they reach a gini coefficient of 0.7 and above in Animation III-4. Our Gini coefficient fluctuated around 0.35. Compared to the same configuration but without inheritance (Animation III-1) which reached a Gini coefficient of about 0.21, this is indeed a substantial increase - also with inheritance we reach a larger number of agents of around 1,000 as compared to around 300 without inheritance.
The Sugarscape book compares this to chapter II, Animation II-4 for which they report a Gini coefficient of around 0.5 which we could reproduce as well. The question remains, why it is lower (lower inequality) with inheritance?

The baseline is that this shows that inheritance indeed has an influence on the inequality in a population. Thus we deemed that our results are qualitatively the same as the make the same point. Still there must be some mechanisms going on behind the scenes which are unspecified in the original Sugarscape.

\subsection{Cultural dynamics}
We could replicate the cultural dynamics of AnimationIII-6 / Figure III-8: after 2700 steps either one culture (red / blue) dominates both hills or each hill is dominated by a different ulture. We wrote a test for it in which we run the simulation for 2.700 steps and then check if either culture dominates with a ratio of 95\% or if they are equal dominant with 45\%. Because always a few agents stay stationary on sugarlevel 1 (they have a metabolism of 1 and cant see far enough to move towards the hills, thus stay always on same spot because no improvement and grow back to 1 after 1 step), there are a few agents which never participate in the cultural process and thus no complete convergence can happen. This is accordance with \cite{weaver_replicating_2009}.

\subsection{Combat}
Unfortunately \cite{weaver_replicating_2009} didn't implement combat, so we couldn't compare it to their dynamics. Also, we weren't able to replicate the dynamics found in the Sugarscape book: the two tribes always formed a clear battlefront where some agents engage in combat, for example when one single agent strays too far from its tribe and comes into vision of the other tribe it will be killed almost always immediately. This is because crossing the sugar valley is costly: this agent wont harvest as much as the agents staying on their hill thus will be less wealthy and thus easier killed off. Also retaliation is not possible without any of its own tribe anywhere near.

We didn't see a single run where an agent of an opposite tribe "invaded" the other tribes hill and ran havoc killing off the entire tribe. We don't see how this can happen: the two tribes start in opposite corners and quickly occupy the respective sugar hills. So both tribes are acting on average the same and also because of the number of agents no single agent can gather extreme amounts of wealth - the wealth should rise in both tribes equally on average. Thus it is very unlikely that a super-wealthy agent emerges, which makes the transition to the other side and starts killing off agents at large. First: a super-wealthy agent is unlikely to emerge, second making the transition to the other side is costly and also low probability, third the other tribe is quite wealthy as well having harvested for the same time the sugar hill, thus it might be that the agent might kill a few but the closer it gets to the center of the tribe the less like is a kill due to retaliation avoidance - the agent will simply get killed by others.

Also it is unclear in case of AnimationIII-11 if the R rule also applies to agents which get killed in combat. Nothing in the book makes this clear and we left it untouched so that agents who only die from age (original R rule) are replaced. This will lead to a near extinction of the whole population quite quickly as agents kill each other off until 1 single agent is left which will never get killed in combat because there are no other agents who could kill it - instead it will enter an infinite die and  reborn cycle thanks to the R rule.

\subsection{Spice}
The book specifies for AnimationIV-1 a vision between 1-10 and a metabolism between 1-5. The last one seems to be quite strange because the maximum sugar / spice an agent can find is 4 which means that agents with metabolism of either 5 will die no matter what they do because the can never harvest enough to satisfy their metabolism. When running our implementation with this configuration the number of agents quickly drops from 400 to 105 and continues to slowly degrade below 90 after around 1000 steps.
The implementation of \cite{weaver_replicating_2009} used a slightly different configuration for AnimationIV-1, where they set vision to 1-6 and metabolism to 1-4. Their dynamics stabilise to 97 agents after around 500+ steps. When we use the same configuration as theirs, we produce the same dynamics.
Also it is worth nothing that our visual output is strikingly similar to both the book AnimationIV-1 and \cite{weaver_replicating_2009}.

\subsection{Trading}
For trading we had a look at the NetLogo implementation of \cite{weaver_replicating_2009}: there an agent engages in trading with its neighbours \textit{over multiple rounds} until either MRSs cross over or no trade has happened anymore. Because \cite{weaver_replicating_2009} were able to exactly replicate the dynamics of the trading time series we assume that their implementation is correct. We think that the fact that an agent interact with its neighbours over multiple rounds is made not very clear in the book. The only hint is found on page 102: \textit{"This process is repeated until no further gains from trades are possible."} which is not very clear and does not specify exactly what is going on: does the agent engage with all neighbours again? is the ordering random? Another hint is found on page 105 where trading is to be stopped after MRS crossover to prevent an infinite loop. Unfortunately this is missing in the Agent trade rule T on page 105. Additional information on this is found in footnote 23 on page 107. Further on page 107: \textit{"If exchange of the commodities will not cause the agents' MRSs to cross over then the transaction occurs, the agents recompute their MRSs, and bargaining begins anew."}. This is probably the clearest hint that trading could occur over multiple rounds.

We still managed to exactly replicate the trading dynamics as shown in the book in Figure IV-3, Figure IV-4 and Figure IV-5. The book is also pretty specific on the dynamics of the trading prices standard deviation: on page 109 the authors specify that at t=1000 the standard deviation will have always fallen below 0.05 (Figure IV-5), thus we implemented a property test which tests for exactly that property. Unfortunately we didn't reach the same magnitude of the trading volume where ours is much lower around 50 but it is equally erratic, so we attribute these differences to other missing specifications or different measurements because the price dynamics match that well already so we can safely assume that our trading implementation is correct.

According to the book, Carrying Capacity (Animation II-2) is increased by Trade (page 111/112). To check this it is important to compare it not against AnimationII-2 but a variation of the configuration for it where spice is enabled, otherwise the results are not comparable because carrying capacity changes substantially when spice is on the environment and trade turned off. We could replicate the findings of the book: the carrying capacity increases slightly when trading is turned on. Also does the average vision decrease and the average metabolism increase. This makes perfect sense: trading allows genetically weaker agents to survive which results in a slightly higher carrying capacity but shows a weaker genetic performance of the population.

According to the book, increasing the agent vision leads to a faster convergence towards the (near) equilibrium price (page 117/118/119, Figure IV-8 and Figure IV-9). We could replicate this behaviour as well.

According to the book, when enabling R rule and giving agents a finite life span between 60 and 100 this will lead to price dispersion: the trading prices will not converge around the equilibrium and the standard deviation will fluctuate wildly (page 120, Figure IV-10 and Figure IV-11). We could replicate this behaviour as well.

The Gini coefficient should be higher when trading is enabled (page 122, Figure IV-13) - We could replicate this behaviour.

Finite lives with sexual reproduction lead to prices which don't converge (page 123, Figure IV-14). We could reproduce this as well but it was important to set the parameters to reasonable values: increasing number of agents from 200 to 400, metabolism to 1-4 and vision to 1-6, most important the initial endowments back to 5-25 (both sugar and spice) otherwise hardly any mating would happen because the agents need too much wealth to engage (only fertile when have gathered more than initial endowment). What was kind of interesting is that in this scenario the trading volume of sugar is substantially higher than the spice volume - about 3 times as high. 

From this part, we didn't implement: Effect of Culturally Varying Preferences, page 124 - 126, Externalities and Price Disequilibrium: The effect of Pollution, page 126 - 118, On The Evolution of Foresight page 129 / 130. 

%\section{Lending (Credit)}
%Not really much information to validate was available and the \cite{weaver_replicating_2009} implementation ran into an exception so there was not much to validate against. What was unexpected was that this was the most complex behaviour to implement, with lots of subtle details to take care of (spice on/off, inheritance,...).
%Note that we implemented lending of sugar and spice, although it looks from the book (Animation IV-5) that they only implemented it for sugar.

\subsection{Diseases}
We were able to exactly replicate the behaviour of Animation V-1 and Animation V-2: in the first case the population rids itself of all diseases (maximum 10) which happens pretty quickly, in less than 100 ticks. In the second case the population fails to do so because of the much larger number of diseases (25) in circulation. We used the same parameters as in the book. 
The authors of \cite{weaver_replicating_2009} could only replicate the first animation exactly and the second was only deemed "good". Their implementation differs slightly from ours: In their case a disease can be passed to an agent who is immune to it - this is not possible in ours. In their case if an agent has already the disease, the transmitting agent selects a new disease, the other agent has not yet - this is not the case in our implementation and we think this is unreasonable to follow: it would require too much information and is also unrealistic.
We wrote regression tests which check for animation V-1 that after 100 ticks there are no more infected agents and for animation V-2 that after 1000 ticks there are still infected agents left and they dominate: there are more infected than recovered agents.

\section{Discussion}
In this appendix we showed how to use QuickCheck to formalise and check hypotheses about an \textit{exploratory} agent-based model, in which no ground truth exists. Due to ABS stochastic nature in general it became obvious that to get a good measure of a hypotheses validity we need to emulate failure using the \texttt{cover} function of QuickCheck. This allowed us to show that the hypotheses we have presented are sufficiently valid for the task at hand and can indeed be used for expressing and formalising emergent properties of the model and also as regression tests within a TDD cycle.

%What is particularly powerful is that one has complete control and insight over the changed state before and after e.g. a function was called on an agent: thus it is very easy to check if the function just tested has changed the agent-state itself or the environment: the new environment is returned after running the agent and can be checked for equality of the initial one - if the environments are not the same, one simply lets the test fail. This behaviour is very hard to emulate in OOP because one can not exclude side-effect at compile time, which means that some implicit data-change might slip away unnoticed. In FP we get this for free.

% what a waste of space
%\section*{Author Biographies}
%\textbf{\uppercase{JONATHAN THALER}} is a PhD. student.
%\textbf{\uppercase{PEER-OLAF SIEBERS}} is Jonathans Supervisor.

\end{document}