% interactapasample.tex
% v1.05 - August 2017
\documentclass[]{interact}

\usepackage{minted} % for HaskellCode
\usepackage{epstopdf}% To incorporate .eps illustrations using PDFLaTeX, etc.
\usepackage[caption=false]{subfig}% Support for small, `sub' figures and tables
%\usepackage[nolists,tablesfirst]{endfloat}% To `separate' figures and tables from text if required
%\usepackage[doublespacing]{setspace}% To produce a `double spaced' document if required
%\setlength\parindent{24pt}% To increase paragraph indentation when line spacing is doubled

% UNCOMMENT WHEN USING BIBITEM
%\usepackage[longnamesfirst,sort]{natbib}% Citation support using natbib.sty
%\bibpunct[, ]{(}{)}{;}{a}{,}{,}% Citation support using natbib.sty
%\renewcommand\bibfont{\fontsize{10}{12}\selectfont}% To set the list of references in 10 point font using natbib.sty
%
% define HaskellCode command for nice formatting of Haskell Code
\newminted[HaskellCode]{haskell}{fontsize=\footnotesize}

\usepackage[natbibapa,nodoi]{apacite}% Citation support using apacite.sty. Commands using natbib.sty MUST be deactivated first!
\setlength\bibhang{12pt}% To set the indentation in the list of references using apacite.sty. Commands using natbib.sty MUST be deactivated first!
\renewcommand\bibliographytypesize{\fontsize{10}{12}\selectfont}% To set the list of references in 10 point font using apacite.sty. Commands using natbib.sty MUST be deactivated first!

\theoremstyle{plain}% Theorem-like structures provided by amsthm.sty
\newtheorem{theorem}{Theorem}[section]
\newtheorem{lemma}[theorem]{Lemma}
\newtheorem{corollary}[theorem]{Corollary}
\newtheorem{proposition}[theorem]{Proposition}

\theoremstyle{definition}
\newtheorem{definition}[theorem]{Definition}
\newtheorem{example}[theorem]{Example}

\theoremstyle{remark}
\newtheorem{remark}{Remark}
\newtheorem{notation}{Notation}

\begin{document}

\articletype{ARTICLE TEMPLATE}% Specify the article type or omit as appropriate

% 1. Author details. All authors of a manuscript should include their full name and affiliation on the cover page of the manuscript. Where available, please also include ORCiDs and social media handles (Facebook, Twitter or LinkedIn). One author will need to be identified as the corresponding author, with their email address normally displayed in the article PDF (depending on the journal) and the online article. Authors’ affiliations are the affiliations where the research was conducted. If any of the named co-authors moves affiliation during the peer-review process, the new affiliation can be given as a footnote. Please note that no changes to affiliation can be made after your paper is accepted. Read more on authorship.
\title{Specification Testing of Agent-Based Simulation using Property-Based Testing.}

\author{
\name{Jonathan Thaler \textsuperscript{a}\thanks{CONTACT Jonathan Thaler. Email: jonathan.thaler@nottingham.ac.uk} and Peer-Olaf Siebers\textsuperscript{a}}
\affil{\textsuperscript{a}School Of Computer Science, University of Nottingham, 7301 Wollaton Rd, Nottingham, UK;}
}

\maketitle

% 2. Should contain an unstructured abstract of 200 words. 
% 3. You can opt to include a video abstract with your article. Find out how these can help your work reach a wider audience, and what to think about when filming.
\begin{abstract}
This paper explores how to use random property-based testing on a technical level to encode and test specifications of agent-based simulations (ABS). The claim is that opposed to unit testing, random property-based testing is a much more natural fit to test ABS due to both stochastic nature. The paper shows how to test full agent- and model-specifications, in the case of an agents behaviour, its transition probabilities and model invariants. The outcome are specifications expressed directly in code, which relate whole classes of random input to expected classes of output. During test execution, random test data is generated automatically, potentially covering the equivalent of thousands of unit tests, run within seconds. The expressiveness and power of property-based testing is not only limited to be part of a test-driven development process where it acts as specification, verification and regression test but can be integrated as a fundamental part of the model development process, supporting the hypothesis and discovery making process. By incorporating this powerful technique into the simulation development process the confidence in the correctness of an implementation increases dramatically, something of fundamental importance for ABS in general and for ABS supporting far-reaching policy decisions in particular.
\end{abstract}

% 4. Between 3 and 6 keywords. Read making your article more discoverable, including information on choosing a title and search engine optimization.
\begin{keywords}
Testing; Test Driven Development; Model Specification;
\end{keywords}

% 5 Funding details - not required for this paper

% 6. Disclosure statement. This is to acknowledge any financial interest or benefit that has arisen from the direct applications of your research. - not required in this paper

% 7. Supplemental online material. Supplemental material can be a video, dataset, fileset, sound file or anything which supports (and is pertinent to) your paper. We publish supplemental material online via Figshare. 
% TODO

% 8. Figures. Figures should be high quality (1200 dpi for line art, 600 dpi for grayscale and 300 dpi for colour, at the correct size). Figures should be supplied in one of our preferred file formats: EPS, PS, JPEG, GIF, or Microsoft Word (DOC or DOCX). For information relating to other file types, please consult our Submission of electronic artwork document.

% 9. Tables. Tables should present new information rather than duplicating what is in the text. Readers should be able to interpret the table without reference to the text. Please supply editable files.

% 10. Equations. If you are submitting your manuscript as a Word document, please ensure that equations are editable. 

% 11. Units. Please use SI units (non-italicized).

% WORD LIMITS: find . -name '*.tex' | xargs wc
% Please include a word count for your paper. A typical article for this journal should be no more than 6000 words; this limit does not include the abstract, endnotes, tables, figures, figure captions and legends, or references. 

%*******************************************************************************
%*********************************** First Chapter *****************************
%*******************************************************************************

\chapter{Introduction}  %Title of the First Chapter
I noticed that it is pretty hard to convince an agent-based economics specialist who is not a computer scientist about a pure functional approach. My conjecture is that the implementation technique and method does not matter much to them because they have very little knowledge about programming and are almost always self-taught - they don't know about software-engineering, nothing about proper software-design and architecture, nothing about software-maintenance, nothing about unit-testing,... In the end they just "hack" the simulation in whatever language they are able to: C++, Visual Basic, Java or toolboxes like Netlogo. For them it is all about to \textit{get things done somehow} and not to get things done the right way or in a beautiful way - the way and the method doesn't matter, its just a necessary evil which needs to be done. Thus if functional programming could make their lives easier, then they will definitely welcome it. But functional programming is, i think, harder to learn and harder to understand - so one needs to provide an abstraction through EDSL. So I REALLY need to come up with convincing arguments why to use pure functional approaches in ACE THEY can understand, otherwise I will be lost and not heard (not published,...). \\

What ACE economists care for:

\begin{itemize}
\item Very: Qualitative modelling with quantitative results
\item Yes: Easy reproducibility
\item Likely: Reasoning about convergence?
\item Likely: EDSL
\end{itemize}

My contributions are: pure functional framework, functional agent-model for market-simulations, EDSL for market-simulations, qualitative / implicit modelling with quanitative results, reasoning in my framework about convergence \\

IDEA: could I develop non-causal modelling (models are expressed in terms of non-directed equations, modelled in signal-relations) to allow for qualitative modelling for the agent-based economists? See hybrid modelling paper of Yampa. \textbf{THIS WOULD BE A HUGE NOVEL CONTRIBUTION TO ACE ESPECIALLY WHEN COMBINED WITH AN EDSL AND PROVIDING FULL REFERENTIAL TRANSPARENCY TO KEEP THE ABILITY TO REASON ABOUT CONVERGENCE}. This should be covered in the "EDSL"-paper.

TODO: maybe i should really focus only on market models? otherwise too much? \\

central novelty of my PhD: model specification = runnable code. possible through EDSL. but only in specific subfield of ACE: market-models. need a functional description of the model, then translate it to model specification in EDSL and then run it to see dynamics. But: model specification moves closer to functional programming languages. \\

another novelty approach: model specification through qualitative instead of quantiative approaches. is this possible? \\

WHY FUNCTIONAL? "because its the ultimate approach to scientific computing": fewer bugs due to mutable state (why? is thos shown obkectively by someone?), shorter (again as above, productivity), more expressive and closer to math, EDSL, EDSL=model=simulation, better parallelising due to referental transparency, reasoning \\

scientific results need to be reproduced, especially when they have high impact. a more formal approach of specifying the model and the simulation (model=simulation) could lead to easier sharing and easier reporduction without ambigouites \\

pure functional agent-model \& theory, EDSL framework in Haskell for ACE

\begin{enumerate}
\item Which kind of problem do we have?
\item What aim is there? Solving the problem? 
\item How the aim is achieved by enumerating VERY CLEAR objectives.
\item What the impact one expects (hypothesis) and what it is (after results).
\end{enumerate}

Note: It is not in the interest of the researcher to develop new economic theories but to research the use of functional methods (programming and specification) in agent-based computational economics (ACE).

NOTE: Get the reader’s attention early in the introduction: motivation, significance, originality and novelty.

\section{Methods}
Methods need to be selected to implement the simulations. Special emphasis will be put on functional ones which will then be compared to established methods in the field of ABM/S and ACE. \\

Claim: non-programming environments are considered to be not powerful enough to capture the complexity of ACE implementations thus a programming approach to ACE will be always required.

\section{Scenarios}
To apply and test functional methods in ACE, four scenarios of ACE are selected and then the methods applied and compared with each other to see how each of them perform in comparison. The 4 selected scenarios represent a selection of the challenges posed in ACE: from very abstract ones to very operational ones.

\section{Comparison}
Each of the selected scenarios is then implemented using the selected methods where each solution is then compared against the following criteria: 

\begin{enumerate}
\item suitability for scientific computation
\item robustness
\item error-sources
\item testability
\item stability
\item extendability
\item size of code
\item maintainability
\item time taken for development
\item verification \& correctness
\item replications \& parallelism
\item EDSL
\end{enumerate}

This will then allow to compare the different methods against each other and to show under which circumstances functional methods shine and when they should not be used.

\section{Agent-Based Modelling and Simulation (ABM/S)}
ABM/S is a method of modelling and simulating a system where the global behaviour may be unknown but the behaviour and interactions of the parts making up the system is of knowledge (Wooldrige, M. (2009). An Introduction to MultiAgent Systems. John Wiley & Sons). Those parts, called agents, are modelled and simulated out of which then the aggregate global behaviour of the whole system emerges. Thus the central aspect of ABM/S is the concept of an Agent which can be understood as a metaphor for a pro-active unit, able to spawn new Agents, and interacting with other Agents in a network of neighbours by exchange of messages. The implementation of Agents can vary and strongly depends on the programming language and the kind of domain the simulation and model is situated in.

\section{Agent-Based Economics (ACE)}
According to Leigh Tesfatsion (Tesfatsion, L. (2006). Agent-based computational economics: A constructive approach to economic theory. In Tesfatsion, L. and Judd, K. L., editors, Handbook of Computational Economics, volume 2, chapter 16, pages 831–880. Elsevier, 1 edition.), one of the leading figures, ACE is "[...] computational modelling of economic processes (including whole economies) as open-ended dynamic systems of interacting agents." - thus lending perfectly to the use of ABM/S as already the name suggests. Whereas classical economic models fall short by only looking at the average, pure rational, individual interacting in anonymous markets, the ACE approach looks at heterogeneous, non-rational individuals interacting with each other in networks (Kirman, A. (2010). Complex Economics: Individual and Collective Rationality. Routledge, London ; New York, NY.). Thus ACE can be understood as a combination of computer-science, cognitive/social science and evolutionary economics.

\section{Functional programming}
TODO: read \cite{Backus1978}

The state-of-the-art approach to implementing Agents are object-oriented methods and programming as the metaphor of an Agent as presented above lends itself very naturally to object-orientation (OO). The author of this thesis claims that OO in the hands of inexperienced or ignorant programmers is dangerous, leading to bugs and hardly maintainable and extensible code. The reason for this is that OO provides very powerful techniques of organising and structuring programs through Classes, Type Hierarchies and Objects, which, when misused, lead to the above mentioned problems. Also major problems, which experts face as well as beginners are 1. state is highly scattered across the program which disguises the flow of data in complex simulations and 2. objects don’t compose as well as functions. The reason for this is that objects always carry around some internal state which makes it obviously much more complicated as complex dependencies can be introduced according to the internal state.
All this is tackled by (pure) functional programming which abandons the concept of global state, Objects and Classes and makes data-flow explicit. This then allows to reason about correctness, termination and other properties of the program e.g. if a given function exhibits side-effects or not. Other benefits are fewer lines of code, easier maintainability and ultimately fewer bugs thus making functional programming the ideal choice for scientific computing and simulation and thus also for ACE. A very powerful feature of functional programming is Lazy evaluation. It allows to describe infinite data-structures and functions producing an infinite stream of output but which are only computed as currently needed. Thus the decision of how many is decoupled from how to (Hughes, J. (1989). Why functional programming matters. Comput. J., 32(2):98–107.).
The most powerful aspect using pure functional programming however is that it allows the design of embedded domain specific languages (EDSL). In this case one develops and programs primitives e.g. types and functions in a host language (embed) in a way that they can be combined. The combination of these primitives then looks like a language specific to a given domain, in the case of this thesis ACE. The ease of development of EDSLs in pure functional programming is also a proof of the superior extensibility and composability of pure functional languages over OO (Henderson P. (1982). Functional Geometry. Proceedings of the 1982 ACM Symposium on LISP and Functional Programming.).
One of the most compelling example to utilize pure functional programming is the reporting of Hudak (Hudak P., Jones M. (1994). Haskell vs. Ada vs. C++ vs. Awk vs. ... An Experiment in Software Prototyping Productivity. Department of Computer Science, Yale University.)  where in a prototyping contest of DARPA the Haskell prototype was by far the shortest with 85 lines of code. Also the Jury mistook the code as specification because the prototype did actually implement a small EDSL which is a perfect proof how close EDSL can get to and look like a specification.

Functional languages can best be characterized by their way computation works: instead of \textit{how} something is computed, \textit{what} is computed is described. Thus functional programming follows a declarative instead of an imperative style of programming. The key points are:
\begin{itemize}
\item No assignment statements - variables values can never change once given a value.
\item Function calls have no side-effect and will only compute the results - this makes order of execution irrelevant, as due to the lack of side-effects the logical point in \textit{time} when the function is calculated within the program-execution does not matter.
\item higher-order functions
\item lazy evaluation
\item Looping is achieved using recursion, mostly through the use of the general fold or the more specific map.
\item Pattern-matching
\end{itemize}

This alone does not really explain the \textit{real} advantages of functional programming and one must look for better motivations using functional programming languages. One motivation is given in \cite{Hughes1989} which is a great paper explaining to non-functional programmers what the significance of functional programming is and helping functional programmers putting functional languages to maximum use by showing the real power and advantages of functional languages. The main conclusion is that \textit{modularity}, which is the key to successful programming, can be achieved best using higher-order functions and lazy evaluation provided in functional languages like Haskell. \cite{Hughes1989} argues that the ability to divide problems into sub-problems depends on the ability to glue the sub-problems together which depends strongly on the programming-language and \cite{Hughes1989} argues that in this ability functional languages are superior to structured programming.

TODO: comparison of functional and object-oriented programming. My points are:
\begin{itemize}
\item The way state can be changed and treated - distributed over multiple objects - is often very difficult to understand.
\item Inheritance is a dangerous thing if not used with care because inheritance introduces very strong dependencies which cannot be changed during runtime anymore.
\item Objects don't compose very well: \url{http://zeroturnaround.com/rebellabs/why-the-debate-on-object-oriented-vs-functional-programming-is-all-about-composition/}
\item (Nearly) impossible to reason about programs
\end{itemize}

In conclusion the upsides of functional programming as opposed to OO are:
\begin{itemize}
\item Much more explicit flow of data \& control
\item Much better compose-able
\item Much better parallelism
\end{itemize}

\section{Related Research}
Tim Sweeney, CTO of Epic Games gave an invited talk about how "future programming languages could help us write better code" by "supplying stronger typing, reduce run-time failures;  and the need for pervasive concurrency support, both implicit and explicit, to effectively exploit the several forms of parallelism present in games and graphics." \cite{Sweeney2006}. Although the fields of games and agent-based simulations seem to be very different in the end, they have also very important similarities: both are simulations which perform numerical computations and update objects - in games they are called "game-objects" and in abm they are called agents but they are in fact the same thing - in a loop either concurrently or sequential. His key-points were:

\begin{itemize}
\item Dependent types as the remedy of most of the run-time failures.
\item Parallelism for numerical computation: these are pure functional algorithms, operate locally on mutable state. Haskell ST, STRef solution enables encapsulating local heaps and mutability within referentially transparent code.
\item Updating game-objects (agents) concurrently using STM: update all objects concurrently in arbitrary order, with each update wrapped in atomic block - depends on collisions if performance goes up.
\end{itemize}

\section{Property-based testing}
\label{sec:proptesting}
Property-based testing allows to formulate \textit{functional specifications} in code which then a property-based testing library tries to falsify by \textit{automatically} generating test data, covering as much cases as possible. When a case is found for which the property fails, the library then reduces the test data to its simplest form for which the test still fails, for example shrinking a list to a smaller size. It is clear to see that this kind of testing is especially suited to ABS, because we can formulate specifications, meaning we describe \textit{what} to test instead of \textit{how} to test. Also the deductive nature of falsification in property-based testing suits very well the constructive and exploratory nature of ABS. Further, the automatic test generation can make testing of large scenarios in ABS feasible because it does not require the programmer to specify all test cases by hand, as is required in traditional unit tests.

Property-based testing was introduced in \cite{claessen_quickcheck_2000,claessen_testing_2002} where the authors present the QuickCheck library in Haskell, which tries to falsify the specifications by \textit{randomly} sampling the test space. %We argue, that the stochastic sampling nature of this approach is particularly well suited to ABS, because it is itself almost always driven by stochastic events and randomness in the agents behaviour, thus this correlation should make it straightforward to map ABS to property-testing.
%The main challenge when using QuickCheck, as will be shown later, is to write \textit{custom} test data generators for agents and the environment which cover the space sufficiently enough to not miss out on important test cases.
According to the authors of QuickCheck \textit{"The major limitation is that there is no measurement of test coverage."} \cite{claessen_quickcheck_2000}. Although QuickCheck provides help to report the distribution of test cases it is not able to measure the coverage of tests in general. This could lead to the case that test cases which would fail are never tested because of the stochastic nature of QuickCheck. Fortunately, the library provides mechanisms for the developer to measure coverage in specific test cases where the data and its expected distribution is known to the developer. This is a powerful tool for testing randomness in ABS as will be shown in the next chapters.

\medskip

As a remedy for the potential coverage problems of QuickCheck, there exists also a deterministic property-testing library called SmallCheck \cite{runciman_smallcheck_2008}, which instead of randomly sampling the test space, enumerates test cases exhaustively up to some depth. It is based on two observations, derived from model-checking, that (1) \textit{"If a program fails to meet its specification in some cases, it almost always fails in some simple case"} and (2) \textit{"If a program does not fail in any simple case, it hardly ever fails in any case} \cite{runciman_smallcheck_2008}. This non-stochastic approach to property-based testing might be a complementary addition in some cases where the tests are of non-stochastic nature with a search space too large to test manually by unit testing but small enough to enumerate exhaustively. The main difficulty and weakness of using SmallCheck is to reduce the dimensionality of the test case depth search to prevent combinatorial explosion, which would lead to exponential number of cases. Thus one can see QuickCheck and SmallCheck as complementary instead of in opposition to each other.

\subsection{A brief overview of QuickCheck}
To give a good understanding of how property-based testing works with \\ QuickCheck, we give a few examples of property tests on lists, which are directly expressed as functions in Haskell. Such a function has to return a \texttt{Bool} which indicates \texttt{True} in case the test succeeds or \texttt{False} if not and can take input arguments which data is automatically generated by QuickCheck.

\begin{HaskellCode}
-- append operator (++) is associative
append_associative :: [Int] -> [Int] -> [Int] -> Bool
append_associative xs ys zs = (xs ++ ys) ++ zs == xs ++ (ys ++ zs)

-- The reverse of a reversed list is the original list
reverse_reverse :: [Int] -> Bool
reverse_reverse xs = reverse (reverse xs) == xs

-- reverse is distributive over append (++)
-- This test fails for explanatory reasons, for a correct 
-- property xs and ys need to be swapped on the right-hand side!
reverse_distributive :: [Int] -> [Int] -> Bool
reverse_distributive xs ys = reverse (xs ++ ys) == reverse xs ++ reverse ys

-- running the tests
main :: IO ()
main = do
  quickCheck append_associative
  quickCheck reverse_reverse
  quickCheck reverse_distributive
\end{HaskellCode}

When we run the tests using \textit{main}, we get the following output:

\begin{verbatim}
+++ OK, passed 100 tests.
+++ OK, passed 100 tests.
*** Failed! Falsifiable (after 5 tests and 6 shrinks):    
[0]
[1]
\end{verbatim}

We see that QuickCheck generates 100 test cases for each property test and it does this by generating random data for the input arguments. We have not specified any data for our input arguments because QuickCheck is able to provide a suitable data generator through type inference. For lists and all the existing Haskell types there exist custom data generators already. We have to use a monomorphic list, in our case \texttt{Int}, and cannot use polymorphic lists because QuickCheck would not know how to generate data for a polymorphic type. Still, by appealing to genericity and polymorphism, we get the guarantee that the test case is the same for all types of a lists.

QuickCheck generates 100 test cases by default and requires all of them to pass. If there is a test case which fails, the overall property test fails and QuickCheck shrinks the input to a minimal size, which still fails and reports it as a counter example. This is the case in the last property test \texttt{reverse\_distributive} which is wrong as \textit{xs} and \textit{ys} need to be swapped on the right-hand side. In this run, QuickCheck found a counter example to the property after 5 tests and applied 6 shrinks to find the minimal failing example of \texttt{xs = [0]} and \texttt{ys = [1]}. If we swap \texttt{xs} and \texttt{ys}, the property test passes 100 test cases just like the other two did. It is possible to configure QuickCheck to generate more or less random test cases, which can be used to increase the coverage if the sampling space is quite large - this will become useful later.

\subsubsection{Generators}
QuickCheck comes with a lot of data generators for existing types like \texttt{String, Int, Double, []}, but in case one wants to randomize custom data types one has to write custom data generators. There are two ways to do this. Either fix them at compile time by writing an \texttt{Arbitrary} instance or write a run-time generator running in the \texttt{Gen} Monad. The advantage of having an \texttt{Arbitrary} instance is that the custom data type can then be used as random argument to a function as in the examples above.

Lets implement a custom data generator for the \texttt{SIRState} for both cases. We start with the run-time option, running in the \texttt{Gen} Monad:

\begin{HaskellCode}
genSIRState :: Gen SIRState
genSIRState = elements [Susceptible, Infected, Recovered]
\end{HaskellCode}

This implementation makes use of the \texttt{elements :: [a] $\rightarrow$ Gen a} functions, which picks a random element from a non-empty list with uniform probability. If a skewed distribution is needed, one can use the \texttt{frequency :: [(Int, Gen a)] $\rightarrow$ Gen a} function, where a frequency can be specified for each element. For example generating on average 80\% \texttt{Susceptible}, 15\% \texttt{Infected} and 5\% \texttt{Recovered} can be achieved using this function:

\begin{HaskellCode}
genSIRState :: Gen SIRState
genSIRState = frequency [(80, Susceptible), (15, Infected), (5, Recovered)]
\end{HaskellCode}

Implementing an \texttt{Arbitrary} instance is straightforward, one only needs to implement the \texttt{arbitrary :: Gen a} method:

\begin{HaskellCode}
instance Arbitrary SIRState where
  arbitrary = genSIRState
\end{HaskellCode}

When we have a random \texttt{Double} as input to a function but want to restrict its random range to (0,1) because it reflects a probability, we can do this easily with \texttt{newtype} and implementing an \texttt{Arbitrary} instance. The same can be done for limiting the simulation duration to a lower range than the full \texttt{Double} range.

\begin{HaskellCode}
newtype Probability = P Double
newtype TimeRange   = T Double

instance Arbitrary Probability where
  arbitrary = P <$> choose (0, 1)
  
instance Arbitrary TimeRange where
  arbitrary = T <$> choose (0, 50)
\end{HaskellCode}

The simulations we run all rely on a random-number generator, thus we need a randomly initialised random-number generator each time we run a simulation. This can be easily achieved by drawing a seed from the full \texttt{Int} range and creating an \texttt{StdGen} from it:

\begin{HaskellCode}
genStdGen :: Gen StdGen
-- min/maxBound are defined in the Haskell Prelude and
-- define the smallest and largest value of a Bounded type 
genStdGen = mkStdGen <$> choose (minBound, maxBound)

instance Arbitrary StdGen where
  arbitrary = genStdGen
\end{HaskellCode}
%$

This generator then can be used to write another custom data generator which generates simulation runs. Here we give an example for the time-driven SIR:

\begin{HaskellCode}
genTimeSIR :: [SIRState]  -- ^ Population
           -> Double      -- ^ Contact rate (beta)
           -> Double      -- ^ Infectivity (gamma)
           -> Double      -- ^ Illness duration (delta)
           -> Double      -- ^ Time Delta
           -> Double      -- ^ Time Limit
           -> Gen [(Double, (Int, Int, Int))]
genTimeSIR as beta gamma delta dt tMax 
  = runTimeSIR as beta gamma delta dt tMax <$> genStdGen
\end{HaskellCode}
%$

\subsubsection{Distributions}
As already mentioned, QuickCheck provides functions to measure the coverage of test cases. This can be done using the 
\texttt{label :: Testable prop $\Rightarrow$ String $\rightarrow$ prop $\rightarrow$ Property} function. It takes a \texttt{String} as first argument and a testable property and constructs a \texttt{Property}. QuickCheck collects all generated labels, counts their occurrences and reports their distribution. For example it could be used to get a rough idea about the length of the random lists created in the \texttt{reverse\_reverse} property shown above:

\begin{HaskellCode}
reverse_reverse_label :: [Int] -> Property
reverse_reverse_label xs  
  = label ("length of random-list is " ++ show (length xs)) 
          (reverse (reverse xs) == xs)
\end{HaskellCode}
%$
When running the test, we see the following output:

\begin{verbatim}
+++ OK, passed 100 tests:
 5% length of random-list is 27
 5% length of random-list is 0
 4% length of random-list is 19
 ...
\end{verbatim}

\subsubsection{Coverage}
The most powerful functions to work with test-case distributions though are \texttt{cover} and \texttt{checkCoverage}. The function \texttt{cover :: Testable prop $\Rightarrow$ Double $\rightarrow$ Bool $\rightarrow$ String $\rightarrow$ prop $\rightarrow$ Property} allows to explicitly specify that a given percentage of successful test cases belong to a given class. The first argument is the expected percentage; the second argument is a \texttt{Bool} indicating whether the current test case belongs to the class or not; the third argument is a label for the coverage; the fourth argument is the property which needs to hold for the test case to succeed. 

Lets look at an example where we use \texttt{cover} to express that we expect 15\% of all test cases to have a random list with at least 50 elements.

\begin{HaskellCode}
reverse_reverse_cover :: [Int] -> Property
reverse_reverse_cover xs  
  = cover 15 (length xs >= 50) "Length of random list at least 50"
             (reverse (reverse xs) == xs)
\end{HaskellCode}

When repeatedly running the test, we see the following output:

\begin{verbatim}
+++ OK, passed 100 tests (10% length of random list at least 50).
Only 10% Length of random-list at least 50, but expected 15%.
+++ OK, passed 100 tests (21% length of random list at least 50).
\end{verbatim}

As can be seen, QuickCheck runs the default 100 test cases and prints a warning if the expected coverage is not reached. This is a useful feature but it is up to us to decide whether 100 test cases are suitable and whether we can really claim that the given coverage will be reached or not. Fortunately, QuickCheck provides the powerful function \texttt{checkCoverage :: Testable prop $\Rightarrow$ prop $\rightarrow$ Property} which does this for us. When \texttt{checkCoverage} is used, QuickCheck will run an increasing number of test cases until it can decide whether the percentage in \texttt{cover} was reached or cannot be reached at all. The way QuickCheck does it, is by using sequential statistical hypothesis testing \cite{wald_sequential_1992}, thus if QuickCheck comes to the conclusion that the given percentage can or cannot be reached, it is based on a robust statistical test giving strong confidence in the result.

When we run the example from above but now with \texttt{checkCoverage} we get the following output:

\begin{verbatim}
+++ OK, passed 12800 tests 
    (15.445% length of random-list at least 50).
\end{verbatim}

We see that after QuickCheck has run 12,800 tests it came to the statistically robust conclusion that indeed at least 15\% of the test cases have a random list with at least 50 elements. 

\subsubsection{Emulating failure}
As already mentioned, \textit{all} test cases have to pass for the whole property test to succeed. If just a single test case fails, the whole property test fails. This requirement is sometimes too strong, especially when we are dealing with stochastic systems like ABS.

The function \texttt{cover} can be used to emulate failure of test cases and get a measure of failure. Instead of computing the \texttt{True/False} property in the last \texttt{prop} argument, we set the last argument always to \texttt{True} and compute the \texttt{True/False} property in the second \texttt{Bool} argument, indicating whether the test case belongs to the class of passed tests or not. This has the effect that \textit{all} test cases are successful but that we get a distribution of failed and successful ones. In combination with \texttt{checkCoverage}, this is a particularly powerful pattern for testing ABS, which allows us to test hypotheses and statistical tests on distributions as will be shown in the following chapters.

\section{An event-driven agent-based SIR model}
\label{sec:sirmodel}
The explanatory SIR model is a very well studied and understood compartment model from epidemiology \cite{kermack_contribution_1927}, which allows to simulate the dynamics of an infectious disease like influenza, tuberculosis, chicken pox, rubella and measles spreading through a population. The reason for choosing this model is its simplicity as it is easy to understand fully but complex enough to develop basic concepts of pure functional ABS, which are then extended and deepened in the much more complex Sugarscape model of the next section.

In this model, people in a population of size $N$ can be in either one of the three states \textit{Susceptible}, \textit{Infected} or \textit{Recovered} at a particular time, where it is assumed that initially there is at least one infected person in the population. People interact \textit{on average} with a given rate of $\beta$ other people per time unit and become infected with a given probability $\gamma$ when interacting with an infected person. When infected, a person recovers \textit{on average} after $\delta$ time units and is then immune to further infections. An interaction between infected persons does not lead to reinfection, thus these interactions are ignored in this model. This definition gives rise to three compartments with the transitions seen in Figure \ref{fig:sir_transitions}.

\begin{figure}
	\centering
	\includegraphics[width=.7\textwidth, angle=0]{./fig/SIR_transitions.png}
	\caption{States and transitions in the SIR compartment model.}
	\label{fig:sir_transitions}
\end{figure}

In this paper we want to implement an agent-based simulation of this model, where we follow TODO: cite macal, translating the informal specification into an an event-driven agent-based approach. 

\begin{figure}
	\centering
	\includegraphics[width=0.7\textwidth, angle=0]{./fig/sir_eventdriven.png}
	\caption{Dynamics of the SIR compartment model using an event-driven agent-based approach. Population Size $N$ = 1,000, contact rate $\beta =  \frac{1}{5}$, infection probability $\gamma = 0.05$, illness duration $\delta = 15$ with initially 1 infected agent.}
	\label{fig:sir_sd_dynamics}
\end{figure}

We start by giving the full \textit{specification} of the susceptible, infected and recovered agent by stating the input-to-output event relations. The susceptible agent is specified as follows:

TODO is there some diagram form (BPNL or other process language, e.g. UML), with which we can express the SIR agents event behaviour? would be more concise than only describing it in word

\begin{enumerate}
	\item \texttt{MakeContact} - if the agent receives this event it will output $\beta$ \texttt{Contact ai Susceptible} events, where \texttt{ai} is the agents own id. The events have to be scheduled immediately without delay, thus having the current time as scheduling timestamp. The receivers of the events are uniformly randomly chosen from the agent population. The agent doesn't change its state, stays \texttt{Susceptible} and does not schedule any other events than the ones mentioned.
	
	\item \texttt{Contact \_ Infected} - if the agent receives this event there is a chance of uniform probability $\gamma$ (infectivity) that the agent becomes \texttt{Infected}. If this happens, the agent will schedule a \texttt{Recover} event to itself into the future, where the time is drawn randomly from the exponential distribution with $\lambda = \delta$ (illness duration). If the agent does not become infected, it will not change its state, stays \texttt{Susceptible} and does not schedule any events.
	
	\item \texttt{Contact \_ \_} or \texttt{Recover} - if the agent receives any of these other events it will not change its state, stays \texttt{Susceptible} and does not schedule any events.
\end{enumerate}

This specification implicitly covers that a susceptible agent can never transition from a \texttt{Susceptible} to a \texttt{Recovered} state within a single event as it can only make the transition to \texttt{Infected} or stay \texttt{Susceptible}. The infected agent is specified as follows:

\begin{enumerate}
	\item \texttt{Recover} - if the agent receives this, it will not schedule any events and make the transition to the \texttt{Recovered} state.
	
	\item \texttt{Contact sender Susceptible} - if the agent receives this, it will reply immediately with \texttt{Contact ai Infected} to \textit{sender}, where \texttt{ai} is the infected agents' id and the scheduling timestamp is the current time. It will not schedule any events and stays \texttt{Infected}.
	
	\item In case of any other event, the agent will not schedule any events and stays \texttt{Infected}.
\end{enumerate}

This specification implicitly covers that an infected agent never goes back to the \texttt{Susceptible} state as it can only make the transition to \texttt{Recovered} or stay \texttt{Infected}. From the specification of the susceptible agent it becomes clear that a susceptible agent who became infected, will always recover as the transition to \texttt{Infected} includes the scheduling of \texttt{Recovered} to itself. 

\medskip

The \textit{recovered} agent specification is very simple. It stays \texttt{Recovered} forever and does not schedule any events.

\section{Testing agent specifications}
\label{sec:method}
%TODO: put emphasis on statistical robust testing using cover and checkCoverage,
%TODO: introduce concepts along writing the main body. focus on event-driven sir.
%TODO: it would be great if i can show how property-based testing found a bug in an implementation

We start by encoding the invariants of the susceptible agent directly into Haskell, implementing a function which takes all necessary parameters and returns a \texttt{Bool} indicating whether the invariants hold or not. The encoding is straightforward when using pattern matching and it nearly reads like a formal specification due to the declarative nature of functional programming.

\begin{HaskellCode}
susceptibleProps :: SIREvent              -- ^ Random event sent to agent
                 -> SIRState              -- ^ Output state of the agent
                 -> [QueueItem SIREvent]  -- ^ Events the agent scheduled
                 -> AgentId               -- ^ Agent id of the agent
                 -> Bool
-- received Recover => stay Susceptible, no event scheduled
susceptibleProps Recover Susceptible es _ = null es
-- received MakeContact => stay Susceptible, check events
susceptibleProps MakeContact Susceptible es ai
  = checkMakeContactInvariants ai es cor 
-- received Contact _ Recovered => stay Susceptible, no event scheduled
susceptibleProps (Contact _ Recovered) Susceptible es _ = null es
-- received Contact _ Susceptible => stay Susceptible, no event scheduled
susceptibleProps (Contact _ Susceptible) Susceptible es _  = null es
-- received Contact _ Infected, didn't get Infected, no event scheduled
susceptibleProps (Contact _ Infected) Susceptible es _ = null es
-- received Contact _ Infected AND got infected, check events
susceptibleProps (Contact _ Infected) Infected es ai
  = checkInfectedInvariants ai es
-- all other cases are invalid and result in a failed test case
susceptibleProps _ _ _ _ = False
\end{HaskellCode}

Next, we give the implementation for the \texttt{checkMakeContactInvariants} function. We omit a detailed implementation of \texttt{checkInfectedInvariants} as it works in a similar way and its details do not add anything conceptually new. The function \texttt{checkMakeContactInvariants} encodes the invariants which have to hold when the susceptible agent receives a \texttt{MakeContact} event:

\begin{HaskellCode}
checkInfectedInvariants :: AgentId              -- ^ Agent id of the agent 
                        -> [QueueItem SIREvent] -- ^ Events the agent scheduled
                        -> Bool
checkInfectedInvariants sender 
  -- expect exactly one Recovery event
  [QueueItem receiver (Event Recover) t'] 
  -- receiver is sender (self) and scheduled into the future
  = sender == receiver && t' >= t 
-- all other cases are invalid
checkInfectedInvariants _ _ = False
\end{HaskellCode}

%The \texttt{checkMakeContactInvariants} is a bit more complex:
%
%\begin{HaskellCode}
%checkMakeContactInvariants :: AgentId              -- ^ Agent id of the agent 
%                           -> [QueueItem SIREvent] -- ^ Events the agent scheduled
%                           -> Int                  -- ^ Contact Rate
%                           -> Bool
%checkMakeContactInvariants sender es contactRate
%    -- make sure there has to be exactly one MakeContact event and
%    -- exactly contactRate Contact events
%    = invOK && hasMakeCont && numCont == contactRate
%  where
%    (invOK, hasMakeCont, numCont) 
%      = foldr checkMakeContactInvariantsAux (True, False, 0) es
%
%    checkMakeContactInvariantsAux :: QueueItem SIREvent 
%                                  -> (Bool, Bool, Int)
%                                  -> (Bool, Bool, Int)
%    checkMakeContactInvariantsAux 
%        (QueueItem (Contact sender' Susceptible) receiver t') (b, mkb, n)
%      = (b && sender == sender'   -- sender in Contact must be self
%           && receiver `elem` ais -- receiver of Contact must be in agent ids
%           && t == t', mkb, n+1)  -- Contact event is scheduled immediately
%    checkMakeContactInvariantsAux 
%        (QueueItem MakeContact receiver t') (b, mkb, n) 
%      = (b && receiver == sender  -- receiver of MakeContact is agent itself
%           && t' == t + 1         -- MakeContact scheduled 1 timeunit into future
%           &&  not mkb, True, n)  -- there can only be one MakeContact event
%    checkMakeContactInvariantsAux _ (_, _, _) 
%      = (False, False, 0)         -- other patterns are invalid
%\end{HaskellCode}

\subsection{Writing a property test}
What is left is to actually write a property test using QuickCheck. We are making heavy use of random parameters to express that the properties have to hold invariant of the model parameters. We make use of additional data generator modifiers: \texttt{Positive} ensures that the value generated is positive; \texttt{NonEmptyList} ensures that the randomly generated list is not empty.

QuickCheck comes with a lot of data generators for existing types like \texttt{String, Int, Double, []}, but in case one wants to randomize custom data types one has to write custom data generators. There are two ways to do this. Either fix them at compile time by writing an \texttt{Arbitrary} instance or write a run-time generator running in the \texttt{Gen} context. The advantage of having an \texttt{Arbitrary} instance is that the custom data type can then be used as random argument to a function..

This implementation makes use of the \texttt{elements :: [a] $\rightarrow$ Gen a} functions, which picks a random element from a non-empty list with uniform probability. If a skewed distribution is needed, one can use the \texttt{frequency :: [(Int, Gen a)] $\rightarrow$ Gen a} function, where a frequency can be specified for each element.

\begin{HaskellCode}
genEventFreq :: Int
             -> Int
             -> Int
             -> (Int, Int, Int)
             -> [AgentId]
             -> Gen SIREvent
genEventFreq mcf _ rcf _ []  
  = frequency [ (mcf, return MakeContact), (rcf, return Recover)]
genEventFreq mcf cof rcf (s,i,r) ais
  = frequency [ (mcf, return MakeContact)
              , (cof, do
                  ss <- frequency [ (s, return Susceptible)
                                  , (i, return Infected)
                                  , (r, return Recovered)]
                  ai <- elements ais
                  return (Contact ai ss))
              , (rcf, return Recover)]
         
genEvent :: [AgentId] -> Gen SIREvent
genEvent = genEventFreq 1 1 1 (1,1,1) 
\end{HaskellCode}

When we have a random \texttt{Double} as input to a function but want to restrict its random range to (0,1) because it reflects a probability, we can do this easily with \texttt{newtype} and implementing an \texttt{Arbitrary} instance. The same can be done for limiting the simulation duration to a lower range than the full \texttt{Double} range. Implementing an \texttt{Arbitrary} instance is straightforward, one only needs to implement the \texttt{arbitrary :: Gen a} method:

\begin{HaskellCode}
newtype Probability = P Double
newtype TimeRange   = T Double

instance Arbitrary Probability where
  arbitrary = P <$> choose (0, 1)
  
instance Arbitrary TimeRange where
  arbitrary = T <$> choose (0, 50)
\end{HaskellCode}

We are now equipped with all functionality to implement the property test.

\begin{HaskellCode}
prop_susceptible_invariants :: Positive Int         -- ^ Contact rate (beta)
                            -> Probability          -- ^ Infectivity (gamma)
                            -> Positive Double      -- ^ Illness duration (delta)
                            -> Positive Double      -- ^ Current simulation time
                            -> NonEmptyList AgentId -- ^ population agent ids
                            -> Gen Property
prop_susceptible_invariants 
  (Positive beta) (P gamma) (Positive delta) (Positive t) (NonEmpty ais) = do
  -- generate random event, requires the population agent ids
  evt <- genEvent ais
  -- run susceptible random agent with given parameters
  (ai, ao, es) <- genRunSusceptibleAgent beta gamma delta t ais evt
  -- check properties
  return (label (labelTestCase ao) (property (susceptibleProps evt ao es ai)))
  where
    labelTestCase :: SIRState -> String
    labelTestCase Infected    = "Susceptible -> Infected"
    labelTestCase Susceptible = "Susceptible"
    labelTestCase Recovered   = "INVALID"
\end{HaskellCode}

Due to the large random sampling space with 5 parameters, we increase the number of test cases to generate to 100,000. We also label the test cases to generate a distribution of the transitions. The case where the agents output state is \texttt{Recovered} is marked as "INVALID" as it must never occur, otherwise the test will fail, due to the invariants encoded above.

\begin{verbatim}
+++ OK, passed 100000 tests (6.77s):
94.522% Susceptible
 5.478% Susceptible -> Infected
\end{verbatim}

All 100,000 test cases go through within 6.7 seconds. The distribution of the transitions shows that we indeed cover both cases a susceptible agent can react within one event. It either stays susceptible or makes the transition to infection. The fact that there is no transition to recovered shows that the implementation is correct - for a transition to recovered we would need to send an additional, second event to the agent.

Encoding of the invariants and writing property tests for the infected agents follows the same idea and is not repeated here. Next, we show how to test transition probabilities using the powerful statistical hypothesis testing feature of QuickCheck.

\subsection{Encoding transition probabilities}
In the specifications from section \ref{sec:sirmodel} there are probabilistic state transitions, for example an infected agent \textit{will} recover after a given time, which is randomly distributed with the exponential distribution. The susceptible agent \textit{might} become infected, depending on the events it receives and the infectivity ($\gamma$) parameter. We look now into how we can encode these probabilistic properties using the powerful \texttt{cover} and \texttt{checkCoverage} feature of QuickCheck.

The function \texttt{cover :: Testable prop $\Rightarrow$ Double $\rightarrow$ Bool $\rightarrow$ String $\rightarrow$ prop $\rightarrow$ Property} allows to explicitly specify that a given percentage of successful test cases belong to a given class. The first argument is the expected percentage; the second argument is a \texttt{Bool} indicating whether the current test case belongs to the class or not; the third argument is a label for the coverage; the fourth argument is the property which needs to hold for the test case to succeed. 

QuickCheck provides the powerful function \texttt{checkCoverage :: Testable prop $\Rightarrow$ prop $\rightarrow$ Property} which does this for us. When \texttt{checkCoverage} is used, QuickCheck will run an increasing number of test cases until it can decide whether the percentage in \texttt{cover} was reached or cannot be reached at all. The way QuickCheck does it, is by using sequential statistical hypothesis testing \cite{wald_sequential_1992}, thus if QuickCheck comes to the conclusion that the given percentage can or cannot be reached, it is based on a robust statistical test giving strong confidence in the result.

We follow the same approach as in encoding the invariants of the susceptible agent but instead of checking the invariants, we compute the probability for each case. In this property test we cannot randomise the model parameters because this would lead to random coverage. This might seem like a disadvantage but we do not really have a choice here, still the model parameters can be adjusted arbitrarily and the property must hold. %Note that we do not provide the details of computing the probabilities of each input-to-output case as it is quite technical and of not much importance - it is only a matter of multiplication and divisions amongst the event-frequencies and model parameters.
We make use of the \texttt{cover} function together with \texttt{checkCoverage}, which ensures that we get a statistical robust estimate whether the expected percentages can be reached or not. Implementing this property test is then simply a matter of computing the probabilities and of case analysis over the random input event and the agents output.

\begin{HaskellCode}
...
case evt of 
  Recover -> 
    cover recoverPerc True 
     ("Susceptible receives Recover, expected " ++ show recoverPerc) True
...
\end{HaskellCode}

Note the usage pattern of \texttt{cover} where we unconditionally include the test case into the coverage class so all test cases pass. The reason for this is that we are just interested in testing the coverage, which is in fact the property we want to test. We could have combined this test into the previous one but then we couldn't have used randomised model parameters. For this reason, and to keep the concerns separated we opted for two different tests, which makes them also much more readable.

%\begin{HaskellCode}
%prop_susceptible_proabilities :: Positive Double      -- ^ Current simulation time
%                              -> NonEmptyList AgentId -- ^ Agent ids of the population
%                              -> Property
%prop_susceptible_proabilities (Positive t) (NonEmpty ais) = checkCoverage (do
%  -- fixed model parameters, otherwise random coverage
%  let cor = 5
%      inf = 0.05
%      ild = 15.0
%
%   -- compute distributions for all cases
%  let recoverPerc       = ...
%      makeContPerc      = ...
%      contactRecPerc    = ...
%      contactSusPerc    = ...
%      contactInfSusPerc = ...
%      contactInfInfPerc = ...
%
%  -- generate a random event
%  evt <- genEvent ais
%  -- run susceptible random agent with given parameters
%  (_, ao, _) <- genRunSusceptibleAgent cor inf ild t ais evt
%
%  -- encode expected distributions
%  return $ property $
%    case evt of 
%      Recover -> 
%        cover recoverPerc True 
%          ("Susceptible receives Recover, expected " ++ 
%           show recoverPerc) True
%      MakeContact -> 
%        cover makeContPerc True 
%          ("Susceptible receives MakeContact, expected " ++ 
%           show makeContPerc) True
%      (Contact _ Recovered) -> 
%        cover contactRecPerc True 
%          ("Susceptible receives Contact * Recovered, expected " ++ 
%           show contactRecPerc) True
%      (Contact _ Susceptible) -> 
%        cover contactSusPerc True 
%          ("Susceptible receives Contact * Susceptible, expected " ++ 
%           show contactSusPerc) True
%      (Contact _ Infected) -> 
%        case ao of
%          Susceptible ->
%            cover contactInfSusPerc True 
%              ("Susceptible receives Contact * Infected, stays Susceptible " ++
%               ", expected " ++ show contactInfSusPerc) True
%          Infected ->
%            cover contactInfInfPerc True 
%              ("Susceptible receives Contact * Infected, becomes Infected, " ++
%               ", expected " ++ show contactInfInfPerc) True
%          _ ->
%            cover 0 True "Impossible Case, expected 0" True
%\end{HaskellCode}

When running the property test we get the following output:

\begin{footnotesize}
\begin{verbatim}
+++ OK, passed 819200 tests (7.32s):
33.3582% Susceptible receives MakeContact, expected 33.33%
33.2578% Susceptible receives Recover, expected 33.33%
11.1643% Susceptible receives Contact * Recovered, expected 11.11%
11.1096% Susceptible receives Contact * Susceptible, expected 11.11%
10.5616% Susceptible receives Contact * Infected, stays Susceptible, expected 10.56%
 0.5485% Susceptible receives Contact * Infected, becomes Infected, expected 0.56%
\end{verbatim}
\end{footnotesize}

After 819,200 (!) test cases QuickCheck comes to the conclusion that the distributions generated by the test cases reflect the expected distributions and passes the property test. We see that the values do not match exactly in some cases but by using sequential statistical hypothesis testing, QuickCheck is able to conclude that the coverage are statistically equal.

\section{Discussion}

\subsection{Other Models}
TODO: mention that we have also implemented other models, which also work without time-semantics (all agents make a move at discrete time-steps and do not really rely on some notion of time). 

\subsection{Time-Semantics}
The main reason for building our pure functional ABMS approach on top of Yampa was to leverage the powerful time-semantics of Yampa which allows us to implement important concepts of ABMS:

state-chart: agents are at all time of their life-cycle in one state and can switch between multiple states using transitions 
timed transitions: transition to another state/behaviour happens at a discrete time
rate transitions: transition happens with a given rate
message transition: transition upon receiving a given message 

\subsection{Agents as Signals}
Due to the underlying nature and motivation of Functional Reactive Programming (und im speziellen) Yampa, Agents can be seen as Signals which is generated and consumed by a Signal-Function which is the behaviour of an Agent. If an Agent does not change the OUTPUT-signal is constant, if the agent changes e.g. by sending a message, changing its state,... the OUTPUT signal changes. A dead agent has no signal at all.

\subsection{Time-Sampling}
sampling rate depends on the transition times \& rates of the model. when e.g. the contact rate is 5 then the sampling dt should be below 0.2

\subsection{System Dynamics}
can emulate system dynamics due to the parallel update-strategy and continuous time-flow semantics

\subsection{Discrete Event Simulation}
DES in FrABMS? how easily can we implement server/queue systems? do they also look like a specification? potential problem: ordering of messages is not guaranteed by now

\subsection{Advantages}
advantages:
	- no side-effects within agents leads to much safer code
	- edsl for time-semantics
	- declarative style: agent-implementation looks like a model-specification
	- reasoning and verification
	- sequential and parallel
	- powerful time-semantics
	- arrowized programming is optional and only required when utilizing yampas time-semantics. if the model does not rely on time-semantics, it can use monadic-programming by building on the existing monadic functions in the EDSL which allow to run in the State-Monad which simplifies things very much
	- when to use yampas arrowized programing: time-semantics, simple state-chart agents 
	- when not using yampas facilities: in all the other cases e.g. SugarScape is such a case as it proceeds in unit time-steps and all agents act in every time-step
	- can implement System Dynamics building on Yampas facilities with total ease	
	- get replications for free without having to worry about side-effects and can even run them in parallel without headaches
	- cant mess around with time because delta-time is hidden from you (intentional design-decision by Yampa). this would be only very difficult and cumbersome to achieve in an object-oriented approach. TODO: experiment with it in Java - how could we actually implement this? I think it is impossible: may only achieve this through complicated application of patterns and inheritance but then has the problem of how to update the dt and more important how to deal with functions like integral which accumulates a value through closures and continuations. We could do this in OO by having a general base-class e.g. ContinuousTime which provides functions like updateDt and integrate, but we could only accumulate a single integral value.
	- reproducibility statically guaranteed
	- cannot mess around with dt
	- code == specification
	- rule out serious class of bugs
	- different time-sampling leads to different results e.g. in wildfire \& SIR but not in Prisoners Dilemma. why? probabilistic time-sampling?
	- reasoning about equivalence between SD and ABS implementation in the same framework
	- recursive implementations
	
	- we can statically guarantee the reproducibility of the simulation because: no side effects possible within the agents which would result in differences between same runs (e.g. file access, networking, threading), also timedeltas are fixed and do not depend on rendering performance or userinput	
	
\subsection{Disadvantages}
disadvantages:
	- performance is low
	- reasoning about performance is very difficult
	- very steep learning curve for non-functional programmers
	- learning a new EDSL
	- think ABMS different: when to use async messages, when to use sync conversations


[ ] important: increasing sampling freqzency and increasing number of steps so that the same number of simulation steps are executed should lead to same results. but it doesnt. why?
[ ] hypothesis: if time-semantics are involved then event ordering becomes relevant for emergent patterns. there are no tine semantics in heroes and cowards but in the prisoners dilemma
[ ] can we implement different types of agents interacting with each other in the same simulation ? with different behaviour funcs, digferent state? yes, also not possible in NetLogo to my knowledge. but they must have the same messages, emvironment 

[ ] Hypothesis: we can combine with FrABS agent-based simulation and system dynamics (this has been proved by example!)

\chapter{Conclusions}
\label{ch:conclusions}

This chapter concludes the whole thesis and outlines future research. Roughly 20\% exists already.

%we now know how to engineer time- and event-driven ABS with complex state both in the agent and environment, main difficulty is direct agent-interaction (see macal classification into 4 types of ABS), compile-time guaranteed reproducibility, explicit handling of complex state (read only, read/write), concurrency explicit and limited to STM, very promising concurrency but direct agent-interactions main problem (erlang as a rescue?), main drawbacks: everything is explicit, performance

\section{Further Research}
clearly outline the ideas for further research

\subsection{A general purpose library}
generalise concepts explored into a pure functional ABS library in Haskell (called chimera)

\subsection{Dependent and linear types}
dependent types and linear types are the next big step, towards a stronger formalisation of agents and ABS,
focus on the equilibrium - totality correspondence

\subsection{Concurrent event-driven ABS}
stm based concurrency for event-driven ABS using parallel DES. challenge is the time-warp implementation using monads. in general it should be easy to roll-back agents actions but with monads we have to be careful - for some monads rolling back is not neccessary e.g. rand and reader, for others it is, and for some it is impossible e.g. IO

\bibliographystyle{apacite}
\bibliography{references}

\end{document}