\section{Related Work}
\label{sec:related}

Research on code testing of ABS is quite new with few publications so far. Our own work \cite{thaler_show_2019} is the first paper to introduce property-based testing to ABS. In it we show on a conceptual level that property-based testing allows to do both verification and validation of an implementation. However, in this work we do not go into technical details of actual implementations nor how to use property-based testing on a technical level.

TODO: rephrase
Collier et al. \cite{collier_test-driven_2013} were the first to discuss how to apply TDD to ABS, using unit testing \cite{beck_test_2002} to verify the correctness of the implementation up to a certain level. They show how to implement unit tests within the RePast Framework  and make the important point that such a software needs to be designed to be sufficiently modular otherwise testing becomes too cumbersome and involves too many parts. A similar approach has been discussed for Discrete Event Simulation in the AnyLogic software toolkit \cite{asta_investigation_2014}. 

TODO: rephrase
The work \cite{onggo_test-driven_2016} proposes Test Driven Simulation Modeling (TDSM), which combines techniques from TDD to simulation modeling. The authors present a case study for maritime search operations where they employ ABS. They emphasize that simulation modeling is an iterative process, where changes are made to existing parts, making a TDD approach to simulation modeling a good match. They present how to validate their model against analytical solutions from theory using unit tests by running the whole simulation within a unit test and then perform a statistical comparison against a formal specification.

TODO: rephrase
The authors of \cite{gurcan_generic_2013} give an in-depth and detailed overview over verification, validation and testing of agent-based models and simulations and proposes a generic framework for it. The authors present a generic UML class-model for their framework which they then implement in the two ABS frameworks RePast and MASON. Both of them are implemented in Java and the authors provide a detailed description how their generic testing framework architecture works and how it utilizes JUnit to run automated tests. To demonstrate their framework they provide also a case study of an agent-based simulation of synaptic connectivity where they provide an in-depth explanation of their levels of test together with code.

TODO: rephrase
The work of \cite{onggo_test-driven_2016} explicitly mention the problem of test coverage, which would often require to write a large number of tests manually to cover the parameter ranges sufficiently enough. Property-based testing addresses exactly this problem by \textit{automating} the test-data generation. Note that this is closely related to data-generators \cite{gurcan_generic_2013} and load generators and random testing \cite{burnstein_practical_2010} but property-based testing goes one step further by integrating this into a specification language directly into code, emphasizing a declarative approach and pushing the generators behind the scenes, making them transparent and focusing on the specification rather than on the data-generation. 