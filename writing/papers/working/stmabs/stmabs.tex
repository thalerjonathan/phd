%% BioMed_Central_Tex_Template_v1.06
%%                                      %
%  bmc_article.tex            ver: 1.06 %
%                                       %

%%IMPORTANT: do not delete the first line of this template
%%It must be present to enable the BMC Submission system to
%%recognise this template!!

%%%%%%%%%%%%%%%%%%%%%%%%%%%%%%%%%%%%%%%%%
%%                                     %%
%%  LaTeX template for BioMed Central  %%
%%     journal article submissions     %%
%%                                     %%
%%          <8 June 2012>              %%
%%                                     %%
%%                                     %%
%%%%%%%%%%%%%%%%%%%%%%%%%%%%%%%%%%%%%%%%%


%%%%%%%%%%%%%%%%%%%%%%%%%%%%%%%%%%%%%%%%%%%%%%%%%%%%%%%%%%%%%%%%%%%%%
%%                                                                 %%
%% For instructions on how to fill out this Tex template           %%
%% document please refer to Readme.html and the instructions for   %%
%% authors page on the biomed central website                      %%
%% http://www.biomedcentral.com/info/authors/                      %%
%%                                                                 %%
%% Please do not use \input{...} to include other tex files.       %%
%% Submit your LaTeX manuscript as one .tex document.              %%
%%                                                                 %%
%% All additional figures and files should be attached             %%
%% separately and not embedded in the \TeX\ document itself.       %%
%%                                                                 %%
%% BioMed Central currently use the MikTex distribution of         %%
%% TeX for Windows) of TeX and LaTeX.  This is available from      %%
%% http://www.miktex.org                                           %%
%%                                                                 %%
%%%%%%%%%%%%%%%%%%%%%%%%%%%%%%%%%%%%%%%%%%%%%%%%%%%%%%%%%%%%%%%%%%%%%

%%% additional documentclass options:
%  [doublespacing]
%  [linenumbers]   - put the line numbers on margins

%%% loading packages, author definitions

%\documentclass[twocolumn]{bmcart}% uncomment this for twocolumn layout and comment line below
%\documentclass[linenumbers, doublespacing]{bmcart}
\documentclass{bmcart}

%%% Load packages
%\usepackage{amsthm,amsmath}
%\RequirePackage{natbib}
%\RequirePackage[authoryear]{natbib}% uncomment this for author-year bibliography
%\RequirePackage{hyperref}
\usepackage[utf8]{inputenc} %unicode support
%\usepackage[applemac]{inputenc} %applemac support if unicode package fails
%\usepackage[latin1]{inputenc} %UNIX support if unicode package fails

\usepackage{hyperref}
\usepackage{booktabs} % For formal tables
\usepackage{float}
\usepackage{graphicx}
\usepackage{subcaption}
\usepackage{ifthen}
%\usepackage{minted}
\usepackage{verbatim}
\usepackage{multirow}

%%%%%%%%%%%%%%%%%%%%%%%%%%%%%%%%%%%%%%%%%%%%%%%%%
%%                                             %%
%%  If you wish to display your graphics for   %%
%%  your own use using includegraphic or       %%
%%  includegraphics, then comment out the      %%
%%  following two lines of code.               %%
%%  NB: These line *must* be included when     %%
%%  submitting to BMC.                         %%
%%  All figure files must be submitted as      %%
%%  separate graphics through the BMC          %%
%%  submission process, not included in the    %%
%%  submitted article.                         %%
%%                                             %%
%%%%%%%%%%%%%%%%%%%%%%%%%%%%%%%%%%%%%%%%%%%%%%%%%

%\def\includegraphic{}
%\def\includegraphics{}

%%% Put your definitions there:
\startlocaldefs
\endlocaldefs


%%% Begin ...
\begin{document}

%\newminted[HaskellCode]{haskell}{fontsize=\footnotesize}

%%% Start of article front matter
\begin{frontmatter}

\begin{fmbox}
\dochead{Research}

%%%%%%%%%%%%%%%%%%%%%%%%%%%%%%%%%%%%%%%%%%%%%%
%%                                          %%
%% Enter the title of your article here     %%
%%                                          %%
%%%%%%%%%%%%%%%%%%%%%%%%%%%%%%%%%%%%%%%%%%%%%%

\title{A tale of lock-free Agents: Towards Software Transactional Memory in parallel Agent-Based Simulation}
%\subtitle{Towards Software Transactional Memory in parallel Agent-Based Simulation}

%%%%%%%%%%%%%%%%%%%%%%%%%%%%%%%%%%%%%%%%%%%%%%
%%                                          %%
%% Enter the authors here                   %%
%%                                          %%
%% Specify information, if available,       %%
%% in the form:                             %%
%%   <key>={<id1>,<id2>}                    %%
%%   <key>=                                 %%
%% Comment or delete the keys which are     %%
%% not used. Repeat \author command as much %%
%% as required.                             %%
%%                                          %%
%%%%%%%%%%%%%%%%%%%%%%%%%%%%%%%%%%%%%%%%%%%%%%

\author[
   addressref={aff1},                   % id's of addresses, e.g. {aff1,aff2}
   corref={aff1},                       % id of corresponding address, if any
   %noteref={n1},                        % id's of article notes, if any
   email={jonathan.thaler@nottingham.ac.uk}   % email address
]{\inits{JT}\fnm{Jonathan} \snm{Thaler}}
\author[
   addressref={aff1},
   email={peer-olaf.siebers@nottingham.ac.uk}
]{\inits{POS}\fnm{Peer-Olaf} \snm{Siebers}}

%%%%%%%%%%%%%%%%%%%%%%%%%%%%%%%%%%%%%%%%%%%%%%
%%                                          %%
%% Enter the authors' addresses here        %%
%%                                          %%
%% Repeat \address commands as much as      %%
%% required.                                %%
%%                                          %%
%%%%%%%%%%%%%%%%%%%%%%%%%%%%%%%%%%%%%%%%%%%%%%

\address[id=aff1]{%                           % unique id
  \orgname{University of Nottingham}, % university, etc
  \street{7301 Wollaton Rd},                     %
  \postcode{NG8 1BB}                                % post or zip code
  \city{Nottingham},                              % city
  \cny{UK}                                    % country
}

%%%%%%%%%%%%%%%%%%%%%%%%%%%%%%%%%%%%%%%%%%%%%%
%%                                          %%
%% Enter short notes here                   %%
%%                                          %%
%% Short notes will be after addresses      %%
%% on first page.                           %%
%%                                          %%
%%%%%%%%%%%%%%%%%%%%%%%%%%%%%%%%%%%%%%%%%%%%%%

\begin{artnotes}
%%\note{Sample of title note}     % note to the article
%\note[id=n1]{Equal contributor} % note, connected to author
\end{artnotes}

\end{fmbox}% comment this for two column layout

%%%%%%%%%%%%%%%%%%%%%%%%%%%%%%%%%%%%%%%%%%%%%%
%%                                          %%
%% The Abstract begins here                 %%
%%                                          %%
%% Please refer to the Instructions for     %%
%% authors on http://www.biomedcentral.com  %%
%% and include the section headings         %%
%% accordingly for your article type.       %%
%%                                          %%
%%%%%%%%%%%%%%%%%%%%%%%%%%%%%%%%%%%%%%%%%%%%%%

\begin{abstractbox}

% TODO:
% Major changes:
% 1. implement fairer lock-based implementations: use ReadWrite lock in case of SIR, use cell-based lock in SugarScape. Don't delete old results. Due to my new hardware I need a full re-run of all experiments. This allows to properly address point 2, which also the JOS referees made: more robust statistics and results of performance measurements and about reproducibility. This cost me a lof of time because I did it very naively by hand, instead of using a proper tool. Therefore, see point 2:
%	NOTE: atomicModifyIORef' is strictly speaking not a lock-based but actually a lock-free implementation (!!) based on IO instead of STM: it does not aquire any lock but uses hardware features of the CPU to swap a pointer https://stackoverflow.com/questions/10102881/haskell-how-does-atomicmodifyioref-work. mention this unique feature of haskell, which is not possible in other languages (TODO: is this so? do we have here really a unique feature of haskell?). A downside is that it only works in case there is only a single IORef in the program and that we are still running in IO - the performance however is even better than STM (!!!)
% 2. use criterion or gauge to automately re-run all tests with much more and robust detail on performance.
% 3. incorporate minor changes outlined by the reviewers

\begin{abstract} % abstract
With the decline of Moore's law and the ever increasing availability of cheap massively parallel hardware, it becomes more and more important to embrace parallel programming methods to implement Agent-Based Simulations (ABS). This has been acknowledge in the field a while ago and numerous research on distributed parallel ABS exists, focusing primarily on Parallel Discrete Event Simulation as the underlying mechanism. However, these concepts and tools are inherently difficult to master and apply and often overkill in case implementers simply want to parallelise their own, custom agent-based model implementation. However, with the established programming languages in the field, Python, Java and C++, it is not easy to address the complexities of parallel programming due to unrestricted side effects and the intricacies of low-level locking semantics. Therefore, in this paper we propose the use of a lock-free approach to parallel ABS using Software Transactional Memory (STM) in conjunction with the pure functional programming language Haskell, which in combination, removes some of the problems and complexities of parallel implementations in imperative approaches.

We present two case studies where we compare the performance of lock-based and lock-free STM implementations in two different well known Agent-Based Models, where we investigate both the scaling performance under increasing number of CPUs and the scaling performance under increasing number of agents. We show that the lock-free STM implementations consistently outperform the lock-based ones and scale much better to increasing number of CPUs both on local machines and on Amazon Cloud Services. Further, by utilizing the pure functional language Haskell we gain the benefits of immutable data and lack of unrestricted side effects guaranteed at compile-time, making validation easier and leading to increased confidence in the correctness of an implementation, something of fundamental importance and benefit in parallel programming in general and scientific computing like ABS in particular.
\end{abstract}

%%%%%%%%%%%%%%%%%%%%%%%%%%%%%%%%%%%%%%%%%%%%%%
%%                                          %%
%% The keywords begin here                  %%
%%                                          %%
%% Put each keyword in separate \kwd{}.     %%
%%                                          %%
%%%%%%%%%%%%%%%%%%%%%%%%%%%%%%%%%%%%%%%%%%%%%%

\begin{keyword}
\kwd{Agent-Based Simulation}
\kwd{Software Transactional Memory}
\kwd{Parallel Programming}
\kwd{Haskell}
\end{keyword}

% MSC classifications codes, if any
%\begin{keyword}[class=AMS]
%\kwd[Primary ]{}
%\kwd{}
%\kwd[; secondary ]{}
%\end{keyword}

\end{abstractbox}
%
%\end{fmbox}% uncomment this for twcolumn layout

\end{frontmatter}

%%%%%%%%%%%%%%%%%%%%%%%%%%%%%%%%%%%%%%%%%%%%%%
%%                                          %%
%% The Main Body begins here                %%
%%                                          %%
%% Please refer to the instructions for     %%
%% authors on:                              %%
%% http://www.biomedcentral.com/info/authors%%
%% and include the section headings         %%
%% accordingly for your article type.       %%
%%                                          %%
%% See the Results and Discussion section   %%
%% for details on how to create sub-sections%%
%%                                          %%
%% use \cite{...} to cite references        %%
%%  \cite{koon} and                         %%
%%  \cite{oreg,khar,zvai,xjon,schn,pond}    %%
%%  \nocite{smith,marg,hunn,advi,koha,mouse}%%
%%                                          %%
%%%%%%%%%%%%%%%%%%%%%%%%%%%%%%%%%%%%%%%%%%%%%%

%%%%%%%%%%%%%%%%%%%%%%%%% start of article main body
% <put your article body there>


%*******************************************************************************
%*********************************** First Chapter *****************************
%*******************************************************************************

\chapter{Introduction}  %Title of the First Chapter
I noticed that it is pretty hard to convince an agent-based economics specialist who is not a computer scientist about a pure functional approach. My conjecture is that the implementation technique and method does not matter much to them because they have very little knowledge about programming and are almost always self-taught - they don't know about software-engineering, nothing about proper software-design and architecture, nothing about software-maintenance, nothing about unit-testing,... In the end they just "hack" the simulation in whatever language they are able to: C++, Visual Basic, Java or toolboxes like Netlogo. For them it is all about to \textit{get things done somehow} and not to get things done the right way or in a beautiful way - the way and the method doesn't matter, its just a necessary evil which needs to be done. Thus if functional programming could make their lives easier, then they will definitely welcome it. But functional programming is, i think, harder to learn and harder to understand - so one needs to provide an abstraction through EDSL. So I REALLY need to come up with convincing arguments why to use pure functional approaches in ACE THEY can understand, otherwise I will be lost and not heard (not published,...). \\

What ACE economists care for:

\begin{itemize}
\item Very: Qualitative modelling with quantitative results
\item Yes: Easy reproducibility
\item Likely: Reasoning about convergence?
\item Likely: EDSL
\end{itemize}

My contributions are: pure functional framework, functional agent-model for market-simulations, EDSL for market-simulations, qualitative / implicit modelling with quanitative results, reasoning in my framework about convergence \\

IDEA: could I develop non-causal modelling (models are expressed in terms of non-directed equations, modelled in signal-relations) to allow for qualitative modelling for the agent-based economists? See hybrid modelling paper of Yampa. \textbf{THIS WOULD BE A HUGE NOVEL CONTRIBUTION TO ACE ESPECIALLY WHEN COMBINED WITH AN EDSL AND PROVIDING FULL REFERENTIAL TRANSPARENCY TO KEEP THE ABILITY TO REASON ABOUT CONVERGENCE}. This should be covered in the "EDSL"-paper.

TODO: maybe i should really focus only on market models? otherwise too much? \\

central novelty of my PhD: model specification = runnable code. possible through EDSL. but only in specific subfield of ACE: market-models. need a functional description of the model, then translate it to model specification in EDSL and then run it to see dynamics. But: model specification moves closer to functional programming languages. \\

another novelty approach: model specification through qualitative instead of quantiative approaches. is this possible? \\

WHY FUNCTIONAL? "because its the ultimate approach to scientific computing": fewer bugs due to mutable state (why? is thos shown obkectively by someone?), shorter (again as above, productivity), more expressive and closer to math, EDSL, EDSL=model=simulation, better parallelising due to referental transparency, reasoning \\

scientific results need to be reproduced, especially when they have high impact. a more formal approach of specifying the model and the simulation (model=simulation) could lead to easier sharing and easier reporduction without ambigouites \\

pure functional agent-model \& theory, EDSL framework in Haskell for ACE

\begin{enumerate}
\item Which kind of problem do we have?
\item What aim is there? Solving the problem? 
\item How the aim is achieved by enumerating VERY CLEAR objectives.
\item What the impact one expects (hypothesis) and what it is (after results).
\end{enumerate}

Note: It is not in the interest of the researcher to develop new economic theories but to research the use of functional methods (programming and specification) in agent-based computational economics (ACE).

NOTE: Get the reader’s attention early in the introduction: motivation, significance, originality and novelty.

\section{Methods}
Methods need to be selected to implement the simulations. Special emphasis will be put on functional ones which will then be compared to established methods in the field of ABM/S and ACE. \\

Claim: non-programming environments are considered to be not powerful enough to capture the complexity of ACE implementations thus a programming approach to ACE will be always required.

\section{Scenarios}
To apply and test functional methods in ACE, four scenarios of ACE are selected and then the methods applied and compared with each other to see how each of them perform in comparison. The 4 selected scenarios represent a selection of the challenges posed in ACE: from very abstract ones to very operational ones.

\section{Comparison}
Each of the selected scenarios is then implemented using the selected methods where each solution is then compared against the following criteria: 

\begin{enumerate}
\item suitability for scientific computation
\item robustness
\item error-sources
\item testability
\item stability
\item extendability
\item size of code
\item maintainability
\item time taken for development
\item verification \& correctness
\item replications \& parallelism
\item EDSL
\end{enumerate}

This will then allow to compare the different methods against each other and to show under which circumstances functional methods shine and when they should not be used.

\section{Agent-Based Modelling and Simulation (ABM/S)}
ABM/S is a method of modelling and simulating a system where the global behaviour may be unknown but the behaviour and interactions of the parts making up the system is of knowledge (Wooldrige, M. (2009). An Introduction to MultiAgent Systems. John Wiley & Sons). Those parts, called agents, are modelled and simulated out of which then the aggregate global behaviour of the whole system emerges. Thus the central aspect of ABM/S is the concept of an Agent which can be understood as a metaphor for a pro-active unit, able to spawn new Agents, and interacting with other Agents in a network of neighbours by exchange of messages. The implementation of Agents can vary and strongly depends on the programming language and the kind of domain the simulation and model is situated in.

\section{Agent-Based Economics (ACE)}
According to Leigh Tesfatsion (Tesfatsion, L. (2006). Agent-based computational economics: A constructive approach to economic theory. In Tesfatsion, L. and Judd, K. L., editors, Handbook of Computational Economics, volume 2, chapter 16, pages 831–880. Elsevier, 1 edition.), one of the leading figures, ACE is "[...] computational modelling of economic processes (including whole economies) as open-ended dynamic systems of interacting agents." - thus lending perfectly to the use of ABM/S as already the name suggests. Whereas classical economic models fall short by only looking at the average, pure rational, individual interacting in anonymous markets, the ACE approach looks at heterogeneous, non-rational individuals interacting with each other in networks (Kirman, A. (2010). Complex Economics: Individual and Collective Rationality. Routledge, London ; New York, NY.). Thus ACE can be understood as a combination of computer-science, cognitive/social science and evolutionary economics.

\section{Functional programming}
TODO: read \cite{Backus1978}

The state-of-the-art approach to implementing Agents are object-oriented methods and programming as the metaphor of an Agent as presented above lends itself very naturally to object-orientation (OO). The author of this thesis claims that OO in the hands of inexperienced or ignorant programmers is dangerous, leading to bugs and hardly maintainable and extensible code. The reason for this is that OO provides very powerful techniques of organising and structuring programs through Classes, Type Hierarchies and Objects, which, when misused, lead to the above mentioned problems. Also major problems, which experts face as well as beginners are 1. state is highly scattered across the program which disguises the flow of data in complex simulations and 2. objects don’t compose as well as functions. The reason for this is that objects always carry around some internal state which makes it obviously much more complicated as complex dependencies can be introduced according to the internal state.
All this is tackled by (pure) functional programming which abandons the concept of global state, Objects and Classes and makes data-flow explicit. This then allows to reason about correctness, termination and other properties of the program e.g. if a given function exhibits side-effects or not. Other benefits are fewer lines of code, easier maintainability and ultimately fewer bugs thus making functional programming the ideal choice for scientific computing and simulation and thus also for ACE. A very powerful feature of functional programming is Lazy evaluation. It allows to describe infinite data-structures and functions producing an infinite stream of output but which are only computed as currently needed. Thus the decision of how many is decoupled from how to (Hughes, J. (1989). Why functional programming matters. Comput. J., 32(2):98–107.).
The most powerful aspect using pure functional programming however is that it allows the design of embedded domain specific languages (EDSL). In this case one develops and programs primitives e.g. types and functions in a host language (embed) in a way that they can be combined. The combination of these primitives then looks like a language specific to a given domain, in the case of this thesis ACE. The ease of development of EDSLs in pure functional programming is also a proof of the superior extensibility and composability of pure functional languages over OO (Henderson P. (1982). Functional Geometry. Proceedings of the 1982 ACM Symposium on LISP and Functional Programming.).
One of the most compelling example to utilize pure functional programming is the reporting of Hudak (Hudak P., Jones M. (1994). Haskell vs. Ada vs. C++ vs. Awk vs. ... An Experiment in Software Prototyping Productivity. Department of Computer Science, Yale University.)  where in a prototyping contest of DARPA the Haskell prototype was by far the shortest with 85 lines of code. Also the Jury mistook the code as specification because the prototype did actually implement a small EDSL which is a perfect proof how close EDSL can get to and look like a specification.

Functional languages can best be characterized by their way computation works: instead of \textit{how} something is computed, \textit{what} is computed is described. Thus functional programming follows a declarative instead of an imperative style of programming. The key points are:
\begin{itemize}
\item No assignment statements - variables values can never change once given a value.
\item Function calls have no side-effect and will only compute the results - this makes order of execution irrelevant, as due to the lack of side-effects the logical point in \textit{time} when the function is calculated within the program-execution does not matter.
\item higher-order functions
\item lazy evaluation
\item Looping is achieved using recursion, mostly through the use of the general fold or the more specific map.
\item Pattern-matching
\end{itemize}

This alone does not really explain the \textit{real} advantages of functional programming and one must look for better motivations using functional programming languages. One motivation is given in \cite{Hughes1989} which is a great paper explaining to non-functional programmers what the significance of functional programming is and helping functional programmers putting functional languages to maximum use by showing the real power and advantages of functional languages. The main conclusion is that \textit{modularity}, which is the key to successful programming, can be achieved best using higher-order functions and lazy evaluation provided in functional languages like Haskell. \cite{Hughes1989} argues that the ability to divide problems into sub-problems depends on the ability to glue the sub-problems together which depends strongly on the programming-language and \cite{Hughes1989} argues that in this ability functional languages are superior to structured programming.

TODO: comparison of functional and object-oriented programming. My points are:
\begin{itemize}
\item The way state can be changed and treated - distributed over multiple objects - is often very difficult to understand.
\item Inheritance is a dangerous thing if not used with care because inheritance introduces very strong dependencies which cannot be changed during runtime anymore.
\item Objects don't compose very well: \url{http://zeroturnaround.com/rebellabs/why-the-debate-on-object-oriented-vs-functional-programming-is-all-about-composition/}
\item (Nearly) impossible to reason about programs
\end{itemize}

In conclusion the upsides of functional programming as opposed to OO are:
\begin{itemize}
\item Much more explicit flow of data \& control
\item Much better compose-able
\item Much better parallelism
\end{itemize}

\section{Related Research}
Tim Sweeney, CTO of Epic Games gave an invited talk about how "future programming languages could help us write better code" by "supplying stronger typing, reduce run-time failures;  and the need for pervasive concurrency support, both implicit and explicit, to effectively exploit the several forms of parallelism present in games and graphics." \cite{Sweeney2006}. Although the fields of games and agent-based simulations seem to be very different in the end, they have also very important similarities: both are simulations which perform numerical computations and update objects - in games they are called "game-objects" and in abm they are called agents but they are in fact the same thing - in a loop either concurrently or sequential. His key-points were:

\begin{itemize}
\item Dependent types as the remedy of most of the run-time failures.
\item Parallelism for numerical computation: these are pure functional algorithms, operate locally on mutable state. Haskell ST, STRef solution enables encapsulating local heaps and mutability within referentially transparent code.
\item Updating game-objects (agents) concurrently using STM: update all objects concurrently in arbitrary order, with each update wrapped in atomic block - depends on collisions if performance goes up.
\end{itemize}

\section{Background}

\subsection{Schelling Segregation}
We follow in our implementation the original paper of Schelling as in \cite{schelling_dynamic_1971} where we focus on the \textit{Area Distribution} section (Schelling starts with movement in a linear, 1-dimensional world where agents are able to move to the nearest point which meets the agents satisfaction but this is not what we follow here). One assumes a discrete 2-dimensional lattice-world with NxM fields. Each field is either occupied by an agent of a given color (e.g. Red or Green) or is free. Each field has 8 neighbours, which denotes a Moore-Neighbourhood. In Schellings original work the lattice-world is limited at its borders but we assume a torus world which is wrapped around in both the x- and y-dimensions resulting in 8 neighbours also for fields at the border. The occupation density was set by Schelling to be about 70\%-75\% which he identifies as being a setting which allows the agents to move around freely without making the lattice-world too sparse.
Now the agents make their move sequentially one after another. In each move an agent calculates the number of neighbours which are of equal color. If the number satisfies the agents needs about the neighbourhood then the agent is regarded as being 'happy' and will stay on this field. On the other hand the agent moves to the nearest unoccupied field which satisfies its needs. An agent which moves selects an unoccupied place randomly relative from its current place within a rectangle of side-length 2r where its current place is at the center. The interpretation for that behaviour is that agents won't move too far as it could be costly. Also children might attend a school in this area or the family has friends in this area, so they don't want to break that.



Agents just move depending on their movement-strategy to another place if they are not happy on the current one - they don't care how the target place is in the present or in the future, they will decide again in the next time-step. The interpretation for that behaviour is: agents want to 'just get out' at any cost, not caring what the future place will look like - it might be better or worse but they will see then.

\subsubsection{Optimizing behaviour}
TODO: define utility

The original schelling model didn't have a move-optimizing behaviour, meaning agents are just binary: if it is happy it will not move, if it is unhappy it will move but they won't care where they move. We introduce local move-optimizing behaviours which can be interpreted as being realistic in the real-world. It is important to note that we focus on \textit{local} instead of \textit{global} move-optimization: the agents are limited in their reasoning-capabilities and have limited information available: they cannot check out \textit{every} place and pick the globally best one.\\

\subsubsection{Anticipating behaviour}
Schelling explicitly mentions in \cite{schelling_dynamic_1971} that nobody anticipates moves of others. This is what we introduce using the recursive simulation.

TODO: is this optimizing behaviour in the spirit of schellings original work? 

\paragraph{Optimizing future} Agents pick an unoccupied random place and move to it if it increases their utility in the future. The interpretation for that behaviour is: agents heard about a place which will be cool in the future.

\paragraph{Optimizing present \& future} Agents pick an unoccupied random place and move to it if it increases their utility in the now and in the future. The interpretation for that behaviour is: agents heard about a cool spot in town, check it out and move to it if they like it but they also anticipate the coolness of the place in the future and if it seems that the place is going down then they won't move there.

\subsection{Related Research}
TODO: \cite{kirman_complex_2010} mention kirman complex economics where he investigates the model more in depth


\section{STM and ABS}

For a proof-of-concept we changed the reference implementation of the agent-based SIR model on a 2D-grid as described in the paper in Appendix \ref{app:pfe}. In it, a State Monad is used to share the grid across all agents where all agents are run after each other to guarantee exclusive access to the state. We replaced the State Monad by the STM Monad, share the grid through a \textit{TVar} and run every agent within its own thread. All agents are run at the same time but synchronise after each time-step which is done through the main-thread.

We make STM the innermost Monad within a RandT transformer:
\begin{HaskellCode}
type SIRMonad g   = RandT g STM
type SIRAgent g   = SF (SIRMonad g) () ()
\end{HaskellCode}

In each step we use an \textit{MVar} to let the agents block on the next $\Delta t$ and let the main-thread block for all results. After each step we output the environment by reading it from the \textit{TVar}:
\begin{HaskellCode}
-- this is run in the main-thread
simulationStep :: TVar SIREnv
               -> [MVar DTime]
               -> [MVar ()]
               -> Int
               -> IO SIREnv
simulationStep env dtVars retVars _i = do
  -- tell all threads to continue with the corresponding DTime
  mapM_ (`putMVar` dt) dtVars
  -- wait for results, ignoring them, only [()]
  mapM_ takeMVar retVars
  -- read last version of environment
  readTVarIO env
\end{HaskellCode}

Each agent runs within its own thread. It will block for the posting of the next $\Delta t$ where it then will run the MSF stack with the given $\Delta t$ and atomically transacting the STM action. It will then post the result of the computation to the main-thread to signal it has finished. Note that the number of steps the agent will run is hard-coded and comes from the main-thread so that no infinite blocking occurs and the thread shuts down gracefully.

\begin{HaskellCode}
createAgentThread :: RandomGen g 
                  => Int 
                  -> TVar SIREnv
                  -> MVar DTime
                  -> g
                  -> (Disc2dCoord, SIRState)
                  -> IO (MVar ())
createAgentThread steps env dtVar rng0 a = do
    let sf = uncurry (sirAgent env) a
    -- create the var where the result will be posted to
    retVar <- newEmptyMVar
    _ <- forkIO (sirAgentThreadAux steps sf rng0 retVar)
    return retVar
  where
    agentThread :: RandomGen g 
                => Int
                -> SIRAgent g
                -> g
                -> MVar ()
                -> IO ()
    agentThread 0 _ _ _ = return ()
    agentThread n sf rng retVar = do
      -- wait for next dt to compute next step
      dt <- takeMVar dtVar

      -- compute next step
      let sfReader = unMSF sf ()
          sfRand   = runReaderT sfReader dt
          sfSTM    = runRandT sfRand rng
      ((_, sf'), rng') <- atomically sfSTM 
      
      -- post result to main thread
      putMVar retVar ()
      
      agentThread (n - 1) sf' rng' retVar
\end{HaskellCode}

\section{Case Study 1 - Spatial SIR (First Encounter)} 
\label{sec:cs_sir}

Our first case study is the SIR model which is a very well studied and understood compartment model from epidemiology \cite{kermack_contribution_1927} which allows to simulate the dynamics of an infectious disease like influenza, tuberculosis, chicken pox, rubella and measles spreading through a population \cite{enns_its_2010}.

In it, people in a population of size $N$ can be in either one of three states \textit{Susceptible}, \textit{Infected} or \textit{Recovered} at a particular time, where it is assumed that initially there is at least one infected person in the population. People interact \textit{on average} with a given rate of $\beta$ other people per time-unit and become infected with a given probability $\gamma$ when interacting with an infected person. When infected, a person recovers \textit{on average} after $\delta$ time-units and is then immune to further infections. An interaction between infected persons does not lead to re-infection, thus these interactions are ignored in this model. 

We followed in our agent-based implementation of the SIR model the work \cite{macal_agent-based_2010} but extended it by placing the agents on a discrete 2D grid using a Moore (8) neighbourhood TODO: cite my own PFE paper. In this case agents interact with each other indirectly through the shared discrete 2D grid by writing their current state on their cell which neighbours can read. A visualisation can be seen in Figure \ref{fig:vis_sir}.

It is important to note that due to the continuous-time nature of the SIR model, our implementation follows the time-driven \cite{meyer_event-driven_2014} approach and maps naturally to the continuous time-semantics and state-transitions provided by FRP. By sampling the system with very small $\Delta t$ this means that we have comparatively very few writes to the shared environment which will become important when discussing the performance results.

\begin{figure}
\begin{center}
	\begin{tabular}{c c}
		\begin{subfigure}[b]{0.4\textwidth}
			\centering
			\includegraphics[width=1\textwidth, angle=0]{./fig/sir/vis/51x51agents_t50_01dt.png}
			\caption{$t = 50$}
			\label{fig:vis_51x51agents_t50_01dt}
		\end{subfigure}
    	
    	&
  
		\begin{subfigure}[b]{0.4\textwidth}
			\centering
			\includegraphics[width=1\textwidth, angle=0]{./fig/sir/vis/51x51agents_t100_01dt.png}
			\caption{$t = 100$}
			\label{fig:vis_51x51agents_t100_01dt}
		\end{subfigure}
	\end{tabular}
	
	\caption{Simulating the agent-based SIR model on a 51x51 2D grid with Moore neighbourhood, a single infected agent at the center, contact rate $\beta = \frac{1}{5}$, infection probability $\gamma = 0.05$ and illness duration $\delta = 15$ . Simulation run until $t = 100$ with fixed $\Delta t = 0.1$. The susceptible agents are rendered as blue hollow circles for better contrast.}
	\label{fig:vis_sir}
\end{center}
\end{figure}

\subsection{Experiment Design}
In this case study we compare the performance of the following implementations under varying numbers of CPU cores and agent numbers:

\begin{enumerate}
	\item Sequential - This is the original implementation we also discuss in TODO: cite my own PFE paper. In it the discrete 2D grid is shared amongst all agents using the State Monad. Agents are run sequentially after another thus ensuring exclusive read/write access to it. Because we are neither running in the STM or IO Monad there is no way we can run this implementation concurrently.
	\item STM - This is the same implementation like the State Monad but instead of sharing the discrete 2D grid in a State Monad, agents run in the STM Monad and have access to the discrete 2D grid through a transactional variable \textit{TVar}. This means that the reads and writes of the discrete 2D grid are exactly the same but happen always through the \textit{TVar}. Also each agent is run within its own thread, thus enabling true concurrency when the simulation is actually run on multiple cores (which can be configured by the Haskell Runtime System).
	\item Lock-Based - This is exactly the same implementation like the STM Monad but instead of running in STM, the agents now run in IO. They share the discrete 2D grid using an \textit{IORef} and have access to an \textit{MVar} to synchronise access to the it. Also each agent is run within its own thread.
	\item RePast - To have an idea where the functional implementation is performance-wise compared to the established object-oriented methods, we implemented a Java version of the SIR model using RePast with the State-Chart feature. This implementation cannot run on multiple cores concurrently but gives a good estimate of the single core performance of imperative approaches. Also there exists a RePast High Performance Computing library for implementing large-scale distributed simulations in C++ - we leave this for further research as an implementation and comparison is out of scope of this paper.
\end{enumerate}

Each experiment was run until $t = 100$ and stepped using $\Delta t = 0.1$ except in RePast for which we don't have access to the underlying implementation of the state-chart and left it as it is. For each experiment we conducted 8 runs on our machine (see Table \ref{tab:machine_specs}) under no additional work-load and report both the average and standard deviation. Further, we checked the visual outputs and the dynamics and they look qualitatively the same to the reference implementation of the State Monad TODO: cite my own PFE paper. In the experiments we varied the number of agents (grid size) and the number of cores when running concurrently - the numbers are always indicated clearly. For varying the number of cores we compiled the executable using \textit{stack} and the \textit{threaded} option and executed it with \textit{stack} using the +RTS -Nx option where x is the number of cores between 1 and 4. 

\begin{table}
	\centering
	\begin{tabular}{ c || c }
		OS & Fedora 28 64-bit \\ \hline
		RAM & 16 GByte \\ \hline
		CPU & Intel Core i5-4670K @ 3.40GHz x 4 \\ \hline
		HD & 250Gbyte SSD \\ \hline
		Haskell & GHC 8.2.2 \\ \hline
		Java & OpenJDK 1.8.0 \\ \hline
		RePast & 2.5.0.a
	\end{tabular}
	
	\caption{Machine and Software Specs for all experiments}
	\label{tab:machine_specs}
\end{table}

\subsection{Constant Grid Size, Varying Cores}
In this experiment we held the grid size constant to 51 x 51 (2,601 agents) and varied the cores where possible. The results are reported in Table \ref{tab:constgrid_varyingcores}.

\begin{table}
	\centering
  	\begin{tabular}{ c || c | c  }
                    & Cores & Duration       \\ \hline \hline 
    	Sequential  & 1     & 100.33 (0.434) \\ \hline \hline
   		STM         & 1     & 53.182 (0.393) \\ \hline
   		STM         & 2     & 27.817 (0.555) \\ \hline
   		STM         & 3     & 21.776 (0.388) \\ \hline
   		STM         & 4     & 20.201 (0.789) \\ \hline \hline
   		Lock-Based  & 1     & 60.564 (0.265) \\ \hline 
   		Lock-Based  & 2     & 42.779 (0.421) \\ \hline 
   		Lock-Based  & 3     & 38.586 (0.451) \\ \hline 
   		Lock-Based  & 4     & 41.555 (0.445) \\ \hline \hline
   		RePast      & 1     & \textbf{10.822} (0.377)
  	\end{tabular}
  	
  	\caption{Experiments on constant 51x51 (2,601 agents) grid with varying number of cores.}
	\label{tab:constgrid_varyingcores}
\end{table}

%TODO: re-run the 3-core and 4-core versions of IO, i don't understand why on larger grid-sizes 4-core is faster. do 16 runs each
Comparing the performance and scaling on multiple cores of the STM and Lock-Based implementations shows that the lock-free STM implementation significantly outperforms the Lock-Based one and scales better to multiple cores. The Lock-Based implementation performs best with 3 cores and shows slightly worse performance on 4 cores as can be seen in Figure \ref{fig:core_duration_stm_io}. This is no surprise because the more cores are running at the same time, the more contention for the lock, thus the more likely synchronisation happening, resulting in more potential for reduced performance. This is not an issue in STM because no locks are taken in advance. 

Comparing the reference \textit{State} implementation shows that it is the slowest by far - even the single core STM and Lock-Based implementations outperform it by far. Also our profiling results reported about 30\% increased memory footprint for the State implementation. This shows that the State Monad is a rather slow and memory intense approach sharing data but guarantees purity and excludes any non-deterministic side-effects which is not the case in STM and IO.

What comes a bit as a surprise is that the single core RePast implementation significantly outperforms \textit{all} other implementations, even when they run on multiple cores and even with RePast doing complex visualisation in addition (something the functional implementations don't do). We attribute this to the conceptually slower approach of functional programming. We might could have optimised parts of the code but leave this for further research.

\begin{figure}
	\centering
	\includegraphics[width=0.6\textwidth, angle=0]{./fig/sir/core_duration_stm_io.png}
	\caption{Comparison of performance and scaling on multiple cores of STM vs. IO. Note that the Lock-Based implementation performs worse on 4 cores than on 3.}
	\label{fig:core_duration_stm_io}
\end{figure}

\subsection{Varying Grid Size, Constant Cores}
In this experiment we varied the grid size and used constantly 4 cores. Because in the previous experiment, Lock-Based performed best on 3 cores, we additionally ran Lock-Based on 3 cores as well. The results for STM are reported in Table \ref{tab:varyinggrid_constcores_stm}, for Lock-Based in Tables \ref{tab:varyinggrid_constcores4_IO}, \ref{tab:varyinggrid_constcores3_IO} and Repast in Table \ref{tab:varyinggrid_constcores_repast}. Again, note that the RePast experiments all ran on a single (1) core and were conducted to have a rough estimate where the functional approach is in comparison to the imperative.

\begin{table}
	\centering
  	\begin{tabular}{ c || c }
        Grid-Size          & Duration                \\ \hline \hline 
   		51 x 51 (2,601)    & 20.201          (0.789) \\ \hline
   		101 x 101 (1,0201) & \textbf{74.493} (0.524) \\ \hline
   		151 x 151 (22,801) & \textbf{168.47} (1.783) \\ \hline
   		201 x 201 (40,401) & \textbf{302.43} (3.931) \\ \hline
   		251 x 251 (63,001) & \textbf{495.73} 0
  	\end{tabular}

  	\caption{STM Monad experiments on varying grid sizes on 4 cores.}
	\label{tab:varyinggrid_constcores_stm}
\end{table}


\begin{table}
	\centering
  	\begin{tabular}{ c || c }
        Grid-Size          & Duration 		\\ \hline \hline 
   		51 x 51 (2,601)    & 41.914 (1.073) \\ \hline
   		101 x 101 (10,201) & 170.55 (1.115) \\ \hline
   		151 x 151 (22,801) & 376.89 0 		\\ \hline
   		201 x 201 (40,401) & 672.01 0 		\\ \hline 
   		251 x 251 (63,001) & 1,027.27 0
  	\end{tabular}
  	
  	\caption{Lock-Based experiments on varying grid sizes on 4 cores.}
	\label{tab:varyinggrid_constcores4_IO}
\end{table}

\begin{table}
	\centering
  	\begin{tabular}{ c || c  }
        Grid-Size         & Duration  		\\ \hline \hline 
   		51 x 51   (2,601)  & 38.614 (0.397) \\ \hline
   		101 x 101 (1,0201) & 171.61 (3.016) \\ \hline
   		151 x 151 (22,801) & 404.11	0 		\\ \hline
   		201 x 201 (40,401) & 720.65 0 		\\ \hline 
   		251 x 251 (63,001) & 1,117.27 0 
  	\end{tabular}
  	
  	\caption{Lock-Based experiments on varying grid sizes on 3 cores.}
	\label{tab:varyinggrid_constcores3_IO}
\end{table}

\begin{table}
	\centering
  	\begin{tabular}{ c || c }
        Grid-Size          & Duration  				 \\ \hline \hline 
   		51 x 51 (2,601)    & \textbf{10.822} (0.377) \\ \hline
   		101 x 101 (10,201) & 107.40 (1.306) 		 \\ \hline
   		151 x 151 (22,801) & 464.017  0 			 \\ \hline
   		201 x 201 (40,401) & 1,227.68 0 			 \\ \hline 
   		251 x 251 (63,001) & 3,283.63 0 
  	\end{tabular}
  	
  	\caption{Repast experiments on varying grid sizes on a single (1) core.}
	\label{tab:varyinggrid_constcores_repast}
\end{table}

We plotted the results in Figure \ref{fig:stm_io_repast_varyinggrid_performance}. It is clear that the lock-free STM implementation outperforms the lock-based Lock-Based implementation by a substantial factor. Surprisingly, the Lock-Based implementation on 4 core scales just slightly better with increasing agents number than on 3 cores, something we wouldn't have anticipated based on the results seen in Table \ref{tab:constgrid_varyingcores}. Also  while on a 51x51 grid the single (1) core Java RePast version outperforms the 4 core Haskell STM version by about 200\%. The figure is inverted on a 251x251 grid where the 4 core Haskell STM version outperforms the single core Java Repast version by over 600\%. This might not be entirely surprising because we compare single (1) core against multi-core performance - still the scaling is indeed impressive and we would never have anticipated an increase of over 600\%.

\begin{figure}
\begin{center}
	\begin{tabular}{c c}
		\begin{subfigure}[b]{0.5\textwidth}
			\centering
			\includegraphics[width=1\textwidth, angle=0]{./fig/sir/stm_io_repast_varyinggrid_performance.png}
			\caption{Normal Scale}
		\end{subfigure}
    	&
		\begin{subfigure}[b]{0.5\textwidth}
			\centering
			\includegraphics[width=1\textwidth, angle=0]{./fig/sir/stm_io_repast_varyinggrid_performance_loglog.png}
			\caption{Logarithmic scale on both axes}
		\end{subfigure}
    \end{tabular}
	\caption{Comparison of STM (Table \ref{tab:varyinggrid_constcores_stm}), Lock-Based (Table \ref{tab:varyinggrid_constcores4_IO}, Table 					\ref{tab:varyinggrid_constcores3_IO}) and RePast (single core) (Table \ref{tab:varyinggrid_constcores_repast}) performance. TODO: re-create the figure when all experiments had 8 runs.}
	\label{fig:stm_io_repast_varyinggrid_performance}
\end{center}
\end{figure}

%\begin{figure}
%	\centering
%	\includegraphics[width=0.6\textwidth, angle=0]{./fig/sir/stm_io_repast_varyinggrid_performance.png}
%	\caption{Comparison of STM (Table \ref{tab:varyinggrid_constcores_stm}), Lock-Based (Table \ref{tab:tab:varyinggrid_constcores4_IO}, Table \ref{tab:varyinggrid_constcores3_IO}) and RePast (single core) (Table \ref{tab:varyinggrid_constcores_repast}) performance. TODO: re-create the figure when all experiments had 8 runs.}
%	\label{fig:stm_io_repast_varyinggrid_performance}
%\end{figure}

\subsection{Retries}
Of very much interest when using STM is the retry-ratio, which obviously depends highly on the read-write patterns of the respective model. We used the stm-stats library to record statistics of commits, retries and the ratio. In these experiments we only averaged over 4 runs because they all arrived at a ratio of 0.0. The results are reported in Table \ref{tab:retries_stm}.

\begin{table}
	\centering
  	\begin{tabular}{ c || c | c | c }
        Grid-Size 		   & Commits    & Retries         & Ratio \\ \hline \hline 
   		51 x 51 (2,601)    & 2,601,000  & 1306.5 (83.9)   & 0.0 \\ \hline
   		101 x 101 (10,201) & 10,201,000 & 3712.5 (308.42) & 0.0 \\ \hline
   		151 x 151 (22,801) & 22,801,000 & 8189.5 (342.12) & 0.0 \\ \hline
   		201 x 201 (40,401) & 40,401,000 & 13285 (0.0)     & 0.0 \\ \hline 
   		251 x 251 (63,001) & 63,001,000 & 21217 (0.0)     & 0.0
  	\end{tabular}
  	
  	\caption{Retries Ratio of STM Monad experiments on varying grid sizes on 4 cores.}
	\label{tab:retries_stm}
\end{table}

Independent of the number of agents we always have a retry-ratio of 0.0. This indicates that this model is \textit{very} well suited to STM, which is also directly reflected in the substantial better performance over the Lock-Based implementation. Obviously this ratio stems from the fact, that in our implementation we have \textit{very} few writes (only when an agent changes e.g. from Susceptible to Infected or from Infected to Recovered) and mostly reads. Also we conducted runs on lower number of cores which resulted in fewer retries, which was what we expected.

%\begin{figure}
%	\centering
%	\includegraphics[width=0.6\textwidth, angle=0]{./fig/sir/retries_stm.png}
%	\caption{Scaling of retries by agent count. TODO: re-create the figure when all experiments had 4 runs.}
%	\label{fig:retries_stm}
%\end{figure}

\subsection{Discussion}
Interpretation of the performance data leads to the following insights:
\begin{enumerate}
	%\item On a single core, no transaction retries should happen, the results support that assumption.
	\item Running in STM and sharing state using a transactional variable is much more time- and memory-efficient than running in the State Monad but potentially sacrifices determinism: repeated runs might not lead to same dynamics despite same initial conditions.
	\item Running STM on multiple cores concurrently \textit{does} lead to a significant performance improvement \textit{for that model}.
	\item STM outperforms the Lock-Based implementation substantially and scales much better to multiple cores.
	\item STM on single (1) core is still about twice as slow than an object-oriented Java RePast implementation on a single (1) core.
	\item STM on multiple cores dramatically outperforms the single (1) core object-oriented Java RePast implementation on a single (1) core on instances with large agent numbers and scales much better to increasing number of agents.
\end{enumerate}

\section{Case Study 2: SugarScape}
\label{sec:cs_sugarscape}

One of the first models in Agent-Based Simulation was the seminal Sugarscape model developed by Epstein and Axtell in 1996 \cite{epstein_growing_1996}. Their aim was to \textit{grow} an artificial society by simulation and connect observations in their simulation to phenomenon observed in real-world societies. In this model a population of agents move around in a discrete 2D environment, where sugar grows, and interact with each other and the environment in many different ways. The main features of this model are (amongst others): searching, harvesting and consuming of resources, wealth and age distributions, population dynamics under sexual reproduction, cultural processes and transmission, combat and assimilation, bilateral decentralized trading (bartering) between agents with endogenous demand and supply, disease processes transmission and immunology.

We implemented the \textit{Carrying Capacity} (p. 30) section of Chapter II of the book \cite{epstein_growing_1996}. There, in each step agents search (move) to the cell with the most sugar they see within their vision, harvest all of it from the environment and consume sugar based on their metabolism. Sugar regrows in the environment over time. Only one agent can occupy a cell at a time. Agents don't age and cannot die from age. If agents run out of sugar due to their metabolism, they die from starvation and are removed from the simulation. The authors report that the initial number of agents quickly drops and stabilises around a level depending on the model parameters. This is in accordance with our results as we show in Figure \ref{fig:vis_sugarscape} and guarantees that we don't run out of agents. The model parameters are as follows:

\begin{itemize}
	\item Sugar Endowment: each agent has an initial sugar endowment randomly uniform distributed between 5 and 25 units;
	\item Sugar Metabolism: each agent has a sugar metabolism randomly uniform distributed between 1 and 5;
	\item Agent Vision: each agent has a vision randomly uniform distributed between 1 and 6, same for each of the 4 directions (N, W, S, E);
	\item Sugar Growback: sugar grows back by 1 unit per step until the maximum capacity of a cell is reached;
	\item Agent Number: initially 500 agents;
	\item Environment Size: 50 x 50 cells with toroid boundaries which wrap around in both x and y dimension.
\end{itemize}

\subsection{Experiment Design}
In this case study we compare the performance of four (4) implementations under varying numbers of CPU cores and agent numbers. The code of all implementations can be accessed freely from the \href{https://github.com/thalerjonathan/haskell-stm-sugarscape}{code repository}~\cite{thaler_stm_sugarscape_repository}.

\begin{enumerate}
	\item Sequential - This is the reference implementation, where all agents are run after another (including the environment). The environment is represented using an indexed array \cite{array_hackage} and shared amongst the agents using a read and write state context. 
	
	\item Lock-Based - This is the same implementation as \textit{Sequential}, but all agents are run concurrently within the \texttt{IO} context. The environment is also represented as an indexed array but shared using a global reference between the agents which acquire and release a lock when accessing it. Note that the semantics of Sugarscape do not support the implementation of either a read-write lock or atomic modification approach as in the SIR model. In the SIR model, the agents write conditionally to \textit{their own} cell, but this is not the case in Sugarscape, where agents need a consistent view of the whole environment for the whole duration of an agent execution due to the fact that agents do not only write their own locations but also to other locations. If this is not handled correctly, data races are happening and threads overwrite data from other threads, ultimately resulting in incorrect dynamics.
	
	\item STM TVar - This is the same implementation as \textit{Sequential}, but all agents are run concurrently within the \texttt{STM} context. The environment is also represented as an indexed array but shared using a \texttt{TVar} between the agents.
	
	\item STM TArray - This is the same implementation as \textit{Sequential}, but all agents are run concurrently within the \texttt{STM} context. The environment is represented and shared between the agents using a \texttt{TArray}. 
\end{enumerate}

\paragraph{Ordering} The model specification requires to shuffle agents before every step (\cite{epstein_growing_1996}, footnote 12 on page 26). In the \textit{Sequential} approach we do this explicitly but in the \textit{Lock-Based} and both \textit{STM} approaches we assume this to happen automatically due to race conditions from concurrency, thus we arrive at an effectively shuffled processing of agents because we implicitly assume that the order of the agents is \textit{effectively} random in every step. The important difference between the two approaches is that in the \textit{Sequential} approach we have full control over this randomness but in the \textit{STM} and \textit{Lock-Based} not. This has the consequence that repeated runs with the same initial conditions might lead to slightly different results. 
This decision leaves the execution order of the agents ultimately to Haskell's Runtime System and the underlying operating system. We are aware that by doing this, we make assumptions that the threads run uniformly distributed (fair) but such assumptions should not be made in concurrent programming. As a result we can expect this fact to produces non-uniform distributions of agent runs but we assumed that for this model this does not have a significance influence. In case of doubt, we could resort to shuffling the agents before running them in every step. This problem, where also the influence of nondeterministic ordering on the correctness and results of ABS has to be analysed, deserves in-depth research on its own and is therefore beyond the focus of this paper. As a potential direction for such an investigation, we refer to the technique of property-based testing as shown in \cite{thaler_show_2019}.

Note that in the concurrent implementations we have two options for running the environment: either asynchronously as a concurrent agent at the same time with the population agents or synchronously after all agents have run. We must be careful though as running the environment as a concurrent agent can be seen as conceptually wrong because the time when the regrowth of the sugar happens is now completely random. In this case it could happen that sugar regrows in the very first transaction or in the very last, different in each step, which can be seen as a violation of the model specifications. Thus we do not run the environment concurrently with the agents but synchronously after all agents have run.

\medskip

The experiment setup is the same as in the SIR case study, with the same hardware (see Table \ref{tab:machine_specs}), with measurements done under no additional workload using the microbenchmarking library Criterion \cite{criterion_serpentine, criterion_hackage} as well. However, as the Sugarscape model is stepped using natural numbers we ran each measurement until $t = 1000$ and stepped it using $\Delta t = 1$. In the experiments we varied the number of agents as well as the number of cores when running concurrently. We checked the visual outputs and the dynamics and they look qualitatively the same as the reference \textit{Sequential}. As in the SIR case study, a rigorous, statistical comparison of all implementations, to investigate the effects of concurrency on the dynamics, is quite involved and therefore beyond the focus of this paper but as a remedy we refer to the use of property-based testing, as shown in \cite{thaler_show_2019}.

\subsection{Constant Agent Size}
In this experiment we compare the performance of all implementations on varying numbers of cores. The results are reported in Table \ref{tab:sugarscape_varyingcores_constagents} and plotted in Figure \ref{fig:sugarscape_varyingcores_constagents}. 

\begin{table}
	\centering
  	\begin{tabular}{ c || c | c | c | c }
        Cores  & Sequential  & Lock-Based            & TVar         & TArray                \\ \hline \hline 
    		1      & 25.2 (0.36) & \textbf{21.0} (0.12)  & 21.1 (0.25)  & 42.0 (2.20)           \\ \hline
   		2      & -           & \textbf{20.0} (0.12)  & 22.2 (0.21)  & 24.5 (1.07)           \\ \hline
   		3      & -           & 21.9 (0.19)           & 23.6 (0.12)  & \textbf{19.7} (1.05)  \\ \hline
   		4      & -           & 24.0 (0.17)           & 25.2 (0.16)  & \textbf{18.9} (0.58)  \\ \hline
   		5      & -           & 26.7 (0.17)           & 31.0 (0.24)  & \textbf{20.3} (0.87)  \\ \hline
   		6      & -           & 29.3 (0.57)           & 35.2 (0.12)  & \textbf{21.2} (1.49)  \\ \hline
   		7      & -           & 30.0 (0.12)           & 38.7 (0.42)  & \textbf{21.0} (0.41)  \\ \hline
   		8      & -           & 31.2 (0.29)           & 49.0 (0.41)  & \textbf{21.1} (0.64)  \\ \hline \hline
   	\end{tabular}
 
  	\caption{Performance comparison of \textit{Sequential}, \textit{Lock-Based}, \textit{TVar} and \textit{TArray} Sugarscape implementations under varying cores with 50x50 environment and 500 initial agents. Timings in seconds (lower is better), standard deviation in parentheses.}
	\label{tab:sugarscape_varyingcores_constagents}
\end{table}

\begin{table}
	\centering
  	\begin{tabular}{ c || c | c }
        Cores & TVar  & TArray  \\ \hline \hline 
    		1     & 0.00  & 0.00    \\ \hline
   		2     & 1.04  & 0.02    \\ \hline
   		3     & 2.15  & 0.04    \\ \hline
   		4     & 3.20  & 0.06    \\ \hline
   		5     & 4.06  & 0.07    \\ \hline
   		6     & 5.02  & 0.09    \\ \hline
   		7     & 6.09  & 0.10    \\ \hline
   		8     & 8.45  & 0.11    \\ \hline \hline
   	\end{tabular}
 
  	\caption{Retry ratio comparison (lower is better) of the \textit{TVar} and \textit{TArray} Sugarscape implementations under varying cores with 50x50 environment and 500 initial agents.}
	\label{tab:sugarscape_retry_ratios}
\end{table}

As expected, the \textit{Sequential} implementation is the slowest, with \textit{TArray} being the fastest one except on 1 and 2 cores, where unexpectedly the \textit{Lock-Based} implementation performed best. Interestingly the \textit{TVar} implementation was the worst performing one of the concurrent implementations.

The reason for the bad performance of \textit{TVar} is that using a \texttt{TVar} to share the environment is a very inefficient choice: \textit{every} write to a cell leads to a retry independent whether the reading agent reads that changed cell or not, because the data structure can not distinguish between individual cells. By using a \texttt{TArray} we can avoid the situation where a write to a cell in a far distant location of the environment will lead to a retry of an agent which never even touched that cell. The inefficiency of \textit{TVar} is also reflected in the fact that the \textit{Lock-Based} implementation outperforms it on all cores. The sweet spot is in both cases at 3 cores, after which decreasing performance is the result. This is due to very similar approaches because both operate on the whole environment instead of only the cells as \textit{TArray} does. In case of the \textit{Lock-Based} approach, the lock contention increases, whereas in the \textit{TVar} approach, the retries start to dominate (see Table \ref{tab:sugarscape_retry_ratios}).

Interestingly, the performance of the \textit{TArray} implementation is the \textit{worst} amongst all on 1 core. We attribute this to the overhead incurred by STM, which dramatically adds up in terms of a sequential execution.

\subsection{Scaling up Agents}
So far we kept the initial number of agents at 500, which due to the model specification, quickly drops and stabilises around 200 due to the carrying capacity of the environment as can be seen in Figure \ref{fig:vis_sugarscape_t50_dynamics} and which is also described in the book \cite{epstein_growing_1996} section \textit{Carrying Capacity} (p. 30).

We now measure the performance of our approaches under increased number of agents. For this we slightly change the implementation: always when an agent dies it spawns a new one which is inspired by the ageing and birthing feature of Chapter III in the book \cite{epstein_growing_1996}. This ensures that we keep the number of agents roughly constant (still fluctuates but doesn't drop to low levels) over the whole duration. This ensures a constant load of concurrent agents interacting with each other and demonstrates also the ability to terminate and fork threads dynamically during the simulation.

Except for the \textit{Sequential} approach we ran all experiments with 4 cores. We looked into the performance of 500, 1,000, 1,500, 2,000 and 2,500 (maximum possible capacity of the 50x50 environment). The results are reported in Table \ref{tab:sugarscape_varyingagents_constcores} and plotted in Figure \ref{fig:sugarscape_varyingagents_constcores}.

\begin{table}
	\centering
  	\begin{tabular}{ c || c | c | c | c }
        Agents  & Sequential    & Lock-Based    & TVar          & TArray                \\ \hline \hline 
    	    500     & 70.1 (0.41)   & 67.9 (0.13)   & 69.1 (0.34)   & \textbf{25.7} (0.42)  \\ \hline
   		1,000   & 145.0 (0.11)  & 130.0 (0.28)  & 136.0 (0.16)  & \textbf{38.8} (1.43)  \\ \hline
   		1,500   & 220.0 (0.14)  & 183.0 (0.83)  & 192.0 (0.73)  & \textbf{40.1} (0.25)  \\ \hline
   		2,000   & 213.0 (0.69)  & 181.0 (0.84)  & 214.0 (0.53)  & \textbf{49.9} (0.82)  \\ \hline
   		2,500   & 193.0 (0.16)  & 272.0 (0.81)  & 147.0 (0.32)  & \textbf{55.2} (1.04)  \\ \hline \hline
   	\end{tabular}
  	
  	\caption{Performance comparison of \textit{Sequential}, \textit{Lock-Based}, \textit{TVar} and \textit{TArray} Sugarscape implementations with varying agent numbers and 50x50 environment on 4 cores (except \textit{Sequential}). Timings in seconds (lower is better), standard deviation in parentheses.}
	\label{tab:sugarscape_varyingagents_constcores}
\end{table}

As expected, the \textit{TArray} implementation outperforms all others substantially and scales up much smother. Also, \textit{Lock-Based} performs better than the \textit{TVar}.

What seems to be very surprising is that in the \textit{Sequential} and \textit{TVar} cases the performance with 2,500 agents is \textit{better} than the one with 2,000 agents. The reason for this is that in the case of 2,500 agents, an agent can't move anywhere because all cells are already occupied. In this case the agent won't rank the cells in order of their payoff (max sugar) to move to but just stays where it is. We hypothesize that due to Haskells laziness the agents actually never look at the content of the cells in this case but only the number which means that the cells themselves are never evaluated which further increases performance. This leads to the better performance in case of \textit{Sequential} and \textit{TVar} because both exploit laziness. 
In the case of the \textit{Lock-Based} approach we still arrive at a lower performance because the limiting factor are the unconditional locks. In the case of the \textit{TArray} approach we also arrive at a lower performance because it seems that STM perform reads on the neighbouring cells which are not subject to lazy evaluation.

In case of the \textit{Sequential} implementation with 2,000 agents we also arrive at a better performance than with 1,500, due to less space of the agents for free movement, exploiting laziness as in the case with 2,500 agents. In the case of the \textit{Lock-Based} approach we see similar behaviour, where the performance with 2,000 agents is better than with 1,500. It is not quite clear why this is the case, given the dramatically \textit{lower} performance with 2,500 agents but it seems that 2,000 agents create much less lock contention due to lower free space, whereas 2,500 agents create a lot more lock contention due to no free space available at all.

We also measured the average retries both for \textit{TVar} and \textit{TArray} under 2,500 agents where the \textit{TArray} approach shows best scaling performance with 0.01 retries whereas \textit{TVar} averages at 3.28 retries. Again this can be attributed to the better transactional data structure which reduces retry ratio substantially to near-zero levels.

\subsection{Going Large-Scale}
To test how far we can scale up the number of cores in the \textit{TArray} case, we ran the two experiments, carrying capacity (500 agents) and rebirthing (2500 agents), on an Amazon EC \texttt{m5ad.16xlarge} instance with 16, 32 and 64 cores to see if we run into decreasing returns. The results are reported in Table \ref{tab:sug_varying_cores_amazon}.

\begin{table}
	\centering
  	\begin{tabular}{ c || c | c }
        Cores  & Carrying Capacity  & Rebirthing    \\ \hline \hline 
    	    16     & 11.9 (0.21)        & 46.6 (0.07)   \\ \hline
   		32     & 12.8 (0.29)        & 76.4 (0.01)   \\ \hline
   		64     & 14.6 (0.09)        & 99.1 (0.01)   \\ \hline \hline
   	\end{tabular}

  	\caption{Sugarscape \textit{TArray} performance on 16, 32 and 64 cores an Amazon EC \texttt{m5ad.16xlarge} instance. Timings in seconds (lower is better). Retry ratios in parentheses.}
	\label{tab:sug_varying_cores_amazon}
\end{table}

Unlike in the SIR model, Sugarscapes STM \textit{TArray} implementation does not scale up beyond 16 cores. We attribute this to a mix of retries and Amdahl's law. As retries are much more expensive in the case of Sugarscape compared to SIR, even a small increase in the retry ratio (see Table \ref{tab:sugarscape_retry_ratios}), leads to reduced performance. On the other hand, although the retry ratio decreases as the number of cores increases, the ratio of parallelisable work diminishes and we get bound by the sequential part of the program.

\subsection{Comparison with other approaches}
The paper \cite{lysenko_framework_2008} reports a performance of 2,000 steps per second on a GPU on a 128x128 grid. Our best performing implementation, \textit{TArray} with 500 rebirthing agents, arrives at a performance of 39 steps per second and is therefore clearly slower. However, the very high performance on the GPU does not concern us here as it follows a very different approach than ours. We focus on speeding up implementations on the CPU as directly as possible without locking overhead. When following a GPU approach one needs to map the model to the GPU which is a delicate and non-trivial matter. With our approach we show that speed up with concurrency is very possible without the low-level locking details or the need to map to GPU. Also some features like bilateral trading between agents, where a pair of agents needs to come to a conclusion over multiple synchronous steps, is difficult to implement on a GPU whereas this should be not as hard using STM.

Note that we kept the grid size constant because we implemented the environment as a single agent which works sequentially on the cells to regrow the sugar. Obviously this doesn't really scale up on parallel hardware and experiments which we haven't included here due to lack of space, show that the performance goes down dramatically when we increase the environment to 128x128 with same number of agents. This is the result of Amdahl's law where the environment becomes the limiting \textit{sequential} factor of the simulation. Depending on the underlying data structure used for the environment we have two options to solve this problem. In the case of the \textit{Sequential} and \textit{TVar} implementation we build on an indexed array, which can be updated in parallel using the existing data-parallel support in Haskell. In the case of the \textit{TArray} approach we have no option but to run the update of every cell within its own thread. We leave both for further research as it is beyond the scope of this paper.

\subsection{Summary}
This case study showed clearly that besides being substantially faster than the \textit{Sequential} implementation, an \textit{STM} implementation with the right transactional data structure is also able to perform considerably better than a \textit{Lock-Based} approach even in the case of the Sugarscape model which has a much higher complexity in terms of agent behaviour and dramatically increased number of writes to the environment.

Further, this case study demonstrated that the selection of the right transactional data structure is of fundamental importance when using STM. Selecting the right transactional data structure is highly model-specific and can lead to dramatically different performance results. In this case study the \textit{TArray} performed best due to many writes but in the SIR case study a \textit{TVar} showed good enough results due to the very low number of writes. When not carefully selecting the right transactional data structure, which supports fine-grained concurrency, a lock-based implementation might perform as well or even outperform the STM approach as can be seen when using the \texttt{TVar}.

Although the \texttt{TArray} is the better transactional data structure overall, it might come with an overhead, performing worse on low number of cores than a \textit{TVar}, \textit{Lock-Based} or even \textit{Sequential} approach, as seen with \textit{TArray} on 1 core. However, it has the benefit of quickly scaling up to multiple cores. Depending on the transactional data structure, scaling up to multiple cores hits a limit at some point. In the case of the \texttt{TVar} the best performance is reached with 3 cores. With the \texttt{TArray} we reached this limit around 16 cores.

The comparison between the \textit{Lock-Based} approach and the \textit{TArray} implementation seems to be a bit unfair due to a very different locking structure. A more suitable comparison would be to use an indexed Array with a tuple of \texttt{(MVar, IORef)}, holding a synchronisation primitive and reference for each cell to support fine-grained locking on the cell level. This would seem to be a more just comparison to the \textit{TArray} where fine-grained transactions happen on the cell level. However, due to the model semantics, this approach is actually not possible. As already expressed in the experiments description, in Sugarscape an agent needs a consistent view of the whole environment for the whole duration of an agent execution due to the fact that agents don't only write their own locations but change also other locations. If we would use an indexed array we would also run into data races because the agents need to hold all relevant cells, which can't be grabbed in one atomic instruction but only one after another, which makes it highly susceptible to data races. Therefore, we could run into deadlocks if two agents are acquiring locks because they are taken after another and therefore subject to races where they end up holding a lock the other needs.

\chapter{Conclusions}
\label{ch:conclusions}

This chapter concludes the whole thesis and outlines future research. Roughly 20\% exists already.

%we now know how to engineer time- and event-driven ABS with complex state both in the agent and environment, main difficulty is direct agent-interaction (see macal classification into 4 types of ABS), compile-time guaranteed reproducibility, explicit handling of complex state (read only, read/write), concurrency explicit and limited to STM, very promising concurrency but direct agent-interactions main problem (erlang as a rescue?), main drawbacks: everything is explicit, performance

\section{Further Research}
clearly outline the ideas for further research

\subsection{A general purpose library}
generalise concepts explored into a pure functional ABS library in Haskell (called chimera)

\subsection{Dependent and linear types}
dependent types and linear types are the next big step, towards a stronger formalisation of agents and ABS,
focus on the equilibrium - totality correspondence

\subsection{Concurrent event-driven ABS}
stm based concurrency for event-driven ABS using parallel DES. challenge is the time-warp implementation using monads. in general it should be easy to roll-back agents actions but with monads we have to be careful - for some monads rolling back is not neccessary e.g. rand and reader, for others it is, and for some it is impossible e.g. IO

\section{Further Research}
\label{sec:further_research}

We see this paper as an intermediary and necessary step towards dependent types for which we first needed to understand the potentials and limitations of a non-dependently typed pure functional approach in Haskell. Dependent types are extremely promising in functional programming as they allow us to express stronger guarantees about the correctness of programs and go as far as allowing to formulate programs and types as constructive proofs \cite{wadler_propositions_2015} which must be total by definition \cite{thompson_type_1991}, \cite{altenkirch_why_2005}, \cite{altenkirch_pi_2010}, \cite{program_homotopy_2013}. So far no research using dependent types in agent-based simulation exists at all and it is not clear whether dependent types make sense in this context. In our next paper we want to explore this for the first time and ask more specifically how we can add dependent types to our pure functional approach, which conceptual implications this has for ABS and what we gain from doing so. We plan on using Idris \cite{brady_idris_2013}, \cite{brady_type-driven_2017} as the language of choice as it is very close to Haskell with focus on real-world application and running programs as opposed to other languages with dependent types e.g. Agda and Coq which serve primarily as proof assistants.
It would be of immense interest whether we could apply dependent types to the model meta-level or not - this boils down to the question if we can encode our model specification in a dependent type way. This would allow the ABS community for the first time to reason about a proper formalisation of a model. We plan to implement a total and terminating implementation of our approach which would be a formal proof-by-construction that the agent-based approach of the SIR model terminates after a finite number of steps.


%%%%%%%%%%%%%%%%%%%%%%%%%%%%%%%%%%%%%%%%%%%%%%
%%                                          %%
%% Backmatter begins here                   %%
%%                                          %%
%%%%%%%%%%%%%%%%%%%%%%%%%%%%%%%%%%%%%%%%%%%%%%

\begin{backmatter}

%\section*{Abbreviations}
%ABS - Agent Based Simulation
%STM - Software Transactional Memory
%PDES - Parallel Discrete Event Simulation

\section*{Declarations}
\section*{Availability of data and materials}
The datasets used and/or analysed during the current study are available from the corresponding author on reasonable request.

\section*{Competing interests}
The authors declare that they have no competing interests.

\section*{Funding}
Not applicable.

\section*{Authors' contributions}
JT initiated the idea and the research, did the implementation, experiments, performance measurements, and writing. POS supervised the work, gave feedback and supported the writing process. All authors read and approved the final manuscript.

\section*{Acknowledgements}
The authors would like to thank J. Hey and M. Handley for constructive feedback, comments and valuable discussions.

\section*{Authors' information}
\textbf{\uppercase{JONATHAN THALER}} is a Ph.D. student at the University of Nottingham and part of the Intelligent Modelling and Analysis Group (\url{http://www.cs.nott.ac.uk/~psxjat/}). His main research interest is the benefits and drawbacks of using pure functional programming with Haskell for implementing Agent-Based Simulations.

\textbf{\uppercase{Dr. PEER-OLAF SIEBERS}} is an Assistant Professor at the School of Computer Science, University of Nottingham, UK (\url{http://www.cs.nott.ac.uk/~pszps/}). His main research interest is the application of computer simulation to study human-centric complex adaptive systems. He is a strong advocate of Object Oriented Agent-Based Social Simulation. This is a novel and highly interdisciplinary research field, involving disciplines like Social Science, Economics, Psychology, Operations Research, Geography, and Computer Science. His current research focuses on Urban Sustainability and he is a co-investigator in several related projects and a member of the university's "Sustainable and Resilient Cities" Research Priority Area management team.

%%%%%%%%%%%%%%%%%%%%%%%%%%%%%%%%%%%%%%%%%%%%%%%%%%%%%%%%%%%%%
%%                  The Bibliography                       %%
%%                                                         %%
%%  Bmc_mathpys.bst  will be used to                       %%
%%  create a .BBL file for submission.                     %%
%%  After submission of the .TEX file,                     %%
%%  you will be prompted to submit your .BBL file.         %%
%%                                                         %%
%%                                                         %%
%%  Note that the displayed Bibliography will not          %%
%%  necessarily be rendered by Latex exactly as specified  %%
%%  in the online Instructions for Authors.                %%
%%                                                         %%
%%%%%%%%%%%%%%%%%%%%%%%%%%%%%%%%%%%%%%%%%%%%%%%%%%%%%%%%%%%%%

% if your bibliography is in bibtex format, use those commands:
\bibliographystyle{vancouver} % Style BST file (bmc-mathphys, vancouver, spbasic).
\bibliography{references}      % Bibliography file (usually '*.bib' )
% for author-year bibliography (bmc-mathphys or spbasic)
% a) write to bib file (bmc-mathphys only)
% @settings{label, options="nameyear"}
% b) uncomment next line
%\nocite{label}

% or include bibliography directly:
% \begin{thebibliography}
% \bibitem{b1}
% \end{thebibliography}

%%%%%%%%%%%%%%%%%%%%%%%%%%%%%%%%%%%
%%                               %%
%% Figures                       %%
%%                               %%
%% NB: this is for captions and  %%
%% Titles. All graphics must be  %%
%% submitted separately and NOT  %%
%% included in the Tex document  %%
%%                               %%
%%%%%%%%%%%%%%%%%%%%%%%%%%%%%%%%%%%

%%
%% Do not use \listoffigures as most will included as separate files

\section*{Figures}
\begin{figure}[h!]
	\includegraphics[width=0.6\textwidth, angle=0]{./stm_abs.png}
	\caption{Diagram of the parallel time-driven lock-step approach.}
	\label{fig:stm_abs_structure}
\end{figure}

\begin{figure}[h!]
	\includegraphics[width=0.4\textwidth, angle=0]{./sir_vis.png}
	\caption{Simulating the spatial SIR model with a Moore neighbourhood, a single infected agent at the center, contact rate $\beta = \frac{1}{5}$, infection probability $\gamma = 0.05$ and illness duration $\delta = 15$ . Infected agents are indicated by red circles, recovered agents by green ones. The susceptible agents are rendered as blue hollow circles for better contrast.}
	\label{fig:vis_sir}
\end{figure}

\begin{figure}[h!]
	\includegraphics[width=0.6\textwidth, angle=0]{./sir_core_duration_stm_io.png}
	\caption{Performance and scaling comparison of \textit{STM} and \textit{Lock-Based} SIR implementations on multiple cores. Note that the Lock-Based implementation performs worse on 4 than on 3 cores due to lock contention.}
	\label{fig:core_duration_stm_io}
\end{figure}

\begin{figure}[h!]
	\centering
	\includegraphics[width=0.7\textwidth, angle=0]{./sir_stm_io_varyinggrid_performance.png}
	\caption{Performance comparison of \textit{STM} and \textit{Lock-Based} SIR implementations on varying grid sizes.}
	\label{fig:varyinggrid_constcores}
\end{figure}

\begin{figure}[h!]
\begin{center}
	\begin{tabular}{c c}
		\begin{subfigure}[b]{0.4\textwidth}
			\centering
			\includegraphics[width=1\textwidth, angle=0]{./sugarscape_t60_environment.png}
			\caption{Visualisation of the Sugarscape at $t = 50$}
			\label{fig:vis_sugarscape_t50_environment}
		\end{subfigure}
    	
    	&
  
		\begin{subfigure}[b]{0.6\textwidth}
			\centering
			\includegraphics[width=1\textwidth, angle=0]{./sugarscape_t60_dynamics.png}
			\caption{Dynamics population size over 50 steps}
			\label{fig:vis_sugarscape_t50_dynamics}
		\end{subfigure}
	\end{tabular}
	
	\caption{Visualisation of our SugarScape implementation and dynamics of the population size over 50 steps. The white numbers in the blue agent circles are the agents unique ids.}
	\label{fig:vis_sugarscape}
\end{center}
\end{figure}

\begin{figure}[h!]
	\includegraphics[width=0.7\textwidth, angle=0]{./sug_varying_cores.png}
	\caption{Comparison of steps per seconds (higher is better) and retries of various Sugarscape implementations with constant agent numbers. Using a 50x50 grid and 500 initial agents on varying cores.}
	\label{fig:varying_cores}
\end{figure}

\begin{figure}[h!]
	\includegraphics[width=0.6\textwidth, angle=0]{./sug_varying_agents.png}
	\caption{Steps per second (higher is better) of various Sugarscape implementations with varying agent numbers. Using 50x50 grid and varying number of agents with 4 (and 3) cores except the \textit{Sequential} (1 core) implementation.}
	\label{fig:state_results_agentsscale_time}
\end{figure}


%%%%%%%%%%%%%%%%%%%%%%%%%%%%%%%%%%%
%%                               %%
%% Tables                        %%
%%                               %%
%%%%%%%%%%%%%%%%%%%%%%%%%%%%%%%%%%%

%% Use of \listoftables is discouraged.
%%
%\section*{Tables}
%\begin{table}[h!]
%\caption{Sample table title. This is where the description of the table should go.}
%      \begin{tabular}{cccc}
%        \hline
%           & B1  &B2   & B3\\ \hline
%        A1 & 0.1 & 0.2 & 0.3\\
%        A2 & ... & ..  & .\\
%        A3 & ..  & .   & .\\ \hline
%      \end{tabular}
%\end{table}

%%%%%%%%%%%%%%%%%%%%%%%%%%%%%%%%%%%
%%                               %%
%% Additional Files              %%
%%                               %%
%%%%%%%%%%%%%%%%%%%%%%%%%%%%%%%%%%%


\end{backmatter}
\end{document}