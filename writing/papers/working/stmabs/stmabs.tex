\documentclass[format=acmsmall, review=true, screen=true]{acmart}

\usepackage{booktabs} % For formal tables

\usepackage{float}
\usepackage{graphicx}
\usepackage{subcaption}
\usepackage{ifthen}
\usepackage{minted}
\usepackage{verbatim}
\usepackage{multirow}

% Metadata Information
\acmJournal{TOMACS}
\acmVolume{28}
\acmNumber{4}
\acmArticle{0}
\acmYear{2018}
\acmMonth{10}
\copyrightyear{2018}
%\acmArticleSeq{9}

% Copyright
%\setcopyright{none}
%\setcopyright{acmcopyright}
\setcopyright{acmlicensed}
%\setcopyright{rightsretained}
%\setcopyright{usgov}
%\setcopyright{usgovmixed}
%\setcopyright{cagov}
%\setcopyright{cagovmixed}

% DOI
\acmDOI{0000001.0000001}

% Paper history
\received{October 2018}

% Document starts
\begin{document}

%TODO 
% side-effects: remove hyphen
% remove sequential comparisons of java/GPU/repast
% add property-based testing as method to check that two implementations are the same / concurrency has no influence. cite my own paper from summersim
% use wording Agent-Based Modelling (ABM) instead of ABS
% amazon EC not S2!!

\newminted[HaskellCode]{haskell}{fontsize=\footnotesize}

\title[A Tale Of Lock-Free Agents]{A Tale Of Lock-Free Agents}
\subtitle{Towards Software Transactional Memory in parallel Agent-Based Simulation}

\author{Jonathan Thaler}
\orcid{https://orcid.org/0000-0001-8736-0479}
\email{jonathan.thaler@nottingham.ac.uk}
\author{Peer-Olaf Siebers}
\email{peer-olaf.siebers@nottingham.ac.uk}
\affiliation{
  \institution{University of Nottingham}
  \streetaddress{7301 Wollaton Rd}
  \city{Nottingham}
  \postcode{NG8 1BB}
  \country{United Kingdom}}

\begin{abstract}
With the decline of Moore's law and the ever increasing availability of cheap massively parallel hardware, it becomes more and more important to embrace parallel programming methods to implement Agent-Based Simulations (ABS). This has been acknowledge in the field a while ago and numerous research on distributed parallel ABS exists, focusing primarily on Parallel Discrete Event Simulation as the underlying mechanism. However, these concepts and tools are inherently difficult to master and apply and often overkill in case implementers simply want to parallelise their own, custom agent-based model implementation. However, with the established programming languages in the field, Python, Java and C++, it is not easy to address the complexities of parallel programming due to unrestricted side effects and the intricacies of low-level locking semantics. Therefore, in this paper we propose the use of a lock-free approach to parallel ABS using Software Transactional Memory (STM) in conjunction with the pure functional programming language Haskell, which in combination, removes some of the problems and complexities of parallel implementations in imperative approaches.

We present two case studies where we compare the performance of lock-based and lock-free STM implementations in two different well known Agent-Based Models, where we investigate both the scaling performance under increasing number of CPUs and the scaling performance under increasing number of agents. We show that the lock-free STM implementations consistently outperform the lock-based ones and scale much better to increasing number of CPUs both on local machines and on Amazon Cloud Services. Further, by utilizing the pure functional language Haskell we gain the benefits of immutable data and lack of unrestricted side effects guaranteed at compile-time, making validation easier and leading to increased confidence in the correctness of an implementation, something of fundamental importance and benefit in parallel programming in general and scientific computing like ABS in particular.
\end{abstract}

%
% The code below should be generated by the tool at
% http://dl.acm.org/ccs.cfm
% Please copy and paste the code instead of the example below.
%
\begin{CCSXML}
<ccs2012>
<concept>
<concept_id>10010147.10010341.10010349.10010355</concept_id>
<concept_desc>Computing methodologies~Agent / discrete models</concept_desc>
<concept_significance>500</concept_significance>
</concept>
<concept>
<concept_id>10010147.10010341.10010349.10010362</concept_id>
<concept_desc>Computing methodologies~Massively parallel and high-performance simulations</concept_desc>
<concept_significance>500</concept_significance>
</concept>
<concept>
<concept_id>10010147.10010341.10010366.10010368</concept_id>
<concept_desc>Computing methodologies~Simulation languages</concept_desc>
<concept_significance>300</concept_significance>
</concept>
</ccs2012>
\end{CCSXML}

\ccsdesc[500]{Computing methodologies~Agent / discrete models}
\ccsdesc[500]{Computing methodologies~Massively parallel and high-performance simulations}
\ccsdesc[300]{Computing methodologies~Simulation languages}

\keywords{Agent-Based Simulation, Software Transactional Memory, Parallel Programming, Concurrency, Functional Programming, Haskell}

\maketitle

%*******************************************************************************
%*********************************** First Chapter *****************************
%*******************************************************************************

\chapter{Introduction}  %Title of the First Chapter
I noticed that it is pretty hard to convince an agent-based economics specialist who is not a computer scientist about a pure functional approach. My conjecture is that the implementation technique and method does not matter much to them because they have very little knowledge about programming and are almost always self-taught - they don't know about software-engineering, nothing about proper software-design and architecture, nothing about software-maintenance, nothing about unit-testing,... In the end they just "hack" the simulation in whatever language they are able to: C++, Visual Basic, Java or toolboxes like Netlogo. For them it is all about to \textit{get things done somehow} and not to get things done the right way or in a beautiful way - the way and the method doesn't matter, its just a necessary evil which needs to be done. Thus if functional programming could make their lives easier, then they will definitely welcome it. But functional programming is, i think, harder to learn and harder to understand - so one needs to provide an abstraction through EDSL. So I REALLY need to come up with convincing arguments why to use pure functional approaches in ACE THEY can understand, otherwise I will be lost and not heard (not published,...). \\

What ACE economists care for:

\begin{itemize}
\item Very: Qualitative modelling with quantitative results
\item Yes: Easy reproducibility
\item Likely: Reasoning about convergence?
\item Likely: EDSL
\end{itemize}

My contributions are: pure functional framework, functional agent-model for market-simulations, EDSL for market-simulations, qualitative / implicit modelling with quanitative results, reasoning in my framework about convergence \\

IDEA: could I develop non-causal modelling (models are expressed in terms of non-directed equations, modelled in signal-relations) to allow for qualitative modelling for the agent-based economists? See hybrid modelling paper of Yampa. \textbf{THIS WOULD BE A HUGE NOVEL CONTRIBUTION TO ACE ESPECIALLY WHEN COMBINED WITH AN EDSL AND PROVIDING FULL REFERENTIAL TRANSPARENCY TO KEEP THE ABILITY TO REASON ABOUT CONVERGENCE}. This should be covered in the "EDSL"-paper.

TODO: maybe i should really focus only on market models? otherwise too much? \\

central novelty of my PhD: model specification = runnable code. possible through EDSL. but only in specific subfield of ACE: market-models. need a functional description of the model, then translate it to model specification in EDSL and then run it to see dynamics. But: model specification moves closer to functional programming languages. \\

another novelty approach: model specification through qualitative instead of quantiative approaches. is this possible? \\

WHY FUNCTIONAL? "because its the ultimate approach to scientific computing": fewer bugs due to mutable state (why? is thos shown obkectively by someone?), shorter (again as above, productivity), more expressive and closer to math, EDSL, EDSL=model=simulation, better parallelising due to referental transparency, reasoning \\

scientific results need to be reproduced, especially when they have high impact. a more formal approach of specifying the model and the simulation (model=simulation) could lead to easier sharing and easier reporduction without ambigouites \\

pure functional agent-model \& theory, EDSL framework in Haskell for ACE

\begin{enumerate}
\item Which kind of problem do we have?
\item What aim is there? Solving the problem? 
\item How the aim is achieved by enumerating VERY CLEAR objectives.
\item What the impact one expects (hypothesis) and what it is (after results).
\end{enumerate}

Note: It is not in the interest of the researcher to develop new economic theories but to research the use of functional methods (programming and specification) in agent-based computational economics (ACE).

NOTE: Get the reader’s attention early in the introduction: motivation, significance, originality and novelty.

\section{Methods}
Methods need to be selected to implement the simulations. Special emphasis will be put on functional ones which will then be compared to established methods in the field of ABM/S and ACE. \\

Claim: non-programming environments are considered to be not powerful enough to capture the complexity of ACE implementations thus a programming approach to ACE will be always required.

\section{Scenarios}
To apply and test functional methods in ACE, four scenarios of ACE are selected and then the methods applied and compared with each other to see how each of them perform in comparison. The 4 selected scenarios represent a selection of the challenges posed in ACE: from very abstract ones to very operational ones.

\section{Comparison}
Each of the selected scenarios is then implemented using the selected methods where each solution is then compared against the following criteria: 

\begin{enumerate}
\item suitability for scientific computation
\item robustness
\item error-sources
\item testability
\item stability
\item extendability
\item size of code
\item maintainability
\item time taken for development
\item verification \& correctness
\item replications \& parallelism
\item EDSL
\end{enumerate}

This will then allow to compare the different methods against each other and to show under which circumstances functional methods shine and when they should not be used.

\section{Agent-Based Modelling and Simulation (ABM/S)}
ABM/S is a method of modelling and simulating a system where the global behaviour may be unknown but the behaviour and interactions of the parts making up the system is of knowledge (Wooldrige, M. (2009). An Introduction to MultiAgent Systems. John Wiley & Sons). Those parts, called agents, are modelled and simulated out of which then the aggregate global behaviour of the whole system emerges. Thus the central aspect of ABM/S is the concept of an Agent which can be understood as a metaphor for a pro-active unit, able to spawn new Agents, and interacting with other Agents in a network of neighbours by exchange of messages. The implementation of Agents can vary and strongly depends on the programming language and the kind of domain the simulation and model is situated in.

\section{Agent-Based Economics (ACE)}
According to Leigh Tesfatsion (Tesfatsion, L. (2006). Agent-based computational economics: A constructive approach to economic theory. In Tesfatsion, L. and Judd, K. L., editors, Handbook of Computational Economics, volume 2, chapter 16, pages 831–880. Elsevier, 1 edition.), one of the leading figures, ACE is "[...] computational modelling of economic processes (including whole economies) as open-ended dynamic systems of interacting agents." - thus lending perfectly to the use of ABM/S as already the name suggests. Whereas classical economic models fall short by only looking at the average, pure rational, individual interacting in anonymous markets, the ACE approach looks at heterogeneous, non-rational individuals interacting with each other in networks (Kirman, A. (2010). Complex Economics: Individual and Collective Rationality. Routledge, London ; New York, NY.). Thus ACE can be understood as a combination of computer-science, cognitive/social science and evolutionary economics.

\section{Functional programming}
TODO: read \cite{Backus1978}

The state-of-the-art approach to implementing Agents are object-oriented methods and programming as the metaphor of an Agent as presented above lends itself very naturally to object-orientation (OO). The author of this thesis claims that OO in the hands of inexperienced or ignorant programmers is dangerous, leading to bugs and hardly maintainable and extensible code. The reason for this is that OO provides very powerful techniques of organising and structuring programs through Classes, Type Hierarchies and Objects, which, when misused, lead to the above mentioned problems. Also major problems, which experts face as well as beginners are 1. state is highly scattered across the program which disguises the flow of data in complex simulations and 2. objects don’t compose as well as functions. The reason for this is that objects always carry around some internal state which makes it obviously much more complicated as complex dependencies can be introduced according to the internal state.
All this is tackled by (pure) functional programming which abandons the concept of global state, Objects and Classes and makes data-flow explicit. This then allows to reason about correctness, termination and other properties of the program e.g. if a given function exhibits side-effects or not. Other benefits are fewer lines of code, easier maintainability and ultimately fewer bugs thus making functional programming the ideal choice for scientific computing and simulation and thus also for ACE. A very powerful feature of functional programming is Lazy evaluation. It allows to describe infinite data-structures and functions producing an infinite stream of output but which are only computed as currently needed. Thus the decision of how many is decoupled from how to (Hughes, J. (1989). Why functional programming matters. Comput. J., 32(2):98–107.).
The most powerful aspect using pure functional programming however is that it allows the design of embedded domain specific languages (EDSL). In this case one develops and programs primitives e.g. types and functions in a host language (embed) in a way that they can be combined. The combination of these primitives then looks like a language specific to a given domain, in the case of this thesis ACE. The ease of development of EDSLs in pure functional programming is also a proof of the superior extensibility and composability of pure functional languages over OO (Henderson P. (1982). Functional Geometry. Proceedings of the 1982 ACM Symposium on LISP and Functional Programming.).
One of the most compelling example to utilize pure functional programming is the reporting of Hudak (Hudak P., Jones M. (1994). Haskell vs. Ada vs. C++ vs. Awk vs. ... An Experiment in Software Prototyping Productivity. Department of Computer Science, Yale University.)  where in a prototyping contest of DARPA the Haskell prototype was by far the shortest with 85 lines of code. Also the Jury mistook the code as specification because the prototype did actually implement a small EDSL which is a perfect proof how close EDSL can get to and look like a specification.

Functional languages can best be characterized by their way computation works: instead of \textit{how} something is computed, \textit{what} is computed is described. Thus functional programming follows a declarative instead of an imperative style of programming. The key points are:
\begin{itemize}
\item No assignment statements - variables values can never change once given a value.
\item Function calls have no side-effect and will only compute the results - this makes order of execution irrelevant, as due to the lack of side-effects the logical point in \textit{time} when the function is calculated within the program-execution does not matter.
\item higher-order functions
\item lazy evaluation
\item Looping is achieved using recursion, mostly through the use of the general fold or the more specific map.
\item Pattern-matching
\end{itemize}

This alone does not really explain the \textit{real} advantages of functional programming and one must look for better motivations using functional programming languages. One motivation is given in \cite{Hughes1989} which is a great paper explaining to non-functional programmers what the significance of functional programming is and helping functional programmers putting functional languages to maximum use by showing the real power and advantages of functional languages. The main conclusion is that \textit{modularity}, which is the key to successful programming, can be achieved best using higher-order functions and lazy evaluation provided in functional languages like Haskell. \cite{Hughes1989} argues that the ability to divide problems into sub-problems depends on the ability to glue the sub-problems together which depends strongly on the programming-language and \cite{Hughes1989} argues that in this ability functional languages are superior to structured programming.

TODO: comparison of functional and object-oriented programming. My points are:
\begin{itemize}
\item The way state can be changed and treated - distributed over multiple objects - is often very difficult to understand.
\item Inheritance is a dangerous thing if not used with care because inheritance introduces very strong dependencies which cannot be changed during runtime anymore.
\item Objects don't compose very well: \url{http://zeroturnaround.com/rebellabs/why-the-debate-on-object-oriented-vs-functional-programming-is-all-about-composition/}
\item (Nearly) impossible to reason about programs
\end{itemize}

In conclusion the upsides of functional programming as opposed to OO are:
\begin{itemize}
\item Much more explicit flow of data \& control
\item Much better compose-able
\item Much better parallelism
\end{itemize}

\section{Related Research}
Tim Sweeney, CTO of Epic Games gave an invited talk about how "future programming languages could help us write better code" by "supplying stronger typing, reduce run-time failures;  and the need for pervasive concurrency support, both implicit and explicit, to effectively exploit the several forms of parallelism present in games and graphics." \cite{Sweeney2006}. Although the fields of games and agent-based simulations seem to be very different in the end, they have also very important similarities: both are simulations which perform numerical computations and update objects - in games they are called "game-objects" and in abm they are called agents but they are in fact the same thing - in a loop either concurrently or sequential. His key-points were:

\begin{itemize}
\item Dependent types as the remedy of most of the run-time failures.
\item Parallelism for numerical computation: these are pure functional algorithms, operate locally on mutable state. Haskell ST, STRef solution enables encapsulating local heaps and mutability within referentially transparent code.
\item Updating game-objects (agents) concurrently using STM: update all objects concurrently in arbitrary order, with each update wrapped in atomic block - depends on collisions if performance goes up.
\end{itemize}

%\section{Related Work}
\label{sec:rel_work}

In his master thesis \cite{bezirgiannis_improving_2013} the author investigated Haskells' parallel and concurrency features to implement (amongst others) \textit{HLogo}, a Haskell clone of the NetLogo \cite{wilensky_introduction_2015} simulation package, focusing on using STM for a limited form of agent-interactions. \textit{HLogo} is basically a re-implementation of NetLogos API in Haskell where agents run within an unrestricted context (known as \textit{IO}) and thus can also make use of STM functionality. The benchmarks show that this approach does indeed result in a speed-up especially under larger agent-populations. The authors' thesis can be seen as one of the first works on ABS using Haskell. Despite the concurrency and parallel aspect our work share, our approach is rather different: we avoid IO within the agents under all costs and explore the use of STM more on a conceptual level rather than implementing a ABS library and compare our case-studies with lock-based and imperative implementations.

There exists some research \cite{di_stefano_using_2005, varela_modelling_2004, sher_agent-based_2013} using the functional programming language Erlang \cite{armstrong_erlang_2010} to implement concurrent ABS. The language is inspired by the actor model \cite{agha_actors:_1986} and was created in 1986 by Joe Armstrong for Eriksson for developing distributed high reliability software in telecommunications. The actor model can be seen as quite influential to the development of the concept of agents in ABS, which borrowed it from Multi Agent Systems \cite{wooldridge_introduction_2009}. It emphasises message-passing concurrency with share-nothing semantics (no shared state between agents), which maps nicely to functional programming concepts. Erlang implements light-weight processes, which allows to spawn thousands of them without heavy memory overhead. The mentioned papers investigate how the actor model can be used to close the conceptual gap between agent-specifications, which focus on message-passing and their implementation. Further they also showed that using this kind of concurrency allows to overcome some problems of low level concurrent programming as well.
Also \cite{bezirgiannis_improving_2013} ported NetLogos API to Erlang mapping agents to concurrently running processes, which interact with each other by message-passing. With some restrictions on the agent-interactions this model worked, which shows at using concurrent message-passing for parallel ABS is at least \textit{conceptually} feasible.

The work \cite{lysenko_framework_2008} discusses a framework which allows to map Agent-Based Simulations to Graphics Processing Units (GPU). Amongst others they use the SugarScape model \cite{epstein_growing_1996} and scale it up to millions of agents on very large environment grids. They reported an impressive speed-up of a factor of 9,000. Although their work is conceptually very different we can draw inspiration from their work in terms of performance measurement and comparison of the SugarScape model.

In \cite{thaler_pure_2019} the authors showed how to implement a spatial SIR model in pure Haskell using Functional Reactive Programming \cite{hudak_arrows_2003}. They report quite low performance but mention that STM may be a way to considerably speed up the simulation. We follow their approach in implementation technique, using Functional Reactive Programming and Monadic Stream Functions \cite{perez_functional_2016} (we don't go into implementation details as it is out of the scope of this paper) and use the spatial SIR model as the first case-study.

%The authors of Software Transactional Memory vs. Locking in a Functional Language \cite{castor_software_2011} investigated in a controlled experiment whether the promises of STM being less error-prone than lock-based approaches are valid or not. They assessed 51 undergraduates in two groups with one using STM and the other locks and compared the errors in the resulting programs, lines of code and time to finish the assessment. They report no statistically significant difference in both approaches. 

% In, A survey on parallel and distributed multi-agent systems for high performance computing simulations \cite{rousset_survey_2016}, the authors investigate all currently existing distributed MAS where most of them use MPI (message passing interface) for communication across machines.

\section{Background}

\subsection{Schelling Segregation}
We follow in our implementation the original paper of Schelling as in \cite{schelling_dynamic_1971} where we focus on the \textit{Area Distribution} section (Schelling starts with movement in a linear, 1-dimensional world where agents are able to move to the nearest point which meets the agents satisfaction but this is not what we follow here). One assumes a discrete 2-dimensional lattice-world with NxM fields. Each field is either occupied by an agent of a given color (e.g. Red or Green) or is free. Each field has 8 neighbours, which denotes a Moore-Neighbourhood. In Schellings original work the lattice-world is limited at its borders but we assume a torus world which is wrapped around in both the x- and y-dimensions resulting in 8 neighbours also for fields at the border. The occupation density was set by Schelling to be about 70\%-75\% which he identifies as being a setting which allows the agents to move around freely without making the lattice-world too sparse.
Now the agents make their move sequentially one after another. In each move an agent calculates the number of neighbours which are of equal color. If the number satisfies the agents needs about the neighbourhood then the agent is regarded as being 'happy' and will stay on this field. On the other hand the agent moves to the nearest unoccupied field which satisfies its needs. An agent which moves selects an unoccupied place randomly relative from its current place within a rectangle of side-length 2r where its current place is at the center. The interpretation for that behaviour is that agents won't move too far as it could be costly. Also children might attend a school in this area or the family has friends in this area, so they don't want to break that.



Agents just move depending on their movement-strategy to another place if they are not happy on the current one - they don't care how the target place is in the present or in the future, they will decide again in the next time-step. The interpretation for that behaviour is: agents want to 'just get out' at any cost, not caring what the future place will look like - it might be better or worse but they will see then.

\subsubsection{Optimizing behaviour}
TODO: define utility

The original schelling model didn't have a move-optimizing behaviour, meaning agents are just binary: if it is happy it will not move, if it is unhappy it will move but they won't care where they move. We introduce local move-optimizing behaviours which can be interpreted as being realistic in the real-world. It is important to note that we focus on \textit{local} instead of \textit{global} move-optimization: the agents are limited in their reasoning-capabilities and have limited information available: they cannot check out \textit{every} place and pick the globally best one.\\

\subsubsection{Anticipating behaviour}
Schelling explicitly mentions in \cite{schelling_dynamic_1971} that nobody anticipates moves of others. This is what we introduce using the recursive simulation.

TODO: is this optimizing behaviour in the spirit of schellings original work? 

\paragraph{Optimizing future} Agents pick an unoccupied random place and move to it if it increases their utility in the future. The interpretation for that behaviour is: agents heard about a place which will be cool in the future.

\paragraph{Optimizing present \& future} Agents pick an unoccupied random place and move to it if it increases their utility in the now and in the future. The interpretation for that behaviour is: agents heard about a cool spot in town, check it out and move to it if they like it but they also anticipate the coolness of the place in the future and if it seems that the place is going down then they won't move there.

\subsection{Related Research}
TODO: \cite{kirman_complex_2010} mention kirman complex economics where he investigates the model more in depth


\section{STM and ABS}

For a proof-of-concept we changed the reference implementation of the agent-based SIR model on a 2D-grid as described in the paper in Appendix \ref{app:pfe}. In it, a State Monad is used to share the grid across all agents where all agents are run after each other to guarantee exclusive access to the state. We replaced the State Monad by the STM Monad, share the grid through a \textit{TVar} and run every agent within its own thread. All agents are run at the same time but synchronise after each time-step which is done through the main-thread.

We make STM the innermost Monad within a RandT transformer:
\begin{HaskellCode}
type SIRMonad g   = RandT g STM
type SIRAgent g   = SF (SIRMonad g) () ()
\end{HaskellCode}

In each step we use an \textit{MVar} to let the agents block on the next $\Delta t$ and let the main-thread block for all results. After each step we output the environment by reading it from the \textit{TVar}:
\begin{HaskellCode}
-- this is run in the main-thread
simulationStep :: TVar SIREnv
               -> [MVar DTime]
               -> [MVar ()]
               -> Int
               -> IO SIREnv
simulationStep env dtVars retVars _i = do
  -- tell all threads to continue with the corresponding DTime
  mapM_ (`putMVar` dt) dtVars
  -- wait for results, ignoring them, only [()]
  mapM_ takeMVar retVars
  -- read last version of environment
  readTVarIO env
\end{HaskellCode}

Each agent runs within its own thread. It will block for the posting of the next $\Delta t$ where it then will run the MSF stack with the given $\Delta t$ and atomically transacting the STM action. It will then post the result of the computation to the main-thread to signal it has finished. Note that the number of steps the agent will run is hard-coded and comes from the main-thread so that no infinite blocking occurs and the thread shuts down gracefully.

\begin{HaskellCode}
createAgentThread :: RandomGen g 
                  => Int 
                  -> TVar SIREnv
                  -> MVar DTime
                  -> g
                  -> (Disc2dCoord, SIRState)
                  -> IO (MVar ())
createAgentThread steps env dtVar rng0 a = do
    let sf = uncurry (sirAgent env) a
    -- create the var where the result will be posted to
    retVar <- newEmptyMVar
    _ <- forkIO (sirAgentThreadAux steps sf rng0 retVar)
    return retVar
  where
    agentThread :: RandomGen g 
                => Int
                -> SIRAgent g
                -> g
                -> MVar ()
                -> IO ()
    agentThread 0 _ _ _ = return ()
    agentThread n sf rng retVar = do
      -- wait for next dt to compute next step
      dt <- takeMVar dtVar

      -- compute next step
      let sfReader = unMSF sf ()
          sfRand   = runReaderT sfReader dt
          sfSTM    = runRandT sfRand rng
      ((_, sf'), rng') <- atomically sfSTM 
      
      -- post result to main thread
      putMVar retVar ()
      
      agentThread (n - 1) sf' rng' retVar
\end{HaskellCode}

\section{Case Study 1 - Spatial SIR (First Encounter)} 
\label{sec:cs_sir}

Our first case study is the SIR model which is a very well studied and understood compartment model from epidemiology \cite{kermack_contribution_1927} which allows to simulate the dynamics of an infectious disease like influenza, tuberculosis, chicken pox, rubella and measles spreading through a population \cite{enns_its_2010}.

In it, people in a population of size $N$ can be in either one of three states \textit{Susceptible}, \textit{Infected} or \textit{Recovered} at a particular time, where it is assumed that initially there is at least one infected person in the population. People interact \textit{on average} with a given rate of $\beta$ other people per time-unit and become infected with a given probability $\gamma$ when interacting with an infected person. When infected, a person recovers \textit{on average} after $\delta$ time-units and is then immune to further infections. An interaction between infected persons does not lead to re-infection, thus these interactions are ignored in this model. 

We followed in our agent-based implementation of the SIR model the work \cite{macal_agent-based_2010} but extended it by placing the agents on a discrete 2D grid using a Moore (8) neighbourhood TODO: cite my own PFE paper. In this case agents interact with each other indirectly through the shared discrete 2D grid by writing their current state on their cell which neighbours can read. A visualisation can be seen in Figure \ref{fig:vis_sir}.

It is important to note that due to the continuous-time nature of the SIR model, our implementation follows the time-driven \cite{meyer_event-driven_2014} approach and maps naturally to the continuous time-semantics and state-transitions provided by FRP. By sampling the system with very small $\Delta t$ this means that we have comparatively very few writes to the shared environment which will become important when discussing the performance results.

\begin{figure}
\begin{center}
	\begin{tabular}{c c}
		\begin{subfigure}[b]{0.4\textwidth}
			\centering
			\includegraphics[width=1\textwidth, angle=0]{./fig/sir/vis/51x51agents_t50_01dt.png}
			\caption{$t = 50$}
			\label{fig:vis_51x51agents_t50_01dt}
		\end{subfigure}
    	
    	&
  
		\begin{subfigure}[b]{0.4\textwidth}
			\centering
			\includegraphics[width=1\textwidth, angle=0]{./fig/sir/vis/51x51agents_t100_01dt.png}
			\caption{$t = 100$}
			\label{fig:vis_51x51agents_t100_01dt}
		\end{subfigure}
	\end{tabular}
	
	\caption{Simulating the agent-based SIR model on a 51x51 2D grid with Moore neighbourhood, a single infected agent at the center, contact rate $\beta = \frac{1}{5}$, infection probability $\gamma = 0.05$ and illness duration $\delta = 15$ . Simulation run until $t = 100$ with fixed $\Delta t = 0.1$. The susceptible agents are rendered as blue hollow circles for better contrast.}
	\label{fig:vis_sir}
\end{center}
\end{figure}

\subsection{Experiment Design}
In this case study we compare the performance of the following implementations under varying numbers of CPU cores and agent numbers:

\begin{enumerate}
	\item Sequential - This is the original implementation we also discuss in TODO: cite my own PFE paper. In it the discrete 2D grid is shared amongst all agents using the State Monad. Agents are run sequentially after another thus ensuring exclusive read/write access to it. Because we are neither running in the STM or IO Monad there is no way we can run this implementation concurrently.
	\item STM - This is the same implementation like the State Monad but instead of sharing the discrete 2D grid in a State Monad, agents run in the STM Monad and have access to the discrete 2D grid through a transactional variable \textit{TVar}. This means that the reads and writes of the discrete 2D grid are exactly the same but happen always through the \textit{TVar}. Also each agent is run within its own thread, thus enabling true concurrency when the simulation is actually run on multiple cores (which can be configured by the Haskell Runtime System).
	\item Lock-Based - This is exactly the same implementation like the STM Monad but instead of running in STM, the agents now run in IO. They share the discrete 2D grid using an \textit{IORef} and have access to an \textit{MVar} to synchronise access to the it. Also each agent is run within its own thread.
	\item RePast - To have an idea where the functional implementation is performance-wise compared to the established object-oriented methods, we implemented a Java version of the SIR model using RePast with the State-Chart feature. This implementation cannot run on multiple cores concurrently but gives a good estimate of the single core performance of imperative approaches. Also there exists a RePast High Performance Computing library for implementing large-scale distributed simulations in C++ - we leave this for further research as an implementation and comparison is out of scope of this paper.
\end{enumerate}

Each experiment was run until $t = 100$ and stepped using $\Delta t = 0.1$ except in RePast for which we don't have access to the underlying implementation of the state-chart and left it as it is. For each experiment we conducted 8 runs on our machine (see Table \ref{tab:machine_specs}) under no additional work-load and report both the average and standard deviation. Further, we checked the visual outputs and the dynamics and they look qualitatively the same to the reference implementation of the State Monad TODO: cite my own PFE paper. In the experiments we varied the number of agents (grid size) and the number of cores when running concurrently - the numbers are always indicated clearly. For varying the number of cores we compiled the executable using \textit{stack} and the \textit{threaded} option and executed it with \textit{stack} using the +RTS -Nx option where x is the number of cores between 1 and 4. 

\begin{table}
	\centering
	\begin{tabular}{ c || c }
		OS & Fedora 28 64-bit \\ \hline
		RAM & 16 GByte \\ \hline
		CPU & Intel Core i5-4670K @ 3.40GHz x 4 \\ \hline
		HD & 250Gbyte SSD \\ \hline
		Haskell & GHC 8.2.2 \\ \hline
		Java & OpenJDK 1.8.0 \\ \hline
		RePast & 2.5.0.a
	\end{tabular}
	
	\caption{Machine and Software Specs for all experiments}
	\label{tab:machine_specs}
\end{table}

\subsection{Constant Grid Size, Varying Cores}
In this experiment we held the grid size constant to 51 x 51 (2,601 agents) and varied the cores where possible. The results are reported in Table \ref{tab:constgrid_varyingcores}.

\begin{table}
	\centering
  	\begin{tabular}{ c || c | c  }
                    & Cores & Duration       \\ \hline \hline 
    	Sequential  & 1     & 100.33 (0.434) \\ \hline \hline
   		STM         & 1     & 53.182 (0.393) \\ \hline
   		STM         & 2     & 27.817 (0.555) \\ \hline
   		STM         & 3     & 21.776 (0.388) \\ \hline
   		STM         & 4     & 20.201 (0.789) \\ \hline \hline
   		Lock-Based  & 1     & 60.564 (0.265) \\ \hline 
   		Lock-Based  & 2     & 42.779 (0.421) \\ \hline 
   		Lock-Based  & 3     & 38.586 (0.451) \\ \hline 
   		Lock-Based  & 4     & 41.555 (0.445) \\ \hline \hline
   		RePast      & 1     & \textbf{10.822} (0.377)
  	\end{tabular}
  	
  	\caption{Experiments on constant 51x51 (2,601 agents) grid with varying number of cores.}
	\label{tab:constgrid_varyingcores}
\end{table}

%TODO: re-run the 3-core and 4-core versions of IO, i don't understand why on larger grid-sizes 4-core is faster. do 16 runs each
Comparing the performance and scaling on multiple cores of the STM and Lock-Based implementations shows that the lock-free STM implementation significantly outperforms the Lock-Based one and scales better to multiple cores. The Lock-Based implementation performs best with 3 cores and shows slightly worse performance on 4 cores as can be seen in Figure \ref{fig:core_duration_stm_io}. This is no surprise because the more cores are running at the same time, the more contention for the lock, thus the more likely synchronisation happening, resulting in more potential for reduced performance. This is not an issue in STM because no locks are taken in advance. 

Comparing the reference \textit{State} implementation shows that it is the slowest by far - even the single core STM and Lock-Based implementations outperform it by far. Also our profiling results reported about 30\% increased memory footprint for the State implementation. This shows that the State Monad is a rather slow and memory intense approach sharing data but guarantees purity and excludes any non-deterministic side-effects which is not the case in STM and IO.

What comes a bit as a surprise is that the single core RePast implementation significantly outperforms \textit{all} other implementations, even when they run on multiple cores and even with RePast doing complex visualisation in addition (something the functional implementations don't do). We attribute this to the conceptually slower approach of functional programming. We might could have optimised parts of the code but leave this for further research.

\begin{figure}
	\centering
	\includegraphics[width=0.6\textwidth, angle=0]{./fig/sir/core_duration_stm_io.png}
	\caption{Comparison of performance and scaling on multiple cores of STM vs. IO. Note that the Lock-Based implementation performs worse on 4 cores than on 3.}
	\label{fig:core_duration_stm_io}
\end{figure}

\subsection{Varying Grid Size, Constant Cores}
In this experiment we varied the grid size and used constantly 4 cores. Because in the previous experiment, Lock-Based performed best on 3 cores, we additionally ran Lock-Based on 3 cores as well. The results for STM are reported in Table \ref{tab:varyinggrid_constcores_stm}, for Lock-Based in Tables \ref{tab:varyinggrid_constcores4_IO}, \ref{tab:varyinggrid_constcores3_IO} and Repast in Table \ref{tab:varyinggrid_constcores_repast}. Again, note that the RePast experiments all ran on a single (1) core and were conducted to have a rough estimate where the functional approach is in comparison to the imperative.

\begin{table}
	\centering
  	\begin{tabular}{ c || c }
        Grid-Size          & Duration                \\ \hline \hline 
   		51 x 51 (2,601)    & 20.201          (0.789) \\ \hline
   		101 x 101 (1,0201) & \textbf{74.493} (0.524) \\ \hline
   		151 x 151 (22,801) & \textbf{168.47} (1.783) \\ \hline
   		201 x 201 (40,401) & \textbf{302.43} (3.931) \\ \hline
   		251 x 251 (63,001) & \textbf{495.73} 0
  	\end{tabular}

  	\caption{STM Monad experiments on varying grid sizes on 4 cores.}
	\label{tab:varyinggrid_constcores_stm}
\end{table}


\begin{table}
	\centering
  	\begin{tabular}{ c || c }
        Grid-Size          & Duration 		\\ \hline \hline 
   		51 x 51 (2,601)    & 41.914 (1.073) \\ \hline
   		101 x 101 (10,201) & 170.55 (1.115) \\ \hline
   		151 x 151 (22,801) & 376.89 0 		\\ \hline
   		201 x 201 (40,401) & 672.01 0 		\\ \hline 
   		251 x 251 (63,001) & 1,027.27 0
  	\end{tabular}
  	
  	\caption{Lock-Based experiments on varying grid sizes on 4 cores.}
	\label{tab:varyinggrid_constcores4_IO}
\end{table}

\begin{table}
	\centering
  	\begin{tabular}{ c || c  }
        Grid-Size         & Duration  		\\ \hline \hline 
   		51 x 51   (2,601)  & 38.614 (0.397) \\ \hline
   		101 x 101 (1,0201) & 171.61 (3.016) \\ \hline
   		151 x 151 (22,801) & 404.11	0 		\\ \hline
   		201 x 201 (40,401) & 720.65 0 		\\ \hline 
   		251 x 251 (63,001) & 1,117.27 0 
  	\end{tabular}
  	
  	\caption{Lock-Based experiments on varying grid sizes on 3 cores.}
	\label{tab:varyinggrid_constcores3_IO}
\end{table}

\begin{table}
	\centering
  	\begin{tabular}{ c || c }
        Grid-Size          & Duration  				 \\ \hline \hline 
   		51 x 51 (2,601)    & \textbf{10.822} (0.377) \\ \hline
   		101 x 101 (10,201) & 107.40 (1.306) 		 \\ \hline
   		151 x 151 (22,801) & 464.017  0 			 \\ \hline
   		201 x 201 (40,401) & 1,227.68 0 			 \\ \hline 
   		251 x 251 (63,001) & 3,283.63 0 
  	\end{tabular}
  	
  	\caption{Repast experiments on varying grid sizes on a single (1) core.}
	\label{tab:varyinggrid_constcores_repast}
\end{table}

We plotted the results in Figure \ref{fig:stm_io_repast_varyinggrid_performance}. It is clear that the lock-free STM implementation outperforms the lock-based Lock-Based implementation by a substantial factor. Surprisingly, the Lock-Based implementation on 4 core scales just slightly better with increasing agents number than on 3 cores, something we wouldn't have anticipated based on the results seen in Table \ref{tab:constgrid_varyingcores}. Also  while on a 51x51 grid the single (1) core Java RePast version outperforms the 4 core Haskell STM version by about 200\%. The figure is inverted on a 251x251 grid where the 4 core Haskell STM version outperforms the single core Java Repast version by over 600\%. This might not be entirely surprising because we compare single (1) core against multi-core performance - still the scaling is indeed impressive and we would never have anticipated an increase of over 600\%.

\begin{figure}
\begin{center}
	\begin{tabular}{c c}
		\begin{subfigure}[b]{0.5\textwidth}
			\centering
			\includegraphics[width=1\textwidth, angle=0]{./fig/sir/stm_io_repast_varyinggrid_performance.png}
			\caption{Normal Scale}
		\end{subfigure}
    	&
		\begin{subfigure}[b]{0.5\textwidth}
			\centering
			\includegraphics[width=1\textwidth, angle=0]{./fig/sir/stm_io_repast_varyinggrid_performance_loglog.png}
			\caption{Logarithmic scale on both axes}
		\end{subfigure}
    \end{tabular}
	\caption{Comparison of STM (Table \ref{tab:varyinggrid_constcores_stm}), Lock-Based (Table \ref{tab:varyinggrid_constcores4_IO}, Table 					\ref{tab:varyinggrid_constcores3_IO}) and RePast (single core) (Table \ref{tab:varyinggrid_constcores_repast}) performance. TODO: re-create the figure when all experiments had 8 runs.}
	\label{fig:stm_io_repast_varyinggrid_performance}
\end{center}
\end{figure}

%\begin{figure}
%	\centering
%	\includegraphics[width=0.6\textwidth, angle=0]{./fig/sir/stm_io_repast_varyinggrid_performance.png}
%	\caption{Comparison of STM (Table \ref{tab:varyinggrid_constcores_stm}), Lock-Based (Table \ref{tab:tab:varyinggrid_constcores4_IO}, Table \ref{tab:varyinggrid_constcores3_IO}) and RePast (single core) (Table \ref{tab:varyinggrid_constcores_repast}) performance. TODO: re-create the figure when all experiments had 8 runs.}
%	\label{fig:stm_io_repast_varyinggrid_performance}
%\end{figure}

\subsection{Retries}
Of very much interest when using STM is the retry-ratio, which obviously depends highly on the read-write patterns of the respective model. We used the stm-stats library to record statistics of commits, retries and the ratio. In these experiments we only averaged over 4 runs because they all arrived at a ratio of 0.0. The results are reported in Table \ref{tab:retries_stm}.

\begin{table}
	\centering
  	\begin{tabular}{ c || c | c | c }
        Grid-Size 		   & Commits    & Retries         & Ratio \\ \hline \hline 
   		51 x 51 (2,601)    & 2,601,000  & 1306.5 (83.9)   & 0.0 \\ \hline
   		101 x 101 (10,201) & 10,201,000 & 3712.5 (308.42) & 0.0 \\ \hline
   		151 x 151 (22,801) & 22,801,000 & 8189.5 (342.12) & 0.0 \\ \hline
   		201 x 201 (40,401) & 40,401,000 & 13285 (0.0)     & 0.0 \\ \hline 
   		251 x 251 (63,001) & 63,001,000 & 21217 (0.0)     & 0.0
  	\end{tabular}
  	
  	\caption{Retries Ratio of STM Monad experiments on varying grid sizes on 4 cores.}
	\label{tab:retries_stm}
\end{table}

Independent of the number of agents we always have a retry-ratio of 0.0. This indicates that this model is \textit{very} well suited to STM, which is also directly reflected in the substantial better performance over the Lock-Based implementation. Obviously this ratio stems from the fact, that in our implementation we have \textit{very} few writes (only when an agent changes e.g. from Susceptible to Infected or from Infected to Recovered) and mostly reads. Also we conducted runs on lower number of cores which resulted in fewer retries, which was what we expected.

%\begin{figure}
%	\centering
%	\includegraphics[width=0.6\textwidth, angle=0]{./fig/sir/retries_stm.png}
%	\caption{Scaling of retries by agent count. TODO: re-create the figure when all experiments had 4 runs.}
%	\label{fig:retries_stm}
%\end{figure}

\subsection{Discussion}
Interpretation of the performance data leads to the following insights:
\begin{enumerate}
	%\item On a single core, no transaction retries should happen, the results support that assumption.
	\item Running in STM and sharing state using a transactional variable is much more time- and memory-efficient than running in the State Monad but potentially sacrifices determinism: repeated runs might not lead to same dynamics despite same initial conditions.
	\item Running STM on multiple cores concurrently \textit{does} lead to a significant performance improvement \textit{for that model}.
	\item STM outperforms the Lock-Based implementation substantially and scales much better to multiple cores.
	\item STM on single (1) core is still about twice as slow than an object-oriented Java RePast implementation on a single (1) core.
	\item STM on multiple cores dramatically outperforms the single (1) core object-oriented Java RePast implementation on a single (1) core on instances with large agent numbers and scales much better to increasing number of agents.
\end{enumerate}

\section{Case Study 2: Sugarscape (Second Encounter)}
\label{sec:cs_sugarscape}

One of the first models in Agent-Based Simulation was the seminal Sugarscape model developed by Epstein and Axtell in 1996 \cite{epstein_growing_1996}. Their aim was to \textit{grow} an artificial society by simulation and connect observations in their simulation to phenomenon observed in real-world societies. In this model a population of agents move around in a discrete 2D environment where sugar grows and interact with each other and the environment in many different ways. The main features of this model are (amongst others): searching, harvesting and consuming of resources, wealth and age distributions, population dynamics under sexual reproduction, cultural processes and transmission, combat and assimilation, bilateral decentralized trading (bartering) between agents with endogenous demand and supply, disease processes transmission and immunology.

We implemented the \textit{Carrying Capacity} (p. 30) section of Chapter II of the book \cite{epstein_growing_1996}. There, in each step agents search (move) to the cell with the highest sugar they see within their vision, harvest all of it from the environment and consume sugar because of their metabolism. Sugar regrows in the environment over time. Only one agent can occupy a cell at a time. Agents don't age and cannot die from age. If agents run out of sugar due to their metabolism, they die from starvation and are removed from the simulation. The authors report that the initial number of agents quickly drops and stabilises around a level depending on the model parameters. This is in accordance with our results as we show in Figure \ref{fig:vis_sugarscape} and guarantees that we don't run out of agents. The model parameters are as follows:

\begin{itemize}
	\item Sugar Endowment: each agent has an initial sugar endowment randomly uniform distributed between 5 and 25 units.
	\item Sugar Metabolism: each agent has a sugar metabolism randomly uniform distributed between 1 and 5.
	\item Agent Vision: each agent has a vision randomly uniform distributed between 1 and 6, same for each of the 4 directions (N, W, S, E). 
	\item Sugar Growback: sugar grows back by 1.0 unit per step until the maximum capacity of a cell is reached.
	\item Agent Number: initially 500 agents.
	\item Environment Size: 50 x 50 cells with toroid boundaries which wrap around in both x and y dimension.
\end{itemize}

\begin{figure}
\begin{center}
	\begin{tabular}{c c}
		\begin{subfigure}[b]{0.4\textwidth}
			\centering
			\includegraphics[width=1\textwidth, angle=0]{./fig/sugarscape/vis/sugarscape_t60_environment.png}
			\caption{Visualisation of the Sugarscape at $t = 50$. TODO: retake the pictures}
			\label{fig:vis_sugarscape_t50_environment}
		\end{subfigure}
    	
    	&
  
		\begin{subfigure}[b]{0.6\textwidth}
			\centering
			\includegraphics[width=1\textwidth, angle=0]{./fig/sugarscape/vis/sugarscape_t60_dynamics.png}
			\caption{Dynamics population size over 50 steps. TODO: retake the picture.}
			\label{fig:vis_sugarscape_t50_dynamics}
		\end{subfigure}
	\end{tabular}
	
	\caption{Visualisation of our SugarScape implementation and dynamics of the population size over 50 steps. The white numbers in the blue agent circles are the agents unique ids.}
	\label{fig:vis_sugarscape}
\end{center}
\end{figure}

\subsection{Experiment Design}
We compare three different implementations

\begin{enumerate}
	\item Sequential - All agents are run after another (including the environment) and the environment is shared amongst the agents using the State Monad.
	\item Lock-Based - All agents are run concurrently and the environment is shared using an \textit{IORef} amongst the agents which acquire and release a lock when accessing it.
	\item STM TVar - All agents are run concurrently and the environment is shared using a \textit{TVar} amongst the agents.
	\item STM TArray - All agents are run concurrently and the environment is shared using a \textit{TArray} amongst the agents. 
\end{enumerate}

The model specification requires to shuffle agents before every step (Footnote 12 on page 26). In the \textit{Sequential} approach we do this explicitly but in both STM approaches this happens automatically due to race-conditions in concurrency thus we arrive at an effectively shuffled processing of agents: we can assume that the order of the agents is \textit{effectively} random in every step. The important difference between the two approaches is that in the State approach we have full control over this randomness but in the STM not - also this means that repeated runs with the same initial conditions might lead to slightly different results.
Note that in the concurrent implementations we could have two options for running the environment: either running it asynchronously as a concurrent agent at the same time with the population agents or synchronously after all agents have run. We must be careful though as running the environment as a concurrent agent can be seen as conceptually wrong because the time when the regrowth of the sugar happens is now completely random. It could happen in the very first transaction or in the very last, different in each step, which can be seen as a violation of the model specifications (TODO: reference the book where it shows that environment grows after / before all agents).

We follow \cite{lysenko_framework_2008} and measure the average updates per second of the simulation over 60 seconds.

For each experiment we conducted 8 runs on our machine (see Table \ref{tab:machine_specs}) under no additional work-load and report the average. In the experiments we varied the number of cores when running concurrently - the numbers are always indicated clearly. For varying the number of cores we compiled the executable using \textit{stack} and the \textit{threaded} option and executed it with \textit{stack} using the \textit{+RTS -Nx} option where x is the number of cores between 1 and 4.

Note that we omit the graphical rendering in the functional approach because it is a serious bottleneck taking up substantial amount of the simulation time. Although visual output is crucial in ABS, it is not what we are interested here thus we completely omit it and only output the number of agents in the simulation at each step piped into a file, thus omitting slow output to the console. Note that we need to produce \textit{some} output because of Haskells laziness - if we wouldn't output anything from the simulation then the expressions would actually never be fully evaluated thus resulting in ridiculous high number of steps per second but which obviously don't really reflect the true computations done.

\subsection{Constant Agent Size}
In this first approach we compare the performance of all implementations on varying numbers of cores. The results are reported in Table \ref{tab:varying_cores} and can be seen in Figure \ref{fig:varying_cores}. 

\begin{table}
	\centering
  	\begin{tabular}{ c || c | c | c }
                   & Cores & Steps & Retries  \\ \hline \hline 
    	Sequential & 1     & 39.4  & N/A      \\ \hline \hline   

    	Lock-Based & 1     & 43.0  & N/A       \\ \hline
    	Lock-Based & 2     & 51.8  & N/A       \\ \hline
    	Lock-Based & 3     & 57.4  & N/A       \\ \hline
    	Lock-Based & 4     & 58.1  & N/A       \\ \hline \hline   
   		
   		STM TVar   & 1     & 47.3  & 0.0       \\ \hline
   		STM TVar   & 2     & 53.5  & 1.1       \\ \hline
   		STM TVar   & 3     & 57.1  & 2.2 	   \\ \hline
   		STM TVar   & 4     & 53.0  & 3.2	   \\ \hline \hline   
   		
   		STM TArray & 1     & 45.4  & 0.0 	   \\ \hline
   		STM TArray & 2     & 65.3  & 0.02      \\ \hline
   		STM TArray & 3     & 75.7  & 0.04      \\ \hline
   		STM TArray & 4     & 84.4  & 0.05	   \\ \hline \hline   
   	\end{tabular}
  	
  	\caption{Steps per second and retries on 50x50 grid and 500 initial agents on varying cores.}
	\label{tab:varying_cores}
\end{table}

\begin{figure}
	\centering
	\includegraphics[width=0.7\textwidth, angle=0]{./fig/sugarscape/varying_cores.png}
	\caption{Steps per second and retries on 50x50 grid and 500 initial agents on varying cores.}
	\label{fig:varying_cores}
\end{figure}

As expected, the \textit{Sequential} implementation is the slowest. 

Clearly the concurrent \textit{STM} implementation outperforms the \textit{Sequential} one but the results are below expectations - clearly the speed-up is not as much as we hoped for. This is immediately reflected in the retry-ratios which rise up to more than 3 on 4 cores which means that each agent re-tries its computation \textit{on average} 3 times in each step. Can we do better?


TODO: the Tarray seems to scale up by 10 steps per second for every core added, it would be interesting to see how far this could go


Second, using \textit{TVar} to share the environment is a very inefficient choice: \textit{every} write to a cell leads to a retry independent whether the reading agent read that changed cell or not because the data-structure can not distinguish between individual cells.

The first shortcoming is already addressed by running the environment synchronously after all agents have run. When looking at the results we see that running the environment synchronously might have led to a correct implementation but the performance difference is insignificant. It seems that the choice of the \textit{TVar} is the limiting factor. This is also the second shortcoming and can be addressed by using a \textit{TArray} instead. 

Let us now switch to an implementation using the \textit{TArray} data-structure. In this implementation we replaced the \textit{TVar} by a \textit{TArray} data-structure which should reduce the number of retries substantially and thus improve performance by a considerable factor. By using a \textit{TArray} we can avoid the situation where a write to a cell in a far distant location of the environment will lead to a retry of an agent which never even touched that cell. Also we ran the environment synchronously. The results are reported in Table \ref{tab:tarray_results_syncenv_time} and can be seen in Figure \ref{fig:tarray_results_syncenv_time}.

Now we are arriving at substantial performance improvements, which is directly reflected in the retry-ratios which are close to 0. Also this makes the point of this section crystal clear: selecting the right transactional data-structure is paramount for best performance when using STM. Out of interest we ran the \textit{TArray} implementation with a concurrent environment to see how much impact this has, and indeed it has some impact and reduces performance by quite some factor but is still considerable faster than a \textit{TVar} synchronous environment approach. What is interesting is that the performance on 2 cores drops below the one of 1 core TODO: why?.

TODO: figure which combines all the previous figures into one: TVar sync and async with TArray sync and async

\subsection{Scaling up Agents}
So far we always kept the initial number of agents at 500, which due to the model specification, quickly drops to around 200 and stabilises around this value due to the carrying capacity of the environment as described in the book \cite{epstein_growing_1996} section \textit{Carrying Capacity} (p. 30).

We now want to see the scaling property of our approaches when increasing the number of agents. For this we slightly change the implementation: always when an agent dies it spawns a new one. This ensures that we keep the number of agents always constant (still fluctuates slightly between 500 and 490) over the whole duration. This ensures a constant load of concurrent processes interacting with each other and demonstrates also the ability to terminate and fork threads dynamically during the simulation.

Except for the \textit{Sequential} approach we ran all experiments with 4 cores with a concurrent environment. We looked into the performance of 500, 1,000, 1,500, 2,000 and 2,500 (maximum possible capacity of the 50x50 environment). We also measured the average retries both for \textit{TVar} and \textit{TArray} under 2,500 agents where the \textit{TArray} approach shows best scaling performance with 0.01 retries whereas \textit{TVar} averages at 3.28 retries. Again this can be attributed to the better transactional data-structure which reduces retry-ratio substantially to near-zero levels. The results are reported in Table \ref{tab:state_results_agentsscale_time} and can be seen in Figure \ref{fig:state_results_agentsscale_time}.

TODO: re-run all experiments, select same spot on rebirth otherwise will take too much time to find a new spot e.g. when 2,500 agents

\begin{table}
	\centering
  	\begin{tabular}{ c || c | c | c | c }
        Agents  & Sequential & Lock-Based & TVar       & TArray        \\ \hline \hline 
    	500     & TODO 14.1       & TODO			  &	TODO 21.1       & TODO \textbf{74.4} \\ \hline
   		1,000   & TODO 6.8        & TODO 			  & TODO 11.3       & TODO \textbf{56.8} \\ \hline
   		1,500   & TODO 4.5        & TODO 			  & TODO 8.1        & TODO \textbf{45.2} \\ \hline
   		2,000   & TODO 3.3        & TODO 			  & TODO 6.2        & TODO \textbf{37.0} \\ \hline 
   		2,500   & TODO 2.6        & TODO 			  & TODO 5.2        & TODO \textbf{31.7}
   	\end{tabular}
  	
  	\caption{Steps per second on 50x50 grid and varying number of agents.}
	\label{tab:state_results_agentsscale_time}
\end{table}

\begin{figure}
	\centering
	\includegraphics[width=0.6\textwidth, angle=0]{./fig/sugarscape/varying_agents.png}
	\caption{Steps per second on 50x50 grid and varying number of agents. TODO: re-render figure}
	\label{fig:state_results_agentsscale_time}
\end{figure}

\subsection{Comparison with other approaches}
The paper \cite{lysenko_framework_2008} reports a performance of 17 steps in RePast, 18 steps in MASON (both non-parallel) and 2000 steps per second on a GPU on a 128x128 grid. Although our \textit{Sequential} implementation which runs non-parallel as well outperforms the RePast and MASON implementations one must be very well aware that these results were generated in 2008, on 10 year older hardware - the performance might have caught up by now and even outperform our functional \textit{Sequential} approach. 

Indeed, when we run the SugarScape example of RePast with the same model parameters as ours on the same machine (see Table \ref{tab:machine_specs}) we arrive at roughly 450 steps per second - a factor of more than 5 faster than even our STM \textit{TArray} implementation on 4 cores. This might seem quite shocking, even more so because RePast also performs visual output, rendering the SugarScape in every step. When scaling up the agents to 2,500 the RePast version arrives around roughly 95 steps per second which is still faster by a factor of 3 than our 4 core \textit{TArray} implementation. Still our research is just a first step and might result in future work increasing performance.

The very high performance on the GPU does not concern us here as it follows a very different approach than we do here. Our focus is on speeding up implementations on the CPU as directly as possible without locking overhead. When following a GPU approach one needs to map the model to the GPU which is a delicate and non-trivial approach. With our approach we show that speed-up with concurrency is very possible without the low-level locking details or the need to map to GPU.

Note that we kept the grid-size constant because we implemented the environment as a single agent which works sequentially on the cells to regrow the sugar. Obviously this doesn't really scale up on parallel hardware and indeed, the performance goes down dramatically when we increase the environment to 128x128 with same number of agents. Obviously this is the result of Amdahls law where the environment becomes the limiting factor of the simulation. Depending on the underlying data-structure used for the environment we have two options to solving this problem. In the case of the \textit{Sequential} and \textit{TVar} implementation we build on an indexed array which we can updated in parallel using the existing data-parallel support in Haskell. In the case of the \textit{TArray} approach we have no option but to run the update of every cell within its own thread. We leave both for further research as it is out of scope of this paper. 

\subsection{Discussion}
In this section basically drives home the important point that selecting the right transactional data-structure is of utmost importance to maximise performance when scaling up to multiple cores. Unfortunately for this model the performance is nowhere comparable to imperative approaches which we attribute to the inherent deeper complexity of the model where it seems that imperative implementations seem to have an advantage.



\chapter{Conclusions}
\label{ch:conclusions}

This chapter concludes the whole thesis and outlines future research. Roughly 20\% exists already.

%we now know how to engineer time- and event-driven ABS with complex state both in the agent and environment, main difficulty is direct agent-interaction (see macal classification into 4 types of ABS), compile-time guaranteed reproducibility, explicit handling of complex state (read only, read/write), concurrency explicit and limited to STM, very promising concurrency but direct agent-interactions main problem (erlang as a rescue?), main drawbacks: everything is explicit, performance

\section{Further Research}
clearly outline the ideas for further research

\subsection{A general purpose library}
generalise concepts explored into a pure functional ABS library in Haskell (called chimera)

\subsection{Dependent and linear types}
dependent types and linear types are the next big step, towards a stronger formalisation of agents and ABS,
focus on the equilibrium - totality correspondence

\subsection{Concurrent event-driven ABS}
stm based concurrency for event-driven ABS using parallel DES. challenge is the time-warp implementation using monads. in general it should be easy to roll-back agents actions but with monads we have to be careful - for some monads rolling back is not neccessary e.g. rand and reader, for others it is, and for some it is impossible e.g. IO

\section{Further Research}
\label{sec:further_research}

We see this paper as an intermediary and necessary step towards dependent types for which we first needed to understand the potentials and limitations of a non-dependently typed pure functional approach in Haskell. Dependent types are extremely promising in functional programming as they allow us to express stronger guarantees about the correctness of programs and go as far as allowing to formulate programs and types as constructive proofs \cite{wadler_propositions_2015} which must be total by definition \cite{thompson_type_1991}, \cite{altenkirch_why_2005}, \cite{altenkirch_pi_2010}, \cite{program_homotopy_2013}. So far no research using dependent types in agent-based simulation exists at all and it is not clear whether dependent types make sense in this context. In our next paper we want to explore this for the first time and ask more specifically how we can add dependent types to our pure functional approach, which conceptual implications this has for ABS and what we gain from doing so. We plan on using Idris \cite{brady_idris_2013}, \cite{brady_type-driven_2017} as the language of choice as it is very close to Haskell with focus on real-world application and running programs as opposed to other languages with dependent types e.g. Agda and Coq which serve primarily as proof assistants.
It would be of immense interest whether we could apply dependent types to the model meta-level or not - this boils down to the question if we can encode our model specification in a dependent type way. This would allow the ABS community for the first time to reason about a proper formalisation of a model. We plan to implement a total and terminating implementation of our approach which would be a formal proof-by-construction that the agent-based approach of the SIR model terminates after a finite number of steps.

\begin{acks}
The authors would like to thank J. Hey and M. Handley for constructive feedback, comments and valuable discussions.
\end{acks}

% Bibliography
\bibliographystyle{ACM-Reference-Format}
\bibliography{references.bib}

\end{document}
