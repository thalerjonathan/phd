\section{Related Work}
\label{sec:rel_work}

%%%%%%%%%%%%%%%%%%%%%%%%%%%%%%%%%%%%%%%%%%%%%%%%%%%%%%%%%%%%%%%%%%%%%%%%%%%%%%%%%%%%%%%%%%%%%%%%%%%%%%%%%%%%%%%%%%%%%%%%%%
%% DOI: 10.1016/j.procs.2013.05.231 \cite{suryanarayanan_pdes-mas_2013}
%The work \cite{suryanarayanan_pdes-mas_2013} gives a detailed and in-depth discussion of the internals and implementation of their PDES-MAS system, and using the Tileworld ABM as benchmark provide the performance behaviour of it. It is a powerful system, providing mechanisms for the adaptive partitioning of the simulation shared state, addressing load balancing, synchronisation and interest management (avoiding all to all communications) in an integrated, adaptive and fully transparent manner. This work provides a good insight into how complex the development of a correctly working PDES system is. The authors make the important point, that MAS models are data-centric by nature, requiring a large number of shared variables, which has important implications parallelisation. The fact that it is a data-centric problem is exploited by STM to a very high degree as it allows to approach parallel programming as if it were a data-centric problem without the issues of low-level synchronisation.
%
%% DOI: 10.1109/5.910853 \cite{logan_distributed_2001}
%The rather conceptual work \cite{logan_distributed_2001} proposes a general, distributed simulation framework for multiagent systems. It discusses distributed discrete-event simulation as technique for simulating multi agent systems and addresses a number of key problems: decomposition of the environment, load balancing, modelling, communication and shared state variables, which the authors mention as the central problem of parallelisation. They base their ideas on spheres of influence to minimise communication and synchronisation between agent processes and propose an algorithm to decomposition of the
%shared state into specific communication logical processes. Again, the main focus is on the management of share variable and how do best parallelise access to them without violating causality principle.
%
%% DOI: 10.1177/0037549708096691 \cite{lees_using_2008}
%The work \cite{lees_using_2008} compares the performance of various optimistic synchronisation algorithms for parallel MAS simulation using PDES. 
%
%% DOI 10.1145/2517449 \cite{suryanarayanan_synchronised_2013}
%The work \cite{suryanarayanan_synchronised_2013} discusses range queries in PDES-MAS system.
%
%% SPADES (dl.acm.org/citation.cfm?id=1030926) \cite{riley_next_2003}
%OK
%
%% MACE3J (DOI: 10.1145/544862.544918) \cite{gasser_mace3j:_2002}
%OK
%
%% James II (DOI: 10.1109/ANSS.2007.34) plug 'n simulate \cite{himmelspach_plugn_2007}
%OK
%
%% DOI: 10.1002/cpe.1280 RePast-HLA which has ported RePast on HLA \cite{minson_distributing_2008}
%OK
%
%% DOI: 10.22360/SpringSim.2016.HPC.046 RepastHPC uses both optimistic and pessimistic PDES \cite{gorur_repast_2016}
%OK
%
%% DOI: 10.1145/2769458.2769462 hardware transactional memory has been investigated in the context of PDES not for agents though \cite{hay_experiments_2015}
%OK
%
%% DOI: 10.1016/j.cosrev.2017.03.001 \cite{abar_agent_2017}
%OK
%
%% \cite{cicirelli_efficient_2015} [2] Cicirelli, F., Giordano, A., & Nigro, L. (2015). Efficient environment management for distributed simulation of large‐scale situated multi‐agent systems. Concurrency and Computation: Practice and Experience, 27(3), 610-632.
%% DOI: 10.1145/2535417 interest management \cite{liu_interest_2014}
% NOT RELEVANT

%%%%%%%%%%%%%%%%%%%%%%%%%%%%%%%%%%%%%%%%%%%%%%%%%%%%%%%%%%%%%%%%%%%%%%%%%%%%%%%%%%%%%%%%%%%%%%%%%%%%%%%%%%%%%%%%%%%%%%%%%%
In his master thesis \cite{bezirgiannis_improving_2013} the author investigates Haskells' parallel and concurrency features to implement (amongst others) \textit{HLogo}, a Haskell clone of the NetLogo \cite{wilensky_introduction_2015} simulation package, focusing on using STM for a limited form of agent-interactions. \textit{HLogo} is basically a re-implementation of NetLogos API in Haskell where agents run within an unrestricted context (known as \textit{IO}) and thus can also make use of STM functionality. The benchmarks show that this approach does indeed result in a speed-up especially under larger agent-populations. The authors' thesis can be seen as one of the first works on ABS using Haskell. Despite the concurrency and parallel aspect our work share, our approach is rather different: we avoid IO within the agents under all costs and explore the use of STM more on a conceptual level rather than implementing a ABS library and compare our case-studies with lock-based and imperative implementations.

There exists some research \cite{di_stefano_using_2005, varela_modelling_2004, sher_agent-based_2013} using the functional programming language Erlang \cite{armstrong_erlang_2010} to implement concurrent ABS. The language is inspired by the actor model \cite{agha_actors:_1986} and was created in 1986 by Joe Armstrong for Eriksson for developing distributed high reliability software in telecommunications. The actor model can be seen as quite influential to the development of the concept of agents in ABS, which borrowed it from Multi Agent Systems \cite{wooldridge_introduction_2009}. It emphasises message-passing concurrency with share-nothing semantics (no shared state between agents), which maps nicely to functional programming concepts. Erlang implements light-weight processes, which allows to spawn thousands of them without heavy memory overhead. The mentioned papers investigate how the actor model can be used to close the conceptual gap between agent-specifications, which focus on message-passing and their implementation. Further they show that using this kind of concurrency allows to overcome some problems of low level concurrent programming as well.
Also \cite{bezirgiannis_improving_2013} ported NetLogos API to Erlang mapping agents to concurrently running processes, which interact with each other by message-passing. With some restrictions on the agent-interactions this model worked, which shows that using concurrent message-passing for parallel ABS is at least \textit{conceptually} feasible.

The work \cite{lysenko_framework_2008} discusses a framework, which allows to map Agent-Based Simulations to Graphics Processing Units (GPU). Amongst others they use the SugarScape model \cite{epstein_growing_1996} and scale it up to millions of agents on very large environment grids. They reported an impressive speed-up of a factor of 9,000. Although their work is conceptually very different we can draw inspiration from their work in terms of performance measurement and comparison of the SugarScape model.

In \cite{thaler_pure_2019} the authors showed how to implement a spatial SIR model in pure Haskell using Functional Reactive Programming \cite{hudak_arrows_2003}. They report quite low performance but mention that STM may be a way to considerably speed up the simulation. We follow their approach in implementation technique, using Functional Reactive Programming and Monadic Stream Functions \cite{perez_functional_2016} (we don't go into implementation details as it is out of the scope of this paper) and use the spatial SIR model as the first case-study.

%The authors of Software Transactional Memory vs. Locking in a Functional Language \cite{castor_software_2011} investigated in a controlled experiment whether the promises of STM being less error-prone than lock-based approaches are valid or not. They assessed 51 undergraduates in two groups with one using STM and the other locks and compared the errors in the resulting programs, lines of code and time to finish the assessment. They report no statistically significant difference in both approaches. 

% In, A survey on parallel and distributed multi-agent systems for high performance computing simulations \cite{rousset_survey_2016}, the authors investigate all currently existing distributed MAS where most of them use MPI (message passing interface) for communication across machines.