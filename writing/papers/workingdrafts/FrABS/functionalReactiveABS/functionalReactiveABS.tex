%\documentclass[a4paper, 10pt, conference]{../../templates/IEEEconf/IEEEconf}
\documentclass[10pt, conference, onecolumn]{../../../templates/IEEEtran/IEEEtran}
%\documentclass[10pt, journal]{../../templates/IEEEtran/IEEEtran}

\usepackage{graphicx}
\usepackage{caption} 
\usepackage{subcaption}
\usepackage{hyperref}
\usepackage{listings}
\usepackage{hhline}
\usepackage{float}
\usepackage{amssymb}
\usepackage[autostyle=true]{csquotes}
\usepackage{amsmath}
\usepackage{marvosym}
\usepackage{minted}

\font\subtitlefont=cmr12 at 18pt

\title{Spreading the disease \\ {\huge Simulating epidemics using Functional Reactive Programming}}

% IEEEtran journal authors
%\author{Jonathan Thaler, ̃Peer-Olaf Siebers \\ School of Computer Science \\ University of Nottingham%
%\thanks{jonathan.thaler@nottingham.ac.uk}%
%\thanks{peer-olaf.siebers@nottingham.ac.uk}
%}

% IEEEtran conference authors
\author{
	\IEEEauthorblockN{Jonathan Thaler}
	\IEEEauthorblockA{School of Computer Science\\
		University of Nottingham\\
		jonathan.thaler@nottingham.ac.uk}
		
	\and
		
	\IEEEauthorblockN{Thorsten Altenkirch}
	\IEEEauthorblockA{School of Computer Science\\
		University of Nottingham\\
		thorsten.altenkirch@nottingham.ac.uk}
}

%\IEEEpubid{0000--0000/00\$00.00 ̃\copyright ̃2015 IEEE}

% IEEEconf authors
%\author{
%	Jonathan Thaler \\
%	\email{jonathan.thaler@nottingham.ac.uk} \\
%	\begin{affiliation}
%		School of Computer Science, University of Nottingham
%	\end{affiliation} \\
%	\and 
%	Peer-Olaf Siebers \\
%	\email{peer-olaf.siebers@nottingham.ac.uk} \\
%	\begin{affiliation}
%		School of Computer Science, University of Nottingham
%	\end{affiliation} 
%	\and 
%	Thorsten Altenkirch \\
%	\email{thorsten.altenkirch@nottingham.ac.uk} \\
%	\begin{affiliation}
%		School of Computer Science, University of Nottingham
%	\end{affiliation} 
%}

\begin{document}
\maketitle 

\begin{abstract}
TODO: need to select the right journal for publication, something more practical functional programming

Agent-Based Simulation (ABS) is a methodology in which a system is simulated in a bottom-up approach by modelling the micro interactions of its constituting parts, called agents, out of which the global macro system behaviour emerges. So far, the Haskell community hasn't been much in contact with the community of ABS due to the latter's primary focus on the object-oriented programming paradigm. This paper tries to bridge the gap between those two communities by introducing the Haskell community to the concepts of ABS. We do this by deriving an agent-based implementation for the simple SIR model from epidemiology. In our approach we leverage the basic concepts of ABS with functional reactive programming from Yampa which results in a surprisingly fresh, powerful and convenient EDSL for formulating ABS in Haskell.
\end{abstract}

\begin{IEEEkeywords}
Functional Reactive Programming, Agent-Based Simulation
\end{IEEEkeywords}

%*******************************************************************************
%*********************************** First Chapter *****************************
%*******************************************************************************

\chapter{Introduction}  %Title of the First Chapter
I noticed that it is pretty hard to convince an agent-based economics specialist who is not a computer scientist about a pure functional approach. My conjecture is that the implementation technique and method does not matter much to them because they have very little knowledge about programming and are almost always self-taught - they don't know about software-engineering, nothing about proper software-design and architecture, nothing about software-maintenance, nothing about unit-testing,... In the end they just "hack" the simulation in whatever language they are able to: C++, Visual Basic, Java or toolboxes like Netlogo. For them it is all about to \textit{get things done somehow} and not to get things done the right way or in a beautiful way - the way and the method doesn't matter, its just a necessary evil which needs to be done. Thus if functional programming could make their lives easier, then they will definitely welcome it. But functional programming is, i think, harder to learn and harder to understand - so one needs to provide an abstraction through EDSL. So I REALLY need to come up with convincing arguments why to use pure functional approaches in ACE THEY can understand, otherwise I will be lost and not heard (not published,...). \\

What ACE economists care for:

\begin{itemize}
\item Very: Qualitative modelling with quantitative results
\item Yes: Easy reproducibility
\item Likely: Reasoning about convergence?
\item Likely: EDSL
\end{itemize}

My contributions are: pure functional framework, functional agent-model for market-simulations, EDSL for market-simulations, qualitative / implicit modelling with quanitative results, reasoning in my framework about convergence \\

IDEA: could I develop non-causal modelling (models are expressed in terms of non-directed equations, modelled in signal-relations) to allow for qualitative modelling for the agent-based economists? See hybrid modelling paper of Yampa. \textbf{THIS WOULD BE A HUGE NOVEL CONTRIBUTION TO ACE ESPECIALLY WHEN COMBINED WITH AN EDSL AND PROVIDING FULL REFERENTIAL TRANSPARENCY TO KEEP THE ABILITY TO REASON ABOUT CONVERGENCE}. This should be covered in the "EDSL"-paper.

TODO: maybe i should really focus only on market models? otherwise too much? \\

central novelty of my PhD: model specification = runnable code. possible through EDSL. but only in specific subfield of ACE: market-models. need a functional description of the model, then translate it to model specification in EDSL and then run it to see dynamics. But: model specification moves closer to functional programming languages. \\

another novelty approach: model specification through qualitative instead of quantiative approaches. is this possible? \\

WHY FUNCTIONAL? "because its the ultimate approach to scientific computing": fewer bugs due to mutable state (why? is thos shown obkectively by someone?), shorter (again as above, productivity), more expressive and closer to math, EDSL, EDSL=model=simulation, better parallelising due to referental transparency, reasoning \\

scientific results need to be reproduced, especially when they have high impact. a more formal approach of specifying the model and the simulation (model=simulation) could lead to easier sharing and easier reporduction without ambigouites \\

pure functional agent-model \& theory, EDSL framework in Haskell for ACE

\begin{enumerate}
\item Which kind of problem do we have?
\item What aim is there? Solving the problem? 
\item How the aim is achieved by enumerating VERY CLEAR objectives.
\item What the impact one expects (hypothesis) and what it is (after results).
\end{enumerate}

Note: It is not in the interest of the researcher to develop new economic theories but to research the use of functional methods (programming and specification) in agent-based computational economics (ACE).

NOTE: Get the reader’s attention early in the introduction: motivation, significance, originality and novelty.

\section{Methods}
Methods need to be selected to implement the simulations. Special emphasis will be put on functional ones which will then be compared to established methods in the field of ABM/S and ACE. \\

Claim: non-programming environments are considered to be not powerful enough to capture the complexity of ACE implementations thus a programming approach to ACE will be always required.

\section{Scenarios}
To apply and test functional methods in ACE, four scenarios of ACE are selected and then the methods applied and compared with each other to see how each of them perform in comparison. The 4 selected scenarios represent a selection of the challenges posed in ACE: from very abstract ones to very operational ones.

\section{Comparison}
Each of the selected scenarios is then implemented using the selected methods where each solution is then compared against the following criteria: 

\begin{enumerate}
\item suitability for scientific computation
\item robustness
\item error-sources
\item testability
\item stability
\item extendability
\item size of code
\item maintainability
\item time taken for development
\item verification \& correctness
\item replications \& parallelism
\item EDSL
\end{enumerate}

This will then allow to compare the different methods against each other and to show under which circumstances functional methods shine and when they should not be used.

\section{Agent-Based Modelling and Simulation (ABM/S)}
ABM/S is a method of modelling and simulating a system where the global behaviour may be unknown but the behaviour and interactions of the parts making up the system is of knowledge (Wooldrige, M. (2009). An Introduction to MultiAgent Systems. John Wiley & Sons). Those parts, called agents, are modelled and simulated out of which then the aggregate global behaviour of the whole system emerges. Thus the central aspect of ABM/S is the concept of an Agent which can be understood as a metaphor for a pro-active unit, able to spawn new Agents, and interacting with other Agents in a network of neighbours by exchange of messages. The implementation of Agents can vary and strongly depends on the programming language and the kind of domain the simulation and model is situated in.

\section{Agent-Based Economics (ACE)}
According to Leigh Tesfatsion (Tesfatsion, L. (2006). Agent-based computational economics: A constructive approach to economic theory. In Tesfatsion, L. and Judd, K. L., editors, Handbook of Computational Economics, volume 2, chapter 16, pages 831–880. Elsevier, 1 edition.), one of the leading figures, ACE is "[...] computational modelling of economic processes (including whole economies) as open-ended dynamic systems of interacting agents." - thus lending perfectly to the use of ABM/S as already the name suggests. Whereas classical economic models fall short by only looking at the average, pure rational, individual interacting in anonymous markets, the ACE approach looks at heterogeneous, non-rational individuals interacting with each other in networks (Kirman, A. (2010). Complex Economics: Individual and Collective Rationality. Routledge, London ; New York, NY.). Thus ACE can be understood as a combination of computer-science, cognitive/social science and evolutionary economics.

\section{Functional programming}
TODO: read \cite{Backus1978}

The state-of-the-art approach to implementing Agents are object-oriented methods and programming as the metaphor of an Agent as presented above lends itself very naturally to object-orientation (OO). The author of this thesis claims that OO in the hands of inexperienced or ignorant programmers is dangerous, leading to bugs and hardly maintainable and extensible code. The reason for this is that OO provides very powerful techniques of organising and structuring programs through Classes, Type Hierarchies and Objects, which, when misused, lead to the above mentioned problems. Also major problems, which experts face as well as beginners are 1. state is highly scattered across the program which disguises the flow of data in complex simulations and 2. objects don’t compose as well as functions. The reason for this is that objects always carry around some internal state which makes it obviously much more complicated as complex dependencies can be introduced according to the internal state.
All this is tackled by (pure) functional programming which abandons the concept of global state, Objects and Classes and makes data-flow explicit. This then allows to reason about correctness, termination and other properties of the program e.g. if a given function exhibits side-effects or not. Other benefits are fewer lines of code, easier maintainability and ultimately fewer bugs thus making functional programming the ideal choice for scientific computing and simulation and thus also for ACE. A very powerful feature of functional programming is Lazy evaluation. It allows to describe infinite data-structures and functions producing an infinite stream of output but which are only computed as currently needed. Thus the decision of how many is decoupled from how to (Hughes, J. (1989). Why functional programming matters. Comput. J., 32(2):98–107.).
The most powerful aspect using pure functional programming however is that it allows the design of embedded domain specific languages (EDSL). In this case one develops and programs primitives e.g. types and functions in a host language (embed) in a way that they can be combined. The combination of these primitives then looks like a language specific to a given domain, in the case of this thesis ACE. The ease of development of EDSLs in pure functional programming is also a proof of the superior extensibility and composability of pure functional languages over OO (Henderson P. (1982). Functional Geometry. Proceedings of the 1982 ACM Symposium on LISP and Functional Programming.).
One of the most compelling example to utilize pure functional programming is the reporting of Hudak (Hudak P., Jones M. (1994). Haskell vs. Ada vs. C++ vs. Awk vs. ... An Experiment in Software Prototyping Productivity. Department of Computer Science, Yale University.)  where in a prototyping contest of DARPA the Haskell prototype was by far the shortest with 85 lines of code. Also the Jury mistook the code as specification because the prototype did actually implement a small EDSL which is a perfect proof how close EDSL can get to and look like a specification.

Functional languages can best be characterized by their way computation works: instead of \textit{how} something is computed, \textit{what} is computed is described. Thus functional programming follows a declarative instead of an imperative style of programming. The key points are:
\begin{itemize}
\item No assignment statements - variables values can never change once given a value.
\item Function calls have no side-effect and will only compute the results - this makes order of execution irrelevant, as due to the lack of side-effects the logical point in \textit{time} when the function is calculated within the program-execution does not matter.
\item higher-order functions
\item lazy evaluation
\item Looping is achieved using recursion, mostly through the use of the general fold or the more specific map.
\item Pattern-matching
\end{itemize}

This alone does not really explain the \textit{real} advantages of functional programming and one must look for better motivations using functional programming languages. One motivation is given in \cite{Hughes1989} which is a great paper explaining to non-functional programmers what the significance of functional programming is and helping functional programmers putting functional languages to maximum use by showing the real power and advantages of functional languages. The main conclusion is that \textit{modularity}, which is the key to successful programming, can be achieved best using higher-order functions and lazy evaluation provided in functional languages like Haskell. \cite{Hughes1989} argues that the ability to divide problems into sub-problems depends on the ability to glue the sub-problems together which depends strongly on the programming-language and \cite{Hughes1989} argues that in this ability functional languages are superior to structured programming.

TODO: comparison of functional and object-oriented programming. My points are:
\begin{itemize}
\item The way state can be changed and treated - distributed over multiple objects - is often very difficult to understand.
\item Inheritance is a dangerous thing if not used with care because inheritance introduces very strong dependencies which cannot be changed during runtime anymore.
\item Objects don't compose very well: \url{http://zeroturnaround.com/rebellabs/why-the-debate-on-object-oriented-vs-functional-programming-is-all-about-composition/}
\item (Nearly) impossible to reason about programs
\end{itemize}

In conclusion the upsides of functional programming as opposed to OO are:
\begin{itemize}
\item Much more explicit flow of data \& control
\item Much better compose-able
\item Much better parallelism
\end{itemize}

\section{Related Research}
Tim Sweeney, CTO of Epic Games gave an invited talk about how "future programming languages could help us write better code" by "supplying stronger typing, reduce run-time failures;  and the need for pervasive concurrency support, both implicit and explicit, to effectively exploit the several forms of parallelism present in games and graphics." \cite{Sweeney2006}. Although the fields of games and agent-based simulations seem to be very different in the end, they have also very important similarities: both are simulations which perform numerical computations and update objects - in games they are called "game-objects" and in abm they are called agents but they are in fact the same thing - in a loop either concurrently or sequential. His key-points were:

\begin{itemize}
\item Dependent types as the remedy of most of the run-time failures.
\item Parallelism for numerical computation: these are pure functional algorithms, operate locally on mutable state. Haskell ST, STRef solution enables encapsulating local heaps and mutability within referentially transparent code.
\item Updating game-objects (agents) concurrently using STM: update all objects concurrently in arbitrary order, with each update wrapped in atomic block - depends on collisions if performance goes up.
\end{itemize}

\section{Defining Agent-Based Simulation}
Agent-Based Simulation (ABS) is a methodology to model and simulate a system where the global behaviour may be unknown but the behaviour and interactions of the parts making up the system is of knowledge. Those parts, called agents, are modelled and simulated out of which then the aggregate global behaviour of the whole system emerges. So the central aspect of ABS is the concept of an agent which can be understood as a metaphor for a pro-active unit, situated in an environment, able to spawn new agents and interacting with other agents in some neighbourhood by exchange of messages \cite{wooldridge_introduction_2009}. We informally assume the following about our agents TODO: need some references here, we cannot claim this without citation here (cite Peers book):

\begin{itemize}
	\item They are uniquely addressable entities with some internal state over which they have full, exclusive control.
	\item They are pro-active which means they can initiate actions on their own e.g. change their internal state, send messages, create new agents, terminate themselves.
	\item They are situated in an environment and can interact with it.
	\item They can interact with other agents which are situated in the same environment by means of messaging.
\end{itemize} 

Epstein \cite{epstein_generative_2012} identifies ABS to be especially applicable for analysing \textit{"spatially distributed systems of heterogeneous autonomous actors with bounded information and computing capacity"}. Thus in the line of the simulation types \textit{Statistic} $^\dag$, \textit{Markov} $^\ddag$, \textit{System Dynamics} $^\S$, \textit{Discrete Event} $^\mp$, ABS is the most powerful one as it allows to model  the following:

\begin{itemize}
	\item Linearity \& Non-Linearity $^{\dag \ddag \S \mp}$ - the dynamics of the simulation can exhibit both linear and non-linear behaviour. 
	\item Time $^{\dag \ddag \S \mp}$ - agents act over time, time is also the source of pro-activity.
	\item States $^{\ddag \S \mp}$ - agents encapsulate some state which can be accessed and changed during the simulation.
	\item Feedback-Loops $^{\S \mp}$ - because agents act continuously and their actions influence each other and themselves, feedback-loops are the norm in ABS. 
	\item Heterogeneity $^{\mp}$ - although agents can have same properties like height, sex,... the actual values can vary arbitrarily between agents.
	\item Interactions - agents can be modelled after interactions with an environment or other agents, making this a unique feature of ABS, not possible in the other simulation models.
	\item Spatiality \& Networks - agents can be situated within e.g. a spatial (discrete 2d, continuous 3d,...) or network environment, making this also a unique feature of ABS, not possible in the other simulation models.
\end{itemize}

\section{The SIR Model}
To explain the concepts of ABS and of our functional reactive approach to it, we introduce the SIR model as a motivating example. It is a very well studied and understood compartment model from epidemiology which allows to simulate the dynamics of an infectious disease spreading through a population. In this model, people in a population of size $N$ can be in either one of three states \textit{Susceptible}, \textit{Infected} or \textit{Recovered} at a particular time, where it is assumed that initially there is at least one infected person in the population. People interact with each other \textit{on average} with a given rate $\beta$ per time-unit and get infected with a given probability $\gamma$ when interacting with an infected person. When infected, a person recovers \textit{on average} after $\delta$ time-units and is then immune to further infections. An interaction between infected persons does not lead to re-infection, thus these interactions are ignored in this model. This definition gives rise to three compartments with the transitions as seen in Figure \ref{fig:sir_transitions}.

\begin{figure}
	\centering
	\includegraphics[width=.4\textwidth, angle=0]{./../shared/fig/diagrams/SIR_transitions.png}
	\caption{Transitions in the SIR compartment model.}
	\label{fig:sir_transitions}
\end{figure}

The dynamics of this model over time can be formalized using the System Dynamics (SD) approach which models a system through differential equations. For the SIR model we get the following equations:

\begin{align*}
\frac{\mathrm d S}{\mathrm d t} &= -infectionRate \\ 
\frac{\mathrm d I}{\mathrm d t} &= infectionRate - recoveryRate \\ 
\frac{\mathrm d R}{\mathrm d t} &= recoveryRate 
\end{align*}

\begin{align*}
infectionRate &= \frac{I \beta S \gamma}{N} \\
recoveryRate &= \frac{I}{\delta} 
\end{align*}

Solving these equations is then done by integrating over time. In the SD terminology, the integrals are called \textit{Stocks} and the values over which is integrated over time are called \textit{Flows} \footnote{The $1+$ in $I(t)$ amounts to the initially infected agent - if there wouldn't be a single infected one, the system would immediately reach equilibrium.}.

\begin{align*}
S(t) &= N + \int_0^t -infectionRate\, \mathrm{d}t \\
I(t) &= 1 + \int_0^t infectionRate - recoveryRate\, \mathrm{d}t \\
R(t) &= \int_0^t recoveryRate\, \mathrm{d}t
\end{align*}

There exist a huge number of software packages which allow to conveniently express SD models using a visual approach like in Figure \ref{fig:sir_sd_stockflow_diagramm}.

\begin{figure}
	\centering
	\includegraphics[width=.4\textwidth, angle=0]{./../shared/fig/diagrams/SIR_SD_STOCKFLOW_DIAGRAMM.png}
	\caption{A visual representation of the SD stocks and flows of the SIR compartment model. Picture taken using AnyLogic Personal Learning Edition 8.1.0.}
	\label{fig:sir_sd_stockflow_diagramm}
\end{figure}

Running the SD simulation over time results in the dynamics as shown in Figure \ref{fig:sir_sd_dynamics} with the given variables.

\begin{figure}
	\centering
	\includegraphics[width=.4\textwidth, angle=0]{./../shared/fig/frsd/SIR_SD_1000agents_150t_001dt.png}
	\caption{Dynamics of the SIR compartment model using the System Dynamics approach. Population Size $N$ = 1000, contact rate $\beta =  \frac{1}{5}$, infection probability $\gamma = 0.05$, illness duration $\delta = 15$ with initially 1 infected agent. Simulation run for 150 time-steps.}
	\label{fig:sir_sd_dynamics}
\end{figure}

\subsection*{An Agent-Based approach}
The SD approach is inherently top-down because the emergent property of the system is formalized in differential equations. The question is if such a top-down behaviour can be emulated using ABS, which is inherently bottom-up. Also the question is if there are fundamental drawbacks and benefits when doing so using ABS. Such questions were asked before and modelling the SIR model using an agent-based approach is indeed possible. It is important to note that SD treats the population completely continuous which results in non-discrete values of stocks e.g. 3.1415 infected persons. Thus the fundamental approach to map the SIR model to an ABS is to discretize the population and model each person in the population as an individual agent. The transition  between the states are no longer happening according to continuous differential equations but due to discrete events caused both by interactions amongst the agents and time-outs.

\begin{itemize}
	\item Every agent makes \textit{on average} contact with $\beta$ random other agents per time unit. In ABS we can only contact discrete agents thus we model this by generating a random event on average every $\beta$ time units. Note that we need to sample from an exponential CDF because the rate is proportional to the size of the population as \cite{borshchev_system_2004} pointed out.
	
	\item An agent does not know the other agents' state when making contact with it, thus we need a mechanism in which agents reveal their state in which they are in \textit{at the moment of making contact}. Obviously the already mentioned messaging which allows agents to interact is perfectly suited to do this.
	\begin{itemize}
		\item \textit{Susceptibles}: These agents make contact with other random agents (excluding themselves) with a "Susceptible" message. They can be seen to be the drivers of the dynamics.
		\item \textit{Infected}: These agents only reply to incoming "Susceptible" messages with an "Infected" message to the sender. Note that they themselves do \textit{not} make contact pro-actively but only react to incoming one. 
		\item \textit{Recovered}: These agents do not need to send messages because contacting it or being contacted by it has no influence on the state.
	\end{itemize}
	
	\item Transition of susceptible to infected state - a susceptible agent needs to have made contact with an infected agent which happens when it receives an "Infected" message. If this happens an infection occurs with a probability of $\gamma$. The infection can be calculated by drawing $p$ from a uniform random-distribution between 0 and 1 - infection occurs in case of $\gamma >= p $. Note that this needs to be done for \textit{every} received "Infected" message.
	
	\item Transition of infected to recovered - a person recovers \textit{on average} after $\delta$ time unites. This is implemented by drawing the duration from an exponential distribution \cite{borshchev_system_2004} with $\lambda = \frac{1}{\delta}$ and making the transition after this duration.
\end{itemize}

In Figure \ref{fig:sir_abs_anylogic_agents} we give the dynamics simulating the SIR model with the agent-based approach.

%TODO: replace these pictures with ones generated by FrABS
%
%TODO: reproducing about the same dynamics of the SD-solution (1.0 dt)
%	- super-sampling: 	contact-rate ss high, illness time-out low 
%	- agent number:		1000 vs. 10.000 agents
%	- Susceptibles making contact and infected response VS. only Infected make contact
%	- update-strat:		Sequential vs. Parallel
%	- making contact: susceptible only vs. susceptible AND infected
%	- do conversations make a difference?
%	- does a delayed switch (dSwitch) in transitions makes a difference?
	
\begin{figure*}
\begin{center}

	\begin{tabular}{c c c}
		\begin{subfigure}[b]{0.3\textwidth}
			\centering
			\includegraphics[width=1\textwidth, angle=0]{./../shared/fig/frabs/SIR_100agents_150t_01dt_parallel.png}
			\caption{100 Agents}
			\label{fig:pd_seq}
		\end{subfigure}
    	&
		\begin{subfigure}[b]{0.3\textwidth}
			\centering
			\includegraphics[width=1\textwidth, angle=0]{./../shared/fig/frabs/SIR_1000agents_150t_01dt_parallel.png}
			\caption{1,000 Agents}
			\label{fig:pd_seq}
		\end{subfigure}
    	&
		\begin{subfigure}[b]{0.3\textwidth}
			\centering
			\includegraphics[width=1\textwidth, angle=0]{./../shared/fig/frabs/SIR_10000agents_150t_01dt_parallel.png}
			\caption{10,000 Agents}
			\label{fig:hac_seq}
		\end{subfigure}
	\end{tabular}
	
	\caption{Approximating the continuous dynamics of the system dynamics simulation using the agent-based approach. Model-parameters are the same ($\beta = \frac{1}{5}$, $\gamma = 0.05$, $\delta = 15$ with initially 1 infected agent) except population size. All simulations run for 150 time-steps.} 
	\label{fig:sir_abs_anylogic_agents}
\end{center}
\end{figure*}

As previously mentioned the agent-based approach is a discrete one which means that with increasing number of agents, the discrete dynamics approximate the continuous dynamics of the SD simulation. Still the dynamics of 10,000 Agents do not match the dynamics of the SD simulation perfectly. This is because as opposed to the SD simulation the agent-based approach is inherently a stochastic one as we continuously draw from random-distributions which drive our state-transitions. What we see in Figure \ref{fig:sir_abs_anylogic_agents} is then just a single run where the dynamics would result in slightly different shapes when run with a different random-number generator seed. The agent-based approach thus generates a distribution of dynamics over which ones needs to average to arrive at the correct solution. This can be done using replications in which the simulation is run with the exact same parameters multiple times but each with a different random-number generator see. The resulting dynamics are then averaged and the result is then regarded as the correct dynamics.
We have done this as can be seen in Figure \ref{fig:sir_abs_agents_repls}, using 10 replications, which matches the SD dynamics to a very satisfactory level \footnote{Note that in the replications we are using 10 initially infected agents to ensure that no simulation run will terminate too early (meaning that the disease gets extinct after a few time steps) which would offset the dynamics completely. This happens due to "unlucky" random distributions which can be repaired by introducing more initially infected agents which increases the probability of spreading the disease in the very early stage of the simulation drastically. We found that when using 10 initially infected agents in a population of 10,000 (which amounts to 0.1\%) is enough to never result in an early terminating simulation. This is also a fundamental difference between SD and ABS: the dynamics of the agent-based approach can result in a wide range of scenarios which includes also the one in which the disease gets extinct in the early stages (a lucky coincidence for mankind) - this is simply not possible in the SD approach. So we can argue that ABS is much closer to reality than SD as it allows to explore alternate futures in the dynamics.}.

\begin{figure*}
\begin{center}

	\begin{tabular}{c c c}
		\begin{subfigure}[b]{0.3\textwidth}
			\centering
			\includegraphics[width=1\textwidth, angle=0]{./../shared/fig/frabs/SIR_100agents_150t_01dt_parallel_10replications.png}
			\caption{100 Agents}
			\label{fig:sir_abs_agents_repls_100}
		\end{subfigure}
    	&
		\begin{subfigure}[b]{0.3\textwidth}
			\centering
			\includegraphics[width=1\textwidth, angle=0]{./../shared/fig/frabs/SIR_1000agents_150t_01dt_parallel_10replications.png}
			\caption{1,000 Agents}
			\label{fig:sir_abs_agents_repls_1000}
		\end{subfigure}
    	&
		\begin{subfigure}[b]{0.3\textwidth}
			\centering
			\includegraphics[width=1\textwidth, angle=0]{./../shared/fig/frabs/SIR_1000agents_150t_01dt_parallel_10replications.png}
			\caption{TODO: create run with 10 replications of 10,000 Agents}
			\label{fig:sir_abs_agents_repls_10000}
		\end{subfigure}
	\end{tabular}
	
	\caption{Dynamics of Figure \ref{fig:sir_abs_anylogic_agents} averaged over 10 replications with initially 10 infected agents.} 
	\label{fig:sir_abs_agents_repls}
\end{center}
\end{figure*}

For a more in-depth introduction of how to approximate an SD model by ABS see \cite{macal_agent-based_2010} who discusses a general approach and how to compare dynamics and \cite{borshchev_system_2004} which explain the need to draw the illness-duration from an exponential-distribution.

We will derive the implementation of this approach in the next section, with the complete code provided in Appendix \ref{app:abs_code}. As will be seen our approach allows us express this behaviour very explicitly and is looking very much like a formal ABS specification of the problem. 

\section{Functional Reactive SIR}
In this section we will derive the implementation of the agent-based approach to the SIR model, with the complete code provided in Appendix \ref{app:abs_code}. As will be seen our approach allows us express the agent behaviour very explicitly and is looking very much like a formal ABS specification of the problem. The challenges one faces when implementing an ABS plain, without support from a library are manifold. Generally one faces the following challenges:

\begin{itemize}
	\item Agent Representation \ref{sub:agent_rep} - how do we represent an agent in Haskell?
	\item Agent-Agent Interaction \ref{sub:agent_agent_inter} - how can agents interact with other agents in Haskell without resorting to the IO Monad?
	\item Environment representation \ref{sub:env_rep} - how can we represent an environment which must have the ability to update itself e.g. regrow some resources?
	\item Agent-Environment interaction \ref{sub:agent_env_inter} - how can agents interact (read / write) with the environment?
	\item Agent Updating \ref{sub:agent_updt} - how are the agents organised, how are they updated and how is it managed (deleting, adding during simulation) in Haskell without resorting to the IO Monad?
\end{itemize}

In the next subsections we will discuss each point by deriving a functional reactive implementation of the agent-based SIR model. For us it is absolutely paramount that the simulation should be pure and never run in the IO Monad (except of course the surrounding Yampa loop which allows rendering and output). The complete source-code can be seen in Appendix \ref{app:abs_code}. For this research we implemented a prototype library called \textit{FrABS} which is available under \url{https://github.com/thalerjonathan/phd/tree/master/coding/libraries/frABS}. We plan on releasing it on Hackage in the future.

\subsection{Agent Representation}
\label{sub:agent_rep}
An agent can be seen as a tuple $<id, s, m, e, b>$.
\begin{itemize}
	\item \textbf{id} - the unique identifier of the agent
	\item \textbf{s} - the generic state of the agent
	\item \textbf{m} - the messages the agent understands
	\item \textbf{e} - the environment the agent can interact with
	\item \textbf{b} - the behaviour of the agent
\end{itemize}

The id is simply represented as an Integer and must be unique for all currently existing agents in the system as it is used for message delivery. %A stronger requirement would be that the id of an agent is unique for the whole simulation-run and will never be reused - this would support replications and operations requiring unique agent-ids.

Each agent may have a generic state which could be represented by any data type or compound data. A SIR agent's state can be represented using the an Algebraic Data Type as follows:
\begin{minted}[fontsize=\footnotesize]{haskell}
data SIRState = Susceptible | Infected | Recovered
\end{minted}

The behaviour of the agent is a signal-function which maps a tuple of an AgentIn and the environment to an AgentOut and the environment. It has the following signature \footnote{Note that we omit the type-parameters in the following code-listings unless it is needed for clarity. Still it is important to keep in mind that all AgentIn and AgentOut are parameterised with \textit{s} representing the type of its state, \textit{m} representing the type of the messages and \textit{e} representing the type of environment.} 
\begin{minted}[fontsize=\footnotesize]{haskell}
type AgentBehaviour s m e = SF (AgentIn s m e, e) (AgentOut s m e, e)
\end{minted}

\textit{AgentIn} provides the necessary data to the agent-behaviour: its \textit{id}, incoming messages, the current state \textit{s} and a random-number generator. \textit{AgentOut} allows the agent to communicate changes: kill itself, create new agents, sending messages, an updated state \textit{s} and a changed random-number generator. Both types are opaque and access to them is only possible through the provided functions. The behaviour also gets the environment passed in, which the agent can read and also write by changing it and returning it along side the \textit{AgentOut}. It is important to note that the environment is completely generic and we do not induce any type bounds on it. Obviously \textit{AgentIn} is read-only whereas \textit{AgentOut} is both read- and write-able. The first thing an agent-behaviour does is creating the default AgentOut from the existing AgentIn as is done in line 94 in Appendix \ref{app:abs_code}.

\begin{minted}[fontsize=\footnotesize]{haskell}
agentOutFromIn :: AgentIn -> AgentOut
\end{minted}

This will copy the relevant fields over to \textit{AgentOut} on which one now primarily acts. The read-only and read/write character of both types is also reflected in the EDSL where most of the functions implemented also work that way: they may read the \textit{AgentIn} and read/write an \textit{AgentOut}. Relevant functions for working on the agent-definition are:

\begin{minted}[fontsize=\footnotesize]{haskell}
type AgentId = Int

agentId :: AgentIn -> AgentId
kill :: AgentOut -> AgentOut
isDead :: AgentOut -> Bool
onStart :: (AgentOut -> AgentOut) -> AgentIn -> AgentOut -> AgentOut

agentState :: AgentOut -> s
setAgentState :: s -> AgentOut -> AgentOut
updateAgentState :: (s -> s) -> AgentOut -> AgentOut

createAgent :: AgentDef -> AgentOut -> AgentOut
\end{minted}

The function \textit{kill} marks an agent for removal after the current iteration. The function \textit{isDead} checks if the agent is marked for removal. The function \textit{onStart} allows to change the \textit{AgentOut} in the case of the start-event which happens on the very first time the agent runs. The function \textit{agentState} returns the agents' state \textit{s}, \textit{setAgentState} allows to change the state of the agent by overriding it and \textit{updateAgentState} allows to change it by keeping parts of it. The function \textit{createAgent} allows to add an agent-definition where \textit{AgentDef} simply contains the initial state, behaviour and id of the agent, amongst others - see line 50 - 57 in Appendix \ref{app:abs_code}. to the AgentOut which results in creating a new agent from the given definition which will be active in the next iteration of the simulation. 

Having these functions we build some reactive primitives into our EDSL meaning that they return signal-functions themselves. We start with the following functions:

\begin{minted}[fontsize=\footnotesize]{haskell}
doOnce :: (AgentOut -> AgentOut) -> SF AgentOut AgentOut
doOnceR :: AgentBehaviour -> AgentBehaviour
doNothing :: AgentBehaviour

setAgentStateR :: s -> AgentBehaviour
updateAgentStateR :: (s -> s) -> AgentBehaviour
\end{minted}

The \textit{doOnce} function may seem strange at first but allows conveniently make actions (which are changing the \textit{AgentOut}) only once e.g. when making the transition from Susceptible to Infected changing the state to Infected just once as can be seen in line 114 of Appendix \ref{app:abs_code}. A more striking example would be to send a message just once after a transition. The \textit{doOnceR} function is the reactive version, which allows to run an agent behaviour only once. The function \textit{doNothing} provides a convenient way of an agent sink which is basically an agent which does literally nothing - the resulting agent behaviour just transforms the \textit{AgentIn} to \textit{AgentOut} using the previously mentioned function \textit{agentOutFromIn}.

Often we want some more reactive behaviour e.g. making a transition from one behaviour to another on a given event. For this we provide the following:

\begin{minted}[fontsize=\footnotesize]{haskell}
type EventSource = SF (AgentIn, AgentOut) (AgentOut, Event ())
transitionOnEvent :: EventSource -> AgentBehaviour -> AgentBehaviour -> AgentBehaviour
\end{minted}

The function \textit{transitionOnEvent} takes an event-source which creates the event, an agent behaviour which is run until the event hits and an agent behaviour which is run at the event and after. The event-source is a signal-function itself to allow maximum of flexibility and gets both \textit{AgentIn} and \textit{AgentOut} and returns a (potentially changed) \textit{AgentOut} and the event upon to switch. 
This function is used for implementing the susceptible agent where we use a \textit{transitionOnEvent} and a specific event-source which generates an event when the susceptible agent got infected as can be seen in lines 81-90 in Appendix \ref{app:abs_code}.

Sometimes we need our transition event to rely on time-semantics e.g. in SIR where an infected agent recovers \textit{on average} after $\delta$ time-units. For this we provide the following function which can be seen in line 104 in Appendix \ref{app:abs_code}:

\begin{minted}[fontsize=\footnotesize]{haskell}
transitionAfterExp :: RandomGen g => g -> Double -> AgentBehaviour -> AgentBehaviour -> AgentBehaviour
\end{minted}

It takes a random-number generator, the \textit{average} time-out, the behaviour to run before the time-out and the behaviour to run after the time-out where the function will return then the according behaviour. For implementing this behaviour we initially used Yampas \textit{after} function which generates an event after given time-units but this would not result in the correct dynamics as we rather need to create a random-distribution of time-outs than a deterministic time-out which occurs always after the same time. For this we implemented our own function, called \textit{afterExp}, which now takes a random-number generator a time-out and some value of type b and creates a signal-function which ignores its input and creates an event \textit{on average} after DTime.

\begin{minted}[fontsize=\footnotesize]{haskell}
afterExp :: RandomGen g => g -> DTime -> b -> SF a (Event b)
\end{minted}

\subsection{Agent-Agent Interaction}
\label{sub:agent_agent_inter}

Agent-agent interaction is the means of an agent to directly address another agent and vice versa. Inspired by the actor model we implement  \textit{messaging} with share-nothing semantics. In this case the agent sends messages which will arrive at the receiver in the next step of the simulation, thus being kind of asynchronous - a round-trip would always take at least two steps, independent of the sampling time. Depending on the semantics of the model we sometimes need synchronous interactions e.g. when only one agent can change the environment or decisions need to be made within one step - this wouldn't be possible with the asynchronous messaging. For this we introduced the concept of \textit{conversations} which allow two agents to interact with each other for an arbitrary number of requests and replies without the simulation being advanced - time is halted and only the two agents are active until they finish their conversation.

\subsubsection{Messaging}
Each Agent can send a message to another agent through \textit{AgentOut} where incoming messages are queued in the \textit{AgentIn} in unspecified order and can be processed when the agent is running the next time. The agent is free to ignore the messages and if it does not process them in the current step, they will simply be lost. This is in fundamental contrast to the actor model where messages stay in the message-box of the receiving actor until the actor has processed them. We chose a different approach as time has a different meaning in ABS than in a system of actors where there is basically no global notion of time.
Note that due to the fact we don't have method-calls in FP, messaging will always take some time, which depends on the sampling interval of the system. This was not obviously clear when implementing ABS in an object-oriented way because there we can communicate through method calls which are a way of interaction which takes no simulation-time.
For messaging, we need a set of messages \textit{m} the agents understand. In the case of the SIR model we simply use the following:

\begin{minted}[fontsize=\footnotesize]{haskell}
data SIRMsg = Contact SIRState
\end{minted}

In addition we provide the following functions in our EDSL to support messaging.

\begin{minted}[fontsize=\footnotesize]{haskell}
type AgentMessage m = (AgentId, m)

sendMessage :: AgentMessage m -> AgentOut -> AgentOut
sendMessageTo :: AgentId -> m -> AgentOut -> AgentOut
sendMessages :: [AgentMessage m] -> AgentOut -> AgentOut
broadcastMessage :: m -> [AgentId] -> AgentOut -> AgentOut

hasMessage :: (Eq m) => m -> AgentIn -> Bool
onMessage :: (AgentMessage -> acc -> acc) -> AgentIn -> acc -> acc
onMessageFrom :: AgentId -> (AgentMessage -> acc -> acc) -> AgentIn -> acc -> acc
onMessageType :: (Eq m) => m -> (AgentMessage -> acc -> acc) -> AgentIn -> acc -> acc
\end{minted}

Most of the functions are pretty self-explanatory, we will shortly explain the \textit{onMessage*}. The function \textit{onMessage} provides a way to react to incoming messages by using a callback function which manipulates an accumulator, thus resembling the workings of fold. The functions \textit{onMessageFrom} and \textit{onMessageType} provide the same functionality but filter the messages accordingly. We can now write implement the functionality of an infected agent which replies to an incoming \textit{Contact} message with another \textit{Contact Infected} message as can be seen in line 73 of Appendix \ref{app:abs_code}.

Sometimes we also need discrete semantics like changing the behaviour of an agent on reception of a specific message. For this we provide the function \textit{transitionOnMessage} which works the same way as \textit{transitionOnEvent} but now on a message instead.

\begin{minted}[fontsize=\footnotesize]{haskell}
transitionOnMessage :: (Eq m) => m -> AgentBehaviour -> AgentBehaviour -> AgentBehaviour
\end{minted}

Of course messaging sometimes may have specific time-semantics as in our SIR model. There susceptible agents make contact with $\beta$ other agents on average \textit{per time unit}. To implement this we randomly need to generate messages with a given frequency within some time-interval by drawing from the exponential random-distribution. This is already supported by Yampa using \textit{occasionally} and we have built on it a the following:

\begin{minted}[fontsize=\footnotesize]{haskell}
type MessageSource m e = e -> AgentOut -> (AgentOut, AgentMessage m)

sendMessageOccasionallySrc :: RandomGen g => g -> Double -> MessageSource -> SF (AgentOut, e) AgentOut

constMsgReceiverSource :: m -> AgentId -> MessageSource 
randomNeighbourNodeMsgSource :: m -> MessageSource s m (Network l)
randomNeighbourCellMsgSource :: (s -> Discrete2dCoord) -> m -> Bool -> MessageSource s m (Discrete2d AgentId)
randomAgentIdMsgSource :: m -> Bool -> MessageSource s m [AgentId]
\end{minted}

The function \textit{sendMessageOccasionallySrc} takes a random-number generator, the frequency of messages to generate \textit{on average per time-unit} a message-source and returns a signal-function which takes a tuple of an \textit{AgentOut} and environment and returns an \textit{AgentOut}. This signal-function which performs the actual generating of the messages needs to be fed in the tuple but only returns the changed \textit{AgentOut} but not the environment - this guarantees statically at compile-time that the environment cannot be changed in this process. This is also directly reflected in the type of \textit{MessageSource} which takes an environment and \textit{AgentOut} and returns a tuple with a changed \textit{AgentOut} and a message. We provide pre-defined messages-sources like \textit{constMsgReceiverSource} which always generates the same message, \textit{randomNeighbourNodeMsgSource} which picks a random neighbour from a network-environment (see below), \textit{randomNeighbourCellMsgSource} which picks a random neighbour from a discrete 2D grid environment (see below) and \textit{randomAgentIdMsgSource} which randomly picks an element from an environment which is a list of \textit{AgentId} (omitting the sender True/False). The susceptible agent builds on this function to make contact with other agents as can be seen in line 95-99 of Appendix \ref{app:abs_code}.

\subsubsection{Conversations}
The messaging as implemented above works well for one-directional, virtual asynchronous interaction where we don't need a reply at the same time. A perfect use-case for messaging is making contact with neighbours in the SIRS-model: the agent sends the contact message but does not need any response from the receiver, the receiver handles the message and may get infected but does not need to communicate this back to the sender. 
A different case is when agents need to transact in the same time-step or interact over multiple steps: agent A interacts with agent B where the semantics of the model need an immediate response from agent B - which can lead to further interactions initiated by agent A. An example would be negotiating a trading price between two agents to buy and sell goods between each other and then execute the trade. This must happen in the same time-step as constraints need to be considered which could be violated in asynchronous interactions. Basically the concept is always the same and is rooted in the fact that these interactions needs to transact in the current time-step as all of the actions only work on a 1:1 relationship and could violate resource-constraints.
For this we introduce the concept of a \textit{conversation} between agents. It allows an agent A to initiate a conversation with another agent B in which the simulation is virtually halted and both can exchange an arbitrary number of messages through calling and responding without time passing, something not possible without this concept because in each iteration the time advances. After either one agent has finished with the conversation it will terminate it and the simulation will continue with the updated agents. It is important to understand that \textit{both} agents can change their state and the envirionment in a conversation. The conversation concept is implemented at the moment in the way that the initiating agent A has all the freedom in sending messages, starting a new conversation,... but that the receiving agent B is only able to change its state but is not allowed to send messages or start conversations in this process. Technically speaking: agent A can manipulate an \textit{AgentOut} whereas agent B can only read its \textit{AgentIn} and manipulate its state.
When looking at conversations they may look like an emulation of method-calls but they are more powerful: a receiver can be unavailable to conversations or simply refuse to handle this conversation. This follows the concept of an active actor which can decide what happens with the incoming interaction-request, instead of the passive object which cannot decide whether the method-call is really executed or not.

\begin{minted}[fontsize=\footnotesize]{haskell}
type AgentConversationReceiver s m e = AgentIn -> e -> AgentMessage m -> Maybe (s, m, e)
type AgentConversationSender m e = AgentOut -> e -> Maybe (AgentMessage m) -> (AgentOute, e)
                                        
conversation :: AgentMessage m -> AgentConversationSender -> AgentOut -> AgentOut
conversationEnd :: AgentOut -> AgentOut 
\end{minted}

The conversation sender is the initiator of the conversation which can only be an agent which is just run and has started a conversation with a call to the function \textit{conversation}. After the agent has run, the simulation system will detect the request for a conversation and start processing it by looking up the receiver and calling the functions with passing the values back and forth. While a conversation is active, time does not advance and other agents do not act. Note that due to its nature, conversations are only available when using the \textit{sequential} update-strategy (see below). Note that conversations are inherently non-reactive because they are synchronous interactions, involve no time-semantics because time is halted, thus it makes no sense to implement conversations as signal-functions.

\subsection{Environment representation}
\label{sub:env_rep}

Agents have access to an environment which itself is \textit{not} an agent - although it can have its own behaviour e.g. regrowing resources, it cannot send or receive messages from agents. Thus we treat the environment completely generic by allowing any given type captured in the type-variable \textit{e}. Environment behaviour is optional but if required, implemented using a signal-function which simply maps \textit{e} to \textit{e}:

\begin{minted}[fontsize=\footnotesize]{haskell}
type EnvironmentBehaviour e = SF e e
\end{minted}

By allowing a signal-function as the environment behaviour gives us the opportunity to implement reactive behaviour and time-semantics in environments as well. Again it is essential to note that throughout the whole simulation implementation we never put any bounds on the environment nor make assumptions about its type.

Although environments can be anything, even be of unit-type \textit{()} if no environment is required at all, there exist a few standard environments in ABS which are provided in all ABS packages. We provide implementations for them and discuss them below. Note that we don't provide the APIs of the environments here as it is out of the scope of this paper.

\subsubsection{Network}
A network environment gives agents access to a network, represented by a graph where the nodes are agent-ids and the edges represent neighbourhood information. We implemented fully-connected, Erdos-Renyi and Barbasi-Albert networks. In our case the networks are undirected and the labels can be labelled, carrying arbitrary data or being unlabelled, having unit-type. Agents can then perform the usual graph algorithms on these networks. 

\subsubsection{Discrete 2D}
A discrete 2d environment gives agents access to a 2D grid with dimensions of $N \times M \in \mathbb{N}$ cells. The cells are of a generic type $c$ and can thus be anything from \textit{AgentId} to resource-sites with single or multiple occupants. Such an environment has a defined neighbourhood of either Moore (8 neighbours) or Von-Neumann (4 neighbours). Agents can then query the environment for cells using neighbourhoods, radius or specific positions, change the cells and update them in the environment.

\subsubsection{Continuous 2D}
A continuous 2d environment gives agents access to a continuous 2D space with dimensions of $N \times M \in \mathbb{R}$. This space can contain an arbitrary number of objects of the generic type \textit{o} where each of them has a coordinate within this space. Agents can query for objects within a given radius, add, remove and update them. Also we provide functions to move objects either in a given or random direction.

\subsection{Agent-Environment interaction}
\label{sub:agent_env_inter}
In ABS agents almost always are \textit{situated within} an environment. We follow a subtle different approach and implement it in a way that agents have access to a generic environment of type $e$ as discussed above instead of being situated within. It is important to note the subtle difference of agents having \textit{access to} the environments instead of \textit{being situated} within them. This allows to free us making assumptions within an environment how agents use these environments and also allows us to stack multiple environments e.g. agents moving on a discrete 2D grid but relying on neighbourhood from a network.
Our SIR implementation uses a list of all \textit{AgentId} as the environment which means that every agent knows all the existing agents of the simulation and can address them - see line 6 in Appendix \ref{app:abs_code}. We could have used a Network environment using a fully-connected graph but the memory-consumptions of the library \textit{FGL} we are using for graphs are unacceptable in case of fully-connected networks of a larger numbers of agents (10,000). Each agent gets the environment passed in through the \textit{AgentIn} and can change it by passing a changed version of the environment out through \textit{AgentOut}. 

\subsection{Agent Updating}
\label{sub:agent_updt}
For agents to be pro-active, they need to be able to perceive time. Also agents must have the opportunity to react to incoming messages and manipulate the environment. The work of (TODO: cite our own paper on update-strategies) identifies four possible ways of doing this where we only implemented the \textit{sequential-} and \textit{parallel-strategy}. The other two strategies being the  \textit{concurrent-} and \textit{actor-strategy}, both requiring to run within the STM-Monad, which is not possible with Yampa. The author of \cite{perez_functional_2016} implemented a library called \textit{Dunai}, which is the same as Yampa but capable of running in an arbitrary Monad - we leave this for further research. Implementing these iteration-strategies using Haskell and FRP is not as straight-forward as in imperative effectful languages because one does not have mutable data which can be updated in-place.
We implement both update-strategies basically by running all agents behaviour signal-functions every $\Delta t$, so when running a simulation for a duration of \textit{t} the number of steps is $\frac{t}{\Delta t}$. It is important to realise that in our approach of a single behaviour function we merge pro-activity, time-dependent behaviour and message receiving behaviour. A different approach would be to have callbacks for messages in addition to the normal agent-behaviour but this would be quite cluttered and inelegant.

In both the sequential and parallel update-strategy each iteration must also output a potentially changed environment. As already discussed this is implemented as a signal-function which, when available, is then run after each iteration, to make the environment pro-active as well. An example of an environment behaviour would be to regrow some good on each cell according to some rate per time-unit.

\subsubsection{Sequential}
In this strategy the agents are updated one after another where the changes (messages sent, environment changed,...) of one agent are visible to agents updated after. Basically this strategy is implemented as a variant of \textit{fold} which allows to feed output of one agent (e.g. messages and the environment) forward to the other agents while iterating over the list of agents. For each agent the agent behaviour signal-function is called with the current \textit{AgentIn} as input to retrieve the according \textit{AgentOut}. The messages of the \textit{AgentOut} are then distributed to the \textit{AgentIn} of the receiving agents.
The environment which is passed in and returned as well, will then be passed forward to the next agent$_{i + 1}$ in the current iteration. The last environment is then the final environment in the current iteration and will be returned together with the current list of \textit{AgentOut} (see below). As previously mentioned, conversations are \textit{only} possible within this update-strategy because only in this strategy agents act after another which is a fundamental requirement for conversations to make sense and work correctly.

\subsubsection{Parallel}
The parallel strategy is \textit{much} easier to implement than the sequential but is of course not applicable to all models because of its different semantics. Basically this strategy is implemented as a \textit{map} over all agents which calls each agent behaviour signal-function with the agents \textit{AgentIn} to retrieve the new \textit{AgentOut}. Then the messages are distributed amongst all agents.
Each agent receives a copy of the environment upon which it can work and return a changed one. Thus after one iteration there are \textit{N} versions of environments where \textit{N} is equals to the number of agents. These environments must then be folded into a final one which is always domain-specific thus the model implementer needs to provide a corresponding function.

\begin{minted}[fontsize=\footnotesize]{haskell}
type EnvironmentFolding e = [e] -> e
\end{minted}

Of course not all models have environments which can be changed and in the SIR model we indeed use a list of AgentIds which won't change during execution, meaning the agents only read it. Because of this, the environment folding function is optional and when none is provided the environments returned by the agents are ignored and always the initial one is provided.
To make this more explicit we introduce a wrapper which wraps a signal-function which is the same as agent-behaviour but omits the environment from the out tuples. When this wrapper is used one can guarantee statically at compile-time that the environment cannot be changed by the agent-behaviour. We also provide a function which completely ignores the environment, which allows to reason already at compile time that no environment access will happen ever in the given signal-function.

\begin{minted}[fontsize=\footnotesize]{haskell}
type ReactiveBehaviourIgnoreEnv = SF AgentIn AgentOut
type ReactiveBehaviourReadEnv e = SF (AgentIn, e) AgentOut

ignoreEnv :: ReactiveBehaviourIgnoreEnv -> AgentBehaviour
readEnv :: ReactiveBehaviourReadEnv -> AgentBehaviour
\end{minted}

We use both functions in our SIR implementation. The function \textit{readEnv} is used in line 83 of Appendix \ref{app:abs_code} to make sure the behaviour of a susceptible agent can read the environment but never change it. The function \textit{ignoreEnv} is used in line 108 of Appendix \ref{app:abs_code} to make sure the behaviour of an infected agent never accesses the environment.

\section{Non-Reactive Features}
Not all models have such explicit time-semantics as the SIR model. Such models just assume that the agents act in some order but don't rely on any time-outs, timed transitions or rates. These models are more of an imperative nature and map therefore naturally to a monadic style of programming using the \textit{do} notation. Unfortunately it is not possible to do monadic programming within a signal-function, thus to support the programming of such imperative models, we implemented wrapper-functions which allow to provide both non-monadic and monadic functionality which runs within a wrapper signal-function.

\subsection{Non-monadic (pure) wrapping}
\begin{minted}[fontsize=\footnotesize]{haskell}
agentPure :: (e -> Double -> AgentIn -> AgentOut -> (AgentOut, e)) -> AgentBehaviour
agentPureReadEnv :: (e -> Double -> AgentIn -> AgentOut -> AgentOut) -> AgentBehaviour
agentPureIgnoreEnv :: (Double -> AgentIn -> AgentOut -> AgentOut) -> AgentBehaviour
\end{minted}

The function \textit{agentPure} wraps a non-monadic (pure) function and wraps it in a signal-function. This pure function has the environment, the current local time of the agent, the \textit{AgentIn} and a default \textit{AgentOut} as arguments and must return the tuple of \textit{AgentOut} and the environment.
The function \textit{agentPureReadEnv} works the same as \textit{agentPure} but omits the environment from the return type thus ensuring statically at compile-time that an agent which implements its behaviour in this way can only read the environment but never change it.
The function \textit{agentPureIgnoreEnv} is an even more restrictive version and omits environment from the agents behaviour altogether thus ensuring statically at compile-time that an agent which wraps its behaviour in this function will never access the environment.

\subsection{Monadic wrapping}
The monadic wrappers work basically the same way as the pure version, with the difference that they run within the state-monad with \textit{AgentOut} being the state to pass around.

\begin{minted}[fontsize=\footnotesize]{haskell}
agentMonadic :: (e -> Double -> AgentIn -> State AgentOut e) -> AgentBehaviour
agentMonadicReadEnv :: (e -> Double -> AgentIn -> State AgentOut ()) -> AgentBehaviour
agentMonadicIgnoreEnv :: (Double -> AgentIn -> State AgentOut ()) -> AgentBehaviour
\end{minted}

We also provide monadic versions of the pure primitives of our EDSL which work on \textit{AgentOut} as discussed above. They run in the State Monad where the state is the \textit{AgentOut} as this is the data which is manipulated step-by-step for the final output. 

\begin{minted}[fontsize=\footnotesize]{haskell}
agentIdM :: State (AgentOut s m e) AgentId

sendMessageM :: AgentMessage m -> State AgentOut ()
sendMessageToM :: AgentId -> m -> State AgentOut ()
onMessageM :: (Monad mon) => (acc -> AgentMessage m -> mon acc) -> AgentIn -> acc -> mon acc

createAgentM :: AgentDef -> State AgentOut ()

killM :: State AgentOut ()
isDeadM :: State AgentOut Bool

agentStateM :: State AgentOut s
updateAgentStateM :: (s -> s) -> State AgentOut ()
setAgentStateM :: s -> State AgentOut ()
agentStateFieldM :: (s -> t) -> State AgentOut t
\end{minted}

For the environment-behaviour we provide a monadic wrapper as well.

\begin{minted}[fontsize=\footnotesize]{haskell}
type EnvironmentMonadicBehaviour e = Double -> State e ()
environmentMonadic :: EnvironmentMonadicBehaviour e -> EnvironmentBehaviour e
\end{minted}

\subsection{Randomness}
Most of the ABS are inherent stochastic processes and so is the agent-based SIR implementation. This means that agents behaviour depends on random-numbers. So far the drawing from these random-number was hidden behind functions but sometimes an agent needs to draw a random-number directly. For this we provide each agent with its own random-number generator which must be initialized when creating the agent-definition as seen in line 48-59 of Appendix \ref{app:abs_code}. The agent can then draw random-numbers in a pure way without having to resort to the IO Monad. Still it can become very cumbersome because the changed random-number generator needs to be updated in the opaque \textit{AgentOut} - for this we provide a few convenience functions and monadic implementations.

\begin{minted}[fontsize=\footnotesize]{haskell}
agentRandom :: Rand StdGen a -> AgentOut -> (a, AgentOut)
agentRandomM :: Rand StdGen a -> State AgentOut a

randomBoolM :: (RandomGen g) => Double -> Rand g Bool
\end{minted}

Both functions \textit{agentRandom} and \textit{agentRandomM} allow to run a random action implemented in the Rand Monad acting on a StdGen generator. Both need an instance of an \textit{AgentOut} which runs the random action on the agents random-number generator, updates it and returns the random value.
The function \textit{randomBoolM} is such a random action implemented in the Rand Monad and draws a random boolean which is True with a given probability. It is use to randomly determine if a susceptible agent got infected in case of a \textit{Contact Infected} message as can be seen in line 67 of Appendix \ref{app:abs_code}. Note that this random action runs in \textit{onMessageM} which allows to run on a generic monad. Note there exist more random action e.g. to pick at random from a list, to randomly generate within a range, to shuffle a list and to draw from a random distribution. We didn't discuss them here as they follow the same principle.

\section{Running and Replicating the Simulation}
For actually running the simulation we provide four different approaches. 

\begin{minted}[fontsize=\footnotesize]{haskell}
type AgentObservable s = (AgentId, s)
type SimulationStepOut s e = (Time, [AgentObservable s], e)
type AgentObservableAggregator s e a = SimulationStepOut s e -> a

simulateIOInit :: [AgentDef] -> e -> SimulationParams e
                    -> (ReactHandle () (SimulationStepOut s e) -> Bool -> SimulationStepOut s e -> IO Bool)
                    -> IO (ReactHandle () (SimulationStepOut s e))
                    
simulateTime :: [AgentDef] -> e -> SimulationParams e -> DTime -> Time -> [SimulationStepOut s e]
simulateAggregateTime :: [AgentDef] -> e -> SimulationParams -> DTime -> Time -> AgentObservableAggregator a -> [a]
simulateTimeDeltas :: [AgentDef] -> e -> SimulationParams e -> [DTime] -> [SimulationStepOut s e]
\end{minted}

The first one \textit{simulateIOInit} allows for output in the IO Monad, which is useful when one wants to run the simulation for an undefined number of steps and visualise each step by rendering it to a window using a rendering library e.g. \textit{Gloss}. Note that we provide substantial rendering functionality for the environments but don't discuss it here as it is out of the scope of the paper. The \textit{ReactHandle} is part of Yampas function \textit{reactinit} and we refer to Yampas documentation for further details. The second approach \textit{simulateTime} runs the simulation for a given time with given $\Delta t$ and then returns the output for all $\Delta t$. The third approach \textit{simulateTimeDeltas} works the same as the second one but one can provide a list of all $\Delta t$. The function \textit{simulateAggregateTime} allows to transform the output of each step into a different representation, as happens in our SIR implementation where we aggregate the list of all observable agent-outputs into a tuple holding the number of susceptible, infected an recovered agents (see line 132 and 135 - 140 of Appendix \ref{app:abs_code}).
In all approaches at every $\Delta t$ we output a tuple of the current global simulation time, a list of \textit{AgentId} with their states \textit{s} and the environment \textit{e}. All approaches take a list of initial agent definitions, the initial environment and simulation parameters which are obtained calling the function \textit{initSimulation}:

\begin{minted}[fontsize=\footnotesize]{haskell}
data UpdateStrategy = Sequential | Parallel deriving (Eq)

initSimulation :: UpdateStrategy
                    -> Maybe (EnvironmentBehaviour e)
                    -> Maybe (EnvironmentFolding e)
                    -> Bool
                    -> Maybe Int
                    -> IO (SimulationParams e)
\end{minted}

This function takes as parameters the update-strategy with which to run this simulation, an optional environment behaviour, an optional environment folding, a boolean which determines whether the agents are shuffled after each iteration and an optional initial random-number generator seed. The shuffling is necessary in some models which run in the sequential strategy to uniformly distribute the probability of an agent to run at a fixed position.

\subsection*{Replications}
As already described in the previous sections, sometimes it is necessary to run replications. This means running the same simulation multiple times but each with a different random-number generator and averaging the results. For this we provide replicators for agents and environments which can create a new \textit{AgentDef} and environment \textit{e} from the initial ones using the provided random-number generator for the current replication. Although it would be possible to use a default agent replicator which only replaces the random-number generator in \textit{AgentDef}, in the case of our SIR implementation we also need to replace the behaviour signal-function which takes a random-number generator as well.

\begin{minted}[fontsize=\footnotesize]{haskell}
type AgentDefReplicator = StdGen -> AgentDef -> (AgentDef, StdGen)
type EnvironmentReplicator e = StdGen -> e -> (e, StdGen)

data ReplicationConfig e = ReplicationConfig {
    replCfgCount            :: Int,
    replCfgAgentReplicator  :: AgentDefReplicator,
    replCfgEnvReplicator    :: EnvironmentReplicator e
}

runReplications :: [AgentDef] -> e -> SimulationParams e -> DTime
                    -> Time -> ReplicationConfig s m e -> [[SimulationStepOut s e]]
\end{minted}

The function \textit{runReplications} works the same way as the previously mentioned \textit{simulateTime} but takes an additional replication configuration and returns lists of \textit{SimulationStepOut} of length \textit{replCfgCount}.

\section{Agent-Based Dynamics}
We can now run simulations of our agent-based approach and see whether they reach the SD dynamics of Figure \ref{fig:sir_sd_dynamics}. In Figure \ref{fig:sir_abs_approximating_1dt} the dynamics of a first naive attempt using 1,000 agents with $\Delta t= 1.0$ can be seen. 

\begin{figure}
\begin{center}
	\begin{tabular}{c c}
		\begin{subfigure}[b]{0.3\textwidth}
			\centering
			\includegraphics[width=1\textwidth, angle=0]{./../shared/fig/frabs/SIR_1000agents_150t_1dt_NOSS_parallel.png}
			\caption{$\Delta t = 1.0$}
			\label{fig:sir_abs_approximating_1dt}
		\end{subfigure}
    	&
		\begin{subfigure}[b]{0.3\textwidth}
			\centering
			\includegraphics[width=1\textwidth, angle=0]{./../shared/fig/frabs/SIR_1000agents_150t_05dt_NOSS_parallel.png}
			\caption{$\Delta t = 0.5$}
			\label{fig:sir_abs_approximating_05dt}
		\end{subfigure}
    	
    	\\
    	
		\begin{subfigure}[b]{0.3\textwidth}
			\centering
			\includegraphics[width=1\textwidth, angle=0]{./../shared/fig/frabs/SIR_1000agents_150t_02dt_NOSS_parallel.png}
			\caption{$\Delta t = 0.2$}
			\label{fig:sir_abs_approximating_02dt}
		\end{subfigure}
		& 
		\begin{subfigure}[b]{0.3\textwidth}
			\centering
			\includegraphics[width=1\textwidth, angle=0]{./../shared/fig/frabs/SIR_1000agents_150t_01dt_NOSS_parallel.png}
			\caption{$\Delta t = 0.1$}
			\label{fig:sir_abs_approximating_01dt}
		\end{subfigure}
	\end{tabular}
	
	\caption{Naive simulation of SIR using agent-based approach. Population Size $N$ = 1,000, contact rate $\beta = \frac{1}{5}$, infection probability $\gamma = 0.05$, illness duration $\delta = 15$ with initially 1 infected agent. Simulation run for 150 time-steps with various $\Delta t$.} 
	\label{fig:sir_abs_dynamics_naive}
\end{center}
\end{figure}

%TODO: reproducing about the same dynamics of the SD-solution (1.0 dt)
%	- super-sampling: 	contact-rate ss high, illness time-out low 
%	- agent number:		1000 vs. 10.000 agents
%	- Susceptibles making contact and infected response VS. only Infected make contact
%	- update-strat:		Sequential vs. Parallel
%	- making contact: susceptible only vs. susceptible AND infected
%	- do conversations make a difference?
%	- does a delayed switch (dSwitch) in transitions makes a difference?

Clearly something is going wrong as the dynamics do not resemble the ones of SD in any way with only 10 agents making the transition to infected to recovered. The problem is that we are running into sampling issues. TODO: explain deeper and better

\subsection{Sampling the System}
When sampling the system, the correct $\Delta t$ must be selected which depends on the highest frequency which occurs in a time-reactive function in the whole system. For example in the SIR model we want infected agents to make on average contact with $\beta = 5$ other agents per time-unit, which means with a frequency of $\frac{1}{5}$. This functionality is built on Yampas function \textit{occasionally} which behaviour we investigated under differing $\Delta t$ with the above frequency. In this investigation we simply sampled occasionally with different $\Delta t$ for a duration of $t = 1,000$ and the event-frequency of $\frac{1}{5}$. The result can be seen in Figure \ref{fig:sampling_occasionally_5evts} and is quite striking. The plot clearly shows that occasionally needs a quite high sampling frequency even for a comparatively low event-frequency, which becomes of course worse for higher event-frequencies.

The other time-reactive function which occurs in the SIR model is the timed transition from infected to recovered which occurs on average with an exponential random-distribution after $\delta = 15$. This functionality is built on a custom implementation of Yampas \textit{after} which creates an event after a time-out of the passed in time-duration drawn from an exponential random-distribution. Clearly this function has different semantics as although it also continuously emit events over time - \textit{NoEvent} before the time was hit, and \textit{Event b} after the time hit - the relevant point is that it switches to Event at some discrete point in time. This is implemented as simply adding up the $\Delta t$ until the accumulator is greater of equal than the previously drawn exponential time-out. We also investigated the behaviour of this function under varying $\Delta t$ using a time-out of $\delta = 15$. Our approach was to sample the \textit{afterExp} until an event occurs and then see when it has occurred. We run this with 10,000 replications with different random-number seeds and average the resulting times. The results can be seen in Figure \ref{fig:sampling_afterExp_5time}. The result is striking in another way: this function seems to be pretty invariant to the time-deltas, for obvious reasons: we are basically just interested in the "after"-condition of the whole semantics whereas in occasionally we are interested in the "repeatedly"-conditions. If we under-sample the \textit{afterExp} then we can be off by one $\Delta t$. If we under-sample \textit{occasionally} we keep loosing events - the less difference between $\Delta t$ and event-frequency, the more events we lose. Of course \textit{afterExp} can also be used for very short time-outs e.g. $\frac{1}{5}$. We have investigated the behaviour of this function for various $\Delta t$ as well as seen in Figure \ref{fig:sampling_afterExp_02time}. Here the result is much more striking and shows that \textit{afterExp} is vulnerable to small time-outs as well as \textit{occasionally}.  
To show that \textit{occasionally} is not vulnerable to very low frequencies of e.g. one event every 5 time-steps we plotted the behaviour of this under varying time-steps in Figure \ref{fig:sampling_occasionally_02evts}. The result shows that for low frequencies occasionally works fine with larger $\Delta t$.

\begin{figure}
\begin{center}
	\begin{tabular}{c c}
	\begin{subfigure}[b]{0.5\textwidth}
			\centering
			\includegraphics[width=.6\textwidth, angle=0]{./../shared/fig/sampling/samplingTest_occasionally_5evts.png}
			\caption{Sampling \textit{occasional} with a frequency of $\frac{1}{5}$ (average of 5 events per time-unit). The theoretical average is 5000 events within this time-frame.}
			\label{fig:sampling_occasionally_5evts}
		\end{subfigure}
		& 
		\begin{subfigure}[b]{0.5\textwidth}
			\centering
			\includegraphics[width=.6\textwidth, angle=0]{./../shared/fig/sampling/samplingTest_occasionally_02evts.png}
			\caption{Sampling \textit{occasional} with a frequency of 5 (average of 0.2 events per time-unit). The theoretical average is 200 events within this time-frame.}
			\label{fig:sampling_occasionally_02evts}
		\end{subfigure}
		
		\\
		
		\begin{subfigure}[b]{0.5\textwidth}
			\centering
			\includegraphics[width=.6\textwidth, angle=0]{./../shared/fig/sampling/samplingTest_afterExp_5time.png}
			\caption{Sampling \textit{afterExp} with an average time-out of 5.}
			\label{fig:sampling_afterExp_5time}
		\end{subfigure}
		& 
		\begin{subfigure}[b]{0.5\textwidth}
			\centering
			\includegraphics[width=.6\textwidth, angle=0]{./../shared/fig/sampling/samplingTest_afterExp_02time.png}
			\caption{Sampling \textit{afterExp} with an average time-out of 0.2.}
			\label{fig:sampling_afterExp_02time}
		\end{subfigure}
	\end{tabular}
	
	\caption{Sampling the \textit{afterExp} and \textit{occasional} functions to visualise the influence of sampling frequencies on the occurrence of the respective events. $\Delta t$ are [ 5, 2, 1, $\frac{1}{2}$, $\frac{1}{5}$, $\frac{1}{10}$, $\frac{1}{20}$, $\frac{1}{50}$, $\frac{1}{100}$ ]. The experiments for \textit{afterExp} used 10,000 replications. The experiments for \textit{occasional} ran for $t= 1,000$ with 100 replications.} 
	\label{fig:sampling_tests}
\end{center}
\end{figure}

Using these observation we run simulations with varying $\Delta t$ with $\Delta = 0.5$, $\Delta = 0.2$ and $\Delta = 0.1$ with the results visible in Figures \ref{fig:sir_abs_approximating_05dt}, \ref{fig:sir_abs_approximating_02dt} and \ref{fig:sir_abs_approximating_01dt} but still when decreasing $\Delta t$ we don't approach the SD dynamics. As previously mentioned the agent-based approach is a discrete one which means that with increasing number of agents, the discrete dynamics approximate the continuous dynamics of the SD simulation. We run further simulations with $\Delta = 0.1$ but with varying agent numbers to see the influence with the results seen in Figure \ref{fig:sir_abs_approximating}.

\begin{figure}
\begin{center}
	\begin{tabular}{c c}
		\begin{subfigure}[b]{0.3\textwidth}
			\centering
			\includegraphics[width=1\textwidth, angle=0]{./../shared/fig/frabs/SIR_100agents_150t_01dt_NOSS_parallel.png}
			\caption{100 Agents}
			\label{fig:sir_abs_approximating_100}
		\end{subfigure}
    	&
		\begin{subfigure}[b]{0.3\textwidth}
			\centering
			\includegraphics[width=1\textwidth, angle=0]{./../shared/fig/frabs/SIR_1000agents_150t_01dt_NOSS_parallel.png}
			\caption{1,000 Agents}
			\label{fig:sir_abs_approximating_1000}
		\end{subfigure}
    	
    	\\
    	
		\begin{subfigure}[b]{0.3\textwidth}
			\centering
			\includegraphics[width=1\textwidth, angle=0]{./../shared/fig/frabs/SIR_5000agents_150t_01dt_NOSS_parallel.png}
			\caption{5,000 Agents}
			\label{fig:sir_abs_approximating_5000}
		\end{subfigure}
		& 
		\begin{subfigure}[b]{0.3\textwidth}
			\centering
			\includegraphics[width=1\textwidth, angle=0]{./../shared/fig/frabs/SIR_10000agents_150t_01dt_NOSS_parallel.png}
			\caption{10,000 Agents}
			\label{fig:sir_abs_approximating_10000}
		\end{subfigure}
	\end{tabular}
	
	\caption{Varying agent numbers with same model-parameters except population size. All simulations run for 150 time-steps with $\Delta t = 0.1$}
	\label{fig:sir_abs_approximating}
\end{center}
\end{figure}

Still the dynamics of 10,000 Agents do not match the dynamics of the SD simulation perfectly. This is because as opposed to the SD simulation, which is deterministic, the agent-based approach is inherently a stochastic one as we continuously draw from random-distributions which drive our state-transitions. What we see in Figure \ref{fig:sir_abs_approximating} is then just a single run where the dynamics would result in slightly different shapes when run with a different random-number generator seed. The agent-based approach thus generates a distribution of dynamics over which ones needs to average to arrive at the correct solution. This can be done using replications in which the simulation is run with the exact same parameters multiple times but each with a different random-number generator see. The resulting dynamics are then averaged and the result is then regarded as the correct dynamics.
We have done this as can be seen in Figure \ref{fig:sir_abs_agents_repls}, using 10 replications, which matches the SD dynamics to a very satisfactory level. Note that in the replications we are using 10 initially infected agents to ensure that no simulation run will terminate too early (meaning that the disease gets extinct after a few time steps) which would offset the dynamics completely. This happens due to "unlucky" random distributions which can be repaired by introducing more initially infected agents which increases the probability of spreading the disease in the very early stage of the simulation drastically. We found that when using 10 initially infected agents in a population of 5,000 (which amounts to 0.2\%) is enough to never result in an early terminating simulation. In the case of 100 agents 10 initially infected ones might be too much and distorts the dynamics but this is irrelevant in this case. This is also a fundamental difference between SD and ABS: the dynamics of the agent-based approach can result in a wide range of scenarios which includes also the one in which the disease gets extinct in the early stages (a lucky coincidence for mankind) - this is simply not possible in the SD approach. So we can argue that ABS is much closer to reality than SD as it allows to explore alternate futures in the dynamics.

\begin{figure}
\begin{center}
	\begin{tabular}{c c}
		\begin{subfigure}[b]{0.3\textwidth}
			\centering
			\includegraphics[width=1\textwidth, angle=0]{./../shared/fig/frabs/SIR_100agents_150t_01dt_NOSS_parallel_10replications.png}
			\caption{100 Agents}
			\label{fig:sir_abs_agents_repls_100}
		\end{subfigure}
    	&
		\begin{subfigure}[b]{0.3\textwidth}
			\centering
			\includegraphics[width=1\textwidth, angle=0]{./../shared/fig/frabs/SIR_1000agents_150t_01dt_NOSS_parallel_10replications.png}
			\caption{1,000 Agents}
			\label{fig:sir_abs_agents_repls_1000}
		\end{subfigure}
    	
    	\\
    	
		\begin{subfigure}[b]{0.3\textwidth}
			\centering
			\includegraphics[width=1\textwidth, angle=0]{./../shared/fig/frabs/SIR_5000agents_150t_01dt_NOSS_parallel_10replications.png}
			\caption{5,000 Agents}
			\label{fig:sir_abs_agents_repls_5000}
		\end{subfigure}
		&
		\begin{subfigure}[b]{0.3\textwidth}
			\centering
			\includegraphics[width=1\textwidth, angle=0]{./../shared/fig/frabs/SIR_10000agents_150t_01dt_NOSS_parallel_10replications.png}
			\caption{10,000 Agents}
			\label{fig:sir_abs_agents_repls_10000}
		\end{subfigure}
	\end{tabular}
	
	\caption{Dynamics of Figure \ref{fig:sir_abs_approximating} averaged over 10 replications with initially 10 infected agents.} 
	\label{fig:sir_abs_agents_repls}
\end{center}
\end{figure}

When comparing the results of the dynamics of the agent-based approach from Figure \ref{fig:sir_abs_approximating} and Figure \ref{fig:sir_abs_agents_repls} to the SD dynamics of Figure \ref{fig:sir_sd_dynamics} it becomes apparent that by increasing the number of agents the dynamics approximate the SD dynamics with increasing accuracy. Still although using 5,000 agents and replications seem to be not enough yet, we need to increase our number of agents to 10,000

Still although using a quite small $\Delta t = 0.1$ and using replications we are nowhere close to the SD dynamics. The only option we have is to further decrease $\Delta t$. Of course performance is a big issue and it decreases as $\Delta t$ get smaller and smaller. This is because when running a simulation for a duration of $t$ and sampling it with $\Delta t$ when the steps to calculate is $\frac{t}{\Delta t}$. In each step all agents are run, messages delivered and environments folded and updated which implies that the more steps the lower the performance. If we could perform super-sampling just for the given high-frequency functions with the whole system running in lower frequency then we could achieve a substantial performance boost.

\subsection{Super-Sampling}
In Yampa there exists a function \textit{embed} which allows to run a given signal-function with provided $\Delta t$ but the problem is that this function does not really help because it does not return a signal-function. What we need is a signal-function which takes the number of super-samples \textit{n}, the signal-function \textit{sf} to sample and returns a new signal-function which performs super-sampling on it:

\begin{minted}[fontsize=\footnotesize]{haskell}
superSampling :: Int -> SF a b -> SF a [b]
\end{minted}

It does this by evaluating \textit{sf} for \textit{n} times, each with $\Delta t = \frac{\Delta t}{n}$ and the same input argument \textit{a} for all \textit{n} evaluations. At time 0 no super-sampling is done and just a single output of \textit{sf} is calculated. A list of \textit{b} is returned with length of \textit{n} containing the result of the \textit{n} evaluations of \textit{sf}. If 0 or less super samples are requested exactly one is calculated.

We ran tests super-sampling both \textit{occasionally} Figure \ref{fig:sampling_occasionally_ss_02evts}, Figure \ref{fig:sampling_occasionally_ss_5evts} and \textit{afterExp} Figure \ref{fig:sampling_afterExp_ss_5time}, Figure \ref{fig:sampling_afterExp_ss_02time}. They work the same way as above except that now $\Delta t = 1.0$ but using increasing numbers of super-samples. The results are as expected: as the number of super-samples increase, so increases the accuracy.

\begin{figure*}
\begin{center}
	\begin{tabular}{c c}
		\begin{subfigure}[b]{0.5\textwidth}
			\centering
			\includegraphics[width=.6\textwidth, angle=0]{./../shared/fig/sampling/samplingTest_occasionally_ss_02evts.png}
			\caption{Super-Sampling the \textit{occasional} function with event-frequency of 5 (average of 0.2 events per time-unit). The theoretical average is 20 event within this time-frame.}
			\label{fig:sampling_occasionally_ss_02evts}
		\end{subfigure}
	
		&
		
		\begin{subfigure}[b]{0.5\textwidth}
			\centering
			\includegraphics[width=.6\textwidth, angle=0]{./../shared/fig/sampling/samplingTest_occasionally_ss_5evts.png}
			\caption{Super-Sampling the \textit{occasional} function with event-frequency of $\frac{1}{5}$ (average of 5 events per time-unit). The theoretical average is 500 event within this time-frame.}
			\label{fig:sampling_occasionally_ss_5evts}
		\end{subfigure}

		\\
		
		\begin{subfigure}[b]{0.5\textwidth}
			\centering
			\includegraphics[width=.6\textwidth, angle=0]{./../shared/fig/sampling/samplingTest_afterExp_SS_5time.png}
			\caption{Super-Sampling the \textit{afterExp} function with average time-out of 5.}
			\label{fig:sampling_afterExp_ss_5time}
		\end{subfigure}

		&
		
		\begin{subfigure}[b]{0.5\textwidth}
			\centering
			\includegraphics[width=.6\textwidth, angle=0]{./../shared/fig/sampling/samplingTest_afterExp_SS_02time.png}
			\caption{Super-Sampling the \textit{afterExp} function with average time-out of 0.2.}
			\label{fig:sampling_afterExp_ss_02time}
		\end{subfigure}
	\end{tabular}
	
	\caption{Super-Sampling the \textit{afterExp} and \textit{occasional} functions to visualize the influence of increasing number of super-samples on the average occurrence of the respective events. The $\Delta t = 1.0$ in both cases with super-samples of [1, 2, 5, 10, 100, 1000]. The experiments for \textit{afterExp} used 10,000 replications. The experiments for \textit{occasional} ran for $t = 100$ with 100 replications.} 
	\label{fig:supersampling_tests}
\end{center}
\end{figure*}

At first this might not seem to be a real win as we still need to calculate a big number of samples every time. The big win comes though when these super-sampled signal-functions are embedded in a larger system which could run on a comparatively low frequency of $\Delta t = 1.0$. So we are then increasing the sampling-frequency just where we need it and keep the frequency low where it is not required.

We are using super-sampling in our SIR implementation to increase performance. We do this by setting $\Delta t = 1.0$ and super-sampling the relevant functions with time-semantics which are \textit{transitionAfterExp} and \textit{sendMessageOccationallySrc}. For both we provide in our EDSL versions which support super-sampling:

\begin{minted}[fontsize=\footnotesize]{haskell}
sendMessageOccasionallySrcSS :: RandomGen g => g -> Double -> Int -> MessageSource 
                                -> SF (AgentOut, e) AgentOut
                                
transitionAfterExpSS :: RandomGen g => g -> Double -> Int 
                        -> AgentBehaviour -> AgentBehaviour -> AgentBehaviour
\end{minted}

Both now take an additional parameter which determines the number of super-samples to be calculated. According to the above observations of the \textit{occasionally} and \textit{afterExp} functions which are the foundations of both of the functions we sample \textit{sendMessageOccasionallySrcSS} with 20 super-samples and \textit{transitionAfterExpSS} with 2. This will ensure that by using $\Delta t = 1.0$ we only calculate $t$ steps when running a simulation for $t$ time but that we sample our relevant functions with enough resolution to capture its frequencies. Optimally we should increase the number of super-samples for \textit{sendMessageOccasionallySrcSS} to about 100. This will result in lower performance as \textit{every} agent will perform this super-sampling. So in the end it is a struggle of performance vs. sufficiently close approximation. We define the number of super-samples in lines 29 and 32 and use the functions in line 96 and 106 of Appendix \ref{app:abs_code}.

TODO: 10.000 with SS and dt = 1.0 with ss

Unfortunately when setting $\Delta t = 1.0$ the dynamics of the agent-based approach won't approach the dynamics of the SD, despite using super-sampling as can be seen in Figure \ref{fig:sir_10000_1dt}.

\begin{figure}
\begin{center}
	\begin{tabular}{c c}
		\begin{subfigure}[b]{0.5\textwidth}
			\centering
			\includegraphics[width=.8\textwidth, angle=0]{./../shared/fig/frabs/SIR_10000agents_150t_1dt_parallel.png}
			\caption{$\Delta t = 1.0$}
			\label{fig:sir_10000_1dt}
		\end{subfigure}
	
		&
		
		\begin{subfigure}[b]{0.5\textwidth}
			\centering
			\includegraphics[width=.8\textwidth, angle=0]{./../shared/fig/frabs/SIR_10000agents_150t_01dt_parallel.png}
			\caption{$\Delta t = 0.1$}
			\label{fig:sir_10000_01dt}
		\end{subfigure}
	\end{tabular}
	
	\caption{Comparing the influence of different $\Delta t$. Both dynamics were generated with the same configuration of 10,000 agents, super-sampling enabled as described and the same model-parameters. When using $\Delta t = 1.0$, the dynamics do not match the ones of the SD approach, whereas in the case of $\Delta t = 0.1$, they can be seen as matching completely.} 
	\label{fig:sir_10000_dt_comparisons}
\end{center}
\end{figure}

When reflecting on the messaging mechanism it becomes apparent that a round-trip from sender to receiver and back takes $2 \Delta t$. A round-trip happens in our agent-based SIR approach to implement the transition from infected to susceptible - susceptible agents send \textit{Contact Susceptible} messages to random agents (except itself) where only infected agents reply with a \textit{Contact Infected} message. This means that it takes $2 \Delta t$ until a susceptible agent might get infected. This becomes an issue if we want to match the dynamics of our agent-based approach to the one of SD in which no time-delay happens - the agents act instantaneous with each other during one time-step. 
We have two solutions for this problem: either we resort to \textit{conversations} or we increase the global sampling frequency of the system which matches the \textit{message frequency} of messages which are subject to round-trips. Implementing conversations is only available in the \textit{sequential} update-strategy and is much more involved, so we followed the approach of increasing the frequency. As can be seen in Figure \ref{fig:sir_10000_01dt} when setting $t\Delta = 0.1$ the resulting dynamics are a sufficiently good approximation to the SD solution.

\section{Spatiality}
When emulating the dynamics of the SIR model using an agent-based approach the question arises what we ultimately gain from doing so when we could have generated the dynamics much quicker and smoother using the SD approach. The difference is that the agent-based approach is a stochastic one and can thus also generate "degenerated" dynamics e.g. in which the disease dies out after a few steps or even can't spread from patient zero - in this case ABS is clearly a benefit as it allows to investigate \textit{alternative futures}, something not possible with SD in which the disease will never die out prematurely when there are non-zero infected agents. \\
Another advantage of ABS over the system-dynamics approach is that agents can be heterogeneous and make use of spatial- and/or network-information defining the neighbourhood. We can thus simulate the spread of the disease throughout a population which is laid out on a 2D grid or one can investigate spreading of the disease throughout a network of agents where some are vaccinated and others not. We provide already suitable environments to simulate these cases and show an example of spreading the disease on a 2D grid in Figure \ref{fig:sir_spatial}.  

\begin{figure*}
\begin{center}
	\begin{tabular}{c c}
		\begin{subfigure}[b]{0.4\textwidth}
			\centering
			\includegraphics[width=.6\textwidth, angle=0]{./../shared/fig/spatial/SIR_spatial_52x52_92time.png}
			\caption{$t = 92$}
			\label{fig:sir_spatial_92}
		\end{subfigure}

		& 

		\begin{subfigure}[b]{0.4\textwidth}
			\centering
			\includegraphics[width=.6\textwidth, angle=0]{./../shared/fig/spatial/SIR_spatial_52x52_200time.png}
			\caption{$t = 200$}
			\label{fig:sir_spatial_200}
		\end{subfigure}

		\\
		
		\begin{subfigure}[b]{0.4\textwidth}
			\centering
			\includegraphics[width=.6\textwidth, angle=0]{./../shared/fig/spatial/SIR_spatial_52x52_440time.png}
			\caption{$t = 440$}
			\label{fig:sir_spatial_440}
		\end{subfigure}
		
		& 
		
		\begin{subfigure}[b]{0.4\textwidth}
			\centering
			\includegraphics[width=.6\textwidth, angle=0]{./../shared/fig/spatial/SIR_spatial_52x52_873time.png}
			\caption{$t = 873$}
			\label{fig:sir_spatial_873}
		\end{subfigure}
	\end{tabular}
	
	\caption{Simulating SIR on a 52x52 grid with Moore neighbourhood using $\Delta t = 1$ TODO: at this point we can not talk about dt yet. Blue are susceptible, red are infected, green are recovered. The green areas act as protection as infected cannot cross the recovered border: this is particularly visible in the lower right corner of \ref{fig:sir_spatial_440} where the disease has been contained in the blue island and has no means to escape. It may seem that the few remaining infected agents in the top left corner of \ref{fig:sir_spatial_440} will die out soon but still it needs more than the already running simulation-time until the disease actually dies out with the last patient recovering at center top of \ref{fig:sir_spatial_873} at $t = 873$. The dynamics of this simulation run can be seen in Figure \ref{fig:sir_spatial_dynamics}.} 
	\label{fig:sir_spatial}
\end{center}
\end{figure*}

\begin{figure}
	\centering
	\includegraphics[width=.6\textwidth, angle=0]{./../shared/fig/spatial/SIR_spatial_dynamics_52x52_900time_1dt_parallel.png}
	\caption{Dynamics of the spatial SIR simulation from Figure \ref{fig:sir_spatial}.}
	\label{fig:sir_spatial_dynamics}
\end{figure}

Note that the dynamics of the spatial SIR simulation which are seen in Figure \ref{fig:sir_spatial_dynamics} look very different from the SD dynamics of Figure \ref{fig:sir_sd_dynamics}. This is due to a much more restricted neighbourhood which results in far fewer infected agents at a time and a lower number of recovered agents at the end of the epidemic, meaning that fewer agents got infected overall.

When using a 2D grid or network one needs to set them up in the initialization code so there is a little more work to do there but the implementation of the agents differ just in one single line, which is where the neighbourhood is picked (see line 100 of Appendix \ref{app:abs_code}). Instead of \textit{randomAgentIdMsgSource} one uses either \textit{randomNeighbourNodeMsgSource} in the case of a network or \textit{randomNeighbourCellMsgSource} in case of a 2D grid.

\section{Randomness and Super-Sampling}
TODO: maybe this subsection is unimportant

It is important to note that if we disable super-sampling and run the simulation for a given time \textit{t} but with two different $\Delta t$ we would end up with two different results, even if the $\Delta t$ are small enough to sample the time-dependent functions sufficiently. Also if we use super-sampling with $\Delta t = 1.0$ and create the exact same number of samples as when using no super-sampling but smaller $\Delta t$, then we also end up with different results.
The reason for this behaviour is that most of the time-dependent functions ultimately build upon drawing from random-distributions. With different $\Delta t$ we are generating a different number of random-samples, which would result in different random-number sequences which in turn ultimately leads to slightly different dynamics. When generating a plot of the dynamics this is not as visible, also this is the reason why one generates multiple replications, but this behaviour becomes strikingly apparent when simulating the SIR model on a 2D grid as can be seen in Figure \ref{fig:sir_abs_timeDeltas_randomness}.

\begin{figure*}
\begin{center}
	\begin{tabular}{c c}
		\begin{subfigure}[b]{0.4\textwidth}
			\centering
			\includegraphics[width=.7\textwidth, angle=0]{./../shared/fig/randomness/SIR_32x32_200time_01delta_noSS.png}
			\caption{$\Delta t = 0.1$, no super-sampling}
			\label{fig:sir_abs_timeDeltas_randomness_dt01}
		\end{subfigure}
	
		& 
		
		\begin{subfigure}[b]{0.4\textwidth}
			\centering
			\includegraphics[width=.7\textwidth, angle=0]{./../shared/fig/randomness/SIR_32x32_200time_001delta_noSS.png}
			\caption{$\Delta t = 0.01$, no super-sampling}
			\label{fig:sir_abs_timeDeltas_randomness_dt001}
		\end{subfigure}
		
		\\

		\begin{subfigure}[b]{0.4\textwidth}
			\centering
			\includegraphics[width=.7\textwidth, angle=0]{./../shared/fig/randomness/SIR_32x32_200time_10delta_ss10.png}
			\caption{$\Delta t = 1.0$ with 10 super-samples}
			\label{fig:sir_abs_timeDeltas_randomness_dt10_ss10}
		\end{subfigure}
		
		&
		
		\begin{subfigure}[b]{0.4\textwidth}
			\centering
			\includegraphics[width=.7\textwidth, angle=0]{./../shared/fig/randomness/SIR_32x32_200time_10delta_ss100.png}
			\caption{$\Delta t = 1.0$ with 100 super-samples}
			\label{fig:sir_abs_timeDeltas_randomness_dt10_ss100}
		\end{subfigure}
	\end{tabular}
	
	\caption{Comparing results on 32x32 grid after $t = 200$ but with different $\Delta t$ and different number of super-samples.}
	\label{fig:sir_abs_timeDeltas_randomness}
\end{center}
\end{figure*}

\section{Agents as Signals}
Due to the underlying nature and motivation of FRP and Yampa, agents can be seen as signals which are generated and consumed by a signal-function which is the behaviour of an agent.  If an agent does not change, the output-signal should be constant amd if the agent changes e.g. by sending a message, changing its state,... the output-signal should change as well. A dead agent then should have no signal at all.
The question is if the agents of our agent-based SIR implementation are true signals: do the dynamics stay constant when we sample the system with $\Delta t = 0$? We hypothesize that our agents are true signals, thus they should not change when time does not change because they are completely time-dependent and rely completely on time-semantics. When actually running the simulation with $\Delta t = 0$ one gets the results as seen in Figure \ref{fig:sir_abs_zero_dt}.

\begin{figure*}
\begin{center}
	\begin{tabular}{c c}
		\begin{subfigure}[b]{0.5\textwidth}
			\centering
			\includegraphics[width=.8\textwidth, angle=0]{./../shared/fig/dtzero/SIR_ABS_zeroDt_start.png}
			\caption{$\Delta t = 0$ from step 0 to 50.}
			\label{fig:sd_plot_10dt}
		\end{subfigure}
	
		& 
		
		\begin{subfigure}[b]{0.5\textwidth}
			\centering
			\includegraphics[width=.8\textwidth, angle=0]{./../shared/fig/dtzero/SIR_ABS_zeroDt_mid.png}
			\caption{$\Delta t = 0$ from step 51 to 101.}
			\label{fig:sd_plot_0.01dt}
		\end{subfigure}
	\end{tabular}
	
	\caption{Dynamics of agent-based SIR implementation of 1,000 agents running with $\Delta t = 1$ with ranges of $\Delta t = 0$ marked with two vertical black lines.}
	\label{fig:sir_abs_zero_dt}
\end{center}
\end{figure*}

As can be seen  the dynamics are becoming constant \textit{but} with a minor delay: infected increases a bit while susceptible decreases as can be seen in Figure \ref{fig:sir_abs_zero_dt_zoom}. This is due to the delay of message delivery which takes one $\Delta t$, independent of its value - messages are also delivered when $\Delta t = 0$. Only message-generating functions, which depend on non-zero $\Delta t$ to generate messages, will then stop generating messages. Reactive functions which act on incoming messages can still create change as they do not rely on time-semantics but just on the discrete event of a message arrival - which is the case in the transition from susceptible to infected.

\begin{figure}
	\centering
	\includegraphics[width=.4\textwidth, angle=0]{./../shared/fig/dtzero/SIR_ABS_zeroDt_mid_zoom.png}
	\caption{Zoom-in to step 51, marked with the black line from where on $\Delta t = 0$ for the next 50 steps. The recovered ratio stays constant but a few agents get infected even \textit{after} having switched to $\Delta t = 0$ which happens due to the message delivery lag. After all messages have been delivered, the signal stays constant until non-zero $\Delta t$ are turned on again.}
	\label{fig:sir_abs_zero_dt_zoom}
\end{figure}

Note that agents of models with no time-semantics won't exhibit this behaviour - the dynamics will change even in case of $\Delta t = 0$ as agents act on every update and don't care about $\Delta t$ and just assume that every update occurs after $\Delta t$ independent of the actual value of it. We implemented the function \textit{doRepeatedlyEvery} which allows to transform a time-agnostic agent-behaviour into one. It is built on Yampas \textit{repeatedly} function and has the following signature:

\begin{minted}[fontsize=\footnotesize]{haskell}
doRepeatedlyEvery :: Time -> AgentBehaviour -> AgentBehaviour
\end{minted}

This function takes a time interval and an agent behaviour signal-function and returns a new agent behaviour signal-function which runs the argument signal-function every time-interval. Note that this function is subject to sampling issues too e.g. when the time-interval is very small one needs to run the simulation with a $\Delta t \leq Time$ otherwise the dynamics would show delayed activation of the agent behaviour.

\section{Discussion}

\subsection{Other Models}
TODO: mention that we have also implemented other models, which also work without time-semantics (all agents make a move at discrete time-steps and do not really rely on some notion of time). 

\subsection{Time-Semantics}
The main reason for building our pure functional ABMS approach on top of Yampa was to leverage the powerful time-semantics of Yampa which allows us to implement important concepts of ABMS:

state-chart: agents are at all time of their life-cycle in one state and can switch between multiple states using transitions 
timed transitions: transition to another state/behaviour happens at a discrete time
rate transitions: transition happens with a given rate
message transition: transition upon receiving a given message 

\subsection{Agents as Signals}
Due to the underlying nature and motivation of Functional Reactive Programming (und im speziellen) Yampa, Agents can be seen as Signals which is generated and consumed by a Signal-Function which is the behaviour of an Agent. If an Agent does not change the OUTPUT-signal is constant, if the agent changes e.g. by sending a message, changing its state,... the OUTPUT signal changes. A dead agent has no signal at all.

\subsection{Time-Sampling}
sampling rate depends on the transition times \& rates of the model. when e.g. the contact rate is 5 then the sampling dt should be below 0.2

\subsection{System Dynamics}
can emulate system dynamics due to the parallel update-strategy and continuous time-flow semantics

\subsection{Discrete Event Simulation}
DES in FrABMS? how easily can we implement server/queue systems? do they also look like a specification? potential problem: ordering of messages is not guaranteed by now

\subsection{Advantages}
advantages:
	- no side-effects within agents leads to much safer code
	- edsl for time-semantics
	- declarative style: agent-implementation looks like a model-specification
	- reasoning and verification
	- sequential and parallel
	- powerful time-semantics
	- arrowized programming is optional and only required when utilizing yampas time-semantics. if the model does not rely on time-semantics, it can use monadic-programming by building on the existing monadic functions in the EDSL which allow to run in the State-Monad which simplifies things very much
	- when to use yampas arrowized programing: time-semantics, simple state-chart agents 
	- when not using yampas facilities: in all the other cases e.g. SugarScape is such a case as it proceeds in unit time-steps and all agents act in every time-step
	- can implement System Dynamics building on Yampas facilities with total ease	
	- get replications for free without having to worry about side-effects and can even run them in parallel without headaches
	- cant mess around with time because delta-time is hidden from you (intentional design-decision by Yampa). this would be only very difficult and cumbersome to achieve in an object-oriented approach. TODO: experiment with it in Java - how could we actually implement this? I think it is impossible: may only achieve this through complicated application of patterns and inheritance but then has the problem of how to update the dt and more important how to deal with functions like integral which accumulates a value through closures and continuations. We could do this in OO by having a general base-class e.g. ContinuousTime which provides functions like updateDt and integrate, but we could only accumulate a single integral value.
	- reproducibility statically guaranteed
	- cannot mess around with dt
	- code == specification
	- rule out serious class of bugs
	- different time-sampling leads to different results e.g. in wildfire \& SIR but not in Prisoners Dilemma. why? probabilistic time-sampling?
	- reasoning about equivalence between SD and ABS implementation in the same framework
	- recursive implementations
	
	- we can statically guarantee the reproducibility of the simulation because: no side effects possible within the agents which would result in differences between same runs (e.g. file access, networking, threading), also timedeltas are fixed and do not depend on rendering performance or userinput	
	
\subsection{Disadvantages}
disadvantages:
	- performance is low
	- reasoning about performance is very difficult
	- very steep learning curve for non-functional programmers
	- learning a new EDSL
	- think ABMS different: when to use async messages, when to use sync conversations


[ ] important: increasing sampling freqzency and increasing number of steps so that the same number of simulation steps are executed should lead to same results. but it doesnt. why?
[ ] hypothesis: if time-semantics are involved then event ordering becomes relevant for emergent patterns. there are no tine semantics in heroes and cowards but in the prisoners dilemma
[ ] can we implement different types of agents interacting with each other in the same simulation ? with different behaviour funcs, digferent state? yes, also not possible in NetLogo to my knowledge. but they must have the same messages, emvironment 

[ ] Hypothesis: we can combine with FrABS agent-based simulation and system dynamics (this has been proved by example!)

\section{Related Research}

\cite{schneider_towards_2012} and \cite{vendrov_frabjous:_2014} present a domain-specific language for developing functional reactive agent-based simulations. This language called FRABJOUS is very human readable and easily understandable by domain-experts. It is not directly implemented in FRP/Haskell/Yampa but is compiled to Haskell/Yampa code which they claim is also readable. This is the direction we want to head but we don't want this intermediate step but look for how a most simple domain-specific language embedded in Haskell would look like. In this paper we explicitly dive deep into FRP And Yampa and see how we can combine the best of both.

\section{Conclusions}
\label{sec:conclusions}

Our approach is radically different from traditional approaches in the ABS community. First it builds on the already quite powerful FRP paradigm. Second, due to our continuous time approach, it forces one to think properly of time-semantics of the model and how small $\Delta t$ should be. Third it requires to think about agent interactions in a new way instead of being just method-calls.

Because no part of the simulation runs in the IO Monad and we do not use unsafePerformIO we can rule out a serious class of bugs caused by implicit data-dependencies and side-effects which can occur in traditional imperative implementations.

Also we can statically guarantee the reproducibility of the simulation, which means that repeated runs with the same initial conditions are guaranteed to result in the same dynamics. Although we allow side-effects within agents, we restrict them to only the Random and State Monad in a controlled, deterministic way and never use the IO Monad which guarantees the absence of non-deterministic side effects within the agents and other parts of the simulation.

Determinism is also ensured by fixing the $\Delta t$ and not making it dependent on the performance of e.g. a rendering-loop or other system-dependent sources of non-determinism as described by \cite{perez_testing_2017}. Also by using FRP we gain all the benefits from it and can use research on testing, debugging and exploring FRP systems \cite{perez_testing_2017, perez_back_2017}.

\subsection*{Issues}
Currently, the performance of the system is not comparable to imperative implementations but our research was not focusing on this aspect. We leave the investigation and optimization of the performance aspect of our approach for further research.

Despite the strengths and benefits we get by leveraging on FRP, there are errors that are not raised at compile time, e.g. we can still have infinite loops and run-time errors. This was for example investigated in \cite{sculthorpe_safe_2009} where the authors use dependent types to avoid some run-time errors in FRP. We suggest that one could go further and develop a domain specific type system for FRP that makes the FRP based ABS more predictable and that would support further mathematical analysis of its properties. Furthermore, moving to dependent types would pose a unique benefit over the traditional object-oriented approach and should allow us to express and guarantee even more properties at compile time. We leave this for further research.

In our pure functional approach, agent identity is not as clear as in traditional object-oriented programming, where an agent can be hidden behind a polymorphic interface which is much more abstract than in our approach. Also the identity of an agent is much clearer in object-oriented programming due to the concept of object-identity and the encapsulation of data and methods.

We can conclude that the main difficulty of a pure functional approach evolves around the communication and interaction between agents, which is a direct consequence of the issue with agent identity. Agent interaction is straight-forward in object-oriented programming, where it is achieved using method-calls mutating the internal state of the agent, but that comes at the cost of a new class of bugs due to implicit data flow. In pure functional programming these data flows are explicit but our current approach of feeding back the states of all agents as inputs is not very general and we have added further mechanisms of agent interaction which we had to omit due to lack of space.

\section*{Acknowledgments}
The authors would like to thank I. Perez, H. Nilsson, J. Greensmith for constructive comments and valuable discussions.

\bibliographystyle{../../../templates/IEEEtran/bibtex/IEEEtran}
\bibliography{../../../../references/phdReferences.bib}

\appendices

\newpage
\section{Functional reactive code of the agent-based SIR model}
\label{app:abs_code}

\begin{minted}[fontsize=\footnotesize, linenos]{haskell}
data SIRState = Susceptible | Infected | Recovered deriving (Eq)
data SIRMsg = Contact SIRState deriving (Eq)

type SIRAgentState = SIRState

type SIREnvironment = [AgentId]

type SIRAgentDef = AgentDef SIRAgentState SIRMsg SIREnvironment
type SIRAgentBehaviour = AgentBehaviour SIRAgentState SIRMsg SIREnvironment
type SIRAgentBehaviourReadEnv = ReactiveBehaviourReadEnv SIRAgentState SIRMsg SIREnvironment
type SIRAgentBehaviourIgnoreEnv = ReactiveBehaviourIgnoreEnv SIRAgentState SIRMsg SIREnvironment
type SIRAgentIn = AgentIn SIRAgentState SIRMsg SIREnvironment
type SIRAgentOut = AgentOut SIRAgentState SIRMsg SIREnvironment
type SIRAgentObservable = AgentObservable SIRAgentState

type SIREventSource = EventSource SIRAgentState SIRMsg SIREnvironment

-------------------------------------------------------------------------------
infectivity :: Double
infectivity = 0.05

contactRate :: Double
contactRate = 5

illnessDuration :: Double
illnessDuration = 15

contactSS :: Int
contactSS = 20

illnessTimeoutSS :: Int
illnessTimeoutSS = 2

-------------------------------------------------------------------------------
createSIRNumInfected :: Int -> Int -> IO ([SIRAgentDef], SIREnvironment)
createSIRNumInfected agentCount numInfected = do
    let agentIds = [0 .. (agentCount-1)]
    let infectedIds = take numInfected agentIds
    let susceptibleIds = drop numInfected agentIds

    adefsSusceptible <- mapM (sirAgent Susceptible) susceptibleIds
    adefsInfected <- mapM (sirAgent Infected) infectedIds

    return (adefsSusceptible ++ adefsInfected, agentIds)

sirAgent :: SIRState -> AgentId -> IO SIRAgentDef
sirAgent initS aid = do
    rng <- newStdGen
    let beh = sirAgentBehaviour rng initS
    let adef = AgentDef { 
          adId = aid
        , adState = initS
        , adBeh = beh
        , adInitMessages = NoEvent
        , adConversation = Nothing
        , adRng = rng 
        }

    return adef
   
-------------------------------------------------------------------------------
-- UTILITIES
gotInfected :: SIRAgentIn -> Rand StdGen Bool
gotInfected ain = onMessageM gotInfectedAux ain False
  where
    gotInfectedAux :: Bool -> AgentMessage SIRMsg -> Rand StdGen Bool
    gotInfectedAux False (_, Contact Infected) = randomBoolM infectivity
    gotInfectedAux x _ = return x



respondToContactWith :: SIRState -> SIRAgentIn -> SIRAgentOut -> SIRAgentOut
respondToContactWith state ain ao = onMessage respondToContactWithAux ain ao
  where
    respondToContactWithAux :: AgentMessage SIRMsg -> SIRAgentOut -> SIRAgentOut
    respondToContactWithAux (senderId, Contact _) ao = sendMessage (senderId, Contact state) ao

-- SUSCEPTIBLE
sirAgentSuceptible :: RandomGen g => g -> SIRAgentBehaviour
sirAgentSuceptible g = 
	transitionOnEvent 
		sirAgentInfectedEvent 
		(readEnv $ sirAgentSusceptibleBehaviour g) 
		(sirAgentInfected g)

sirAgentInfectedEvent :: SIREventSource
sirAgentInfectedEvent = proc (ain, ao) -> do
    let (isInfected, ao') = agentRandom (gotInfected ain) ao 
    infectionEvent <- edge -< isInfected
    returnA -< (ao', infectionEvent)

sirAgentSusceptibleBehaviour :: RandomGen g => g -> SIRAgentBehaviourReadEnv
sirAgentSusceptibleBehaviour g = proc (ain, e) -> do
    ao' <- doOnce (setAgentState Susceptible) -< agentOutFromIn ain
    returnA -< sendMessageOccasionallySrcSS 
    			g
    			(1 / contactRate)
    			contactSS
    			(randomAgentIdMsgSource (Contact Susceptible) True) -< (ao', e)

-- INFECTED
sirAgentInfected :: RandomGen g => g -> SIRAgentBehaviour
sirAgentInfected g = 
	transitionAfterExpSS 
		g 
		illnessDuration 
		illnessTimeoutSS 
		(ignoreEnv $ sirAgentInfectedBehaviour g) 
		sirAgentRecovered

sirAgentInfectedBehaviour :: RandomGen g => g -> SIRAgentBehaviourIgnoreEnv
sirAgentInfectedBehaviour g = proc ain -> do
    ao' <- doOnce (setAgentState Infected) -< agentOutFromIn ain
    returnA -< respondToContactWith Infected ain ao'

-- RECOVERED
sirAgentRecovered :: SIRAgentBehaviour
sirAgentRecovered = doOnceR $ setAgentStateR Recovered

-- INITIAL CASES
sirAgentBehaviour :: RandomGen g => g -> SIRState -> SIRAgentBehaviour
sirAgentBehaviour g Susceptible = sirAgentSuceptible g
sirAgentBehaviour g Infected = sirAgentInfected g
sirAgentBehaviour _ Recovered = sirAgentRecovered

-------------------------------------------------------------------------------
runSIR :: IO ()
runSIR = do
    -- parallel strategy, no updating/folding of environment, no shuffling, rng-seed of 42
    params <- initSimulation Parallel Nothing Nothing False (Just 42)
    (initAdefs, initEnv) <- createSIRNumInfected agentCount numInfected
    let dynamics = simulateAggregateTime initAdefs initEnv params dt t aggregate
    print dynamics
	
aggregate :: (Time, [SIRAgentObservable], SIREnvironment) -> (Time, Double, Double, Double)
aggregate (t, aobs, _) = (t, susceptibleCount, infectedCount, recoveredCount)
  where
    susceptibleCount = fromIntegral $ length $ filter ((Susceptible==) . snd) aobs
    infectedCount = fromIntegral $ length $ filter ((Infected==) . snd) aobs
    recoveredCount = fromIntegral $ length $ filter ((Recovered==) . snd) aobs
\end{minted}

\end{document}