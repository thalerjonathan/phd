\section{Reasoning}
[ ] reasoning: sd is always reproducible because all runs outside the io moand. this means that potential RNG seeds are always the same and do not depend on outside states e.g. time
[ ] for the agents this means that repeated runs with the same seed are guaranteed to result in the same dynamics as there are again no possibilities to introduce e.g. random seeds or values from the outside which may change between runs like time, files,...

-> spatial and network behaviour is EXACTLY the same except selection of neighbours => what can we reason about it regarding the dynamics?

TODO: can we formally show that the SIR approximates the SD model?

		-> my emulation of SD using ABS is really an implementation of the SD model and follows it - they are equivalent
		-> my ABS implementation is the same as / equivalent to the SD emulation
			=> thus if i can show that my SD emulation is equlas to the SD model
			=> AND that the ABS implementation is the same as the SD emulation
			=> THEN the ABS implementation is an SD implementation, and we have shown this in code for the first time in ABS


i need to get a deep understanding in writing correct code and reasoning about correctness in Haskell - look into papers:
\url{https://wiki.haskell.org/Research_papers/Testing_and_correctness}
\url{https://www.reddit.com/r/haskell/comments/4tagq3/examples_of_realworld_haskell_usage_where/}
\url{https://stackoverflow.com/questions/4077970/can-haskell-functions-be-proved-model-checked-verified-with-correctness-properti}\\

reasoning about equivalence between SD and ABS implementation in the same framework