%\documentclass[a4paper, 10pt, conference]{../../templates/IEEEconf/IEEEconf}
\documentclass[10pt, onecolumn, conference]{../../../templates/IEEEtran/IEEEtran}
%\documentclass[10pt, journal]{../../templates/IEEEtran/IEEEtran}

\usepackage{graphicx}
\usepackage{caption} 
\usepackage{subcaption}
\usepackage{hyperref}
\usepackage{listings}
\usepackage{hhline}
\usepackage{float}
\usepackage{amssymb}
\usepackage[autostyle=true]{csquotes}
\usepackage{amsmath}
\usepackage{marvosym}
\usepackage{minted}

\font\subtitlefont=cmr12 at 18pt

\title{Hybrid Agent-Based Simulation \\ {\large Towards pure functional multimethod simulation in Haskell}}

% IEEEtran journal authors
%\author{Jonathan Thaler, ̃Peer-Olaf Siebers \\ School of Computer Science \\ University of Nottingham%
%\thanks{jonathan.thaler@nottingham.ac.uk}%
%\thanks{peer-olaf.siebers@nottingham.ac.uk}
%}

% IEEEtran conference authors
\author{
	\IEEEauthorblockN{Jonathan Thaler}
	\IEEEauthorblockA{School of Computer Science\\
		University of Nottingham\\
		jonathan.thaler@nottingham.ac.uk}
		
	\and
		
	\IEEEauthorblockN{Peer-Olaf Siebers}
	\IEEEauthorblockA{School of Computer Science\\
		University of Nottingham\\
		peer-olaf.siebers@nottingham.ac.uk}
}

%\IEEEpubid{0000--0000/00\$00.00 ̃\copyright ̃2015 IEEE}

% IEEEconf authors
%\author{
%	Jonathan Thaler \\
%	\email{jonathan.thaler@nottingham.ac.uk} \\
%	\begin{affiliation}
%		School of Computer Science, University of Nottingham
%	\end{affiliation} \\
%	\and 
%	Peer-Olaf Siebers \\
%	\email{peer-olaf.siebers@nottingham.ac.uk} \\
%	\begin{affiliation}
%		School of Computer Science, University of Nottingham
%	\end{affiliation} 
%	\and 
%	Thorsten Altenkirch \\
%	\email{thorsten.altenkirch@nottingham.ac.uk} \\
%	\begin{affiliation}
%		School of Computer Science, University of Nottingham
%	\end{affiliation} 
%}

\begin{document}
\maketitle 

\begin{abstract}
TODO: the main punch is that our approach combines the best of the three simulation methodologies:
	- SD part: 	it can represent continuous time (as well as discrete) with continuous data-flows from agents which act at the same time (parallel update), can express the formulas directly in code, there exists also a small EDSL for expressing SD in our approach, can guarantee reproducibility and no drawing of random-numbers in our approach
		-> drawback over real SD: none known so far 
	- DES part: it can represent discrete time with events occurring at discrete points in time which cause an instant change in the system
		-> drawback over real DES: time does not advance discretely to the next event which results of course not in the performance of a real DES system
	- ABS part:	the entities of the system (=agents) can be heterogenous and pro-active in time and can have arbitrary neighbourhood (2d/3d discrete/continuous, network,...)
		-> drawback over classic ABS: none known so far 

TODO: give examples of all 3 approaches: SD \& ABS: SIR model, DES: simulation of a queuing system

TODO: describe the different approach which is necessary because of being functional 
	- data-flows \& update-strategies: sequential, parallel, concurrent, actor
	- how state is handled

TODO: main benefits
	- being explicit and polymorph about side-effects: can have 'pure' (no side-effects except state), 'random' (can draw random-numbers), 'IO' (all bets are off), STM (concurrency) agents
	- hybrid between SD and ABS due to continuous time AND parallel dataFlow (parallel update-strategy)
	- being update-strategy polymorph (TODO: this is just an asumption atm, need to prove this): 4 different update-strategies, one agent implementation
	- parallel update-strategy: lack of implicit side-effects makes it work without any danger of data-interference
	- recursive simulation
	- reasoning about correctness
	- reasoning about dynamics 
	- testing with quickcheck much more convenient
	- expressivity:
		-> 1:1 mapping of SD to code: can express the SD formulas directly in code
		-> directly expressing state-charts in code
	
TODO: what we need to show / future work
	- can we do DES? e.g. single queue with multiple servers? also specialist vs. generalist
	- implement concurrent and actor update-strategies
	- reasoning about correctness: implement Gintis \& Ionescous papers 
	- reasoning about dynamics: implement Gintis \& Ionescous papers
	
TODO: describing how things are treated different
	- time is represented using the FRP concept: Signal-Functions which are sampled at (fixed) time-deltas, the dt is never visible directly but only reflected in the code and read-only.
	- no method calls => continuous data-flow instead
	- no global shared mutable environment, having different options:
		-> non-active read-only (SIR): no agent, as additional argument to each agent
		-> pro-active read-only (?): environment as agent, broadcast environment updates as data-flow
		-> non-active read/write (?): no agent, shared data as STM as additional argument to each agent
		-> pro-active read/write (Sugarscape): environment as, shared data as STM as additional argument to each agent
	- parallel update only, sequential is deliberately abandoned due to:
		-> reality does not behave this way
		-> if we need transactional behaviour, can use STM which is more explicit
		-> it is translates directly to a map which is very easy to reason about (sequential is basically a fold which is much more difficult to reason about)
		-> is more natural in functional programming
	- state is handled using FRP: recursive arrows and continuations
		
So far, the pure functional paradigm hasn't got much attention in Agent-Based Simulation (ABS) where the dominant programming paradigm is object-orientation, with Java, Python and C++ being its most prominent representatives. We claim that pure functional programming using Haskell is very well suited to implement complex, real-world agent-based models and brings with it a number of benefits. To show that we implemented the seminal Sugarscape model in Haskell in our library \textit{FrABS} which allows to do ABS the first time in the pure functional programming language Haskell. To achieve this we leverage the basic concepts of ABS with functional reactive programming using Yampa. The result is a surprisingly fresh approach to ABS as it allows to incorporate discrete time-semantics similar to Discrete Event Simulation and continuous time-flows as in System Dynamics. In this paper we will show the novel approach of functional reactive ABS through the example of the SIR model, discuss implications, benefits and best practices.
\end{abstract}

\begin{IEEEkeywords}
Haskell, Functional Programming, Verification
\end{IEEEkeywords}

%*******************************************************************************
%*********************************** First Chapter *****************************
%*******************************************************************************

\chapter{Introduction}  %Title of the First Chapter
I noticed that it is pretty hard to convince an agent-based economics specialist who is not a computer scientist about a pure functional approach. My conjecture is that the implementation technique and method does not matter much to them because they have very little knowledge about programming and are almost always self-taught - they don't know about software-engineering, nothing about proper software-design and architecture, nothing about software-maintenance, nothing about unit-testing,... In the end they just "hack" the simulation in whatever language they are able to: C++, Visual Basic, Java or toolboxes like Netlogo. For them it is all about to \textit{get things done somehow} and not to get things done the right way or in a beautiful way - the way and the method doesn't matter, its just a necessary evil which needs to be done. Thus if functional programming could make their lives easier, then they will definitely welcome it. But functional programming is, i think, harder to learn and harder to understand - so one needs to provide an abstraction through EDSL. So I REALLY need to come up with convincing arguments why to use pure functional approaches in ACE THEY can understand, otherwise I will be lost and not heard (not published,...). \\

What ACE economists care for:

\begin{itemize}
\item Very: Qualitative modelling with quantitative results
\item Yes: Easy reproducibility
\item Likely: Reasoning about convergence?
\item Likely: EDSL
\end{itemize}

My contributions are: pure functional framework, functional agent-model for market-simulations, EDSL for market-simulations, qualitative / implicit modelling with quanitative results, reasoning in my framework about convergence \\

IDEA: could I develop non-causal modelling (models are expressed in terms of non-directed equations, modelled in signal-relations) to allow for qualitative modelling for the agent-based economists? See hybrid modelling paper of Yampa. \textbf{THIS WOULD BE A HUGE NOVEL CONTRIBUTION TO ACE ESPECIALLY WHEN COMBINED WITH AN EDSL AND PROVIDING FULL REFERENTIAL TRANSPARENCY TO KEEP THE ABILITY TO REASON ABOUT CONVERGENCE}. This should be covered in the "EDSL"-paper.

TODO: maybe i should really focus only on market models? otherwise too much? \\

central novelty of my PhD: model specification = runnable code. possible through EDSL. but only in specific subfield of ACE: market-models. need a functional description of the model, then translate it to model specification in EDSL and then run it to see dynamics. But: model specification moves closer to functional programming languages. \\

another novelty approach: model specification through qualitative instead of quantiative approaches. is this possible? \\

WHY FUNCTIONAL? "because its the ultimate approach to scientific computing": fewer bugs due to mutable state (why? is thos shown obkectively by someone?), shorter (again as above, productivity), more expressive and closer to math, EDSL, EDSL=model=simulation, better parallelising due to referental transparency, reasoning \\

scientific results need to be reproduced, especially when they have high impact. a more formal approach of specifying the model and the simulation (model=simulation) could lead to easier sharing and easier reporduction without ambigouites \\

pure functional agent-model \& theory, EDSL framework in Haskell for ACE

\begin{enumerate}
\item Which kind of problem do we have?
\item What aim is there? Solving the problem? 
\item How the aim is achieved by enumerating VERY CLEAR objectives.
\item What the impact one expects (hypothesis) and what it is (after results).
\end{enumerate}

Note: It is not in the interest of the researcher to develop new economic theories but to research the use of functional methods (programming and specification) in agent-based computational economics (ACE).

NOTE: Get the reader’s attention early in the introduction: motivation, significance, originality and novelty.

\section{Methods}
Methods need to be selected to implement the simulations. Special emphasis will be put on functional ones which will then be compared to established methods in the field of ABM/S and ACE. \\

Claim: non-programming environments are considered to be not powerful enough to capture the complexity of ACE implementations thus a programming approach to ACE will be always required.

\section{Scenarios}
To apply and test functional methods in ACE, four scenarios of ACE are selected and then the methods applied and compared with each other to see how each of them perform in comparison. The 4 selected scenarios represent a selection of the challenges posed in ACE: from very abstract ones to very operational ones.

\section{Comparison}
Each of the selected scenarios is then implemented using the selected methods where each solution is then compared against the following criteria: 

\begin{enumerate}
\item suitability for scientific computation
\item robustness
\item error-sources
\item testability
\item stability
\item extendability
\item size of code
\item maintainability
\item time taken for development
\item verification \& correctness
\item replications \& parallelism
\item EDSL
\end{enumerate}

This will then allow to compare the different methods against each other and to show under which circumstances functional methods shine and when they should not be used.

\section{Agent-Based Modelling and Simulation (ABM/S)}
ABM/S is a method of modelling and simulating a system where the global behaviour may be unknown but the behaviour and interactions of the parts making up the system is of knowledge (Wooldrige, M. (2009). An Introduction to MultiAgent Systems. John Wiley & Sons). Those parts, called agents, are modelled and simulated out of which then the aggregate global behaviour of the whole system emerges. Thus the central aspect of ABM/S is the concept of an Agent which can be understood as a metaphor for a pro-active unit, able to spawn new Agents, and interacting with other Agents in a network of neighbours by exchange of messages. The implementation of Agents can vary and strongly depends on the programming language and the kind of domain the simulation and model is situated in.

\section{Agent-Based Economics (ACE)}
According to Leigh Tesfatsion (Tesfatsion, L. (2006). Agent-based computational economics: A constructive approach to economic theory. In Tesfatsion, L. and Judd, K. L., editors, Handbook of Computational Economics, volume 2, chapter 16, pages 831–880. Elsevier, 1 edition.), one of the leading figures, ACE is "[...] computational modelling of economic processes (including whole economies) as open-ended dynamic systems of interacting agents." - thus lending perfectly to the use of ABM/S as already the name suggests. Whereas classical economic models fall short by only looking at the average, pure rational, individual interacting in anonymous markets, the ACE approach looks at heterogeneous, non-rational individuals interacting with each other in networks (Kirman, A. (2010). Complex Economics: Individual and Collective Rationality. Routledge, London ; New York, NY.). Thus ACE can be understood as a combination of computer-science, cognitive/social science and evolutionary economics.

\section{Functional programming}
TODO: read \cite{Backus1978}

The state-of-the-art approach to implementing Agents are object-oriented methods and programming as the metaphor of an Agent as presented above lends itself very naturally to object-orientation (OO). The author of this thesis claims that OO in the hands of inexperienced or ignorant programmers is dangerous, leading to bugs and hardly maintainable and extensible code. The reason for this is that OO provides very powerful techniques of organising and structuring programs through Classes, Type Hierarchies and Objects, which, when misused, lead to the above mentioned problems. Also major problems, which experts face as well as beginners are 1. state is highly scattered across the program which disguises the flow of data in complex simulations and 2. objects don’t compose as well as functions. The reason for this is that objects always carry around some internal state which makes it obviously much more complicated as complex dependencies can be introduced according to the internal state.
All this is tackled by (pure) functional programming which abandons the concept of global state, Objects and Classes and makes data-flow explicit. This then allows to reason about correctness, termination and other properties of the program e.g. if a given function exhibits side-effects or not. Other benefits are fewer lines of code, easier maintainability and ultimately fewer bugs thus making functional programming the ideal choice for scientific computing and simulation and thus also for ACE. A very powerful feature of functional programming is Lazy evaluation. It allows to describe infinite data-structures and functions producing an infinite stream of output but which are only computed as currently needed. Thus the decision of how many is decoupled from how to (Hughes, J. (1989). Why functional programming matters. Comput. J., 32(2):98–107.).
The most powerful aspect using pure functional programming however is that it allows the design of embedded domain specific languages (EDSL). In this case one develops and programs primitives e.g. types and functions in a host language (embed) in a way that they can be combined. The combination of these primitives then looks like a language specific to a given domain, in the case of this thesis ACE. The ease of development of EDSLs in pure functional programming is also a proof of the superior extensibility and composability of pure functional languages over OO (Henderson P. (1982). Functional Geometry. Proceedings of the 1982 ACM Symposium on LISP and Functional Programming.).
One of the most compelling example to utilize pure functional programming is the reporting of Hudak (Hudak P., Jones M. (1994). Haskell vs. Ada vs. C++ vs. Awk vs. ... An Experiment in Software Prototyping Productivity. Department of Computer Science, Yale University.)  where in a prototyping contest of DARPA the Haskell prototype was by far the shortest with 85 lines of code. Also the Jury mistook the code as specification because the prototype did actually implement a small EDSL which is a perfect proof how close EDSL can get to and look like a specification.

Functional languages can best be characterized by their way computation works: instead of \textit{how} something is computed, \textit{what} is computed is described. Thus functional programming follows a declarative instead of an imperative style of programming. The key points are:
\begin{itemize}
\item No assignment statements - variables values can never change once given a value.
\item Function calls have no side-effect and will only compute the results - this makes order of execution irrelevant, as due to the lack of side-effects the logical point in \textit{time} when the function is calculated within the program-execution does not matter.
\item higher-order functions
\item lazy evaluation
\item Looping is achieved using recursion, mostly through the use of the general fold or the more specific map.
\item Pattern-matching
\end{itemize}

This alone does not really explain the \textit{real} advantages of functional programming and one must look for better motivations using functional programming languages. One motivation is given in \cite{Hughes1989} which is a great paper explaining to non-functional programmers what the significance of functional programming is and helping functional programmers putting functional languages to maximum use by showing the real power and advantages of functional languages. The main conclusion is that \textit{modularity}, which is the key to successful programming, can be achieved best using higher-order functions and lazy evaluation provided in functional languages like Haskell. \cite{Hughes1989} argues that the ability to divide problems into sub-problems depends on the ability to glue the sub-problems together which depends strongly on the programming-language and \cite{Hughes1989} argues that in this ability functional languages are superior to structured programming.

TODO: comparison of functional and object-oriented programming. My points are:
\begin{itemize}
\item The way state can be changed and treated - distributed over multiple objects - is often very difficult to understand.
\item Inheritance is a dangerous thing if not used with care because inheritance introduces very strong dependencies which cannot be changed during runtime anymore.
\item Objects don't compose very well: \url{http://zeroturnaround.com/rebellabs/why-the-debate-on-object-oriented-vs-functional-programming-is-all-about-composition/}
\item (Nearly) impossible to reason about programs
\end{itemize}

In conclusion the upsides of functional programming as opposed to OO are:
\begin{itemize}
\item Much more explicit flow of data \& control
\item Much better compose-able
\item Much better parallelism
\end{itemize}

\section{Related Research}
Tim Sweeney, CTO of Epic Games gave an invited talk about how "future programming languages could help us write better code" by "supplying stronger typing, reduce run-time failures;  and the need for pervasive concurrency support, both implicit and explicit, to effectively exploit the several forms of parallelism present in games and graphics." \cite{Sweeney2006}. Although the fields of games and agent-based simulations seem to be very different in the end, they have also very important similarities: both are simulations which perform numerical computations and update objects - in games they are called "game-objects" and in abm they are called agents but they are in fact the same thing - in a loop either concurrently or sequential. His key-points were:

\begin{itemize}
\item Dependent types as the remedy of most of the run-time failures.
\item Parallelism for numerical computation: these are pure functional algorithms, operate locally on mutable state. Haskell ST, STRef solution enables encapsulating local heaps and mutability within referentially transparent code.
\item Updating game-objects (agents) concurrently using STM: update all objects concurrently in arbitrary order, with each update wrapped in atomic block - depends on collisions if performance goes up.
\end{itemize}

\section{Background}

\subsection{Schelling Segregation}
We follow in our implementation the original paper of Schelling as in \cite{schelling_dynamic_1971} where we focus on the \textit{Area Distribution} section (Schelling starts with movement in a linear, 1-dimensional world where agents are able to move to the nearest point which meets the agents satisfaction but this is not what we follow here). One assumes a discrete 2-dimensional lattice-world with NxM fields. Each field is either occupied by an agent of a given color (e.g. Red or Green) or is free. Each field has 8 neighbours, which denotes a Moore-Neighbourhood. In Schellings original work the lattice-world is limited at its borders but we assume a torus world which is wrapped around in both the x- and y-dimensions resulting in 8 neighbours also for fields at the border. The occupation density was set by Schelling to be about 70\%-75\% which he identifies as being a setting which allows the agents to move around freely without making the lattice-world too sparse.
Now the agents make their move sequentially one after another. In each move an agent calculates the number of neighbours which are of equal color. If the number satisfies the agents needs about the neighbourhood then the agent is regarded as being 'happy' and will stay on this field. On the other hand the agent moves to the nearest unoccupied field which satisfies its needs. An agent which moves selects an unoccupied place randomly relative from its current place within a rectangle of side-length 2r where its current place is at the center. The interpretation for that behaviour is that agents won't move too far as it could be costly. Also children might attend a school in this area or the family has friends in this area, so they don't want to break that.



Agents just move depending on their movement-strategy to another place if they are not happy on the current one - they don't care how the target place is in the present or in the future, they will decide again in the next time-step. The interpretation for that behaviour is: agents want to 'just get out' at any cost, not caring what the future place will look like - it might be better or worse but they will see then.

\subsubsection{Optimizing behaviour}
TODO: define utility

The original schelling model didn't have a move-optimizing behaviour, meaning agents are just binary: if it is happy it will not move, if it is unhappy it will move but they won't care where they move. We introduce local move-optimizing behaviours which can be interpreted as being realistic in the real-world. It is important to note that we focus on \textit{local} instead of \textit{global} move-optimization: the agents are limited in their reasoning-capabilities and have limited information available: they cannot check out \textit{every} place and pick the globally best one.\\

\subsubsection{Anticipating behaviour}
Schelling explicitly mentions in \cite{schelling_dynamic_1971} that nobody anticipates moves of others. This is what we introduce using the recursive simulation.

TODO: is this optimizing behaviour in the spirit of schellings original work? 

\paragraph{Optimizing future} Agents pick an unoccupied random place and move to it if it increases their utility in the future. The interpretation for that behaviour is: agents heard about a place which will be cool in the future.

\paragraph{Optimizing present \& future} Agents pick an unoccupied random place and move to it if it increases their utility in the now and in the future. The interpretation for that behaviour is: agents heard about a cool spot in town, check it out and move to it if they like it but they also anticipate the coolness of the place in the future and if it seems that the place is going down then they won't move there.

\subsection{Related Research}
TODO: \cite{kirman_complex_2010} mention kirman complex economics where he investigates the model more in depth


\section{Reasoning}
i need to get a deep understanding in writing correct code and reasoning about correctness in Haskell - look into papers:
\url{https://wiki.haskell.org/Research_papers/Testing_and_correctness}
\url{https://www.reddit.com/r/haskell/comments/4tagq3/examples_of_realworld_haskell_usage_where/}
\url{https://stackoverflow.com/questions/4077970/can-haskell-functions-be-proved-model-checked-verified-with-correctness-properti}\\

\section{Testing}

\cite{perez_testing_2017}

\subsection{Time-Traveling}

\cite{perez_back_2017}

\section{Time-Traveling}
\cite{perez_back_2017}

[ ] time in FrABS: when 0dt then still actions can occur when not relying on time semantics
[ ] what about time-travel in abms for introspection during running it? this is much easier in FrABS

\section{Equivalence of SD and ABS implementation}
After having shown that both the SD and ABS approaches are equivalent by visually comparing the dynamics of Figure \ref{fig:sir_sd_dynamics} to the ones of Figure \ref{fig:sir_10000_01dt}, we ask the question whether we can also show the equivalence of both approaches from their implementation.
First we need to show that the SD implementation is correct as in \textit{is an implementation of the mathematical definition} as it is the foundation upon we build our equivalence.

\subsection{Correctness of SD implementation}
The mathematical formulas for susceptible, infected an recovered stocks with the infection- and recover-rates are directly translated into the SD implementation as seen in Appendix \ref{app:sd_code}. We can thus assume that the translation from the mathematical definition to Haskell code is correct as it \textit{is} exactly the same. This leaves us with the question how the values of the stocks and flows are distributed between each other. When solving the mathematical equations numerically, one is dividing time into infinitely small intervals and calculates the new value of each formula at each time-interval at the same time: the values update all at the same time.
The same needs to happen in our SD implementation which is indeed the case: \textit{runSD} as seen in line 127 uses \textit{simulateTime} behind the scenes with the \textit{parallel} update-strategy which uses a \textit{map} to iterate over all agents. This implies by definition of map that all stocks and flows, which are in fact agents, update virtually at the same time. Due to pureness of runSD we can rule out side-effects which could only occur when running in the IO Monad or using unsafePerformIO \footnote{Again we use unsafePerformIO in initializing the previously mentioned agent-id generator but it is never used within the runSD implementation or any of the stocks and flows.}. 
It is clear now that the values update at the same time and calculate the correct values as defined in the mathematical formulas. The distribution of the values happens by using absolute stock- and flow-ids and it is easy to check that the ids used in \textit{flowInFrom/flowOutTo} and \textit{stockInFrom/stockOutTo} build the correct connections as seen in the Stock-And-Flow diagram in Figure \ref{fig:sir_sd_stockflow_diagramm}.
This leaves us with the implementation of \textit{integral}. Theoretically we arrive at the correct mathematical solution if we use infinitely small $\Delta t$ but this is of course impossible using a computer. When looking at the implementation of \textit{integral} it becomes apparent that it uses the rectangle-rule. TODO: need more details and look a bit into theory of numerically solving integral (e.g. classic newton, or runge-kutta). The rectangle rule needs small $\Delta t$ for sufficient accuracy as we have shown in Figure \ref{fig:sd_plots}.
For our reference dynamics in Figure \ref{fig:sir_sd_dynamics} we used $\Delta t = 0.001$, which computes 150,000 steps. When comparing it with dynamics as seen in Figure \ref{fig:sd_anylogic} generated by the professional software package AnyLogic Personal Learning Edition 8.1.0 we can safely say that a $\Delta t = 0.001$ is sufficiently small to generate matching accuracy.

\begin{figure}
	\centering
	\includegraphics[width=.4\textwidth, angle=0]{./../shared/fig/anylogic/SIR_SD_DYNAMICS_ANYLOGIC.png}
	\caption{Dynamics of the SIR model using the SD approach generated with AnyLogic Personal Learning Edition 8.1.0. Population Size $N$ = 1,000, contact rate $\beta =  \frac{1}{5}$, infection probability $\gamma = 0.05$, illness duration $\delta = 15$ with initially 1 infected agent. Simulation run for 150 time-steps.}
	\label{fig:sd_anylogic}
\end{figure}

\subsection{Correctness of ABS implementation}
The question is now what \textit{correctness} of an ABS implementation means. This is indeed a very non-trivial problem and is highly dependent on the model. In our case of the agent-based SIR implementation we define correctness to mean \textit{qualitatively approximating the SD dynamics}. This means that to show that our ABS implementation is correct, we need to show that it is equivalent to the SD implementation. TODO: this is a bit short, maybe we can say more about this. need to read a few papers and get better understanding on this topic

Also we ask the question: why is it that the susceptible agents must make the contact and the infected only replying? can't just the infected agents make contact thus saving a round-trip? also why are not both making contact? When comparing the two contact-policies as seem in Figure \ref{fig:sir_abs_correctness_contactpolicies}, it is clear that both don't match neither the SD dynamics of Figure \ref{fig:sir_sd_dynamics} and ABM dynamics of Figure \ref{fig:sir_10000_01dt}.

\begin{figure*}
\begin{center}

	\begin{tabular}{c c}
		\begin{subfigure}[b]{0.5\textwidth}
			\centering
			\includegraphics[width=1\textwidth, angle=0]{./../shared/fig/frabs/INFECTED_CONTACT_ONLY_SIR_10000agents_150t_01dt_parallel.png}
			\caption{Infected agents make contact.}
			\label{fig:sir_abs_correctness_contact_infected}
		\end{subfigure}
    	&
		\begin{subfigure}[b]{0.5\textwidth}
			\centering
			\includegraphics[width=1\textwidth, angle=0]{./../shared/fig/frabs/INFECTED_AND_SUSCEPTIBLE_CONTACT_SIR_10000agents_150t_01dt_parallel.png}
			\caption{Susceptible and Infected agents make contact.}
			\label{fig:sir_abs_correctness_contact_both}
		\end{subfigure}
	\end{tabular}
	
	\caption{Dynamics of different contact policies. Using same model parameters as in Figure \ref{fig:sir_10000_01dt}.} 
	\label{fig:sir_abs_correctness_contactpolicies}
\end{center}
\end{figure*}

\subsection{Equivalence}
TODO: need to think about this very deeply, but basically it is all about probabilities which increase with number of messages which increase with number of infected. we need somehow to match the number of messages a susceptible receives to the stocks and flows formulas
TODO: can we show why the infected themselves do not (and must not) make contact and only reply to incoming contacts?

\section{Agent-Based Dynamics}
We can now run simulations of our agent-based approach and see whether they reach the SD dynamics of Figure \ref{fig:sir_sd_dynamics}. In Figure \ref{fig:sir_abs_approximating_1dt} the dynamics of a first naive attempt using 1,000 agents with $\Delta t= 1.0$ can be seen. 

\begin{figure}
\begin{center}
	\begin{tabular}{c c}
		\begin{subfigure}[b]{0.3\textwidth}
			\centering
			\includegraphics[width=1\textwidth, angle=0]{./../shared/fig/frabs/SIR_1000agents_150t_1dt_NOSS_parallel.png}
			\caption{$\Delta t = 1.0$}
			\label{fig:sir_abs_approximating_1dt}
		\end{subfigure}
    	&
		\begin{subfigure}[b]{0.3\textwidth}
			\centering
			\includegraphics[width=1\textwidth, angle=0]{./../shared/fig/frabs/SIR_1000agents_150t_05dt_NOSS_parallel.png}
			\caption{$\Delta t = 0.5$}
			\label{fig:sir_abs_approximating_05dt}
		\end{subfigure}
    	
    	\\
    	
		\begin{subfigure}[b]{0.3\textwidth}
			\centering
			\includegraphics[width=1\textwidth, angle=0]{./../shared/fig/frabs/SIR_1000agents_150t_02dt_NOSS_parallel.png}
			\caption{$\Delta t = 0.2$}
			\label{fig:sir_abs_approximating_02dt}
		\end{subfigure}
		& 
		\begin{subfigure}[b]{0.3\textwidth}
			\centering
			\includegraphics[width=1\textwidth, angle=0]{./../shared/fig/frabs/SIR_1000agents_150t_01dt_NOSS_parallel.png}
			\caption{$\Delta t = 0.1$}
			\label{fig:sir_abs_approximating_01dt}
		\end{subfigure}
	\end{tabular}
	
	\caption{Naive simulation of SIR using agent-based approach. Population Size $N$ = 1,000, contact rate $\beta = \frac{1}{5}$, infection probability $\gamma = 0.05$, illness duration $\delta = 15$ with initially 1 infected agent. Simulation run for 150 time-steps with various $\Delta t$.} 
	\label{fig:sir_abs_dynamics_naive}
\end{center}
\end{figure}

%TODO: reproducing about the same dynamics of the SD-solution (1.0 dt)
%	- super-sampling: 	contact-rate ss high, illness time-out low 
%	- agent number:		1000 vs. 10.000 agents
%	- Susceptibles making contact and infected response VS. only Infected make contact
%	- update-strat:		Sequential vs. Parallel
%	- making contact: susceptible only vs. susceptible AND infected
%	- do conversations make a difference?
%	- does a delayed switch (dSwitch) in transitions makes a difference?

Clearly something is going wrong as the dynamics do not resemble the ones of SD in any way with only 10 agents making the transition to infected to recovered. The problem is that we are running into sampling issues. TODO: explain deeper and better

\subsection{Sampling the System}
When sampling the system, the correct $\Delta t$ must be selected which depends on the highest frequency which occurs in a time-reactive function in the whole system. For example in the SIR model we want infected agents to make on average contact with $\beta = 5$ other agents per time-unit, which means with a frequency of $\frac{1}{5}$. This functionality is built on Yampas function \textit{occasionally} which behaviour we investigated under differing $\Delta t$ with the above frequency. In this investigation we simply sampled occasionally with different $\Delta t$ for a duration of $t = 1,000$ and the event-frequency of $\frac{1}{5}$. The result can be seen in Figure \ref{fig:sampling_occasionally_5evts} and is quite striking. The plot clearly shows that occasionally needs a quite high sampling frequency even for a comparatively low event-frequency, which becomes of course worse for higher event-frequencies.

The other time-reactive function which occurs in the SIR model is the timed transition from infected to recovered which occurs on average with an exponential random-distribution after $\delta = 15$. This functionality is built on a custom implementation of Yampas \textit{after} which creates an event after a time-out of the passed in time-duration drawn from an exponential random-distribution. Clearly this function has different semantics as although it also continuously emit events over time - \textit{NoEvent} before the time was hit, and \textit{Event b} after the time hit - the relevant point is that it switches to Event at some discrete point in time. This is implemented as simply adding up the $\Delta t$ until the accumulator is greater of equal than the previously drawn exponential time-out. We also investigated the behaviour of this function under varying $\Delta t$ using a time-out of $\delta = 15$. Our approach was to sample the \textit{afterExp} until an event occurs and then see when it has occurred. We run this with 10,000 replications with different random-number seeds and average the resulting times. The results can be seen in Figure \ref{fig:sampling_afterExp_5time}. The result is striking in another way: this function seems to be pretty invariant to the time-deltas, for obvious reasons: we are basically just interested in the "after"-condition of the whole semantics whereas in occasionally we are interested in the "repeatedly"-conditions. If we under-sample the \textit{afterExp} then we can be off by one $\Delta t$. If we under-sample \textit{occasionally} we keep loosing events - the less difference between $\Delta t$ and event-frequency, the more events we lose. Of course \textit{afterExp} can also be used for very short time-outs e.g. $\frac{1}{5}$. We have investigated the behaviour of this function for various $\Delta t$ as well as seen in Figure \ref{fig:sampling_afterExp_02time}. Here the result is much more striking and shows that \textit{afterExp} is vulnerable to small time-outs as well as \textit{occasionally}.  
To show that \textit{occasionally} is not vulnerable to very low frequencies of e.g. one event every 5 time-steps we plotted the behaviour of this under varying time-steps in Figure \ref{fig:sampling_occasionally_02evts}. The result shows that for low frequencies occasionally works fine with larger $\Delta t$.

\begin{figure}
\begin{center}
	\begin{tabular}{c c}
	\begin{subfigure}[b]{0.5\textwidth}
			\centering
			\includegraphics[width=.6\textwidth, angle=0]{./../shared/fig/sampling/samplingTest_occasionally_5evts.png}
			\caption{Sampling \textit{occasional} with a frequency of $\frac{1}{5}$ (average of 5 events per time-unit). The theoretical average is 5000 events within this time-frame.}
			\label{fig:sampling_occasionally_5evts}
		\end{subfigure}
		& 
		\begin{subfigure}[b]{0.5\textwidth}
			\centering
			\includegraphics[width=.6\textwidth, angle=0]{./../shared/fig/sampling/samplingTest_occasionally_02evts.png}
			\caption{Sampling \textit{occasional} with a frequency of 5 (average of 0.2 events per time-unit). The theoretical average is 200 events within this time-frame.}
			\label{fig:sampling_occasionally_02evts}
		\end{subfigure}
		
		\\
		
		\begin{subfigure}[b]{0.5\textwidth}
			\centering
			\includegraphics[width=.6\textwidth, angle=0]{./../shared/fig/sampling/samplingTest_afterExp_5time.png}
			\caption{Sampling \textit{afterExp} with an average time-out of 5.}
			\label{fig:sampling_afterExp_5time}
		\end{subfigure}
		& 
		\begin{subfigure}[b]{0.5\textwidth}
			\centering
			\includegraphics[width=.6\textwidth, angle=0]{./../shared/fig/sampling/samplingTest_afterExp_02time.png}
			\caption{Sampling \textit{afterExp} with an average time-out of 0.2.}
			\label{fig:sampling_afterExp_02time}
		\end{subfigure}
	\end{tabular}
	
	\caption{Sampling the \textit{afterExp} and \textit{occasional} functions to visualise the influence of sampling frequencies on the occurrence of the respective events. $\Delta t$ are [ 5, 2, 1, $\frac{1}{2}$, $\frac{1}{5}$, $\frac{1}{10}$, $\frac{1}{20}$, $\frac{1}{50}$, $\frac{1}{100}$ ]. The experiments for \textit{afterExp} used 10,000 replications. The experiments for \textit{occasional} ran for $t= 1,000$ with 100 replications.} 
	\label{fig:sampling_tests}
\end{center}
\end{figure}

Using these observation we run simulations with varying $\Delta t$ with $\Delta = 0.5$, $\Delta = 0.2$ and $\Delta = 0.1$ with the results visible in Figures \ref{fig:sir_abs_approximating_05dt}, \ref{fig:sir_abs_approximating_02dt} and \ref{fig:sir_abs_approximating_01dt} but still when decreasing $\Delta t$ we don't approach the SD dynamics. As previously mentioned the agent-based approach is a discrete one which means that with increasing number of agents, the discrete dynamics approximate the continuous dynamics of the SD simulation. We run further simulations with $\Delta = 0.1$ but with varying agent numbers to see the influence with the results seen in Figure \ref{fig:sir_abs_approximating}.

\begin{figure}
\begin{center}
	\begin{tabular}{c c}
		\begin{subfigure}[b]{0.3\textwidth}
			\centering
			\includegraphics[width=1\textwidth, angle=0]{./../shared/fig/frabs/SIR_100agents_150t_01dt_NOSS_parallel.png}
			\caption{100 Agents}
			\label{fig:sir_abs_approximating_100}
		\end{subfigure}
    	&
		\begin{subfigure}[b]{0.3\textwidth}
			\centering
			\includegraphics[width=1\textwidth, angle=0]{./../shared/fig/frabs/SIR_1000agents_150t_01dt_NOSS_parallel.png}
			\caption{1,000 Agents}
			\label{fig:sir_abs_approximating_1000}
		\end{subfigure}
    	
    	\\
    	
		\begin{subfigure}[b]{0.3\textwidth}
			\centering
			\includegraphics[width=1\textwidth, angle=0]{./../shared/fig/frabs/SIR_5000agents_150t_01dt_NOSS_parallel.png}
			\caption{5,000 Agents}
			\label{fig:sir_abs_approximating_5000}
		\end{subfigure}
		& 
		\begin{subfigure}[b]{0.3\textwidth}
			\centering
			\includegraphics[width=1\textwidth, angle=0]{./../shared/fig/frabs/SIR_10000agents_150t_01dt_NOSS_parallel.png}
			\caption{10,000 Agents}
			\label{fig:sir_abs_approximating_10000}
		\end{subfigure}
	\end{tabular}
	
	\caption{Varying agent numbers with same model-parameters except population size. All simulations run for 150 time-steps with $\Delta t = 0.1$}
	\label{fig:sir_abs_approximating}
\end{center}
\end{figure}

Still the dynamics of 10,000 Agents do not match the dynamics of the SD simulation perfectly. This is because as opposed to the SD simulation, which is deterministic, the agent-based approach is inherently a stochastic one as we continuously draw from random-distributions which drive our state-transitions. What we see in Figure \ref{fig:sir_abs_approximating} is then just a single run where the dynamics would result in slightly different shapes when run with a different random-number generator seed. The agent-based approach thus generates a distribution of dynamics over which ones needs to average to arrive at the correct solution. This can be done using replications in which the simulation is run with the exact same parameters multiple times but each with a different random-number generator see. The resulting dynamics are then averaged and the result is then regarded as the correct dynamics.
We have done this as can be seen in Figure \ref{fig:sir_abs_agents_repls}, using 10 replications, which matches the SD dynamics to a very satisfactory level. Note that in the replications we are using 10 initially infected agents to ensure that no simulation run will terminate too early (meaning that the disease gets extinct after a few time steps) which would offset the dynamics completely. This happens due to "unlucky" random distributions which can be repaired by introducing more initially infected agents which increases the probability of spreading the disease in the very early stage of the simulation drastically. We found that when using 10 initially infected agents in a population of 5,000 (which amounts to 0.2\%) is enough to never result in an early terminating simulation. In the case of 100 agents 10 initially infected ones might be too much and distorts the dynamics but this is irrelevant in this case. This is also a fundamental difference between SD and ABS: the dynamics of the agent-based approach can result in a wide range of scenarios which includes also the one in which the disease gets extinct in the early stages (a lucky coincidence for mankind) - this is simply not possible in the SD approach. So we can argue that ABS is much closer to reality than SD as it allows to explore alternate futures in the dynamics.

\begin{figure}
\begin{center}
	\begin{tabular}{c c}
		\begin{subfigure}[b]{0.3\textwidth}
			\centering
			\includegraphics[width=1\textwidth, angle=0]{./../shared/fig/frabs/SIR_100agents_150t_01dt_NOSS_parallel_10replications.png}
			\caption{100 Agents}
			\label{fig:sir_abs_agents_repls_100}
		\end{subfigure}
    	&
		\begin{subfigure}[b]{0.3\textwidth}
			\centering
			\includegraphics[width=1\textwidth, angle=0]{./../shared/fig/frabs/SIR_1000agents_150t_01dt_NOSS_parallel_10replications.png}
			\caption{1,000 Agents}
			\label{fig:sir_abs_agents_repls_1000}
		\end{subfigure}
    	
    	\\
    	
		\begin{subfigure}[b]{0.3\textwidth}
			\centering
			\includegraphics[width=1\textwidth, angle=0]{./../shared/fig/frabs/SIR_5000agents_150t_01dt_NOSS_parallel_10replications.png}
			\caption{5,000 Agents}
			\label{fig:sir_abs_agents_repls_5000}
		\end{subfigure}
		&
		\begin{subfigure}[b]{0.3\textwidth}
			\centering
			\includegraphics[width=1\textwidth, angle=0]{./../shared/fig/frabs/SIR_10000agents_150t_01dt_NOSS_parallel_10replications.png}
			\caption{10,000 Agents}
			\label{fig:sir_abs_agents_repls_10000}
		\end{subfigure}
	\end{tabular}
	
	\caption{Dynamics of Figure \ref{fig:sir_abs_approximating} averaged over 10 replications with initially 10 infected agents.} 
	\label{fig:sir_abs_agents_repls}
\end{center}
\end{figure}

When comparing the results of the dynamics of the agent-based approach from Figure \ref{fig:sir_abs_approximating} and Figure \ref{fig:sir_abs_agents_repls} to the SD dynamics of Figure \ref{fig:sir_sd_dynamics} it becomes apparent that by increasing the number of agents the dynamics approximate the SD dynamics with increasing accuracy. Still although using 5,000 agents and replications seem to be not enough yet, we need to increase our number of agents to 10,000

Still although using a quite small $\Delta t = 0.1$ and using replications we are nowhere close to the SD dynamics. The only option we have is to further decrease $\Delta t$. Of course performance is a big issue and it decreases as $\Delta t$ get smaller and smaller. This is because when running a simulation for a duration of $t$ and sampling it with $\Delta t$ when the steps to calculate is $\frac{t}{\Delta t}$. In each step all agents are run, messages delivered and environments folded and updated which implies that the more steps the lower the performance. If we could perform super-sampling just for the given high-frequency functions with the whole system running in lower frequency then we could achieve a substantial performance boost.

\subsection{Super-Sampling}
In Yampa there exists a function \textit{embed} which allows to run a given signal-function with provided $\Delta t$ but the problem is that this function does not really help because it does not return a signal-function. What we need is a signal-function which takes the number of super-samples \textit{n}, the signal-function \textit{sf} to sample and returns a new signal-function which performs super-sampling on it:

\begin{minted}[fontsize=\footnotesize]{haskell}
superSampling :: Int -> SF a b -> SF a [b]
\end{minted}

It does this by evaluating \textit{sf} for \textit{n} times, each with $\Delta t = \frac{\Delta t}{n}$ and the same input argument \textit{a} for all \textit{n} evaluations. At time 0 no super-sampling is done and just a single output of \textit{sf} is calculated. A list of \textit{b} is returned with length of \textit{n} containing the result of the \textit{n} evaluations of \textit{sf}. If 0 or less super samples are requested exactly one is calculated.

We ran tests super-sampling both \textit{occasionally} Figure \ref{fig:sampling_occasionally_ss_02evts}, Figure \ref{fig:sampling_occasionally_ss_5evts} and \textit{afterExp} Figure \ref{fig:sampling_afterExp_ss_5time}, Figure \ref{fig:sampling_afterExp_ss_02time}. They work the same way as above except that now $\Delta t = 1.0$ but using increasing numbers of super-samples. The results are as expected: as the number of super-samples increase, so increases the accuracy.

\begin{figure*}
\begin{center}
	\begin{tabular}{c c}
		\begin{subfigure}[b]{0.5\textwidth}
			\centering
			\includegraphics[width=.6\textwidth, angle=0]{./../shared/fig/sampling/samplingTest_occasionally_ss_02evts.png}
			\caption{Super-Sampling the \textit{occasional} function with event-frequency of 5 (average of 0.2 events per time-unit). The theoretical average is 20 event within this time-frame.}
			\label{fig:sampling_occasionally_ss_02evts}
		\end{subfigure}
	
		&
		
		\begin{subfigure}[b]{0.5\textwidth}
			\centering
			\includegraphics[width=.6\textwidth, angle=0]{./../shared/fig/sampling/samplingTest_occasionally_ss_5evts.png}
			\caption{Super-Sampling the \textit{occasional} function with event-frequency of $\frac{1}{5}$ (average of 5 events per time-unit). The theoretical average is 500 event within this time-frame.}
			\label{fig:sampling_occasionally_ss_5evts}
		\end{subfigure}

		\\
		
		\begin{subfigure}[b]{0.5\textwidth}
			\centering
			\includegraphics[width=.6\textwidth, angle=0]{./../shared/fig/sampling/samplingTest_afterExp_SS_5time.png}
			\caption{Super-Sampling the \textit{afterExp} function with average time-out of 5.}
			\label{fig:sampling_afterExp_ss_5time}
		\end{subfigure}

		&
		
		\begin{subfigure}[b]{0.5\textwidth}
			\centering
			\includegraphics[width=.6\textwidth, angle=0]{./../shared/fig/sampling/samplingTest_afterExp_SS_02time.png}
			\caption{Super-Sampling the \textit{afterExp} function with average time-out of 0.2.}
			\label{fig:sampling_afterExp_ss_02time}
		\end{subfigure}
	\end{tabular}
	
	\caption{Super-Sampling the \textit{afterExp} and \textit{occasional} functions to visualize the influence of increasing number of super-samples on the average occurrence of the respective events. The $\Delta t = 1.0$ in both cases with super-samples of [1, 2, 5, 10, 100, 1000]. The experiments for \textit{afterExp} used 10,000 replications. The experiments for \textit{occasional} ran for $t = 100$ with 100 replications.} 
	\label{fig:supersampling_tests}
\end{center}
\end{figure*}

At first this might not seem to be a real win as we still need to calculate a big number of samples every time. The big win comes though when these super-sampled signal-functions are embedded in a larger system which could run on a comparatively low frequency of $\Delta t = 1.0$. So we are then increasing the sampling-frequency just where we need it and keep the frequency low where it is not required.

We are using super-sampling in our SIR implementation to increase performance. We do this by setting $\Delta t = 1.0$ and super-sampling the relevant functions with time-semantics which are \textit{transitionAfterExp} and \textit{sendMessageOccationallySrc}. For both we provide in our EDSL versions which support super-sampling:

\begin{minted}[fontsize=\footnotesize]{haskell}
sendMessageOccasionallySrcSS :: RandomGen g => g -> Double -> Int -> MessageSource 
                                -> SF (AgentOut, e) AgentOut
                                
transitionAfterExpSS :: RandomGen g => g -> Double -> Int 
                        -> AgentBehaviour -> AgentBehaviour -> AgentBehaviour
\end{minted}

Both now take an additional parameter which determines the number of super-samples to be calculated. According to the above observations of the \textit{occasionally} and \textit{afterExp} functions which are the foundations of both of the functions we sample \textit{sendMessageOccasionallySrcSS} with 20 super-samples and \textit{transitionAfterExpSS} with 2. This will ensure that by using $\Delta t = 1.0$ we only calculate $t$ steps when running a simulation for $t$ time but that we sample our relevant functions with enough resolution to capture its frequencies. Optimally we should increase the number of super-samples for \textit{sendMessageOccasionallySrcSS} to about 100. This will result in lower performance as \textit{every} agent will perform this super-sampling. So in the end it is a struggle of performance vs. sufficiently close approximation. We define the number of super-samples in lines 29 and 32 and use the functions in line 96 and 106 of Appendix \ref{app:abs_code}.

TODO: 10.000 with SS and dt = 1.0 with ss

Unfortunately when setting $\Delta t = 1.0$ the dynamics of the agent-based approach won't approach the dynamics of the SD, despite using super-sampling as can be seen in Figure \ref{fig:sir_10000_1dt}.

\begin{figure}
\begin{center}
	\begin{tabular}{c c}
		\begin{subfigure}[b]{0.5\textwidth}
			\centering
			\includegraphics[width=.8\textwidth, angle=0]{./../shared/fig/frabs/SIR_10000agents_150t_1dt_parallel.png}
			\caption{$\Delta t = 1.0$}
			\label{fig:sir_10000_1dt}
		\end{subfigure}
	
		&
		
		\begin{subfigure}[b]{0.5\textwidth}
			\centering
			\includegraphics[width=.8\textwidth, angle=0]{./../shared/fig/frabs/SIR_10000agents_150t_01dt_parallel.png}
			\caption{$\Delta t = 0.1$}
			\label{fig:sir_10000_01dt}
		\end{subfigure}
	\end{tabular}
	
	\caption{Comparing the influence of different $\Delta t$. Both dynamics were generated with the same configuration of 10,000 agents, super-sampling enabled as described and the same model-parameters. When using $\Delta t = 1.0$, the dynamics do not match the ones of the SD approach, whereas in the case of $\Delta t = 0.1$, they can be seen as matching completely.} 
	\label{fig:sir_10000_dt_comparisons}
\end{center}
\end{figure}

When reflecting on the messaging mechanism it becomes apparent that a round-trip from sender to receiver and back takes $2 \Delta t$. A round-trip happens in our agent-based SIR approach to implement the transition from infected to susceptible - susceptible agents send \textit{Contact Susceptible} messages to random agents (except itself) where only infected agents reply with a \textit{Contact Infected} message. This means that it takes $2 \Delta t$ until a susceptible agent might get infected. This becomes an issue if we want to match the dynamics of our agent-based approach to the one of SD in which no time-delay happens - the agents act instantaneous with each other during one time-step. 
We have two solutions for this problem: either we resort to \textit{conversations} or we increase the global sampling frequency of the system which matches the \textit{message frequency} of messages which are subject to round-trips. Implementing conversations is only available in the \textit{sequential} update-strategy and is much more involved, so we followed the approach of increasing the frequency. As can be seen in Figure \ref{fig:sir_10000_01dt} when setting $t\Delta = 0.1$ the resulting dynamics are a sufficiently good approximation to the SD solution.

\section{Discussion}

\subsection{Other Models}
TODO: mention that we have also implemented other models, which also work without time-semantics (all agents make a move at discrete time-steps and do not really rely on some notion of time). 

\subsection{Time-Semantics}
The main reason for building our pure functional ABMS approach on top of Yampa was to leverage the powerful time-semantics of Yampa which allows us to implement important concepts of ABMS:

state-chart: agents are at all time of their life-cycle in one state and can switch between multiple states using transitions 
timed transitions: transition to another state/behaviour happens at a discrete time
rate transitions: transition happens with a given rate
message transition: transition upon receiving a given message 

\subsection{Agents as Signals}
Due to the underlying nature and motivation of Functional Reactive Programming (und im speziellen) Yampa, Agents can be seen as Signals which is generated and consumed by a Signal-Function which is the behaviour of an Agent. If an Agent does not change the OUTPUT-signal is constant, if the agent changes e.g. by sending a message, changing its state,... the OUTPUT signal changes. A dead agent has no signal at all.

\subsection{Time-Sampling}
sampling rate depends on the transition times \& rates of the model. when e.g. the contact rate is 5 then the sampling dt should be below 0.2

\subsection{System Dynamics}
can emulate system dynamics due to the parallel update-strategy and continuous time-flow semantics

\subsection{Discrete Event Simulation}
DES in FrABMS? how easily can we implement server/queue systems? do they also look like a specification? potential problem: ordering of messages is not guaranteed by now

\subsection{Advantages}
advantages:
	- no side-effects within agents leads to much safer code
	- edsl for time-semantics
	- declarative style: agent-implementation looks like a model-specification
	- reasoning and verification
	- sequential and parallel
	- powerful time-semantics
	- arrowized programming is optional and only required when utilizing yampas time-semantics. if the model does not rely on time-semantics, it can use monadic-programming by building on the existing monadic functions in the EDSL which allow to run in the State-Monad which simplifies things very much
	- when to use yampas arrowized programing: time-semantics, simple state-chart agents 
	- when not using yampas facilities: in all the other cases e.g. SugarScape is such a case as it proceeds in unit time-steps and all agents act in every time-step
	- can implement System Dynamics building on Yampas facilities with total ease	
	- get replications for free without having to worry about side-effects and can even run them in parallel without headaches
	- cant mess around with time because delta-time is hidden from you (intentional design-decision by Yampa). this would be only very difficult and cumbersome to achieve in an object-oriented approach. TODO: experiment with it in Java - how could we actually implement this? I think it is impossible: may only achieve this through complicated application of patterns and inheritance but then has the problem of how to update the dt and more important how to deal with functions like integral which accumulates a value through closures and continuations. We could do this in OO by having a general base-class e.g. ContinuousTime which provides functions like updateDt and integrate, but we could only accumulate a single integral value.
	- reproducibility statically guaranteed
	- cannot mess around with dt
	- code == specification
	- rule out serious class of bugs
	- different time-sampling leads to different results e.g. in wildfire \& SIR but not in Prisoners Dilemma. why? probabilistic time-sampling?
	- reasoning about equivalence between SD and ABS implementation in the same framework
	- recursive implementations
	
	- we can statically guarantee the reproducibility of the simulation because: no side effects possible within the agents which would result in differences between same runs (e.g. file access, networking, threading), also timedeltas are fixed and do not depend on rendering performance or userinput	
	
\subsection{Disadvantages}
disadvantages:
	- performance is low
	- reasoning about performance is very difficult
	- very steep learning curve for non-functional programmers
	- learning a new EDSL
	- think ABMS different: when to use async messages, when to use sync conversations


[ ] important: increasing sampling freqzency and increasing number of steps so that the same number of simulation steps are executed should lead to same results. but it doesnt. why?
[ ] hypothesis: if time-semantics are involved then event ordering becomes relevant for emergent patterns. there are no tine semantics in heroes and cowards but in the prisoners dilemma
[ ] can we implement different types of agents interacting with each other in the same simulation ? with different behaviour funcs, digferent state? yes, also not possible in NetLogo to my knowledge. but they must have the same messages, emvironment 

[ ] Hypothesis: we can combine with FrABS agent-based simulation and system dynamics (this has been proved by example!)

\section*{Acknowledgments}
The authors would like to thank I. Perez, H. Nilsson, J. Greensmith for constructive comments and valuable discussions.

\bibliographystyle{../../../templates/IEEEtran/bibtex/IEEEtran}
\bibliography{../../../../references/phdReferences.bib}

\appendices

\newpage
\section{Examples}
In this appendix we give a list of all the examples we have implemented and discuss implementation details relevant \footnote{The examples are freely available under \url{https://github.com/thalerjonathan/phd/tree/master/coding/libraries/frABS/examples}}. The examples were implemented as use-cases to drive the development of \textit{FrABS} and to give code samples of known models which show how to use this new approach. Note that we do not give an explanation of each model as this would be out of scope of this paper but instead give the major references from which an understanding of the model can be obtained.

We distinguish between the following attributes
\begin{itemize}
	\item Implementation - Which style was used? Either Pure, Monadic or Reactive. Examples could have been implemented in all of them.
	\item Yampa Time-Semantics - Does the implemented model make use of Yampas time-semantics e.g. occasional, after,...? Yes / No.
	\item Update-Strategy - Which update-strategy is required for the given example? It is either Sequential or Parallel or both. In the case of Sequential Agents may be shuffled or not.
	\item Environment - Which kind of environment is used in the given example? Possibilities are 2D/3D Discrete/Continuous or Network. In case of a Parallel Update-Strategy, collapsing may become necessary, depending on the semantics of the model. Also it is noted if the environment has behaviour. Note that an implementation may also have no environment which is noted as None. Although every model implemented in \textit{FrABs} needs to set up some environment, it is not required to use it in the implementation.
	\item Recursive - Is this implementation making use of the recursive features of \textit{FrABS} Yes/No (only available in sequential updating)?
	\item Conversations - Is this implementation making use of the conversations features of \textit{FrABS} Yes/No (only available in sequential updating)?
\end{itemize}

\subsection{Sugarscape}
This is a full implementation of the famous Sugarscape model as described by Epstein \& Axtell in their book \cite{epstein_growing_1996}. The model description itself has no real time-semantics, the agents act in every time-step. Only the environment may change its behaviour after a given number of steps but this is easily expressed without time-semantics as described in the model by Epstein \& Axtell \footnote{Note that this implementation has about 2600 lines of code which - although it includes both a pure and monadic implementation - is significant lower than e.g. the Java-implementation \url{http://sugarscape.sourceforge.net/} with about 6000. Of course it is difficult to compare such measures as we do not include FrABS itself into our measure.}.

\begin{center}
\begin{tabular}{l || l }
\textbf{Implementation}			& Pure, Monadic \\
\textbf{Yampa Time-Semantics}	& No \\
\textbf{Update-Strategy}		& Sequential, shuffling \\
\textbf{Environment}			& 2D Discrete, behaviour \\
\textbf{Recursive}				& No \\
\textbf{Conversations}			& Yes \\
\end{tabular}
\end{center}

\subsection{Agent\_Zero}
This is an implementation of the \textit{Parable 1} from the book of Epstein \cite{epstein_agent_zero:_2014}.

\begin{center}
\begin{tabular}{l || l }
\textbf{Implementation}			& Pure, Monadic \\
\textbf{Yampa Time-Semantics}	& No \\
\textbf{Update-Strategy}		& Parallel, Sequential, shuffling \\
\textbf{Environment}			& 2D Discrete, behaviour, collapsing \\
\textbf{Recursive}				& No \\
\textbf{Conversations}			& No \\
\end{tabular}
\end{center}

\subsection{Schelling Segregation}
This is an implementation of \cite{schelling_dynamic_1971} with extended agent-behaviour which allows to study dynamics of different optimization behaviour: local or global, nearest/random, increasing/binary/future. This is also the only 'real' model in which the recursive features were applied \footnote{The example of Recursive ABS is just a plain how-to example without any real deeper implications.}.

\begin{center}
\begin{tabular}{l || l }
\textbf{Implementation}			& Pure \\
\textbf{Yampa Time-Semantics}	& No \\
\textbf{Update-Strategy}		& Sequential, shuffling \\
\textbf{Environment}			& 2D Discrete \\
\textbf{Recursive}				& Yes (optional) \\
\textbf{Conversations}			& No \\
\end{tabular}
\end{center}

\subsection{Prisoners Dilemma}
This is an implementation of the Prisoners Dilemma on a 2D Grid as discussed in the papers of \cite{nowak_evolutionary_1992}, \cite{huberman_evolutionary_1993} and TODO: cite my own paper on update-strategies.

TODO: implement

\subsection{Heroes \& Cowards}
This is an implementation of the Heroes \& Cowards Game as introduced in \cite{wilensky_introduction_2015} and discussed more in depth in TODO: cite my own paper on update-strategies.

TODO: implement

\subsection{SIRS}
This is an early, non-reactive implementation of a spatial version of the SIRS compartment model found in epidemiology. Note that although the SIRS model itself includes time-semantics, in this implementation no use of Yampas facilities were made. Timed transitions and making contact was implemented directly into the model which results in contacts being made on every iteration, independent of the sampling time. Also in this sample only the infected agents make contact with others, which is not quite correct when wanting to approximate the System Dynamics model (see below). It is primarily included as a comparison to the later implementations (Fr*SIRS) of the same model  which make full use of \textit{FrABS} and to see the huge differences the usage of Yampas time-semantics can make.

\begin{center}
\begin{tabular}{l || l }
\textbf{Implementation}			& Pure, Monadic \\
\textbf{Yampa Time-Semantics}	& No \\
\textbf{Update-Strategy}		& Parallel, Sequential with shuffling \\
\textbf{Environment}			& 2D Discrete \\
\textbf{Recursive}				& No \\
\textbf{Conversations}			& No \\
\end{tabular}
\end{center}

\subsection{Fr(Spatial$|$Network)SIRS}
This is the reactive implementations of both 2D spatial and network (complete graph, Erdos-Renyi and Barbasi-Albert) versions of the SIRS compartment model. Unlike SIRS these examples make full use of the time-semantics provided by Yampa and show the real strength provided by \textit{FrABS}.

\begin{center}
\begin{tabular}{l || l }
\textbf{Implementation}			& Reactive \\
\textbf{Yampa Time-Semantics}	& Yes \\
\textbf{Update-Strategy}		& Parallel \\
\textbf{Environment}			& 2D Discrete, Network \\
\textbf{Recursive}				& No \\
\textbf{Conversations}			& No \\
\end{tabular}
\end{center}

\subsection{System Dynamics SIR}
This is an emulation of the System Dynamics model of the SIR compartment model in epidemiology. It was implemented as a proof-of-concept to show that \textit{FrABS} is able to implement even System Dynamic models because of its continuous-time and time-semantic features. Connections between stocks \& flows are hardcoded, after all System Dynamics completely lacks the concept of spatial- or network-effects. Note that describing the implementation as Reactive may seem not appropriate as in System Dynamics we are not dealing with any events or reactions to it - it is all about a continuous flow between stocks. In this case we wanted to express with Reactive that it is implemented using the Arrowized notion of Yampa which is required when one wants to use Yampas time-semantics anyway.

\begin{center}
\begin{tabular}{l || l }
\textbf{Implementation}			& Reactive \\
\textbf{Yampa Time-Semantics}	& Yes \\
\textbf{Update-Strategy}		& Parallel \\
\textbf{Environment}			& None \\
\textbf{Recursive}				& No \\
\textbf{Conversations}			& No \\
\end{tabular}
\end{center}

\subsection{WildFire}
This is an implementation of a very simple Wildfire model inspired by an example from AnyLogic\texttrademark with the same name.

\begin{center}
\begin{tabular}{l || l }
\textbf{Implementation}			& Reactive \\
\textbf{Yampa Time-Semantics}	& Yes \\
\textbf{Update-Strategy}		& Parallel \\
\textbf{Environment}			& 2D Discrete \\
\textbf{Recursive}				& No \\
\textbf{Conversations}			& No \\
\end{tabular}
\end{center}

\subsection{Double Auction}
This is a basic implementation of a double-auction process of a model described by \cite{breuer_endogenous_2015}. This model is not relying on any environment at the moment but could make use of networks in the future for matching offers.

\begin{center}
\begin{tabular}{l || l }
\textbf{Implementation}			& Pure, Monadic \\
\textbf{Yampa Time-Semantics}	& No \\
\textbf{Update-Strategy}		& Parallel \\
\textbf{Environment}			& None \\
\textbf{Recursive}				& No \\
\textbf{Conversations}			& No \\
\end{tabular}
\end{center}

\subsection{Proof of concepts}
\subsubsection{Recursive ABS} This example shows the very basics of how to implement a recursive ABS using \textit{FrABS}. Note that recursive features only work within the sequential strategy.

\begin{center}
\begin{tabular}{l || l }
\textbf{Implementation}			& Pure \\
\textbf{Yampa Time-Semantics}	& No \\
\textbf{Update-Strategy}		& Sequential \\
\textbf{Environment}			& None \\
\textbf{Recursive}				& Yes \\
\textbf{Conversations}			& No \\
\end{tabular}
\end{center}

\subsubsection{Conversation} This example shows the very basics of how to implement conversations in \textit{FrABS}. Note that conversations only work within the sequential strategy.

\begin{center}
\begin{tabular}{l || l }
\textbf{Implementation}			& Pure \\
\textbf{Yampa Time-Semantics}	& No \\
\textbf{Update-Strategy}		& Sequential \\
\textbf{Environment}			& None \\
\textbf{Recursive}				& No \\
\textbf{Conversations}			& Yes \\
\end{tabular}
\end{center}

%\newpage
%\section{Recursive Agent-Based Simulation}
The idea for this paper arose from my idea of \textit{anticipating agents}, which can project their actions in the future. Because this paper is not as polished as the draft for programming paradigms, we opted not to include it as an appendix and only give its basic ideas and results for the experiments conducted so far. Note that we were not able to find any research regarding recursive ABS \footnote{We found a paper on recursive simulation in general \cite{gilmer_recursive_2000} which focuses on military simulation implemented in C++. Its main findings are that deterministic models seem to benefit significantly from using recursions of the simulation for the decision making process and that when using stochastic models this benefit seems to be lost.}.
In Recursive ABS agents are able to halt time and 'play through' an arbitrary number of actions, compare their outcome and then to resume time and continue with a specifically chosen action e.g. the best performing or the one in which they haven't died. More precisely, what we want is to give an agent the ability to run the simulation recursively a number of times where the this number is not determined initially but can depend on the outcome of the recursive simulation. So Recursive ABS gives each Agent the ability to run the simulation locally from its point of view to anticipate its actions in the future and change them in the present.
We investigate the famous Schelling Segregation \cite{schelling_dynamic_1971} and endow our agents with the ability to project their actions into the future by recursively running simulations. Based on the outcome of the recursions they are then able to determine whether their move increases their utility in the future or not. The main finding for now is that it does not increase the convergence speed to equilibrium but can lead to extreme volatility of dynamics although the system seems to be near to complete equilibrium. In the case of a 10x10 field it was observed that although the system was nearly in its steady state - all but one agent were satisfied - the move of a single agent caused the system to become completely unstable and depart from its near-equilibrium state to a highly volatile and unstable state.

This approach of course rises a few questions and issues. The main problem of our approach is that, depending on ones view-point, it is violating the principles of locality of information and limit of computing power. To recursively run the simulation the agent which initiates the recursion is feeding in all the states of the other agents and calculates the outcome of potentially multiple of its own steps, each potentially multiple recursion-layers deep and each recursion-layer multiple time-steps long. Both requires that each agent has perfect information about the complete simulation \textit{and} can compute these 3-dimensional recursions, which scale exponentially. In the social sciences where agents are often designed to have only very local information and perform low-cost computations it is very difficult or impossible to motivate the usage of recursive simulations - it simply does not match the assumptions of the real world, the social sciences want to model. In general simulations, where it is much more commonly accepted to assume perfect information and potentially infinite amount of computing power this approach is easily motivated by a constructive argument: it is possible to build, thus we build it.
Another fundamental question regards the meaning and epistemology behind an entity running simulations. Of course, this strongly depends on the context: in ACE it may be understood as a search for optimizing behaviour, in Social Simulation it may be interpreted as a kind of free will: the agent who is initiating the recursion can be seen as 'knowing' that it is running inside a simulation, thus in this context free will is seen as being able to anticipate ones actions and change them.
When talking about recursion it is always the question of the depth of the recursion and because as we are running on computers we need to terminate at some point. Accelerating Turing machines (also known as Zeno Machine) are theoretically able to calculate an infinite regress but this raises again epistemological questions and can be seen as having religious character as discussed e.g. in Tiplers Omega Point, Bostroms simulation argument \cite{bostrom_are_2003} and its theological implications \cite{steinhart_theological_2010}. So the ultimate question this research leaves is what the outcome would be when running a recursive ABS on a Zeno Machine/Accelerated Turing Machine? \footnote{Anyway this would mean we have infinite amount of computing power - I am sure that in this case we don't worry the slightest about recursive ABS any more.}

At the moment this idea lies dormant as the intention was just to develop it far enough to give a proof-of-concept and see some results. Having achieved this we arrived at the conclusion, that the results are not really ground-breaking. This stems from the fact that Schelling segregation is not the best model to demonstrate this technique and that we are thus lacking the right model in which recursive ABS is the real killer-feature. Also to pursue this direction further and treat it in-depth, would require much more time and give the PhD a complete different spin. Still it is useful in supporting our move towards pure functional ABS as we are convinced that recursion is comparably easy to implement because the language is built on it and due to the lack of side-effects \footnote{Actually implementing it was \textit{really hard} but we wouldn't dare to implement this into an object-oriented language or into an object-oriented ABS framework.}.

\newpage
\section{FrSD SIR Code}

\begin{minted}[fontsize=\footnotesize, linenos]{haskell}
totalPopulation :: Double
totalPopulation = 1000

infectivity :: Double
infectivity = 0.05

contactRate :: Double
contactRate = 5

avgIllnessDuration :: Double
avgIllnessDuration = 15

-- Hard-coded ids for stocks & flows interaction
susceptibleStockId :: AgentId
susceptibleStockId = 0

infectiousStockId :: AgentId
infectiousStockId = 1

recoveredStockId :: AgentId
recoveredStockId = 2

infectionRateFlowId :: AgentId
infectionRateFlowId = 3

recoveryRateFlowId :: AgentId
recoveryRateFlowId = 4

-------------------------------------------------------------------------------
-- STOCKS
susceptibleStock :: Stock
susceptibleStock initValue = proc ain -> do
    let infectionRate = flowInFrom infectionRateFlowId ain

    stockValue <- (initValue+) ^<< integral -< (-infectionRate)
    
    let ao = agentOutFromIn ain
    let ao0 = setDomainState stockValue ao
    let ao1 = stockOutTo stockValue infectionRateFlowId ao0

    returnA -< ao1

infectiousStock :: Stock
infectiousStock initValue = proc ain -> do
    let infectionRate = flowInFrom infectionRateFlowId ain
    let recoveryRate = flowInFrom recoveryRateFlowId ain

    stockValue <- (initValue+) ^<< integral -< (infectionRate - recoveryRate)
    
    let ao = agentOutFromIn ain
    let ao0 = setDomainState stockValue ao
    let ao1 = stockOutTo stockValue infectionRateFlowId ao0 
    let ao2 = stockOutTo stockValue recoveryRateFlowId ao1
    
    returnA -< ao2

recoveredStock :: Stock
recoveredStock initValue = proc ain -> do
    let recoveryRate = flowInFrom recoveryRateFlowId ain

    stockValue <- (initValue+) ^<< integral -< recoveryRate
    
    let ao = agentOutFromIn ain
    let ao' = setDomainState stockValue ao

    returnA -< ao'

-------------------------------------------------------------------------------
-- FLOWS
infectionRateFlow :: Flow
infectionRateFlow = proc ain -> do
    let susceptible = stockInFrom susceptibleStockId ain 
    let infectious = stockInFrom infectiousStockId ain

    let flowValue = (infectious * contactRate * susceptible * infectivity) / totalPopulation
    
    let ao = agentOutFromIn ain
    let ao' = flowOutTo flowValue susceptibleStockId ao
    let ao'' = flowOutTo flowValue infectiousStockId ao'

    returnA -< ao''

recoveryRateFlow :: Flow
recoveryRateFlow = proc ain -> do
    let infectious = stockInFrom infectiousStockId ain

    let flowValue = infectious / avgIllnessDuration
    
    let ao = agentOutFromIn ain
    let ao' = flowOutTo flowValue infectiousStockId ao
    let ao'' = flowOutTo flowValue recoveredStockId ao'

    returnA -< ao''

-------------------------------------------------------------------------------
createSysDynSIR :: [SDDef]
createSysDynSIR = 
    [ susStock
    , infStock
    , recStock
    , infRateFlow
    , recRateFlow
    ]
  where
    initialSusceptibleStockValue = totalPopulation - 1
    initialInfectiousStockValue = 1
    initialRecoveredStockValue = 0

    susStock = createStock susceptibleStockId initialSusceptibleStockValue susceptibleStock
    infStock = createStock infectiousStockId initialInfectiousStockValue infectiousStock
    recStock = createStock recoveredStockId initialRecoveredStockValue recoveredStock

    infRateFlow = createFlow infectionRateFlowId infectionRateFlow
    recRateFlow = createFlow recoveryRateFlowId recoveryRateFlow

-------------------------------------------------------------------------------
runSysDynSIRSteps :: IO ()
runSysDynSIRSteps = print dynamics
  where
    -- SD run completely deterministic, this is reflected also in the types of 
    -- the createSysDynSIR and runSD functions which are pure-functions  
    sdDefs = createSysDynSIR
    sdObs = runSD sdDefs dt t 
            
    dynamics = map calculateDynamics sdObs
    
-- NOTE: here we rely on the fact the we have exactly three stocks and sort them by their id to access them
--          stock id 0: Susceptible
--          stock id 1: Infectious
--          stock id 2: Recovered
--          the remaining items are the flows
calculateDynamics :: [SDObservable] -> (Double, Double, Double)
calculateDynamics unsortedStocks = (susceptibleCount, infectedCount, recoveredCount) 
  where
    stocks = sortBy (\s1 s2 -> compare (fst s1) (fst s2)) unsortedStocks
    ((_, susceptibleCount) : (_, infectedCount) : (_, recoveredCount) : _) = stocks
\end{minted}

\end{document}