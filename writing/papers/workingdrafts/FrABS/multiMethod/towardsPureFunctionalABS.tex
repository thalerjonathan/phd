%\documentclass[a4paper, 10pt, conference]{../../templates/IEEEconf/IEEEconf}
\documentclass[10pt, onecolumn, conference]{../../../templates/IEEEtran/IEEEtran}
%\documentclass[10pt, journal]{../../templates/IEEEtran/IEEEtran}

\usepackage{graphicx}
\usepackage{caption} 
\usepackage{subcaption}
\usepackage{hyperref}
\usepackage{listings}
\usepackage{hhline}
\usepackage{float}
\usepackage{amssymb}
\usepackage[autostyle=true]{csquotes}
\usepackage{amsmath}
\usepackage{marvosym}
\usepackage{minted}

\font\subtitlefont=cmr12 at 18pt

\title{Towards pure functional agent-based simulation}

% IEEEtran journal authors
%\author{Jonathan Thaler, ̃Peer-Olaf Siebers \\ School of Computer Science \\ University of Nottingham%
%\thanks{jonathan.thaler@nottingham.ac.uk}%
%\thanks{peer-olaf.siebers@nottingham.ac.uk}
%}

% IEEEtran conference authors
\author{
	\IEEEauthorblockN{Jonathan Thaler}
	\IEEEauthorblockA{School of Computer Science\\
		University of Nottingham\\
		jonathan.thaler@nottingham.ac.uk}
		
	\and
		
	\IEEEauthorblockN{Peer-Olaf Siebers}
	\IEEEauthorblockA{School of Computer Science\\
		University of Nottingham\\
		peer-olaf.siebers@nottingham.ac.uk}
}

%\IEEEpubid{0000--0000/00\$00.00 ̃\copyright ̃2015 IEEE}

% IEEEconf authors
%\author{
%	Jonathan Thaler \\
%	\email{jonathan.thaler@nottingham.ac.uk} \\
%	\begin{affiliation}
%		School of Computer Science, University of Nottingham
%	\end{affiliation} \\
%	\and 
%	Peer-Olaf Siebers \\
%	\email{peer-olaf.siebers@nottingham.ac.uk} \\
%	\begin{affiliation}
%		School of Computer Science, University of Nottingham
%	\end{affiliation} 
%	\and 
%	Thorsten Altenkirch \\
%	\email{thorsten.altenkirch@nottingham.ac.uk} \\
%	\begin{affiliation}
%		School of Computer Science, University of Nottingham
%	\end{affiliation} 
%}

\begin{document}
\maketitle 

\begin{abstract}
TODO: the main punch is that our approach combines the best of the three simulation methodologies:
	- SD part: 	it can represent continuous time (as well as discrete) with continuous data-flows from agents which act at the same time (parallel update), can express the formulas directly in code, there exists also a small EDSL for expressing SD in our approach, can guarantee reproducibility and no drawing of random-numbers in our approach
		-> drawback over real SD: none known so far 
	- DES part: it can represent discrete time with events occurring at discrete points in time which cause an instant change in the system
		-> drawback over real DES: time does not advance discretely to the next event which results of course not in the performance of a real DES system
	- ABS part:	the entities of the system (=agents) can be heterogenous and pro-active in time and can have arbitrary neighbourhood (2d/3d discrete/continuous, network,...)
		-> drawback over classic ABS: none known so far 

TODO: give examples of all 3 approaches: SD \& ABS: SIR model, DES: simulation of a queuing system

TODO: describe the different approach which is necessary because of being functional 
	- data-flows \& update-strategies: sequential, parallel, concurrent, actor
	- how state is handled

TODO: main benefits
	- being explicit and polymorph about side-effects: can have 'pure' (no side-effects except state), 'random' (can draw random-numbers), 'IO' (all bets are off), STM (concurrency) agents
	- hybrid between SD and ABS due to continuous time AND parallel dataFlow (parallel update-strategy)
	- being update-strategy polymorph (TODO: this is just an asumption atm, need to prove this): 4 different update-strategies, one agent implementation
	- parallel update-strategy: lack of implicit side-effects makes it work without any danger of data-interference
	- recursive simulation
	- reasoning about correctness
	- reasoning about dynamics 
	- testing with quickcheck much more convenient
	- expressivity:
		-> 1:1 mapping of SD to code: can express the SD formulas directly in code
		-> directly expressing state-charts in code
	
TODO: what we need to show / future work
	- can we do DES? e.g. single queue with multiple servers? also specialist vs. generalist
	- reasoning about correctness: implement Gintis \& Ionescous papers 
	- reasoning about dynamics: implement Gintis \& Ionescous papers
	
TODO: describing how things are treated different
	- time is represented using the FRP concept: Signal-Functions which are sampled at (fixed) time-deltas, the dt is never visible directly but only reflected in the code and read-only.
	- no method calls => continuous data-flow instead
	- no global shared mutable environment, having different options:
		-> non-active read-only (SIR): no agent, as additional argument to each agent
		-> pro-active read-only (?): environment as agent, broadcast environment updates as data-flow
		-> non-active read/write (?): no agent, shared data as STM as additional argument to each agent
		-> pro-active read/write (Sugarscape): environment as, shared data as STM as additional argument to each agent
	- parallel update only, sequential is deliberately abandoned due to:
		-> reality does not behave this way
		-> if we need transactional behaviour, can use STM which is more explicit
		-> it is translates directly to a map which is very easy to reason about (sequential is basically a fold which is much more difficult to reason about)
		-> is more natural in functional programming
		-> it exists for 'transactional' reasons where we need mutual exclusive access to environment / other agents
			-> we provide a more explicit mechanism for this: Agent Transactions
	- state is handled using FRP: recursive arrows and continuations
	- still need transactions between two agents e.g. trading occurs over multiple steps (makeoffer, accept/refuse, finalize/abort) 
		-> exactly define what TX means in ABS
			-> exclusive between 2 agents
			-> state-changes which occur over multiple steps and are only visible to the other agents after the TX has commited
			-> no read/write access to this state is allowed to other agents while the TX is active
			-> a TX executes in a single time-step and can have an arbitrary number of tx-steps
		-> it is easily possible using method-calls in OOP but in our pure functional approach it is not possible
		-> parallel execution is being a problem here as TX between agents are very easy with sequential
		-> an agent must be able to transact with as many other agents as it wants to in the same time-step
		-> no time passes between transactions
		=> what we need is a 'all agents transact at the same time'
			-> basically we can implement it by running the SFs of the agents involved in the TX repeatedly with dt=0 until there are no more active TXs
			-> continuations (SFs) are perfectly suited for this as we can 'rollback' easily by using the SF before the TX has started
			
TODO: defining agent-based simulation
	- look into existing literature (e.g. peers paper, any logic book,...)

TODO: need to find a formal definition on agent-based simulation
	- start with wooldridge book
	- derive a more functional approach then in my paper
	
So far, the pure functional paradigm hasn't got much attention in Agent-Based Simulation (ABS) where the dominant programming paradigm is object-orientation, with Java, Python and C++ being its most prominent representatives. We claim that pure functional programming using Haskell is very well suited to implement complex, real-world agent-based models and brings with it a number of benefits. To show that we implemented the seminal Sugarscape model in Haskell in our library \textit{FrABS} which allows to do ABS the first time in the pure functional programming language Haskell. To achieve this we leverage the basic concepts of ABS with functional reactive programming using Yampa. The result is a surprisingly fresh approach to ABS as it allows to incorporate discrete time-semantics similar to Discrete Event Simulation and continuous time-flows as in System Dynamics. In this paper we will show the novel approach of functional reactive ABS through the example of the SIR model, discuss implications, benefits and best practices.
\end{abstract}

\begin{IEEEkeywords}
Haskell, Functional Programming, Verification
\end{IEEEkeywords}

%*******************************************************************************
%*********************************** First Chapter *****************************
%*******************************************************************************

\chapter{Introduction}  %Title of the First Chapter
I noticed that it is pretty hard to convince an agent-based economics specialist who is not a computer scientist about a pure functional approach. My conjecture is that the implementation technique and method does not matter much to them because they have very little knowledge about programming and are almost always self-taught - they don't know about software-engineering, nothing about proper software-design and architecture, nothing about software-maintenance, nothing about unit-testing,... In the end they just "hack" the simulation in whatever language they are able to: C++, Visual Basic, Java or toolboxes like Netlogo. For them it is all about to \textit{get things done somehow} and not to get things done the right way or in a beautiful way - the way and the method doesn't matter, its just a necessary evil which needs to be done. Thus if functional programming could make their lives easier, then they will definitely welcome it. But functional programming is, i think, harder to learn and harder to understand - so one needs to provide an abstraction through EDSL. So I REALLY need to come up with convincing arguments why to use pure functional approaches in ACE THEY can understand, otherwise I will be lost and not heard (not published,...). \\

What ACE economists care for:

\begin{itemize}
\item Very: Qualitative modelling with quantitative results
\item Yes: Easy reproducibility
\item Likely: Reasoning about convergence?
\item Likely: EDSL
\end{itemize}

My contributions are: pure functional framework, functional agent-model for market-simulations, EDSL for market-simulations, qualitative / implicit modelling with quanitative results, reasoning in my framework about convergence \\

IDEA: could I develop non-causal modelling (models are expressed in terms of non-directed equations, modelled in signal-relations) to allow for qualitative modelling for the agent-based economists? See hybrid modelling paper of Yampa. \textbf{THIS WOULD BE A HUGE NOVEL CONTRIBUTION TO ACE ESPECIALLY WHEN COMBINED WITH AN EDSL AND PROVIDING FULL REFERENTIAL TRANSPARENCY TO KEEP THE ABILITY TO REASON ABOUT CONVERGENCE}. This should be covered in the "EDSL"-paper.

TODO: maybe i should really focus only on market models? otherwise too much? \\

central novelty of my PhD: model specification = runnable code. possible through EDSL. but only in specific subfield of ACE: market-models. need a functional description of the model, then translate it to model specification in EDSL and then run it to see dynamics. But: model specification moves closer to functional programming languages. \\

another novelty approach: model specification through qualitative instead of quantiative approaches. is this possible? \\

WHY FUNCTIONAL? "because its the ultimate approach to scientific computing": fewer bugs due to mutable state (why? is thos shown obkectively by someone?), shorter (again as above, productivity), more expressive and closer to math, EDSL, EDSL=model=simulation, better parallelising due to referental transparency, reasoning \\

scientific results need to be reproduced, especially when they have high impact. a more formal approach of specifying the model and the simulation (model=simulation) could lead to easier sharing and easier reporduction without ambigouites \\

pure functional agent-model \& theory, EDSL framework in Haskell for ACE

\begin{enumerate}
\item Which kind of problem do we have?
\item What aim is there? Solving the problem? 
\item How the aim is achieved by enumerating VERY CLEAR objectives.
\item What the impact one expects (hypothesis) and what it is (after results).
\end{enumerate}

Note: It is not in the interest of the researcher to develop new economic theories but to research the use of functional methods (programming and specification) in agent-based computational economics (ACE).

NOTE: Get the reader’s attention early in the introduction: motivation, significance, originality and novelty.

\section{Methods}
Methods need to be selected to implement the simulations. Special emphasis will be put on functional ones which will then be compared to established methods in the field of ABM/S and ACE. \\

Claim: non-programming environments are considered to be not powerful enough to capture the complexity of ACE implementations thus a programming approach to ACE will be always required.

\section{Scenarios}
To apply and test functional methods in ACE, four scenarios of ACE are selected and then the methods applied and compared with each other to see how each of them perform in comparison. The 4 selected scenarios represent a selection of the challenges posed in ACE: from very abstract ones to very operational ones.

\section{Comparison}
Each of the selected scenarios is then implemented using the selected methods where each solution is then compared against the following criteria: 

\begin{enumerate}
\item suitability for scientific computation
\item robustness
\item error-sources
\item testability
\item stability
\item extendability
\item size of code
\item maintainability
\item time taken for development
\item verification \& correctness
\item replications \& parallelism
\item EDSL
\end{enumerate}

This will then allow to compare the different methods against each other and to show under which circumstances functional methods shine and when they should not be used.

\section{Agent-Based Modelling and Simulation (ABM/S)}
ABM/S is a method of modelling and simulating a system where the global behaviour may be unknown but the behaviour and interactions of the parts making up the system is of knowledge (Wooldrige, M. (2009). An Introduction to MultiAgent Systems. John Wiley & Sons). Those parts, called agents, are modelled and simulated out of which then the aggregate global behaviour of the whole system emerges. Thus the central aspect of ABM/S is the concept of an Agent which can be understood as a metaphor for a pro-active unit, able to spawn new Agents, and interacting with other Agents in a network of neighbours by exchange of messages. The implementation of Agents can vary and strongly depends on the programming language and the kind of domain the simulation and model is situated in.

\section{Agent-Based Economics (ACE)}
According to Leigh Tesfatsion (Tesfatsion, L. (2006). Agent-based computational economics: A constructive approach to economic theory. In Tesfatsion, L. and Judd, K. L., editors, Handbook of Computational Economics, volume 2, chapter 16, pages 831–880. Elsevier, 1 edition.), one of the leading figures, ACE is "[...] computational modelling of economic processes (including whole economies) as open-ended dynamic systems of interacting agents." - thus lending perfectly to the use of ABM/S as already the name suggests. Whereas classical economic models fall short by only looking at the average, pure rational, individual interacting in anonymous markets, the ACE approach looks at heterogeneous, non-rational individuals interacting with each other in networks (Kirman, A. (2010). Complex Economics: Individual and Collective Rationality. Routledge, London ; New York, NY.). Thus ACE can be understood as a combination of computer-science, cognitive/social science and evolutionary economics.

\section{Functional programming}
TODO: read \cite{Backus1978}

The state-of-the-art approach to implementing Agents are object-oriented methods and programming as the metaphor of an Agent as presented above lends itself very naturally to object-orientation (OO). The author of this thesis claims that OO in the hands of inexperienced or ignorant programmers is dangerous, leading to bugs and hardly maintainable and extensible code. The reason for this is that OO provides very powerful techniques of organising and structuring programs through Classes, Type Hierarchies and Objects, which, when misused, lead to the above mentioned problems. Also major problems, which experts face as well as beginners are 1. state is highly scattered across the program which disguises the flow of data in complex simulations and 2. objects don’t compose as well as functions. The reason for this is that objects always carry around some internal state which makes it obviously much more complicated as complex dependencies can be introduced according to the internal state.
All this is tackled by (pure) functional programming which abandons the concept of global state, Objects and Classes and makes data-flow explicit. This then allows to reason about correctness, termination and other properties of the program e.g. if a given function exhibits side-effects or not. Other benefits are fewer lines of code, easier maintainability and ultimately fewer bugs thus making functional programming the ideal choice for scientific computing and simulation and thus also for ACE. A very powerful feature of functional programming is Lazy evaluation. It allows to describe infinite data-structures and functions producing an infinite stream of output but which are only computed as currently needed. Thus the decision of how many is decoupled from how to (Hughes, J. (1989). Why functional programming matters. Comput. J., 32(2):98–107.).
The most powerful aspect using pure functional programming however is that it allows the design of embedded domain specific languages (EDSL). In this case one develops and programs primitives e.g. types and functions in a host language (embed) in a way that they can be combined. The combination of these primitives then looks like a language specific to a given domain, in the case of this thesis ACE. The ease of development of EDSLs in pure functional programming is also a proof of the superior extensibility and composability of pure functional languages over OO (Henderson P. (1982). Functional Geometry. Proceedings of the 1982 ACM Symposium on LISP and Functional Programming.).
One of the most compelling example to utilize pure functional programming is the reporting of Hudak (Hudak P., Jones M. (1994). Haskell vs. Ada vs. C++ vs. Awk vs. ... An Experiment in Software Prototyping Productivity. Department of Computer Science, Yale University.)  where in a prototyping contest of DARPA the Haskell prototype was by far the shortest with 85 lines of code. Also the Jury mistook the code as specification because the prototype did actually implement a small EDSL which is a perfect proof how close EDSL can get to and look like a specification.

Functional languages can best be characterized by their way computation works: instead of \textit{how} something is computed, \textit{what} is computed is described. Thus functional programming follows a declarative instead of an imperative style of programming. The key points are:
\begin{itemize}
\item No assignment statements - variables values can never change once given a value.
\item Function calls have no side-effect and will only compute the results - this makes order of execution irrelevant, as due to the lack of side-effects the logical point in \textit{time} when the function is calculated within the program-execution does not matter.
\item higher-order functions
\item lazy evaluation
\item Looping is achieved using recursion, mostly through the use of the general fold or the more specific map.
\item Pattern-matching
\end{itemize}

This alone does not really explain the \textit{real} advantages of functional programming and one must look for better motivations using functional programming languages. One motivation is given in \cite{Hughes1989} which is a great paper explaining to non-functional programmers what the significance of functional programming is and helping functional programmers putting functional languages to maximum use by showing the real power and advantages of functional languages. The main conclusion is that \textit{modularity}, which is the key to successful programming, can be achieved best using higher-order functions and lazy evaluation provided in functional languages like Haskell. \cite{Hughes1989} argues that the ability to divide problems into sub-problems depends on the ability to glue the sub-problems together which depends strongly on the programming-language and \cite{Hughes1989} argues that in this ability functional languages are superior to structured programming.

TODO: comparison of functional and object-oriented programming. My points are:
\begin{itemize}
\item The way state can be changed and treated - distributed over multiple objects - is often very difficult to understand.
\item Inheritance is a dangerous thing if not used with care because inheritance introduces very strong dependencies which cannot be changed during runtime anymore.
\item Objects don't compose very well: \url{http://zeroturnaround.com/rebellabs/why-the-debate-on-object-oriented-vs-functional-programming-is-all-about-composition/}
\item (Nearly) impossible to reason about programs
\end{itemize}

In conclusion the upsides of functional programming as opposed to OO are:
\begin{itemize}
\item Much more explicit flow of data \& control
\item Much better compose-able
\item Much better parallelism
\end{itemize}

\section{Related Research}
Tim Sweeney, CTO of Epic Games gave an invited talk about how "future programming languages could help us write better code" by "supplying stronger typing, reduce run-time failures;  and the need for pervasive concurrency support, both implicit and explicit, to effectively exploit the several forms of parallelism present in games and graphics." \cite{Sweeney2006}. Although the fields of games and agent-based simulations seem to be very different in the end, they have also very important similarities: both are simulations which perform numerical computations and update objects - in games they are called "game-objects" and in abm they are called agents but they are in fact the same thing - in a loop either concurrently or sequential. His key-points were:

\begin{itemize}
\item Dependent types as the remedy of most of the run-time failures.
\item Parallelism for numerical computation: these are pure functional algorithms, operate locally on mutable state. Haskell ST, STRef solution enables encapsulating local heaps and mutability within referentially transparent code.
\item Updating game-objects (agents) concurrently using STM: update all objects concurrently in arbitrary order, with each update wrapped in atomic block - depends on collisions if performance goes up.
\end{itemize}

\section{Agent-Based Simulation Defined}
Agent-Based Simulation (ABS) is a methodology to model and simulate a system where the global behaviour may be unknown but the behaviour and interactions of the parts making up the system is of knowledge. Those parts, called agents, are modelled and simulated out of which then the aggregate global behaviour of the whole system emerges. Epstein \cite{epstein_generative_2012} identifies ABS to be especially applicable for analysing \textit{"spatially distributed systems of heterogeneous autonomous actors with bounded information and computing capacity"}. Thus in the line of the simulation methods \textit{Statistic} $^\dag$, \textit{Markov} $^\ddag$, \textit{System Dynamics} $^\S$, \textit{Discrete Event} $^\mp$, ABS is the most recent development and the most powerful one as it subsumes it's predecessors features and goes beyond:

\begin{itemize}
	\item Linearity \& Non-Linearity $^{\dag \ddag \S \mp}$ - the dynamics of the simulation can exhibit both linear and non-linear behaviour. 
	\item Time $^{\dag \ddag \S \mp}$ - agents act over time, time is also the source of pro-activity.
	\item States $^{\ddag \S \mp}$ - agents encapsulate some state which can be accessed and changed during the simulation.
	\item Feedback-Loops $^{\S \mp}$ - because agents act continuously and their actions influence each other and themselves, feedback-loops are the norm in ABS. 
	\item Heterogeneity $^{\mp}$ - although agents can have same properties like height, sex,... the actual values can vary arbitrarily between agents.
	\item Interactions - agents can be modelled after interactions with an environment or other agents, making this a unique feature of ABS, not possible in the other simulation models.
	\item Spatiality \& Networks - agents can be situated within e.g. a spatial (discrete 2d, continuous 3d,...) or network environment, making this also a unique feature of ABS, not possible in the other simulation models.
\end{itemize}

\subsection{Deriving central concepts}
Before we can approach a functional view on ABS, we need to identify the central concepts of ABS on a more technical level. Unfortunately there does not exist a commonly agreed technical definition of ABS but we can draw inspiration from the closely related field of Multi-Agent Systems (MAS). It is important to understand that MAS and ABS are two different fields where in MAS the focus is much more on technical details implementing a system of interacting intelligent agents within a highly complex environment with the focus primarily on solving AI problems.

\subsubsection{Agents}
The central aspect of ABS is the concept of an agent. In MAS \cite{wooldridge_introduction_2009}, \cite{weiss_multiagent_2013} agents can be informally defined as:

\begin{itemize}
	\item They are uniquely addressable entities with some internal state over which they have full, exclusive control.
	\item They are situated in an environment which they can observe and act upon.
	\item They can interact with other agents which are situated in the same environment.
	\item They are pro-active which means they can initiate actions on their own e.g. change their internal state, interact with other agents, create new agents, terminate themselves, interact with the environment,...
\end{itemize}

\subsubsection{Environment}
The other important concept is the one of an environment. In MAS \cite{wooldridge_introduction_2009}, \cite{weiss_multiagent_2013} one distinguishes between different types of environments (based on \cite{russell_artificial_2010}):

\begin{itemize}
	\item Accessible vs. inaccessible - in an accessible environment an agent can obtain complete and accurate information from the environment. In ABS environments are generally implemented as being accessible.
	\item Deterministic vs. non-deterministic - in a deterministic environment the actions of an agent have no uncertainty and are guaranteed to have a single effect. In ABS environments are generally implemented as being deterministic.
	%\item Episodic vs. non-episodic - in an episodic environment agents act only on the current state and do not project into the future. In ABS environments are generally episodic.
	\item Static vs. dynamic - a static environment only changes due to the agents actions whereas a dynamic one has other processes which operate on it. In ABS both static and dynamic environments are common.
	\item Discrete vs. continuous - a discrete environment has only a fixed, finite number of states and actions whereas a continuous is potentially unlimited. In ABS both discrete and continuous environments are common.
\end{itemize}

Note that in MAS the focus is much more on the environment rather than on the agents where the environment is almost always a highly complex one and the agents may intelligently act on it. In ABS the focus is rather on the agents and their interactions where the environment plays a role but is not of central interest as it is almost always deterministic.

\subsection{Deriving a formal view}
In order to explore how we can implement an ABS in a pure functional way we need a sufficiently formal view on it. This will help us expressing the concepts in Haskell as formal, mathematical specifications translate easily into functional programming. There exists formalisations of MAS \cite{wooldridge_introduction_2009} but unfortunately they are not very helpful in our context as its formalization is tailored much more towards optimizing, intelligent and reasoning behaviour of agents within a highly complex and uncertain environment. What we need for ABS is a more agent-oriented approach.

An ABS is a time simulation 
Central to an ABS is the notion of time which is advanced either in discrete or continuous time-steps where discrete means advancing by a natural number time-delta and continuous by a real-valued time-delta. 
feed back

notion of time, which is also the source of proactivity
simulation is stepped in discrete or continuous time-steps
in each time-step the agents have to opportunity to act
agents can read/write the environment
agents can interact with the other agents through communication
agents can updated themselves

\section*{Acknowledgments}
The authors would like to thank I. Perez, H. Nilsson, J. Greensmith for constructive comments and valuable discussions.

\bibliographystyle{../../../templates/IEEEtran/bibtex/IEEEtran}
\bibliography{../../../../references/phdReferences.bib}

%\appendices

%\newpage
%\section{Examples}
In this appendix we give a list of all the examples we have implemented and discuss implementation details relevant \footnote{The examples are freely available under \url{https://github.com/thalerjonathan/phd/tree/master/coding/libraries/frABS/examples}}. The examples were implemented as use-cases to drive the development of \textit{FrABS} and to give code samples of known models which show how to use this new approach. Note that we do not give an explanation of each model as this would be out of scope of this paper but instead give the major references from which an understanding of the model can be obtained.

We distinguish between the following attributes
\begin{itemize}
	\item Implementation - Which style was used? Either Pure, Monadic or Reactive. Examples could have been implemented in all of them.
	\item Yampa Time-Semantics - Does the implemented model make use of Yampas time-semantics e.g. occasional, after,...? Yes / No.
	\item Update-Strategy - Which update-strategy is required for the given example? It is either Sequential or Parallel or both. In the case of Sequential Agents may be shuffled or not.
	\item Environment - Which kind of environment is used in the given example? Possibilities are 2D/3D Discrete/Continuous or Network. In case of a Parallel Update-Strategy, collapsing may become necessary, depending on the semantics of the model. Also it is noted if the environment has behaviour. Note that an implementation may also have no environment which is noted as None. Although every model implemented in \textit{FrABs} needs to set up some environment, it is not required to use it in the implementation.
	\item Recursive - Is this implementation making use of the recursive features of \textit{FrABS} Yes/No (only available in sequential updating)?
	\item Conversations - Is this implementation making use of the conversations features of \textit{FrABS} Yes/No (only available in sequential updating)?
\end{itemize}

\subsection{Sugarscape}
This is a full implementation of the famous Sugarscape model as described by Epstein \& Axtell in their book \cite{epstein_growing_1996}. The model description itself has no real time-semantics, the agents act in every time-step. Only the environment may change its behaviour after a given number of steps but this is easily expressed without time-semantics as described in the model by Epstein \& Axtell \footnote{Note that this implementation has about 2600 lines of code which - although it includes both a pure and monadic implementation - is significant lower than e.g. the Java-implementation \url{http://sugarscape.sourceforge.net/} with about 6000. Of course it is difficult to compare such measures as we do not include FrABS itself into our measure.}.

\begin{center}
\begin{tabular}{l || l }
\textbf{Implementation}			& Pure, Monadic \\
\textbf{Yampa Time-Semantics}	& No \\
\textbf{Update-Strategy}		& Sequential, shuffling \\
\textbf{Environment}			& 2D Discrete, behaviour \\
\textbf{Recursive}				& No \\
\textbf{Conversations}			& Yes \\
\end{tabular}
\end{center}

\subsection{Agent\_Zero}
This is an implementation of the \textit{Parable 1} from the book of Epstein \cite{epstein_agent_zero:_2014}.

\begin{center}
\begin{tabular}{l || l }
\textbf{Implementation}			& Pure, Monadic \\
\textbf{Yampa Time-Semantics}	& No \\
\textbf{Update-Strategy}		& Parallel, Sequential, shuffling \\
\textbf{Environment}			& 2D Discrete, behaviour, collapsing \\
\textbf{Recursive}				& No \\
\textbf{Conversations}			& No \\
\end{tabular}
\end{center}

\subsection{Schelling Segregation}
This is an implementation of \cite{schelling_dynamic_1971} with extended agent-behaviour which allows to study dynamics of different optimization behaviour: local or global, nearest/random, increasing/binary/future. This is also the only 'real' model in which the recursive features were applied \footnote{The example of Recursive ABS is just a plain how-to example without any real deeper implications.}.

\begin{center}
\begin{tabular}{l || l }
\textbf{Implementation}			& Pure \\
\textbf{Yampa Time-Semantics}	& No \\
\textbf{Update-Strategy}		& Sequential, shuffling \\
\textbf{Environment}			& 2D Discrete \\
\textbf{Recursive}				& Yes (optional) \\
\textbf{Conversations}			& No \\
\end{tabular}
\end{center}

\subsection{Prisoners Dilemma}
This is an implementation of the Prisoners Dilemma on a 2D Grid as discussed in the papers of \cite{nowak_evolutionary_1992}, \cite{huberman_evolutionary_1993} and TODO: cite my own paper on update-strategies.

TODO: implement

\subsection{Heroes \& Cowards}
This is an implementation of the Heroes \& Cowards Game as introduced in \cite{wilensky_introduction_2015} and discussed more in depth in TODO: cite my own paper on update-strategies.

TODO: implement

\subsection{SIRS}
This is an early, non-reactive implementation of a spatial version of the SIRS compartment model found in epidemiology. Note that although the SIRS model itself includes time-semantics, in this implementation no use of Yampas facilities were made. Timed transitions and making contact was implemented directly into the model which results in contacts being made on every iteration, independent of the sampling time. Also in this sample only the infected agents make contact with others, which is not quite correct when wanting to approximate the System Dynamics model (see below). It is primarily included as a comparison to the later implementations (Fr*SIRS) of the same model  which make full use of \textit{FrABS} and to see the huge differences the usage of Yampas time-semantics can make.

\begin{center}
\begin{tabular}{l || l }
\textbf{Implementation}			& Pure, Monadic \\
\textbf{Yampa Time-Semantics}	& No \\
\textbf{Update-Strategy}		& Parallel, Sequential with shuffling \\
\textbf{Environment}			& 2D Discrete \\
\textbf{Recursive}				& No \\
\textbf{Conversations}			& No \\
\end{tabular}
\end{center}

\subsection{Fr(Spatial$|$Network)SIRS}
This is the reactive implementations of both 2D spatial and network (complete graph, Erdos-Renyi and Barbasi-Albert) versions of the SIRS compartment model. Unlike SIRS these examples make full use of the time-semantics provided by Yampa and show the real strength provided by \textit{FrABS}.

\begin{center}
\begin{tabular}{l || l }
\textbf{Implementation}			& Reactive \\
\textbf{Yampa Time-Semantics}	& Yes \\
\textbf{Update-Strategy}		& Parallel \\
\textbf{Environment}			& 2D Discrete, Network \\
\textbf{Recursive}				& No \\
\textbf{Conversations}			& No \\
\end{tabular}
\end{center}

\subsection{System Dynamics SIR}
This is an emulation of the System Dynamics model of the SIR compartment model in epidemiology. It was implemented as a proof-of-concept to show that \textit{FrABS} is able to implement even System Dynamic models because of its continuous-time and time-semantic features. Connections between stocks \& flows are hardcoded, after all System Dynamics completely lacks the concept of spatial- or network-effects. Note that describing the implementation as Reactive may seem not appropriate as in System Dynamics we are not dealing with any events or reactions to it - it is all about a continuous flow between stocks. In this case we wanted to express with Reactive that it is implemented using the Arrowized notion of Yampa which is required when one wants to use Yampas time-semantics anyway.

\begin{center}
\begin{tabular}{l || l }
\textbf{Implementation}			& Reactive \\
\textbf{Yampa Time-Semantics}	& Yes \\
\textbf{Update-Strategy}		& Parallel \\
\textbf{Environment}			& None \\
\textbf{Recursive}				& No \\
\textbf{Conversations}			& No \\
\end{tabular}
\end{center}

\subsection{WildFire}
This is an implementation of a very simple Wildfire model inspired by an example from AnyLogic\texttrademark with the same name.

\begin{center}
\begin{tabular}{l || l }
\textbf{Implementation}			& Reactive \\
\textbf{Yampa Time-Semantics}	& Yes \\
\textbf{Update-Strategy}		& Parallel \\
\textbf{Environment}			& 2D Discrete \\
\textbf{Recursive}				& No \\
\textbf{Conversations}			& No \\
\end{tabular}
\end{center}

\subsection{Double Auction}
This is a basic implementation of a double-auction process of a model described by \cite{breuer_endogenous_2015}. This model is not relying on any environment at the moment but could make use of networks in the future for matching offers.

\begin{center}
\begin{tabular}{l || l }
\textbf{Implementation}			& Pure, Monadic \\
\textbf{Yampa Time-Semantics}	& No \\
\textbf{Update-Strategy}		& Parallel \\
\textbf{Environment}			& None \\
\textbf{Recursive}				& No \\
\textbf{Conversations}			& No \\
\end{tabular}
\end{center}

\subsection{Proof of concepts}
\subsubsection{Recursive ABS} This example shows the very basics of how to implement a recursive ABS using \textit{FrABS}. Note that recursive features only work within the sequential strategy.

\begin{center}
\begin{tabular}{l || l }
\textbf{Implementation}			& Pure \\
\textbf{Yampa Time-Semantics}	& No \\
\textbf{Update-Strategy}		& Sequential \\
\textbf{Environment}			& None \\
\textbf{Recursive}				& Yes \\
\textbf{Conversations}			& No \\
\end{tabular}
\end{center}

\subsubsection{Conversation} This example shows the very basics of how to implement conversations in \textit{FrABS}. Note that conversations only work within the sequential strategy.

\begin{center}
\begin{tabular}{l || l }
\textbf{Implementation}			& Pure \\
\textbf{Yampa Time-Semantics}	& No \\
\textbf{Update-Strategy}		& Sequential \\
\textbf{Environment}			& None \\
\textbf{Recursive}				& No \\
\textbf{Conversations}			& Yes \\
\end{tabular}
\end{center}

%\newpage
%\section{Recursive Agent-Based Simulation}
The idea for this paper arose from my idea of \textit{anticipating agents}, which can project their actions in the future. Because this paper is not as polished as the draft for programming paradigms, we opted not to include it as an appendix and only give its basic ideas and results for the experiments conducted so far. Note that we were not able to find any research regarding recursive ABS \footnote{We found a paper on recursive simulation in general \cite{gilmer_recursive_2000} which focuses on military simulation implemented in C++. Its main findings are that deterministic models seem to benefit significantly from using recursions of the simulation for the decision making process and that when using stochastic models this benefit seems to be lost.}.
In Recursive ABS agents are able to halt time and 'play through' an arbitrary number of actions, compare their outcome and then to resume time and continue with a specifically chosen action e.g. the best performing or the one in which they haven't died. More precisely, what we want is to give an agent the ability to run the simulation recursively a number of times where the this number is not determined initially but can depend on the outcome of the recursive simulation. So Recursive ABS gives each Agent the ability to run the simulation locally from its point of view to anticipate its actions in the future and change them in the present.
We investigate the famous Schelling Segregation \cite{schelling_dynamic_1971} and endow our agents with the ability to project their actions into the future by recursively running simulations. Based on the outcome of the recursions they are then able to determine whether their move increases their utility in the future or not. The main finding for now is that it does not increase the convergence speed to equilibrium but can lead to extreme volatility of dynamics although the system seems to be near to complete equilibrium. In the case of a 10x10 field it was observed that although the system was nearly in its steady state - all but one agent were satisfied - the move of a single agent caused the system to become completely unstable and depart from its near-equilibrium state to a highly volatile and unstable state.

This approach of course rises a few questions and issues. The main problem of our approach is that, depending on ones view-point, it is violating the principles of locality of information and limit of computing power. To recursively run the simulation the agent which initiates the recursion is feeding in all the states of the other agents and calculates the outcome of potentially multiple of its own steps, each potentially multiple recursion-layers deep and each recursion-layer multiple time-steps long. Both requires that each agent has perfect information about the complete simulation \textit{and} can compute these 3-dimensional recursions, which scale exponentially. In the social sciences where agents are often designed to have only very local information and perform low-cost computations it is very difficult or impossible to motivate the usage of recursive simulations - it simply does not match the assumptions of the real world, the social sciences want to model. In general simulations, where it is much more commonly accepted to assume perfect information and potentially infinite amount of computing power this approach is easily motivated by a constructive argument: it is possible to build, thus we build it.
Another fundamental question regards the meaning and epistemology behind an entity running simulations. Of course, this strongly depends on the context: in ACE it may be understood as a search for optimizing behaviour, in Social Simulation it may be interpreted as a kind of free will: the agent who is initiating the recursion can be seen as 'knowing' that it is running inside a simulation, thus in this context free will is seen as being able to anticipate ones actions and change them.
When talking about recursion it is always the question of the depth of the recursion and because as we are running on computers we need to terminate at some point. Accelerating Turing machines (also known as Zeno Machine) are theoretically able to calculate an infinite regress but this raises again epistemological questions and can be seen as having religious character as discussed e.g. in Tiplers Omega Point, Bostroms simulation argument \cite{bostrom_are_2003} and its theological implications \cite{steinhart_theological_2010}. So the ultimate question this research leaves is what the outcome would be when running a recursive ABS on a Zeno Machine/Accelerated Turing Machine? \footnote{Anyway this would mean we have infinite amount of computing power - I am sure that in this case we don't worry the slightest about recursive ABS any more.}

At the moment this idea lies dormant as the intention was just to develop it far enough to give a proof-of-concept and see some results. Having achieved this we arrived at the conclusion, that the results are not really ground-breaking. This stems from the fact that Schelling segregation is not the best model to demonstrate this technique and that we are thus lacking the right model in which recursive ABS is the real killer-feature. Also to pursue this direction further and treat it in-depth, would require much more time and give the PhD a complete different spin. Still it is useful in supporting our move towards pure functional ABS as we are convinced that recursion is comparably easy to implement because the language is built on it and due to the lack of side-effects \footnote{Actually implementing it was \textit{really hard} but we wouldn't dare to implement this into an object-oriented language or into an object-oriented ABS framework.}.

\end{document}