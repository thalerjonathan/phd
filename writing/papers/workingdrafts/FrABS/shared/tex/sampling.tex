\section{Sampling the System}
discuss the importance of sampling the system. When sampling the system, the correct time-delta must be selected which depends on the highest frequency which occurs in a time-reactive function in the whole system. For example in the FrSIR model we want infected agents to make on average contact with 5 other agents per time-unit, which means with a frequency of $frac{1}{5}$. This functionality is built on Yampas function occasionally which behaviour we investigated under differing time-deltas with the above frequency. In this investigation we simply sampled occasionally with different time-deltas for a duration of 1000 time-units and the event-frequency of $frac{1}{5}$. The results can be seen in Figure \ref{fig:sampling_occasionally_5evts} and are quite striking. The plot clearly shows that occasionally needs quite high sampling frequency even for comparatively low event-frequency - this becomes of course worse for higher event-frequencies.

\begin{figure}
	\centering
	\includegraphics[width=.6\textwidth, angle=0]{./../shared/fig/samplingTest_occasionally_5evts.png}
	\caption{Sampling the \textit{occasional} function to visualize the influence of sampling frequencies on the event occurrence. Event-frequency of $\frac{1}{5}$ (average of 5 events per time-unit) with time-deltas of [ 5, 2, 1, $\frac{1}{2}$, $\frac{1}{5}$, $\frac{1}{10}$, $\frac{1}{20}$, $\frac{1}{50}$, $\frac{1}{100}$ ] running for 1000 time-units with 100 replications. The theoretical maximum is 5000.}
	\label{fig:sampling_occasionally_5evts}
\end{figure}

The other time-reactive function which occurs in the FrSIR model is the timed transition from infected to recovered which occurs on average with an exponential random-distribution after 15 time-units. This functionality is built on a custom implementation of Yampas after which creates an event after a time-out of the passed in time-duration drawn from an exponential random-distribution. Clearly this function has different semantics as although it also continuously emit events over time - NoEvent before the time was hit, and Event x after the time hit - the relevant point is that it switches to Event at some discrete point in time. This is implemented as simply adding up the time-deltas until the accumulator is GE than the previously drawn exponential time-out. We also investigated the behaviour of this function under varying time-deltas using a time-out of 15 (drawn from an exponential distribution within the function). Our approach was to sample the afterExp until an event occurs (this is one of the occasions where lazy evaluation really shines as one simply repeats the time-delta stream forever but then searches for the first occurrence of an event, which MUST occur at some point due to mathematical exponential distribution and our parameters to it, so it will always terminate) and then see when it has occurred. We run this for 1000 times with different random-number seeds and average the resulting times. The results can be seen in Figure TODO. The result is striking in another way: this function seems to be pretty invariant to the time-deltas, for obvious reasons: we are basically just interested in the "after"-condition of the whole semantics whereas in occasionally we are interested in the "repeatedly"-conditions. If we under the afterExp then we can be off by one time-delta. If we under sample occasionally we keep loosing events, the close time-delta and event-frequency are, the more we lose. Of course afterExp can also be used for very short time-outs e.g. $frac{1}{5}$. We have investigated the behaviour of this function for various time-deltas as well as seen in Figure TODO. Here the result is much more striking and shows that afterExp is vulnerable to small time-outs as well as occasionally. 
To show that occasionally is not vulnerable to very low frequencies of e.g. one event every 5 time-steps we plotted the behaviour of this under varying time-steps in Figure \ref{fig:sampling_occasionally_02evts}. The result shows that for low frequencies occasionally works fine with larger time-deltas

TODO: produce a plot of my afterExp tests with time-out of 15 with time-deltas of 1.0, 0.5, 0.2, 0.1, 0.05, 0.01, 0.001.

TODO: produce a plot of my afterExp tests with time-out of $frac{1}{5}$ with time-deltas of 1.0, 0.5, 0.2, 0.1, 0.05, 0.01, 0.001.

\begin{figure}
	\centering
	\includegraphics[width=.6\textwidth, angle=0]{./../shared/fig/samplingTest_occasionally_02evts.png}
	\caption{Sampling the \textit{occasional} function to visualize the influence of sampling frequencies on the event occurrence. Event-frequency of 5 (average of 0.2 events per time-unit) with time-deltas of [ 5, 2, 1, $\frac{1}{2}$, $\frac{1}{5}$, $\frac{1}{10}$, $\frac{1}{20}$, $\frac{1}{50}$, $\frac{1}{100}$ ] running for 1000 time-units with 100 replications. The theoretical maximum is 200.}
	\label{fig:sampling_occasionally_02evts}
\end{figure}

\subsection{Frequency of an ABS}
TODO: can we derive a formula to calculate the optimal time-delta for a given agent-based model?

\subsection{Performance}
of course performance is a big issue and it decreases as time-deltas get smaller and smaller. if we could perform subsampling just for the given high-frequency function with the remaining system running in lower frequency then we could achieve substantial performance-increase.

formula of calculating-steps = steps per time-unit * time-to-run-the-simulation