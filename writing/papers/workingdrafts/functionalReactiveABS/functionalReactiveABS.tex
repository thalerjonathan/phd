%\documentclass[a4paper, 10pt, conference]{../../templates/IEEEconf/IEEEconf}
\documentclass[10pt, conference]{../../templates/IEEEtran/IEEEtran}
%\documentclass[10pt, journal]{../../templates/IEEEtran/IEEEtran}

\usepackage{graphicx}
\usepackage{caption} 
\usepackage{subcaption}
\usepackage{hyperref}
\usepackage{listings}
\usepackage{hhline}
\usepackage{float}
\usepackage{amssymb}
\usepackage[autostyle=true]{csquotes}
\usepackage{amsmath}

\font\subtitlefont=cmr12 at 18pt

\title{Functional Reactive Agent-Based Simulation\\{\subtitlefont Towards pure functional Agent-Based Simulation}}

% IEEEtran journal authors
%\author{Jonathan Thaler, ̃Peer-Olaf Siebers \\ School of Computer Science \\ University of Nottingham%
%\thanks{jonathan.thaler@nottingham.ac.uk}%
%\thanks{peer-olaf.siebers@nottingham.ac.uk}
%}

% IEEEtran conference authors
\author{
	\IEEEauthorblockN{Jonathan Thaler}
	\IEEEauthorblockA{School of Computer Science\\
		University of Nottingham\\
		jonathan.thaler@nottingham.ac.uk}
		
	\and
		
	\IEEEauthorblockN{Peer-Olaf Siebers}
	\IEEEauthorblockA{School of Computer Science\\
		University of Nottingham\\
		peer-olaf.siebers@nottingham.ac.uk}
		
	\and
		
	\IEEEauthorblockN{Thorsten Altenkirch}
	\IEEEauthorblockA{School of Computer Science\\
		University of Nottingham\\
		thorsten.altenkirch@nottingham.ac.uk}
}

%\IEEEpubid{0000--0000/00\$00.00 ̃\copyright ̃2015 IEEE}

% IEEEconf authors
%\author{
%	Jonathan Thaler \\
%	\email{jonathan.thaler@nottingham.ac.uk} \\
%	\begin{affiliation}
%		School of Computer Science, University of Nottingham
%	\end{affiliation} \\
%	\and 
%	Peer-Olaf Siebers \\
%	\email{peer-olaf.siebers@nottingham.ac.uk} \\
%	\begin{affiliation}
%		School of Computer Science, University of Nottingham
%	\end{affiliation} 
%	\and 
%	Thorsten Altenkirch \\
%	\email{thorsten.altenkirch@nottingham.ac.uk} \\
%	\begin{affiliation}
%		School of Computer Science, University of Nottingham
%	\end{affiliation} 
%}

\begin{document}
\maketitle 

\begin{abstract}


[ ] time in FrABS: when 0dt then still actions can occur when not relying on time semantics
[ ] what about time-travel in abms for introspection during running it? this is much easier in FrABS

TODO: also solve the SIR with an algebra system to have a bullet-proof "proof" that we reproduce the same dynamics. this is only partially a proof our system is correct, but it is not a formal proof, this needs to be done different

TODO: don't sell this paper as an opposing view against OOP (e.g. OOP is bad) but as a positive view: "for the first time it is possible to do ABMS in pure functional programming".

TODO: publishing: 1st version write for a journal in the ABMS community, 2nd version write for a conference in the functional programming community e.g. for the TFP in Kent 2018

TODO: It should be possible to formally show that spatial SIR and WildFire are the same model. NOTE: they are NOT the same, the fundamental difference is that in the WildFire model only the burning cells initiate the ignition - if we compare this to the SIR, the burning cells would be infected agents and although in the spatial SIR model the infected agents make contact with other agents, so do the susceptible ones which does NOT occur in wildfire

TODO: cite my own work on update-strategies

TODO: can we formally show that the SIR approximates the SD model?

TODO: cite papers which discuss how to approximate a SD model by ABS
- Macal (2010) - To Agent-Based Simulation From System Dynamics 
	-> i am very unhappy with this paper: first it does not give concrete parameters for the SD model so it is impossible to replicate. Also i think it has a systematical error as the infected agents make no contact but this is required as evident from the SD-models infection-rate which also incorporates. TODO: write an email to this guy: why are the infectious not contacting the other agents? this seems to be a systematical error
- Borshchev, Filippov (2004) - From System Dynamics and Discrete Event to Practical Agent Based Modeling: Reasons, Techniques, Tools
	-> its VERY IMPORTANT point is that we need to draw the illness-duration from an exponential-distribution because the illness-duration is proportional to the size of the infected. note: this is wrongly expressed, need to find the correct formulation

		-> my emulation of SD using ABS is really an implementation of the SD model and follows it - they are equivalent
		-> my ABS implementation is the same as / equivalent to the SD emulation
			=> thus if i can show that my SD emulation is equlas to the SD model
			=> AND that the ABS implementation is the same as the SD emulation
			=> THEN the ABS implementation is an SD implementation, and we have shown this in code for the first time in ABS
			
TODO: the first (of two) contribution of this paper is: an explanation of one way of how ABS can be done in pure functional programming and its benefits: declarative style where the code looks very much like specification, fewer LoC, fewer Bugs, reasoning and proves

TODO: main second contribution is: show that the SD and the ABS implementation of the SIR model are the same by proving that ABS solves the SD equation. this should be possible by now using reasoning techniques (and quickcheck?)

So far, the pure functional paradigm hasn't got much attention in Agent-Based Simulation (ABS) where the dominant programming paradigm is object-orientation, with Java being its most prominent representative. We claim that pure functional programming using Haskell is very well suited to implement complex, real-world ABS. To prove that we implemented the library \textit{FrABMS} which allows to do Agent-Based Modelling \& Simulation the first time in the pure functional programming language Haskell. To achieve this we leverage the basic concepts of ABS with functional reactive programming using the library Yampa. The result is a surprisingly fresh approach to Agent-Based Modelling \& Simulation as it allows to incorporate discrete time-semantics similar to Discrete Event Simulation and continuous time-flows like System Dynamics. In this paper we will show the novel approach of FrABMS through the example of the SIR model and discuss implications and best practices.
\end{abstract}

\begin{IEEEkeywords}
Agent-Based Simulation, Functional Programming, Haskell
\end{IEEEkeywords}

%*******************************************************************************
%*********************************** First Chapter *****************************
%*******************************************************************************

\chapter{Introduction}  %Title of the First Chapter
I noticed that it is pretty hard to convince an agent-based economics specialist who is not a computer scientist about a pure functional approach. My conjecture is that the implementation technique and method does not matter much to them because they have very little knowledge about programming and are almost always self-taught - they don't know about software-engineering, nothing about proper software-design and architecture, nothing about software-maintenance, nothing about unit-testing,... In the end they just "hack" the simulation in whatever language they are able to: C++, Visual Basic, Java or toolboxes like Netlogo. For them it is all about to \textit{get things done somehow} and not to get things done the right way or in a beautiful way - the way and the method doesn't matter, its just a necessary evil which needs to be done. Thus if functional programming could make their lives easier, then they will definitely welcome it. But functional programming is, i think, harder to learn and harder to understand - so one needs to provide an abstraction through EDSL. So I REALLY need to come up with convincing arguments why to use pure functional approaches in ACE THEY can understand, otherwise I will be lost and not heard (not published,...). \\

What ACE economists care for:

\begin{itemize}
\item Very: Qualitative modelling with quantitative results
\item Yes: Easy reproducibility
\item Likely: Reasoning about convergence?
\item Likely: EDSL
\end{itemize}

My contributions are: pure functional framework, functional agent-model for market-simulations, EDSL for market-simulations, qualitative / implicit modelling with quanitative results, reasoning in my framework about convergence \\

IDEA: could I develop non-causal modelling (models are expressed in terms of non-directed equations, modelled in signal-relations) to allow for qualitative modelling for the agent-based economists? See hybrid modelling paper of Yampa. \textbf{THIS WOULD BE A HUGE NOVEL CONTRIBUTION TO ACE ESPECIALLY WHEN COMBINED WITH AN EDSL AND PROVIDING FULL REFERENTIAL TRANSPARENCY TO KEEP THE ABILITY TO REASON ABOUT CONVERGENCE}. This should be covered in the "EDSL"-paper.

TODO: maybe i should really focus only on market models? otherwise too much? \\

central novelty of my PhD: model specification = runnable code. possible through EDSL. but only in specific subfield of ACE: market-models. need a functional description of the model, then translate it to model specification in EDSL and then run it to see dynamics. But: model specification moves closer to functional programming languages. \\

another novelty approach: model specification through qualitative instead of quantiative approaches. is this possible? \\

WHY FUNCTIONAL? "because its the ultimate approach to scientific computing": fewer bugs due to mutable state (why? is thos shown obkectively by someone?), shorter (again as above, productivity), more expressive and closer to math, EDSL, EDSL=model=simulation, better parallelising due to referental transparency, reasoning \\

scientific results need to be reproduced, especially when they have high impact. a more formal approach of specifying the model and the simulation (model=simulation) could lead to easier sharing and easier reporduction without ambigouites \\

pure functional agent-model \& theory, EDSL framework in Haskell for ACE

\begin{enumerate}
\item Which kind of problem do we have?
\item What aim is there? Solving the problem? 
\item How the aim is achieved by enumerating VERY CLEAR objectives.
\item What the impact one expects (hypothesis) and what it is (after results).
\end{enumerate}

Note: It is not in the interest of the researcher to develop new economic theories but to research the use of functional methods (programming and specification) in agent-based computational economics (ACE).

NOTE: Get the reader’s attention early in the introduction: motivation, significance, originality and novelty.

\section{Methods}
Methods need to be selected to implement the simulations. Special emphasis will be put on functional ones which will then be compared to established methods in the field of ABM/S and ACE. \\

Claim: non-programming environments are considered to be not powerful enough to capture the complexity of ACE implementations thus a programming approach to ACE will be always required.

\section{Scenarios}
To apply and test functional methods in ACE, four scenarios of ACE are selected and then the methods applied and compared with each other to see how each of them perform in comparison. The 4 selected scenarios represent a selection of the challenges posed in ACE: from very abstract ones to very operational ones.

\section{Comparison}
Each of the selected scenarios is then implemented using the selected methods where each solution is then compared against the following criteria: 

\begin{enumerate}
\item suitability for scientific computation
\item robustness
\item error-sources
\item testability
\item stability
\item extendability
\item size of code
\item maintainability
\item time taken for development
\item verification \& correctness
\item replications \& parallelism
\item EDSL
\end{enumerate}

This will then allow to compare the different methods against each other and to show under which circumstances functional methods shine and when they should not be used.

\section{Agent-Based Modelling and Simulation (ABM/S)}
ABM/S is a method of modelling and simulating a system where the global behaviour may be unknown but the behaviour and interactions of the parts making up the system is of knowledge (Wooldrige, M. (2009). An Introduction to MultiAgent Systems. John Wiley & Sons). Those parts, called agents, are modelled and simulated out of which then the aggregate global behaviour of the whole system emerges. Thus the central aspect of ABM/S is the concept of an Agent which can be understood as a metaphor for a pro-active unit, able to spawn new Agents, and interacting with other Agents in a network of neighbours by exchange of messages. The implementation of Agents can vary and strongly depends on the programming language and the kind of domain the simulation and model is situated in.

\section{Agent-Based Economics (ACE)}
According to Leigh Tesfatsion (Tesfatsion, L. (2006). Agent-based computational economics: A constructive approach to economic theory. In Tesfatsion, L. and Judd, K. L., editors, Handbook of Computational Economics, volume 2, chapter 16, pages 831–880. Elsevier, 1 edition.), one of the leading figures, ACE is "[...] computational modelling of economic processes (including whole economies) as open-ended dynamic systems of interacting agents." - thus lending perfectly to the use of ABM/S as already the name suggests. Whereas classical economic models fall short by only looking at the average, pure rational, individual interacting in anonymous markets, the ACE approach looks at heterogeneous, non-rational individuals interacting with each other in networks (Kirman, A. (2010). Complex Economics: Individual and Collective Rationality. Routledge, London ; New York, NY.). Thus ACE can be understood as a combination of computer-science, cognitive/social science and evolutionary economics.

\section{Functional programming}
TODO: read \cite{Backus1978}

The state-of-the-art approach to implementing Agents are object-oriented methods and programming as the metaphor of an Agent as presented above lends itself very naturally to object-orientation (OO). The author of this thesis claims that OO in the hands of inexperienced or ignorant programmers is dangerous, leading to bugs and hardly maintainable and extensible code. The reason for this is that OO provides very powerful techniques of organising and structuring programs through Classes, Type Hierarchies and Objects, which, when misused, lead to the above mentioned problems. Also major problems, which experts face as well as beginners are 1. state is highly scattered across the program which disguises the flow of data in complex simulations and 2. objects don’t compose as well as functions. The reason for this is that objects always carry around some internal state which makes it obviously much more complicated as complex dependencies can be introduced according to the internal state.
All this is tackled by (pure) functional programming which abandons the concept of global state, Objects and Classes and makes data-flow explicit. This then allows to reason about correctness, termination and other properties of the program e.g. if a given function exhibits side-effects or not. Other benefits are fewer lines of code, easier maintainability and ultimately fewer bugs thus making functional programming the ideal choice for scientific computing and simulation and thus also for ACE. A very powerful feature of functional programming is Lazy evaluation. It allows to describe infinite data-structures and functions producing an infinite stream of output but which are only computed as currently needed. Thus the decision of how many is decoupled from how to (Hughes, J. (1989). Why functional programming matters. Comput. J., 32(2):98–107.).
The most powerful aspect using pure functional programming however is that it allows the design of embedded domain specific languages (EDSL). In this case one develops and programs primitives e.g. types and functions in a host language (embed) in a way that they can be combined. The combination of these primitives then looks like a language specific to a given domain, in the case of this thesis ACE. The ease of development of EDSLs in pure functional programming is also a proof of the superior extensibility and composability of pure functional languages over OO (Henderson P. (1982). Functional Geometry. Proceedings of the 1982 ACM Symposium on LISP and Functional Programming.).
One of the most compelling example to utilize pure functional programming is the reporting of Hudak (Hudak P., Jones M. (1994). Haskell vs. Ada vs. C++ vs. Awk vs. ... An Experiment in Software Prototyping Productivity. Department of Computer Science, Yale University.)  where in a prototyping contest of DARPA the Haskell prototype was by far the shortest with 85 lines of code. Also the Jury mistook the code as specification because the prototype did actually implement a small EDSL which is a perfect proof how close EDSL can get to and look like a specification.

Functional languages can best be characterized by their way computation works: instead of \textit{how} something is computed, \textit{what} is computed is described. Thus functional programming follows a declarative instead of an imperative style of programming. The key points are:
\begin{itemize}
\item No assignment statements - variables values can never change once given a value.
\item Function calls have no side-effect and will only compute the results - this makes order of execution irrelevant, as due to the lack of side-effects the logical point in \textit{time} when the function is calculated within the program-execution does not matter.
\item higher-order functions
\item lazy evaluation
\item Looping is achieved using recursion, mostly through the use of the general fold or the more specific map.
\item Pattern-matching
\end{itemize}

This alone does not really explain the \textit{real} advantages of functional programming and one must look for better motivations using functional programming languages. One motivation is given in \cite{Hughes1989} which is a great paper explaining to non-functional programmers what the significance of functional programming is and helping functional programmers putting functional languages to maximum use by showing the real power and advantages of functional languages. The main conclusion is that \textit{modularity}, which is the key to successful programming, can be achieved best using higher-order functions and lazy evaluation provided in functional languages like Haskell. \cite{Hughes1989} argues that the ability to divide problems into sub-problems depends on the ability to glue the sub-problems together which depends strongly on the programming-language and \cite{Hughes1989} argues that in this ability functional languages are superior to structured programming.

TODO: comparison of functional and object-oriented programming. My points are:
\begin{itemize}
\item The way state can be changed and treated - distributed over multiple objects - is often very difficult to understand.
\item Inheritance is a dangerous thing if not used with care because inheritance introduces very strong dependencies which cannot be changed during runtime anymore.
\item Objects don't compose very well: \url{http://zeroturnaround.com/rebellabs/why-the-debate-on-object-oriented-vs-functional-programming-is-all-about-composition/}
\item (Nearly) impossible to reason about programs
\end{itemize}

In conclusion the upsides of functional programming as opposed to OO are:
\begin{itemize}
\item Much more explicit flow of data \& control
\item Much better compose-able
\item Much better parallelism
\end{itemize}

\section{Related Research}
Tim Sweeney, CTO of Epic Games gave an invited talk about how "future programming languages could help us write better code" by "supplying stronger typing, reduce run-time failures;  and the need for pervasive concurrency support, both implicit and explicit, to effectively exploit the several forms of parallelism present in games and graphics." \cite{Sweeney2006}. Although the fields of games and agent-based simulations seem to be very different in the end, they have also very important similarities: both are simulations which perform numerical computations and update objects - in games they are called "game-objects" and in abm they are called agents but they are in fact the same thing - in a loop either concurrently or sequential. His key-points were:

\begin{itemize}
\item Dependent types as the remedy of most of the run-time failures.
\item Parallelism for numerical computation: these are pure functional algorithms, operate locally on mutable state. Haskell ST, STRef solution enables encapsulating local heaps and mutability within referentially transparent code.
\item Updating game-objects (agents) concurrently using STM: update all objects concurrently in arbitrary order, with each update wrapped in atomic block - depends on collisions if performance goes up.
\end{itemize}

\section{Background}

\subsection{Schelling Segregation}
We follow in our implementation the original paper of Schelling as in \cite{schelling_dynamic_1971} where we focus on the \textit{Area Distribution} section (Schelling starts with movement in a linear, 1-dimensional world where agents are able to move to the nearest point which meets the agents satisfaction but this is not what we follow here). One assumes a discrete 2-dimensional lattice-world with NxM fields. Each field is either occupied by an agent of a given color (e.g. Red or Green) or is free. Each field has 8 neighbours, which denotes a Moore-Neighbourhood. In Schellings original work the lattice-world is limited at its borders but we assume a torus world which is wrapped around in both the x- and y-dimensions resulting in 8 neighbours also for fields at the border. The occupation density was set by Schelling to be about 70\%-75\% which he identifies as being a setting which allows the agents to move around freely without making the lattice-world too sparse.
Now the agents make their move sequentially one after another. In each move an agent calculates the number of neighbours which are of equal color. If the number satisfies the agents needs about the neighbourhood then the agent is regarded as being 'happy' and will stay on this field. On the other hand the agent moves to the nearest unoccupied field which satisfies its needs. An agent which moves selects an unoccupied place randomly relative from its current place within a rectangle of side-length 2r where its current place is at the center. The interpretation for that behaviour is that agents won't move too far as it could be costly. Also children might attend a school in this area or the family has friends in this area, so they don't want to break that.



Agents just move depending on their movement-strategy to another place if they are not happy on the current one - they don't care how the target place is in the present or in the future, they will decide again in the next time-step. The interpretation for that behaviour is: agents want to 'just get out' at any cost, not caring what the future place will look like - it might be better or worse but they will see then.

\subsubsection{Optimizing behaviour}
TODO: define utility

The original schelling model didn't have a move-optimizing behaviour, meaning agents are just binary: if it is happy it will not move, if it is unhappy it will move but they won't care where they move. We introduce local move-optimizing behaviours which can be interpreted as being realistic in the real-world. It is important to note that we focus on \textit{local} instead of \textit{global} move-optimization: the agents are limited in their reasoning-capabilities and have limited information available: they cannot check out \textit{every} place and pick the globally best one.\\

\subsubsection{Anticipating behaviour}
Schelling explicitly mentions in \cite{schelling_dynamic_1971} that nobody anticipates moves of others. This is what we introduce using the recursive simulation.

TODO: is this optimizing behaviour in the spirit of schellings original work? 

\paragraph{Optimizing future} Agents pick an unoccupied random place and move to it if it increases their utility in the future. The interpretation for that behaviour is: agents heard about a place which will be cool in the future.

\paragraph{Optimizing present \& future} Agents pick an unoccupied random place and move to it if it increases their utility in the now and in the future. The interpretation for that behaviour is: agents heard about a cool spot in town, check it out and move to it if they like it but they also anticipate the coolness of the place in the future and if it seems that the place is going down then they won't move there.

\subsection{Related Research}
TODO: \cite{kirman_complex_2010} mention kirman complex economics where he investigates the model more in depth


\section{The formal Agent-Model}

TODO: define an agent 

\section{Functional Reactive ABS}
the fundamental problem is that unlike in oo e.g. java there are no objects and no implicit aliases through which to acess and change data: method calls are not there in FP. we must silve the problem of how to represent an agent and how agents can interact with each other

using example SIR
TODO: show how simple state-transition agents work using switch
TODO: show how the time-semantics can be used

\begin{enumerate}
	\item Representing an agent and environment - there are no classes and objects in Haskell.
	\item Interactions among agents and actions of agents on the environment - there are no method-calls and aliases in Haskell.
	\item Implement the necessary update-strategies as discussed in our paper \ref{app:updateStrategies}, where we only focus on sequential- and parallel-strategies - there is no mutable data which can be changed implicitly through side-effects (e.g. the agents, the list of all the agents, the environment).
\end{enumerate}

\subsection{Messaging}
As discussed in the literature reflection in Chapter \ref{chap:refl}, inspired by the actor model we will resort to synchronized, reliable message passing with share nothing semantics to implement agent-agent interactions. Each Agent can send a message to an other agent through AgentOut-Signal where the messages are queued in the AgentIn-Signal and can be processed when the agent is updated the next time. The agent is free to ignore the messages and if it does not process them they will be simply lost.
Note that due to the fact we don't have method-calls in FP, messaging will always take some time, which depends on the sampling interval of the system. This was not obviously clear when implementing ABS in an object-oriented way because there we can communicate through method calls which are a way of interaction which takes no simulation-time.

%TODO: Push vs. Pull. we need push because we need to 'sample' the system at regular time-intervals because agent-behaviour can depend on time as well (pro-active) and not only on messaging. if we had only the latter, a pull approach would suffice.

% my wrongthinking: messaging ALWAYS takes time e.g. send/response roundtrip. conversations dont take time but are restricted for the receiver e.g. the receiver cannot send messages to others or change the environment in a conversation

% because an agent cannot reply within the same timestep sampling interval becomes an issue: if we need a reply within a given time then the sampling interval needs to be at least twice as much
% difference between discrete and continuous: the successor of discrete is defined whereas in the case of continuous it is not? how is the successor defined in the case of continuous time?

\subsection{Conversations}
The messaging as implemented above works well for one-directional, virtual asynchronous interaction where we don't need a reply at the same time. A perfect use-case for messaging is making contact with neighbours in the SIRS-model: the agent sends the contact message but does not need any response from the receiver, the receiver handles the message and may get infected but does not need to communicate this back to the sender. 
A different case is when agents need to transact in the time-step one or multiple times: agent A interacts with agent B where the semantics of the model (and thus messaging) need an immediate response from agent B - which can lead to further interactions initiated by agent A. The Sugarscape model has three use-cases for this: sex, warfare and trading amongst agents all need an immediate response (e.g. wanna mate with me?, I just killed you, wanna trade for this price?). The reason is that we need to transact now as all of the actions only work on a 1:1 relationship and could violate ressource-constraints.
For this we introduce the concept of a conversation between agents. This allows an agent A to initiate a conversation with another agent B in which the simulation is virtually halted and both can exchange an arbitrary number of messages through calling and responding without time passing (something not possible without this concept because in each iteration the time advances). After either one agent has finished with the conversation it will terminate it and the simulation will continue with the updated agents (note the importance here: \textit{both} agents can change their state in a conversation). The conversation-concept is implemented at the moment in the way that the initiating agent A has all the freedom in sending messages, starting a new conversation,... but that the receiving agent B is only able to change its state but is not allowed to send messages or start conversations in this process. Technically speaking: agent A can manipulate an AgentOut whereas agent B can only manipulate its next AgentIn.
When looking at conversations they may look like an emulation of method-calls but they are more powerful: a receiver can be unavailable to conversations or simply refuse to handle this conversation. This follows the concept of an active actor which can decide what happens with the incoming interaction-request, instead of the passive object which cannot decide whether the method-call is really executed or not.

\subsection{Iteration-Strategies}
Building on the foundations laid out in my paper about iteration-strategies in Appendix \ref{app:updateStrategies}, we implement two of the four strategies: sequential- and parallel-strategy. We deliberately ignore the concurrent- and actor-strategy for now and leave this for further research \footnote{Also both strategies would require running in the STM-Monad, which is not possible with Yampa. The work of Ivan Perez in \cite{perez_functional_2016} implemented a library called Dunai, which is the same as Yampa but capable of running in an arbitrary Monad.}.
Implementing iteration-strategies using Haskell and FRP is not as straight-forward as in e.g. Java because one does not have mutable data which can be updated in-place. Although my work on programming paradigms in Appendix \ref{app:paradigms} did not take FRP into account, general concepts apply equally as well.

\subsubsection{Sequential}
In this strategy the agents are updated one after another where the changes (messages sent, environment changed,...) of one agent are visible to agents updated after. Basically this strategy is implemented as a variant of fold which allows to feed output of one agent (e.g. messages and the environment) forward to the other agents while iterating over the list of agents. For each agent the agent-behaviour signal-function is called with the current AgentIn as input to retrieve the according AgentOut. The messages of the AgentOut are then distributed to the receivers AgentIn.
The environment of the agent, which is passed in through AgentIn and returned through AgentOut will then be passed forward to all agents i + 1 AgentIn in the current iteration and override their old environment. Thus all steps of changes made to the environment are visible in the AgentOuts. The last environment is then the final environment in the current iteration and will be returned by the callback function together with the current AgentOuts.

\subsubsection{Parallel}
The parallel strategy is \textit{much} easier to implement than the sequential but is of course not applicable to all models because of it different semantics. Basically this strategy is implemented as a map over all agents which calls each agent-behaviour signal-function with the agents AgentIn to retrieve the new AgentOut. Then the messages are distributed amongst all agents.
A problem in this strategy is that the environment is duplicated to each agent and then each agent can work on it and return a changed environment. Thus after one iteration there are n versions of environments where n is equals to the number of agents. These environments must then be collapsed into a final one which is always domain-specific thus needs to be done through a function provided in the environment itself.

%TODO: functionsl approach to ABS: parallel application to previous states where only one agent is acting and the others are fixed. per step we have n results. for a full iteration we need $(n-1)^2$ applicatioms

\subsection{Environment}
TODO: again cite my own work where I discussed the problem of environments

Each agent has a copy of the environment passed in through the AgentIn and can change it by passing a changed version of the environment out through AgentOut. 
In the sequential update-strategy the environment of the agent i will then be passed to all agents i + 1 AgentIn in the current iteration and override their old environment. Thus all steps of changes made to the environment are visible in the AgentOuts. The last environment is then the final environment int he current iteration and will be returned by the callback function together with the current AgentOuts.
In the parallel update-strategy the environment is duplicated to each agent and then each agent can work on it and return the changed environment. Thus after one iteration there are n versions of environments where n is equals to the number of agents. These environments must then be collapsed into a final one which is always domain-specific thus needs to be done through a function provided in the environment itself.
In both the sequential and parallel update-strategy after one iteration there is one single environment left. An environment can have an optional behaviour which allows the environment to update its cells. This is a regular SF thus having also the time of the simulation available. Note that the environment cannot send messages to agents because it is not an agent itself. An example of an environment behaviour would be to regrow some good on each cell according to some rate per time-unit (inspired by SugarScape regrowing of Sugar).

\subsection{Time-Semantics}
The main reason for building our pure functional ABMS approach on top of Yampa was to leverage the powerful time-semantics of Yampa which allows us to implement important concepts of ABMS:

state-chart: agents are at all time of their life-cycle in one state and can switch between multiple states using transitions 
timed transitions: transition to another state/behaviour happens at a discrete time
rate transitions: transition happens with a given rate
message transition: transition upon receiving a given message 

\subsection{Agents as Signals}
Due to the underlying nature and motivation of Functional Reactive Programming (und im speziellen) Yampa, Agents can be seen as Signals which is generated and consumed by a Signal-Function which is the behaviour of an Agent. If an Agent does not change the OUTPUT-signal is constant, if the agent changes e.g. by sending a message, changing its state,... the OUTPUT signal changes. A dead agent has no signal at all.

\subsection{Time-Sampling}
sampling rate depends on the transition times \& rates of the model. when e.g. the contact rate is 5 then the sampling dt should be below 0.2

\subsection{System Dynamics}
can emulate system dynamics due to the parallel update-strategy and continuous time-flow semantics

\subsection{Discrete Event Simulation}
DES in FrABMS? how easily can we implement server/queue systems? do they also look like a specification? potential problem: ordering of messages is not guaranteed by now

\section{Examples}
In this appendix we give a list of all the examples we have implemented and discuss implementation details relevant \footnote{The examples are freely available under \url{https://github.com/thalerjonathan/phd/tree/master/coding/libraries/frABS/examples}}. The examples were implemented as use-cases to drive the development of \textit{FrABS} and to give code samples of known models which show how to use this new approach. Note that we do not give an explanation of each model as this would be out of scope of this paper but instead give the major references from which an understanding of the model can be obtained.

We distinguish between the following attributes
\begin{itemize}
	\item Implementation - Which style was used? Either Pure, Monadic or Reactive. Examples could have been implemented in all of them.
	\item Yampa Time-Semantics - Does the implemented model make use of Yampas time-semantics e.g. occasional, after,...? Yes / No.
	\item Update-Strategy - Which update-strategy is required for the given example? It is either Sequential or Parallel or both. In the case of Sequential Agents may be shuffled or not.
	\item Environment - Which kind of environment is used in the given example? Possibilities are 2D/3D Discrete/Continuous or Network. In case of a Parallel Update-Strategy, collapsing may become necessary, depending on the semantics of the model. Also it is noted if the environment has behaviour. Note that an implementation may also have no environment which is noted as None. Although every model implemented in \textit{FrABs} needs to set up some environment, it is not required to use it in the implementation.
	\item Recursive - Is this implementation making use of the recursive features of \textit{FrABS} Yes/No (only available in sequential updating)?
	\item Conversations - Is this implementation making use of the conversations features of \textit{FrABS} Yes/No (only available in sequential updating)?
\end{itemize}

\subsection{Sugarscape}
This is a full implementation of the famous Sugarscape model as described by Epstein \& Axtell in their book \cite{epstein_growing_1996}. The model description itself has no real time-semantics, the agents act in every time-step. Only the environment may change its behaviour after a given number of steps but this is easily expressed without time-semantics as described in the model by Epstein \& Axtell \footnote{Note that this implementation has about 2600 lines of code which - although it includes both a pure and monadic implementation - is significant lower than e.g. the Java-implementation \url{http://sugarscape.sourceforge.net/} with about 6000. Of course it is difficult to compare such measures as we do not include FrABS itself into our measure.}.

\begin{center}
\begin{tabular}{l || l }
\textbf{Implementation}			& Pure, Monadic \\
\textbf{Yampa Time-Semantics}	& No \\
\textbf{Update-Strategy}		& Sequential, shuffling \\
\textbf{Environment}			& 2D Discrete, behaviour \\
\textbf{Recursive}				& No \\
\textbf{Conversations}			& Yes \\
\end{tabular}
\end{center}

\subsection{Agent\_Zero}
This is an implementation of the \textit{Parable 1} from the book of Epstein \cite{epstein_agent_zero:_2014}.

\begin{center}
\begin{tabular}{l || l }
\textbf{Implementation}			& Pure, Monadic \\
\textbf{Yampa Time-Semantics}	& No \\
\textbf{Update-Strategy}		& Parallel, Sequential, shuffling \\
\textbf{Environment}			& 2D Discrete, behaviour, collapsing \\
\textbf{Recursive}				& No \\
\textbf{Conversations}			& No \\
\end{tabular}
\end{center}

\subsection{Schelling Segregation}
This is an implementation of \cite{schelling_dynamic_1971} with extended agent-behaviour which allows to study dynamics of different optimization behaviour: local or global, nearest/random, increasing/binary/future. This is also the only 'real' model in which the recursive features were applied \footnote{The example of Recursive ABS is just a plain how-to example without any real deeper implications.}.

\begin{center}
\begin{tabular}{l || l }
\textbf{Implementation}			& Pure \\
\textbf{Yampa Time-Semantics}	& No \\
\textbf{Update-Strategy}		& Sequential, shuffling \\
\textbf{Environment}			& 2D Discrete \\
\textbf{Recursive}				& Yes (optional) \\
\textbf{Conversations}			& No \\
\end{tabular}
\end{center}

\subsection{Prisoners Dilemma}
This is an implementation of the Prisoners Dilemma on a 2D Grid as discussed in the papers of \cite{nowak_evolutionary_1992}, \cite{huberman_evolutionary_1993} and TODO: cite my own paper on update-strategies.

TODO: implement

\subsection{Heroes \& Cowards}
This is an implementation of the Heroes \& Cowards Game as introduced in \cite{wilensky_introduction_2015} and discussed more in depth in TODO: cite my own paper on update-strategies.

TODO: implement

\subsection{SIRS}
This is an early, non-reactive implementation of a spatial version of the SIRS compartment model found in epidemiology. Note that although the SIRS model itself includes time-semantics, in this implementation no use of Yampas facilities were made. Timed transitions and making contact was implemented directly into the model which results in contacts being made on every iteration, independent of the sampling time. Also in this sample only the infected agents make contact with others, which is not quite correct when wanting to approximate the System Dynamics model (see below). It is primarily included as a comparison to the later implementations (Fr*SIRS) of the same model  which make full use of \textit{FrABS} and to see the huge differences the usage of Yampas time-semantics can make.

\begin{center}
\begin{tabular}{l || l }
\textbf{Implementation}			& Pure, Monadic \\
\textbf{Yampa Time-Semantics}	& No \\
\textbf{Update-Strategy}		& Parallel, Sequential with shuffling \\
\textbf{Environment}			& 2D Discrete \\
\textbf{Recursive}				& No \\
\textbf{Conversations}			& No \\
\end{tabular}
\end{center}

\subsection{Fr(Spatial$|$Network)SIRS}
This is the reactive implementations of both 2D spatial and network (complete graph, Erdos-Renyi and Barbasi-Albert) versions of the SIRS compartment model. Unlike SIRS these examples make full use of the time-semantics provided by Yampa and show the real strength provided by \textit{FrABS}.

\begin{center}
\begin{tabular}{l || l }
\textbf{Implementation}			& Reactive \\
\textbf{Yampa Time-Semantics}	& Yes \\
\textbf{Update-Strategy}		& Parallel \\
\textbf{Environment}			& 2D Discrete, Network \\
\textbf{Recursive}				& No \\
\textbf{Conversations}			& No \\
\end{tabular}
\end{center}

\subsection{System Dynamics SIR}
This is an emulation of the System Dynamics model of the SIR compartment model in epidemiology. It was implemented as a proof-of-concept to show that \textit{FrABS} is able to implement even System Dynamic models because of its continuous-time and time-semantic features. Connections between stocks \& flows are hardcoded, after all System Dynamics completely lacks the concept of spatial- or network-effects. Note that describing the implementation as Reactive may seem not appropriate as in System Dynamics we are not dealing with any events or reactions to it - it is all about a continuous flow between stocks. In this case we wanted to express with Reactive that it is implemented using the Arrowized notion of Yampa which is required when one wants to use Yampas time-semantics anyway.

\begin{center}
\begin{tabular}{l || l }
\textbf{Implementation}			& Reactive \\
\textbf{Yampa Time-Semantics}	& Yes \\
\textbf{Update-Strategy}		& Parallel \\
\textbf{Environment}			& None \\
\textbf{Recursive}				& No \\
\textbf{Conversations}			& No \\
\end{tabular}
\end{center}

\subsection{WildFire}
This is an implementation of a very simple Wildfire model inspired by an example from AnyLogic\texttrademark with the same name.

\begin{center}
\begin{tabular}{l || l }
\textbf{Implementation}			& Reactive \\
\textbf{Yampa Time-Semantics}	& Yes \\
\textbf{Update-Strategy}		& Parallel \\
\textbf{Environment}			& 2D Discrete \\
\textbf{Recursive}				& No \\
\textbf{Conversations}			& No \\
\end{tabular}
\end{center}

\subsection{Double Auction}
This is a basic implementation of a double-auction process of a model described by \cite{breuer_endogenous_2015}. This model is not relying on any environment at the moment but could make use of networks in the future for matching offers.

\begin{center}
\begin{tabular}{l || l }
\textbf{Implementation}			& Pure, Monadic \\
\textbf{Yampa Time-Semantics}	& No \\
\textbf{Update-Strategy}		& Parallel \\
\textbf{Environment}			& None \\
\textbf{Recursive}				& No \\
\textbf{Conversations}			& No \\
\end{tabular}
\end{center}

\subsection{Proof of concepts}
\subsubsection{Recursive ABS} This example shows the very basics of how to implement a recursive ABS using \textit{FrABS}. Note that recursive features only work within the sequential strategy.

\begin{center}
\begin{tabular}{l || l }
\textbf{Implementation}			& Pure \\
\textbf{Yampa Time-Semantics}	& No \\
\textbf{Update-Strategy}		& Sequential \\
\textbf{Environment}			& None \\
\textbf{Recursive}				& Yes \\
\textbf{Conversations}			& No \\
\end{tabular}
\end{center}

\subsubsection{Conversation} This example shows the very basics of how to implement conversations in \textit{FrABS}. Note that conversations only work within the sequential strategy.

\begin{center}
\begin{tabular}{l || l }
\textbf{Implementation}			& Pure \\
\textbf{Yampa Time-Semantics}	& No \\
\textbf{Update-Strategy}		& Sequential \\
\textbf{Environment}			& None \\
\textbf{Recursive}				& No \\
\textbf{Conversations}			& Yes \\
\end{tabular}
\end{center}

\section{Discussion}

\subsection{Other Models}
TODO: mention that we have also implemented other models, which also work without time-semantics (all agents make a move at discrete time-steps and do not really rely on some notion of time). 

\subsection{Time-Semantics}
The main reason for building our pure functional ABMS approach on top of Yampa was to leverage the powerful time-semantics of Yampa which allows us to implement important concepts of ABMS:

state-chart: agents are at all time of their life-cycle in one state and can switch between multiple states using transitions 
timed transitions: transition to another state/behaviour happens at a discrete time
rate transitions: transition happens with a given rate
message transition: transition upon receiving a given message 

\subsection{Agents as Signals}
Due to the underlying nature and motivation of Functional Reactive Programming (und im speziellen) Yampa, Agents can be seen as Signals which is generated and consumed by a Signal-Function which is the behaviour of an Agent. If an Agent does not change the OUTPUT-signal is constant, if the agent changes e.g. by sending a message, changing its state,... the OUTPUT signal changes. A dead agent has no signal at all.

\subsection{Time-Sampling}
sampling rate depends on the transition times \& rates of the model. when e.g. the contact rate is 5 then the sampling dt should be below 0.2

\subsection{System Dynamics}
can emulate system dynamics due to the parallel update-strategy and continuous time-flow semantics

\subsection{Discrete Event Simulation}
DES in FrABMS? how easily can we implement server/queue systems? do they also look like a specification? potential problem: ordering of messages is not guaranteed by now

\subsection{Advantages}
advantages:
	- no side-effects within agents leads to much safer code
	- edsl for time-semantics
	- declarative style: agent-implementation looks like a model-specification
	- reasoning and verification
	- sequential and parallel
	- powerful time-semantics
	- arrowized programming is optional and only required when utilizing yampas time-semantics. if the model does not rely on time-semantics, it can use monadic-programming by building on the existing monadic functions in the EDSL which allow to run in the State-Monad which simplifies things very much
	- when to use yampas arrowized programing: time-semantics, simple state-chart agents 
	- when not using yampas facilities: in all the other cases e.g. SugarScape is such a case as it proceeds in unit time-steps and all agents act in every time-step
	- can implement System Dynamics building on Yampas facilities with total ease	
	- get replications for free without having to worry about side-effects and can even run them in parallel without headaches
	- cant mess around with time because delta-time is hidden from you (intentional design-decision by Yampa). this would be only very difficult and cumbersome to achieve in an object-oriented approach. TODO: experiment with it in Java - how could we actually implement this? I think it is impossible: may only achieve this through complicated application of patterns and inheritance but then has the problem of how to update the dt and more important how to deal with functions like integral which accumulates a value through closures and continuations. We could do this in OO by having a general base-class e.g. ContinuousTime which provides functions like updateDt and integrate, but we could only accumulate a single integral value.
	- reproducibility statically guaranteed
	- cannot mess around with dt
	- code == specification
	- rule out serious class of bugs
	- different time-sampling leads to different results e.g. in wildfire \& SIR but not in Prisoners Dilemma. why? probabilistic time-sampling?
	- reasoning about equivalence between SD and ABS implementation in the same framework
	- recursive implementations
	
	- we can statically guarantee the reproducibility of the simulation because: no side effects possible within the agents which would result in differences between same runs (e.g. file access, networking, threading), also timedeltas are fixed and do not depend on rendering performance or userinput	
	
\subsection{Disadvantages}
disadvantages:
	- performance is low
	- reasoning about performance is very difficult
	- very steep learning curve for non-functional programmers
	- learning a new EDSL
	- think ABMS different: when to use async messages, when to use sync conversations


[ ] important: increasing sampling freqzency and increasing number of steps so that the same number of simulation steps are executed should lead to same results. but it doesnt. why?
[ ] hypothesis: if time-semantics are involved then event ordering becomes relevant for emergent patterns. there are no tine semantics in heroes and cowards but in the prisoners dilemma
[ ] can we implement different types of agents interacting with each other in the same simulation ? with different behaviour funcs, digferent state? yes, also not possible in NetLogo to my knowledge. but they must have the same messages, emvironment 

[ ] Hypothesis: we can combine with FrABS agent-based simulation and system dynamics (this has been proved by example!)

\section{Conclusions}
\label{sec:conclusions}

Our approach is radically different from traditional approaches in the ABS community. First it builds on the already quite powerful FRP paradigm. Second, due to our continuous time approach, it forces one to think properly of time-semantics of the model and how small $\Delta t$ should be. Third it requires to think about agent interactions in a new way instead of being just method-calls.

Because no part of the simulation runs in the IO Monad and we do not use unsafePerformIO we can rule out a serious class of bugs caused by implicit data-dependencies and side-effects which can occur in traditional imperative implementations.

Also we can statically guarantee the reproducibility of the simulation, which means that repeated runs with the same initial conditions are guaranteed to result in the same dynamics. Although we allow side-effects within agents, we restrict them to only the Random and State Monad in a controlled, deterministic way and never use the IO Monad which guarantees the absence of non-deterministic side effects within the agents and other parts of the simulation.

Determinism is also ensured by fixing the $\Delta t$ and not making it dependent on the performance of e.g. a rendering-loop or other system-dependent sources of non-determinism as described by \cite{perez_testing_2017}. Also by using FRP we gain all the benefits from it and can use research on testing, debugging and exploring FRP systems \cite{perez_testing_2017, perez_back_2017}.

\subsection*{Issues}
Currently, the performance of the system is not comparable to imperative implementations but our research was not focusing on this aspect. We leave the investigation and optimization of the performance aspect of our approach for further research.

Despite the strengths and benefits we get by leveraging on FRP, there are errors that are not raised at compile time, e.g. we can still have infinite loops and run-time errors. This was for example investigated in \cite{sculthorpe_safe_2009} where the authors use dependent types to avoid some run-time errors in FRP. We suggest that one could go further and develop a domain specific type system for FRP that makes the FRP based ABS more predictable and that would support further mathematical analysis of its properties. Furthermore, moving to dependent types would pose a unique benefit over the traditional object-oriented approach and should allow us to express and guarantee even more properties at compile time. We leave this for further research.

In our pure functional approach, agent identity is not as clear as in traditional object-oriented programming, where an agent can be hidden behind a polymorphic interface which is much more abstract than in our approach. Also the identity of an agent is much clearer in object-oriented programming due to the concept of object-identity and the encapsulation of data and methods.

We can conclude that the main difficulty of a pure functional approach evolves around the communication and interaction between agents, which is a direct consequence of the issue with agent identity. Agent interaction is straight-forward in object-oriented programming, where it is achieved using method-calls mutating the internal state of the agent, but that comes at the cost of a new class of bugs due to implicit data flow. In pure functional programming these data flows are explicit but our current approach of feeding back the states of all agents as inputs is not very general and we have added further mechanisms of agent interaction which we had to omit due to lack of space.

\bibliographystyle{../../templates/IEEEtran/bibtex/IEEEtran}
\bibliography{../../../references/phdReferences.bib}

\appendices

\section{Examples}
In this appendix we give a list of all the examples we have implemented and discuss implementation details relevant \footnote{The examples are freely available under \url{https://github.com/thalerjonathan/phd/tree/master/coding/libraries/frABS/examples}}. The examples were implemented as use-cases to drive the development of \textit{FrABS} and to give code samples of known models which show how to use this new approach. Note that we do not give an explanation of each model as this would be out of scope of this paper but instead give the major references from which an understanding of the model can be obtained.

We distinguish between the following attributes
\begin{itemize}
	\item Implementation - Which style was used? Either Pure, Monadic or Reactive. Examples could have been implemented in all of them.
	\item Yampa Time-Semantics - Does the implemented model make use of Yampas time-semantics e.g. occasional, after,...? Yes / No.
	\item Update-Strategy - Which update-strategy is required for the given example? It is either Sequential or Parallel or both. In the case of Sequential Agents may be shuffled or not.
	\item Environment - Which kind of environment is used in the given example? Possibilities are 2D/3D Discrete/Continuous or Network. In case of a Parallel Update-Strategy, collapsing may become necessary, depending on the semantics of the model. Also it is noted if the environment has behaviour. Note that an implementation may also have no environment which is noted as None. Although every model implemented in \textit{FrABs} needs to set up some environment, it is not required to use it in the implementation.
	\item Recursive - Is this implementation making use of the recursive features of \textit{FrABS} Yes/No (only available in sequential updating)?
	\item Conversations - Is this implementation making use of the conversations features of \textit{FrABS} Yes/No (only available in sequential updating)?
\end{itemize}

\subsection{Sugarscape}
This is a full implementation of the famous Sugarscape model as described by Epstein \& Axtell in their book \cite{epstein_growing_1996}. The model description itself has no real time-semantics, the agents act in every time-step. Only the environment may change its behaviour after a given number of steps but this is easily expressed without time-semantics as described in the model by Epstein \& Axtell \footnote{Note that this implementation has about 2600 lines of code which - although it includes both a pure and monadic implementation - is significant lower than e.g. the Java-implementation \url{http://sugarscape.sourceforge.net/} with about 6000. Of course it is difficult to compare such measures as we do not include FrABS itself into our measure.}.

\begin{center}
\begin{tabular}{l || l }
\textbf{Implementation}			& Pure, Monadic \\
\textbf{Yampa Time-Semantics}	& No \\
\textbf{Update-Strategy}		& Sequential, shuffling \\
\textbf{Environment}			& 2D Discrete, behaviour \\
\textbf{Recursive}				& No \\
\textbf{Conversations}			& Yes \\
\end{tabular}
\end{center}

\subsection{Agent\_Zero}
This is an implementation of the \textit{Parable 1} from the book of Epstein \cite{epstein_agent_zero:_2014}.

\begin{center}
\begin{tabular}{l || l }
\textbf{Implementation}			& Pure, Monadic \\
\textbf{Yampa Time-Semantics}	& No \\
\textbf{Update-Strategy}		& Parallel, Sequential, shuffling \\
\textbf{Environment}			& 2D Discrete, behaviour, collapsing \\
\textbf{Recursive}				& No \\
\textbf{Conversations}			& No \\
\end{tabular}
\end{center}

\subsection{Schelling Segregation}
This is an implementation of \cite{schelling_dynamic_1971} with extended agent-behaviour which allows to study dynamics of different optimization behaviour: local or global, nearest/random, increasing/binary/future. This is also the only 'real' model in which the recursive features were applied \footnote{The example of Recursive ABS is just a plain how-to example without any real deeper implications.}.

\begin{center}
\begin{tabular}{l || l }
\textbf{Implementation}			& Pure \\
\textbf{Yampa Time-Semantics}	& No \\
\textbf{Update-Strategy}		& Sequential, shuffling \\
\textbf{Environment}			& 2D Discrete \\
\textbf{Recursive}				& Yes (optional) \\
\textbf{Conversations}			& No \\
\end{tabular}
\end{center}

\subsection{Prisoners Dilemma}
This is an implementation of the Prisoners Dilemma on a 2D Grid as discussed in the papers of \cite{nowak_evolutionary_1992}, \cite{huberman_evolutionary_1993} and TODO: cite my own paper on update-strategies.

TODO: implement

\subsection{Heroes \& Cowards}
This is an implementation of the Heroes \& Cowards Game as introduced in \cite{wilensky_introduction_2015} and discussed more in depth in TODO: cite my own paper on update-strategies.

TODO: implement

\subsection{SIRS}
This is an early, non-reactive implementation of a spatial version of the SIRS compartment model found in epidemiology. Note that although the SIRS model itself includes time-semantics, in this implementation no use of Yampas facilities were made. Timed transitions and making contact was implemented directly into the model which results in contacts being made on every iteration, independent of the sampling time. Also in this sample only the infected agents make contact with others, which is not quite correct when wanting to approximate the System Dynamics model (see below). It is primarily included as a comparison to the later implementations (Fr*SIRS) of the same model  which make full use of \textit{FrABS} and to see the huge differences the usage of Yampas time-semantics can make.

\begin{center}
\begin{tabular}{l || l }
\textbf{Implementation}			& Pure, Monadic \\
\textbf{Yampa Time-Semantics}	& No \\
\textbf{Update-Strategy}		& Parallel, Sequential with shuffling \\
\textbf{Environment}			& 2D Discrete \\
\textbf{Recursive}				& No \\
\textbf{Conversations}			& No \\
\end{tabular}
\end{center}

\subsection{Fr(Spatial$|$Network)SIRS}
This is the reactive implementations of both 2D spatial and network (complete graph, Erdos-Renyi and Barbasi-Albert) versions of the SIRS compartment model. Unlike SIRS these examples make full use of the time-semantics provided by Yampa and show the real strength provided by \textit{FrABS}.

\begin{center}
\begin{tabular}{l || l }
\textbf{Implementation}			& Reactive \\
\textbf{Yampa Time-Semantics}	& Yes \\
\textbf{Update-Strategy}		& Parallel \\
\textbf{Environment}			& 2D Discrete, Network \\
\textbf{Recursive}				& No \\
\textbf{Conversations}			& No \\
\end{tabular}
\end{center}

\subsection{System Dynamics SIR}
This is an emulation of the System Dynamics model of the SIR compartment model in epidemiology. It was implemented as a proof-of-concept to show that \textit{FrABS} is able to implement even System Dynamic models because of its continuous-time and time-semantic features. Connections between stocks \& flows are hardcoded, after all System Dynamics completely lacks the concept of spatial- or network-effects. Note that describing the implementation as Reactive may seem not appropriate as in System Dynamics we are not dealing with any events or reactions to it - it is all about a continuous flow between stocks. In this case we wanted to express with Reactive that it is implemented using the Arrowized notion of Yampa which is required when one wants to use Yampas time-semantics anyway.

\begin{center}
\begin{tabular}{l || l }
\textbf{Implementation}			& Reactive \\
\textbf{Yampa Time-Semantics}	& Yes \\
\textbf{Update-Strategy}		& Parallel \\
\textbf{Environment}			& None \\
\textbf{Recursive}				& No \\
\textbf{Conversations}			& No \\
\end{tabular}
\end{center}

\subsection{WildFire}
This is an implementation of a very simple Wildfire model inspired by an example from AnyLogic\texttrademark with the same name.

\begin{center}
\begin{tabular}{l || l }
\textbf{Implementation}			& Reactive \\
\textbf{Yampa Time-Semantics}	& Yes \\
\textbf{Update-Strategy}		& Parallel \\
\textbf{Environment}			& 2D Discrete \\
\textbf{Recursive}				& No \\
\textbf{Conversations}			& No \\
\end{tabular}
\end{center}

\subsection{Double Auction}
This is a basic implementation of a double-auction process of a model described by \cite{breuer_endogenous_2015}. This model is not relying on any environment at the moment but could make use of networks in the future for matching offers.

\begin{center}
\begin{tabular}{l || l }
\textbf{Implementation}			& Pure, Monadic \\
\textbf{Yampa Time-Semantics}	& No \\
\textbf{Update-Strategy}		& Parallel \\
\textbf{Environment}			& None \\
\textbf{Recursive}				& No \\
\textbf{Conversations}			& No \\
\end{tabular}
\end{center}

\subsection{Proof of concepts}
\subsubsection{Recursive ABS} This example shows the very basics of how to implement a recursive ABS using \textit{FrABS}. Note that recursive features only work within the sequential strategy.

\begin{center}
\begin{tabular}{l || l }
\textbf{Implementation}			& Pure \\
\textbf{Yampa Time-Semantics}	& No \\
\textbf{Update-Strategy}		& Sequential \\
\textbf{Environment}			& None \\
\textbf{Recursive}				& Yes \\
\textbf{Conversations}			& No \\
\end{tabular}
\end{center}

\subsubsection{Conversation} This example shows the very basics of how to implement conversations in \textit{FrABS}. Note that conversations only work within the sequential strategy.

\begin{center}
\begin{tabular}{l || l }
\textbf{Implementation}			& Pure \\
\textbf{Yampa Time-Semantics}	& No \\
\textbf{Update-Strategy}		& Sequential \\
\textbf{Environment}			& None \\
\textbf{Recursive}				& No \\
\textbf{Conversations}			& Yes \\
\end{tabular}
\end{center}

\section{Recursive Agent-Based Simulation}
The idea for this paper arose from my idea of \textit{anticipating agents}, which can project their actions in the future. Because this paper is not as polished as the draft for programming paradigms, we opted not to include it as an appendix and only give its basic ideas and results for the experiments conducted so far. Note that we were not able to find any research regarding recursive ABS \footnote{We found a paper on recursive simulation in general \cite{gilmer_recursive_2000} which focuses on military simulation implemented in C++. Its main findings are that deterministic models seem to benefit significantly from using recursions of the simulation for the decision making process and that when using stochastic models this benefit seems to be lost.}.
In Recursive ABS agents are able to halt time and 'play through' an arbitrary number of actions, compare their outcome and then to resume time and continue with a specifically chosen action e.g. the best performing or the one in which they haven't died. More precisely, what we want is to give an agent the ability to run the simulation recursively a number of times where the this number is not determined initially but can depend on the outcome of the recursive simulation. So Recursive ABS gives each Agent the ability to run the simulation locally from its point of view to anticipate its actions in the future and change them in the present.
We investigate the famous Schelling Segregation \cite{schelling_dynamic_1971} and endow our agents with the ability to project their actions into the future by recursively running simulations. Based on the outcome of the recursions they are then able to determine whether their move increases their utility in the future or not. The main finding for now is that it does not increase the convergence speed to equilibrium but can lead to extreme volatility of dynamics although the system seems to be near to complete equilibrium. In the case of a 10x10 field it was observed that although the system was nearly in its steady state - all but one agent were satisfied - the move of a single agent caused the system to become completely unstable and depart from its near-equilibrium state to a highly volatile and unstable state.

This approach of course rises a few questions and issues. The main problem of our approach is that, depending on ones view-point, it is violating the principles of locality of information and limit of computing power. To recursively run the simulation the agent which initiates the recursion is feeding in all the states of the other agents and calculates the outcome of potentially multiple of its own steps, each potentially multiple recursion-layers deep and each recursion-layer multiple time-steps long. Both requires that each agent has perfect information about the complete simulation \textit{and} can compute these 3-dimensional recursions, which scale exponentially. In the social sciences where agents are often designed to have only very local information and perform low-cost computations it is very difficult or impossible to motivate the usage of recursive simulations - it simply does not match the assumptions of the real world, the social sciences want to model. In general simulations, where it is much more commonly accepted to assume perfect information and potentially infinite amount of computing power this approach is easily motivated by a constructive argument: it is possible to build, thus we build it.
Another fundamental question regards the meaning and epistemology behind an entity running simulations. Of course, this strongly depends on the context: in ACE it may be understood as a search for optimizing behaviour, in Social Simulation it may be interpreted as a kind of free will: the agent who is initiating the recursion can be seen as 'knowing' that it is running inside a simulation, thus in this context free will is seen as being able to anticipate ones actions and change them.
When talking about recursion it is always the question of the depth of the recursion and because as we are running on computers we need to terminate at some point. Accelerating Turing machines (also known as Zeno Machine) are theoretically able to calculate an infinite regress but this raises again epistemological questions and can be seen as having religious character as discussed e.g. in Tiplers Omega Point, Bostroms simulation argument \cite{bostrom_are_2003} and its theological implications \cite{steinhart_theological_2010}. So the ultimate question this research leaves is what the outcome would be when running a recursive ABS on a Zeno Machine/Accelerated Turing Machine? \footnote{Anyway this would mean we have infinite amount of computing power - I am sure that in this case we don't worry the slightest about recursive ABS any more.}

At the moment this idea lies dormant as the intention was just to develop it far enough to give a proof-of-concept and see some results. Having achieved this we arrived at the conclusion, that the results are not really ground-breaking. This stems from the fact that Schelling segregation is not the best model to demonstrate this technique and that we are thus lacking the right model in which recursive ABS is the real killer-feature. Also to pursue this direction further and treat it in-depth, would require much more time and give the PhD a complete different spin. Still it is useful in supporting our move towards pure functional ABS as we are convinced that recursion is comparably easy to implement because the language is built on it and due to the lack of side-effects \footnote{Actually implementing it was \textit{really hard} but we wouldn't dare to implement this into an object-oriented language or into an object-oriented ABS framework.}.

\end{document}