\section{Background}

\subsection{Agent Based Simulation}
We understand ABS as a method of modelling and simulating a system where the global behaviour may be unknown but the behaviour and interactions of the parts making up the system is of knowledge. Those parts, called agents, are modelled and simulated out of which then the aggregate global behaviour of the whole system emerges. So the central aspect of ABS is the concept of an agent which can be understood as a metaphor for a pro-active unit, situated in an environment, able to spawn new agents and interacting with other agents in a network of neighbours by exchange of messages \cite{wooldridge_introduction_2009}. It is important to note that we focus our understanding of ABS on a very specific kind of agents where the focus is on communicating entities with individual, localized behaviour from out of which the global behaviour of the system emerges. We informally assume the following about our agents:

\begin{itemize}
	\item They are uniquely addressable entities with some internal state.
	\item They can initiate actions on their own e.g. change their internal state, send messages, create new agents, kill themselves.
	\item They can react to messages they receive with actions as above.
	\item They can interact with an environment they are situated in.
\end{itemize} 

We will give a formal model of agents in a subsequent section (TODO: reference)

\subsection{Pure functional paradigm}
TODO: short description what pure functional paradigm is
Object-oriented (OO) programming is the current state-of-the-art method used in implementing ABM/S due to the natural way of mapping concepts and models of ABM/S to an OO-language. Although this dominance in the field we claim that OO has also its serious drawbacks:

\begin{itemize}
\item Mutable State is distributed over multiple objects which is often very difficult to understand, track and control.
\item Inheritance is a dangerous thing if not used properly and with care because it introduces very strong dependencies which cannot be changed during runtime any-more.
\item Objects don't compose very well due to their internal (mutable) state (note that we are aware that there is the concept of immutable objects which are becoming more and more popular but that does not solve the fundamental problem.
\item It is (nearly) impossible to reason about programs.
\end{itemize}

We claim that these drawbacks are non-existent in pure functional programming like Haskell due to the nature of the functional approach. To give an introduction into functional programming is out of scope of this paper but we refer to the classical paper of \cite{hughes_why_1989} which is a great paper explaining to non-functional programmers what the significance of functional programming is and helping functional programmers putting functional languages to maximum use by showing the real power and advantages of functional languages. The main conclusion of this classical paper is that \textit{modularity}, which is the key to successful programming, can be achieved best using higher-order functions and lazy evaluation provided in functional languages like Haskell. \cite{hughes_why_1989} argues that the ability to divide problems into sub-problems depends on the ability to glue the sub-problems together which depends strongly on the programming-language and \cite{hughes_why_1989} argues that in this ability functional languages are superior to structured programming.

\subsubsection{Naive ABS implementation}
TODO: shortcomings
Of course the basic pure functional primitives alone do not make a well structured functional program by themselves as the usage of classes, interfaces, objects and inheritance alone does not make a well structured object-oriented program. What is needed are \textit{patterns} how to use the primitives available in pure functional programs to arrive at well structure programs. In object-orientation much work has been done in the 90s by the highly influential book \cite{gamma_design_1994} whereas in functional programming the major inventions were also done in the 90s by the invention of Monads through \cite{Moggi1989}, \cite{Wadler1990} and \cite{Wadler1995} and beginning of the 2000s by the invention of Arrows through \cite{Hughes2000}.

\paragraph{Higher Order Functions \& Monads} map \& fmap, foldl, applicatives, \cite{hutton_programming_2007} gives a great overview and motivation for using fmap, applicatives and Monads. TODO: explain Monads

\paragraph{Arrows}
\cite{Hughes2004} is a great tutorial about \textit{Arrows} which are very well suited for structuring functional programs with effects.

\begin{quote}
Just like monads, arrow types are useful for the additional operations they support, over and above those that every arrow provides.
\end{quote}

The main difference between Monads and Arrows are that where monadic computations are parameterized only over their output-type, Arrows computations are parameterised both over their input- and output-type thus making Arrows more general.

\begin{quote}
In real applications an arrow often represents some kind of a process, with an input channel of type a, and an output channel of type b.
\end{quote}

In the work \cite{Hughes2004} an example for the usage for Arrows is given in the field of circuit simulation. They use previously introduced streams to advance the simulation in discrete steps to calculate values of circuits thus the implementation is a form of \textit{discrete event simulation} - which is in the direction we are heading already with ABM/S. Also the paper mentions Yampa which is introduced in the section (TODO: reference) on functional reactive programming.


\input{./tex/backgroundFRP.tex}

\subsection{Models}
 
\subsubsection{Sugarscape}
\cite{epstein_growing_1996} introduced the famous \textit{Sugarscape} model which is widely researched in the social sciences. TODO: short explanation
We included this model because it is well known and we wanted to test how much effort it takes to implement it with our approach. Because its complexity is moderate and there are only homogeneous agents it allows us to really show the power of our EDSL.

\subsection{ACE Trading World}
\cite{tesfatsion_agent-based_2006} gives a broad overview of agent-based computational economics (ACE), gives the four primary objectives of it and discusses advantages and disadvantages. She introduces a model called \textit{ACE Trading World} in which she shows how an artificial economy can be implemented without the \textit{Walrasian Auctioneer} but just by agents and their interactions. She gives a detailed mathematical specification in the appendix of the paper which should allow others to implement the simulation. We included this model because of its complexity and different, heterogeneous types of agents.

\subsection{Verification and Validation in ABS}
TODO: \url{http://www2.econ.iastate.edu/tesfatsi/DockingSimModels.pdf}
TODO: \url{http://www2.econ.iastate.edu/tesfatsi/empvalid.htm}
TODO: \url{http://jasss.soc.surrey.ac.uk/10/2/8.html}
TODO: \url{https://www.openabm.org/faq/how-validate-and-calibrate-agent-based-models}
TODO: \url{https://link.springer.com/chapter/10.1007%2F978-3-642-01109-2_10}
TODO: \url{http://www3.nd.edu/~nom/Papers/ADS019_Xiang.pdf}

\subsection{Related Research}
The amount of research on using the pure functional paradigm using Haskell in the field of ABS has been moderate so far.
Most of the papers look into how agents can be specified using the belief-desire-intention paradigm \cite{de_jong_suitability_2014}, \cite{sulzmann_specifying_2007}, \cite{jankovic_functional_2007}.
A library for Discrete Event Simulation (DES) and System Dynamics (SD) in Haskell called \textit{Aivika 3} is described in \cite{sorokin_aivika_2015}. It comes with very basic features for ABS but only allows to specify simple state-based agents with timed transitions.
\cite{jankovic_functional_2007} discuss using functional programming for discrete event simulation (DES) and mention the paradigm of FRP to be very suitable to DES.

\cite{sulzmann_specifying_2007} present an EDSL for Haskell allowing to specify Agents using the BDI model. TODO: We don't go there, thats not our intention. 

\cite{schneider_towards_2012} and \cite{vendrov_frabjous:_2014} present a domain-specific language for developing functional reactive agent-based simulations. This language called FRABJOUS is very human readable and easily understandable by domain-experts. It is not directly implemented in FRP/Haskell/Yampa but is compiled to Haskell/Yampa code which they claim is also readable. This is the direction we want to head but we don't want this intermediate step but look for how a most simple domain-specific language embedded in Haskell would look like. We also don't touch upon FRP and Yampa yet but leave this to further research for another paper of ours.

\cite{klugl_amason:_2013} TODO

TODO: cite julie greensmith paper on haskell