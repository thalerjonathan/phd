\section{The SIR Model}




TODO: also solve the SIR with an algebra system to have a bullet-proof "proof" that we reproduce the same dynamics. this is only partially a proof our system is correct, but it is not a formal proof, this needs to be done different

TODO: It should be possible to formally show that spatial SIR and WildFire are the same model. NOTE: they are NOT the same, the fundamental difference is that in the WildFire model only the burning cells initiate the ignition - if we compare this to the SIR, the burning cells would be infected agents and although in the spatial SIR model the infected agents make contact with other agents, so do the susceptible ones which does NOT occur in wildfire

TODO: cite my own work on update-strategies

TODO: can we formally show that the SIR approximates the SD model?

TODO: cite papers which discuss how to approximate a SD model by ABS
- Macal (2010) - To Agent-Based Simulation From System Dynamics 
	-> i am very unhappy with this paper: first it does not give concrete parameters for the SD model so it is impossible to replicate. Also i think it has a systematical error as the infected agents make no contact but this is required as evident from the SD-models infection-rate which also incorporates. TODO: write an email to this guy: why are the infectious not contacting the other agents? this seems to be a systematical error
- Borshchev, Filippov (2004) - From System Dynamics and Discrete Event to Practical Agent Based Modeling: Reasons, Techniques, Tools
	-> its VERY IMPORTANT point is that we need to draw the illness-duration from an exponential-distribution because the illness-duration is proportional to the size of the infected. note: this is wrongly expressed, need to find the correct formulation

		-> my emulation of SD using ABS is really an implementation of the SD model and follows it - they are equivalent
		-> my ABS implementation is the same as / equivalent to the SD emulation
			=> thus if i can show that my SD emulation is equlas to the SD model
			=> AND that the ABS implementation is the same as the SD emulation
			=> THEN the ABS implementation is an SD implementation, and we have shown this in code for the first time in ABS
			
\subsubsection{SIRS}
[ ] SIRS: timed transitions using after, occasionally sending messages, transitions on messages
$\frac{\mathrm d S}{\mathrm d t} = -infectionRate$ \\
$\frac{\mathrm d I}{\mathrm d t} = infectionRate - recoveryRate$ \\
$\frac{\mathrm d R}{\mathrm d t} = recoveryRate$ \\

$S(t) = N + \int_0^t -infectionRate\, \mathrm{d}t$ \\
$I(t) = 1 + \int_0^t infectionRate - recoveryRate\, \mathrm{d}t$ \\
$R(t) = \int_0^t recoveryRate\, \mathrm{d}t$ \\

$infectionRate = \frac{I \beta S \gamma}{N}$ \\
$recoveryRate = \frac{I}{\delta}$ \\

\subsection{Agent-Based SIR}
Advantage: can incorporate networking or spatial effects. In this example it runs within a discrete 2D environment. FrABMS library also has the ability for networks.

Hypothesis: high-frequency sampling is required when there are rates involved e.g. occacionally.
TODO: compare fully-connected SIR to System Dynamics solution
TODO: look into sequential (shuffled/nonshuffled) vs parallel

\subsection{System Dynamics SIR}
Because we have the powerful time-semantic features of yampa at hand which allows to sample a system at very high frequency with continuous time at hand we can easily implement System Dynamics.
Using parallel iteration-strategy (no shuffling of agents)
Thus in a way we can see FrABMS to be a hybrid approach between ABMS and System Dynamics.

\subsubsection{Stocks}
are completely defined by the formula
InitialValue + Integrate (0 to t) by dt (inflow - outflow)

by messages stocks communicate their current value to the flows which require them.
receive by messages the current vale of all flows relevant to them

\subsubsection{Flows}
Are stateless and can calculate any rate

receive through messages the current values of their relevant Stocks
send their current flow-value to the relevant Stocks