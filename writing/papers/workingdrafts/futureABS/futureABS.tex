\documentclass[a4paper, 10pt, conference]{../../templates/IEEEconf/IEEEconf}

\usepackage{graphicx}
\usepackage{caption} 
\usepackage{subcaption}
\usepackage{hyperref}
\usepackage{listings}
\usepackage{hhline}
\usepackage[utf8]{inputenc}
\usepackage{float}
\usepackage{amssymb}
\usepackage[autostyle=true]{csquotes}
\usepackage{amsmath}

\title{The Future of Agent-Based Simulation}

\author{
	Jonathan Thaler \\
	\email{jonathan.thaler@nottingham.ac.uk} \\
	\begin{affiliation}
		School of Computer Science, University of Nottingham
	\end{affiliation} \\
}

\begin{document}
\maketitle

\begin{abstract}
As Agent-Based Simulation (ABS) is still a young discipline we ask for its future and try to hypothesize in which direction it will develop. We claim that its logical future is to simulate our own world, thus creating a point where simulation and reality converge, an event we postulate and term \textit{Simulation-Convergence}. Just by being able to talk about this convergence hints that there may be strong connections between the real world and ABS - something other simulation tools can hardly claim for themselves. This for the first time will give us a framework which allows us to talk about metaphysics from a computer-scientific view point. As we will show the only really interesting and worthy reason for pursuing the \textit{Simulation-Convergence} is to simulate free will. We argue that the emergent properties of such a simulation of the free will are ideologies. Our final conclusion is that the \textit{Simulation-Convergence} may already have happened and that reality as we experience it is indeed an ABS created by our very self.
\end{abstract}

%*******************************************************************************
%*********************************** First Chapter *****************************
%*******************************************************************************

\chapter{Introduction}  %Title of the First Chapter
I noticed that it is pretty hard to convince an agent-based economics specialist who is not a computer scientist about a pure functional approach. My conjecture is that the implementation technique and method does not matter much to them because they have very little knowledge about programming and are almost always self-taught - they don't know about software-engineering, nothing about proper software-design and architecture, nothing about software-maintenance, nothing about unit-testing,... In the end they just "hack" the simulation in whatever language they are able to: C++, Visual Basic, Java or toolboxes like Netlogo. For them it is all about to \textit{get things done somehow} and not to get things done the right way or in a beautiful way - the way and the method doesn't matter, its just a necessary evil which needs to be done. Thus if functional programming could make their lives easier, then they will definitely welcome it. But functional programming is, i think, harder to learn and harder to understand - so one needs to provide an abstraction through EDSL. So I REALLY need to come up with convincing arguments why to use pure functional approaches in ACE THEY can understand, otherwise I will be lost and not heard (not published,...). \\

What ACE economists care for:

\begin{itemize}
\item Very: Qualitative modelling with quantitative results
\item Yes: Easy reproducibility
\item Likely: Reasoning about convergence?
\item Likely: EDSL
\end{itemize}

My contributions are: pure functional framework, functional agent-model for market-simulations, EDSL for market-simulations, qualitative / implicit modelling with quanitative results, reasoning in my framework about convergence \\

IDEA: could I develop non-causal modelling (models are expressed in terms of non-directed equations, modelled in signal-relations) to allow for qualitative modelling for the agent-based economists? See hybrid modelling paper of Yampa. \textbf{THIS WOULD BE A HUGE NOVEL CONTRIBUTION TO ACE ESPECIALLY WHEN COMBINED WITH AN EDSL AND PROVIDING FULL REFERENTIAL TRANSPARENCY TO KEEP THE ABILITY TO REASON ABOUT CONVERGENCE}. This should be covered in the "EDSL"-paper.

TODO: maybe i should really focus only on market models? otherwise too much? \\

central novelty of my PhD: model specification = runnable code. possible through EDSL. but only in specific subfield of ACE: market-models. need a functional description of the model, then translate it to model specification in EDSL and then run it to see dynamics. But: model specification moves closer to functional programming languages. \\

another novelty approach: model specification through qualitative instead of quantiative approaches. is this possible? \\

WHY FUNCTIONAL? "because its the ultimate approach to scientific computing": fewer bugs due to mutable state (why? is thos shown obkectively by someone?), shorter (again as above, productivity), more expressive and closer to math, EDSL, EDSL=model=simulation, better parallelising due to referental transparency, reasoning \\

scientific results need to be reproduced, especially when they have high impact. a more formal approach of specifying the model and the simulation (model=simulation) could lead to easier sharing and easier reporduction without ambigouites \\

pure functional agent-model \& theory, EDSL framework in Haskell for ACE

\begin{enumerate}
\item Which kind of problem do we have?
\item What aim is there? Solving the problem? 
\item How the aim is achieved by enumerating VERY CLEAR objectives.
\item What the impact one expects (hypothesis) and what it is (after results).
\end{enumerate}

Note: It is not in the interest of the researcher to develop new economic theories but to research the use of functional methods (programming and specification) in agent-based computational economics (ACE).

NOTE: Get the reader’s attention early in the introduction: motivation, significance, originality and novelty.

\section{Methods}
Methods need to be selected to implement the simulations. Special emphasis will be put on functional ones which will then be compared to established methods in the field of ABM/S and ACE. \\

Claim: non-programming environments are considered to be not powerful enough to capture the complexity of ACE implementations thus a programming approach to ACE will be always required.

\section{Scenarios}
To apply and test functional methods in ACE, four scenarios of ACE are selected and then the methods applied and compared with each other to see how each of them perform in comparison. The 4 selected scenarios represent a selection of the challenges posed in ACE: from very abstract ones to very operational ones.

\section{Comparison}
Each of the selected scenarios is then implemented using the selected methods where each solution is then compared against the following criteria: 

\begin{enumerate}
\item suitability for scientific computation
\item robustness
\item error-sources
\item testability
\item stability
\item extendability
\item size of code
\item maintainability
\item time taken for development
\item verification \& correctness
\item replications \& parallelism
\item EDSL
\end{enumerate}

This will then allow to compare the different methods against each other and to show under which circumstances functional methods shine and when they should not be used.

\section{Agent-Based Modelling and Simulation (ABM/S)}
ABM/S is a method of modelling and simulating a system where the global behaviour may be unknown but the behaviour and interactions of the parts making up the system is of knowledge (Wooldrige, M. (2009). An Introduction to MultiAgent Systems. John Wiley & Sons). Those parts, called agents, are modelled and simulated out of which then the aggregate global behaviour of the whole system emerges. Thus the central aspect of ABM/S is the concept of an Agent which can be understood as a metaphor for a pro-active unit, able to spawn new Agents, and interacting with other Agents in a network of neighbours by exchange of messages. The implementation of Agents can vary and strongly depends on the programming language and the kind of domain the simulation and model is situated in.

\section{Agent-Based Economics (ACE)}
According to Leigh Tesfatsion (Tesfatsion, L. (2006). Agent-based computational economics: A constructive approach to economic theory. In Tesfatsion, L. and Judd, K. L., editors, Handbook of Computational Economics, volume 2, chapter 16, pages 831–880. Elsevier, 1 edition.), one of the leading figures, ACE is "[...] computational modelling of economic processes (including whole economies) as open-ended dynamic systems of interacting agents." - thus lending perfectly to the use of ABM/S as already the name suggests. Whereas classical economic models fall short by only looking at the average, pure rational, individual interacting in anonymous markets, the ACE approach looks at heterogeneous, non-rational individuals interacting with each other in networks (Kirman, A. (2010). Complex Economics: Individual and Collective Rationality. Routledge, London ; New York, NY.). Thus ACE can be understood as a combination of computer-science, cognitive/social science and evolutionary economics.

\section{Functional programming}
TODO: read \cite{Backus1978}

The state-of-the-art approach to implementing Agents are object-oriented methods and programming as the metaphor of an Agent as presented above lends itself very naturally to object-orientation (OO). The author of this thesis claims that OO in the hands of inexperienced or ignorant programmers is dangerous, leading to bugs and hardly maintainable and extensible code. The reason for this is that OO provides very powerful techniques of organising and structuring programs through Classes, Type Hierarchies and Objects, which, when misused, lead to the above mentioned problems. Also major problems, which experts face as well as beginners are 1. state is highly scattered across the program which disguises the flow of data in complex simulations and 2. objects don’t compose as well as functions. The reason for this is that objects always carry around some internal state which makes it obviously much more complicated as complex dependencies can be introduced according to the internal state.
All this is tackled by (pure) functional programming which abandons the concept of global state, Objects and Classes and makes data-flow explicit. This then allows to reason about correctness, termination and other properties of the program e.g. if a given function exhibits side-effects or not. Other benefits are fewer lines of code, easier maintainability and ultimately fewer bugs thus making functional programming the ideal choice for scientific computing and simulation and thus also for ACE. A very powerful feature of functional programming is Lazy evaluation. It allows to describe infinite data-structures and functions producing an infinite stream of output but which are only computed as currently needed. Thus the decision of how many is decoupled from how to (Hughes, J. (1989). Why functional programming matters. Comput. J., 32(2):98–107.).
The most powerful aspect using pure functional programming however is that it allows the design of embedded domain specific languages (EDSL). In this case one develops and programs primitives e.g. types and functions in a host language (embed) in a way that they can be combined. The combination of these primitives then looks like a language specific to a given domain, in the case of this thesis ACE. The ease of development of EDSLs in pure functional programming is also a proof of the superior extensibility and composability of pure functional languages over OO (Henderson P. (1982). Functional Geometry. Proceedings of the 1982 ACM Symposium on LISP and Functional Programming.).
One of the most compelling example to utilize pure functional programming is the reporting of Hudak (Hudak P., Jones M. (1994). Haskell vs. Ada vs. C++ vs. Awk vs. ... An Experiment in Software Prototyping Productivity. Department of Computer Science, Yale University.)  where in a prototyping contest of DARPA the Haskell prototype was by far the shortest with 85 lines of code. Also the Jury mistook the code as specification because the prototype did actually implement a small EDSL which is a perfect proof how close EDSL can get to and look like a specification.

Functional languages can best be characterized by their way computation works: instead of \textit{how} something is computed, \textit{what} is computed is described. Thus functional programming follows a declarative instead of an imperative style of programming. The key points are:
\begin{itemize}
\item No assignment statements - variables values can never change once given a value.
\item Function calls have no side-effect and will only compute the results - this makes order of execution irrelevant, as due to the lack of side-effects the logical point in \textit{time} when the function is calculated within the program-execution does not matter.
\item higher-order functions
\item lazy evaluation
\item Looping is achieved using recursion, mostly through the use of the general fold or the more specific map.
\item Pattern-matching
\end{itemize}

This alone does not really explain the \textit{real} advantages of functional programming and one must look for better motivations using functional programming languages. One motivation is given in \cite{Hughes1989} which is a great paper explaining to non-functional programmers what the significance of functional programming is and helping functional programmers putting functional languages to maximum use by showing the real power and advantages of functional languages. The main conclusion is that \textit{modularity}, which is the key to successful programming, can be achieved best using higher-order functions and lazy evaluation provided in functional languages like Haskell. \cite{Hughes1989} argues that the ability to divide problems into sub-problems depends on the ability to glue the sub-problems together which depends strongly on the programming-language and \cite{Hughes1989} argues that in this ability functional languages are superior to structured programming.

TODO: comparison of functional and object-oriented programming. My points are:
\begin{itemize}
\item The way state can be changed and treated - distributed over multiple objects - is often very difficult to understand.
\item Inheritance is a dangerous thing if not used with care because inheritance introduces very strong dependencies which cannot be changed during runtime anymore.
\item Objects don't compose very well: \url{http://zeroturnaround.com/rebellabs/why-the-debate-on-object-oriented-vs-functional-programming-is-all-about-composition/}
\item (Nearly) impossible to reason about programs
\end{itemize}

In conclusion the upsides of functional programming as opposed to OO are:
\begin{itemize}
\item Much more explicit flow of data \& control
\item Much better compose-able
\item Much better parallelism
\end{itemize}

\section{Related Research}
Tim Sweeney, CTO of Epic Games gave an invited talk about how "future programming languages could help us write better code" by "supplying stronger typing, reduce run-time failures;  and the need for pervasive concurrency support, both implicit and explicit, to effectively exploit the several forms of parallelism present in games and graphics." \cite{Sweeney2006}. Although the fields of games and agent-based simulations seem to be very different in the end, they have also very important similarities: both are simulations which perform numerical computations and update objects - in games they are called "game-objects" and in abm they are called agents but they are in fact the same thing - in a loop either concurrently or sequential. His key-points were:

\begin{itemize}
\item Dependent types as the remedy of most of the run-time failures.
\item Parallelism for numerical computation: these are pure functional algorithms, operate locally on mutable state. Haskell ST, STRef solution enables encapsulating local heaps and mutability within referentially transparent code.
\item Updating game-objects (agents) concurrently using STM: update all objects concurrently in arbitrary order, with each update wrapped in atomic block - depends on collisions if performance goes up.
\end{itemize}

\section{Background}

\subsection{Schelling Segregation}
We follow in our implementation the original paper of Schelling as in \cite{schelling_dynamic_1971} where we focus on the \textit{Area Distribution} section (Schelling starts with movement in a linear, 1-dimensional world where agents are able to move to the nearest point which meets the agents satisfaction but this is not what we follow here). One assumes a discrete 2-dimensional lattice-world with NxM fields. Each field is either occupied by an agent of a given color (e.g. Red or Green) or is free. Each field has 8 neighbours, which denotes a Moore-Neighbourhood. In Schellings original work the lattice-world is limited at its borders but we assume a torus world which is wrapped around in both the x- and y-dimensions resulting in 8 neighbours also for fields at the border. The occupation density was set by Schelling to be about 70\%-75\% which he identifies as being a setting which allows the agents to move around freely without making the lattice-world too sparse.
Now the agents make their move sequentially one after another. In each move an agent calculates the number of neighbours which are of equal color. If the number satisfies the agents needs about the neighbourhood then the agent is regarded as being 'happy' and will stay on this field. On the other hand the agent moves to the nearest unoccupied field which satisfies its needs. An agent which moves selects an unoccupied place randomly relative from its current place within a rectangle of side-length 2r where its current place is at the center. The interpretation for that behaviour is that agents won't move too far as it could be costly. Also children might attend a school in this area or the family has friends in this area, so they don't want to break that.



Agents just move depending on their movement-strategy to another place if they are not happy on the current one - they don't care how the target place is in the present or in the future, they will decide again in the next time-step. The interpretation for that behaviour is: agents want to 'just get out' at any cost, not caring what the future place will look like - it might be better or worse but they will see then.

\subsubsection{Optimizing behaviour}
TODO: define utility

The original schelling model didn't have a move-optimizing behaviour, meaning agents are just binary: if it is happy it will not move, if it is unhappy it will move but they won't care where they move. We introduce local move-optimizing behaviours which can be interpreted as being realistic in the real-world. It is important to note that we focus on \textit{local} instead of \textit{global} move-optimization: the agents are limited in their reasoning-capabilities and have limited information available: they cannot check out \textit{every} place and pick the globally best one.\\

\subsubsection{Anticipating behaviour}
Schelling explicitly mentions in \cite{schelling_dynamic_1971} that nobody anticipates moves of others. This is what we introduce using the recursive simulation.

TODO: is this optimizing behaviour in the spirit of schellings original work? 

\paragraph{Optimizing future} Agents pick an unoccupied random place and move to it if it increases their utility in the future. The interpretation for that behaviour is: agents heard about a place which will be cool in the future.

\paragraph{Optimizing present \& future} Agents pick an unoccupied random place and move to it if it increases their utility in the now and in the future. The interpretation for that behaviour is: agents heard about a cool spot in town, check it out and move to it if they like it but they also anticipate the coolness of the place in the future and if it seems that the place is going down then they won't move there.

\subsection{Related Research}
TODO: \cite{kirman_complex_2010} mention kirman complex economics where he investigates the model more in depth


\section{Implementing Reality as ABS}
TODO: in this section think about how reality could be implemented as an ABS

simulation of reality may be possible but that computation alone lacks an important ingredient: the spark of conciousness and free will which are inherently non-computable and thus non-constructive. 
Thus => we live in a construced world
But => we ourselves are non-constructive


what if our conciousness constantly observes us, thus creating ourselves continuously new in every moment, also thus realizing thoughts which pop up. these thoughts we can either adhere to or we can ignore them: this is free will.

free will on a machine is a contradiction. the machine works according to very strict rules. free will can be completely unpredictable. or is free will just an imagination? if one confronts a decision maker within short time with too much information then the outcome it is unpredictable 

pro-activity possible through conciousness: the brain produces thoughts and the conciousness can observe these and decide to follow them or not. This is observable on oneself during meditation!

Free will: deliberately ignore thoughts

\begin{itemize}
\item Who or what implemented the simulation?
\item What is outside this simulation?
\item What is free will in this context? Can it be defined formally?
\item On which hardware does this simulation run? Where does the energy come from?
\item What is the computational complexity of this simulation?
\item What are the memory-requirements of this simulation?
\end{itemize}

free will on a machine is a contradiction. the machine works according to very strict rules. free will can be completely unpredictable. or is free will just an imagination? if one confronts a decision maker within short time with too much information then the outcome it is unpredictable 

\subsection{Simulating Conciousness}
concious ABS: in an ABS some code representing an agent is executed in regular simulated time intervals. this can be seen as the agent 'thinks' itself: if it would not act/think/execute the code it would simply not exist. this is a fundamental concept which comes from the very heart of our reality which flows through to the ABS because the ABS is run on a computer: software exists only in execution. there exists data if software is stored somewhere but it is dead.

the fundamental question is now whether this is the case for a human being or not. put other way: do we exist if we do not think ourself? what is thinking of ourself?

here we can create a parallelism to the agent but put it on a higher level and view it from a different angle: we exist because we are concious. in this concious mode we experience our SELF, feel ourself. in this mode we are able to produce thoughts using our computing device (brain) which are translated to actions through our body.

the conciousness is the same as when an agent is run on a computer: it is allocated to the CPU and becomes alive. but the fundamental difference is that an agent is already executing computations when it is alive: there is no distinction between thinking and conciousness. we humans have the ability to rise ave thinking: to step outside our computation and view it from a metalevel. 

the question is if an agent is capable of doing this as well. there are two possible approaches: either we can simulate the conciouslevel on a cpu or we need a new kind of device for this. My intuition is that for deterministic simulations we can simulate it on CPU but for true conciousness and true artificial life we need a 'concious-generating device' for which i have no idea how to build it and is probably far ahead of its time

thus how can we simulate an agent with conciousness? we need to introduce the conciozs level which observes the thoughts which spring forward: the concious part is the meta-level from which the agent constantly creates itself by constantly thinking itself new 

\subsection{Self-Conciousness \& Free-Will}
self-conciousness: the ability to observe ones thoughts on a metalevel: more or less pronounced. this meta-observation allows to intervene. also origin and unfolding is then possible. thus one can observe oneself from an outer perspective \\

freely choose NOT to obey some impulse. requires self-conciousness \\

computers have neither and cant have neither. why? thus computers as we know them cant be source of true intelligence as they are not able to introspection, to self-reflection. \\
they don't have this ability because the have no ability to \textit{imagine or anticipate the outcome of their actions without actually computing them}

\subsection{Simulation}
We as humans constantly run simulations in our minds when thinking and perceiving the reality: we anticipate our actions, envision what we want to do,... all by \textit{simulating} them in our mind. This is probably the most powerful tool of our intelligence which separates us (probably) from the animal kingdom. This ability to simulate potential / future realities but also changes us, there is a feedback.  So in a case there are 2 levels: reality and the simulations of reality in our minds.
I claim that these simulations may be as real as the reality we are living in where "only" the mind in which the simulation runs differs: in the case of our humans it is ourself, in case of the reality we are situated in it is an entity we would like to call God.
Both spawn a reality which are bewohnt by entities. But were God allows the entities free will, we haven't managed to do that yet. I postulate some stage in human development where we are able to create simulations which are able to simulate all free will outcomes. Tthe entities in the simulation need free will, just as we do. For this to happen they need the ability to simulate their reality as well - this creates a cascade. But the whole point is that the free will and conciousness \textit{has always been there}, passed down from the initial \textit{first} simulation initiator - which we refer to God but which may be just a level on a range of infinite many levels. 

\subsection{Cascading Simulations}
At some point in the existence of a free-will intelligence, it starts to asking for the future. First using religion, then mathematics, then finally computer simulation. But the problem is that such a simulation is too weak to forecast the future because out of simple computation no free will is born. thus the solution of the free-will intelligence is to put itself in a simulation-environment as a seed of free will. this simulation will then play through every decision branches an thus be able to predict possible futures. because within such a simulation the same thing can AND WILL happen at some point, we arrive at cascading simulations within simulations. thus we are at one level of this cascade where our direct outer level is god.

\subsection{On parallel universes, existence as simulation, free will}
We cannot predict the future due to complex interaction of free will of Humankind. To predict it we would have to spawn a new universe running in parallel if a free-will choice occurs. Then again, maybe this is already the case and the whole existence is an extremely huge tree of parallel universes being created from each other and collapsing back into others or being completely determined. \\
The question is then: Where in this tree am I? And maybe time does only advance in discrete steps after a spawn/collapse? \\
When one looks at the existence as a simulation then one can say that it has become unstable because too many actors with free will and too many variables producing unforeseeable consequences. But then, can we make predictions about a simulation from within? Can we talk about the meaning and meta-workings of a system from within it? \\
We always try to treat reality as smooth and predictable without outliers but ignoring catastrophic events - this is what the book "Black Swan" says. My own point of view is that the problem is the way we do science: "we divide and put reality into small boxes of labels/categories and then pile them up, adding piles of theories describing it creating a mountain of unbearable complexity - just to be caught by surprise by the next catastrophic event no one could predict despite the overwhelming amount of complex theories. \\
What's the problem? Theories describe the past. Science needs to move on to the now letting go of the myriads of categories and look at it all as a single complex system/simulation - the world as a simulation, simulating the interaction of free will, allowing it to unfold and see the effects in all facets. \\
The question is whether "Black Swans" are an emergent system property coming from within the simulation or whether they are created from steering forces e.g. God.

\subsection{A magical approach as remedy of the dilemma}
Just as we try to manifest our thoughts and desires using magic we need devices which can do so with our thoughts in a structured way. Computers can be seen as a kind of attempt to achieve these devices but are not able to manifest real creational and metaphysical thoughts but only allow to execute formal models which can be mapped to a specific kind of symbol-manipulation. We need something more powerful: a magic computer. We need to learn how to think in its language but it will allow us to manifest thoughts in a virtual reality. \\
Thus we can say: Programming = Magic. It is a systematic altering reality and manifesting thoughts by encoding them in a systematic way in a system of symbols and rules how to change/apply them (=language).
[ ] we imagine something and then create it
[ ] its purely virtual
[ ] we are naming things
[ ] results can be unpredictable

\subsection{How can humankind survive?}
remove all ideologies
is it possible to live without an ideology?
love is the answer: it is more radical and allows for more change than anything else
free will without love ultimately leads to destruction. this would be the hypothesis of the simulation. 
but then again: what is love? it accepts all live as equal and same value with no right of one to judge and rule over another. even more: it also attributes this to live which kills the loving one

\section{Discussion}

\subsection{Other Models}
TODO: mention that we have also implemented other models, which also work without time-semantics (all agents make a move at discrete time-steps and do not really rely on some notion of time). 

\subsection{Time-Semantics}
The main reason for building our pure functional ABMS approach on top of Yampa was to leverage the powerful time-semantics of Yampa which allows us to implement important concepts of ABMS:

state-chart: agents are at all time of their life-cycle in one state and can switch between multiple states using transitions 
timed transitions: transition to another state/behaviour happens at a discrete time
rate transitions: transition happens with a given rate
message transition: transition upon receiving a given message 

\subsection{Agents as Signals}
Due to the underlying nature and motivation of Functional Reactive Programming (und im speziellen) Yampa, Agents can be seen as Signals which is generated and consumed by a Signal-Function which is the behaviour of an Agent. If an Agent does not change the OUTPUT-signal is constant, if the agent changes e.g. by sending a message, changing its state,... the OUTPUT signal changes. A dead agent has no signal at all.

\subsection{Time-Sampling}
sampling rate depends on the transition times \& rates of the model. when e.g. the contact rate is 5 then the sampling dt should be below 0.2

\subsection{System Dynamics}
can emulate system dynamics due to the parallel update-strategy and continuous time-flow semantics

\subsection{Discrete Event Simulation}
DES in FrABMS? how easily can we implement server/queue systems? do they also look like a specification? potential problem: ordering of messages is not guaranteed by now

\subsection{Advantages}
advantages:
	- no side-effects within agents leads to much safer code
	- edsl for time-semantics
	- declarative style: agent-implementation looks like a model-specification
	- reasoning and verification
	- sequential and parallel
	- powerful time-semantics
	- arrowized programming is optional and only required when utilizing yampas time-semantics. if the model does not rely on time-semantics, it can use monadic-programming by building on the existing monadic functions in the EDSL which allow to run in the State-Monad which simplifies things very much
	- when to use yampas arrowized programing: time-semantics, simple state-chart agents 
	- when not using yampas facilities: in all the other cases e.g. SugarScape is such a case as it proceeds in unit time-steps and all agents act in every time-step
	- can implement System Dynamics building on Yampas facilities with total ease	
	- get replications for free without having to worry about side-effects and can even run them in parallel without headaches
	- cant mess around with time because delta-time is hidden from you (intentional design-decision by Yampa). this would be only very difficult and cumbersome to achieve in an object-oriented approach. TODO: experiment with it in Java - how could we actually implement this? I think it is impossible: may only achieve this through complicated application of patterns and inheritance but then has the problem of how to update the dt and more important how to deal with functions like integral which accumulates a value through closures and continuations. We could do this in OO by having a general base-class e.g. ContinuousTime which provides functions like updateDt and integrate, but we could only accumulate a single integral value.
	- reproducibility statically guaranteed
	- cannot mess around with dt
	- code == specification
	- rule out serious class of bugs
	- different time-sampling leads to different results e.g. in wildfire \& SIR but not in Prisoners Dilemma. why? probabilistic time-sampling?
	- reasoning about equivalence between SD and ABS implementation in the same framework
	- recursive implementations
	
	- we can statically guarantee the reproducibility of the simulation because: no side effects possible within the agents which would result in differences between same runs (e.g. file access, networking, threading), also timedeltas are fixed and do not depend on rendering performance or userinput	
	
\subsection{Disadvantages}
disadvantages:
	- performance is low
	- reasoning about performance is very difficult
	- very steep learning curve for non-functional programmers
	- learning a new EDSL
	- think ABMS different: when to use async messages, when to use sync conversations


[ ] important: increasing sampling freqzency and increasing number of steps so that the same number of simulation steps are executed should lead to same results. but it doesnt. why?
[ ] hypothesis: if time-semantics are involved then event ordering becomes relevant for emergent patterns. there are no tine semantics in heroes and cowards but in the prisoners dilemma
[ ] can we implement different types of agents interacting with each other in the same simulation ? with different behaviour funcs, digferent state? yes, also not possible in NetLogo to my knowledge. but they must have the same messages, emvironment 

[ ] Hypothesis: we can combine with FrABS agent-based simulation and system dynamics (this has been proved by example!)

\section{Conclusions}
\label{sec:conclusions}

Our approach is radically different from traditional approaches in the ABS community. First it builds on the already quite powerful FRP paradigm. Second, due to our continuous time approach, it forces one to think properly of time-semantics of the model and how small $\Delta t$ should be. Third it requires to think about agent interactions in a new way instead of being just method-calls.

Because no part of the simulation runs in the IO Monad and we do not use unsafePerformIO we can rule out a serious class of bugs caused by implicit data-dependencies and side-effects which can occur in traditional imperative implementations.

Also we can statically guarantee the reproducibility of the simulation, which means that repeated runs with the same initial conditions are guaranteed to result in the same dynamics. Although we allow side-effects within agents, we restrict them to only the Random and State Monad in a controlled, deterministic way and never use the IO Monad which guarantees the absence of non-deterministic side effects within the agents and other parts of the simulation.

Determinism is also ensured by fixing the $\Delta t$ and not making it dependent on the performance of e.g. a rendering-loop or other system-dependent sources of non-determinism as described by \cite{perez_testing_2017}. Also by using FRP we gain all the benefits from it and can use research on testing, debugging and exploring FRP systems \cite{perez_testing_2017, perez_back_2017}.

\subsection*{Issues}
Currently, the performance of the system is not comparable to imperative implementations but our research was not focusing on this aspect. We leave the investigation and optimization of the performance aspect of our approach for further research.

Despite the strengths and benefits we get by leveraging on FRP, there are errors that are not raised at compile time, e.g. we can still have infinite loops and run-time errors. This was for example investigated in \cite{sculthorpe_safe_2009} where the authors use dependent types to avoid some run-time errors in FRP. We suggest that one could go further and develop a domain specific type system for FRP that makes the FRP based ABS more predictable and that would support further mathematical analysis of its properties. Furthermore, moving to dependent types would pose a unique benefit over the traditional object-oriented approach and should allow us to express and guarantee even more properties at compile time. We leave this for further research.

In our pure functional approach, agent identity is not as clear as in traditional object-oriented programming, where an agent can be hidden behind a polymorphic interface which is much more abstract than in our approach. Also the identity of an agent is much clearer in object-oriented programming due to the concept of object-identity and the encapsulation of data and methods.

We can conclude that the main difficulty of a pure functional approach evolves around the communication and interaction between agents, which is a direct consequence of the issue with agent identity. Agent interaction is straight-forward in object-oriented programming, where it is achieved using method-calls mutating the internal state of the agent, but that comes at the cost of a new class of bugs due to implicit data flow. In pure functional programming these data flows are explicit but our current approach of feeding back the states of all agents as inputs is not very general and we have added further mechanisms of agent interaction which we had to omit due to lack of space.

\bibliographystyle{acm}
\bibliography{../../../references/phdReferences.bib}

\end{document}