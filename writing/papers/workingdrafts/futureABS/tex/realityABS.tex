\section{Implementing Reality as ABS}
TODO: in this section think about how reality could be implemented as an ABS

simulation of reality may be possible but that computation alone lacks an important ingredient: the spark of conciousness and free will which are inherently non-computable and thus non-constructive. 
Thus => we live in a construced world
But => we ourselves are non-constructive


what if our conciousness constantly observes us, thus creating ourselves continuously new in every moment, also thus realizing thoughts which pop up. these thoughts we can either adhere to or we can ignore them: this is free will.

free will on a machine is a contradiction. the machine works according to very strict rules. free will can be completely unpredictable. or is free will just an imagination? if one confronts a decision maker within short time with too much information then the outcome it is unpredictable 

pro-activity possible through conciousness: the brain produces thoughts and the conciousness can observe these and decide to follow them or not. This is observable on oneself during meditation!

Free will: deliberately ignore thoughts

\begin{itemize}
\item Who or what implemented the simulation?
\item What is outside this simulation?
\item What is free will in this context? Can it be defined formally?
\item On which hardware does this simulation run? Where does the energy come from?
\item What is the computational complexity of this simulation?
\item What are the memory-requirements of this simulation?
\end{itemize}

free will on a machine is a contradiction. the machine works according to very strict rules. free will can be completely unpredictable. or is free will just an imagination? if one confronts a decision maker within short time with too much information then the outcome it is unpredictable 

\subsection{Simulating Conciousness}
concious ABS: in an ABS some code representing an agent is executed in regular simulated time intervals. this can be seen as the agent 'thinks' itself: if it would not act/think/execute the code it would simply not exist. this is a fundamental concept which comes from the very heart of our reality which flows through to the ABS because the ABS is run on a computer: software exists only in execution. there exists data if software is stored somewhere but it is dead.

the fundamental question is now whether this is the case for a human being or not. put other way: do we exist if we do not think ourself? what is thinking of ourself?

here we can create a parallelism to the agent but put it on a higher level and view it from a different angle: we exist because we are concious. in this concious mode we experience our SELF, feel ourself. in this mode we are able to produce thoughts using our computing device (brain) which are translated to actions through our body.

the conciousness is the same as when an agent is run on a computer: it is allocated to the CPU and becomes alive. but the fundamental difference is that an agent is already executing computations when it is alive: there is no distinction between thinking and conciousness. we humans have the ability to rise ave thinking: to step outside our computation and view it from a metalevel. 

the question is if an agent is capable of doing this as well. there are two possible approaches: either we can simulate the conciouslevel on a cpu or we need a new kind of device for this. My intuition is that for deterministic simulations we can simulate it on CPU but for true conciousness and true artificial life we need a 'concious-generating device' for which i have no idea how to build it and is probably far ahead of its time

thus how can we simulate an agent with conciousness? we need to introduce the conciozs level which observes the thoughts which spring forward: the concious part is the meta-level from which the agent constantly creates itself by constantly thinking itself new 

\subsection{Self-Conciousness \& Free-Will}
self-conciousness: the ability to observe ones thoughts on a metalevel: more or less pronounced. this meta-observation allows to intervene. also origin and unfolding is then possible. thus one can observe oneself from an outer perspective \\

freely choose NOT to obey some impulse. requires self-conciousness \\

computers have neither and cant have neither. why? thus computers as we know them cant be source of true intelligence as they are not able to introspection, to self-reflection. \\
they don't have this ability because the have no ability to \textit{imagine or anticipate the outcome of their actions without actually computing them}

\subsection{Simulation}
We as humans constantly run simulations in our minds when thinking and perceiving the reality: we anticipate our actions, envision what we want to do,... all by \textit{simulating} them in our mind. This is probably the most powerful tool of our intelligence which separates us (probably) from the animal kingdom. This ability to simulate potential / future realities but also changes us, there is a feedback.  So in a case there are 2 levels: reality and the simulations of reality in our minds.
I claim that these simulations may be as real as the reality we are living in where "only" the mind in which the simulation runs differs: in the case of our humans it is ourself, in case of the reality we are situated in it is an entity we would like to call God.
Both spawn a reality which are bewohnt by entities. But were God allows the entities free will, we haven't managed to do that yet. I postulate some stage in human development where we are able to create simulations which are able to simulate all free will outcomes. Tthe entities in the simulation need free will, just as we do. For this to happen they need the ability to simulate their reality as well - this creates a cascade. But the whole point is that the free will and conciousness \textit{has always been there}, passed down from the initial \textit{first} simulation initiator - which we refer to God but which may be just a level on a range of infinite many levels. 

\subsection{Cascading Simulations}
At some point in the existence of a free-will intelligence, it starts to asking for the future. First using religion, then mathematics, then finally computer simulation. But the problem is that such a simulation is too weak to forecast the future because out of simple computation no free will is born. thus the solution of the free-will intelligence is to put itself in a simulation-environment as a seed of free will. this simulation will then play through every decision branches an thus be able to predict possible futures. because within such a simulation the same thing can AND WILL happen at some point, we arrive at cascading simulations within simulations. thus we are at one level of this cascade where our direct outer level is god.

\subsection{On parallel universes, existence as simulation, free will}
We cannot predict the future due to complex interaction of free will of Humankind. To predict it we would have to spawn a new universe running in parallel if a free-will choice occurs. Then again, maybe this is already the case and the whole existence is an extremely huge tree of parallel universes being created from each other and collapsing back into others or being completely determined. \\
The question is then: Where in this tree am I? And maybe time does only advance in discrete steps after a spawn/collapse? \\
When one looks at the existence as a simulation then one can say that it has become unstable because too many actors with free will and too many variables producing unforeseeable consequences. But then, can we make predictions about a simulation from within? Can we talk about the meaning and meta-workings of a system from within it? \\
We always try to treat reality as smooth and predictable without outliers but ignoring catastrophic events - this is what the book "Black Swan" says. My own point of view is that the problem is the way we do science: "we divide and put reality into small boxes of labels/categories and then pile them up, adding piles of theories describing it creating a mountain of unbearable complexity - just to be caught by surprise by the next catastrophic event no one could predict despite the overwhelming amount of complex theories. \\
What's the problem? Theories describe the past. Science needs to move on to the now letting go of the myriads of categories and look at it all as a single complex system/simulation - the world as a simulation, simulating the interaction of free will, allowing it to unfold and see the effects in all facets. \\
The question is whether "Black Swans" are an emergent system property coming from within the simulation or whether they are created from steering forces e.g. God.

\subsection{A magical approach as remedy of the dilemma}
Just as we try to manifest our thoughts and desires using magic we need devices which can do so with our thoughts in a structured way. Computers can be seen as a kind of attempt to achieve these devices but are not able to manifest real creational and metaphysical thoughts but only allow to execute formal models which can be mapped to a specific kind of symbol-manipulation. We need something more powerful: a magic computer. We need to learn how to think in its language but it will allow us to manifest thoughts in a virtual reality. \\
Thus we can say: Programming = Magic. It is a systematic altering reality and manifesting thoughts by encoding them in a systematic way in a system of symbols and rules how to change/apply them (=language).
[ ] we imagine something and then create it
[ ] its purely virtual
[ ] we are naming things
[ ] results can be unpredictable

\subsection{How can humankind survive?}
remove all ideologies
is it possible to live without an ideology?
love is the answer: it is more radical and allows for more change than anything else
free will without love ultimately leads to destruction. this would be the hypothesis of the simulation. 
but then again: what is love? it accepts all live as equal and same value with no right of one to judge and rule over another. even more: it also attributes this to live which kills the loving one