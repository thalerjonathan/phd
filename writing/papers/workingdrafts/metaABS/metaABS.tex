%%%%%%%%%%%%%%%%%%%%%%%%%%%%%%%%%%%%%%%%%
% University/School Laboratory Report
% LaTeX Template
% Version 3.1 (25/3/14)
%
% This template has been downloaded from:
% http://www.LaTeXTemplates.com
%
% Original author:
% Linux and Unix Users Group at Virginia Tech Wiki 
% (https://vtluug.org/wiki/Example_LaTeX_chem_lab_report)
%
% License:
% CC BY-NC-SA 3.0 (http://creativecommons.org/licenses/by-nc-sa/3.0/)
%
%%%%%%%%%%%%%%%%%%%%%%%%%%%%%%%%%%%%%%%%%

%----------------------------------------------------------------------------------------
%	PACKAGES AND DOCUMENT CONFIGURATIONS
%----------------------------------------------------------------------------------------


\documentclass[twocolumn]{article}

\usepackage[utf8]{inputenc}
\usepackage{graphicx} % Required for the inclusion of images
\usepackage{amsmath} % Required for some math elements 
\usepackage{glossaries}
\usepackage[toc,page]{appendix}
\usepackage[autostyle=true]{csquotes}
\usepackage{hyperref}
\usepackage{amssymb}
\usepackage{caption} 
\usepackage{hhline}
\usepackage{float}
\usepackage{listings}

\setlength\parindent{0pt} % Removes all indentation from paragraphs

%\usepackage{times} % Uncomment to use the Times New Roman font

%----------------------------------------------------------------------------------------
%	DOCUMENT INFORMATION
%----------------------------------------------------------------------------------------

\title{Meta-ABS\\Recursive Agent-Based Simulation} % Title

\author{Jonathan \textsc{Thaler}} % Author name

\date{\today} % Date for the report

\begin{document}

\maketitle % Insert the title, author and date

% If you wish to include an abstract, uncomment the lines below
\begin{abstract}
In this paper we ask what influence recursive Agent-Based Simulation has on the dynamics of a simulation. We investigate the famous Schelling Segregation and implement our agents with the ability to anticipate their actions by recursively running simulations. Based on the outcomes of the recursions they are then able to determine whether their move increases their utility in the future or not.
We investigate the dynamics of the MetaABS implementation and compare it to the movement-strategy of the original model. We hypothesize that in the case of a deterministic future this approach allows the agents to increase their utility as a group but we hypothesize that this is not the case when the future is non-deterministic as the power to predict is simply lost in this case.
Also we show that alone by looking at the implementation we can raise interesting philosophical questions about agents, anticipation, information, determinism.
The main contribution of this paper is the introduction of recursive agent-based simulation, a completely new method in ABS, which we termed MetaABS.
\end{abstract}

%*******************************************************************************
%*********************************** First Chapter *****************************
%*******************************************************************************

\chapter{Introduction}  %Title of the First Chapter
I noticed that it is pretty hard to convince an agent-based economics specialist who is not a computer scientist about a pure functional approach. My conjecture is that the implementation technique and method does not matter much to them because they have very little knowledge about programming and are almost always self-taught - they don't know about software-engineering, nothing about proper software-design and architecture, nothing about software-maintenance, nothing about unit-testing,... In the end they just "hack" the simulation in whatever language they are able to: C++, Visual Basic, Java or toolboxes like Netlogo. For them it is all about to \textit{get things done somehow} and not to get things done the right way or in a beautiful way - the way and the method doesn't matter, its just a necessary evil which needs to be done. Thus if functional programming could make their lives easier, then they will definitely welcome it. But functional programming is, i think, harder to learn and harder to understand - so one needs to provide an abstraction through EDSL. So I REALLY need to come up with convincing arguments why to use pure functional approaches in ACE THEY can understand, otherwise I will be lost and not heard (not published,...). \\

What ACE economists care for:

\begin{itemize}
\item Very: Qualitative modelling with quantitative results
\item Yes: Easy reproducibility
\item Likely: Reasoning about convergence?
\item Likely: EDSL
\end{itemize}

My contributions are: pure functional framework, functional agent-model for market-simulations, EDSL for market-simulations, qualitative / implicit modelling with quanitative results, reasoning in my framework about convergence \\

IDEA: could I develop non-causal modelling (models are expressed in terms of non-directed equations, modelled in signal-relations) to allow for qualitative modelling for the agent-based economists? See hybrid modelling paper of Yampa. \textbf{THIS WOULD BE A HUGE NOVEL CONTRIBUTION TO ACE ESPECIALLY WHEN COMBINED WITH AN EDSL AND PROVIDING FULL REFERENTIAL TRANSPARENCY TO KEEP THE ABILITY TO REASON ABOUT CONVERGENCE}. This should be covered in the "EDSL"-paper.

TODO: maybe i should really focus only on market models? otherwise too much? \\

central novelty of my PhD: model specification = runnable code. possible through EDSL. but only in specific subfield of ACE: market-models. need a functional description of the model, then translate it to model specification in EDSL and then run it to see dynamics. But: model specification moves closer to functional programming languages. \\

another novelty approach: model specification through qualitative instead of quantiative approaches. is this possible? \\

WHY FUNCTIONAL? "because its the ultimate approach to scientific computing": fewer bugs due to mutable state (why? is thos shown obkectively by someone?), shorter (again as above, productivity), more expressive and closer to math, EDSL, EDSL=model=simulation, better parallelising due to referental transparency, reasoning \\

scientific results need to be reproduced, especially when they have high impact. a more formal approach of specifying the model and the simulation (model=simulation) could lead to easier sharing and easier reporduction without ambigouites \\

pure functional agent-model \& theory, EDSL framework in Haskell for ACE

\begin{enumerate}
\item Which kind of problem do we have?
\item What aim is there? Solving the problem? 
\item How the aim is achieved by enumerating VERY CLEAR objectives.
\item What the impact one expects (hypothesis) and what it is (after results).
\end{enumerate}

Note: It is not in the interest of the researcher to develop new economic theories but to research the use of functional methods (programming and specification) in agent-based computational economics (ACE).

NOTE: Get the reader’s attention early in the introduction: motivation, significance, originality and novelty.

\section{Methods}
Methods need to be selected to implement the simulations. Special emphasis will be put on functional ones which will then be compared to established methods in the field of ABM/S and ACE. \\

Claim: non-programming environments are considered to be not powerful enough to capture the complexity of ACE implementations thus a programming approach to ACE will be always required.

\section{Scenarios}
To apply and test functional methods in ACE, four scenarios of ACE are selected and then the methods applied and compared with each other to see how each of them perform in comparison. The 4 selected scenarios represent a selection of the challenges posed in ACE: from very abstract ones to very operational ones.

\section{Comparison}
Each of the selected scenarios is then implemented using the selected methods where each solution is then compared against the following criteria: 

\begin{enumerate}
\item suitability for scientific computation
\item robustness
\item error-sources
\item testability
\item stability
\item extendability
\item size of code
\item maintainability
\item time taken for development
\item verification \& correctness
\item replications \& parallelism
\item EDSL
\end{enumerate}

This will then allow to compare the different methods against each other and to show under which circumstances functional methods shine and when they should not be used.

\section{Agent-Based Modelling and Simulation (ABM/S)}
ABM/S is a method of modelling and simulating a system where the global behaviour may be unknown but the behaviour and interactions of the parts making up the system is of knowledge (Wooldrige, M. (2009). An Introduction to MultiAgent Systems. John Wiley & Sons). Those parts, called agents, are modelled and simulated out of which then the aggregate global behaviour of the whole system emerges. Thus the central aspect of ABM/S is the concept of an Agent which can be understood as a metaphor for a pro-active unit, able to spawn new Agents, and interacting with other Agents in a network of neighbours by exchange of messages. The implementation of Agents can vary and strongly depends on the programming language and the kind of domain the simulation and model is situated in.

\section{Agent-Based Economics (ACE)}
According to Leigh Tesfatsion (Tesfatsion, L. (2006). Agent-based computational economics: A constructive approach to economic theory. In Tesfatsion, L. and Judd, K. L., editors, Handbook of Computational Economics, volume 2, chapter 16, pages 831–880. Elsevier, 1 edition.), one of the leading figures, ACE is "[...] computational modelling of economic processes (including whole economies) as open-ended dynamic systems of interacting agents." - thus lending perfectly to the use of ABM/S as already the name suggests. Whereas classical economic models fall short by only looking at the average, pure rational, individual interacting in anonymous markets, the ACE approach looks at heterogeneous, non-rational individuals interacting with each other in networks (Kirman, A. (2010). Complex Economics: Individual and Collective Rationality. Routledge, London ; New York, NY.). Thus ACE can be understood as a combination of computer-science, cognitive/social science and evolutionary economics.

\section{Functional programming}
TODO: read \cite{Backus1978}

The state-of-the-art approach to implementing Agents are object-oriented methods and programming as the metaphor of an Agent as presented above lends itself very naturally to object-orientation (OO). The author of this thesis claims that OO in the hands of inexperienced or ignorant programmers is dangerous, leading to bugs and hardly maintainable and extensible code. The reason for this is that OO provides very powerful techniques of organising and structuring programs through Classes, Type Hierarchies and Objects, which, when misused, lead to the above mentioned problems. Also major problems, which experts face as well as beginners are 1. state is highly scattered across the program which disguises the flow of data in complex simulations and 2. objects don’t compose as well as functions. The reason for this is that objects always carry around some internal state which makes it obviously much more complicated as complex dependencies can be introduced according to the internal state.
All this is tackled by (pure) functional programming which abandons the concept of global state, Objects and Classes and makes data-flow explicit. This then allows to reason about correctness, termination and other properties of the program e.g. if a given function exhibits side-effects or not. Other benefits are fewer lines of code, easier maintainability and ultimately fewer bugs thus making functional programming the ideal choice for scientific computing and simulation and thus also for ACE. A very powerful feature of functional programming is Lazy evaluation. It allows to describe infinite data-structures and functions producing an infinite stream of output but which are only computed as currently needed. Thus the decision of how many is decoupled from how to (Hughes, J. (1989). Why functional programming matters. Comput. J., 32(2):98–107.).
The most powerful aspect using pure functional programming however is that it allows the design of embedded domain specific languages (EDSL). In this case one develops and programs primitives e.g. types and functions in a host language (embed) in a way that they can be combined. The combination of these primitives then looks like a language specific to a given domain, in the case of this thesis ACE. The ease of development of EDSLs in pure functional programming is also a proof of the superior extensibility and composability of pure functional languages over OO (Henderson P. (1982). Functional Geometry. Proceedings of the 1982 ACM Symposium on LISP and Functional Programming.).
One of the most compelling example to utilize pure functional programming is the reporting of Hudak (Hudak P., Jones M. (1994). Haskell vs. Ada vs. C++ vs. Awk vs. ... An Experiment in Software Prototyping Productivity. Department of Computer Science, Yale University.)  where in a prototyping contest of DARPA the Haskell prototype was by far the shortest with 85 lines of code. Also the Jury mistook the code as specification because the prototype did actually implement a small EDSL which is a perfect proof how close EDSL can get to and look like a specification.

Functional languages can best be characterized by their way computation works: instead of \textit{how} something is computed, \textit{what} is computed is described. Thus functional programming follows a declarative instead of an imperative style of programming. The key points are:
\begin{itemize}
\item No assignment statements - variables values can never change once given a value.
\item Function calls have no side-effect and will only compute the results - this makes order of execution irrelevant, as due to the lack of side-effects the logical point in \textit{time} when the function is calculated within the program-execution does not matter.
\item higher-order functions
\item lazy evaluation
\item Looping is achieved using recursion, mostly through the use of the general fold or the more specific map.
\item Pattern-matching
\end{itemize}

This alone does not really explain the \textit{real} advantages of functional programming and one must look for better motivations using functional programming languages. One motivation is given in \cite{Hughes1989} which is a great paper explaining to non-functional programmers what the significance of functional programming is and helping functional programmers putting functional languages to maximum use by showing the real power and advantages of functional languages. The main conclusion is that \textit{modularity}, which is the key to successful programming, can be achieved best using higher-order functions and lazy evaluation provided in functional languages like Haskell. \cite{Hughes1989} argues that the ability to divide problems into sub-problems depends on the ability to glue the sub-problems together which depends strongly on the programming-language and \cite{Hughes1989} argues that in this ability functional languages are superior to structured programming.

TODO: comparison of functional and object-oriented programming. My points are:
\begin{itemize}
\item The way state can be changed and treated - distributed over multiple objects - is often very difficult to understand.
\item Inheritance is a dangerous thing if not used with care because inheritance introduces very strong dependencies which cannot be changed during runtime anymore.
\item Objects don't compose very well: \url{http://zeroturnaround.com/rebellabs/why-the-debate-on-object-oriented-vs-functional-programming-is-all-about-composition/}
\item (Nearly) impossible to reason about programs
\end{itemize}

In conclusion the upsides of functional programming as opposed to OO are:
\begin{itemize}
\item Much more explicit flow of data \& control
\item Much better compose-able
\item Much better parallelism
\end{itemize}

\section{Related Research}
Tim Sweeney, CTO of Epic Games gave an invited talk about how "future programming languages could help us write better code" by "supplying stronger typing, reduce run-time failures;  and the need for pervasive concurrency support, both implicit and explicit, to effectively exploit the several forms of parallelism present in games and graphics." \cite{Sweeney2006}. Although the fields of games and agent-based simulations seem to be very different in the end, they have also very important similarities: both are simulations which perform numerical computations and update objects - in games they are called "game-objects" and in abm they are called agents but they are in fact the same thing - in a loop either concurrently or sequential. His key-points were:

\begin{itemize}
\item Dependent types as the remedy of most of the run-time failures.
\item Parallelism for numerical computation: these are pure functional algorithms, operate locally on mutable state. Haskell ST, STRef solution enables encapsulating local heaps and mutability within referentially transparent code.
\item Updating game-objects (agents) concurrently using STM: update all objects concurrently in arbitrary order, with each update wrapped in atomic block - depends on collisions if performance goes up.
\end{itemize}

\section{Background}

\subsection{Schelling Segregation}
We follow in our implementation the original paper of Schelling as in \cite{schelling_dynamic_1971} where we focus on the \textit{Area Distribution} section (Schelling starts with movement in a linear, 1-dimensional world where agents are able to move to the nearest point which meets the agents satisfaction but this is not what we follow here). One assumes a discrete 2-dimensional lattice-world with NxM fields. Each field is either occupied by an agent of a given color (e.g. Red or Green) or is free. Each field has 8 neighbours, which denotes a Moore-Neighbourhood. In Schellings original work the lattice-world is limited at its borders but we assume a torus world which is wrapped around in both the x- and y-dimensions resulting in 8 neighbours also for fields at the border. The occupation density was set by Schelling to be about 70\%-75\% which he identifies as being a setting which allows the agents to move around freely without making the lattice-world too sparse.
Now the agents make their move sequentially one after another. In each move an agent calculates the number of neighbours which are of equal color. If the number satisfies the agents needs about the neighbourhood then the agent is regarded as being 'happy' and will stay on this field. On the other hand the agent moves to the nearest unoccupied field which satisfies its needs. An agent which moves selects an unoccupied place randomly relative from its current place within a rectangle of side-length 2r where its current place is at the center. The interpretation for that behaviour is that agents won't move too far as it could be costly. Also children might attend a school in this area or the family has friends in this area, so they don't want to break that.



Agents just move depending on their movement-strategy to another place if they are not happy on the current one - they don't care how the target place is in the present or in the future, they will decide again in the next time-step. The interpretation for that behaviour is: agents want to 'just get out' at any cost, not caring what the future place will look like - it might be better or worse but they will see then.

\subsubsection{Optimizing behaviour}
TODO: define utility

The original schelling model didn't have a move-optimizing behaviour, meaning agents are just binary: if it is happy it will not move, if it is unhappy it will move but they won't care where they move. We introduce local move-optimizing behaviours which can be interpreted as being realistic in the real-world. It is important to note that we focus on \textit{local} instead of \textit{global} move-optimization: the agents are limited in their reasoning-capabilities and have limited information available: they cannot check out \textit{every} place and pick the globally best one.\\

\subsubsection{Anticipating behaviour}
Schelling explicitly mentions in \cite{schelling_dynamic_1971} that nobody anticipates moves of others. This is what we introduce using the recursive simulation.

TODO: is this optimizing behaviour in the spirit of schellings original work? 

\paragraph{Optimizing future} Agents pick an unoccupied random place and move to it if it increases their utility in the future. The interpretation for that behaviour is: agents heard about a place which will be cool in the future.

\paragraph{Optimizing present \& future} Agents pick an unoccupied random place and move to it if it increases their utility in the now and in the future. The interpretation for that behaviour is: agents heard about a cool spot in town, check it out and move to it if they like it but they also anticipate the coolness of the place in the future and if it seems that the place is going down then they won't move there.

\subsection{Related Research}
TODO: \cite{kirman_complex_2010} mention kirman complex economics where he investigates the model more in depth


\section{Meta ABS}
Informally, Meta-ABS can be understood as giving the agents the ability to project the outcome of their actions into the future. They are able to halt time and 'play through' an arbitrary number of actions, compare their outcome and then to resume time and continue with a specifically chosen action e.g. the best performing or the one in which they haven't died. 
More precisely, what we want is to give an agent the ability to run the simulation recursively a number of times where the this number is not determined initially but can depend on the outcome of the recursive simulation. 

\subsection{Functional description}
In this section we will give a formal description of how Meta-ABS works. Although we look at it from a programming-language agnostic way, we follow a functional description (rather than object-oriented) both in our pseudo-code and description because we think it allows for a much clearer formalization.

First we need to establish a little bit of terminology to be able to unambiguously discuss the formal approach of Meta-ABS
TODO: recursion-depth:
TODO: recursion-replications
TODO: recursion-length

TODO: what is the difference between depth and length?

There are a few serious pitfalls here: 
\begin{enumeration}
	\item When an agent is running a recursion, then we need to restrict the other agents  otherwise we will end up in an infinite regress.
\end{enumeration}



\subsection{Deterministic vs. Non-Deterministic future}
The model as described in Background section is completely deterministic once it is running because it makes no use of a random-number generator and there are no other sources of non-determinism - the next move of an agent is always completely predictable. If we introduce randomness through a random-number generator into our model then the future becomes non-deterministic \textit{if the state of random-number generator when running recursive simulations is different from when the simulation is run non-recursively.}

TODO: What if the agents are shuffled every time before being traversed sequentially? The deterministic iteration is of importance here!

\subsection{Computational complexity}
the computation power grows exponentially with the number of recursion: give a formula depending on number of agents, recursion depth, independent moves of an agent and number of time-steps 
problem: need to escape infinite regress by preventing simulated 'other' agents to simulate themselves: what would be the outcome in a zeno machine/accelerated turing machine?

we are spanning up 3 dimensions: recursion-depth, replications, and time-steps

\subsection{Philosophical implications}

\subsubsection{Omega Point}
tiplers omega point and paper about god and the simulation argument
accelerating turing machine: finishes after 1 time steps

\subsubsection{Emergent Non-Determinism}
the prediction may work for a single agent but what if more and more agents predict their future? within the prediction no recursion is run so no 2nd level anticipation. 
hypothesis: increasing the ratio of predicting agents will decrease the effectiveness of the predictions because the future becomes then in effect non-deterministic => non-determinism as emerging property? is there a limit e.g. up until which ratio does the average utility of the predicting agents increase?

the agent who is initiating the recursion can be seen as 'knowing' that it is running inside a simulation, but the other agents are not able to distinguish between them running on the base level of the simulation or on a recursive level

\subsubsection{Perfect Information}
The main problem of our approach is that, depending on ones view-point, it is violating the principles of locality of information and limit of computing power. To recursively run the simulation the agent which initiates the recursion is feeding in all the states of the other agents and calculates the outcome of potentially multiple of its own steps, each potentially multiple recursion-layers deep and each recursion-layer multiple time-steps long. Both requires that each agent has perfect information about the complete simulation \textit{and} can compute these 3-dimensional recursions, which scale exponentially.
In the social sciences where agents are often designed to have only very local information and perform low-cost computations it is very difficult or impossible to motivate the usage of recursive simulations - it simply does not match the assumptions of the real world, the social sciences want to model.
In general simulations, with no direct link to the real world, where it is much more commonly accepted to assume perfect information and potentially infinite amount of computing power this approach is easily motivated by a constructive argument: it is possible to build, thus we build it.
What we are ultimately interested in is the influence on the dynamics.
Note that we identified the future-optimization technique as being locally. This is still the case despite of using global information for recurring the simulation - the reason for this is that we are talking about two different contexts here.



\section{Agent-Based Dynamics}
We can now run simulations of our agent-based approach and see whether they reach the SD dynamics of Figure \ref{fig:sir_sd_dynamics}. In Figure \ref{fig:sir_abs_approximating_1dt} the dynamics of a first naive attempt using 1,000 agents with $\Delta t= 1.0$ can be seen. 

\begin{figure}
\begin{center}
	\begin{tabular}{c c}
		\begin{subfigure}[b]{0.3\textwidth}
			\centering
			\includegraphics[width=1\textwidth, angle=0]{./../shared/fig/frabs/SIR_1000agents_150t_1dt_NOSS_parallel.png}
			\caption{$\Delta t = 1.0$}
			\label{fig:sir_abs_approximating_1dt}
		\end{subfigure}
    	&
		\begin{subfigure}[b]{0.3\textwidth}
			\centering
			\includegraphics[width=1\textwidth, angle=0]{./../shared/fig/frabs/SIR_1000agents_150t_05dt_NOSS_parallel.png}
			\caption{$\Delta t = 0.5$}
			\label{fig:sir_abs_approximating_05dt}
		\end{subfigure}
    	
    	\\
    	
		\begin{subfigure}[b]{0.3\textwidth}
			\centering
			\includegraphics[width=1\textwidth, angle=0]{./../shared/fig/frabs/SIR_1000agents_150t_02dt_NOSS_parallel.png}
			\caption{$\Delta t = 0.2$}
			\label{fig:sir_abs_approximating_02dt}
		\end{subfigure}
		& 
		\begin{subfigure}[b]{0.3\textwidth}
			\centering
			\includegraphics[width=1\textwidth, angle=0]{./../shared/fig/frabs/SIR_1000agents_150t_01dt_NOSS_parallel.png}
			\caption{$\Delta t = 0.1$}
			\label{fig:sir_abs_approximating_01dt}
		\end{subfigure}
	\end{tabular}
	
	\caption{Naive simulation of SIR using agent-based approach. Population Size $N$ = 1,000, contact rate $\beta = \frac{1}{5}$, infection probability $\gamma = 0.05$, illness duration $\delta = 15$ with initially 1 infected agent. Simulation run for 150 time-steps with various $\Delta t$.} 
	\label{fig:sir_abs_dynamics_naive}
\end{center}
\end{figure}

%TODO: reproducing about the same dynamics of the SD-solution (1.0 dt)
%	- super-sampling: 	contact-rate ss high, illness time-out low 
%	- agent number:		1000 vs. 10.000 agents
%	- Susceptibles making contact and infected response VS. only Infected make contact
%	- update-strat:		Sequential vs. Parallel
%	- making contact: susceptible only vs. susceptible AND infected
%	- do conversations make a difference?
%	- does a delayed switch (dSwitch) in transitions makes a difference?

Clearly something is going wrong as the dynamics do not resemble the ones of SD in any way with only 10 agents making the transition to infected to recovered. The problem is that we are running into sampling issues. TODO: explain deeper and better

\subsection{Sampling the System}
When sampling the system, the correct $\Delta t$ must be selected which depends on the highest frequency which occurs in a time-reactive function in the whole system. For example in the SIR model we want infected agents to make on average contact with $\beta = 5$ other agents per time-unit, which means with a frequency of $\frac{1}{5}$. This functionality is built on Yampas function \textit{occasionally} which behaviour we investigated under differing $\Delta t$ with the above frequency. In this investigation we simply sampled occasionally with different $\Delta t$ for a duration of $t = 1,000$ and the event-frequency of $\frac{1}{5}$. The result can be seen in Figure \ref{fig:sampling_occasionally_5evts} and is quite striking. The plot clearly shows that occasionally needs a quite high sampling frequency even for a comparatively low event-frequency, which becomes of course worse for higher event-frequencies.

The other time-reactive function which occurs in the SIR model is the timed transition from infected to recovered which occurs on average with an exponential random-distribution after $\delta = 15$. This functionality is built on a custom implementation of Yampas \textit{after} which creates an event after a time-out of the passed in time-duration drawn from an exponential random-distribution. Clearly this function has different semantics as although it also continuously emit events over time - \textit{NoEvent} before the time was hit, and \textit{Event b} after the time hit - the relevant point is that it switches to Event at some discrete point in time. This is implemented as simply adding up the $\Delta t$ until the accumulator is greater of equal than the previously drawn exponential time-out. We also investigated the behaviour of this function under varying $\Delta t$ using a time-out of $\delta = 15$. Our approach was to sample the \textit{afterExp} until an event occurs and then see when it has occurred. We run this with 10,000 replications with different random-number seeds and average the resulting times. The results can be seen in Figure \ref{fig:sampling_afterExp_5time}. The result is striking in another way: this function seems to be pretty invariant to the time-deltas, for obvious reasons: we are basically just interested in the "after"-condition of the whole semantics whereas in occasionally we are interested in the "repeatedly"-conditions. If we under-sample the \textit{afterExp} then we can be off by one $\Delta t$. If we under-sample \textit{occasionally} we keep loosing events - the less difference between $\Delta t$ and event-frequency, the more events we lose. Of course \textit{afterExp} can also be used for very short time-outs e.g. $\frac{1}{5}$. We have investigated the behaviour of this function for various $\Delta t$ as well as seen in Figure \ref{fig:sampling_afterExp_02time}. Here the result is much more striking and shows that \textit{afterExp} is vulnerable to small time-outs as well as \textit{occasionally}.  
To show that \textit{occasionally} is not vulnerable to very low frequencies of e.g. one event every 5 time-steps we plotted the behaviour of this under varying time-steps in Figure \ref{fig:sampling_occasionally_02evts}. The result shows that for low frequencies occasionally works fine with larger $\Delta t$.

\begin{figure}
\begin{center}
	\begin{tabular}{c c}
	\begin{subfigure}[b]{0.5\textwidth}
			\centering
			\includegraphics[width=.6\textwidth, angle=0]{./../shared/fig/sampling/samplingTest_occasionally_5evts.png}
			\caption{Sampling \textit{occasional} with a frequency of $\frac{1}{5}$ (average of 5 events per time-unit). The theoretical average is 5000 events within this time-frame.}
			\label{fig:sampling_occasionally_5evts}
		\end{subfigure}
		& 
		\begin{subfigure}[b]{0.5\textwidth}
			\centering
			\includegraphics[width=.6\textwidth, angle=0]{./../shared/fig/sampling/samplingTest_occasionally_02evts.png}
			\caption{Sampling \textit{occasional} with a frequency of 5 (average of 0.2 events per time-unit). The theoretical average is 200 events within this time-frame.}
			\label{fig:sampling_occasionally_02evts}
		\end{subfigure}
		
		\\
		
		\begin{subfigure}[b]{0.5\textwidth}
			\centering
			\includegraphics[width=.6\textwidth, angle=0]{./../shared/fig/sampling/samplingTest_afterExp_5time.png}
			\caption{Sampling \textit{afterExp} with an average time-out of 5.}
			\label{fig:sampling_afterExp_5time}
		\end{subfigure}
		& 
		\begin{subfigure}[b]{0.5\textwidth}
			\centering
			\includegraphics[width=.6\textwidth, angle=0]{./../shared/fig/sampling/samplingTest_afterExp_02time.png}
			\caption{Sampling \textit{afterExp} with an average time-out of 0.2.}
			\label{fig:sampling_afterExp_02time}
		\end{subfigure}
	\end{tabular}
	
	\caption{Sampling the \textit{afterExp} and \textit{occasional} functions to visualise the influence of sampling frequencies on the occurrence of the respective events. $\Delta t$ are [ 5, 2, 1, $\frac{1}{2}$, $\frac{1}{5}$, $\frac{1}{10}$, $\frac{1}{20}$, $\frac{1}{50}$, $\frac{1}{100}$ ]. The experiments for \textit{afterExp} used 10,000 replications. The experiments for \textit{occasional} ran for $t= 1,000$ with 100 replications.} 
	\label{fig:sampling_tests}
\end{center}
\end{figure}

Using these observation we run simulations with varying $\Delta t$ with $\Delta = 0.5$, $\Delta = 0.2$ and $\Delta = 0.1$ with the results visible in Figures \ref{fig:sir_abs_approximating_05dt}, \ref{fig:sir_abs_approximating_02dt} and \ref{fig:sir_abs_approximating_01dt} but still when decreasing $\Delta t$ we don't approach the SD dynamics. As previously mentioned the agent-based approach is a discrete one which means that with increasing number of agents, the discrete dynamics approximate the continuous dynamics of the SD simulation. We run further simulations with $\Delta = 0.1$ but with varying agent numbers to see the influence with the results seen in Figure \ref{fig:sir_abs_approximating}.

\begin{figure}
\begin{center}
	\begin{tabular}{c c}
		\begin{subfigure}[b]{0.3\textwidth}
			\centering
			\includegraphics[width=1\textwidth, angle=0]{./../shared/fig/frabs/SIR_100agents_150t_01dt_NOSS_parallel.png}
			\caption{100 Agents}
			\label{fig:sir_abs_approximating_100}
		\end{subfigure}
    	&
		\begin{subfigure}[b]{0.3\textwidth}
			\centering
			\includegraphics[width=1\textwidth, angle=0]{./../shared/fig/frabs/SIR_1000agents_150t_01dt_NOSS_parallel.png}
			\caption{1,000 Agents}
			\label{fig:sir_abs_approximating_1000}
		\end{subfigure}
    	
    	\\
    	
		\begin{subfigure}[b]{0.3\textwidth}
			\centering
			\includegraphics[width=1\textwidth, angle=0]{./../shared/fig/frabs/SIR_5000agents_150t_01dt_NOSS_parallel.png}
			\caption{5,000 Agents}
			\label{fig:sir_abs_approximating_5000}
		\end{subfigure}
		& 
		\begin{subfigure}[b]{0.3\textwidth}
			\centering
			\includegraphics[width=1\textwidth, angle=0]{./../shared/fig/frabs/SIR_10000agents_150t_01dt_NOSS_parallel.png}
			\caption{10,000 Agents}
			\label{fig:sir_abs_approximating_10000}
		\end{subfigure}
	\end{tabular}
	
	\caption{Varying agent numbers with same model-parameters except population size. All simulations run for 150 time-steps with $\Delta t = 0.1$}
	\label{fig:sir_abs_approximating}
\end{center}
\end{figure}

Still the dynamics of 10,000 Agents do not match the dynamics of the SD simulation perfectly. This is because as opposed to the SD simulation, which is deterministic, the agent-based approach is inherently a stochastic one as we continuously draw from random-distributions which drive our state-transitions. What we see in Figure \ref{fig:sir_abs_approximating} is then just a single run where the dynamics would result in slightly different shapes when run with a different random-number generator seed. The agent-based approach thus generates a distribution of dynamics over which ones needs to average to arrive at the correct solution. This can be done using replications in which the simulation is run with the exact same parameters multiple times but each with a different random-number generator see. The resulting dynamics are then averaged and the result is then regarded as the correct dynamics.
We have done this as can be seen in Figure \ref{fig:sir_abs_agents_repls}, using 10 replications, which matches the SD dynamics to a very satisfactory level. Note that in the replications we are using 10 initially infected agents to ensure that no simulation run will terminate too early (meaning that the disease gets extinct after a few time steps) which would offset the dynamics completely. This happens due to "unlucky" random distributions which can be repaired by introducing more initially infected agents which increases the probability of spreading the disease in the very early stage of the simulation drastically. We found that when using 10 initially infected agents in a population of 5,000 (which amounts to 0.2\%) is enough to never result in an early terminating simulation. In the case of 100 agents 10 initially infected ones might be too much and distorts the dynamics but this is irrelevant in this case. This is also a fundamental difference between SD and ABS: the dynamics of the agent-based approach can result in a wide range of scenarios which includes also the one in which the disease gets extinct in the early stages (a lucky coincidence for mankind) - this is simply not possible in the SD approach. So we can argue that ABS is much closer to reality than SD as it allows to explore alternate futures in the dynamics.

\begin{figure}
\begin{center}
	\begin{tabular}{c c}
		\begin{subfigure}[b]{0.3\textwidth}
			\centering
			\includegraphics[width=1\textwidth, angle=0]{./../shared/fig/frabs/SIR_100agents_150t_01dt_NOSS_parallel_10replications.png}
			\caption{100 Agents}
			\label{fig:sir_abs_agents_repls_100}
		\end{subfigure}
    	&
		\begin{subfigure}[b]{0.3\textwidth}
			\centering
			\includegraphics[width=1\textwidth, angle=0]{./../shared/fig/frabs/SIR_1000agents_150t_01dt_NOSS_parallel_10replications.png}
			\caption{1,000 Agents}
			\label{fig:sir_abs_agents_repls_1000}
		\end{subfigure}
    	
    	\\
    	
		\begin{subfigure}[b]{0.3\textwidth}
			\centering
			\includegraphics[width=1\textwidth, angle=0]{./../shared/fig/frabs/SIR_5000agents_150t_01dt_NOSS_parallel_10replications.png}
			\caption{5,000 Agents}
			\label{fig:sir_abs_agents_repls_5000}
		\end{subfigure}
		&
		\begin{subfigure}[b]{0.3\textwidth}
			\centering
			\includegraphics[width=1\textwidth, angle=0]{./../shared/fig/frabs/SIR_10000agents_150t_01dt_NOSS_parallel_10replications.png}
			\caption{10,000 Agents}
			\label{fig:sir_abs_agents_repls_10000}
		\end{subfigure}
	\end{tabular}
	
	\caption{Dynamics of Figure \ref{fig:sir_abs_approximating} averaged over 10 replications with initially 10 infected agents.} 
	\label{fig:sir_abs_agents_repls}
\end{center}
\end{figure}

When comparing the results of the dynamics of the agent-based approach from Figure \ref{fig:sir_abs_approximating} and Figure \ref{fig:sir_abs_agents_repls} to the SD dynamics of Figure \ref{fig:sir_sd_dynamics} it becomes apparent that by increasing the number of agents the dynamics approximate the SD dynamics with increasing accuracy. Still although using 5,000 agents and replications seem to be not enough yet, we need to increase our number of agents to 10,000

Still although using a quite small $\Delta t = 0.1$ and using replications we are nowhere close to the SD dynamics. The only option we have is to further decrease $\Delta t$. Of course performance is a big issue and it decreases as $\Delta t$ get smaller and smaller. This is because when running a simulation for a duration of $t$ and sampling it with $\Delta t$ when the steps to calculate is $\frac{t}{\Delta t}$. In each step all agents are run, messages delivered and environments folded and updated which implies that the more steps the lower the performance. If we could perform super-sampling just for the given high-frequency functions with the whole system running in lower frequency then we could achieve a substantial performance boost.

\subsection{Super-Sampling}
In Yampa there exists a function \textit{embed} which allows to run a given signal-function with provided $\Delta t$ but the problem is that this function does not really help because it does not return a signal-function. What we need is a signal-function which takes the number of super-samples \textit{n}, the signal-function \textit{sf} to sample and returns a new signal-function which performs super-sampling on it:

\begin{minted}[fontsize=\footnotesize]{haskell}
superSampling :: Int -> SF a b -> SF a [b]
\end{minted}

It does this by evaluating \textit{sf} for \textit{n} times, each with $\Delta t = \frac{\Delta t}{n}$ and the same input argument \textit{a} for all \textit{n} evaluations. At time 0 no super-sampling is done and just a single output of \textit{sf} is calculated. A list of \textit{b} is returned with length of \textit{n} containing the result of the \textit{n} evaluations of \textit{sf}. If 0 or less super samples are requested exactly one is calculated.

We ran tests super-sampling both \textit{occasionally} Figure \ref{fig:sampling_occasionally_ss_02evts}, Figure \ref{fig:sampling_occasionally_ss_5evts} and \textit{afterExp} Figure \ref{fig:sampling_afterExp_ss_5time}, Figure \ref{fig:sampling_afterExp_ss_02time}. They work the same way as above except that now $\Delta t = 1.0$ but using increasing numbers of super-samples. The results are as expected: as the number of super-samples increase, so increases the accuracy.

\begin{figure*}
\begin{center}
	\begin{tabular}{c c}
		\begin{subfigure}[b]{0.5\textwidth}
			\centering
			\includegraphics[width=.6\textwidth, angle=0]{./../shared/fig/sampling/samplingTest_occasionally_ss_02evts.png}
			\caption{Super-Sampling the \textit{occasional} function with event-frequency of 5 (average of 0.2 events per time-unit). The theoretical average is 20 event within this time-frame.}
			\label{fig:sampling_occasionally_ss_02evts}
		\end{subfigure}
	
		&
		
		\begin{subfigure}[b]{0.5\textwidth}
			\centering
			\includegraphics[width=.6\textwidth, angle=0]{./../shared/fig/sampling/samplingTest_occasionally_ss_5evts.png}
			\caption{Super-Sampling the \textit{occasional} function with event-frequency of $\frac{1}{5}$ (average of 5 events per time-unit). The theoretical average is 500 event within this time-frame.}
			\label{fig:sampling_occasionally_ss_5evts}
		\end{subfigure}

		\\
		
		\begin{subfigure}[b]{0.5\textwidth}
			\centering
			\includegraphics[width=.6\textwidth, angle=0]{./../shared/fig/sampling/samplingTest_afterExp_SS_5time.png}
			\caption{Super-Sampling the \textit{afterExp} function with average time-out of 5.}
			\label{fig:sampling_afterExp_ss_5time}
		\end{subfigure}

		&
		
		\begin{subfigure}[b]{0.5\textwidth}
			\centering
			\includegraphics[width=.6\textwidth, angle=0]{./../shared/fig/sampling/samplingTest_afterExp_SS_02time.png}
			\caption{Super-Sampling the \textit{afterExp} function with average time-out of 0.2.}
			\label{fig:sampling_afterExp_ss_02time}
		\end{subfigure}
	\end{tabular}
	
	\caption{Super-Sampling the \textit{afterExp} and \textit{occasional} functions to visualize the influence of increasing number of super-samples on the average occurrence of the respective events. The $\Delta t = 1.0$ in both cases with super-samples of [1, 2, 5, 10, 100, 1000]. The experiments for \textit{afterExp} used 10,000 replications. The experiments for \textit{occasional} ran for $t = 100$ with 100 replications.} 
	\label{fig:supersampling_tests}
\end{center}
\end{figure*}

At first this might not seem to be a real win as we still need to calculate a big number of samples every time. The big win comes though when these super-sampled signal-functions are embedded in a larger system which could run on a comparatively low frequency of $\Delta t = 1.0$. So we are then increasing the sampling-frequency just where we need it and keep the frequency low where it is not required.

We are using super-sampling in our SIR implementation to increase performance. We do this by setting $\Delta t = 1.0$ and super-sampling the relevant functions with time-semantics which are \textit{transitionAfterExp} and \textit{sendMessageOccationallySrc}. For both we provide in our EDSL versions which support super-sampling:

\begin{minted}[fontsize=\footnotesize]{haskell}
sendMessageOccasionallySrcSS :: RandomGen g => g -> Double -> Int -> MessageSource 
                                -> SF (AgentOut, e) AgentOut
                                
transitionAfterExpSS :: RandomGen g => g -> Double -> Int 
                        -> AgentBehaviour -> AgentBehaviour -> AgentBehaviour
\end{minted}

Both now take an additional parameter which determines the number of super-samples to be calculated. According to the above observations of the \textit{occasionally} and \textit{afterExp} functions which are the foundations of both of the functions we sample \textit{sendMessageOccasionallySrcSS} with 20 super-samples and \textit{transitionAfterExpSS} with 2. This will ensure that by using $\Delta t = 1.0$ we only calculate $t$ steps when running a simulation for $t$ time but that we sample our relevant functions with enough resolution to capture its frequencies. Optimally we should increase the number of super-samples for \textit{sendMessageOccasionallySrcSS} to about 100. This will result in lower performance as \textit{every} agent will perform this super-sampling. So in the end it is a struggle of performance vs. sufficiently close approximation. We define the number of super-samples in lines 29 and 32 and use the functions in line 96 and 106 of Appendix \ref{app:abs_code}.

TODO: 10.000 with SS and dt = 1.0 with ss

Unfortunately when setting $\Delta t = 1.0$ the dynamics of the agent-based approach won't approach the dynamics of the SD, despite using super-sampling as can be seen in Figure \ref{fig:sir_10000_1dt}.

\begin{figure}
\begin{center}
	\begin{tabular}{c c}
		\begin{subfigure}[b]{0.5\textwidth}
			\centering
			\includegraphics[width=.8\textwidth, angle=0]{./../shared/fig/frabs/SIR_10000agents_150t_1dt_parallel.png}
			\caption{$\Delta t = 1.0$}
			\label{fig:sir_10000_1dt}
		\end{subfigure}
	
		&
		
		\begin{subfigure}[b]{0.5\textwidth}
			\centering
			\includegraphics[width=.8\textwidth, angle=0]{./../shared/fig/frabs/SIR_10000agents_150t_01dt_parallel.png}
			\caption{$\Delta t = 0.1$}
			\label{fig:sir_10000_01dt}
		\end{subfigure}
	\end{tabular}
	
	\caption{Comparing the influence of different $\Delta t$. Both dynamics were generated with the same configuration of 10,000 agents, super-sampling enabled as described and the same model-parameters. When using $\Delta t = 1.0$, the dynamics do not match the ones of the SD approach, whereas in the case of $\Delta t = 0.1$, they can be seen as matching completely.} 
	\label{fig:sir_10000_dt_comparisons}
\end{center}
\end{figure}

When reflecting on the messaging mechanism it becomes apparent that a round-trip from sender to receiver and back takes $2 \Delta t$. A round-trip happens in our agent-based SIR approach to implement the transition from infected to susceptible - susceptible agents send \textit{Contact Susceptible} messages to random agents (except itself) where only infected agents reply with a \textit{Contact Infected} message. This means that it takes $2 \Delta t$ until a susceptible agent might get infected. This becomes an issue if we want to match the dynamics of our agent-based approach to the one of SD in which no time-delay happens - the agents act instantaneous with each other during one time-step. 
We have two solutions for this problem: either we resort to \textit{conversations} or we increase the global sampling frequency of the system which matches the \textit{message frequency} of messages which are subject to round-trips. Implementing conversations is only available in the \textit{sequential} update-strategy and is much more involved, so we followed the approach of increasing the frequency. As can be seen in Figure \ref{fig:sir_10000_01dt} when setting $t\Delta = 0.1$ the resulting dynamics are a sufficiently good approximation to the SD solution.

\section{Conclusions}
\label{sec:conclusions}

Our approach is radically different from traditional approaches in the ABS community. First it builds on the already quite powerful FRP paradigm. Second, due to our continuous time approach, it forces one to think properly of time-semantics of the model and how small $\Delta t$ should be. Third it requires to think about agent interactions in a new way instead of being just method-calls.

Because no part of the simulation runs in the IO Monad and we do not use unsafePerformIO we can rule out a serious class of bugs caused by implicit data-dependencies and side-effects which can occur in traditional imperative implementations.

Also we can statically guarantee the reproducibility of the simulation, which means that repeated runs with the same initial conditions are guaranteed to result in the same dynamics. Although we allow side-effects within agents, we restrict them to only the Random and State Monad in a controlled, deterministic way and never use the IO Monad which guarantees the absence of non-deterministic side effects within the agents and other parts of the simulation.

Determinism is also ensured by fixing the $\Delta t$ and not making it dependent on the performance of e.g. a rendering-loop or other system-dependent sources of non-determinism as described by \cite{perez_testing_2017}. Also by using FRP we gain all the benefits from it and can use research on testing, debugging and exploring FRP systems \cite{perez_testing_2017, perez_back_2017}.

\subsection*{Issues}
Currently, the performance of the system is not comparable to imperative implementations but our research was not focusing on this aspect. We leave the investigation and optimization of the performance aspect of our approach for further research.

Despite the strengths and benefits we get by leveraging on FRP, there are errors that are not raised at compile time, e.g. we can still have infinite loops and run-time errors. This was for example investigated in \cite{sculthorpe_safe_2009} where the authors use dependent types to avoid some run-time errors in FRP. We suggest that one could go further and develop a domain specific type system for FRP that makes the FRP based ABS more predictable and that would support further mathematical analysis of its properties. Furthermore, moving to dependent types would pose a unique benefit over the traditional object-oriented approach and should allow us to express and guarantee even more properties at compile time. We leave this for further research.

In our pure functional approach, agent identity is not as clear as in traditional object-oriented programming, where an agent can be hidden behind a polymorphic interface which is much more abstract than in our approach. Also the identity of an agent is much clearer in object-oriented programming due to the concept of object-identity and the encapsulation of data and methods.

We can conclude that the main difficulty of a pure functional approach evolves around the communication and interaction between agents, which is a direct consequence of the issue with agent identity. Agent interaction is straight-forward in object-oriented programming, where it is achieved using method-calls mutating the internal state of the agent, but that comes at the cost of a new class of bugs due to implicit data flow. In pure functional programming these data flows are explicit but our current approach of feeding back the states of all agents as inputs is not very general and we have added further mechanisms of agent interaction which we had to omit due to lack of space.

\bibliographystyle{acm}
\bibliography{../../../references/phdReferences}

\end{document}