%\documentclass[format=acmsmall, review=false, screen=true]{acmart}		% ICFP
%\documentclass[format=sigplan, review=true]{acmart}		% HASKELL SYMPOSIUM 
\documentclass[format=sigconf, review=true]{acmart}		% IFL

\usepackage{float}
\usepackage{graphicx}
\usepackage{subcaption}
\usepackage{ifthen}
\usepackage{minted}
\usepackage{verbatim}

% Metadata Information
%% use defaults for review submission.
%\acmConference[HS18]{Haskell Symposium}{2018}{09}
%\acmYear{2018}
%\copyrightyear{2018}
\acmConference[IFL'18]{International Symposium on Implementation and Application of Functional Languages}{August 2019}{Lowell, MA, USA}
\acmYear{2019}
\copyrightyear{2019}
%\acmDOI{} % \acmDOI{10.1145/nnnnnnn.nnnnnnn}

% Copyright
%% use 'none' for review submission.
\setcopyright{none}
%\setcopyright{acmcopyright}	% = copyright transfer to ACM
%\setcopyright{acmlicensed} 		% = retaining copyright but granting ACM exclusive publication rights
%\setcopyright{rightsretained}  % = open access on payment of a fee
%\setcopyright{usgov}
%\setcopyright{usgovmixed}
%\setcopyright{cagov}
%\setcopyright{cagovmixed}

% TODO : get the data
% DOI
% \acmDOI{0000001.0000001}

% TODO: fill in
% Paper history
\received{May 2018}
%\received[revised]{March 2018}
%\received[accepted]{March 2018}

% Document starts
\begin{document}

\newminted[HaskellCode]{haskell}{fontsize=\footnotesize}

% Title portion. Note the short title for running heads
\title[Hands Off My Property!]{Hands Off My Property!}
\subtitle{Debugging And Property-Based Testing Of \\ Pure Functional Agent-Based Simulations}

\author{Jonathan Thaler}
\orcid{https://orcid.org/0000-0001-8736-0479}
\email{jonathan.thaler@nottingham.ac.uk}
\affiliation{%
  \institution{University of Nottingham}
  \streetaddress{7301 Wollaton Rd}
  \city{Nottingham}
  \postcode{NG8 1BB}
  \country{United Kingdom}}

\begin{abstract}
This paper presents a new approach to test the implementation of agent-based simulations, called property-based testing which allows to test specifications much more directly than unit tests. Although being more expressive than unit tests, property-based testing is seen as a complementary and does not make unit-testing obsolete. We present two different models as case-studies in which we will show how to apply property-based testing to exploratory and explanatory agent-based models and what its limits are.
We conduct our implementations in the pure functional programming language Haskell, which is the origin of property-pased testing. We show that simply by switching to such a language one gets rid of a large class of run-time bugs and is able to make stronger guarantees of correctness already at compile time without writing tests for some parts. Further, it makes isolated unit-tests quite easier.

TODO: this would be ideal to submit to an ABS conference so i can also discuss functional programming

TODO: write related research
TODO: write background
TODO: implement case study 1: property-testing of SIR
TODO: implement case study 2: property-testing of Sugarscape
TODO: write conclusion \& further research
\end{abstract}

%
% The code below should be generated by the tool at
% http://dl.acm.org/ccs.cfm
% Please copy and paste the code instead of the example below.
%
% TODO needs to be generated
%\begin{CCSXML}
%<ccs2012>
% <concept>
%  <concept_id>10010520.10010553.10010562</concept_id>
%  <concept_desc>Computer systems organization~Embedded systems</concept_desc>
%  <concept_significance>500</concept_significance>
% </concept>
% <concept>
%  <concept_id>10010520.10010575.10010755</concept_id>
%  <concept_desc>Computer systems organization~Redundancy</concept_desc>
%  <concept_significance>300</concept_significance>
% </concept>
% <concept>
%  <concept_id>10010520.10010553.10010554</concept_id>
%  <concept_desc>Computer systems organization~Robotics</concept_desc>
%  <concept_significance>100</concept_significance>
% </concept>
% <concept>
%  <concept_id>10003033.10003083.10003095</concept_id>
%  <concept_desc>Networks~Network reliability</concept_desc>
%  <concept_significance>100</concept_significance>
% </concept>
%</ccs2012>
%\end{CCSXML}
%
%\ccsdesc[500]{Computer systems organization~Embedded systems}
%\ccsdesc[300]{Computer systems organization~Redundancy}
%\ccsdesc{Computer systems organization~Robotics}
%\ccsdesc[100]{Networks~Network reliability}

%
% End generated code
%

\keywords{Agent-Based Simulation, Property-Based Testing, Model Checking, Haskell}

\maketitle

%*******************************************************************************
%*********************************** First Chapter *****************************
%*******************************************************************************

\chapter{Introduction}  %Title of the First Chapter
I noticed that it is pretty hard to convince an agent-based economics specialist who is not a computer scientist about a pure functional approach. My conjecture is that the implementation technique and method does not matter much to them because they have very little knowledge about programming and are almost always self-taught - they don't know about software-engineering, nothing about proper software-design and architecture, nothing about software-maintenance, nothing about unit-testing,... In the end they just "hack" the simulation in whatever language they are able to: C++, Visual Basic, Java or toolboxes like Netlogo. For them it is all about to \textit{get things done somehow} and not to get things done the right way or in a beautiful way - the way and the method doesn't matter, its just a necessary evil which needs to be done. Thus if functional programming could make their lives easier, then they will definitely welcome it. But functional programming is, i think, harder to learn and harder to understand - so one needs to provide an abstraction through EDSL. So I REALLY need to come up with convincing arguments why to use pure functional approaches in ACE THEY can understand, otherwise I will be lost and not heard (not published,...). \\

What ACE economists care for:

\begin{itemize}
\item Very: Qualitative modelling with quantitative results
\item Yes: Easy reproducibility
\item Likely: Reasoning about convergence?
\item Likely: EDSL
\end{itemize}

My contributions are: pure functional framework, functional agent-model for market-simulations, EDSL for market-simulations, qualitative / implicit modelling with quanitative results, reasoning in my framework about convergence \\

IDEA: could I develop non-causal modelling (models are expressed in terms of non-directed equations, modelled in signal-relations) to allow for qualitative modelling for the agent-based economists? See hybrid modelling paper of Yampa. \textbf{THIS WOULD BE A HUGE NOVEL CONTRIBUTION TO ACE ESPECIALLY WHEN COMBINED WITH AN EDSL AND PROVIDING FULL REFERENTIAL TRANSPARENCY TO KEEP THE ABILITY TO REASON ABOUT CONVERGENCE}. This should be covered in the "EDSL"-paper.

TODO: maybe i should really focus only on market models? otherwise too much? \\

central novelty of my PhD: model specification = runnable code. possible through EDSL. but only in specific subfield of ACE: market-models. need a functional description of the model, then translate it to model specification in EDSL and then run it to see dynamics. But: model specification moves closer to functional programming languages. \\

another novelty approach: model specification through qualitative instead of quantiative approaches. is this possible? \\

WHY FUNCTIONAL? "because its the ultimate approach to scientific computing": fewer bugs due to mutable state (why? is thos shown obkectively by someone?), shorter (again as above, productivity), more expressive and closer to math, EDSL, EDSL=model=simulation, better parallelising due to referental transparency, reasoning \\

scientific results need to be reproduced, especially when they have high impact. a more formal approach of specifying the model and the simulation (model=simulation) could lead to easier sharing and easier reporduction without ambigouites \\

pure functional agent-model \& theory, EDSL framework in Haskell for ACE

\begin{enumerate}
\item Which kind of problem do we have?
\item What aim is there? Solving the problem? 
\item How the aim is achieved by enumerating VERY CLEAR objectives.
\item What the impact one expects (hypothesis) and what it is (after results).
\end{enumerate}

Note: It is not in the interest of the researcher to develop new economic theories but to research the use of functional methods (programming and specification) in agent-based computational economics (ACE).

NOTE: Get the reader’s attention early in the introduction: motivation, significance, originality and novelty.

\section{Methods}
Methods need to be selected to implement the simulations. Special emphasis will be put on functional ones which will then be compared to established methods in the field of ABM/S and ACE. \\

Claim: non-programming environments are considered to be not powerful enough to capture the complexity of ACE implementations thus a programming approach to ACE will be always required.

\section{Scenarios}
To apply and test functional methods in ACE, four scenarios of ACE are selected and then the methods applied and compared with each other to see how each of them perform in comparison. The 4 selected scenarios represent a selection of the challenges posed in ACE: from very abstract ones to very operational ones.

\section{Comparison}
Each of the selected scenarios is then implemented using the selected methods where each solution is then compared against the following criteria: 

\begin{enumerate}
\item suitability for scientific computation
\item robustness
\item error-sources
\item testability
\item stability
\item extendability
\item size of code
\item maintainability
\item time taken for development
\item verification \& correctness
\item replications \& parallelism
\item EDSL
\end{enumerate}

This will then allow to compare the different methods against each other and to show under which circumstances functional methods shine and when they should not be used.

\section{Agent-Based Modelling and Simulation (ABM/S)}
ABM/S is a method of modelling and simulating a system where the global behaviour may be unknown but the behaviour and interactions of the parts making up the system is of knowledge (Wooldrige, M. (2009). An Introduction to MultiAgent Systems. John Wiley & Sons). Those parts, called agents, are modelled and simulated out of which then the aggregate global behaviour of the whole system emerges. Thus the central aspect of ABM/S is the concept of an Agent which can be understood as a metaphor for a pro-active unit, able to spawn new Agents, and interacting with other Agents in a network of neighbours by exchange of messages. The implementation of Agents can vary and strongly depends on the programming language and the kind of domain the simulation and model is situated in.

\section{Agent-Based Economics (ACE)}
According to Leigh Tesfatsion (Tesfatsion, L. (2006). Agent-based computational economics: A constructive approach to economic theory. In Tesfatsion, L. and Judd, K. L., editors, Handbook of Computational Economics, volume 2, chapter 16, pages 831–880. Elsevier, 1 edition.), one of the leading figures, ACE is "[...] computational modelling of economic processes (including whole economies) as open-ended dynamic systems of interacting agents." - thus lending perfectly to the use of ABM/S as already the name suggests. Whereas classical economic models fall short by only looking at the average, pure rational, individual interacting in anonymous markets, the ACE approach looks at heterogeneous, non-rational individuals interacting with each other in networks (Kirman, A. (2010). Complex Economics: Individual and Collective Rationality. Routledge, London ; New York, NY.). Thus ACE can be understood as a combination of computer-science, cognitive/social science and evolutionary economics.

\section{Functional programming}
TODO: read \cite{Backus1978}

The state-of-the-art approach to implementing Agents are object-oriented methods and programming as the metaphor of an Agent as presented above lends itself very naturally to object-orientation (OO). The author of this thesis claims that OO in the hands of inexperienced or ignorant programmers is dangerous, leading to bugs and hardly maintainable and extensible code. The reason for this is that OO provides very powerful techniques of organising and structuring programs through Classes, Type Hierarchies and Objects, which, when misused, lead to the above mentioned problems. Also major problems, which experts face as well as beginners are 1. state is highly scattered across the program which disguises the flow of data in complex simulations and 2. objects don’t compose as well as functions. The reason for this is that objects always carry around some internal state which makes it obviously much more complicated as complex dependencies can be introduced according to the internal state.
All this is tackled by (pure) functional programming which abandons the concept of global state, Objects and Classes and makes data-flow explicit. This then allows to reason about correctness, termination and other properties of the program e.g. if a given function exhibits side-effects or not. Other benefits are fewer lines of code, easier maintainability and ultimately fewer bugs thus making functional programming the ideal choice for scientific computing and simulation and thus also for ACE. A very powerful feature of functional programming is Lazy evaluation. It allows to describe infinite data-structures and functions producing an infinite stream of output but which are only computed as currently needed. Thus the decision of how many is decoupled from how to (Hughes, J. (1989). Why functional programming matters. Comput. J., 32(2):98–107.).
The most powerful aspect using pure functional programming however is that it allows the design of embedded domain specific languages (EDSL). In this case one develops and programs primitives e.g. types and functions in a host language (embed) in a way that they can be combined. The combination of these primitives then looks like a language specific to a given domain, in the case of this thesis ACE. The ease of development of EDSLs in pure functional programming is also a proof of the superior extensibility and composability of pure functional languages over OO (Henderson P. (1982). Functional Geometry. Proceedings of the 1982 ACM Symposium on LISP and Functional Programming.).
One of the most compelling example to utilize pure functional programming is the reporting of Hudak (Hudak P., Jones M. (1994). Haskell vs. Ada vs. C++ vs. Awk vs. ... An Experiment in Software Prototyping Productivity. Department of Computer Science, Yale University.)  where in a prototyping contest of DARPA the Haskell prototype was by far the shortest with 85 lines of code. Also the Jury mistook the code as specification because the prototype did actually implement a small EDSL which is a perfect proof how close EDSL can get to and look like a specification.

Functional languages can best be characterized by their way computation works: instead of \textit{how} something is computed, \textit{what} is computed is described. Thus functional programming follows a declarative instead of an imperative style of programming. The key points are:
\begin{itemize}
\item No assignment statements - variables values can never change once given a value.
\item Function calls have no side-effect and will only compute the results - this makes order of execution irrelevant, as due to the lack of side-effects the logical point in \textit{time} when the function is calculated within the program-execution does not matter.
\item higher-order functions
\item lazy evaluation
\item Looping is achieved using recursion, mostly through the use of the general fold or the more specific map.
\item Pattern-matching
\end{itemize}

This alone does not really explain the \textit{real} advantages of functional programming and one must look for better motivations using functional programming languages. One motivation is given in \cite{Hughes1989} which is a great paper explaining to non-functional programmers what the significance of functional programming is and helping functional programmers putting functional languages to maximum use by showing the real power and advantages of functional languages. The main conclusion is that \textit{modularity}, which is the key to successful programming, can be achieved best using higher-order functions and lazy evaluation provided in functional languages like Haskell. \cite{Hughes1989} argues that the ability to divide problems into sub-problems depends on the ability to glue the sub-problems together which depends strongly on the programming-language and \cite{Hughes1989} argues that in this ability functional languages are superior to structured programming.

TODO: comparison of functional and object-oriented programming. My points are:
\begin{itemize}
\item The way state can be changed and treated - distributed over multiple objects - is often very difficult to understand.
\item Inheritance is a dangerous thing if not used with care because inheritance introduces very strong dependencies which cannot be changed during runtime anymore.
\item Objects don't compose very well: \url{http://zeroturnaround.com/rebellabs/why-the-debate-on-object-oriented-vs-functional-programming-is-all-about-composition/}
\item (Nearly) impossible to reason about programs
\end{itemize}

In conclusion the upsides of functional programming as opposed to OO are:
\begin{itemize}
\item Much more explicit flow of data \& control
\item Much better compose-able
\item Much better parallelism
\end{itemize}

\section{Related Research}
Tim Sweeney, CTO of Epic Games gave an invited talk about how "future programming languages could help us write better code" by "supplying stronger typing, reduce run-time failures;  and the need for pervasive concurrency support, both implicit and explicit, to effectively exploit the several forms of parallelism present in games and graphics." \cite{Sweeney2006}. Although the fields of games and agent-based simulations seem to be very different in the end, they have also very important similarities: both are simulations which perform numerical computations and update objects - in games they are called "game-objects" and in abm they are called agents but they are in fact the same thing - in a loop either concurrently or sequential. His key-points were:

\begin{itemize}
\item Dependent types as the remedy of most of the run-time failures.
\item Parallelism for numerical computation: these are pure functional algorithms, operate locally on mutable state. Haskell ST, STRef solution enables encapsulating local heaps and mutability within referentially transparent code.
\item Updating game-objects (agents) concurrently using STM: update all objects concurrently in arbitrary order, with each update wrapped in atomic block - depends on collisions if performance goes up.
\end{itemize}

\section{Background}

\subsection{Schelling Segregation}
We follow in our implementation the original paper of Schelling as in \cite{schelling_dynamic_1971} where we focus on the \textit{Area Distribution} section (Schelling starts with movement in a linear, 1-dimensional world where agents are able to move to the nearest point which meets the agents satisfaction but this is not what we follow here). One assumes a discrete 2-dimensional lattice-world with NxM fields. Each field is either occupied by an agent of a given color (e.g. Red or Green) or is free. Each field has 8 neighbours, which denotes a Moore-Neighbourhood. In Schellings original work the lattice-world is limited at its borders but we assume a torus world which is wrapped around in both the x- and y-dimensions resulting in 8 neighbours also for fields at the border. The occupation density was set by Schelling to be about 70\%-75\% which he identifies as being a setting which allows the agents to move around freely without making the lattice-world too sparse.
Now the agents make their move sequentially one after another. In each move an agent calculates the number of neighbours which are of equal color. If the number satisfies the agents needs about the neighbourhood then the agent is regarded as being 'happy' and will stay on this field. On the other hand the agent moves to the nearest unoccupied field which satisfies its needs. An agent which moves selects an unoccupied place randomly relative from its current place within a rectangle of side-length 2r where its current place is at the center. The interpretation for that behaviour is that agents won't move too far as it could be costly. Also children might attend a school in this area or the family has friends in this area, so they don't want to break that.



Agents just move depending on their movement-strategy to another place if they are not happy on the current one - they don't care how the target place is in the present or in the future, they will decide again in the next time-step. The interpretation for that behaviour is: agents want to 'just get out' at any cost, not caring what the future place will look like - it might be better or worse but they will see then.

\subsubsection{Optimizing behaviour}
TODO: define utility

The original schelling model didn't have a move-optimizing behaviour, meaning agents are just binary: if it is happy it will not move, if it is unhappy it will move but they won't care where they move. We introduce local move-optimizing behaviours which can be interpreted as being realistic in the real-world. It is important to note that we focus on \textit{local} instead of \textit{global} move-optimization: the agents are limited in their reasoning-capabilities and have limited information available: they cannot check out \textit{every} place and pick the globally best one.\\

\subsubsection{Anticipating behaviour}
Schelling explicitly mentions in \cite{schelling_dynamic_1971} that nobody anticipates moves of others. This is what we introduce using the recursive simulation.

TODO: is this optimizing behaviour in the spirit of schellings original work? 

\paragraph{Optimizing future} Agents pick an unoccupied random place and move to it if it increases their utility in the future. The interpretation for that behaviour is: agents heard about a place which will be cool in the future.

\paragraph{Optimizing present \& future} Agents pick an unoccupied random place and move to it if it increases their utility in the now and in the future. The interpretation for that behaviour is: agents heard about a cool spot in town, check it out and move to it if they like it but they also anticipate the coolness of the place in the future and if it seems that the place is going down then they won't move there.

\subsection{Related Research}
TODO: \cite{kirman_complex_2010} mention kirman complex economics where he investigates the model more in depth


\section{Pure Functional ABS}
TODO: introduce the general concepts of pure functional ABS and what the benefits and drawbacks are in the context of debugging and correctness, using my IFL paper

We argue that due to its fundamental different nature, the functional programming paradigm can overcome some fundamental problems of the established object-oriented approach to ABS. Note that we don't claim that it will solve all the problems and that the Gintis failure wouldn't have happened but we argue that it makes making mistakes much harder, resulting in simulations which are more likely to be correct.

We have investigated the concepts of \textbf{\textit{how}} to do agent-based simulation using the functional programming paradigm, as in the language Haskell, which is described in the paper in Appendix \ref{app:pfe}. The approach we developed is based on Functional Reactive Programming which allows to express discrete- and continuous-time systems in functional programming. Following the conclusions of the paper, we got the following benefits, supporting directly our initial hypothesis and our claims above, giving good reasons \textbf{\textit{why}} to do ABS in a functional way:

\begin{enumerate}
	\item Run-Time robustness by compile-time guarantees - by expressing stronger guarantees already at compile-time we can restrict the classes of bugs which occur at run-time by a substantial amount due to Haskell's strong and static type system.  This implies the lack of dynamic types and dynamic casts \footnote{Note that there exist casts between different numerical types but they are all safe and can never lead to errors at run-time.} which removes a substantial source of bugs.  Note that we can still have run-time bugs in Haskell when our functions are partial.
	\item Purity - By being explicit and polymorphic in the types about side-effects and the ability to handle side-effects explicitly in a controlled way allows to rule out non-deterministic side-effects which guarantees reproducibility due to guaranteed same initial conditions and deterministic computation. Also by being explicit about side-effects e.g. Random-Numbers and State makes it easier to verify and test.
	\item Explicit Data-Flow and Immutable Data - All data must be explicitly passed to functions thus we can rule out implicit data-dependencies because we are excluding IO. This makes reasoning of data-dependencies and data-flow much easier as compared to traditional object-oriented approaches which utilize pointers or references.
	\item Declarative - describing \textit{what} a system is, instead of \textit{how} (imperative) it works. In this way it should be easier to reason about a system and its (expected) behaviour because it is more natural to reason about the behaviour of a system instead of thinking of abstract operational details.
	\item Concurrency and parallelism - due to its pure and 'stateless' nature, functional programming is extremely well suited for massively large-scale applications as it allows adding parallelism without any side-effects and provides very powerful and convenient facilities for concurrent programming. We have explored this more in-depth in Chapter \ref{chap:stm}.
\end{enumerate}

In general, Types guide us in program construction by restricting the operations we can perform on the data. This means that by choosing types this reveals already a lot of our program and data and prevents us from making mistakes e.g. interpreting some binary data as text instead of a number. In strongly statically typed languages the types can do this already at compile-time which allows to rule out certain bugs already at compile-time. In general, we can say that for all bugs which can be ruled out at compile-time, we don't need to write property- or unit-tests, because those bugs cannot - per definition - occur at run-time, so it won't make sense to test their absence at run-time. Also, as Dijkstra famously put it: "Testing shows the presence, not the absence of bugs" - thus by induction we can say that compile-time guarantees save us from a potentially infinite amount of testing.

In general it is well established, that pure functional programming as in Haskell, allows to express much stronger guarantees about the correctness of a program \textit{already at compile-time}. This is in fundamental contrast to imperative object-oriented languages like Java or Python where only primitive guarantees about types - mostly relationships between type-hierarchies - can be expressed at compile-time which directly implies that one needs to perform much more testing (user testing or unit-testing) at \textit{run-time} to check whether the model is sufficiently correct. Thus guaranteeing properties already at compile-time frees us from writing unit-tests which cover these cases or test them at run time because they are \textit{guaranteed to be correct under all circumstances, for all inputs}.

In this regards we see pure functional programming as truly superior to the traditional object oriented approaches: they lead to implementations of models which are more likely correct because we can express more guarantees already at compile-time which directly leads to less bugs which directly increases the probability of the software being a correct implementation of the model. Having established this was only the first step in our paper in Appendix \ref{app:pfe}. 

Although pure functional ABS as in Haskell allows us to leverage on the concepts of functional and its benefits (and drawbacks) we still rely heavily on (property-based) testing to ensure correctness of a simulation because our approach still can have run-time bugs. Thus, the next step, which follows directly, is towards even stronger guarantees at compile-time, by using dependent types. 


\subsection{Implementing}
TODO: my IFL paper
TODO: Back To the Future: Time Travel in FRP \cite{perez_back_2017}

\subsection{Debugging}
TODO: haskell-titan
TODO: Testing and Debugging Functional Reactive Programming \cite{perez_testing_2017}

General there are the following basic verification \& validation requirements to ABS \cite{robinson_simulation:_2014}, which all can be addressed in our \textit{pure} functional approach as described in the paper in Appendix \ref{app:pfe}:

\begin{itemize}
	%\item Modelling progress of time - achieved using functional reactive programming (FRP)
	%\item Modelling variability - achieved using FRP
	\item Fixing random number streams to allow simulations to be repeated under same conditions - ensured by \textit{pure} functional programming and Random Monads
	\item Rely only on past - guaranteed with \textit{Arrowized} FRP
	\item Bugs due to implicitly mutable state - reduced using pure functional programming
	\item Ruling out external sources of non-determinism / randomness - ensured by \textit{pure} functional programming
	\item Deterministic time-delta - ensured by \textit{pure} functional programming
	\item Repeated runs lead to same dynamics - ensured by \textit{pure} functional programming
\end{itemize}

\subsection{Property-Based ABS Testing}
TODO: general approach to property-based testing in ABS

Although (pure) functional programming allows us to have stronger guarantees about the behaviour and absence of bugs of the simulation already at compile-time, we still need to test all the properties of our simulation which we cannot guarantee at compile-time.

We found property-based testing particularly well suited for ABS. Although it is now available in a wide range of programming languages and paradigms, propert-based testing has its origins in Haskell \cite{claessen_quickcheck:_2000, claessen_testing_2002} and we argue that for that reason it really shines in pure functional programming. Property-based testing allows to formulate \textit{functional specifications} in code which then the property-testing library (e.g. QuickCheck \cite{claessen_quickcheck:_2000}) tries to falsify by automatically generating random test-data covering as much cases as possible. When an input is found for which the property fails, the library then reduces it to the most simple one. It is clear to see that this kind of testing is especially suited to ABS, because we can formulate specifications, meaning we describe \textit{what} to test instead of \textit{how} to test (again the declarative nature of functional programming shines through). Also the deductive nature of falsification in property-based testing suits very well the constructive nature of ABS.

Generally we need to distinguish between two types of testing/verification: 1. testing/verification of models for which we have real-world data or an analytical solution which can act as a ground-truth - examples for such models are the SIR model, stock-market simulations, social simulations of all kind and 2. testing/verification of models which are just exploratory and which are only be inspired by real-world phenomena - examples for such models are Epsteins Sugarscape and Agent\_Zero.

\subsubsection{Black-Box Verification}
In black-box Verification one generally feeds input and compares it to expected output. In the case of ABS we have the following examples of black-box test:
\begin{enumerate}
	\item Isolated Agent Behaviour - test isolated agent behaviour under given inputs using and property-based testing.
	\item Interacting Agent Behaviour - test if interaction between agents are correct .
	\item Simulation Dynamics - compare emergent dynamics of the ABS as a whole under given inputs to an analytical solution or real-world dynamics in case there exists some using statistical tests.
	\item Hypotheses- test whether hypotheses are valid or invalid using and property-based testing. % TODO: how can we formulate hypotheses in unit- and/or property-based tests?
\end{enumerate}

%- testing of the final dynamics: how close do they match the analytical solution
%- can we express model properties in tests e.g. quickcheck?
%- property-testing shines here
%- isolated tests: how easy can we test parts of an agent / simulation?

Using black-box verification and property-based testing we can apply for the following use cases for testing ABS in FRP:

\paragraph{Finding optimal $\Delta t$}
The selection of the right $\Delta t$ can be quite difficult in FRP because we have to make assumptions about the system a priori. One could just play it safe with a very conservative, small $\Delta t < 0.1$ but the smaller $\Delta t$, the lower the performance as it multiplies the number of steps to calculate. Obviously one wants to select the \textit{optimal} $\Delta t$, which in the case of ABS is the largest possible $\Delta t$ for which we still get the correct simulation dynamics.
To find out the \textit{optimal} $\Delta t$ one can make direct use of the black-box tests: start with a large $\Delta t = 1.0$ and reduce it by half every time the tests fail until no more tests fail - if for $\Delta t = 1.0$ tests already pass, increasing it may be an option. It is important to note that although isolated agent behaviour tests might result in larger $\Delta t$, in the end when they are run in the aggregate system, one needs to sample the whole system with the smallest $\Delta t$ found amongst all tests. Another option would be to apply super-sampling to just the parts which need a very small $\Delta t$ but this is out of scope of this paper.

\paragraph{Agents as signals}
Agents \textit{might} behave as signals in FRP which means that their behaviour is completely determined by the passing of time: they only change when time changes thus if they are a signal they should stay constant if time stays constant. This means that they should not change in case one is sampling the system with $\Delta t = 0$. Of course to prove whether this will \textit{always} be the case is strictly speaking impossible with a black-box verification but we can gain a good level of confidence with them also because we are staying pure. It is only through white-box verification that we can really guarantee and prove this property.

\subsubsection{White-Box Verification}
White-Box verification is necessary when we need to reason about properties like \textit{forever}, \textit{never}, which cannot be guaranteed from black-box tests. Additional help can be coverage tests with which we can show that all code paths have been covered in our tests.

TODO: List of Common Bugs and Programming Practices to avoid them \cite{vipindeep_list_2005}

We have discussed in this section \textit{how} to approach an ABS implementation from a pure functional perspective using Haskell where we have also briefly touched on \textit{why} one should do so and what the benefits and drawbacks are. In the next two sections we will expand on the \textit{why} by presenting two case-studies which show the benefits of using Haskell in regards of testing and increasing the confidence in the correctness of the implementation.

\section{Case Study I: SIR}
\label{sec:case_SIR}
As first use-case we discuss property-based testing for the \textit{explanatory} agent-based SIR model. It is a very well studied and understood compartment model from epidemiology \cite{kermack_contribution_1927} which allows to simulate the dynamics of an infectious disease like influenza, tuberculosis, chicken pox, rubella and measles spreading through a population. We implemented an agent-based version of this model \footnote{The code is freely accessible from \url{https://github.com/thalerjonathan/phd/tree/master/public/propabs/sir}}, inspired by \cite{macal_agent-based_2010}.

In this model, people in a population of size $N$ can be in either one of three states \textit{Susceptible}, \textit{Infected} or \textit{Recovered} at a particular time, where it is assumed that initially there is at least one infected person in the population. People interact \textit{on average} with a given rate of $\beta$ other people per time-unit and become infected with a given probability $\gamma$ when interacting with an infected person. When infected, a person recovers \textit{on average} after $\delta$ time-units and is then immune to further infections. An interaction between infected persons does not lead to re-infection, thus these interactions are ignored in this model. Due to the models' origin in System Dynamics (SD) \cite{porter_industrial_1962}, there exists a top-down formalisation in SD with the following equations:

\begin{equation}
\begin{aligned}
\frac{\mathrm d S}{\mathrm d t} = -infectionRate \\
\frac{\mathrm d I}{\mathrm d t} = infectionRate - recoveryRate \\
\frac{\mathrm d R}{\mathrm d t} = recoveryRate 
\end{aligned}
\end{equation}

\begin{equation}
\begin{aligned}
infectionRate = \frac{I \beta S \gamma}{N} \\
recoveryRate = \frac{I}{\delta} 
\end{aligned}
\end{equation}

\subsection{Deriving a property}
Our goal is to derive a property which connects the agent-based implementation to the SD equations. The foundation are both the infection- and recovery-rate where the infection-rate determines how many \textit{Susceptible} agents per time-unit become \textit{Infected} and the recovery-rate determines how many \textit{Infected} agents per time-unit become \textit{Recovered}. Lets look at the algorithm of the susceptible agent behaviour, which is key for the infection-rate:

\begin{algorithm}
generate on average $\beta$ make-contact events per time-unit\; 
\If{make-contact event}{
  select random agent \textit{randA} from population\; 
  \If{agent randA infected}{
    become infected with probability $\gamma$\; 
  }  
}
\caption{Susceptible behaviour}
\end{algorithm}

Per time-unit, a susceptible agent makes \textit{on average} contact with $\beta$ other agents, where in the case of a contact with an infected agent, the susceptible agent becomes infected with a given probability $\gamma$. In this description there is another probability hidden, the probability of making contact with an infected agent, which is simply the ratio of number of infected agents to number non-infected agents. We can now derive the formula for the probability of a \textit{Susceptible} agent to become infected: $\frac{\beta * \gamma * \text{number of infected (I)}}{\text{number of non-infected (N)}}$. When we look at the formula we can see that it is conceptually the same representation of the \textit{infection-rate} of the SD specification as shown above - except that it only considers a single \textit{Susceptible} agent instead of the aggregate of \textit{S} susceptible agents. We have now a property we can check using a property-test.

\subsection{Constructing the property-based test}
Having a property (law), we want now to construct a property-test for it. The formula is invariant under random population mixes and thus should hold for varying agent populations where the mix of \textit{Susceptible, Infected and Recovered} agents is random - thus we use QuickCheck to generate the population randomly, the property must still hold.

Obviously we need to pay attention to the fact that we are dealing with a stochastic system thus we can only talk about averages and thus it does not suffice to only run a single agent but we are repeating this for e.g. 10.000 \textit{Susceptible} agents (all with different random-number seeds). 

To check whether this test has passed we compare the required amount of agents which on average should become infected using the above formula to the one from our tests (simply count the agents which got infected and divide by N) and if the value lies within some small $\epsilon$ then we accept the test as passed. Now we can construct the following property-based test as shown in Algorithm \ref{alg:prop_test_infectionrate}.

\begin{algorithm}
\SetKwInOut{Input}{input}\SetKwInOut{Output}{output}
\Input{List \textit{randAs} of random agent-population generated by QuickCheck}
populationCount     = length \textit{randAs}\;
infectedCount       = count \textit{Infected} in \textit{randAs}\;
infectionRate       = infectivity * contactRate * (infectedCount / populationCount)\;

susceptibles = create 10000 \textit{Susceptible} agents\;
countInfected = 0\;
\For{each agent sa in susceptibles}{
  run agent sa for 1.0 time-unit, with list \textit{randAs} as input\;
  \If{agent sa became \textit{Infected} }{
	countInfected = countInfected + 1\;
  }
}

averageInfectionRate = countInfected / (length susceptibles)\;
$\epsilon$ = 0.1\;
\eIf{abs (averageInfectionRate - infectionRate) $\leq \epsilon$}{
  PASS\;
} {
  FAIL\;
}
\caption{Property-based test for infection-rate.}
\end{algorithm}
\label{alg:prop_test_infectionrate}

When running, QuickCheck generates 100 test-cases by randomly generating 100 different \textit{randAs} inputs to the test. All have to pass for the whole property-test to pass, which should be the case with an $\epsilon = 0.1$. 

This is the very power which property-based testing is offering us: we directly express the specification of the original SD model in a test of our agent-based implementation and let QuickCheck generate random test cases for us. This closely ties our implementation to the original specification and raises the confidence to a very high level that it is actually a valid and correct implementation.

%\subsection{Infected Behaviour}
%An infected agent will \textit{always} recover after some finite time, which is \textit{on average} after $\delta$ time-units. Note that this property involves stochastics too, so to test this property we run a large number of infected agents e.g. $N = 10.000$ (all with different random-number seeds) until they recover, record the time of each agents recovery and then average over all recovery times. To check whether this test has passed we compare the average recovery times to $\delta$ and if they lie within some small $\epsilon$ then we accept the test as passed (note again that we could use a t-test for better stochastic robustness but this is not the point of this paper).
%
%TODO: clearly state the property we test
%
%TODO: produce some pseudo-code of how the property-test conceptually works
%
%in the infected agent test we check if the average duration is as specified. does this resemble the recovery rate? or in other words: can we somehow test the recovery rate?
%durationsAvg = sum durations / fromIntegral (length durations)
%
%We use property-testing with QuickCheck in this case as well to generate the set of other agents as input for the infected agents. Strictly speaking this would not be necessary as an infected agent never makes contact with other agents and simply ignores them - we could as well just feed in an empty list. We opted for using QuickCheck for the following reasons:
%
%\begin{itemize}
%	\item We wanted to stick to the interface specification of the agent-implementation as close as possible which asks to pass the states of all agents as input.
%	\item We shouldn't make any assumptions about the actual implementation and if it REALLY ignores the other agents, so we strictly stick to the interface which requires us to input the states of all the other agents.
%	\item The set of other agents is ignored when determining whether the test has failed or not which indicates by construction that the behaviour of an infected agent does not depend on other agents.
%	\item We are not just running a single replication over 10.000 agents but 100 of them which should give black-box verification more strength.
%\end{itemize}
%
%\subsection{Recovered Behaviour}
%A recovered agent will stay recovered \textit{forever}. Obviously we cannot write a property-based test that truly verifies that because it had to run in fact \textit{forever}. In this case we need to resort to white-box verification and look directly at the code and reason whether this property holds true.

\section{Case Study II: SugarScape}
\label{sec:case_sug}
We now look at how property-based testing can be made of use in the \textit{exploratory} Sugarscape model \cite{epstein_growing_1996}. It was one of the first models in ABS, with the aim to \textit{grow} an artificial society by simulation and connect observations in their simulation to phenomenon observed in real-world societies. In this model a population of agents move around in a discrete 2D environment, where sugar grows, and interact with each other and the environment in many different ways. The main features of this model are (amongst others): searching, harvesting and consuming of resources, wealth and age distributions, population dynamics under sexual reproduction, cultural processes and transmission, combat and assimilation, bilateral decentralized trading (bartering) between agents with endogenous demand and supply, disease processes transmission and immunology. For our research we undertook a \textit{full and validated} implementation of the Sugarscape model \footnote{The code can be accessed freely from \url{https://github.com/thalerjonathan/phd/tree/master/public/towards/SugarScape/sequential}}. We undertook a full validation of our implementation against the book \cite{epstein_growing_1996} and a NetLogo implementation \cite{weaver_replicating_nodate} during which we also implemented property tests. Due to lack of space we added a discussion of the validation process as an Appendix \ref{app:validation}.

Whereas in the explanatory SIR case-study we had an analytical solution, inspired by the SD origins of the model, the fundamental difference in the exploratory Sugarscape model is that none such analytical solutions exist. This raises the question, which properties we can actually test in such a model: 
\begin{itemize}
	\item Environment behaviour parts.
	\item Agent behaviour parts.
	\item Hypotheses about emergent properties which when proved to be valid can be seen as regression tests.
\end{itemize}

\subsection{Environment behaviour}
The environment in the Sugarscape model has some very simple behaviour: each site has a sugar level and when harvested by an agent, it regrows back to the full level over time. Depending on the configuration of the model it either grows back immediately within 1 tick or over multiple ticks. We can construct simple property-based tests for these behaviours. In the case the sugar grows back immediately we let QuickCheck generate a random environment and then run the environment behaviour for 1 tick and then check the property that all sites have to be back to their maximum sugar level. In the case of regrow over multiple ticks, we also use QuickCheck to generate a random environment but additionally a random \textit{positive} rate (which is a floating point number) which we then use to calculate the number of steps until full regrowth. After running the random environment for the given number of steps all sites have to be back to full sugar level - we provided pseudo code for this case, see \ref{alg:prop_test_rateregwroth}.

Note that QuickCheck initially doesn't know how to generate a random environment because each site consists of a custom data-structure for which QuickCheck is not able to generate random instances by default. This problem is solved by writing a custom data-generator, for which existing QuickCheck functions can be used e.g. picking the current sugar level of a site from a random range.

%\begin{algorithm}
%\SetKwInOut{Input}{input}\SetKwInOut{Output}{output}
%\Input{Random environment \textit{env} generated by QuickCheck}
%env' = runEnvironmentTicks 1 env\;
%sites = getEnvironmentSites env'\;
%
%\eIf{all sites maxSugarLevel}{
%  PASS\;
%} {
%  FAIL\;
%}
%\caption{Property-based test for immediate regrow of sugar on all sites.}
%\end{algorithm}
%\label{alg:prop_test_fullregrowth}

\begin{algorithm}
\SetKwInOut{Input}{input}\SetKwInOut{Output}{output}
\Input{Random environment \textit{env} generated by QuickCheck}
\Input{Regrowth rate \textit{randRate} (positive floating point) generated by QuickCheck}
maxTicks = maxSugarCapacityOnSites / randRate\;
env' = runEnvironmentTicks maxTicks env\;
sites = getEnvironmentSites env'\;

\eIf{all sites maxSugarLevel}{
  PASS\;
} {
  FAIL\;
}
\caption{Property-based test for rate-based regrow of sugar on all sites.}
\end{algorithm}
\label{alg:prop_test_rateregwroth}

The Sugarscape environment is a torus where the coordinates wrap around in both dimensions. To check whether the implementation of the wrapping-calculation is correct we used both unit- and property-tests. With the unit-tests we carefully constructed all possible cases we could think of and came up with 13 test-cases. With the property-based test we simply defined a single test-case where we expressed the property that after wrapping \textit{any} coordinates, supplied by QuickCheck, the wrapped coordinates have to be within bounds. See pseudo code \ref{alg:prop_test_wrapcoords}.

\begin{algorithm}
\SetKwInOut{Input}{input}\SetKwInOut{Output}{output}
\Input{Random 2d discrete coordinate \textit{randCoord} generated by QuickCheck}
(x, y) = wrapCoordinates randCoord\;

\eIf{(x $\geq$ 0 and x $\leq$ environmentDimX) and (y $\geq$ 0 and y $\leq$ environmentDimY)}{
  PASS\;
} {
  FAIL\;
}
\caption{Property-based test for wrap-coordinates functionality.}
\end{algorithm}
\label{alg:prop_test_wrapcoords}

\subsection{Agent behaviour parts}
We implemented a number of tests for agent functions which just cover the part of an agents behaviour: checks whether an agent has died of age or starved to death, the metabolism, immunisation step, check if an agent is a potential borrower or fertile, lookout, trading transaction. What all these functions have in common is that they are not pure computations like utility functions but require an agent-continuation which means they have access to the agent state, environment and random-number stream. This allows testing to capture the \textit{complete} system state in one location, which allows the checking of much more invariants than in approaches which have implicit side-effects.

We implement custom data-generators for our agent state and environment and its cells and then let QuickCheck generate the random data and us running the agent with the provided data, checking for the properties. An example for such a property is that an agent has starved to death in case its sugar (or spice) level has dropped to 0. The corresponding property-test generates a random agent state and also a random sugar level which we set in the agent state. We then run the function which returns True in case the agent has starved to death. We can then check that this flag is true only if the initial random sugar level was less then or equal 0.

%What is particularly powerful is that one has complete control and insight over the changed state before and after e.g. a function was called on an agent: thus it is very easy to check if the function just tested has changed the agent-state itself or the environment: the new environment is returned after running the agent and can be checked for equality of the initial one - if the environments are not the same, one simply lets the test fail. This behaviour is very hard to emulate in OOP because one can not exclude side-effect at compile time, which means that some implicit data-change might slip away unnoticed. In FP we get this for free.

\subsection{Emergent Properties}
In our validation and verification process of our Sugarscape implementation we put informal descriptions and hypotheses about emergent properties from the Sugarscape book into formal property-tests. Examples for such hypotheses / informal descriptions of emergent properties are e.g. the carrying capacity becomes stable after 100 steps; when agents trade with each other after 1000 steps the standard deviation of trading prices is less than 0.05; when there are cultures after 2700 steps either one culture dominates the other or both are equally present.

The property we test for is whether \textit{the emergent property under test is stable under varying random-number seeds} or not. Put another way, we let QuickCheck generate random number generators and require that the tests all pass with arbitrary random number streams. Unfortunately this revealed that this property didn't hold for all emergent properties. The problem is that QuickCheck generates by default 100 test-cases in for each property-test where all need to pass for the whole property-test to pass - this wasn't the case, where most of the 100 test-cases passed but unfortunately not all. Thus in this case a different approach is required: instead of requiring \textit{every} test to pass we require that \textit{most} tests pass, which can be achieved using a t-test with a confidence interval of e.g. 95\%. This means we won't use QuickCheck anymore and resort to a normal unit-test where we run the simulation 100 times with different random number streams each time and then performing a t-test with a 95\% confidence interval. Note that we are now technically speaking of a unit-test but conceptually it is still a property-test.

In listing \ref{alg:prop_test_trading} we show the pseudo code of a property-test for checking whether after 1000 steps the standard deviation of trading prices is less than 0.05. The test passes if out of 100 runs a 95\% confidence interval is reached using a t-test.

\begin{algorithm}
maxTicks = 1000\;
replications = 100\;
stdAverage = 0.05\;
tradingPriceStdsList = empty list\;

\For{$i\leftarrow 1$ \KwTo replications}{
rng = new random number generator\;
simContext = initSimulation rng\;
out = runSimulation maxTicks simContext\;
tps = extractTradingPrices out\;
tpsStd = calculate standard deviation of tps\;
insert tpsStd into tradingPriceStdsList\;
}

tTestPass = perform 1-sided t-test comparing stdAverage with tradingPriceStdsList on a 0.95 interval\;

\eIf{tTestPass}{
  PASS\;
} {
  FAIL\;
}
\caption{Property-based test for trading prices.}
\end{algorithm}
\label{alg:prop_test_trading}

\chapter{Conclusions}
\label{ch:conclusions}

This chapter concludes the whole thesis and outlines future research. Roughly 20\% exists already.

%we now know how to engineer time- and event-driven ABS with complex state both in the agent and environment, main difficulty is direct agent-interaction (see macal classification into 4 types of ABS), compile-time guaranteed reproducibility, explicit handling of complex state (read only, read/write), concurrency explicit and limited to STM, very promising concurrency but direct agent-interactions main problem (erlang as a rescue?), main drawbacks: everything is explicit, performance

\section{Further Research}
clearly outline the ideas for further research

\subsection{A general purpose library}
generalise concepts explored into a pure functional ABS library in Haskell (called chimera)

\subsection{Dependent and linear types}
dependent types and linear types are the next big step, towards a stronger formalisation of agents and ABS,
focus on the equilibrium - totality correspondence

\subsection{Concurrent event-driven ABS}
stm based concurrency for event-driven ABS using parallel DES. challenge is the time-warp implementation using monads. in general it should be easy to roll-back agents actions but with monads we have to be careful - for some monads rolling back is not neccessary e.g. rand and reader, for others it is, and for some it is impossible e.g. IO

\section{Further Research}
\label{sec:further_research}

We see this paper as an intermediary and necessary step towards dependent types for which we first needed to understand the potentials and limitations of a non-dependently typed pure functional approach in Haskell. Dependent types are extremely promising in functional programming as they allow us to express stronger guarantees about the correctness of programs and go as far as allowing to formulate programs and types as constructive proofs \cite{wadler_propositions_2015} which must be total by definition \cite{thompson_type_1991}, \cite{altenkirch_why_2005}, \cite{altenkirch_pi_2010}, \cite{program_homotopy_2013}. So far no research using dependent types in agent-based simulation exists at all and it is not clear whether dependent types make sense in this context. In our next paper we want to explore this for the first time and ask more specifically how we can add dependent types to our pure functional approach, which conceptual implications this has for ABS and what we gain from doing so. We plan on using Idris \cite{brady_idris_2013}, \cite{brady_type-driven_2017} as the language of choice as it is very close to Haskell with focus on real-world application and running programs as opposed to other languages with dependent types e.g. Agda and Coq which serve primarily as proof assistants.
It would be of immense interest whether we could apply dependent types to the model meta-level or not - this boils down to the question if we can encode our model specification in a dependent type way. This would allow the ABS community for the first time to reason about a proper formalisation of a model. We plan to implement a total and terminating implementation of our approach which would be a formal proof-by-construction that the agent-based approach of the SIR model terminates after a finite number of steps.

\begin{acks}
The authors would like to thank
\end{acks}

% Bibliography
\bibliographystyle{ACM-Reference-Format}
%% Citation style
%% Note: author/year citations are required for papers published as an
%% issue of PACMPL.
%%\citestyle{acmauthoryear}   %% For author/year citations
\bibliography{../../../references/phdReferences.bib}

\end{document}
