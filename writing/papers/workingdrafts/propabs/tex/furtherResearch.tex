\section{Further Research}
\label{sec:further}
TODO


Generally speaking, a dependent type is a type whose definition depends on a value e.g. when we have a pair of natural numbers where the second one is greater than the first, we speak of a dependent type. With this power, they allow to push compile-time guarantees to a new level where we can express nearly arbitrary complex guarantees at compile-time because we can \textit{compute types at compile-time}. This means that types are first-class citizen of the language and go as far as being formal proofs of the correctness of an implementation, allowing to narrow the gap between specification and implementation substantially.

We hypothesise that the use of dependent types allows us to push the judgement of the correctness of a simulation to new, unprecedented level, not possible with the established object-oriented approaches so far. This has the direct consequence that the development process is very different and can reduce the amount of testing (both unit-testing and manual testing) substantially. Because one is implementing a simulation which is (as much as possible) correct-by-construction, the correctness (of parts) can be guaranteed statically.

 %"Will gladly sacrifice 10\% of our performance for 10\% higher productivity"

Summarizing, we expect the following benefits from adding dependent types to ABS:

\begin{enumerate}
	\item Narrowing the gap between the model specification and its implementation reduces the potential for conceptual errors in model-to-code translation.
	\item Less number of tests required due to guarantees being expressed already at compile time.
	\item Higher confidence in correctness due to formal guarantees in code.
\end{enumerate}
