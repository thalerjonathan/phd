\section{Deriving a functional approach}
%2 main points from Henrik: 1.: try to stick to synchronous updates because this is how the real world works. 2.: get rid of globally shared mutable state as it complicates things extremely with reasoning
%clear conceptual formal model of what agents are and then structure my implementation around it

We presented a high-level agent-based approach to the SIR model in the previous section, which focused only on the states and the transitions, but we haven't talked about technical implementation details on how to actually implement such a state-machine. The authors of \cite{thaler_art_2017} discuss two fundamental problems of implementing an agent-based simulation from a programming language agnostic view. The first problem is how agents can be pro-active and the second how interactions between agents can happen. For agents to be pro-active they must be able to perceive the passing of time, which means there must be a concept of an agent-process which executes over time. Interactions between agents can be reduced to the problem of how an agent can expose information about its internal state which can be perceived by other agents. \\
In this section we will derive a pure functional approach for an agent-based simulation of the SIR model in which we will pose solutions to the previously mentiond problems. We will start out with a very naive approach and show its limitations which we overcome by adding FRP. Then in further steps we will add more concepts and generalisations, ending up at the final approach which utilises monadic stream functions (MSF), a generalisation of FRP.

\subsection*{Step I: Naive beginnings}
In our first step we start with modelling the states of the agents for which we simply use an Algebraic Data Type (ADT):

\begin{minted}[fontsize=\footnotesize]{haskell}
data SIRState = Susceptible | Infected | Recovered
\end{minted}

Also agents are ill for some duration meaning we need to keep track when a potentially infected agent recovers. Also as previously mentioned, a simulation is stepped in discrete or continuous time-steps thus we introducing a notion of \textit{time} and $\Delta t$ by defining:

\begin{minted}[fontsize=\footnotesize]{haskell}
type Time      = Double
type TimeDelta = Double
\end{minted}

Then we can represent every agent simply as a tuple of its state and its potential recovery time. We hold all our agents simply in a list and define helper functions:
\begin{minted}[fontsize=\footnotesize]{haskell}
type SIRAgent = (SIRState, Time)
type Agents   = [SIRAgent]

is :: SIRState -> SIRAgent -> Bool
is s (s',_) = s == s'

susceptible :: SIRAgent
susceptible = (Susceptible, 0)

infected :: Time -> SIRAgent
infected t = (Infected, t)

recovered :: SIRAgent
recovered = (Recovered, 0)
\end{minted}

Next we need to think about how to actually step our simulation. For this we define a function which simply steps our simulation with a fixed $\Delta t$ until a given time $t$ where in each step the agents are processed and the output is fed back into the next step.
TODO: need a much better explanation and maybe split up into more steps?
As already mentioned in previous sections, the agent-based implementation of the SIR model is inherently stochastic which means we need access to a random-number generator. We decided to use the Rand Monad at this point as threading a generator through the simulation and the agents is very cumbersome. Thus our simulation stepping runs in the Rand Monad:

\begin{minted}[fontsize=\footnotesize]{haskell}
runSimulation :: RandomGen g 
              => Time 
              -> TimeDelta 
              -> Agents 
              -> Rand g [Agents]
runSimulation tEnd dt as = runSimulationAux 0 dt as []
  where
    runSimulationAux :: RandomGen g 
                     => Time 
                     -> TimeDelta 
                     -> Agents 
                     -> [Agents] 
                     -> Rand g [Agents]
    runSimulationAux t dt as acc
      | t >= tEnd = return $ reverse (as : acc)
      | otherwise = do
        as' <- stepSimulation dt as 
        runSimulationUntilAux (t + dt) dt as' (as : acc)

stepSimulation :: RandomGen g => TimeDelta -> Agents -> Rand g Agents
stepSimulation dt as = mapM (processAgent dt as) as
\end{minted}

Now we can implement the behaviour of an individual agent.
It is important to try to infect the agent for every infected contact and not for a signle one.

\begin{minted}[fontsize=\footnotesize]{haskell}
processAgent :: RandomGen g 
             => TimeDelta 
             -> Agents 
             -> SIRAgent 
             -> Rand g SIRAgent
processAgent _  as    (Susceptible, _) = susceptibleAgent as
processAgent dt _   a@(Infected   , _) = return $ infectedAgent dt a
processAgent _  _   a@(Recovered  , _) = return a
\end{minted}

An agent gets fed all the agents states so it can draw random contacts. Note that this includes also the agent itself thus we would need to omit the agent itself to prevent making contact with itself. We decided against that as it complicates the solution and for larger numbers of agent population the probability for an agent to make contact with itself is so small that it can be neglected.

From our implementation it becomes apparent that only the behaviour of a susceptible agent involves randomness and that a recovered agent is simply a sink: it does nothing - its state stays constant.

Lets look how we can implement the behaviour of a susceptible agent. It simply makes contact on average with a number of other agents and gets infected with a given probability if an agent it has contact with is infected.
When the agent gets infected it calculates also its time of recovery by drawing a random number from the exponential distribution meaning it is ill on average for illnessDuration.

\begin{minted}[fontsize=\footnotesize]{haskell}
susceptibleAgent :: RandomGen g => Agents -> Rand g SIRAgent
susceptibleAgent as = do
    rc <- randomExpM (1 / contactRate)
    cs <- doTimes (floor rc) (makeContact as)
    if elem True cs
      then infect
      else return susceptible

  where
    makeContact :: RandomGen g => Agents -> Rand g Bool
    makeContact as = do
      randContact <- randomElem as
      if (is Infected randContact)
        then randomBoolM infectivity
        else return False

    infect :: RandomGen g => Rand g SIRAgent
    infect = do
      randIllDur  <- randomExpM (1 / illnessDuration)
      return $ infected randIllDur
\end{minted}

The infected agent is trivial. It simply recovers after the given illness duration which is implemented as follows:

\begin{minted}[fontsize=\footnotesize]{haskell}
infectedAgent :: TimeDelta -> SIRAgent -> SIRAgent
infectedAgent dt (_, t) 
    | t' <= 0   = recovered
    | otherwise = infected t'
  where
    t' = t - dt  
\end{minted}

\subsubsection{Results}
TODO: discuss $\Delta t = 1.0$ and until t = 150. Also cannot vary delta

\subsubsection{Pros}
first approach, it works, having random-monad is VERY convenient

\subsubsection{Cons}
- time is explicitly available and needs to be dealt with explicitly
- all agent-states are fed back into every agent
- agent representation and function is not very elegant 


\subsection{Step II: Adding FRP}
As shown in the first step, the need to handle $\Delta t$ explicitly can be quite messy, is unelegant and a potential source of errors, also the explicit handling of the state of an agent and its behavioural function is not very functional. In this step we will 

\item $\Delta t$ is passed explicitly as argument to the agent and needs to be dealt with explicitly. It seems to be not very elegant and a potential source of errors - can we do better and find a more elegant solution? 
	\item The way our agents are represented is not very elegant: the state of the agent is explicitly encoded in an ADT and when processing the agent the function needs always first distinguish between the states. Can we express it in a more implicit, functional way?
	
- time is implicit and cannot be messed with
- agent can switch their behaviour

\subsection{Step III: Adding data-flow}
In this step we will introduce a data-flow mechanism between agents which makes the feedback explicit. As already mentioned in the previous step, by revealing the state of every agent to all other agents makes the interactions implicit and deprives the agent of its control over which other agent sees its data. 
As a remedy we introduce data-flows which allow an agent to send arbitrary data to other agents. The data will be collected from the sending agents and distributed to the receivers after each step, which means that we have a delay of one $\Delta t$ and a round-trip takes $2 \Delta t$ - which is exactly the behaviour we had before: feedback.
This change requires then a different approach of how the agents interact with each other: a susceptible agent then sends to a random agent a data-flow indicating a contact. Only infected agents need to reply to such contact requests by revealing that they are infected. The susceptible agents then need to check for incoming replies which means they were in contact with an infected agent.

First we need a way of addressing agents, which we do by introducing unique agent ids. Also we need a data-package which identifies the receiver and carries the data:
\begin{minted}[fontsize=\footnotesize]{haskell}
type AgentId     = Int
type AgentData d = (AgentId, d)
\end{minted}

Next we need more general input and output types of our agents signal-functions. We introduce a new input type which holds both the agent-id of the agent and the incoming data-flows from other agents:

\begin{minted}[fontsize=\footnotesize]{haskell}
data AgentIn d = AgentIn
  {
    aiId    :: !AgentId
  , aiData  :: ![AgentData d]
  } 
\end{minted}

We also introduce a new output type which holds both the outgoing data-flows to other agents and the observable state the agent wants to reveal to the outside world:

\begin{minted}[fontsize=\footnotesize]{haskell}
data AgentOut o d = AgentOut
  {
    aoData        :: ![AgentData d]
  , aoObservable  :: !o
  }
\end{minted}

Note that by making the observable state explicit in the types we give the agent further control of what it can reveal to the outside world which allows an even stronger separation between the agents internal state / data and what the agent wants the world to see.

Now we can then generalise the agents signal-functions to the following type:
\begin{minted}[fontsize=\footnotesize]{haskell}
type Agent o d    = SF (AgentIn d) (AgentOut o d)
\end{minted}

For our SIR implementation we need concrete types, so we need to define what the type parameters \textit{o} and \textit{d} are. For \textit{d} we simply define an ADT which defines a contact-message, and for \textit{o} which defines the type of the observable state, we use the existing SIR-state. Now we can define the type synonyms for our SIR implementation:
\begin{minted}[fontsize=\footnotesize]{haskell}
data SIRMsg      = Contact SIRState deriving (Show, Eq)
type SIRAgentIn  = AgentIn SIRMsg
type SIRAgentOut = AgentOut SIRState SIRMsg
type SIRAgent    = SF SIRAgentIn SIRAgentOut
\end{minted}

Obviously the existing implementation from step 2 needs to be adjusted. Lets look at the initial agent-behaviour:

\begin{minted}[fontsize=\footnotesize]{haskell}
sirAgent :: RandomGen g => g -> [AgentId] -> SIRState -> SIRAgent
sirAgent g ais  Susceptible = susceptibleAgent g ais
sirAgent g _    Infected    = infectedAgent g
sirAgent _ _    Recovered   = recoveredAgent
\end{minted}

It still takes a random-number generator, the initial sir-state and returns the corresponding signal-function depending on the state now in addition it now takes a list of agent ids. When using data-flow we need to know the ids of the agents we are communicating with - we need to know our neighbourhood, or seen differently: we need to have access to the environment we are situated in. In our case our environment is a fully connected read-only network in which all agents know all other agents. The easiest way of representing a fully connected network is simply using a list.
Again when we look at the functions which are returned we see that recovered agent is still the same: it is a sink which ignores the environment and the random-number generator. 

\begin{minted}[fontsize=\footnotesize]{haskell}
recoveredAgent :: SIRAgent
recoveredAgent = arr (const TODO DOLLAR agentOut Recovered)
\end{minted}

The implementation is nearly the same as in step 2 but instead of returning only the sir-state now the output of an agents signal-function is of type \textit{AgentOut}:

\begin{minted}[fontsize=\footnotesize]{haskell}
agentOut :: o -> AgentOut o d
agentOut o = AgentOut {
    aoData        = []
  , aoObservable  = o
  }
\end{minted}

The behaviour of the infected agent now explicitly ignores the environment which was not apparent in step 2 on this level:

\begin{minted}[fontsize=\footnotesize]{haskell}
infectedAgent :: RandomGen g => g -> SIRAgent
infectedAgent g = 
    switch
    infected 
      (const recoveredAgent)
  where
    infected :: SF SIRAgentIn (SIRAgentOut, Event ())
    infected = proc ain -> do
      recEvt <- occasionally g illnessDuration () -< ()
      let a = event Infected (const Recovered) recEvt
      let ao = respondToContactWith Infected ain (agentOut a)
      returnA -< (ao, recEvt)
\end{minted}

The implementation of the infected agent now basically works the same as in step 2 but it additionally needs to reply to incoming contact data-flows with an "Infected" reply. This makes the difference to step 2 very explicit: in the data-flow approach agents now make explicit contact with each other which means that the susceptible agent sends out contact data-flows to which only infected agents need to reply.
Note that at the moment of recovery the agent can still infect others because it will still reply with Infected. The response mechanism is implemented in \textit{respondToContactWith}:

\begin{minted}[fontsize=\footnotesize]{haskell}
respondToContactWith :: SIRState -> SIRAgentIn -> SIRAgentOut -> SIRAgentOut
respondToContactWith state ain ao = onData respondToContactWithAux ain ao
  where
    respondToContactWithAux :: AgentData SIRMsg -> SIRAgentOut -> SIRAgentOut
    respondToContactWithAux (senderId, Contact _) ao = dataFlow (senderId, Contact state) ao
    
onData :: (AgentData d -> acc -> acc) -> AgentIn d -> acc -> acc
onData dHdl ai a = foldr (\msg acc'-> dHdl msg acc') a (aiData ai)

dataFlow :: AgentData d -> AgentOut o d -> AgentOut o d
dataFlow df ao = ao { aoData = df : aoData ao }
\end{minted}

Note that the order of data packages in a data-flow is not specified and must not matter as it happens virtually at the same time, thus we always append at the front of the outgoing data-flow list.

Lets look at the susceptible agent behaviour. As already mentioned before, the feedback interaction between agents works now very explicit due to the data-flow but needs a different approach in our implementation:

\begin{minted}[fontsize=\footnotesize]{haskell}
susceptibleAgent :: RandomGen g => g -> [AgentId] -> SIRAgent
susceptibleAgent g ais = 
    switch 
      (susceptible g) 
      (const TODO DOLLAR infectedAgent g)
  where
    susceptible :: RandomGen g 
                  => g 
                  -> SF SIRAgentIn (SIRAgentOut, Event ())
    susceptible g0 = proc ain -> do
      rec
        g <- iPre g0 -< g'
        let (infected, g') = runRand (gotInfected infectivity ain) g

      if infected 
        then returnA -< (agentOut Infected, Event ())
        else (do
          makeContact <- occasionally g (1 / contactRate) ()  -< ()
          contactId   <- drawRandomElemSF g                   -< ais

          if isEvent makeContact
            then returnA -< (dataFlow (contactId, Contact Susceptible) TODO DOLLAR agentOut Susceptible, NoEvent)
            else returnA -< (agentOut Susceptible, NoEvent))
            
gotInfected :: RandomGen g => Double -> SIRAgentIn -> Rand g Bool
gotInfected p ain = onDataM gotInfectedAux ain False
  where
    gotInfectedAux :: RandomGen g => Bool -> AgentData SIRMsg -> Rand g Bool
    gotInfectedAux False (_, Contact Infected) = randomBoolM p
    gotInfectedAux x _ = return x
    
onDataM :: (Monad m) 
        => (acc -> AgentData d -> m acc) 
        -> AgentIn d 
        -> acc 
        -> m acc
onDataM dHdl ai acc = foldM dHdl acc (aiData ai)
\end{minted}

Again the implementation is very similar to step 2 with the fundamental difference how contacts are made and how infections occur. First the agent checks if it got infected. This happens if an infected agent replies to the susceptible agents contact AND the susceptible agent got infected with the given probability. Note that \textit{gotInfected} runs in the Random-Monad which we run using \textit{runRand} and the random-number generator. To update our random-number generator to the changed one, we use the \textit{rec} keyword of Arrows, which allows us to refer to a variable after it is defined. In combination with iPre we introduced a local state - the random-number generator - which changes in every step.
If the agent got infected, it simply returns an AgentOut with Infected as observable state and a switching event which indicates the switch to infected behaviour.
If the agent is not infected it draws from occasionally to determine if it should make contact with a random agent. In case it should make contact it simply sends a data-package with the contact susceptible data to the receiver - only an infected agent will reply.

Stepping the simulation now works a little bit different as the input/output types have changed and we need to collect and distribute the data-flow amongst the agents:

\begin{minted}[fontsize=\footnotesize]{haskell}
stepSimulation :: [SIRAgent] -> [SIRAgentIn] -> SF () [SIRAgentOut]
stepSimulation sfs ains =
    dpSwitch
      (\_ sfs' -> (zip ains sfs'))
      sfs
      (switchingEvt >>> notYet)
      cont

  where
    switchingEvt :: SF ((), [SIRAgentOut]) (Event [SIRAgentIn])
    switchingEvt = proc (_, aos) -> do
      let ais      = map aiId ains
          aios     = zip ais aos
          nextAins = distributeData aios
      returnA -< Event nextAins

    cont :: [SIRAgent] -> [SIRAgentIn] -> SF () [SIRAgentOut]
    cont sfs nextAins = stepSimulation sfs nextAins
\end{minted}

The distribution of the data-flows happens in the \textit{switchingEvt} signal-function and is then passed on to the continuation-generation function as in step 2. The difference is that it creates now a list of AgentIn for the next step instead of a list of all the agents sir-states of the previous step. Again the continuation-generation function recursively returns the stepSimulation signal-function. The pairing function of pSwitch is now slightly more straightforward as it just pairs up the AgentIn with its corresponding signal-function.

\subsubsection{Reflection}
It seems that by introducing the data-flow mechanism we have complicated things but this is not so. Data-flows make the feedback between agents explicit and gives the agents full control over the data which is revealed to other agents. This also makes the fact even more explicit, that we cannot fix the connections between the agents already at compile time e.g. by connecting SFs which is done in many Yampa applications (TODO: cite Henrik papers) because agents interact with each other randomly. One can look at the data-flow mechanism as a kind of messaging but there are fundamental differences: messaging almost always comes up as an approach to managing concurrency and involves stateful message-boxes which can be checked an emptied by the receivers - this is not the case with the data-flow mechanism which behaves indeed as a flow where data is not stored in a messagebox but is only present in the current simulation-step and if ignored by the agent will be gone in the next step.
Also by distinguishing by the internal and the observable state of the agent, we give the agent even more control of what is visible to the outside world.
So far we have a pretty decent implementation of an agent-based SIR approach. The next three steps focus - as this 3rd one - on introducing more concepts and generalising our implementation so far. What we are lacking at the moment is a general treatment of the environment and of synchronised transactional behaviour between agents. To be able to conveniently introduce both we want to make use of monads which is not possible using Yampa. In the next step we make the transition to Monadic Stream Functions (MSF) as introduced in Dunai \cite{perez_functional_2016}. The authors of Dunai implement BearRiver which is a re-implementation of Yampa on top of MSF which should allow us to easily replace Yampa with MSFs in our implementation of Step 3.

\subsection{Step IV: Generalising to Monadic Stream Functions}
TODO: write a bit introductory words for this subsection

\subsubsection{Identity Monad}
We start by making the transition to BearRiver by simply replacing Yampas signal-function by BearRivers which is the same but takes an additional type-parameter \textit{m} which indicates the monad. If we replace this type-parameter with the identity monad we should be able to keep the code exactly the same, except from a few type-declarations, because BearRiver re-implements all necessary functions we are using from Yampa \footnote{This was not quite true at the time we wrote this paper, where \textit{occasionally}, \textit{noiseR} and \textit{dpSwitch} were missing. We simply implemented these functions and created a pull request using git.}.
We start by re-defining our general agent signal-function, introducing the monad (stack) our SIR implementation runs in and the sir-agents signal-function:

\begin{minted}[fontsize=\footnotesize]{haskell}
type Agent m o d = SF m (AgentIn d) (AgentOut o d)
type SIRMonad    = Identity
type SIRAgent    = Agent SIRMonad SIRState SIRMsg
\end{minted}

We also have to add the \textit{SIRMonad} to the existing \textit{stepSimulation} type-declarations and we are nearly done. The function \textit{embed} for running the simulation is not provided by BearRiver but by Dunai which has important implications. Dunai does not know and care about time in MSFs, which is exactly what BearRiver builds on top of MSFs. It does so by adding a \textit{ReaderT Double} which carries the $\Delta t$. This means that \textit{embed} returns a computation in the \textit{ReaderT Double} Monad which we need to run explicitly using \textit{runReaderT}. This then results in an identity computation which we simply peel away using \textit{runIdentity}. Here is the complete code of \textit{runSimulation}:

\begin{minted}[fontsize=\footnotesize]{haskell}
runSimulation :: RandomGen g
              => g 
              -> Time 
              -> DTime 
              -> [(AgentId, SIRState)] 
              -> [[SIRState]]
runSimulation g t dt as = map (map aoObservable) aoss
  where
    steps = floor TODO DOLLAR t / dt
    dts = replicate steps ()
    n = length as

    (rngs, _) = rngSplits g n []
    ais = map fst as
    sfs = map (\ (g', (_, s)) -> sirAgent g' ais s) (zip rngs as)
    ains = map (\ (aid, _) -> agentIn aid) as

    aossReader = embed (stepSimulation sfs ains) dts
    aossIdentity = runReaderT aossReader dt
    aoss = runIdentity aossIdentity
\end{minted}

Note that embed does not take a list of $\Delta t$ any more but simply a list of inputs for each step to the top level signal-function.

\subsubsection{Random Monad}
Using the Identity Monad does not gain us anything but it was a first step towards a more general solution. Our next step is to replace the Identity Monad by the Random Monad which will allow us to get rid of the RandomGen arguments to our functions and run the whole simulation within the RandomMonad \textit{again} just as we started but now with the full features functional reactive programming!
We start by re-defining the SIRMonad and SIRAgent:

\begin{minted}[fontsize=\footnotesize]{haskell}
type SIRMonad g = Rand g
type SIRAgent g = Agent (SIRMonad g) SIRState SIRMsg
\end{minted}

Note that we parametrise the Random Monad with a RandomGen g thus this requires to add the RandomGen type-class to all functions where it was not yet added. We also simply remove all RandomGen arguments to all functions except \textit{runSimulation}. The question is now how to access this random monad functionality within the MSF context.
For the function \textit{occasionally}, there exists a monadic pendant \textit{occasionallyM} which requires a MonadRandom type-class. Because we are now running within a MonadRandom instance we simply replace \textit{occasionally} with \textit{occasionallyM}.
Running \textit{gotInfected} is now much easier. Using the function \textit{arrM} of Dunai allows us to run a monadic action in the stack as an arrow. We then directly run gotInfected by lifting it into the random-monad.
This can be seen in the susceptible agent running in the random monad SF:
\begin{minted}[fontsize=\footnotesize]{haskell}
susceptibleAgent :: RandomGen g => [AgentId] -> SIRAgent g
susceptibleAgent ais = 
    switch 
      susceptible
      (const TODO DOLLAR infectedAgent)
  where
    susceptible :: RandomGen g 
                => SF (SIRMonad g) SIRAgentIn (SIRAgentOut, Event ())
    susceptible = proc ain -> do
      infected <- arrM (lift . gotInfected infectivity) -< ain

      if infected 
        then returnA -< (agentOut Infected, Event ())
        else (do
          makeContact <- occasionallyM (1 / contactRate) () -< ()
          contactId   <- drawRandomElemSF                   -< ais

          if isEvent makeContact
            then returnA -< (dataFlow (contactId, Contact Susceptible) TODO DOLLAR agentOut Susceptible, NoEvent)
            else returnA -< (agentOut Susceptible, NoEvent))
\end{minted}

Note also that \textit{drawRandomElemSF} doesn't take a random number generator as well as it has been reimplemented to make full use of the MonadRandom in the stack:

\begin{minted}[fontsize=\footnotesize]{haskell}
drawRandomElemS :: MonadRandom m => SF m [a] a
drawRandomElemS = proc as -> do
  r <- getRandomRS ((0, 1) :: (Double, Double)) -< ()
  let len = length as
  let idx = fromIntegral len * r
  let a =  as !! floor idx
  returnA -< a
\end{minted}

Instead of \textit{noiseR} which requires a RandomGen, it makes use of Dunai \textit{getRandomRS} stream function which simply runs \textit{getRandomR} in the MonadRandom.

Finally because our innermost monad is now the Random Monad instead of the Identity in \textit{runSimulation} we need to replace \textit{runIdentity} by \textit{evalRand}:

\begin{minted}[fontsize=\footnotesize]{haskell}
aossReader = embed (stepSimulation sfs ains) dts
  aossRand = runReaderT aossReader dt
      aoss = evalRand aossRand g
\end{minted}

\subsubsection{Reflections}
By making the transition to MSFs we can now stack arbitrary number of Monads. As an example we could add a StateT monad on the type of AgentOut which would allow to conveniently manipulate the AgentOut e.g. in case where one sends more than one message or the construction of the final AgentOut is spread across multiple functions which allows easy composition. When implementing this one needs to replace the dpSwitch with an individual implementation in which one runs the state monad isolated for each agent.
We could even add the IO monad if our agents require arbitrary IO e.g. reading/writing from files or communicating over TCP/IP. Although one could run in the IO monad, one should not do so as we would loose all guarantees about the reproducibility of our simulation. In ABS we need deterministic behaviour under all circumstances where repeated runs with the same initial conditions, including the random-number generator, should result in the same dynamics. If we allow IO we loose the ability to guarantee the reproducibility at compile-time even if the agents never use IO facilities and just run in the IO for printing debug messages.
So far making the transition to MSFs does not seem as compelling as making the move from the RandomMonad in step 1 to FRP in step 2. Running in the RandomMonad within FRP is convenient but we could achieve the same with passing RandomGen around as we showed in Step 3. In the next step we introduce the concept of a read/write environment which we realise using a StateT monad. This will show the real benefit of the transition to MSFs as without it, implementing a general Environment access would be quite cumbersome.

\subsection{Step V: Adding an environment}
we already have one variant of possible environment scenarios: the read-only one. Now we will introduce a different approach of communication between the agents by introducing a read/write environment.
adding Environment: when in in/out its cumbersome, end up with n copies, pro-active Environment needs to be hacked in, have additional complexities

% this is omited for now as it takes too much time and involves too many details but would need much more explanation still
%\subsection{Step VI: Adding agent transactions}
Imagine two agents A and B want to engage in a bartering process where agent A, is the seller who wants to sell an asset to agent B who is the buyer. Agent A sends Agent B a sell offer depending on how much agent A values this asset. Agent B receives this sell offer, checks if the price satisfies its utility, if it has enough wealth to buy the asset and replies with either a refusal or its own price offer. Agent A then considers agent Bs offer and if it is happy it replies to agent B with an acceptance of the offer, removes the asset from its inventory and increases its wealth. Agent B receives this acceptance offer, puts the asset in its inventory and decreases its wealth (note that this process could involve a potentially arbitrary number of steps without loss of generality).
We can see this behaviour as a kind of multi-step transactional behaviour because agents have to respect their budget constraints which means that they cannot spend more wealth or assets than they have. This implies that they have to 'lock' the asset and the amount of cash they are bartering about during the bartering process. If both come to an agreement they will swap the asset and the cash and if they refuse their offers they have to 'unlock' them.
In classic OO implementations it is quite easy to implement this as normally only one agent is active at a time due to sequential (discrete event scheduling approach) scheduling of the simulation. This allows then agent A which is active, to directly interact with agent B through method calls. The sequential updating ensures that no other agent will touch the asset or cash and the direct method calls ensure a synchronous updating of the mutable state of both objects with no time passing between these updates.

\subsubsection{Implementation}
We start with the implementation of step 4 with the Random Monad and remove the data-flows from AgentIn and AgentOut. We then add a field in AgentOut which allows the agent to indicate that it wants to start a transaction with another agent with an initial data-package. Also we add a field in AgentIn which indicates an incoming transaction request from another agent with the given data-package. In addition we need another field in AgentOut which allows the agent to indicate that it accepts the incoming request:

\begin{minted}[fontsize=\footnotesize]{haskell}
data AgentIn d = AgentIn
  {
    aiId        :: !AgentId
  , aiRequestTx :: !(Event (AgentData d))
  } deriving (Show)

data AgentOut m o d = AgentOut
  {
    aoObservable :: !o
  , aoRequestTx  :: !(Event (AgentData d, AgentTX m o d))
  , aoAcceptTx   :: !(Event (d, AgentTX m o d))
  }
\end{minted}

We run the transactions in the specialised agent-transaction signal-functions \textit{AgentTX} with different input and output types. This allows us to restrict the possible actions of an agent within a transaction:

\begin{minted}[fontsize=\footnotesize]{haskell}
type AgentTX m o d = SF m (AgentTXIn d) (AgentTXOut m o d)
\end{minted}

The input \textit{AgentTXIn} to an agent-transaction holds optional data and flags which indicate that the other agent has either committed or aborted the transaction.

\begin{minted}[fontsize=\footnotesize]{haskell}
data AgentTXIn d = AgentTXIn
  { aiTxData   :: Maybe d
  , aiTxCommit :: Bool
  , aiTxAbort  :: Bool
  }
\end{minted}

The output \textit{AgentTXOut} of an agent-transaction hold optional data a flag to abort the transaction and optional commit data which is Just in case the agent wants to commit. When committing the agent has to provide a potentially changed AgentOut and optionally a new agent behaviour signal-function. If the agent provides a signal-function when committing, the behaviour of the agent after the transaction will be this signal-function. If no signal-function is provided then the original one will be used.

\begin{minted}[fontsize=\footnotesize]{haskell}
data AgentTXOut m o d = AgentTXOut
  { aoTxData   :: Maybe d
  , aoTxCommit :: Maybe (AgentOut m o d, Maybe (Agent m o d))
  , aoTxAbort  :: Bool
  }
\end{minted}

We also provide type aliases for our SIR implementation:
\begin{minted}[fontsize=\footnotesize]{haskell}
type SIRMonad g    = Rand g
data SIRMsg        = Contact SIRState deriving (Show, Eq)
type SIRAgentIn    = AgentIn SIRMsg
type SIRAgentOut g = AgentOut (SIRMonad g) SIRState SIRMsg
type SIRAgent g    = Agent (SIRMonad g) SIRState SIRMsg
type SIRAgentTX g  = AgentTX (SIRMonad g) SIRState SIRMsg
\end{minted}

Stepping the simulation is now slightly more complex as in every step we need to run the transactions. Fortunately it is easy to provide customised implementations of MSFs in dunai, which is a bit more tricky in Yampa and requires to expose internals.

\begin{minted}[fontsize=\footnotesize]{haskell}
stepSimulation :: RandomGen g
               => [SIRAgent g]
               -> [SIRAgentIn] 
               -> SF (SIRMonad g) () [SIRAgentOut g]
stepSimulation sfs ains = MSF TODO DOLLAR \_ -> do
  res <- mapM (\ (ai, sf) -> unMSF sf ai) (zip ains sfs)
  let aos  = fmap fst res
      sfs' = fmap snd res

      ais  = map aiId ains
      aios = zip ais aos

  -- this works only because runTransactions is stateless
  -- and runs the SFs with dt = 0
  ((aios', sfs''), _) <- unMSF runTransactions (aios, sfs')

  let aos'  = map snd aios'
      ains' = map agentIn ais
      ct    = stepSimulation sfs'' ains'

  return (aos', ct)
\end{minted}

The implementation of \textit{runTransactions} is quite involved and omitted here because it would require too much space \footnote{The full code to all steps is freely available under: TODO provide link to a stable subfolder of my git repo.}, but we will give a short informal description.
All agents are iterated in an unspecified sequence and if an agent requests a transaction the other agent is looked up and the transaction-pair is run. This is done recursively until there are no transaction requests any-more (note that through the AgentOut of a committed transaction, an agent can request a new transaction within the same time-step). Running a transaction-pair works as follows:
The target agents signal-function is run again (resulting in a second, or third,... execution, depending on how many transactions have this agent as target) but now with a $\Delta t = 0$. The target agent can then accept the incoming transaction or simply ignore it. If it is ignored the transaction will never start. The fact that the target agent signal-function is run more than once within a simulation step but with a $\Delta t = 0$ requires agents to make their actions time-dependent \textit{but} they must listen to incoming transactions independent of time. The implementation of the infected agent below will make this more clear.
When the transaction is accepted the system switches to running the transaction signal-functions after another with passing the data forward and backward between the two agents. It is most important to note that again the signal functions are run with $\Delta t = 0$ because conceptionally transactions happen \textit{instantaneously} without time advancing. This has important implications, and means that we cannot use any time-accumulating function e.g. integral or after within a transaction - simply because it makes no sense as no time passes. If \textit{both} agents commit the transaction their new AgentOuts will replace the ones for the current simulation-step. If either one agent aborts the transaction the current AgentOuts of the current simulation-step will be used.

We provide a sequence diagram of data-flow in a multi-step negotiation as described in the introduction for a visual explanation of the complex protocol which is going on in a transaction.

Now it is time to look at the new agent implementations which use now the agent-transaction mechanism. The recovered agent is exactly the same but the susceptible and infected agent behaviour are very different now. Lets first look at the susceptible agent:

\begin{minted}[fontsize=\footnotesize]{haskell}
susceptibleAgent :: RandomGen g => [AgentId] -> SIRAgent g
susceptibleAgent ais = proc _ -> do
    makeContact <- occasionallyM (1 / contactRate) () -< ()

    if not TODO DLLAR isEvent makeContact
      then returnA -< agentOut Susceptible
      else (do
        contactId <- drawRandomElemS -< ais
        returnA -< requestTx 
                    (contactId, Contact Susceptible) 
                    susceptibleTx
                    (agentOut Susceptible))
  where
    susceptibleTx :: RandomGen g => SIRAgentTX g
    susceptibleTx = proc txIn -> do
      -- should have always tx data
      if hasTxDataIn txIn 
          then (do
            let (Contact s) = txDataIn txIn 
            -- only infected agents reply, but make it explicit
            if Infected /= s
              -- don't commit with continuation, no change in behaviour
              then returnA -< commitTx (agentOut Susceptible) agentTXOut
              else (do
                infected <- arrM (\_ -> lift TODO DOLLAR randomBoolM infectivity) -< ()
                if infected
                  -- commit with continuation as we switch into infected behaviour
                  then returnA -< commitTxWithCont 
                                    (agentOut Infected) 
                                    infectedAgent
                                    agentTXOut
                  -- don't commit with continuation, no change in behaviour
                  else returnA -< commitTx 
                                    (agentOut Susceptible) agentTXOut))
          else returnA -< abortTx agentTXOut
\end{minted}

Instead of using a switch the susceptible agent behaves completely time-dependent and occasionally starts a new agent-transaction with a random agent. The function \textit{susceptibleTx} handles the reply of the other agent. Note that we only commit with a continuation in case the agent becomes infected.

The infected agent is slightly less complex and still uses the switch mechanism:
\begin{minted}[fontsize=\footnotesize]{haskell}
infectedAgent :: RandomGen g => SIRAgent g
infectedAgent = 
    switch
    infected 
      (const recoveredAgent)
  where
    infected :: RandomGen g => SF (SIRMonad g) SIRAgentIn (SIRAgentOut g, Event ())
    infected = proc ain -> do
      recEvt <- occasionallyM illnessDuration () -< ()
      let a = event Infected (const Recovered) recEvt
      -- note that at the moment of recovery the agent can still infect others
      -- because it will still reply with Infected
      let ao = agentOut a

      if isRequestTx ain 
        then (do
          returnA -< (acceptTX 
                      (Contact Infected)
                      (infectedTx ao)
                      ao, recEvt))
        else returnA -< (ao, recEvt)

    infectedTx :: RandomGen g => SIRAgentOut g -> SIRAgentTX g
    infectedTx ao = proc _ -> do
      -- it is important not to commit with continuation as it
      -- would reset the time of the SF to 0. Still occasionally
      -- would work as it does not accumulate time but functions
      -- like after or integral would fail
      returnA -< commitTx ao agentTXOut
\end{minted}

The agent acts time-dependent which in this case is the transition from infected to recovered - if occasionallyM is run with a dt of 0 then no Event can happen (todo: is this really true??). The agent checks on every function call of infected for incoming transactions and accepts them all, independent of the state - only susceptible agents request transactions anyway. The agent simply replies with a Contact Infected and immediately commits the transaction in the transaction signal-function but does not switch into a new continuation.

\subsubsection{Reflection}
Note that the transactions run in the same monad as the normal agent behaviour signal-function which allows to add an environment as in step 5. In this case care must be taken when one has changed the environment but aborts the transaction as a roll back of the environment won't happen automatically. A different approach would allow to run the TX in a different monad and bring in e.g. the  transactional state monad Control.Monad.Tx which supports rolling back of changes to the state.

The concept of agent-transactions is not explicitly known in the agent-based community and a novel development of this paper. The reason for this is that agent-transactions are already implicitly available in traditional OO implementations in which agents can call each others methods and change their state. By implementing this necessary and important concept in a pure functional approach we arrived at agent-transactions which make these synchronous, instantaneous, one-to-one interactions explicit.