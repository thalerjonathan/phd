\section{Related Work}
\label{sec:related_work}
The amount of research on using pure functional programming with Haskell in the field of ABS has been moderate so far. Most of the papers are more related to the field of Multi Agent Systems (MAS) and look into how agents can be specified using the belief-desire-intention paradigm \cite{de_jong_suitability_2014}, \cite{sulzmann_specifying_2007}, \cite{jankovic_functional_2007}.

A library for Discrete Event Simulation (DES) and System Dynamics (SD) in Haskell called \textit{Aivika 3} is described in the technical report \cite{sorokin_aivika_2015}. It is not pure, as it uses the IO Monad under the hood and comes only with very basic features for event-driven ABS, which allows to specify simple state-based agents with timed transitions.

The authors of \cite{jankovic_functional_2007} discuss using functional programming for DES and explicitly mention the paradigm of FRP to be very suitable to DES.

The authors of \cite{vendrov_frabjous:_2014} present a domain-specific language for developing functional reactive agent-based simulations. This language called FRABJOUS is human readable and easily understandable by domain-experts. It is not directly implemented in FRP/Haskell but is compiled to Yampa code which they claim is also readable. This supports that FRP is a suitable approach to implement ABS in Haskell. Unfortunately, the authors do not discuss their mapping of ABS to FRP on a technical level, which would be of most interest to functional programmers.
