\section{Deriving a pure functional approach}
\label{sec:functional_approach}

We presented a high-level agent-based approach to the SIR model in the previous section, which focused only on the states and the transitions, but we haven't talked about technical implementation. 

In \cite{thaler_art_2017} two fundamental problems of implementing an agent-based simulation from a programming-language agnostic point of view is discussed. The first problem is how agents can be pro-active and the second how interactions and communication between agents can happen. For agents to be pro-active, they must be able to perceive the passing of time, which means there must be a concept of an agent-process which executes over time. Interactions between agents can be reduced to the problem of how an agent can expose information about its internal state which can be perceived by other agents. Further the authors have shown the influence of different deterministic and non-deterministic elements in agent-based simulation on the dynamics and how the influence of non-determinism can completely break them down or result in different dynamics despite same initial conditions. This means that we want to rule out any potential source of non-determinism.

In this section we will derive a pure functional approach for an agent-based simulation of the SIR model in which we will pose solutions to the previously mentioned problems. We will start out with a very naive approach and show its limitations which we overcome by adding FRP. Then in further steps we will add more concepts and generalisations, ending up at the final approach which utilises Monadic Stream Functions, a generalisation of FRP. 

Of paramount importance is to keep our implementations pure which rules out the use of the IO Monad and thus any potential source of non-determinism under all circumstances because we would loose all compile time guarantees about reproducibility. Still we will make use of the Random and State Monad which indeed allow side-effects but the crucial point here is that we restrict side-effects only to these types in a controlled way without allowing general unrestricted effects
\footnote{The code of all steps can be accessed freely through the following URL: \url{https://github.com/thalerjonathan/phd/tree/master/public/purefunctionalepidemics/code}}.

\subsection{Naive beginnings}
We start by modelling the states of the agents:

\begin{HaskellCode}
data SIRState = Susceptible | Infected | Recovered
\end{HaskellCode}

Agents are ill for some duration, meaning we need to keep track when a potentially infected agent recovers. Also a simulation is stepped in discrete or continuous time-steps thus we introduce a notion of \textit{time} and $\Delta t$ by defining:

\begin{HaskellCode}
type Time      = Double
type TimeDelta = Double
\end{HaskellCode}

Now we can represent every agent simply as a tuple of its SIR state and its potential recovery time. We hold all our agents in a list:
\begin{HaskellCode}
type SIRAgent = (SIRState, Time)
type Agents   = [SIRAgent]
\end{HaskellCode}

Next we need to think about how to actually step our simulation. For this we define a function which advances our simulation with a fixed $\Delta t$ until a given time $t$ where in each step the agents are processed and the output is fed back into the next step. This is the source of pro-activity as agents are executed in every time step and can thus initiate actions based on the passing of time.
As already mentioned, the agent-based implementation of the SIR model is inherently stochastic which means we need access to a random-number generator. We decided to use the Random Monad at this point as threading a generator through the simulation and the agents could become very cumbersome. Thus our simulation stepping runs in the Random Monad:

\begin{HaskellCode}
runSimulation :: RandomGen g 
  => Time -> TimeDelta -> Agents -> Rand g [Agents]
runSimulation tEnd dt as = runSimulationAux 0 as []
  where
    runSimulationAux :: RandomGen g 
      => Time -> Agents -> [Agents] -> Rand g [Agents]
    runSimulationAux t as acc
      | t >= tEnd = return (reverse (as : acc))
      | otherwise = do
        as' <- stepSimulation dt as 
        runSimulationAux (t + dt) as' (as : acc)

stepSimulation :: RandomGen g 
  => TimeDelta -> Agents -> Rand g Agents
stepSimulation dt as = mapM (runAgent dt as) as
\end{HaskellCode}

Now we can implement the behaviour of an individual agent. First we need to distinguish between the agents SIR states:

\begin{HaskellCode}
runAgent :: RandomGen g 
  => TimeDelta -> Agents -> SIRAgent -> Rand g SIRAgent
runAgent _  as   (Susceptible, _) = susceptibleAgent as
runAgent dt _  a@(Infected   , _) = return (infectedAgent dt a)
runAgent _  _  a@(Recovered  , _) = return a
\end{HaskellCode}

An agent gets fed the states of all agents in the system from the previous time-step so it can draw random contacts - this is one, very naive way of implementing the interactions between agents. 
Note that the list of states includes also the agent itself thus we would need to omit it to prevent making contact with itself. We decided against that as it complicates the solution and for a larger number of agents the probability for an agent to make contact with itself is so small that it can be neglected. Also making contact with the same SIR state never leads to a state change so it really makes no big difference. %Maybe omit this information, as it is too long and not very relevant

From our implementation it becomes apparent that only the behaviour of a susceptible agent involves randomness and that a recovered agent is simply a sink - it does nothing and stays constant.

Lets look how we can implement the behaviour of a susceptible agent. It simply makes contact on average with a number of other agents and gets infected with a given probability if an agent it has contact with is infected.
When the agent gets infected it calculates also its time of recovery by drawing a random number from the exponential distribution meaning it is ill on average for \textit{illnessDuration}.

\begin{HaskellCode}
susceptibleAgent :: RandomGen g => Agents -> Rand g SIRAgent
susceptibleAgent as = do
    -- draws from exponential distribution
    rc <- randomExpM (1 / contactRate) 
    cs <- forM [1..floor rc] (const (makeContact as))
    if elem True cs
      then infect
      else return (Susceptible, 0)
  where
    makeContact :: RandomGen g => Agents -> Rand g Bool
    makeContact as = do
      randContact <- randomElem as
      case fst randContact of
        -- returns True with given probability 
        Infected -> randomBoolM infectivity 
        _        -> return False

    infect :: RandomGen g => Rand g SIRAgent
    infect = randomExpM (1 / illnessDuration) 
               >>= \rd -> return (Infected, rd)
\end{HaskellCode}

The infected agent is trivial. It simply recovers after the given illness duration which is implemented as follows:

\begin{HaskellCode}
infectedAgent :: TimeDelta -> SIRAgent -> SIRAgent
infectedAgent dt (_, t) 
    | t' <= 0   = (Recovered, 0)
    | otherwise = (Infected, t')
  where
    t' = t - dt  
\end{HaskellCode}

\subsubsection{Results}
When running our naive implementation with increasing population sizes we get the dynamics as seen in Figure \ref{fig:sir_abs_dynamics_naive}. With increasing number of agents \cite{macal_agent-based_2010} our solution becomes increasingly smoother and approaches the SD dynamics from Figure \ref{fig:sir_sd_dynamics}  but doesn't quite match them because we are under-sampling the contact-rate. We will will address this problem in the next section.

\begin{figure}
\begin{center}
	\begin{tabular}{c}
		\begin{subfigure}[b]{0.3\textwidth}
			\centering
			\includegraphics[width=1\textwidth, angle=0]{./fig/step1_randmonad/SIR_100agents_150t_1dt.png}
			\caption{100 Agents}
			\label{fig:sir_abs_approximating_1dt_100agents}
		\end{subfigure}
    	
    	\\
    	
		\begin{subfigure}[b]{0.3\textwidth}
			\centering
			\includegraphics[width=1\textwidth, angle=0]{./fig/step1_randmonad/SIR_1000agents_150t_1dt.png}
			\caption{1,000 Agents}
			\label{fig:sir_abs_approximating_1dt_1000agents}
		\end{subfigure}
	\end{tabular}
	
	\caption{Naive simulation of SIR using the agent-based approach. Varying population size, contact rate $\beta = \frac{1}{5}$, infection probability $\gamma = 0.05$, illness duration $\delta = 15$ with initially 1 infected agent. Simulation run for 150 time-steps with fixed $\Delta t = 1.0$.} 
	\label{fig:sir_abs_dynamics_naive}
\end{center}
\end{figure}

\subsubsection{Discussion}
Reflecting on our first naive approach we can conclude that it already introduced most of the fundamental concepts of ABS
\begin{itemize}
	\item Time - the simulation occurs over virtual time which is modelled explicitly divided into \textit{fixed} $\Delta t$ where at each step all agents are executed.
	\item Agents - we implement each agent as an individual, with the behaviour depending on its state.
	\item Feedback - the output state of the agent in the current time-step $t$ is the input state for the next time-step $t + \Delta t$.
	\item Environment - as environment we implicitly assume a fully-connected network (complete graph) where every agent 'knows' every other agents, including itself and thus can make contact with all of them.
	\item Stochasticity - it is an inherently stochastic simulation, which is indicated by the Random Monad type and the usage of \textit{randomBoolM} and \textit{randomExpM}.
	\item Deterministic - repeated runs with the same initial random-number generator result in same dynamics. This may not come as a surprise but in Haskell we can guarantee that property statically already at compile time because our simulation runs in the Random Monad and \textit{not} in the IO Monad. This guarantees that no external, uncontrollable sources of randomness can interfere with the simulation.
	\item Dynamics - with increasing number of agents the dynamics smooth out \cite{macal_agent-based_2010}.
\end{itemize}

Nonetheless our approach has also weaknesses and dangers:
\begin{enumerate}
	\item $\Delta t$ is passed explicitly as argument to the agent and needs to be dealt with explicitly. This is not very elegant and a potential source of errors - can we do better and find a more elegant solution? 
	\item The way our agents are represented is not very modular. The state of the agent is explicitly encoded in an ADT and when processing the agent, the function needs always first distinguish between the states. Can we express it in a more modular way e.g. continuations?
\end{enumerate}

We now move on to the next section in which we will address these points and the under-sampling issue.

\subsection{Functional Reactive Programming}
\label{sec:implement_frp}
As described in the Section \ref{sec:back_frp}, Arrowized FRP \cite{hughes_generalising_2000} is a way to implement systems  with continuous and discrete time-semantics where the central concept is the signal function, which can be understood as a process over time, mapping an input- to an output-signal. Technically speaking, a signal function is a continuation which allows to capture state using closures and hides away the $\Delta t$, which means that it is never exposed explicitly to the programmer, meaning it cannot be messed with.

As already pointed out, agents need to perceive time, which means that the concept of processes over time is an ideal match for our agents and our system as a whole, thus we will implement them and the whole system as signal functions.

\subsubsection{Implementation}
We start by defining the SIR states as ADT and our agents as signal functions (SF) which receive the SIR states of all agents form the previous step as input and outputs the current SIR state of the agent. This definition, and the fact that Yampa is not monadic, guarantees already at compile, that the agents are isolated from each other, enforcing the \textit{parallel} lock-step semantics of the model.

\begin{HaskellCode}
data SIRState = Susceptible | Infected | Recovered

type SIRAgent = SF [SIRState] SIRState 

sirAgent :: RandomGen g => g -> SIRState -> SIRAgent
sirAgent g Susceptible = susceptibleAgent g
sirAgent g Infected    = infectedAgent g
sirAgent _ Recovered   = recoveredAgent
\end{HaskellCode}

Depending on the initial state we return the corresponding behaviour. Note that we are passing a random-number generator instead of running in the Random Monad because signal functions as implemented in Yampa are not capable of being monadic. 

We see that the recovered agent ignores the random-number generator because a recovered agent does nothing, stays immune forever and can not get infected again in this model. Thus a recovered agent is a consuming state from which there is no escape, it simply acts as a sink which returns constantly \textit{Recovered}:

\begin{HaskellCode}
recoveredAgent :: SIRAgent
recoveredAgent = arr (const Recovered)
\end{HaskellCode}

Next, we implement the behaviour of a susceptible agent. It makes contact \textit{on average} with $\beta$ other random agents. For every \textit{infected} agent it gets into contact with, it becomes infected with a probability of $\gamma$. If an infection happens, it makes the transition to the \textit{Infected} state. To make contact, it gets fed the states of all agents in the system from the previous time-step, so it can draw random contacts - this is one, very naive way of implementing the interactions between agents.

Thus a susceptible agent behaves as susceptible until it becomes infected. Upon infection an \textit{Event} is returned, which results in switching into the \textit{infectedAgent} SF, which causes the agent to behave as an infected agent from that moment on. When an infection event occurs we change the behaviour of an agent using the Yampa combinator \textit{switch}, which is quite elegant and expressive as it makes the change of behaviour at the occurrence of an event explicit. Note that to make contact \textit{on average}, we use Yampas \textit{occasionally} function which requires us to carefully select the right $\Delta t$ for sampling the system as will be shown in results. 

Note the use of \textit{iPre :: a $\rightarrow$ SF a a}, which delays the input signal by one sample, taking an initial value for the output at time zero. The reason for it is that we need to delay the transition from susceptible to infected by one step due to the semantics of the \textit{switch} combinator: whenever the switching event occurs, the signal function into which is switched will be run at the time of the event occurrence. This means that a susceptible agent could make a transition to recovered within one time-step, which we want to prevent, because the semantics should be that only one state-transition can happen per time-step.

\begin{HaskellCode}
susceptibleAgent :: RandomGen g => g -> SIRAgent
susceptibleAgent g 
    = switch 
      -- delay switching by 1 step to prevent against transition
      -- from Susceptible to Recovered within one time-step
      (susceptible g >>> iPre (Susceptible, NoEvent)) 
      (const (infectedAgent g))
  where
    susceptible :: RandomGen g 
      => g -> SF [SIRState] (SIRState, Event ())
    susceptible g = proc as -> do
      makeContact <- occasionally g (1 / contactRate) () -< ()
      if isEvent makeContact
        then (do
          -- draw random element from the list
          a <- drawRandomElemSF g -< as
          case a of
            Infected -> do
              -- returns True with given probability
              i <- randomBoolSF g infectivity -< ()
              if i
                then returnA -< (Infected, Event ())
                else returnA -< (Susceptible, NoEvent)
             _       -> returnA -< (Susceptible, NoEvent))
        else returnA -< (Susceptible, NoEvent)
\end{HaskellCode}

To deal with randomness in an FRP way, we implemented additional signal functions built on the \textit{noiseR} function provided by Yampa. This is an example for the stream character and statefulness of a signal function as it allows to keep track of the changed random-number generator internally through the use of continuations and closures. Here we provide the implementation of \textit{randomBoolSF}. \textit{drawRandomElemSF} works similar but takes a list as input and returns a randomly chosen element from it:

\begin{HaskellCode}
randomBoolSF :: RandomGen g => g -> Double -> SF () Bool
randomBoolSF g p = proc _ -> do
  r <- noiseR ((0, 1) :: (Double, Double)) g -< ()
  returnA -< (r <= p)
\end{HaskellCode}

An infected agent recovers \textit{on average} after $\delta$ time units. This is implemented by drawing the duration from an exponential distribution \cite{borshchev_system_2004} with $\lambda = \frac{1}{\delta}$ and making the transition to the \textit{Recovered} state after this duration. Thus the infected agent behaves as infected until it recovers, on average after the illness duration, after which it behaves as a recovered agent by switching into \textit{recoveredAgent}. As in the case of the susceptible agent, we use the \textit{occasionally} function to generate the event when the agent recovers. Note that the infected agent ignores the states of the other agents as its behaviour is completely independent of them.

\begin{HaskellCode}
infectedAgent :: RandomGen g => g -> SIRAgent
infectedAgent g 
    = switch 
      -- delay switching by 1 step 
      (infected >>> iPre (Infected, NoEvent))
      (const recoveredAgent)
  where
    infected :: SF [SIRState] (SIRState, Event ())
    infected = proc _ -> do
      recEvt <- occasionally g illnessDuration () -< ()
      let a = event Infected (const Recovered) recEvt
      returnA -< (a, recEvt)
\end{HaskellCode}

For running the simulation we use Yampas function \textit{embed}:

\begin{HaskellCode}
runSimulation :: RandomGen g => g -> Time -> DTime 
              -> [SIRState] -> [[SIRState]]
runSimulation g t dt as 
    = embed (stepSimulation sfs as) ((), dts)
  where
    steps     = floor (t / dt)
    dts       = replicate steps (dt, Nothing)
    n         = length as
    (rngs, _) = rngSplits g n [] -- unique rngs for each agent
    sfs       = zipWith sirAgent rngs as
\end{HaskellCode}

What we need to implement next is a closed feedback-loop - the heart of every agent-based simulation. Fortunately, \cite{nilsson_functional_2002, courtney_yampa_2003} discusses implementing this in Yampa. The function \textit{stepSimulation} is an implementation of such a closed feedback-loop. It takes the current signal functions and states of all agents, runs them all in parallel and returns this step's new agent states. Note the use of \textit{notYet}, which is required because in Yampa switching occurs immediately at $t = 0$. If we don't delay the switching at $t = 0$ until the next step, we would enter an infinite switching loop - \textit{notYet} simply delays the first switching until the next time-step.

\begin{HaskellCode}
stepSimulation :: [SIRAgent] -> [SIRState] -> SF () [SIRState]
stepSimulation sfs as =
    dpSwitch
      -- feeding the agent states to each SF
      (\_ sfs' -> (map (\sf -> (as, sf)) sfs'))
      -- the signal functions
      sfs
      -- switching event, ignored at t = 0         
      (switchingEvt >>> notYet)
      -- recursively switch back into stepSimulation         
      stepSimulation                            
  where
    switchingEvt :: SF ((), [SIRState]) (Event [SIRState])
    switchingEvt = arr (\ (_, newAs) -> Event newAs)
\end{HaskellCode}

Yampa provides the \textit{dpSwitch} combinator for running signal functions in parallel, which has the following type-signature:

\begin{HaskellCode}
dpSwitch :: Functor col
         -- routing function
         => (forall sf. a -> col sf -> col (b, sf))
         -- SF collection
         -> col (SF b c)
         -- SF generating switching event     
         -> SF (a, col c) (Event d)
         -- continuation to invoke upon event           
         -> (col (SF b c) -> d -> SF a (col c))
         -> SF a (col c)
\end{HaskellCode}

Its first argument is the pairing-function, which pairs up the input to the signal functions - it has to preserve the structure of the signal function collection. The second argument is the collection of signal functions to run. The third argument is a signal function generating the switching event. The last argument is a function, which generates the continuation after the switching event has occurred. \textit{dpSwitch} returns a new signal function, which runs all the signal functions in parallel and switches into the continuation when the switching event occurs. The d in \textit{dpSwitch} stands for decoupled which guarantees that it delays the switching until the next time-step: the function into which we switch is only applied in the next step, which prevents an infinite loop if we switch into a recursive continuation.

Conceptually, \textit{dpSwitch} allows us to recursively switch back into the \textit{stepSimulation} with the continuations and new states of all the agents after they were run in parallel. 

\subsubsection{Results}
The dynamics generated by this step can be seen in Figure \ref{fig:sir_abs_dynamics_frp}. 

\begin{figure}
\begin{center}
	\begin{tabular}{c c}
		\begin{subfigure}[b]{0.22\textwidth}
			\centering
			\includegraphics[width=1\textwidth, angle=0]{./fig/SIR_Yampa/SIR_Yampa_dt01.png}
			\caption{$\Delta t = 0.1$}
			\label{fig:sir_abs_approximating_01dt_1000agents}
		\end{subfigure}
		
		&
    	
		\begin{subfigure}[b]{0.22\textwidth}
			\centering
			\includegraphics[width=1\textwidth, angle=0]{./fig/SIR_Yampa/SIR_Yampa_dt001.png}
			\caption{$\Delta t = 0.01$}
			\label{fig:sir_abs_approximating_001dt_1000agents}
		\end{subfigure}
	\end{tabular}
	
	\caption{FRP simulation of agent-based SIR showing the influence of different $\Delta t$. Population size of 1,000 with contact rate $\beta = \frac{1}{5}$, infection probability $\gamma = 0.05$, illness duration $\delta = 15$ with initially 1 infected agent. Simulation run for 150 time-steps with respective $\Delta t$.} 
	\label{fig:sir_abs_dynamics_frp}
\end{center}
\end{figure}

By following the FRP approach we assume a continuous flow of time, which means that we need to select a \textit{correct} $\Delta t$, otherwise we would end up with wrong dynamics. The selection of a correct $\Delta t$ depends in our case on \textit{occasionally} in the \textit{susceptible} behaviour, which randomly generates an event on average with \textit{contact rate} following the exponential distribution. To arrive at the correct dynamics, this requires us to sample \textit{occasionally}, and thus the whole system, with small enough $\Delta t$ which matches the frequency of events generated by \textit{contact rate}. If we choose a too large $\Delta t$, we loose events, which will result in wrong dynamics as can be seen in Figure \ref{fig:sir_abs_approximating_01dt_1000agents}. This issue is known as under-sampling and is described in Figure \ref{fig:sampling_issue}.

\begin{figure}
\begin{center}
	\begin{tabular}{c}
		\begin{subfigure}[b]{0.3\textwidth}
			\centering
			\includegraphics[width=1\textwidth, angle=0]{./fig/diagrams/Undersampling.png}
			\caption{Under-sampling}
			\label{fig:undersampling}
		\end{subfigure}
		
		\\
		
		\begin{subfigure}[b]{0.3\textwidth}
			\centering
			\includegraphics[width=1\textwidth, angle=0]{./fig/diagrams/Supersampling.png}
			\caption{Super-sampling}
			\label{fig:supersampling}
		\end{subfigure}
	\end{tabular}
	
	\caption{A visual explanation of under-sampling and super-sampling. The black dots represent the time-steps of the simulation. The red dots represent virtual events which occur at specific points in continuous time. In the case of under-sampling, 3 events occur in between the two time steps but \textit{occasionally} only captures the first one. By increasing the sampling frequency either through a smaller $\Delta t$ or super-sampling all 3 events can be captured.} 
	\label{fig:sampling_issue}
\end{center}
\end{figure}

For tackling this issue we have two options. The first one is to use a smaller $\Delta t$ as can be seen in \ref{fig:sir_abs_approximating_001dt_1000agents}, which results in the whole system being sampled more often, thus reducing performance. The other option is to implement super-sampling and apply it to \textit{occasionally}, which would allow us to run the whole simulation with $\Delta t = 1.0$ and only sample the \textit{occasionally} function with a much higher frequency.

\subsubsection{Discussion}
We can conclude that our first step already introduced most of the fundamental concepts of ABS:
\begin{itemize}
	\item Time - the simulation occurs over virtual time which is modelled explicitly, divided into \textit{fixed} $\Delta t$, where at each step all agents are executed.
	\item Agents - we implement each agent as an individual, with the behaviour depending on its state.
	\item Feedback - the output state of the agent in the current time-step $t$ is the input state for the next time-step $t + \Delta t$.
	\item Environment - as environment we implicitly assume a fully-connected network (complete graph) where every agent 'knows' every other agent, including itself and thus can make contact with all of them.
	\item Stochasticity - it is an inherently stochastic simulation, which is indicated by the random-number generator and the usage of \textit{occasionally}, \textit{randomBoolSF} and \textit{drawRandomElemSF}.
	\item Deterministic - repeated runs with the same initial random-number generator result in same dynamics. This may not come as a surprise but in Haskell we can guarantee that property statically already at compile time because our simulation runs \textit{not} in the IO Monad. This guarantees that no external, uncontrollable sources of non-determinism can interfere with the simulation.
	\item Parallel, lock-step semantics - the simulation implements a \textit{parallel} update-strategy where in each step the agents are run isolated in parallel and don't see the actions of the others until the next step.
\end{itemize}

Using FRP in the instance of Yampa results in a clear, expressive and robust implementation. State is implicitly encoded, depending on which signal function is active. By using explicit time-semantics with \textit{occasionally} we can achieve extremely fine grained stochastics by sampling the system with small $\Delta t$: we are treating it as a truly continuous time-driven agent-based system.

A very severe problem, hard to find with testing but detectable with in-depth validation analysis, is the fact that in the \textit{susceptible} agent the same random-number generator is used in \textit{occasionally}, \textit{drawRandomElemSF} and \textit{randomBoolSF}. This means that all three stochastic functions, which should be independent from each other, are inherently correlated. This is something one wants to prevent under all circumstances in a simulation, as it can invalidate the dynamics on a very subtle level, and indeed we have tested the influence of the correlation in this example and it has an impact. We left this severe bug in for explanatory reasons, as it shows an example where functional programming actually encourages very subtle bugs if one is not careful. A possible but not very elegant solution would be to simply split the initial random-number generator in \textit{sirAgent} three times (using one of the splited generators for the next split) and pass three random-number generators to \textit{susceptible}. A much more elegant solution would be to use the Random Monad which is not possible because Yampa is not monadic.

So far we have an acceptable implementation of an agent-based SIR approach. What we are lacking at the moment is a general treatment of an environment and an elegant solution to the random number correlation. In the next step we make the transition to Monadic Stream Functions as introduced in Dunai \cite{perez_functional_2016}, which allows FRP within a monadic context and gives us a way for an elegant solution to the random number correlation.

\subsection{Generalising to Monadic Stream Functions}
\label{sec:generalising_msfs}
A part of the library Dunai is BearRiver, a wrapper which re-implements Yampa on top of Dunai, which should allow us to easily replace Yampa with MSFs. This will enable us to run arbitrary monadic computations in a signal function, solving our problem of correlated random numbers through the use of the Random Monad.

\subsubsection{Identity Monad}
We start by making the transition to BearRiver by simply replacing Yampas signal function by BearRivers' which is the same but takes an additional type parameter \textit{m}, indicating the monadic context. If we replace this type-parameter with the Identity Monad, we should be able to keep the code exactly the same, because BearRiver re-implements all necessary functions we are using from Yampa. We simply re-define our agent signal function, introducing the monad stack our SIR implementation runs in:

\begin{HaskellCode}
type SIRMonad    = Identity
type SIRAgent    = SF SIRMonad [SIRState] SIRState
\end{HaskellCode}

\subsubsection{Random Monad}
Using the Identity Monad does not gain us anything but it is a first step towards a more general solution. Our next step is to replace the Identity Monad by the Random Monad, which will allow us to run the whole simulation within the Random Monad with the full features of FRP, finally solving the problem of correlated random numbers in an elegant way. We start by re-defining the SIRMonad and SIRAgent:

\begin{HaskellCode}
type SIRMonad g = Rand g
type SIRAgent g = SF (SIRMonad g) [SIRState] SIRState
\end{HaskellCode}

The question is now how to access this Random Monad functionality within the MSF context. For the function \textit{occasionally}, there exists a monadic pendant \textit{occasionallyM} which requires a MonadRandom type-class. Because we are now running within a MonadRandom instance we simply replace \textit{occasionally} with \textit{occasionallyM}. 

\begin{HaskellCode}
occasionallyM :: MonadRandom m => Time -> b -> SF m a (Event b)
-- can be used through the use of arrM and lift
randomBoolM :: RandomGen g => Double -> Rand g Bool
-- this can be used directly as a SF with the arrow notation
drawRandomElemSF :: MonadRandom m => SF m [a] a
\end{HaskellCode}

\subsubsection{Discussion} 
Running in the Random Monad solved the problem of correlated random numbers and elegantly guarantees us that we won't have correlated stochastics as discussed in the previous section. In the next step we introduce the concept of an explicit discrete 2D environment.

\subsection{Adding an environment}
\label{sec:step5_environment}
In this step we will add an environment in which the agents exist and through which they interact with each other. This is a fundamental different approach to agent interaction but is as valid as the approach in the previous steps.

In ABS agents are often situated within a discrete 2D environment \cite{epstein_growing_1996} which is simply a finite $N x M$ grid with either a Moore or von Neumann neighbourhood (Figure \ref{fig:abs_neighbourhoods}). Agents are either static or can move freely around with cells allowing either single or multiple occupants.

We can directly map the SIR model to a discrete 2D environment by placing the agents on a corresponding 2D grid with an unrestricted neighbourhood. The behaviour of the agents is the same but they select their interactions directly from the environment. Also instead of feeding back the states of all agents as inputs, agents now communicate through the environment by revealing their current state to their neighbours by placing it on their cell. Agents can read the states of all their neighbours which tells them if a neighbour is infected or not. This allows us to implement the infection mechanism as in the beginning. For purposes of a more interesting approach, we restrict the neighbourhood to Moore (Figure \ref{fig:moore_neighbourhood}).

\begin{figure}
\begin{center}
	\begin{tabular}{c c}
		\begin{subfigure}[b]{0.2\textwidth}
			\centering
			\includegraphics[width=0.5\textwidth, angle=0]{./fig/diagrams/neumann.png}
			\caption{von Neumann}
			\label{fig:neumann_neighbourhood}
		\end{subfigure}
    	&
		\begin{subfigure}[b]{0.2\textwidth}
			\centering
			\includegraphics[width=0.5\textwidth, angle=0]{./fig/diagrams/moore.png}
			\caption{Moore}
			\label{fig:moore_neighbourhood}
		\end{subfigure}
    \end{tabular}
	\caption{Common neighbourhoods in discrete 2D environments of Agent-Based Simulation.}
	\label{fig:abs_neighbourhoods}
\end{center}
\end{figure}

\subsubsection{Implementation}
We start by defining our discrete 2D environment for which we use an indexed two dimensional array. In each cell the agents will store their current state, thus we use the \textit{SIRState} as type for our array data:

\begin{HaskellCode}
type Disc2dCoord = (Int, Int)
type SIREnv      = Array Disc2dCoord SIRState
\end{HaskellCode}

Next we redefine our monad stack and agent signal function. We use a StateT transformer on top of our Random Monad from the previous step with \textit{SIREnv} as type for the state. Our agent signal function now has unit input and output type, which indicates that the actions of the agents are only visible through side-effects in the monad stack they are running in.

\begin{HaskellCode}
type SIRMonad g = StateT SIREnv (Rand g)
type SIRAgent g = SF (SIRMonad g) () ()
\end{HaskellCode}

The implementation of a susceptible agent is now a bit different and a mix between previous steps. The agent directly queries the environment for its neighbours and randomly selects one of them. The remaining behaviour is similar:

\begin{HaskellCode}
susceptibleAgent :: RandomGen g => Disc2dCoord -> SIRAgent g
susceptibleAgent coord
    = switch susceptible (const (infectedAgent coord))
  where
    susceptible :: RandomGen g 
      => SF (SIRMonad g) () ((), Event ())
    susceptible = proc _ -> do
      makeContact <- occasionallyM (1 / contactRate) () -< ()
      if not (isEvent makeContact)
        then returnA -< ((), NoEvent)
        else (do
          env <- arrM_ (lift get) -< ()
          let ns = neighbours env coord agentGridSize moore
          s <- drawRandomElemS -< ns
          case s of
            Infected -> do
              infected <- arrM_ 
                (lift $ lift $ randomBoolM infectivity) -< ()
              if infected 
                then (do
                  arrM (put . changeCell coord Infected) -< env
                  returnA -< ((), Event ()))
                else returnA -< ((), NoEvent)
            _        -> returnA -< ((), NoEvent))

\end{HaskellCode}
Querying the neighbourhood is done using the \textit{neighbours :: SIREnv -> Disc2dCoord -> Disc2dCoord -> [Disc2dCoord] -> [SIRState]} function. It takes the environment, the coordinate for which to query the neighbours for, the dimensions of the 2D grid and the neighbourhood information and returns the data of all neighbours it could find. Note that on the edge of the environment, it could be the case that fewer neighbours than provided in the neighbourhood information will be found due to clipping.

The behaviour of an infected agent is nearly the same as in the previous step, with the difference that upon recovery the infected agent updates its state in the environment from Infected to Recovered.

Running the simulation with MSFs works slightly different. The function \textit{embed} we used before is not provided by BearRiver but by Dunai which has important implications. Dunai does not know about time in MSFs, which is exactly what BearRiver builds on top of MSFs. It does so by adding a ReaderT Double which carries the $\Delta t$. This is the reason why we need lifts e.g. in case of getting the environment. Thus \textit{embed} returns a computation in the ReaderT Double Monad which we need to peel away using \textit{runReaderT}. This then results in a StateT computation which we evaluate by using \textit{evalStateT} and an initial environment as initial state. This then results in another monadic computation of the Random Monad type which we evaluate using \textit{evalRand} which delivers the final result. Note that instead of returning agent states we simply return a list of environments, one for each step. The agent states can then be extracted from each environment.

\begin{HaskellCode}
runSimulation :: RandomGen g => g -> Time -> DTime 
  -> SIREnv -> [(Disc2dCoord, SIRState)] -> [SIREnv]
runSimulation g t dt env as = evalRand esRand g
  where
    steps    = floor (t / dt)
    dts      = replicate steps ()
    -- initial SFs of all agents
    sfs      = map (uncurry sirAgent) as   
    -- running the simulation   
    esReader = embed (stepSimulation sfs) dts 
    esState  = runReaderT esReader dt 
    esRand   = evalStateT esState env     
\end{HaskellCode}

Due to the different approach of returning the SIREnv in every step, we implemented our own MSF:
\begin{HaskellCode}
stepSimulation :: RandomGen g 
  => [SIRAgent g] -> SF (SIRMonad g) () SIREnv
stepSimulation sfs = MSF (\_ -> do
  -- running all SFs with unit input
  res <- mapM (`unMSF` ()) sfs
  -- extracting continuations, ignore output
  let sfs' = fmap snd res
  -- getting environment of current step   
  env <- get
  -- recursive continuation    
  let ct = stepSimulation sfs'  
  return (env, ct))
\end{HaskellCode}

\subsubsection{Results}
We implemented rendering of the environments using the gloss library which allows us to cycle arbitrarily through the steps and inspect the spreading of the disease over time visually as seen in Figure \ref{fig:sir_env}.

\begin{figure}
\begin{center}
	\begin{tabular}{c c}
		\begin{subfigure}[b]{0.2\textwidth}
			\centering
			\includegraphics[width=1\textwidth, angle=0]{./fig/step5_environment/SIR_environment_30x30agents_t100_01dt.png}
			\caption{$t = 100$}
			\label{fig:sir_env_t100}
		\end{subfigure}
    	
    	&
  
		\begin{subfigure}[b]{0.25\textwidth}
			\centering
			\includegraphics[width=1\textwidth, angle=0]{./fig/step5_environment/SIR_dynamics_30x30agents_300t_01dt.png}
			\caption{Dynamics over time}
			\label{fig:sir_dynamics_30x30agents_300t_01dt}
		\end{subfigure}
	\end{tabular}
	
	\caption{Simulating the agent-based SIR model on a 21x21 2D grid with Moore neighbourhood (Figure \ref{fig:moore_neighbourhood}), a single infected agent at the center and same SIR parameters as in Figure \ref{fig:sir_sd_dynamics}. Simulation run until $t = 200$ with fixed $\Delta t = 0.1$. Last infected agent recovers shortly after $t = 160$. The susceptible agents are rendered as blue hollow circles for better contrast.}
	\label{fig:sir_env}
\end{center}
\end{figure}

Note that the dynamics of the spatial SIR simulation which are seen in Figure \ref{fig:sir_dynamics_30x30agents_300t_01dt} look quite different from the SD dynamics of Figure \ref{fig:sir_sd_dynamics}. This is due to a much more restricted neighbourhood which results in far fewer infected agents at a time and a lower number of recovered agents at the end of the epidemic, meaning that fewer agents got infected overall.

\subsubsection{Discussion}
At first the environment approach might seem a bit overcomplicated and one might ask what we have gained by using an unrestricted neighbourhood where all agents can contact all others. The real win is that we can introduce arbitrary restrictions on the neighbourhood as shown with the Moore neighbourhood.

Of course an environment is not restricted to be a discrete 2D grid and can be anything from a continuous N-dimensional space to a complex network - one only needs to change the type of the StateT monad and provide corresponding neighbourhood querying functions. The ability to place the heterogeneous agents in a generic environment is also the fundamental advantage of an agent-based over the SD approach and allows to simulate much more realistic scenarios. 

\subsubsection{Discussion}
At first the environment approach might seem a bit overcomplicated and one might ask what we have gained by using an unrestricted neighbourhood where all agents can contact all others. The real win is that we can introduce arbitrary restrictions on the neighbourhood as shown using the Moore neighbourhood. Of course the environment is not restricted to a discrete 2D grid and can be anything from a continuous N-dimensional space to a complex network - one only needs to change the type of the StateT monad and provide corresponding neighbourhood querying functions. The ability to place the heterogeneous agents in a generic environment is also the fundamental advantage of an agent-based over the SD approach and allows to simulate much more realistic scenarios. Note that for reasons of clarity we have removed the data-flow approach from this implementation which results in the unit-types of input and output. In a full blown agent-based simulation library we would combine both approaches.

Generally, there exist four different types of environments in agent-based simulation:
\begin{enumerate}
	\item Passive read-only - implemented in the previous steps, where the environment never changes and is passed as static information, e.g. list of neighbours, to each agent.
	\item Passive read/write - implemented in this step. The environment itself is not modelled as an active process but just as shared data which can be accessed and manipulated by the agents.
	\item Active  read-only - can be implemented by adding an environment agent which broadcasts changes in the environment to all agents using the data-flow mechanism.
	\item Active read/write - can be implemented as in this step plus adding an environment agent which reads/writes the environment e.g. regrowing some resources.
\end{enumerate}

Attempting to introduce an active/passive read/write environment to the Yampa implementation would be quite cumbersome. A possible solution could be to add a type-parameter \textit{e} which captures the type of the environment and then pass it in through the input and allow it to be returned in the output of an agent signal function. We would then end up with $n$ copies of the environment - one for each agent - which we need to fold back into a single environment. Having an active environment complicates things even further. All these problems are not an issue when using MSFs with a StateT which is a compelling example for making the transition to the more general MSFs. The convenient thing is that although conceptually all agents act at the same time, technically by using \textit{mapM} in stepSimulation they are run after another, serialising the environment access which gives every agent exclusive read/write access while it is active.

\subsection{Additional Steps}
ABS involves a few more advanced concepts which we didn't fully explore in this paper due to lack of space. Instead we give a short overview and discuss them without presenting code or going into technical details.

\subsubsection{Agent-Transactions}
Agent-transactions are necessary when an arbitrary number of interactions between two agents need to happen instantaneously without time-lag. The use-case for this are price negotiations between multiple agents where each pair of agents needs to come to an agreement in the same time-step \cite{epstein_growing_1996}. In object-oriented programming, the concept of synchronous communication between agents is implemented directly with method calls.

We have implemented synchronous interactions, which we termed agent-transactions in an additional step. We solved it pure functionally by running the signal functions of the transacting agent pair as often as their protocol requires but with $\Delta t=0$, which indicates the instantaneous character of agent-transactions.

\subsubsection{Event Scheduling}
Our approach is inherently time-driven where the system is sampled with fixed $\Delta t$. The other fundamental way to implement an ABS in general, is to follow an event-driven approach \cite{meyer_event-driven_2014} which is based on the theory of Discrete Event Simulation \cite{zeigler_theory_2000}. In such an approach the system is not sampled in fixed $\Delta t$ but advanced as events occur where the system stays constant in between. Depending on the model, in an event-driven approach it may be more natural to express the requirements of the model.

In an additional step we have implemented a rudimentary event-driven approach which allows the scheduling of events but had to omit it due to lack of space. Using the flexibility of MSFs we added a State transformer to the monad stack which allows enqueuing of events into a priority queue. The simulation is advanced by processing the next event at the top of the queue which means running the MSF of the agent which receives the event. The simulation terminates if there are either no more events in the queue or after a given number of events or if the simulation time has advanced to some limit. Having made the transition to MSFs, implementing this feature was quite straight forward which shows the power and strength of the generalised approach to FRP using MSFs.

\subsubsection{Dynamic Agent creation}
In the SIR model, the agent population stays constant - agents don't die and no agents are created during simulation - but some simulations \cite{epstein_growing_1996} require dynamic agent creation and destruction. We can easily add and remove agents signal functions in the recursive switch after each time-step. The only problem is that creating new agents requires unique agent ids but with the transition to MSFs we can add a monadic context which allows agents to draw the next unique agent id when they create a new agent. 