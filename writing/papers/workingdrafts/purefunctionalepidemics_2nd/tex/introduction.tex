\section{Introduction}
The traditional approach to Agent-Based Simulation (ABS) has so far always been object-oriented techniques, due to the influence of the seminal work of Epstein et al \cite{epstein_growing_1996} in which the authors claim "[..] object-oriented programming to be a particularly natural development environment for Sugarscape specifically and artificial societies generally [..]" (p. 179). This work established the metaphor in the ABS community, that \textit{agents map naturally to objects} \cite{north_managing_2007} which still holds up today.

In this paper we fundamentally challenge this metaphor and explore ways of approaching ABS in a pure functional way using Haskell. By doing this we expect to leverage the benefits of pure functional programming \cite{hudak_history_2007}: higher expressivity through declarative code, being polymorph and explicit about side-effects through monads, more robust and less susceptible for bugs due to explicit data flow and lack of implicit side-effects.

As use case we introduce the simple SIR model of epidemiology with which one can simulate epidemics, that is the spreading of an infectious disease through a population, in a realistic way.

Over the course of four steps, we derive all necessary concepts required for a full agent-based implementation. We start from a very simple solution running in the Random Monad which has all general concepts already there and then refine it in various ways, making the transition to Functional Reactive Programming (FRP) \cite{wan_functional_2000} and to Monadic Stream Functions (MSF) \cite{perez_functional_2016}.

The aim of this paper is to show how ABS can be done in \textit{pure} Haskell and what the benefits and drawbacks are. By doing this we give the reader a good understanding of what ABS is, what the challenges are when implementing it and how we solve these in our approach.

The contributions of this paper are:
\begin{itemize}
	\item We present a novel approach to agent-based simulation (ABS) using \textit{declarative} analysis with Functional Reactive Programming (FRP) in which we systematically introduce the concepts of ABS to \textit{pure} functional programming in a step-by-step approach. Also this work presents a new field of application to FRP as to our best knowledge we are the first to discuss the application of Arrowized FRP to ABS on a technical level.
	
	\item In \cite{thaler_art_2017} the authors have shown the influence of different deterministic and non-deterministic elements in agent-based simulation on the final dynamics and how the influence of non-determinism can completely break them down or result in different dynamics despite same initial conditions.
	
	Our approach can guarantee reproducibility already at compile time, which means that repeated runs of the simulation with same initial conditions will always result in the same dynamics, something highly desirable in simulation in general. This can only be achieved through purity, which guarantees the absence of implicit side-effects which allows to rule out non-deterministic influences and IO at compile time through the strong static type system. This becomes only possible in pure functional programming where we can control the side-effects and can program side-effect polymorph, something not possible with traditional object-oriented approaches. Further through purity and the strong static type system we can rule out an important class of run-time bugs e.g. related to dynamic typing, and the lack of implicit data-dependencies which are common in traditional imperative object-oriented approaches.
	
	\item The result of using Arrowized FRP is a conceptually quite elegant approach to ABS which allows expressing continuous time-semantics in a very clear, compositional and declarative way, without having to deal with low-level details related to the progress of time.
\end{itemize}

Section \ref{sec:related_work} discusses related work. In section \ref{sec:background} we define agent-based simulation, introduce functional reactive programming, arrowized programming and monadic stream functions, because our approach builds heavily on these concepts. In section \ref{sec:sir_model} we introduce the SIR model of epidemiology as an example model to explain the concepts of ABS. The heart of the paper is section \ref{sec:functional_approach} in which we derive the concepts of a pure functional approach to ABS in four steps, using the SIR model. Finally, we draw conclusions and discuss issues in section \ref{sec:conclusions} and point to further research in section \ref{sec:further_research}.

