\documentclass[format=acmsmall, review=true, screen=true]{acmart}		% ICFP
%\documentclass[format=sigplan, review=true]{acmart}		% HASKELL SYMPOSIUM 
%\documentclass[format=sigconf, review=true]{acmart}		% IFL

\usepackage{float}
\usepackage{graphicx}
\usepackage{subcaption}
\usepackage{ifthen}
\usepackage{minted}
\usepackage{verbatim}

% Metadata Information
%% use defaults for review submission.
%\acmConference[HS18]{Haskell Symposium}{2018}{09}
%\acmYear{2018}
%\copyrightyear{2018}
\acmConference[IFL'18]{International Symposium on Implementation and Application of Functional Languages}{August 2019}{Lowell, MA, USA}
\acmYear{2019}
\copyrightyear{2019}
%\acmDOI{} % \acmDOI{10.1145/nnnnnnn.nnnnnnn}

% Copyright
%% use 'none' for review submission.
\setcopyright{none}
%\setcopyright{acmcopyright}	% = copyright transfer to ACM
%\setcopyright{acmlicensed} 		% = retaining copyright but granting ACM exclusive publication rights
%\setcopyright{rightsretained}  % = open access on payment of a fee
%\setcopyright{usgov}
%\setcopyright{usgovmixed}
%\setcopyright{cagov}
%\setcopyright{cagovmixed}

% TODO : get the data
% DOI
% \acmDOI{0000001.0000001}

% TODO: fill in
% Paper history
\received{May 2018}
%\received[revised]{March 2018}
%\received[accepted]{March 2018}

% Document starts
\begin{document}

\newminted[HaskellCode]{haskell}{fontsize=\footnotesize}

% Title portion. Note the short title for running heads
\title[Implementing Correct-By-Construction System Dynamics]{Implementing Correct-By-Construction System Dynamics}
\subtitle{A pure functional approach}

\author{Jonathan Thaler}
%\orcid{TODO}
\email{jonathan.thaler@nottingham.ac.uk}
%\author{Peer-Olaf Siebers}
%\email{peer-olaf.siebers@nottingham.ac.uk}
%\author{Thorsten Altenkirch}
%\email{thorsten.altenkirch@nottingham.ac.uk}
\affiliation{%
  \institution{University of Nottingham}
  \streetaddress{7301 Wollaton Rd}
  \city{Nottingham}
  \postcode{NG8 1BB}
  \country{United Kingdom}}

\begin{abstract}
%TODO: select journals
%- ACM Transactions on Modeling and Computer Simulation (TOMACS): https://tomacs.acm.org/
%- ICFP 2019 functional pearl?

TODO: it is yet unclear where i will submit this paper thus the audience is not clear. depending on which audience (simulation or functional programming) i have to rephrase the 'problem' and the 'claims' and 'contributions':
	- for the functional programming people can stay as it is but i might need to clarify on System Dynamics
	- for the simulation people i might need more clarification on Haskell and functional programming and rephrase the 'problem' and the 'claims' and 'contributions' to something they can understand easier. Problem is that i must not fall into the trap of selling it as FP vs OOP because there is FRP also in e.g. Java. 
TODO: write related work

In this short paper we investigate how to implement System Dynamics in the functional programming language Haskell. We use the concept of Functional Reactive Programming which allows to express continuous-time systems in a functional way. We show that System Dynamics map naturally to the abstractions provided by Functional Reactive Programming. Together with Haskell's strong static type system, we arrive at a correct-by-construction implementation which deterministic reproducibility we can guarantee at compile-time.
\end{abstract}

%
% The code below should be generated by the tool at
% http://dl.acm.org/ccs.cfm
% Please copy and paste the code instead of the example below.
%
% TODO needs to be generated
%\begin{CCSXML}
%<ccs2012>
% <concept>
%  <concept_id>10010520.10010553.10010562</concept_id>
%  <concept_desc>Computer systems organization~Embedded systems</concept_desc>
%  <concept_significance>500</concept_significance>
% </concept>
% <concept>
%  <concept_id>10010520.10010575.10010755</concept_id>
%  <concept_desc>Computer systems organization~Redundancy</concept_desc>
%  <concept_significance>300</concept_significance>
% </concept>
% <concept>
%  <concept_id>10010520.10010553.10010554</concept_id>
%  <concept_desc>Computer systems organization~Robotics</concept_desc>
%  <concept_significance>100</concept_significance>
% </concept>
% <concept>
%  <concept_id>10003033.10003083.10003095</concept_id>
%  <concept_desc>Networks~Network reliability</concept_desc>
%  <concept_significance>100</concept_significance>
% </concept>
%</ccs2012>
%\end{CCSXML}
%
%\ccsdesc[500]{Computer systems organization~Embedded systems}
%\ccsdesc[300]{Computer systems organization~Redundancy}
%\ccsdesc{Computer systems organization~Robotics}
%\ccsdesc[100]{Networks~Network reliability}

%
% End generated code
%

\keywords{System Dynamics, Functional Reactive Programming, Haskell}

\maketitle

%*******************************************************************************
%*********************************** First Chapter *****************************
%*******************************************************************************

\chapter{Introduction}  %Title of the First Chapter
I noticed that it is pretty hard to convince an agent-based economics specialist who is not a computer scientist about a pure functional approach. My conjecture is that the implementation technique and method does not matter much to them because they have very little knowledge about programming and are almost always self-taught - they don't know about software-engineering, nothing about proper software-design and architecture, nothing about software-maintenance, nothing about unit-testing,... In the end they just "hack" the simulation in whatever language they are able to: C++, Visual Basic, Java or toolboxes like Netlogo. For them it is all about to \textit{get things done somehow} and not to get things done the right way or in a beautiful way - the way and the method doesn't matter, its just a necessary evil which needs to be done. Thus if functional programming could make their lives easier, then they will definitely welcome it. But functional programming is, i think, harder to learn and harder to understand - so one needs to provide an abstraction through EDSL. So I REALLY need to come up with convincing arguments why to use pure functional approaches in ACE THEY can understand, otherwise I will be lost and not heard (not published,...). \\

What ACE economists care for:

\begin{itemize}
\item Very: Qualitative modelling with quantitative results
\item Yes: Easy reproducibility
\item Likely: Reasoning about convergence?
\item Likely: EDSL
\end{itemize}

My contributions are: pure functional framework, functional agent-model for market-simulations, EDSL for market-simulations, qualitative / implicit modelling with quanitative results, reasoning in my framework about convergence \\

IDEA: could I develop non-causal modelling (models are expressed in terms of non-directed equations, modelled in signal-relations) to allow for qualitative modelling for the agent-based economists? See hybrid modelling paper of Yampa. \textbf{THIS WOULD BE A HUGE NOVEL CONTRIBUTION TO ACE ESPECIALLY WHEN COMBINED WITH AN EDSL AND PROVIDING FULL REFERENTIAL TRANSPARENCY TO KEEP THE ABILITY TO REASON ABOUT CONVERGENCE}. This should be covered in the "EDSL"-paper.

TODO: maybe i should really focus only on market models? otherwise too much? \\

central novelty of my PhD: model specification = runnable code. possible through EDSL. but only in specific subfield of ACE: market-models. need a functional description of the model, then translate it to model specification in EDSL and then run it to see dynamics. But: model specification moves closer to functional programming languages. \\

another novelty approach: model specification through qualitative instead of quantiative approaches. is this possible? \\

WHY FUNCTIONAL? "because its the ultimate approach to scientific computing": fewer bugs due to mutable state (why? is thos shown obkectively by someone?), shorter (again as above, productivity), more expressive and closer to math, EDSL, EDSL=model=simulation, better parallelising due to referental transparency, reasoning \\

scientific results need to be reproduced, especially when they have high impact. a more formal approach of specifying the model and the simulation (model=simulation) could lead to easier sharing and easier reporduction without ambigouites \\

pure functional agent-model \& theory, EDSL framework in Haskell for ACE

\begin{enumerate}
\item Which kind of problem do we have?
\item What aim is there? Solving the problem? 
\item How the aim is achieved by enumerating VERY CLEAR objectives.
\item What the impact one expects (hypothesis) and what it is (after results).
\end{enumerate}

Note: It is not in the interest of the researcher to develop new economic theories but to research the use of functional methods (programming and specification) in agent-based computational economics (ACE).

NOTE: Get the reader’s attention early in the introduction: motivation, significance, originality and novelty.

\section{Methods}
Methods need to be selected to implement the simulations. Special emphasis will be put on functional ones which will then be compared to established methods in the field of ABM/S and ACE. \\

Claim: non-programming environments are considered to be not powerful enough to capture the complexity of ACE implementations thus a programming approach to ACE will be always required.

\section{Scenarios}
To apply and test functional methods in ACE, four scenarios of ACE are selected and then the methods applied and compared with each other to see how each of them perform in comparison. The 4 selected scenarios represent a selection of the challenges posed in ACE: from very abstract ones to very operational ones.

\section{Comparison}
Each of the selected scenarios is then implemented using the selected methods where each solution is then compared against the following criteria: 

\begin{enumerate}
\item suitability for scientific computation
\item robustness
\item error-sources
\item testability
\item stability
\item extendability
\item size of code
\item maintainability
\item time taken for development
\item verification \& correctness
\item replications \& parallelism
\item EDSL
\end{enumerate}

This will then allow to compare the different methods against each other and to show under which circumstances functional methods shine and when they should not be used.

\section{Agent-Based Modelling and Simulation (ABM/S)}
ABM/S is a method of modelling and simulating a system where the global behaviour may be unknown but the behaviour and interactions of the parts making up the system is of knowledge (Wooldrige, M. (2009). An Introduction to MultiAgent Systems. John Wiley & Sons). Those parts, called agents, are modelled and simulated out of which then the aggregate global behaviour of the whole system emerges. Thus the central aspect of ABM/S is the concept of an Agent which can be understood as a metaphor for a pro-active unit, able to spawn new Agents, and interacting with other Agents in a network of neighbours by exchange of messages. The implementation of Agents can vary and strongly depends on the programming language and the kind of domain the simulation and model is situated in.

\section{Agent-Based Economics (ACE)}
According to Leigh Tesfatsion (Tesfatsion, L. (2006). Agent-based computational economics: A constructive approach to economic theory. In Tesfatsion, L. and Judd, K. L., editors, Handbook of Computational Economics, volume 2, chapter 16, pages 831–880. Elsevier, 1 edition.), one of the leading figures, ACE is "[...] computational modelling of economic processes (including whole economies) as open-ended dynamic systems of interacting agents." - thus lending perfectly to the use of ABM/S as already the name suggests. Whereas classical economic models fall short by only looking at the average, pure rational, individual interacting in anonymous markets, the ACE approach looks at heterogeneous, non-rational individuals interacting with each other in networks (Kirman, A. (2010). Complex Economics: Individual and Collective Rationality. Routledge, London ; New York, NY.). Thus ACE can be understood as a combination of computer-science, cognitive/social science and evolutionary economics.

\section{Functional programming}
TODO: read \cite{Backus1978}

The state-of-the-art approach to implementing Agents are object-oriented methods and programming as the metaphor of an Agent as presented above lends itself very naturally to object-orientation (OO). The author of this thesis claims that OO in the hands of inexperienced or ignorant programmers is dangerous, leading to bugs and hardly maintainable and extensible code. The reason for this is that OO provides very powerful techniques of organising and structuring programs through Classes, Type Hierarchies and Objects, which, when misused, lead to the above mentioned problems. Also major problems, which experts face as well as beginners are 1. state is highly scattered across the program which disguises the flow of data in complex simulations and 2. objects don’t compose as well as functions. The reason for this is that objects always carry around some internal state which makes it obviously much more complicated as complex dependencies can be introduced according to the internal state.
All this is tackled by (pure) functional programming which abandons the concept of global state, Objects and Classes and makes data-flow explicit. This then allows to reason about correctness, termination and other properties of the program e.g. if a given function exhibits side-effects or not. Other benefits are fewer lines of code, easier maintainability and ultimately fewer bugs thus making functional programming the ideal choice for scientific computing and simulation and thus also for ACE. A very powerful feature of functional programming is Lazy evaluation. It allows to describe infinite data-structures and functions producing an infinite stream of output but which are only computed as currently needed. Thus the decision of how many is decoupled from how to (Hughes, J. (1989). Why functional programming matters. Comput. J., 32(2):98–107.).
The most powerful aspect using pure functional programming however is that it allows the design of embedded domain specific languages (EDSL). In this case one develops and programs primitives e.g. types and functions in a host language (embed) in a way that they can be combined. The combination of these primitives then looks like a language specific to a given domain, in the case of this thesis ACE. The ease of development of EDSLs in pure functional programming is also a proof of the superior extensibility and composability of pure functional languages over OO (Henderson P. (1982). Functional Geometry. Proceedings of the 1982 ACM Symposium on LISP and Functional Programming.).
One of the most compelling example to utilize pure functional programming is the reporting of Hudak (Hudak P., Jones M. (1994). Haskell vs. Ada vs. C++ vs. Awk vs. ... An Experiment in Software Prototyping Productivity. Department of Computer Science, Yale University.)  where in a prototyping contest of DARPA the Haskell prototype was by far the shortest with 85 lines of code. Also the Jury mistook the code as specification because the prototype did actually implement a small EDSL which is a perfect proof how close EDSL can get to and look like a specification.

Functional languages can best be characterized by their way computation works: instead of \textit{how} something is computed, \textit{what} is computed is described. Thus functional programming follows a declarative instead of an imperative style of programming. The key points are:
\begin{itemize}
\item No assignment statements - variables values can never change once given a value.
\item Function calls have no side-effect and will only compute the results - this makes order of execution irrelevant, as due to the lack of side-effects the logical point in \textit{time} when the function is calculated within the program-execution does not matter.
\item higher-order functions
\item lazy evaluation
\item Looping is achieved using recursion, mostly through the use of the general fold or the more specific map.
\item Pattern-matching
\end{itemize}

This alone does not really explain the \textit{real} advantages of functional programming and one must look for better motivations using functional programming languages. One motivation is given in \cite{Hughes1989} which is a great paper explaining to non-functional programmers what the significance of functional programming is and helping functional programmers putting functional languages to maximum use by showing the real power and advantages of functional languages. The main conclusion is that \textit{modularity}, which is the key to successful programming, can be achieved best using higher-order functions and lazy evaluation provided in functional languages like Haskell. \cite{Hughes1989} argues that the ability to divide problems into sub-problems depends on the ability to glue the sub-problems together which depends strongly on the programming-language and \cite{Hughes1989} argues that in this ability functional languages are superior to structured programming.

TODO: comparison of functional and object-oriented programming. My points are:
\begin{itemize}
\item The way state can be changed and treated - distributed over multiple objects - is often very difficult to understand.
\item Inheritance is a dangerous thing if not used with care because inheritance introduces very strong dependencies which cannot be changed during runtime anymore.
\item Objects don't compose very well: \url{http://zeroturnaround.com/rebellabs/why-the-debate-on-object-oriented-vs-functional-programming-is-all-about-composition/}
\item (Nearly) impossible to reason about programs
\end{itemize}

In conclusion the upsides of functional programming as opposed to OO are:
\begin{itemize}
\item Much more explicit flow of data \& control
\item Much better compose-able
\item Much better parallelism
\end{itemize}

\section{Related Research}
Tim Sweeney, CTO of Epic Games gave an invited talk about how "future programming languages could help us write better code" by "supplying stronger typing, reduce run-time failures;  and the need for pervasive concurrency support, both implicit and explicit, to effectively exploit the several forms of parallelism present in games and graphics." \cite{Sweeney2006}. Although the fields of games and agent-based simulations seem to be very different in the end, they have also very important similarities: both are simulations which perform numerical computations and update objects - in games they are called "game-objects" and in abm they are called agents but they are in fact the same thing - in a loop either concurrently or sequential. His key-points were:

\begin{itemize}
\item Dependent types as the remedy of most of the run-time failures.
\item Parallelism for numerical computation: these are pure functional algorithms, operate locally on mutable state. Haskell ST, STRef solution enables encapsulating local heaps and mutability within referentially transparent code.
\item Updating game-objects (agents) concurrently using STM: update all objects concurrently in arbitrary order, with each update wrapped in atomic block - depends on collisions if performance goes up.
\end{itemize}

\section{Related Work}
TODO: wormholes thesis and paper

\section{Background}

\subsection{Schelling Segregation}
We follow in our implementation the original paper of Schelling as in \cite{schelling_dynamic_1971} where we focus on the \textit{Area Distribution} section (Schelling starts with movement in a linear, 1-dimensional world where agents are able to move to the nearest point which meets the agents satisfaction but this is not what we follow here). One assumes a discrete 2-dimensional lattice-world with NxM fields. Each field is either occupied by an agent of a given color (e.g. Red or Green) or is free. Each field has 8 neighbours, which denotes a Moore-Neighbourhood. In Schellings original work the lattice-world is limited at its borders but we assume a torus world which is wrapped around in both the x- and y-dimensions resulting in 8 neighbours also for fields at the border. The occupation density was set by Schelling to be about 70\%-75\% which he identifies as being a setting which allows the agents to move around freely without making the lattice-world too sparse.
Now the agents make their move sequentially one after another. In each move an agent calculates the number of neighbours which are of equal color. If the number satisfies the agents needs about the neighbourhood then the agent is regarded as being 'happy' and will stay on this field. On the other hand the agent moves to the nearest unoccupied field which satisfies its needs. An agent which moves selects an unoccupied place randomly relative from its current place within a rectangle of side-length 2r where its current place is at the center. The interpretation for that behaviour is that agents won't move too far as it could be costly. Also children might attend a school in this area or the family has friends in this area, so they don't want to break that.



Agents just move depending on their movement-strategy to another place if they are not happy on the current one - they don't care how the target place is in the present or in the future, they will decide again in the next time-step. The interpretation for that behaviour is: agents want to 'just get out' at any cost, not caring what the future place will look like - it might be better or worse but they will see then.

\subsubsection{Optimizing behaviour}
TODO: define utility

The original schelling model didn't have a move-optimizing behaviour, meaning agents are just binary: if it is happy it will not move, if it is unhappy it will move but they won't care where they move. We introduce local move-optimizing behaviours which can be interpreted as being realistic in the real-world. It is important to note that we focus on \textit{local} instead of \textit{global} move-optimization: the agents are limited in their reasoning-capabilities and have limited information available: they cannot check out \textit{every} place and pick the globally best one.\\

\subsubsection{Anticipating behaviour}
Schelling explicitly mentions in \cite{schelling_dynamic_1971} that nobody anticipates moves of others. This is what we introduce using the recursive simulation.

TODO: is this optimizing behaviour in the spirit of schellings original work? 

\paragraph{Optimizing future} Agents pick an unoccupied random place and move to it if it increases their utility in the future. The interpretation for that behaviour is: agents heard about a place which will be cool in the future.

\paragraph{Optimizing present \& future} Agents pick an unoccupied random place and move to it if it increases their utility in the now and in the future. The interpretation for that behaviour is: agents heard about a cool spot in town, check it out and move to it if they like it but they also anticipate the coolness of the place in the future and if it seems that the place is going down then they won't move there.

\subsection{Related Research}
TODO: \cite{kirman_complex_2010} mention kirman complex economics where he investigates the model more in depth


\subsection{The SIR model}
\label{sec:sir_model}

The explanatory SIR model is a thoroughly studied and well understood compartment model from epidemiology \cite{kermack_contribution_1927}, which allows simulation of the dynamics of an infectious disease like influenza, tuberculosis, chicken pox, rubella and measles spreading through a population. The reason for choosing this model is its simplicity. It is easy to understand fully but complex enough to develop basic concepts of pure functional ABS, which are then extended and deepened in the much more complex Sugarscape model explained in the next section.

In this model, people in a population of size $N$ can be in either one of three states: \textit{Susceptible}, \textit{Infected} or \textit{Recovered}, at any particular time. It is assumed that initially there is at least one infected person in the population. People interact \textit{on average} with a given rate of $\beta$ other people per time unit, and become infected with a given probability $\gamma$ when interacting with an infected person. When infected, a person recovers \textit{on average} after $\delta$ time units and is then immune to further infections. An interaction between infected persons does not lead to reinfection, thus these interactions are ignored in this model. This definition gives rise to three compartments with the transitions seen in Figure \ref{fig:sir_transitions}.

\begin{figure}
	\centering
	\includegraphics[width=.7\textwidth, angle=0]{./fig/timedriven/SIR_transitions.png}
	\caption[States and transitions in the SIR compartment model]{States and transitions in the SIR compartment model.}
	\label{fig:sir_transitions}
\end{figure}

This model was also formalized using System Dynamics \cite{porter_industrial_1962}. In System Dynamics a system is modelled through differential equations, which allow expressing continuous systems, changing over time. They are solved by numerically integrating over time, which gives rise to the respective dynamics. The SIR model is modelled using the following equation, with the dynamics shown in Figure \ref{fig:sir_sd_dynamics} .

\begin{equation}
\begin{aligned}
\frac{\mathrm d S}{\mathrm d t} = -infectionRate \\
\frac{\mathrm d I}{\mathrm d t} = infectionRate - recoveryRate \\
\frac{\mathrm d R}{\mathrm d t} = recoveryRate 
\end{aligned}
\end{equation}

\begin{equation}
\begin{aligned}
infectionRate = \frac{I \beta S \gamma}{N} \\
recoveryRate = \frac{I}{\delta} 
\end{aligned}
\end{equation}

\begin{figure}
	\centering
	\includegraphics[width=0.5\textwidth, angle=0]{./fig/timedriven/SIR_SD_1000agents_150t_001dt.png}
	\caption[Dynamics of the SIR compartment model using the System Dynamics approach]{Dynamics of the SIR compartment model using the System Dynamics approach. Population Size $N$ = 1,000, contact rate $\beta =  \frac{1}{5}$, infection probability $\gamma = 0.05$, illness duration $\delta = 15$ with initially 1 infected agent. Simulation run for 150 time steps. Generated using our pure functional System Dynamics approach (see Appendix \ref{app:sd_simulation}).}
	\label{fig:sir_sd_dynamics}
\end{figure}

The approach of mapping the SIR model to an ABS is to discretise the population and model each person in the population as an individual agent. The transitions between the states are happening due to discrete events caused both by interactions amongst the agents and timeouts. The major advantage of ABS over System Dynamics is that it allows for the incorporation of spatiality and heterogeneity of a population, for example accounting for different sexes and ages. This is not directly possible with other simulation methods of System Dynamics or Discrete Event Simulation \cite{zeigler_theory_2000}.

This is directly related to a networked SIR model, where the interactions between agents are restricted by either a statically fixed or dynamically evolving network. Various network types exist, allowing for simulation of various scenarios. Very small communities where all agents are in contact with each other are modelled by a fully connected network. Real world scenarios where a few agents act as hubs are modelled by complex networks \cite{BarabasiAlbert_EmergenceScaling, Jackson2008, Newman_ComplexNetworks, WattsStrogatz_DynamicsSmallWorld}. In this thesis we do not impose restrictions on the connections among agents and always assume a fully connected network. Adding various network types to our thesis would unnecessarily complicate things in the beginning but would not constitute anything fundamentally new in terms of research. However, the use of complex networks, which in general are generated randomly, constitute an interesting direction for further research especially in the context of randomised property-based testing in ABS, which we discuss in Chapters \ref{ch:agentspec} and \ref{ch:sir_invariants}.

In the ABS classification of \cite{macal_everything_2016}, this model can be seen as an \textit{Interactive ABMS}: agents are individual heterogeneous agents with diverse set characteristics; they have autonomic, dynamic, endogenously defined behaviour; interactions happen between other agents and the environment through observed states, behaviours of other agents and the state of the environment.

\section[Implementation]{Implementation \footnote{Code available under\\ \url{https://github.com/thalerjonathan/phd/tree/master/coding/papers/iteratingABM/}}}
In this section we give a brief overview of comparing implementing the update-strategies in three languages which fundamentally differ among each other. We wanted to cover a wide range of different types of languages but didn't include a language where the memory-management falls in the hands of the developer. This would be the case e.g. in C++. This was looked into partially by \cite{dawson_opening_2014} but the focus of this paper is not on this issue as it would complicated things dramatically. All used languages are garbage-collected / the developer does not need to care how memory is cleared up.

For testing the suitability we selected a variety of simple models we implemented in each language with mostly all strategies. The selected models are \textit{Heroes \& Cowards}, \textit{SIRS}, \textit{Wildfire} and the \textit{Spatial Game} mentioned in \cite{huberman_evolutionary_1993}. We lack the space to explain all models but all are well known and can be easily found, looked up and understood on the Internet. They span different challenges to the ABS implementation: sending messages, accessing the environment, spawning new Agents, killing existing ones, discrete and continuous model. We also can confirm that all the reference-models proposed in \cite{isaac_abm_2011} and the StupidModel 1-16 by Railsback (TODO: cite) can be faithfully capture using our new terminology. Also we could show that the  Parallel-Strategy is the only strategy able to reproduce the pattern of the \textit{Spatial Game} due to the semantics of the model which require that all the Agents play the game at virtually the same time - which is only possible in the Parallel-Strategy.



\subsection{Java}
This language is included as the benchmark of object-oriented (OO) imperative languages as it is extremely popular in the ABS community and widely used in implementing their models and Environments. It comes with a comprehensive programming library, has nice object-oriented features built in and powerful synchronization primitives at built in at language-level. \\

\paragraph{Ease of Use}
Being experienced Java-Programmers we found that implementing all the strategies was straight-forward and easy thanks to the languages features. Especially parallelism and concurrency is quite very easy due to elegant and powerful built-in synchronization primitives so high-performance and large number of agents possible due to aliasing and side-effects.

\paragraph{Benefits}
high-performance and large number of agents possible due to aliasing and side-effects.

\paragraph{Deficits}
We couldn't identify something which absolutely didn't work, that's also why Java can be regarded as a very safe decision when opting which language to use for implementing ABS.
A downside is that one must take care when accessing memory in case of parallel or concurrent strategy. Due to the availability of aliasing and side-effects in the language and the type-system, it can't be guaranteed that access to memory happens only when its safe - something which is possible in Haskell (see below).
Care must be taken when references are sent by messages to other agents in case of parallel or concurrency

although the interfaces encourage it we cannot prevent the agents to use agent-references and directly accessing. a workaround would be to create new agent-instances in every iteration-step which would make old references useless but this doesn't protect us from concurrency issues with a current iteration (the copying must take place within a synchronized block, thus implicitly assuming ordering, something we don't want) and besides, can always work around and update the references.
that's the toll of side-effects: faster execution but less control over abuse
tried to clone agents in each step and let them collect their messages => extremely slow

\paragraph{Natural Strategy}
We think that the Sequential Strategy with Immediate Message-Handling is the most natural Strategy to express in Java due to its heavy reliance on side-effects through references (aliases) and shared thread of execution. Also most of the models work this way and its thus a save decision to start with Java.





 
\subsection{Haskell}
This language is included to put to test whether such a pure functional declarative programming language is suitable for full-blown ABS. What distinguishes it is its complete lack of implicit side-effects, global data, mutable variables and objects. The central concept is the function into which all data has to be passed in and out explicitly through statically typed arguments and return values: data-flow is completely explicit.

\paragraph{Ease of Use}
Being Haskell-Beginners we initially thought that it would be suitable at best for just implementing the Parallel-Strategy but after implementing all the strategies in it we found out that Haskell is extremely well suited to implement all of them. We think this stems from the following facts that it has no implicit side-effects which reduces bugs considerably and reveals the data-flow very explicitly.

dont have objects with methods which can call between each other but we need some way of representing agents. this is done using a struct type with a behaviour function and messaging mechanisms. important: agents are not carried arround but messages are sent to a receiver identified by an id. This is also a big difference to java where don't really need this kind of abstraction due to the use of objects and their 'messaging'. messaging mechanisms have up- and downsides, elaborate on it.

\paragraph{Benefits}
the type-system prevents us from introducing implicit side-effects, global references are only possible in code marked in its types as producing side-effects, something we explicitly avoided and were able to throughout the whole implementation. Thus arriving at a much safer version in the parallel iteration-strategies with separate threads of execution.
Also it is a must-have for STM, although it makes things more difficult in the beginning, in the end it turns out to be a blessing because one can guarantee that side-effects won't occur. We have taken care that the agents all run in side-effect free code.
concurrency and actors extremely elegant possible through: STM which only possible in languages without side-effect	
STM: implementing concurrency is a piece-of-cake
the conc-version and the act-version of the agent-implementations look EXACTLY the same	 BUT we lost the ability to step the simulation!!!

extremely powerful static typesystem: in combination with side-effect free this results in the semantics of an update-strategy to be reflected in the Agent-Transformer function and the messaging-interface. This means a user of this approach can be guided by the types and can't abuse them. Thus the lesson learned here is that \textit{if one tries to abuse the types of the agent-transformer or work around, then this is an indication that the update-strategy one has selected does not match the semantics of the model one wants to implement}. If this happens in Java, it is much more easier to work around by introducing global variables or side-effects but this is not possible in Haskell. \\
Thus our conclusion in using Haskell is that it is an extremely underestimated language in ABM/S which should be explored much more as we have shown that it really shines in this context and we believe that it could be pushed further even more.

\paragraph{Deficits}
still much work to do to capture large and complex models (see further research), performance is a big issue but this has not been about performance (2000 Agents are enough)
but still it is not suitable for big models with heterogeneous agents, there things are lacking: see further research

to implement immediate message-handling in haskell would be very cumbersome: 
drag all agents around, could run into recursion, state-handling becomes cumbersome, => leads to re-building an OO kind-of system in a pure functional language => don't do that! The conclusion is that for now this is left as the single drawback, which is not appropriately implementable using Haskell. Thats really where Java shines with its mutable Objects and Side-Effects.
TODO: could we really implement synchronous message-handling in Haskell? we would need to pass always ALL agents around and return them in every call
it is not possible to send a message directly to an agent
This is the single main drawback of the Haskell implementation and although it only shows up in the Seq Strategy it would be of interest if there is an elegant, pure functional software-architecture which allows messages sent from Agent A to Agent B be immediately - as in: the same execution thread - handled by Agent B. 

\paragraph{Natural Strategy}
note that the difference between SEQ and PAR in Haskell is in the end a 'fold' over the agents in the case of SEQ and a 'map' in the case of PAR




\subsection{Scala with Actors}
This multi-paradigm functional language is included to test the usefulness of the \textit{Act} strategy for implementing ABM/S. The language comes with an Actor-library inspired by \cite{agha_actors:_1986} and resembles the approach of Erlang which allows a very natural implementation of the strategy.

\paragraph{Ease of Use}

\paragraph{Benefits}

\paragraph{Deficits}

\paragraph{Natural Strategy}

\section{Discussion}

\subsection{Other Models}
TODO: mention that we have also implemented other models, which also work without time-semantics (all agents make a move at discrete time-steps and do not really rely on some notion of time). 

\subsection{Time-Semantics}
The main reason for building our pure functional ABMS approach on top of Yampa was to leverage the powerful time-semantics of Yampa which allows us to implement important concepts of ABMS:

state-chart: agents are at all time of their life-cycle in one state and can switch between multiple states using transitions 
timed transitions: transition to another state/behaviour happens at a discrete time
rate transitions: transition happens with a given rate
message transition: transition upon receiving a given message 

\subsection{Agents as Signals}
Due to the underlying nature and motivation of Functional Reactive Programming (und im speziellen) Yampa, Agents can be seen as Signals which is generated and consumed by a Signal-Function which is the behaviour of an Agent. If an Agent does not change the OUTPUT-signal is constant, if the agent changes e.g. by sending a message, changing its state,... the OUTPUT signal changes. A dead agent has no signal at all.

\subsection{Time-Sampling}
sampling rate depends on the transition times \& rates of the model. when e.g. the contact rate is 5 then the sampling dt should be below 0.2

\subsection{System Dynamics}
can emulate system dynamics due to the parallel update-strategy and continuous time-flow semantics

\subsection{Discrete Event Simulation}
DES in FrABMS? how easily can we implement server/queue systems? do they also look like a specification? potential problem: ordering of messages is not guaranteed by now

\subsection{Advantages}
advantages:
	- no side-effects within agents leads to much safer code
	- edsl for time-semantics
	- declarative style: agent-implementation looks like a model-specification
	- reasoning and verification
	- sequential and parallel
	- powerful time-semantics
	- arrowized programming is optional and only required when utilizing yampas time-semantics. if the model does not rely on time-semantics, it can use monadic-programming by building on the existing monadic functions in the EDSL which allow to run in the State-Monad which simplifies things very much
	- when to use yampas arrowized programing: time-semantics, simple state-chart agents 
	- when not using yampas facilities: in all the other cases e.g. SugarScape is such a case as it proceeds in unit time-steps and all agents act in every time-step
	- can implement System Dynamics building on Yampas facilities with total ease	
	- get replications for free without having to worry about side-effects and can even run them in parallel without headaches
	- cant mess around with time because delta-time is hidden from you (intentional design-decision by Yampa). this would be only very difficult and cumbersome to achieve in an object-oriented approach. TODO: experiment with it in Java - how could we actually implement this? I think it is impossible: may only achieve this through complicated application of patterns and inheritance but then has the problem of how to update the dt and more important how to deal with functions like integral which accumulates a value through closures and continuations. We could do this in OO by having a general base-class e.g. ContinuousTime which provides functions like updateDt and integrate, but we could only accumulate a single integral value.
	- reproducibility statically guaranteed
	- cannot mess around with dt
	- code == specification
	- rule out serious class of bugs
	- different time-sampling leads to different results e.g. in wildfire \& SIR but not in Prisoners Dilemma. why? probabilistic time-sampling?
	- reasoning about equivalence between SD and ABS implementation in the same framework
	- recursive implementations
	
	- we can statically guarantee the reproducibility of the simulation because: no side effects possible within the agents which would result in differences between same runs (e.g. file access, networking, threading), also timedeltas are fixed and do not depend on rendering performance or userinput	
	
\subsection{Disadvantages}
disadvantages:
	- performance is low
	- reasoning about performance is very difficult
	- very steep learning curve for non-functional programmers
	- learning a new EDSL
	- think ABMS different: when to use async messages, when to use sync conversations


[ ] important: increasing sampling freqzency and increasing number of steps so that the same number of simulation steps are executed should lead to same results. but it doesnt. why?
[ ] hypothesis: if time-semantics are involved then event ordering becomes relevant for emergent patterns. there are no tine semantics in heroes and cowards but in the prisoners dilemma
[ ] can we implement different types of agents interacting with each other in the same simulation ? with different behaviour funcs, digferent state? yes, also not possible in NetLogo to my knowledge. but they must have the same messages, emvironment 

[ ] Hypothesis: we can combine with FrABS agent-based simulation and system dynamics (this has been proved by example!)

\chapter{Conclusions}
\label{ch:conclusions}

This chapter concludes the whole thesis and outlines future research. Roughly 20\% exists already.

%we now know how to engineer time- and event-driven ABS with complex state both in the agent and environment, main difficulty is direct agent-interaction (see macal classification into 4 types of ABS), compile-time guaranteed reproducibility, explicit handling of complex state (read only, read/write), concurrency explicit and limited to STM, very promising concurrency but direct agent-interactions main problem (erlang as a rescue?), main drawbacks: everything is explicit, performance

\section{Further Research}
clearly outline the ideas for further research

\subsection{A general purpose library}
generalise concepts explored into a pure functional ABS library in Haskell (called chimera)

\subsection{Dependent and linear types}
dependent types and linear types are the next big step, towards a stronger formalisation of agents and ABS,
focus on the equilibrium - totality correspondence

\subsection{Concurrent event-driven ABS}
stm based concurrency for event-driven ABS using parallel DES. challenge is the time-warp implementation using monads. in general it should be easy to roll-back agents actions but with monads we have to be careful - for some monads rolling back is not neccessary e.g. rand and reader, for others it is, and for some it is impossible e.g. IO

\begin{acks}
The authors would like to thank J. Hey for constructive feedback, comments and valuable discussions.
\end{acks}

% Bibliography
\bibliographystyle{ACM-Reference-Format}
%% Citation style
%% Note: author/year citations are required for papers published as an
%% issue of PACMPL.
%%\citestyle{acmauthoryear}   %% For author/year citations
\bibliography{../../../references/phdReferences.bib}

\clearpage

\appendix

\section{Complete Code}
\label{app:code}
This appendix presents the complete code \footnote{The complete project, including visualisation and exporter to Matlab, is freely available on the Git Repository \url{https://github.com/thalerjonathan/phd/tree/master/public/sdhaskell/SIR}} which was introduced step-by-step in Section \ref{sec:impl}.

\begin{HaskellCode}
populationSize :: Double
populationSize = 1000

infectedCount :: Double
infectedCount = 1

contactRate :: Double
contactRate = 5

infectivity :: Double
infectivity = 0.05

illnessDuration :: Double
illnessDuration = 15

type SIRStep = (Time, Double, Double, Double)

sir :: SF () SIRStep
sir = loopPre (0, initSus, initInf, initRec) sirFeedback
  where
    initSus = populationSize - infectedCount
    initInf = infectedCount
    initRec = 0

    sirFeedback :: SF ((), SIRStep) (SIRStep, SIRStep)
    sirFeedback = proc (_, (_, s, i, _)) -> do
      let infectionRate = (i * contactRate * s * infectivity) / populationSize
          recoveryRate  = i / illnessDuration

      t <- time -< ()

      s' <- (initSus+) ^<< integral -< (-infectionRate)
      i' <- (initInf+) ^<< integral -< (infectionRate - recoveryRate)
      r' <- (initRec+) ^<< integral -< recoveryRate

      returnA -< dupe (t, s', i', r')

    dupe :: a -> (a, a)
    dupe a = (a, a)

runSD :: Time -> DTime -> [SIRStep]
runSD t dt = embed sir ((), steps)
  where
    steps = replicate (floor (t / dt)) (dt, Nothing)
\end{HaskellCode}

\end{document}
