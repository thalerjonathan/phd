\section{Further Research} % (Lived happily ever after...)
\label{sec:further}

So far we only implemented a tiny bit of the Sugarscape model and left out the later chapters which are more involved as they incorporate direct synchronous communication between agents. Such mechanisms are very difficult to approach in GPU based approaches \cite{lysenko_framework_2008} but should be quite straightforward in STM using \textit{TChan} and retries. We leave this for future research.

We have not focused on implementing an approach like \textit{Sense-Think-Act} cycle as mentioned in \cite{xiao_survey_2018}. This could offer lot of potential for parallelisation due to sense and think happening isolated from each agent without interfering with global shared data. We expect additional speed-up from such an approach but leave this for future research.

So far we only looked at a time-driven models (note that the Sugarscape is basically a time-driven model where each agent acts in each step) where all agents are run concurrently in lock-step. It would be of interest whether we can apply STM and concurrency to an event-driven approach as well. Generally one could run agents concurrently and undo actions when there are inconsistencies - something which pure functional programming in Haskell together with STM supports out of the box. Such an approach should be theoretically be implemented using Parallel Discrete Event Simulation (PDES) \cite{fujimoto_parallel_1990} and it would be interesting to see how an event-driven ABS approach based on an underlying PDES implementation would perform.

In our Sugarscape implementation we didn't scale up the environment because it is the limiting factor due to its sequential nature. Partitioning the environment into subsets which can be updated concurrently / parallel could speed up the environment updating as well which should be particularly easy in functional programming and using STM \textit{TArray}. We leave this for future research.

Our implementations focus only on local parallelisation and concurrency and avoided true distributed computing as it was out of the scope of this paper. It would be of interest to see how we can map functional agent based simulation to distributed computing using a message-based approach found in the Cloud Haskell framework.