%\documentclass[a4paper, 10pt, conference]{../../templates/IEEEconf/IEEEconf}
\documentclass[10pt, conference]{../../templates/IEEEtran/IEEEtran}
%\documentclass[10pt, journal]{../../templates/IEEEtran/IEEEtran}

\usepackage{graphicx}
\usepackage{caption} 
\usepackage{subcaption}
\usepackage{hyperref}
\usepackage{listings}
\usepackage{hhline}
\usepackage{float}
\usepackage{amssymb}
\usepackage[autostyle=true]{csquotes}
\usepackage{amsmath}
\usepackage{marvosym}

\font\subtitlefont=cmr12 at 18pt

\title{Towards pure functional Agent-Based Simulation}

% IEEEtran journal authors
%\author{Jonathan Thaler, ̃Peer-Olaf Siebers \\ School of Computer Science \\ University of Nottingham%
%\thanks{jonathan.thaler@nottingham.ac.uk}%
%\thanks{peer-olaf.siebers@nottingham.ac.uk}
%}

% IEEEtran conference authors
\author{
	\IEEEauthorblockN{Jonathan Thaler}
	\IEEEauthorblockA{School of Computer Science\\
		University of Nottingham\\
		jonathan.thaler@nottingham.ac.uk}
		
	\and
		
	\IEEEauthorblockN{Peer-Olaf Siebers}
	\IEEEauthorblockA{School of Computer Science\\
		University of Nottingham\\
		peer-olaf.siebers@nottingham.ac.uk}
}

%\IEEEpubid{0000--0000/00\$00.00 ̃\copyright ̃2015 IEEE}

% IEEEconf authors
%\author{
%	Jonathan Thaler \\
%	\email{jonathan.thaler@nottingham.ac.uk} \\
%	\begin{affiliation}
%		School of Computer Science, University of Nottingham
%	\end{affiliation} \\
%	\and 
%	Peer-Olaf Siebers \\
%	\email{peer-olaf.siebers@nottingham.ac.uk} \\
%	\begin{affiliation}
%		School of Computer Science, University of Nottingham
%	\end{affiliation} 
%	\and 
%	Thorsten Altenkirch \\
%	\email{thorsten.altenkirch@nottingham.ac.uk} \\
%	\begin{affiliation}
%		School of Computer Science, University of Nottingham
%	\end{affiliation} 
%}

\begin{document}
\maketitle 

\begin{abstract}
So far, the pure functional paradigm hasn't got much attention in Agent-Based Simulation (ABS) where the dominant programming paradigm is object-orientation, with Java, Python and C++ being its most prominent representatives. We claim that pure functional programming using Haskell is very well suited to implement complex, real-world agent-based models and brings with it a number of benefits. To show that we implemented the library \textit{FrABS} which allows to do ABS the first time in the pure functional programming language Haskell. To achieve this we leverage the basic concepts of ABS with functional reactive programming using Yampa. The result is a surprisingly fresh approach to ABS as it allows to incorporate discrete time-semantics similar to Discrete Event Simulation and continuous time-flows as in System Dynamics. In this paper we will show the novel approach of FrABS through the example of the SIR model, discuss implications, benefits and best practices.
\end{abstract}

\begin{IEEEkeywords}
Haskell, Functional Programming, Verification
\end{IEEEkeywords}

\section{Testing}

\cite{perez_testing_2017}

\subsection{Time-Traveling}

\cite{perez_back_2017}

CONTENT:
- do not introduce SIR model in that length, also don't discuss SD and ABS, only minimal definition of what we understand as ABS, ignore definition of SD completely
- good introduction to pure functional programming in Haskell: this is VERY difficult as it is a VAST topic where one can get lost quickly. focus on the central concepts: no assignment, recursion, pattern matching, static type-system with higher-kinded polymorphism
- focus on the benefits of the pure functional approach
	-> program looks very much like a specification
	-> can rule out bugs at compile time
	-> can guarantee reproducibility at compile time
	-> 2 update-strategies without the need of different
	-> testing using quickcheck, testing = writing program spec
	-> reasoning: TODO

\subsection{Reasoning}
i need to get a deep understanding in writing correct code and reasoning about correctness in Haskell - look into papers:
\url{https://wiki.haskell.org/Research_papers/Testing_and_correctness}
\url{https://www.reddit.com/r/haskell/comments/4tagq3/examples_of_realworld_haskell_usage_where/}
\url{https://stackoverflow.com/questions/4077970/can-haskell-functions-be-proved-model-checked-verified-with-correctness-properti}

\subsection{Costy bugs due to language features}
[ ] knight capital glitch
[ ] mars lander
[ ] moon landing
[ ] ?
[ ] ethereum \& blockchain technology

- The 3 major benefits of the approach I claim
	1. code == spec
	2. can rule out serious class of bugs
	3. we can perform reasoning about the simulation in code
	need to be metricated: e.g. this is really only possible in Haskell and not in Java. This needs thorough thinking about which metrics are used, how they can be aquired, how they can be compared,...
	
- I NEED TO SHOW HOW I CAN MAKE HASKELL RELEVANT IN THE FIELD OF ABS
	-> as far as I know so far no reasoning has been done in the way I intend to do it in the field of ABS. My hypothesis is that it is really only possible in Haskell due to its explicit side-effects, type-system, declarative style,... 
		-> TODO: need to check if this is really unique to haskell
	-> the functional-reactive approach seems to bring a new view to ABS with an embedded language for explicit time-semantics. Together with parallel/sequential updating this allows implementing System-Dynamics and agents which rely on continuous time-semantics e.g. SIR-Agents. Maybe i invented a hybrid between SD and ABS? Also what about time-traveling? The problem is that this is not really clear as i hypothesize that is completely novel approach to ABS - again I need to check this!
		-> TODO: is this really unique to functional reactive? E.g. what about Repast, NetLogo, AnyLogic, other Java-Frameworks? 
	-> maybe i have to admit that its not as unique as thought\\
	
In General i need to show that
- Haskells general benefits \& drawbacks over other Languages in the Field of ABS (e.g. Java, NetLogo, Repast) e.g. declarative style, reasoning, explicit about side-effects, performance, difficult to reason about performance, space-leaks difficult. So this focuses on the general comparison between the established technologies of ABS and Haskell but not yet on Haskells suitability in comparison to these other technologies. Here we talk about reasoning, side-effects, performance IN GENERAL TERMS, NOT SPECIFIC TO ABS. We need to distinguish between 
	-> general technicalities e.g. lambda-calculus (denotational formalism) or turing-machine (operational formalism) foundations, declarative style, lazy-evaluation allows to split the producer from the consumer, explicit about side-effects, not possible for in-order updates,...
	-> and in what they result e.g. fewer lines of code, ruling out of bugs, reasoning, lower performance, difficult to reason about space-time 
	
- Haskells suitability to implement ABS in comparison to other languages and technologies in the Field. Here the focus is on general problems in ABS and how they can and are solved using Haskell e.g. send message, changing environment, handling of time, replications, parallelism/concurrency,...

- Why using Haskell in ABS - do the general benefits / drawbacks apply equally well? Are there unique advantages? Can we do things in Haskell which are not possible in other technologies or just very hard? E.g. the hybrid-approach I created with FRP: how unique is it e.g. can other technologies easily implement it as well? Other potential advantages: recursive simulation. Here we DO NOT concentrate on general technicalities but see how they apply when using it for ABS and if they create a unique benefit for Haskell in ABS.

i need to show that different programming languages and paradigms have different power and are differently well suited to specific problems: the ultimate claim i need to show is that haskell is more powerful than java or C++ - the question is if this also makes it superior in applying it to problems: being more powerful, can all problems of java be solved better in haskell as well? this is i believe not the case e.g. gui- or game- programming. the question then is: what is the power of a programming language? can we measure it?

so what i need to show is how well haskell and its power are suited for implementing ABS. does the fact that haskell is much more powerful than existing technologies in ABS lead to the point that it is better suited for ABS? in fact it is power vs. better suited

\subsection{The power of a language}
[ ] more expressive: we can express complex problems more directly and with less  overhead. note that this is domain-specifix: the mechanisms of a language allow to create abstractions which solve the domain-specific problem. the better these mechanisms support one in this task, the more powerful the language is in the given domain. now we end up by defining what "better" support means
[ ] one could in pronciple do system programming in haskell by provoding bindings to code written in c and / or assembly but when the program is dominated by calls to these bindings then one could as well work directly in these lower languages and saves one from the overhead of the bindings
[ ] but very often a domain consists of multiple subdomains.
[ ] my hypothesis is that haskell is not well suited for domains which are dominated by managing and manipulating a global mutable state through side-effects / effectful computations. examples are gui-programming and computer games (state spread accross GPU and cpu, user input,...). this does not mean that it is not possible to implememt these things in haskell (it has been done with some sucess) but that the solution becomes too complex at some point.
[ ] conciesness
[ ] low ceremony
[ ] susceptibility to bugs
[ ] verbosity
[ ] reasoning about performance
[ ] reasoning about space requirements

\subsection{Measuring a language}
Define scientific measures: e.g. Lines Of Code (show relation to Bugs \& Defects, which is an objective measure: http://www.stevemcconnell.com/est.htm, \url{https://softwareengineering.stackexchange.com/questions/185660/is-the-average-number-of-bugs-per-loc-the-same-for-different-programming-languag}, Book: Code Complete, \url{https://www.mayerdan.com/ruby/2012/11/11/bugs-per-line-of-code-ratio}), also experience reports by companies which show that Haskell has huge benefits when applied to the same domain of a previous implementation of a different language, post on stack overflow / research gate / reddit, read experience reports from \url{http://cufp.org/2015/} Also need to show the problem of operational reasoning as opposed to denotational reasoning

\subsection{The Abstraction Hierarchy}

1st: Functional vs. Object Oriented
2nd: Haskell vs. Java
3rd: FrABS vs. Repast

\bibliographystyle{../../templates/IEEEtran/bibtex/IEEEtran}
\bibliography{../../../references/phdReferences.bib}

\appendices

\newpage
\section{Examples}
In this appendix we give a list of all the examples we have implemented and discuss implementation details relevant \footnote{The examples are freely available under \url{https://github.com/thalerjonathan/phd/tree/master/coding/libraries/frABS/examples}}. The examples were implemented as use-cases to drive the development of \textit{FrABS} and to give code samples of known models which show how to use this new approach. Note that we do not give an explanation of each model as this would be out of scope of this paper but instead give the major references from which an understanding of the model can be obtained.

We distinguish between the following attributes
\begin{itemize}
	\item Implementation - Which style was used? Either Pure, Monadic or Reactive. Examples could have been implemented in all of them.
	\item Yampa Time-Semantics - Does the implemented model make use of Yampas time-semantics e.g. occasional, after,...? Yes / No.
	\item Update-Strategy - Which update-strategy is required for the given example? It is either Sequential or Parallel or both. In the case of Sequential Agents may be shuffled or not.
	\item Environment - Which kind of environment is used in the given example? Possibilities are 2D/3D Discrete/Continuous or Network. In case of a Parallel Update-Strategy, collapsing may become necessary, depending on the semantics of the model. Also it is noted if the environment has behaviour. Note that an implementation may also have no environment which is noted as None. Although every model implemented in \textit{FrABs} needs to set up some environment, it is not required to use it in the implementation.
	\item Recursive - Is this implementation making use of the recursive features of \textit{FrABS} Yes/No (only available in sequential updating)?
	\item Conversations - Is this implementation making use of the conversations features of \textit{FrABS} Yes/No (only available in sequential updating)?
\end{itemize}

\subsection{Sugarscape}
This is a full implementation of the famous Sugarscape model as described by Epstein \& Axtell in their book \cite{epstein_growing_1996}. The model description itself has no real time-semantics, the agents act in every time-step. Only the environment may change its behaviour after a given number of steps but this is easily expressed without time-semantics as described in the model by Epstein \& Axtell \footnote{Note that this implementation has about 2600 lines of code which - although it includes both a pure and monadic implementation - is significant lower than e.g. the Java-implementation \url{http://sugarscape.sourceforge.net/} with about 6000. Of course it is difficult to compare such measures as we do not include FrABS itself into our measure.}.

\begin{center}
\begin{tabular}{l || l }
\textbf{Implementation}			& Pure, Monadic \\
\textbf{Yampa Time-Semantics}	& No \\
\textbf{Update-Strategy}		& Sequential, shuffling \\
\textbf{Environment}			& 2D Discrete, behaviour \\
\textbf{Recursive}				& No \\
\textbf{Conversations}			& Yes \\
\end{tabular}
\end{center}

\subsection{Agent\_Zero}
This is an implementation of the \textit{Parable 1} from the book of Epstein \cite{epstein_agent_zero:_2014}.

\begin{center}
\begin{tabular}{l || l }
\textbf{Implementation}			& Pure, Monadic \\
\textbf{Yampa Time-Semantics}	& No \\
\textbf{Update-Strategy}		& Parallel, Sequential, shuffling \\
\textbf{Environment}			& 2D Discrete, behaviour, collapsing \\
\textbf{Recursive}				& No \\
\textbf{Conversations}			& No \\
\end{tabular}
\end{center}

\subsection{Schelling Segregation}
This is an implementation of \cite{schelling_dynamic_1971} with extended agent-behaviour which allows to study dynamics of different optimization behaviour: local or global, nearest/random, increasing/binary/future. This is also the only 'real' model in which the recursive features were applied \footnote{The example of Recursive ABS is just a plain how-to example without any real deeper implications.}.

\begin{center}
\begin{tabular}{l || l }
\textbf{Implementation}			& Pure \\
\textbf{Yampa Time-Semantics}	& No \\
\textbf{Update-Strategy}		& Sequential, shuffling \\
\textbf{Environment}			& 2D Discrete \\
\textbf{Recursive}				& Yes (optional) \\
\textbf{Conversations}			& No \\
\end{tabular}
\end{center}

\subsection{Prisoners Dilemma}
This is an implementation of the Prisoners Dilemma on a 2D Grid as discussed in the papers of \cite{nowak_evolutionary_1992}, \cite{huberman_evolutionary_1993} and TODO: cite my own paper on update-strategies.

TODO: implement

\subsection{Heroes \& Cowards}
This is an implementation of the Heroes \& Cowards Game as introduced in \cite{wilensky_introduction_2015} and discussed more in depth in TODO: cite my own paper on update-strategies.

TODO: implement

\subsection{SIRS}
This is an early, non-reactive implementation of a spatial version of the SIRS compartment model found in epidemiology. Note that although the SIRS model itself includes time-semantics, in this implementation no use of Yampas facilities were made. Timed transitions and making contact was implemented directly into the model which results in contacts being made on every iteration, independent of the sampling time. Also in this sample only the infected agents make contact with others, which is not quite correct when wanting to approximate the System Dynamics model (see below). It is primarily included as a comparison to the later implementations (Fr*SIRS) of the same model  which make full use of \textit{FrABS} and to see the huge differences the usage of Yampas time-semantics can make.

\begin{center}
\begin{tabular}{l || l }
\textbf{Implementation}			& Pure, Monadic \\
\textbf{Yampa Time-Semantics}	& No \\
\textbf{Update-Strategy}		& Parallel, Sequential with shuffling \\
\textbf{Environment}			& 2D Discrete \\
\textbf{Recursive}				& No \\
\textbf{Conversations}			& No \\
\end{tabular}
\end{center}

\subsection{Fr(Spatial$|$Network)SIRS}
This is the reactive implementations of both 2D spatial and network (complete graph, Erdos-Renyi and Barbasi-Albert) versions of the SIRS compartment model. Unlike SIRS these examples make full use of the time-semantics provided by Yampa and show the real strength provided by \textit{FrABS}.

\begin{center}
\begin{tabular}{l || l }
\textbf{Implementation}			& Reactive \\
\textbf{Yampa Time-Semantics}	& Yes \\
\textbf{Update-Strategy}		& Parallel \\
\textbf{Environment}			& 2D Discrete, Network \\
\textbf{Recursive}				& No \\
\textbf{Conversations}			& No \\
\end{tabular}
\end{center}

\subsection{System Dynamics SIR}
This is an emulation of the System Dynamics model of the SIR compartment model in epidemiology. It was implemented as a proof-of-concept to show that \textit{FrABS} is able to implement even System Dynamic models because of its continuous-time and time-semantic features. Connections between stocks \& flows are hardcoded, after all System Dynamics completely lacks the concept of spatial- or network-effects. Note that describing the implementation as Reactive may seem not appropriate as in System Dynamics we are not dealing with any events or reactions to it - it is all about a continuous flow between stocks. In this case we wanted to express with Reactive that it is implemented using the Arrowized notion of Yampa which is required when one wants to use Yampas time-semantics anyway.

\begin{center}
\begin{tabular}{l || l }
\textbf{Implementation}			& Reactive \\
\textbf{Yampa Time-Semantics}	& Yes \\
\textbf{Update-Strategy}		& Parallel \\
\textbf{Environment}			& None \\
\textbf{Recursive}				& No \\
\textbf{Conversations}			& No \\
\end{tabular}
\end{center}

\subsection{WildFire}
This is an implementation of a very simple Wildfire model inspired by an example from AnyLogic\texttrademark with the same name.

\begin{center}
\begin{tabular}{l || l }
\textbf{Implementation}			& Reactive \\
\textbf{Yampa Time-Semantics}	& Yes \\
\textbf{Update-Strategy}		& Parallel \\
\textbf{Environment}			& 2D Discrete \\
\textbf{Recursive}				& No \\
\textbf{Conversations}			& No \\
\end{tabular}
\end{center}

\subsection{Double Auction}
This is a basic implementation of a double-auction process of a model described by \cite{breuer_endogenous_2015}. This model is not relying on any environment at the moment but could make use of networks in the future for matching offers.

\begin{center}
\begin{tabular}{l || l }
\textbf{Implementation}			& Pure, Monadic \\
\textbf{Yampa Time-Semantics}	& No \\
\textbf{Update-Strategy}		& Parallel \\
\textbf{Environment}			& None \\
\textbf{Recursive}				& No \\
\textbf{Conversations}			& No \\
\end{tabular}
\end{center}

\subsection{Proof of concepts}
\subsubsection{Recursive ABS} This example shows the very basics of how to implement a recursive ABS using \textit{FrABS}. Note that recursive features only work within the sequential strategy.

\begin{center}
\begin{tabular}{l || l }
\textbf{Implementation}			& Pure \\
\textbf{Yampa Time-Semantics}	& No \\
\textbf{Update-Strategy}		& Sequential \\
\textbf{Environment}			& None \\
\textbf{Recursive}				& Yes \\
\textbf{Conversations}			& No \\
\end{tabular}
\end{center}

\subsubsection{Conversation} This example shows the very basics of how to implement conversations in \textit{FrABS}. Note that conversations only work within the sequential strategy.

\begin{center}
\begin{tabular}{l || l }
\textbf{Implementation}			& Pure \\
\textbf{Yampa Time-Semantics}	& No \\
\textbf{Update-Strategy}		& Sequential \\
\textbf{Environment}			& None \\
\textbf{Recursive}				& No \\
\textbf{Conversations}			& Yes \\
\end{tabular}
\end{center}

\newpage
\section{Recursive Agent-Based Simulation}
The idea for this paper arose from my idea of \textit{anticipating agents}, which can project their actions in the future. Because this paper is not as polished as the draft for programming paradigms, we opted not to include it as an appendix and only give its basic ideas and results for the experiments conducted so far. Note that we were not able to find any research regarding recursive ABS \footnote{We found a paper on recursive simulation in general \cite{gilmer_recursive_2000} which focuses on military simulation implemented in C++. Its main findings are that deterministic models seem to benefit significantly from using recursions of the simulation for the decision making process and that when using stochastic models this benefit seems to be lost.}.
In Recursive ABS agents are able to halt time and 'play through' an arbitrary number of actions, compare their outcome and then to resume time and continue with a specifically chosen action e.g. the best performing or the one in which they haven't died. More precisely, what we want is to give an agent the ability to run the simulation recursively a number of times where the this number is not determined initially but can depend on the outcome of the recursive simulation. So Recursive ABS gives each Agent the ability to run the simulation locally from its point of view to anticipate its actions in the future and change them in the present.
We investigate the famous Schelling Segregation \cite{schelling_dynamic_1971} and endow our agents with the ability to project their actions into the future by recursively running simulations. Based on the outcome of the recursions they are then able to determine whether their move increases their utility in the future or not. The main finding for now is that it does not increase the convergence speed to equilibrium but can lead to extreme volatility of dynamics although the system seems to be near to complete equilibrium. In the case of a 10x10 field it was observed that although the system was nearly in its steady state - all but one agent were satisfied - the move of a single agent caused the system to become completely unstable and depart from its near-equilibrium state to a highly volatile and unstable state.

This approach of course rises a few questions and issues. The main problem of our approach is that, depending on ones view-point, it is violating the principles of locality of information and limit of computing power. To recursively run the simulation the agent which initiates the recursion is feeding in all the states of the other agents and calculates the outcome of potentially multiple of its own steps, each potentially multiple recursion-layers deep and each recursion-layer multiple time-steps long. Both requires that each agent has perfect information about the complete simulation \textit{and} can compute these 3-dimensional recursions, which scale exponentially. In the social sciences where agents are often designed to have only very local information and perform low-cost computations it is very difficult or impossible to motivate the usage of recursive simulations - it simply does not match the assumptions of the real world, the social sciences want to model. In general simulations, where it is much more commonly accepted to assume perfect information and potentially infinite amount of computing power this approach is easily motivated by a constructive argument: it is possible to build, thus we build it.
Another fundamental question regards the meaning and epistemology behind an entity running simulations. Of course, this strongly depends on the context: in ACE it may be understood as a search for optimizing behaviour, in Social Simulation it may be interpreted as a kind of free will: the agent who is initiating the recursion can be seen as 'knowing' that it is running inside a simulation, thus in this context free will is seen as being able to anticipate ones actions and change them.
When talking about recursion it is always the question of the depth of the recursion and because as we are running on computers we need to terminate at some point. Accelerating Turing machines (also known as Zeno Machine) are theoretically able to calculate an infinite regress but this raises again epistemological questions and can be seen as having religious character as discussed e.g. in Tiplers Omega Point, Bostroms simulation argument \cite{bostrom_are_2003} and its theological implications \cite{steinhart_theological_2010}. So the ultimate question this research leaves is what the outcome would be when running a recursive ABS on a Zeno Machine/Accelerated Turing Machine? \footnote{Anyway this would mean we have infinite amount of computing power - I am sure that in this case we don't worry the slightest about recursive ABS any more.}

At the moment this idea lies dormant as the intention was just to develop it far enough to give a proof-of-concept and see some results. Having achieved this we arrived at the conclusion, that the results are not really ground-breaking. This stems from the fact that Schelling segregation is not the best model to demonstrate this technique and that we are thus lacking the right model in which recursive ABS is the real killer-feature. Also to pursue this direction further and treat it in-depth, would require much more time and give the PhD a complete different spin. Still it is useful in supporting our move towards pure functional ABS as we are convinced that recursion is comparably easy to implement because the language is built on it and due to the lack of side-effects \footnote{Actually implementing it was \textit{really hard} but we wouldn't dare to implement this into an object-oriented language or into an object-oriented ABS framework.}.

\end{document}