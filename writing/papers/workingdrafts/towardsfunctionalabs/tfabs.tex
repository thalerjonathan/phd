\documentclass[format=acmsmall, review=false, screen=true]{acmart}		% ICFP
%\documentclass[format=sigplan, review=true]{acmart}		% HASKELL SYMPOSIUM 
%\documentclass[format=sigconf, review=true]{acmart}		% IFL

%\usepackage{float}
%\usepackage{graphicx}
%\usepackage{subcaption}
%\usepackage{ifthen}
\usepackage{minted}
%\usepackage{verbatim}

% Metadata Information
%% use defaults for review submission.
%\acmConference[HS18]{Haskell Symposium}{2018}{09}
%\acmYear{2018}
%\copyrightyear{2018}
\acmConference[IFL'18]{International Symposium on Implementation and Application of Functional Languages}{August 2019}{Lowell, MA, USA}
\acmYear{2019}
\copyrightyear{2019}
%\acmDOI{} % \acmDOI{10.1145/nnnnnnn.nnnnnnn}

% Copyright
%% use 'none' for review submission.
\setcopyright{none}
%\setcopyright{acmcopyright}	% = copyright transfer to ACM
%\setcopyright{acmlicensed} 		% = retaining copyright but granting ACM exclusive publication rights
%\setcopyright{rightsretained}  % = open access on payment of a fee
%\setcopyright{usgov}
%\setcopyright{usgovmixed}
%\setcopyright{cagov}
%\setcopyright{cagovmixed}

% TODO : get the data
% DOI
% \acmDOI{0000001.0000001}

% TODO: fill in
% Paper history
\received{May 2018}
%\received[revised]{March 2018}
%\received[accepted]{March 2018}

% Document starts
\begin{document}

\newminted[HaskellCode]{haskell}{fontsize=\small}

% Title portion. Note the short title for running heads
\title[The Agent's new Cloths]{The Agent's new Cloths}
\subtitle{Towards functional programming in Agent-Based Simulation}

\author{Jonathan Thaler}
%\orcid{TODO}
\email{jonathan.thaler@nottingham.ac.uk}
\author{Thorsten Altenkirch}
\email{thorsten.altenkirch@nottingham.ac.uk}
\affiliation{%
  \institution{University of Nottingham}
  \streetaddress{7301 Wollaton Rd}
  \city{Nottingham}
  \postcode{NG8 1BB}
  \country{United Kingdom}}

\begin{abstract}
TODO: select journals
- ACM Transactions on Modeling and Computer Simulation (TOMACS): https://tomacs.acm.org/
- Journal of Simulation: https://www.tandfonline.com/toc/tjsm20/current (https://www.tandfonline.com/doi/10.1057/jos.2010.3)
- JASSS: http://jasss.soc.surrey.ac.uk/JASSS.html

TODO: although we approach it from a high level, we have to present a bit of code at various points otherwise its too vague. 

IMPORTANT: don't over-exaggerate: instead of harder, say hard. instead of easier, say easy

TODO: it is paramount not to write against the established approach but for the functional approach. not to try to come up with arguments AGAINST the object-oriented approach but IN FAVOUR for the functional approach. In the end: don't tell the people that what they do sucks and that i am the saviour with my new method but: that i have a new method which might be of interest as it has a few nice advantages.

So far, the pure functional paradigm hasn't got much attention in Agent-Based Simulation (ABS) where the dominant programming paradigm is object-orientation, with Java, Python and C++ being its most prominent representatives. We claim that functional programming using Haskell is very well suited to implement complex, real-world agent-based models and brings with it a number of benefits. In this paper we will introduce the reader to the functional programming paradigm and explain how it can be applied to implementing ABS. Further we discuss benefits and advantages. As use-case we implemented the seminal Sugarscape model in Haskell.
\end{abstract}

%
% The code below should be generated by the tool at
% http://dl.acm.org/ccs.cfm
% Please copy and paste the code instead of the example below.
%
% TODO needs to be generated
%\begin{CCSXML}
%<ccs2012>
% <concept>
%  <concept_id>10010520.10010553.10010562</concept_id>
%  <concept_desc>Computer systems organization~Embedded systems</concept_desc>
%  <concept_significance>500</concept_significance>
% </concept>
% <concept>
%  <concept_id>10010520.10010575.10010755</concept_id>
%  <concept_desc>Computer systems organization~Redundancy</concept_desc>
%  <concept_significance>300</concept_significance>
% </concept>
% <concept>
%  <concept_id>10010520.10010553.10010554</concept_id>
%  <concept_desc>Computer systems organization~Robotics</concept_desc>
%  <concept_significance>100</concept_significance>
% </concept>
% <concept>
%  <concept_id>10003033.10003083.10003095</concept_id>
%  <concept_desc>Networks~Network reliability</concept_desc>
%  <concept_significance>100</concept_significance>
% </concept>
%</ccs2012>
%\end{CCSXML}
%
%\ccsdesc[500]{Computer systems organization~Embedded systems}
%\ccsdesc[300]{Computer systems organization~Redundancy}
%\ccsdesc{Computer systems organization~Robotics}
%\ccsdesc[100]{Networks~Network reliability}

%
% End generated code
%

\keywords{Agent-Based Simulation, Functional Programming, Haskell}

\maketitle

%*******************************************************************************
%*********************************** First Chapter *****************************
%*******************************************************************************

\chapter{Introduction}  %Title of the First Chapter
I noticed that it is pretty hard to convince an agent-based economics specialist who is not a computer scientist about a pure functional approach. My conjecture is that the implementation technique and method does not matter much to them because they have very little knowledge about programming and are almost always self-taught - they don't know about software-engineering, nothing about proper software-design and architecture, nothing about software-maintenance, nothing about unit-testing,... In the end they just "hack" the simulation in whatever language they are able to: C++, Visual Basic, Java or toolboxes like Netlogo. For them it is all about to \textit{get things done somehow} and not to get things done the right way or in a beautiful way - the way and the method doesn't matter, its just a necessary evil which needs to be done. Thus if functional programming could make their lives easier, then they will definitely welcome it. But functional programming is, i think, harder to learn and harder to understand - so one needs to provide an abstraction through EDSL. So I REALLY need to come up with convincing arguments why to use pure functional approaches in ACE THEY can understand, otherwise I will be lost and not heard (not published,...). \\

What ACE economists care for:

\begin{itemize}
\item Very: Qualitative modelling with quantitative results
\item Yes: Easy reproducibility
\item Likely: Reasoning about convergence?
\item Likely: EDSL
\end{itemize}

My contributions are: pure functional framework, functional agent-model for market-simulations, EDSL for market-simulations, qualitative / implicit modelling with quanitative results, reasoning in my framework about convergence \\

IDEA: could I develop non-causal modelling (models are expressed in terms of non-directed equations, modelled in signal-relations) to allow for qualitative modelling for the agent-based economists? See hybrid modelling paper of Yampa. \textbf{THIS WOULD BE A HUGE NOVEL CONTRIBUTION TO ACE ESPECIALLY WHEN COMBINED WITH AN EDSL AND PROVIDING FULL REFERENTIAL TRANSPARENCY TO KEEP THE ABILITY TO REASON ABOUT CONVERGENCE}. This should be covered in the "EDSL"-paper.

TODO: maybe i should really focus only on market models? otherwise too much? \\

central novelty of my PhD: model specification = runnable code. possible through EDSL. but only in specific subfield of ACE: market-models. need a functional description of the model, then translate it to model specification in EDSL and then run it to see dynamics. But: model specification moves closer to functional programming languages. \\

another novelty approach: model specification through qualitative instead of quantiative approaches. is this possible? \\

WHY FUNCTIONAL? "because its the ultimate approach to scientific computing": fewer bugs due to mutable state (why? is thos shown obkectively by someone?), shorter (again as above, productivity), more expressive and closer to math, EDSL, EDSL=model=simulation, better parallelising due to referental transparency, reasoning \\

scientific results need to be reproduced, especially when they have high impact. a more formal approach of specifying the model and the simulation (model=simulation) could lead to easier sharing and easier reporduction without ambigouites \\

pure functional agent-model \& theory, EDSL framework in Haskell for ACE

\begin{enumerate}
\item Which kind of problem do we have?
\item What aim is there? Solving the problem? 
\item How the aim is achieved by enumerating VERY CLEAR objectives.
\item What the impact one expects (hypothesis) and what it is (after results).
\end{enumerate}

Note: It is not in the interest of the researcher to develop new economic theories but to research the use of functional methods (programming and specification) in agent-based computational economics (ACE).

NOTE: Get the reader’s attention early in the introduction: motivation, significance, originality and novelty.

\section{Methods}
Methods need to be selected to implement the simulations. Special emphasis will be put on functional ones which will then be compared to established methods in the field of ABM/S and ACE. \\

Claim: non-programming environments are considered to be not powerful enough to capture the complexity of ACE implementations thus a programming approach to ACE will be always required.

\section{Scenarios}
To apply and test functional methods in ACE, four scenarios of ACE are selected and then the methods applied and compared with each other to see how each of them perform in comparison. The 4 selected scenarios represent a selection of the challenges posed in ACE: from very abstract ones to very operational ones.

\section{Comparison}
Each of the selected scenarios is then implemented using the selected methods where each solution is then compared against the following criteria: 

\begin{enumerate}
\item suitability for scientific computation
\item robustness
\item error-sources
\item testability
\item stability
\item extendability
\item size of code
\item maintainability
\item time taken for development
\item verification \& correctness
\item replications \& parallelism
\item EDSL
\end{enumerate}

This will then allow to compare the different methods against each other and to show under which circumstances functional methods shine and when they should not be used.

\section{Agent-Based Modelling and Simulation (ABM/S)}
ABM/S is a method of modelling and simulating a system where the global behaviour may be unknown but the behaviour and interactions of the parts making up the system is of knowledge (Wooldrige, M. (2009). An Introduction to MultiAgent Systems. John Wiley & Sons). Those parts, called agents, are modelled and simulated out of which then the aggregate global behaviour of the whole system emerges. Thus the central aspect of ABM/S is the concept of an Agent which can be understood as a metaphor for a pro-active unit, able to spawn new Agents, and interacting with other Agents in a network of neighbours by exchange of messages. The implementation of Agents can vary and strongly depends on the programming language and the kind of domain the simulation and model is situated in.

\section{Agent-Based Economics (ACE)}
According to Leigh Tesfatsion (Tesfatsion, L. (2006). Agent-based computational economics: A constructive approach to economic theory. In Tesfatsion, L. and Judd, K. L., editors, Handbook of Computational Economics, volume 2, chapter 16, pages 831–880. Elsevier, 1 edition.), one of the leading figures, ACE is "[...] computational modelling of economic processes (including whole economies) as open-ended dynamic systems of interacting agents." - thus lending perfectly to the use of ABM/S as already the name suggests. Whereas classical economic models fall short by only looking at the average, pure rational, individual interacting in anonymous markets, the ACE approach looks at heterogeneous, non-rational individuals interacting with each other in networks (Kirman, A. (2010). Complex Economics: Individual and Collective Rationality. Routledge, London ; New York, NY.). Thus ACE can be understood as a combination of computer-science, cognitive/social science and evolutionary economics.

\section{Functional programming}
TODO: read \cite{Backus1978}

The state-of-the-art approach to implementing Agents are object-oriented methods and programming as the metaphor of an Agent as presented above lends itself very naturally to object-orientation (OO). The author of this thesis claims that OO in the hands of inexperienced or ignorant programmers is dangerous, leading to bugs and hardly maintainable and extensible code. The reason for this is that OO provides very powerful techniques of organising and structuring programs through Classes, Type Hierarchies and Objects, which, when misused, lead to the above mentioned problems. Also major problems, which experts face as well as beginners are 1. state is highly scattered across the program which disguises the flow of data in complex simulations and 2. objects don’t compose as well as functions. The reason for this is that objects always carry around some internal state which makes it obviously much more complicated as complex dependencies can be introduced according to the internal state.
All this is tackled by (pure) functional programming which abandons the concept of global state, Objects and Classes and makes data-flow explicit. This then allows to reason about correctness, termination and other properties of the program e.g. if a given function exhibits side-effects or not. Other benefits are fewer lines of code, easier maintainability and ultimately fewer bugs thus making functional programming the ideal choice for scientific computing and simulation and thus also for ACE. A very powerful feature of functional programming is Lazy evaluation. It allows to describe infinite data-structures and functions producing an infinite stream of output but which are only computed as currently needed. Thus the decision of how many is decoupled from how to (Hughes, J. (1989). Why functional programming matters. Comput. J., 32(2):98–107.).
The most powerful aspect using pure functional programming however is that it allows the design of embedded domain specific languages (EDSL). In this case one develops and programs primitives e.g. types and functions in a host language (embed) in a way that they can be combined. The combination of these primitives then looks like a language specific to a given domain, in the case of this thesis ACE. The ease of development of EDSLs in pure functional programming is also a proof of the superior extensibility and composability of pure functional languages over OO (Henderson P. (1982). Functional Geometry. Proceedings of the 1982 ACM Symposium on LISP and Functional Programming.).
One of the most compelling example to utilize pure functional programming is the reporting of Hudak (Hudak P., Jones M. (1994). Haskell vs. Ada vs. C++ vs. Awk vs. ... An Experiment in Software Prototyping Productivity. Department of Computer Science, Yale University.)  where in a prototyping contest of DARPA the Haskell prototype was by far the shortest with 85 lines of code. Also the Jury mistook the code as specification because the prototype did actually implement a small EDSL which is a perfect proof how close EDSL can get to and look like a specification.

Functional languages can best be characterized by their way computation works: instead of \textit{how} something is computed, \textit{what} is computed is described. Thus functional programming follows a declarative instead of an imperative style of programming. The key points are:
\begin{itemize}
\item No assignment statements - variables values can never change once given a value.
\item Function calls have no side-effect and will only compute the results - this makes order of execution irrelevant, as due to the lack of side-effects the logical point in \textit{time} when the function is calculated within the program-execution does not matter.
\item higher-order functions
\item lazy evaluation
\item Looping is achieved using recursion, mostly through the use of the general fold or the more specific map.
\item Pattern-matching
\end{itemize}

This alone does not really explain the \textit{real} advantages of functional programming and one must look for better motivations using functional programming languages. One motivation is given in \cite{Hughes1989} which is a great paper explaining to non-functional programmers what the significance of functional programming is and helping functional programmers putting functional languages to maximum use by showing the real power and advantages of functional languages. The main conclusion is that \textit{modularity}, which is the key to successful programming, can be achieved best using higher-order functions and lazy evaluation provided in functional languages like Haskell. \cite{Hughes1989} argues that the ability to divide problems into sub-problems depends on the ability to glue the sub-problems together which depends strongly on the programming-language and \cite{Hughes1989} argues that in this ability functional languages are superior to structured programming.

TODO: comparison of functional and object-oriented programming. My points are:
\begin{itemize}
\item The way state can be changed and treated - distributed over multiple objects - is often very difficult to understand.
\item Inheritance is a dangerous thing if not used with care because inheritance introduces very strong dependencies which cannot be changed during runtime anymore.
\item Objects don't compose very well: \url{http://zeroturnaround.com/rebellabs/why-the-debate-on-object-oriented-vs-functional-programming-is-all-about-composition/}
\item (Nearly) impossible to reason about programs
\end{itemize}

In conclusion the upsides of functional programming as opposed to OO are:
\begin{itemize}
\item Much more explicit flow of data \& control
\item Much better compose-able
\item Much better parallelism
\end{itemize}

\section{Related Research}
Tim Sweeney, CTO of Epic Games gave an invited talk about how "future programming languages could help us write better code" by "supplying stronger typing, reduce run-time failures;  and the need for pervasive concurrency support, both implicit and explicit, to effectively exploit the several forms of parallelism present in games and graphics." \cite{Sweeney2006}. Although the fields of games and agent-based simulations seem to be very different in the end, they have also very important similarities: both are simulations which perform numerical computations and update objects - in games they are called "game-objects" and in abm they are called agents but they are in fact the same thing - in a loop either concurrently or sequential. His key-points were:

\begin{itemize}
\item Dependent types as the remedy of most of the run-time failures.
\item Parallelism for numerical computation: these are pure functional algorithms, operate locally on mutable state. Haskell ST, STRef solution enables encapsulating local heaps and mutability within referentially transparent code.
\item Updating game-objects (agents) concurrently using STM: update all objects concurrently in arbitrary order, with each update wrapped in atomic block - depends on collisions if performance goes up.
\end{itemize}

\section{Bugs and Errors in Agent-Based Simulation}
TODO: general introduction %https://en.wikipedia.org/wiki/Software_bug

The problem of correctness in agent-based simulations became more apparent in the work of Ionescu et al \cite{ionescu_dependently-typed_2012} which tried to replicate the work of Gintis \cite{gintis_emergence_2006}. In his work Gintis claimed to have found a mechanism in bilateral decentralized exchange which resulted in walrasian general equilibrium without the neo-classical approach of a tatonement process through a central auctioneer. This was a major break-through for economics as the theory of walrasian general equilibrium is non-constructive as it only postulates the properties of the equilibrium \cite{colell_microeconomic_1995} but does not explain the process and dynamics through which this equilibrium can be reached or constructed - Gintis seemed to have found just this process. Ionescu et al. \cite{ionescu_dependently-typed_2012} failed and were only able to solve the problem by directly contacting Gintis which provided the code - the definitive formal reference. It was found that there was a bug in the code which led to the "revolutionary" results which were seriously damaged through this error. They also reported ambiguity between the informal model description in Gintis paper and the actual implementation. TODO: it is still not clear what this bug was, find out! look at the master thesis 

This is supported by a talk \cite{sweeney_next_2006}, in which Tim Sweeney, CEO of Epic Games, discusses the use of main-stream imperative object-oriented programming languages (C++) in the context of Game Programming. Although the fields of games and ABS seem to be very different, in the end they have also very important similarities: both are simulations which perform numerical computations and update objects in a loop either concurrently or sequential \cite{gregory_game_2018}. Sweeney reports that reliability suffers from dynamic failure in such languages e.g. random memory overwrites, memory leaks, accessing arrays out-of-bounds, dereferencing null pointers, integer overflow, accessing uninitialized variables. He reports that 50\% of all bugs in the Game Engine Middleware Unreal can be traced back to such problems and presents dependent types as a potential rescue to those problems.

TODO: list common bugs in object-oriented / imperative programming
TODO: java solved many problems 
TODO: still object-oriented / imperative ultimately struggle when it comes to concurrency / parallelism due to their mutable nature.

TODO: \cite{vipindeep_list_2005}

TODO: software errors can be costly %https://raygun.com/blog/costly-software-errors-history/
TODO: bugs per loc %https://www.mayerdan.com/ruby/2012/11/11/bugs-per-line-of-code-ratio

\input{./tex/fpBackground.tex}

\section{Advanced Concepts}
In this section we give a brief overview over advanced concepts found in functional programming. 

\subsection{Parallelism and Concurrency}
TODO: write this section

TODO: in haskell we can distinguish between parallelism and concurrency in the types: parallelism is pure, concurrency is impure

TODO: explain STM, Problem: live locks, For a technical, in-depth discussion on Software Transactional Memory in Haskell we refer to the following papers: \citep{harris_composable_2005, osullivan_real_2008}.

\subsection{Functional Reactive Programming}
\label{sec:frp}
Functional Reactive Programming (FRP) is a way to implement systems with continuous and discrete time-semantics in functional programming. The central concept in FRP is the Signal Function which can be understood as a process over time which maps an input- to an output-signal. A signal in turn, can be understood as a value
which varies over time. Thus, signal functions have an awareness of the passing of time by having access to a $\Delta t$ which are positive time-steps with which the system is sampled. In general, signal functions can be understood to be computations that represent processes, which have an input of a specific type, process it and output a new type. Note that this is an important building block to represent agents in functional programming: by implementing agents as signal functions allows us to implement them as processes which act continuously over time, which implies a time-driven approach to ABS. We have also applied the concept of FRP to event-driven ABS \citep{meyer_event-driven_2014}.

FRP provides a number of functions for expressing time-semantics, generating events and making state-changes of the system. They allow to change system behaviour in case of events, run signal functions, generate deterministic (after fixed time) and stochastic (exponential arrival rate) events and provide random-number streams. 

For a technical, in-depth discussion on FRP in Haskell we refer to the following papers: \citep{wan_functional_2000, hughes_generalising_2000, hughes_programming_2005, nilsson_functional_2002, hudak_arrows_2003, courtney_yampa_2003, perez_functional_2016, perez_extensible_2017}

\subsection{Property-Based Testing}
TODO: write this section

Although property-based testing has been brought to non-functional languages like Java and Python as well, it has its origins in Haskell and it is here where it truly shines.

We found property-based testing particularly well suited for ABS. Although it is now available in a wide range of programming languages and paradigms, propert-based testing has its origins in Haskell \citep{claessen_quickcheck:_2000, claessen_testing_2002} and we argue that for that reason it really shines in pure functional programming. Property-based testing allows to formulate \textit{functional specifications} in code which then the property-testing library (e.g. QuickCheck \citep{claessen_quickcheck:_2000}) tries to falsify by automatically generating random test-data covering as much cases as possible. When an input is found for which the property fails, the library then reduces it to the most simple one. It is clear to see that this kind of testing is especially suited to ABS, because we can formulate specifications, meaning we describe \textit{what} to test instead of \textit{how} to test (again the declarative nature of functional programming shines through). Also the deductive nature of falsification in property-based testing suits very well the constructive nature of ABS.

For a technical, in-depth discussion on property-based testing in Haskell we refer to the following papers: \citep{claessen_quickcheck:_2000, claessen_testing_2002}.

%\subsection{Software Transactional Memory}
%Although there exist STM implementations in non-functional languages like Java and Python, due to the nature of Haskells type-system, the use of STM has unique benefits in this setting.
%
%Concurrent programming is notoriously difficult to get right because reasoning about the interactions of multiple concurrently running threads and low level operational details of synchronisation primitives and locks is \textit{very hard}. The main problems are:
%
%\begin{itemize}
%	\item Race conditions due to forgotten locks.
%	\item Deadlocks resulting from inconsistent lock ordering.
%	\item Corruption caused by uncaught exceptions.
%	\item Lost wakeups induced by omitted notifications.
%\end{itemize}
%
%Worse, concurrency does not compose. It is utterly difficult to write two functions (or methods in an object) acting on concurrent data which can be composed into a larger concurrent behaviour. The reason for it is that one has to know about internal details of locking, which breaks encapsulation and makes composition depend on knowledge about their implementation. Also it is impossible to compose two functions e.g. where one withdraws some amount of money from an account and the other deposits this amount of money into a different account: one ends up with a temporary state where the money is in none of either accounts, creating an inconsistency - a potential source for errors because threads can be rescheduled at any time.
%
%STM promises to solve all these problems for a very low cost. In STM one executes actions atomically where modifications made in such an action are invisible to other threads until the action is performed. Also the thread in which this action is run, doesn't see changes made by other threads - thus execution of STM actions are isolated. When a transaction exists one of the following things will occur:
%
%\begin{enumerate}
%	\item If no other thread concurrently modified the same data as us, all of our modifications will simultaneously become visible to other threads.
%	\item Otherwise, our modifications are discarded without being performed, and our block of actions is automatically restarted.
%\end{enumerate}
%
%Note that the ability to \textit{restart} a block of actions without any visible effects is only possible due to the nature of Haskells type-system which allows being explicit about side-effects: by restricting the effects to STM only ensures that no uncontrolled effects, which cannot be rolled-back, occur.
%
%STM is implemented using optimistic synchronisation. This means that instead of locking access to shared data, each thread keeps a transaction log for each read and write to shared data it makes. When the transaction exists, this log is checked whether other threads have written to memory it has read - it checks whether it has a consistent view to the shared data or not. This might look like a serious overhead but the implementations are very mature by now, being very performant and the benefits outweigh its costs by far.
%
%Applying this to our agents is very simple: because we already use Dunai / BearRiver as our FRP library, we can run in arbitrary Monadic contexts. This allows us to simply run agents within an STM Monad and execute each agent in their own thread. This allows then the agents to communicate concurrently with each other using the STM primitives without problems of explicit concurrency, making the concurrent nature of an implementation very transparent. Further through optimistic synchronisation we should arrive at a much better performance than with low level locking.
%
%Problem: live locks
%
%For a technical, in-depth discussion on Software Transactional Memory in Haskell we refer to the following papers: \citep{harris_composable_2005, osullivan_real_2008}.

\section{Related Research}

\cite{schneider_towards_2012} and \cite{vendrov_frabjous:_2014} present a domain-specific language for developing functional reactive agent-based simulations. This language called FRABJOUS is very human readable and easily understandable by domain-experts. It is not directly implemented in FRP/Haskell/Yampa but is compiled to Haskell/Yampa code which they claim is also readable. This is the direction we want to head but we don't want this intermediate step but look for how a most simple domain-specific language embedded in Haskell would look like. In this paper we explicitly dive deep into FRP And Yampa and see how we can combine the best of both.

\section{A functional approach}
The restrictions (see list in concepts section), functional programming imposes, directly removes serious sources of bugs which leads to simulation which is more likely to be correct. These restrictions force us to solve the fundamental concepts in ABS implementation differently. Note that we could fall back to using IO throughout all the simulation in which case we have access to mutable references but then we lose important compile-time guarantees and introduce those serious sources of bugs we want to get rid of - also testing becomes more complicated and not as strong anymore because we cannot guarantee at compile time that no random IO stuff is happening within the agents. Also note that obviously no one would do random IO stuff in an agent (e.g. read from a file, open connection to server...) but one must not understimate the value of guaranteeing its absence at compile-time.

Due to the fundamentally different approaches of pure Functional Programming (pure FP) an ABS needs to be implemented fundamentally different as well compared to traditional object-oriented approaches (OO). We face the following challenges:

\begin{enumerate}
	\item How can we represent an Agent? \\
	In OO the obvious approach is to map an agent directly onto an object which encapsulates data and provides methods which implement the agents actions. Obviously we don't have objects in pure FP thus we need to find a different approach to represent the agents actions and to encapsulate its state.
	
	\item How can we represent state in an Agent? \\
	In the classic OO approach one represents the state of an Agent explicitly in mutable member variables of the object which implements the Agent. As already mentioned we don't have objects in pure FP and state is immutable which leaves us with the very tricky question how to represent state of an Agent which can be actually updated.
	
	\item How can we implement proactivity of an Agent? \\
	In the classic OO approach one would either expose the current time-delta in a mutable variable and implement time-dependent functions or ignore it at all and assume agents act on every step. At first this seems to be not a big deal in pure FP but when considering that it is yet unclear how to represent Agents and their state, which is directly related to time-dependent and reactive behaviour it raises the question how we can implement time-varying and reactive behaviour in a purely functional way.
	
	\item How can we implement the agent-agent interaction? \\
	In the classic OO approach Agents can directly invoke other Agents methods which makes direct Agent interaction \textit{very} easy. Again this is obviously not possible in pure FP as we don't have objects with methods and mutable state inside.
		
	\item How can we represent an environment and its various types? \\
	In the classic OO approach an environment is almost always a mutable object which can be easily made dynamic by implementing a method which changes its state and then calling it every step as well. In pure FP we struggle with this for the same reasons we face when deciding how to represent an Agent, its state and proactivity.
	
	\item How can we implement the agent-environment interaction? \\
	In the classic OO approach agents simply have access to the environment either through global mechanisms (e.g. Singleton or simply global variable) or passed as parameter to a method and call methods which change the environment. Again we don't have this in pure FP as we don't have objects and globally mutable state.
	
	\item How can we step the simulation? \\
	In the classic OO approach agents are run one after another (with being optionally shuffled before to uniformly distribute the ordering) which ensures mutual exclusive access in the agent-agent and agent-environment interactions. Obviously in pure FP we cannot iteratively mutate a global state.

	\item An Agent-Interface \\
	How can we interface with the agent: how does the agent reveal information about itself? We will see that functional programming supports very strong encapsulation of local state which is not accessible and mutable from outside. This makes testing in some respekt easier, in some harder. 
\end{enumerate}

\subsection{Agent representation, state and proactivity}
Whereas in imperative programming (the OO which we refer to in this paper is built on the imperative paradigm) the fundamental building block is the destructive assignment, in FP the building blocks are obviously functions which can be evaluated.
Thus we have no other choice than to represent our Agents using a function which implements their behaviour. This function must be time-aware somehow and allow us to react to time-changes and inputs. Fortunately there exists already an approach to time-aware, reactive programming which is termed Functional Reactive Programming (FRP). This paradigm has evolved over the year and current modern FRP is built around the concept of a signal-function which transforms an input-signal into an output-signal. An input-signal can be seen as a time-varying value. Signal-functions are implemented as continuations which allows to capture local state using closures. Modern FRP also provides feedback functions which provides convenient methods to capture and update local state from the previous time-step with an initial state provided at time = 0.

- pure functions don't have a notion of communication as opposed to method calls in object-oriented languages like java

- time is represented using the FRP concept: Signal-Functions which are sampled at (fixed) time-deltas, the dt is never visible directly but only reflected in the code and read-only.
- no method calls => continuous data-flow instead
	
Viewing agent-agent interaction as simple method calls implies the following:
- it takes no time
- it has a synchronous and transactional character
- an agent gives up control over its data / actions or at least there is always the danger that it exposes too much of its interface and implementation details. 
- agents equals objects, which is definitely NOT true. Agents 

data-flow
synchronous agent transactions

- still need transactions between two agents e.g. trading occurs over multiple steps (makeoffer, accept/refuse, finalize/abort) 
		-> exactly define what TX means in ABS
			-> exclusive between 2 agents
			-> state-changes which occur over multiple steps and are only visible to the other agents after the TX has commited
			-> no read/write access to this state is allowed to other agents while the TX is active
			-> a TX executes in a single time-step and can have an arbitrary number of tx-steps
		-> it is easily possible using method-calls in OOP but in our pure functional approach it is not possible
		-> parallel execution is being a problem here as TX between agents are very easy with sequential
		-> an agent must be able to transact with as many other agents as it wants to in the same time-step
		-> no time passes between transactions
		=> what we need is a 'all agents transact at the same time'
			-> basically we can implement it by running the SFs of the agents involved in the TX repeatedly with dt=0 until there are no more active TXs
			-> continuations (SFs) are perfectly suited for this as we can 'rollback' easily by using the SF before the TX has started
		
\subsection{Environment representation and interaction}

no global shared mutable environment, having different options:
- non-active read-only (SIR): no agent, as additional argument to each agent
- pro-active read-only (?): environment as agent, broadcast environment updates as data-flow
- non-active read/write (?): no agent, StateT in agents monad stack
- pro-active read/write (Sugarscape): environment as agent, StateT in agents monad stack

care must be taken in case of agent-transactions: when aborting/refusing all changes to the environment must be rolled back => instead of StateT use a transactional monad which allows us to revert changes to a save point at the start of the TX. if we drag the environment through all agents then we could easily revert changes but that then requires to hard-code the environment concept deep into the simulation scheduling/stepping which brings lots of inconveniences, also it would need us to fold the resulting multiple environments back into a single. If we had an environment-centric view then probably this is what we want but in ABS the focus is on the agents

question is if the TX sf runs in the same monad aw the agent or not. i opt for identity monad which prevents modification of the Environment in a transaction

also need to motivate the dt=0 in all TX processing: conceptually it all happens instantaneously (although arbitration is sequential) but agents must act time-sensitive

for environment we need transactional and shared state behaviour where we can have mutual exclusive access to shared data but also roll back changes we made. it should run deterministic when running agents not truly parallel. solution: run environment in a transactional state monad (TX monad). although the agents are executed in parallel in the end it (map) runs sequentially. this passes a mutable state through all agents which can act on it an roll back actions e.g. in case of a failed agent TX. if we dont need transactional behaviour then just use StateT monad. this ensures determinism. pro active environment is also easily possible by writing to the state. this approach behaves like sequential transactional although the agents run in parallel but how is this possible when using mapMSF ?

\subsection{Stepping the simulation}

- parallel update only, sequential is deliberately abandoned due to:
		-> reality does not behave this way
		-> if we need transactional behaviour, can use STM which is more explicit
		-> it is translates directly to a map which is very easy to reason about (sequential is basically a fold which is much more difficult to reason about)
		-> is more natural in functional programming
		-> it exists for 'transactional' reasons where we need mutual exclusive access to environment / other agents
			-> we provide a more explicit mechanism for this: Agent Transactions
			

\section{Multi-Method Simulation}

\subsection{System Dynamics}
towards paper: sd is nearly correct by construction

\subsection{Discrete Event Simulation}
use MSFs and event-queue

\section{Discussion}

\subsection{Other Models}
TODO: mention that we have also implemented other models, which also work without time-semantics (all agents make a move at discrete time-steps and do not really rely on some notion of time). 

\subsection{Time-Semantics}
The main reason for building our pure functional ABMS approach on top of Yampa was to leverage the powerful time-semantics of Yampa which allows us to implement important concepts of ABMS:

state-chart: agents are at all time of their life-cycle in one state and can switch between multiple states using transitions 
timed transitions: transition to another state/behaviour happens at a discrete time
rate transitions: transition happens with a given rate
message transition: transition upon receiving a given message 

\subsection{Agents as Signals}
Due to the underlying nature and motivation of Functional Reactive Programming (und im speziellen) Yampa, Agents can be seen as Signals which is generated and consumed by a Signal-Function which is the behaviour of an Agent. If an Agent does not change the OUTPUT-signal is constant, if the agent changes e.g. by sending a message, changing its state,... the OUTPUT signal changes. A dead agent has no signal at all.

\subsection{Time-Sampling}
sampling rate depends on the transition times \& rates of the model. when e.g. the contact rate is 5 then the sampling dt should be below 0.2

\subsection{System Dynamics}
can emulate system dynamics due to the parallel update-strategy and continuous time-flow semantics

\subsection{Discrete Event Simulation}
DES in FrABMS? how easily can we implement server/queue systems? do they also look like a specification? potential problem: ordering of messages is not guaranteed by now

\subsection{Advantages}
advantages:
	- no side-effects within agents leads to much safer code
	- edsl for time-semantics
	- declarative style: agent-implementation looks like a model-specification
	- reasoning and verification
	- sequential and parallel
	- powerful time-semantics
	- arrowized programming is optional and only required when utilizing yampas time-semantics. if the model does not rely on time-semantics, it can use monadic-programming by building on the existing monadic functions in the EDSL which allow to run in the State-Monad which simplifies things very much
	- when to use yampas arrowized programing: time-semantics, simple state-chart agents 
	- when not using yampas facilities: in all the other cases e.g. SugarScape is such a case as it proceeds in unit time-steps and all agents act in every time-step
	- can implement System Dynamics building on Yampas facilities with total ease	
	- get replications for free without having to worry about side-effects and can even run them in parallel without headaches
	- cant mess around with time because delta-time is hidden from you (intentional design-decision by Yampa). this would be only very difficult and cumbersome to achieve in an object-oriented approach. TODO: experiment with it in Java - how could we actually implement this? I think it is impossible: may only achieve this through complicated application of patterns and inheritance but then has the problem of how to update the dt and more important how to deal with functions like integral which accumulates a value through closures and continuations. We could do this in OO by having a general base-class e.g. ContinuousTime which provides functions like updateDt and integrate, but we could only accumulate a single integral value.
	- reproducibility statically guaranteed
	- cannot mess around with dt
	- code == specification
	- rule out serious class of bugs
	- different time-sampling leads to different results e.g. in wildfire \& SIR but not in Prisoners Dilemma. why? probabilistic time-sampling?
	- reasoning about equivalence between SD and ABS implementation in the same framework
	- recursive implementations
	
	- we can statically guarantee the reproducibility of the simulation because: no side effects possible within the agents which would result in differences between same runs (e.g. file access, networking, threading), also timedeltas are fixed and do not depend on rendering performance or userinput	
	
\subsection{Disadvantages}
disadvantages:
	- performance is low
	- reasoning about performance is very difficult
	- very steep learning curve for non-functional programmers
	- learning a new EDSL
	- think ABMS different: when to use async messages, when to use sync conversations


[ ] important: increasing sampling freqzency and increasing number of steps so that the same number of simulation steps are executed should lead to same results. but it doesnt. why?
[ ] hypothesis: if time-semantics are involved then event ordering becomes relevant for emergent patterns. there are no tine semantics in heroes and cowards but in the prisoners dilemma
[ ] can we implement different types of agents interacting with each other in the same simulation ? with different behaviour funcs, digferent state? yes, also not possible in NetLogo to my knowledge. but they must have the same messages, emvironment 

[ ] Hypothesis: we can combine with FrABS agent-based simulation and system dynamics (this has been proved by example!)

\chapter{Conclusions}
\label{ch:conclusions}

This chapter concludes the whole thesis and outlines future research. Roughly 20\% exists already.

%we now know how to engineer time- and event-driven ABS with complex state both in the agent and environment, main difficulty is direct agent-interaction (see macal classification into 4 types of ABS), compile-time guaranteed reproducibility, explicit handling of complex state (read only, read/write), concurrency explicit and limited to STM, very promising concurrency but direct agent-interactions main problem (erlang as a rescue?), main drawbacks: everything is explicit, performance

\section{Further Research}
clearly outline the ideas for further research

\subsection{A general purpose library}
generalise concepts explored into a pure functional ABS library in Haskell (called chimera)

\subsection{Dependent and linear types}
dependent types and linear types are the next big step, towards a stronger formalisation of agents and ABS,
focus on the equilibrium - totality correspondence

\subsection{Concurrent event-driven ABS}
stm based concurrency for event-driven ABS using parallel DES. challenge is the time-warp implementation using monads. in general it should be easy to roll-back agents actions but with monads we have to be careful - for some monads rolling back is not neccessary e.g. rand and reader, for others it is, and for some it is impossible e.g. IO

\section{Further Research}
\label{sec:further_research}

We see this paper as an intermediary and necessary step towards dependent types for which we first needed to understand the potentials and limitations of a non-dependently typed pure functional approach in Haskell. Dependent types are extremely promising in functional programming as they allow us to express stronger guarantees about the correctness of programs and go as far as allowing to formulate programs and types as constructive proofs \cite{wadler_propositions_2015} which must be total by definition \cite{thompson_type_1991}, \cite{altenkirch_why_2005}, \cite{altenkirch_pi_2010}, \cite{program_homotopy_2013}. So far no research using dependent types in agent-based simulation exists at all and it is not clear whether dependent types make sense in this context. In our next paper we want to explore this for the first time and ask more specifically how we can add dependent types to our pure functional approach, which conceptual implications this has for ABS and what we gain from doing so. We plan on using Idris \cite{brady_idris_2013}, \cite{brady_type-driven_2017} as the language of choice as it is very close to Haskell with focus on real-world application and running programs as opposed to other languages with dependent types e.g. Agda and Coq which serve primarily as proof assistants.
It would be of immense interest whether we could apply dependent types to the model meta-level or not - this boils down to the question if we can encode our model specification in a dependent type way. This would allow the ABS community for the first time to reason about a proper formalisation of a model. We plan to implement a total and terminating implementation of our approach which would be a formal proof-by-construction that the agent-based approach of the SIR model terminates after a finite number of steps.

\begin{acks}
The authors would like to thank
\end{acks}

% Bibliography
\bibliographystyle{ACM-Reference-Format}
%% Citation style
%% Note: author/year citations are required for papers published as an
%% issue of PACMPL.
%%\citestyle{acmauthoryear}   %% For author/year citations
\bibliography{../../../references/phdReferences.bib}

\end{document}
