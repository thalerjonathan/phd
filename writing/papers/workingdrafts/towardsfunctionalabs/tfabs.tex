% This file provides a template to format JASSS articles v.0.5, 2015-03-24

% In order to be compiled, the following packages should be installed in your system:
% graphicx,xcolor, booktabs,amsmath, ifthen, geometry, authblk, natbib, endnotes
 
 % Please use pdflatex

% The font used is Source Sans Pro, normally included in Tex Live and other  major LaTeX distributions
% Location at CTAN: http://www.ctan.org/tex-archive/fonts/sourcesanspro/
% See also: http://www.tug.dk/FontCatalogue/sourcesanspro/

%%%%%%%%%%%%%%%%%%%%%%%%%%%%%%%%%%%%%%%%%%%%%%

\documentclass{JASSS}
	
%%%%%%%%%%%%%%%%%%%%%%%%%%%%%%%%%%%%%%%%%%%%%%

% Editorial fields (to be set in case of publication)	
% Please leave this section untouched
%\doinum{10.18564/jasss.xxxx}
%\volume{xx}
%\issue{x}
%\article{x}
%\pubyear{20xx}
%\received{dd-mmm-yyyy}
%\accepted{dd-mmm-yyyy}
%\published{dd-mmm-yyyy}

%%%%%%%%%%%%%%%%%%%%%%%%%%%%%%%%%%%%%%%%%%%%%% 

% title, authors and affiliations	

\title{The Agent's new Cloths \\ {\large Towards functional programming in Agent-Based Simulation}}

%All authors should be included in the submission. To anonymise the submission, set to 'true' the \reviewcopy command below. However, before submitting  you can check that all the informations you entered are correct by temporarily setting it to 'false'. Please remember to set it back to 'true' before the submission.
\reviewcopy{true} 

\author[1]{Jonathan Thaler}
\author[1]{Peer-Olaf Siebers}
\author[1]{Thorsten Altenkirch}
\affil[1]{University of Nottingham, 7301 Wollaton Rd, Nottingham, NG8 1BB, United Kingdom}

% Subsequent author should be included using the following template. You can add more in case of need, just remember to appropriately set the corresponding number.  Please check the the authblk package documentation  in case of doubts

%\author[2]{Second author here}
%\affil[2]{Affiliation of the second author here}

%\author[3]{Third author here}
%\affil[3]{Affiliation of the third author here}

%\author[4]{fourth author here}
%\affil[4]{fourth of the third author here}

% In case of multiple affiliation for the same author:

%\author[1,2]{Author name here}
%\affil[1]{First affiliation}
%\affil[2]{Second affiliation}

%  In case of several authors sharing the same affiliation:

%\author[1]{First author name}
%\author[1]{Second author name}
%\affil[1]{Affiliation}

% email for the corresponding author
\email{jonathan.thaler@nottingham.ac.uk}

%%%%%%%%%%%%%%%%%%%%%%%%%%%%%%%%%%%%%%%%%%%%%%

% NOTES. Please use endnotes. Notes should be placed after the main text and appendices and before the references. Also remember to uncomment the \theendnotes commands at the end of the document (just before the references)

%%%%%%%%%%%%%%%%%%%%%%%%%%%%%%%%%%%%%%%%%%%%%%

% REFERENCES should be included using the \citep, \citet, etc. commands provided by the natbib package

\usepackage{natbib}
	\setcitestyle{authoryear,round,aysep={}}
	
%%%%%%%%%%%%%%%%%%%%%%%%%%%%%%%%%%%%%%%%%%%%%%	

% EXTRA PACKAGES. Please place here any extra package you need along with your own command definitions
%\usepackage{graphicx}
%\usepackage{xcolor}
%\usepackage{booktabs}
%\usepackage{amsmath}
%\usepackage{ifthen}
%\usepackage{geometry}
%\usepackage{authblk}
%\usepackage{natbib}
%\usepackage{endnotes}
%\usepackage{minted}
\usepackage{hyperref}

%%%%%%%%%%%%%%%%%%%%%%%%%%%%%%%%%%%%%%%%%%%%%%

\begin{document}
\maketitle 

%%%%%%%%%%%%%%%%%%%%%%%%%%%%%%%%%%%%%%%%%%%%%%

% Abstract and keywords

\begin{abstract}
TODO: parallelism for free because all isolated e.g. running multiple replications or parameter-variations

TODO: it is paramount not to write against the established approach but for the functional approach. not to try to come up with arguments AGAINST the object-oriented approach but IN FAVOUR for the functional approach. In the end: dont tell the people that what they do sucks and that i am the saviour with my new method but: that i have a new method which might be of interest as it has a few nice advantages.

So far, the pure functional paradigm hasn't got much attention in Agent-Based Simulation (ABS) where the dominant programming paradigm is object-orientation, with Java, Python and C++ being its most prominent representatives. We claim that functional programming using Haskell is very well suited to implement complex, real-world agent-based models and brings with it a number of benefits. In this paper we will introduce the reader to the functional programming paradigm and explain how it can be applied to implementing ABS. Further we discuss benefits and advantages. As use-case we implemented the seminal Sugarscape model in Haskell.
\end{abstract}

\begin{keywords}
Agent-Based Simulation, Functional Programming, Haskell
\end{keywords}

%%%%%%%%%%%%%%%%%%%%%%%%%%%%%%%%%%%%%%%%%%%%%%
% Start of  paragraph numbering. Please leave this untouched
\parano{}

%%%%%%%%%%%%%%%%%%%%%%%%%%%%%%%%%%%%%%%%%%%%%%

% MAIN TEXT

%*******************************************************************************
%*********************************** First Chapter *****************************
%*******************************************************************************

\chapter{Introduction}  %Title of the First Chapter
I noticed that it is pretty hard to convince an agent-based economics specialist who is not a computer scientist about a pure functional approach. My conjecture is that the implementation technique and method does not matter much to them because they have very little knowledge about programming and are almost always self-taught - they don't know about software-engineering, nothing about proper software-design and architecture, nothing about software-maintenance, nothing about unit-testing,... In the end they just "hack" the simulation in whatever language they are able to: C++, Visual Basic, Java or toolboxes like Netlogo. For them it is all about to \textit{get things done somehow} and not to get things done the right way or in a beautiful way - the way and the method doesn't matter, its just a necessary evil which needs to be done. Thus if functional programming could make their lives easier, then they will definitely welcome it. But functional programming is, i think, harder to learn and harder to understand - so one needs to provide an abstraction through EDSL. So I REALLY need to come up with convincing arguments why to use pure functional approaches in ACE THEY can understand, otherwise I will be lost and not heard (not published,...). \\

What ACE economists care for:

\begin{itemize}
\item Very: Qualitative modelling with quantitative results
\item Yes: Easy reproducibility
\item Likely: Reasoning about convergence?
\item Likely: EDSL
\end{itemize}

My contributions are: pure functional framework, functional agent-model for market-simulations, EDSL for market-simulations, qualitative / implicit modelling with quanitative results, reasoning in my framework about convergence \\

IDEA: could I develop non-causal modelling (models are expressed in terms of non-directed equations, modelled in signal-relations) to allow for qualitative modelling for the agent-based economists? See hybrid modelling paper of Yampa. \textbf{THIS WOULD BE A HUGE NOVEL CONTRIBUTION TO ACE ESPECIALLY WHEN COMBINED WITH AN EDSL AND PROVIDING FULL REFERENTIAL TRANSPARENCY TO KEEP THE ABILITY TO REASON ABOUT CONVERGENCE}. This should be covered in the "EDSL"-paper.

TODO: maybe i should really focus only on market models? otherwise too much? \\

central novelty of my PhD: model specification = runnable code. possible through EDSL. but only in specific subfield of ACE: market-models. need a functional description of the model, then translate it to model specification in EDSL and then run it to see dynamics. But: model specification moves closer to functional programming languages. \\

another novelty approach: model specification through qualitative instead of quantiative approaches. is this possible? \\

WHY FUNCTIONAL? "because its the ultimate approach to scientific computing": fewer bugs due to mutable state (why? is thos shown obkectively by someone?), shorter (again as above, productivity), more expressive and closer to math, EDSL, EDSL=model=simulation, better parallelising due to referental transparency, reasoning \\

scientific results need to be reproduced, especially when they have high impact. a more formal approach of specifying the model and the simulation (model=simulation) could lead to easier sharing and easier reporduction without ambigouites \\

pure functional agent-model \& theory, EDSL framework in Haskell for ACE

\begin{enumerate}
\item Which kind of problem do we have?
\item What aim is there? Solving the problem? 
\item How the aim is achieved by enumerating VERY CLEAR objectives.
\item What the impact one expects (hypothesis) and what it is (after results).
\end{enumerate}

Note: It is not in the interest of the researcher to develop new economic theories but to research the use of functional methods (programming and specification) in agent-based computational economics (ACE).

NOTE: Get the reader’s attention early in the introduction: motivation, significance, originality and novelty.

\section{Methods}
Methods need to be selected to implement the simulations. Special emphasis will be put on functional ones which will then be compared to established methods in the field of ABM/S and ACE. \\

Claim: non-programming environments are considered to be not powerful enough to capture the complexity of ACE implementations thus a programming approach to ACE will be always required.

\section{Scenarios}
To apply and test functional methods in ACE, four scenarios of ACE are selected and then the methods applied and compared with each other to see how each of them perform in comparison. The 4 selected scenarios represent a selection of the challenges posed in ACE: from very abstract ones to very operational ones.

\section{Comparison}
Each of the selected scenarios is then implemented using the selected methods where each solution is then compared against the following criteria: 

\begin{enumerate}
\item suitability for scientific computation
\item robustness
\item error-sources
\item testability
\item stability
\item extendability
\item size of code
\item maintainability
\item time taken for development
\item verification \& correctness
\item replications \& parallelism
\item EDSL
\end{enumerate}

This will then allow to compare the different methods against each other and to show under which circumstances functional methods shine and when they should not be used.

\section{Agent-Based Modelling and Simulation (ABM/S)}
ABM/S is a method of modelling and simulating a system where the global behaviour may be unknown but the behaviour and interactions of the parts making up the system is of knowledge (Wooldrige, M. (2009). An Introduction to MultiAgent Systems. John Wiley & Sons). Those parts, called agents, are modelled and simulated out of which then the aggregate global behaviour of the whole system emerges. Thus the central aspect of ABM/S is the concept of an Agent which can be understood as a metaphor for a pro-active unit, able to spawn new Agents, and interacting with other Agents in a network of neighbours by exchange of messages. The implementation of Agents can vary and strongly depends on the programming language and the kind of domain the simulation and model is situated in.

\section{Agent-Based Economics (ACE)}
According to Leigh Tesfatsion (Tesfatsion, L. (2006). Agent-based computational economics: A constructive approach to economic theory. In Tesfatsion, L. and Judd, K. L., editors, Handbook of Computational Economics, volume 2, chapter 16, pages 831–880. Elsevier, 1 edition.), one of the leading figures, ACE is "[...] computational modelling of economic processes (including whole economies) as open-ended dynamic systems of interacting agents." - thus lending perfectly to the use of ABM/S as already the name suggests. Whereas classical economic models fall short by only looking at the average, pure rational, individual interacting in anonymous markets, the ACE approach looks at heterogeneous, non-rational individuals interacting with each other in networks (Kirman, A. (2010). Complex Economics: Individual and Collective Rationality. Routledge, London ; New York, NY.). Thus ACE can be understood as a combination of computer-science, cognitive/social science and evolutionary economics.

\section{Functional programming}
TODO: read \cite{Backus1978}

The state-of-the-art approach to implementing Agents are object-oriented methods and programming as the metaphor of an Agent as presented above lends itself very naturally to object-orientation (OO). The author of this thesis claims that OO in the hands of inexperienced or ignorant programmers is dangerous, leading to bugs and hardly maintainable and extensible code. The reason for this is that OO provides very powerful techniques of organising and structuring programs through Classes, Type Hierarchies and Objects, which, when misused, lead to the above mentioned problems. Also major problems, which experts face as well as beginners are 1. state is highly scattered across the program which disguises the flow of data in complex simulations and 2. objects don’t compose as well as functions. The reason for this is that objects always carry around some internal state which makes it obviously much more complicated as complex dependencies can be introduced according to the internal state.
All this is tackled by (pure) functional programming which abandons the concept of global state, Objects and Classes and makes data-flow explicit. This then allows to reason about correctness, termination and other properties of the program e.g. if a given function exhibits side-effects or not. Other benefits are fewer lines of code, easier maintainability and ultimately fewer bugs thus making functional programming the ideal choice for scientific computing and simulation and thus also for ACE. A very powerful feature of functional programming is Lazy evaluation. It allows to describe infinite data-structures and functions producing an infinite stream of output but which are only computed as currently needed. Thus the decision of how many is decoupled from how to (Hughes, J. (1989). Why functional programming matters. Comput. J., 32(2):98–107.).
The most powerful aspect using pure functional programming however is that it allows the design of embedded domain specific languages (EDSL). In this case one develops and programs primitives e.g. types and functions in a host language (embed) in a way that they can be combined. The combination of these primitives then looks like a language specific to a given domain, in the case of this thesis ACE. The ease of development of EDSLs in pure functional programming is also a proof of the superior extensibility and composability of pure functional languages over OO (Henderson P. (1982). Functional Geometry. Proceedings of the 1982 ACM Symposium on LISP and Functional Programming.).
One of the most compelling example to utilize pure functional programming is the reporting of Hudak (Hudak P., Jones M. (1994). Haskell vs. Ada vs. C++ vs. Awk vs. ... An Experiment in Software Prototyping Productivity. Department of Computer Science, Yale University.)  where in a prototyping contest of DARPA the Haskell prototype was by far the shortest with 85 lines of code. Also the Jury mistook the code as specification because the prototype did actually implement a small EDSL which is a perfect proof how close EDSL can get to and look like a specification.

Functional languages can best be characterized by their way computation works: instead of \textit{how} something is computed, \textit{what} is computed is described. Thus functional programming follows a declarative instead of an imperative style of programming. The key points are:
\begin{itemize}
\item No assignment statements - variables values can never change once given a value.
\item Function calls have no side-effect and will only compute the results - this makes order of execution irrelevant, as due to the lack of side-effects the logical point in \textit{time} when the function is calculated within the program-execution does not matter.
\item higher-order functions
\item lazy evaluation
\item Looping is achieved using recursion, mostly through the use of the general fold or the more specific map.
\item Pattern-matching
\end{itemize}

This alone does not really explain the \textit{real} advantages of functional programming and one must look for better motivations using functional programming languages. One motivation is given in \cite{Hughes1989} which is a great paper explaining to non-functional programmers what the significance of functional programming is and helping functional programmers putting functional languages to maximum use by showing the real power and advantages of functional languages. The main conclusion is that \textit{modularity}, which is the key to successful programming, can be achieved best using higher-order functions and lazy evaluation provided in functional languages like Haskell. \cite{Hughes1989} argues that the ability to divide problems into sub-problems depends on the ability to glue the sub-problems together which depends strongly on the programming-language and \cite{Hughes1989} argues that in this ability functional languages are superior to structured programming.

TODO: comparison of functional and object-oriented programming. My points are:
\begin{itemize}
\item The way state can be changed and treated - distributed over multiple objects - is often very difficult to understand.
\item Inheritance is a dangerous thing if not used with care because inheritance introduces very strong dependencies which cannot be changed during runtime anymore.
\item Objects don't compose very well: \url{http://zeroturnaround.com/rebellabs/why-the-debate-on-object-oriented-vs-functional-programming-is-all-about-composition/}
\item (Nearly) impossible to reason about programs
\end{itemize}

In conclusion the upsides of functional programming as opposed to OO are:
\begin{itemize}
\item Much more explicit flow of data \& control
\item Much better compose-able
\item Much better parallelism
\end{itemize}

\section{Related Research}
Tim Sweeney, CTO of Epic Games gave an invited talk about how "future programming languages could help us write better code" by "supplying stronger typing, reduce run-time failures;  and the need for pervasive concurrency support, both implicit and explicit, to effectively exploit the several forms of parallelism present in games and graphics." \cite{Sweeney2006}. Although the fields of games and agent-based simulations seem to be very different in the end, they have also very important similarities: both are simulations which perform numerical computations and update objects - in games they are called "game-objects" and in abm they are called agents but they are in fact the same thing - in a loop either concurrently or sequential. His key-points were:

\begin{itemize}
\item Dependent types as the remedy of most of the run-time failures.
\item Parallelism for numerical computation: these are pure functional algorithms, operate locally on mutable state. Haskell ST, STRef solution enables encapsulating local heaps and mutability within referentially transparent code.
\item Updating game-objects (agents) concurrently using STM: update all objects concurrently in arbitrary order, with each update wrapped in atomic block - depends on collisions if performance goes up.
\end{itemize}

\section{Background}

\subsection{Schelling Segregation}
We follow in our implementation the original paper of Schelling as in \cite{schelling_dynamic_1971} where we focus on the \textit{Area Distribution} section (Schelling starts with movement in a linear, 1-dimensional world where agents are able to move to the nearest point which meets the agents satisfaction but this is not what we follow here). One assumes a discrete 2-dimensional lattice-world with NxM fields. Each field is either occupied by an agent of a given color (e.g. Red or Green) or is free. Each field has 8 neighbours, which denotes a Moore-Neighbourhood. In Schellings original work the lattice-world is limited at its borders but we assume a torus world which is wrapped around in both the x- and y-dimensions resulting in 8 neighbours also for fields at the border. The occupation density was set by Schelling to be about 70\%-75\% which he identifies as being a setting which allows the agents to move around freely without making the lattice-world too sparse.
Now the agents make their move sequentially one after another. In each move an agent calculates the number of neighbours which are of equal color. If the number satisfies the agents needs about the neighbourhood then the agent is regarded as being 'happy' and will stay on this field. On the other hand the agent moves to the nearest unoccupied field which satisfies its needs. An agent which moves selects an unoccupied place randomly relative from its current place within a rectangle of side-length 2r where its current place is at the center. The interpretation for that behaviour is that agents won't move too far as it could be costly. Also children might attend a school in this area or the family has friends in this area, so they don't want to break that.



Agents just move depending on their movement-strategy to another place if they are not happy on the current one - they don't care how the target place is in the present or in the future, they will decide again in the next time-step. The interpretation for that behaviour is: agents want to 'just get out' at any cost, not caring what the future place will look like - it might be better or worse but they will see then.

\subsubsection{Optimizing behaviour}
TODO: define utility

The original schelling model didn't have a move-optimizing behaviour, meaning agents are just binary: if it is happy it will not move, if it is unhappy it will move but they won't care where they move. We introduce local move-optimizing behaviours which can be interpreted as being realistic in the real-world. It is important to note that we focus on \textit{local} instead of \textit{global} move-optimization: the agents are limited in their reasoning-capabilities and have limited information available: they cannot check out \textit{every} place and pick the globally best one.\\

\subsubsection{Anticipating behaviour}
Schelling explicitly mentions in \cite{schelling_dynamic_1971} that nobody anticipates moves of others. This is what we introduce using the recursive simulation.

TODO: is this optimizing behaviour in the spirit of schellings original work? 

\paragraph{Optimizing future} Agents pick an unoccupied random place and move to it if it increases their utility in the future. The interpretation for that behaviour is: agents heard about a place which will be cool in the future.

\paragraph{Optimizing present \& future} Agents pick an unoccupied random place and move to it if it increases their utility in the now and in the future. The interpretation for that behaviour is: agents heard about a cool spot in town, check it out and move to it if they like it but they also anticipate the coolness of the place in the future and if it seems that the place is going down then they won't move there.

\subsection{Related Research}
TODO: \cite{kirman_complex_2010} mention kirman complex economics where he investigates the model more in depth


\section{A functional approach}
The restrictions (see list in concepts section), functional programming imposes, directly removes serious sources of bugs which leads to simulation which is more likely to be correct. These restrictions force us to solve the fundamental concepts in ABS implementation differently. Note that we could fall back to using IO throughout all the simulation in which case we have access to mutable references but then we lose important compile-time guarantees and introduce those serious sources of bugs we want to get rid of - also testing becomes more complicated and not as strong anymore because we cannot guarantee at compile time that no random IO stuff is happening within the agents. Also note that obviously no one would do random IO stuff in an agent (e.g. read from a file, open connection to server...) but one must not understimate the value of guaranteeing its absence at compile-time.

Due to the fundamentally different approaches of pure Functional Programming (pure FP) an ABS needs to be implemented fundamentally different as well compared to traditional object-oriented approaches (OO). We face the following challenges:

\begin{enumerate}
	\item How can we represent an Agent? \\
	In OO the obvious approach is to map an agent directly onto an object which encapsulates data and provides methods which implement the agents actions. Obviously we don't have objects in pure FP thus we need to find a different approach to represent the agents actions and to encapsulate its state.
	
	\item How can we represent state in an Agent? \\
	In the classic OO approach one represents the state of an Agent explicitly in mutable member variables of the object which implements the Agent. As already mentioned we don't have objects in pure FP and state is immutable which leaves us with the very tricky question how to represent state of an Agent which can be actually updated.
	
	\item How can we implement proactivity of an Agent? \\
	In the classic OO approach one would either expose the current time-delta in a mutable variable and implement time-dependent functions or ignore it at all and assume agents act on every step. At first this seems to be not a big deal in pure FP but when considering that it is yet unclear how to represent Agents and their state, which is directly related to time-dependent and reactive behaviour it raises the question how we can implement time-varying and reactive behaviour in a purely functional way.
	
	\item How can we implement the agent-agent interaction? \\
	In the classic OO approach Agents can directly invoke other Agents methods which makes direct Agent interaction \textit{very} easy. Again this is obviously not possible in pure FP as we don't have objects with methods and mutable state inside.
		
	\item How can we represent an environment and its various types? \\
	In the classic OO approach an environment is almost always a mutable object which can be easily made dynamic by implementing a method which changes its state and then calling it every step as well. In pure FP we struggle with this for the same reasons we face when deciding how to represent an Agent, its state and proactivity.
	
	\item How can we implement the agent-environment interaction? \\
	In the classic OO approach agents simply have access to the environment either through global mechanisms (e.g. Singleton or simply global variable) or passed as parameter to a method and call methods which change the environment. Again we don't have this in pure FP as we don't have objects and globally mutable state.
	
	\item How can we step the simulation? \\
	In the classic OO approach agents are run one after another (with being optionally shuffled before to uniformly distribute the ordering) which ensures mutual exclusive access in the agent-agent and agent-environment interactions. Obviously in pure FP we cannot iteratively mutate a global state.

	\item An Agent-Interface \\
	How can we interface with the agent: how does the agent reveal information about itself? We will see that functional programming supports very strong encapsulation of local state which is not accessible and mutable from outside. This makes testing in some respekt easier, in some harder. 
\end{enumerate}

\subsection{Agent representation, state and proactivity}
Whereas in imperative programming (the OO which we refer to in this paper is built on the imperative paradigm) the fundamental building block is the destructive assignment, in FP the building blocks are obviously functions which can be evaluated.
Thus we have no other choice than to represent our Agents using a function which implements their behaviour. This function must be time-aware somehow and allow us to react to time-changes and inputs. Fortunately there exists already an approach to time-aware, reactive programming which is termed Functional Reactive Programming (FRP). This paradigm has evolved over the year and current modern FRP is built around the concept of a signal-function which transforms an input-signal into an output-signal. An input-signal can be seen as a time-varying value. Signal-functions are implemented as continuations which allows to capture local state using closures. Modern FRP also provides feedback functions which provides convenient methods to capture and update local state from the previous time-step with an initial state provided at time = 0.

- pure functions don't have a notion of communication as opposed to method calls in object-oriented languages like java

- time is represented using the FRP concept: Signal-Functions which are sampled at (fixed) time-deltas, the dt is never visible directly but only reflected in the code and read-only.
- no method calls => continuous data-flow instead
	
Viewing agent-agent interaction as simple method calls implies the following:
- it takes no time
- it has a synchronous and transactional character
- an agent gives up control over its data / actions or at least there is always the danger that it exposes too much of its interface and implementation details. 
- agents equals objects, which is definitely NOT true. Agents 

data-flow
synchronous agent transactions

- still need transactions between two agents e.g. trading occurs over multiple steps (makeoffer, accept/refuse, finalize/abort) 
		-> exactly define what TX means in ABS
			-> exclusive between 2 agents
			-> state-changes which occur over multiple steps and are only visible to the other agents after the TX has commited
			-> no read/write access to this state is allowed to other agents while the TX is active
			-> a TX executes in a single time-step and can have an arbitrary number of tx-steps
		-> it is easily possible using method-calls in OOP but in our pure functional approach it is not possible
		-> parallel execution is being a problem here as TX between agents are very easy with sequential
		-> an agent must be able to transact with as many other agents as it wants to in the same time-step
		-> no time passes between transactions
		=> what we need is a 'all agents transact at the same time'
			-> basically we can implement it by running the SFs of the agents involved in the TX repeatedly with dt=0 until there are no more active TXs
			-> continuations (SFs) are perfectly suited for this as we can 'rollback' easily by using the SF before the TX has started
		
\subsection{Environment representation and interaction}

no global shared mutable environment, having different options:
- non-active read-only (SIR): no agent, as additional argument to each agent
- pro-active read-only (?): environment as agent, broadcast environment updates as data-flow
- non-active read/write (?): no agent, StateT in agents monad stack
- pro-active read/write (Sugarscape): environment as agent, StateT in agents monad stack

care must be taken in case of agent-transactions: when aborting/refusing all changes to the environment must be rolled back => instead of StateT use a transactional monad which allows us to revert changes to a save point at the start of the TX. if we drag the environment through all agents then we could easily revert changes but that then requires to hard-code the environment concept deep into the simulation scheduling/stepping which brings lots of inconveniences, also it would need us to fold the resulting multiple environments back into a single. If we had an environment-centric view then probably this is what we want but in ABS the focus is on the agents

question is if the TX sf runs in the same monad aw the agent or not. i opt for identity monad which prevents modification of the Environment in a transaction

also need to motivate the dt=0 in all TX processing: conceptually it all happens instantaneously (although arbitration is sequential) but agents must act time-sensitive

for environment we need transactional and shared state behaviour where we can have mutual exclusive access to shared data but also roll back changes we made. it should run deterministic when running agents not truly parallel. solution: run environment in a transactional state monad (TX monad). although the agents are executed in parallel in the end it (map) runs sequentially. this passes a mutable state through all agents which can act on it an roll back actions e.g. in case of a failed agent TX. if we dont need transactional behaviour then just use StateT monad. this ensures determinism. pro active environment is also easily possible by writing to the state. this approach behaves like sequential transactional although the agents run in parallel but how is this possible when using mapMSF ?

\subsection{Stepping the simulation}

- parallel update only, sequential is deliberately abandoned due to:
		-> reality does not behave this way
		-> if we need transactional behaviour, can use STM which is more explicit
		-> it is translates directly to a map which is very easy to reason about (sequential is basically a fold which is much more difficult to reason about)
		-> is more natural in functional programming
		-> it exists for 'transactional' reasons where we need mutual exclusive access to environment / other agents
			-> we provide a more explicit mechanism for this: Agent Transactions
			

% Place the main text here. Please use only \section, \subsection, and \subsubsection sectioning commands to structure your text. Do NOT use lower sectioning commands, including \paragraph and \subparagraph

%For bulleted, numbered and description lists the class provides three asterisked environments to replace the standard LaTeX ones: itemize*, enumerate*, and description*. Please use these ones as the standard environment may cause issues with the paragraph numbering system.

%\begin{itemize*}
%     \item 
%     \item
%     \item
% \end{itemize*} 

% hyperlinks (to models, videos, etc.) can be included via the \href command (remember to put \usepackage{hyperref} in the preamble). Check the hyperref documentation for details

%%%%%%%%%%%%%%%%%%%%%%%%%%%%%%%%%%%%%%%%%%%%%%

% FIGURES AND TABLES

% Figures should be placed in the desired position within the text. Please follow the template below.
% Figure widths can be set using absolute dimensions (e.g., [width = 12cm]) or relative ones (e.g., [width = 0.8/textwidth]). We strongly suggest to use the latter option, as this allows automatic adaptation to different paper widths

%\begin{figure}[!t]
%\centering
%\includegraphics[width=????\textwidth]{????}
%\caption{}
%\label{fig:????}
%\end{figure}

% Tables should be placed in the desired position within the text. Please follow the template below

%\begin{table}[!t]
%	\centering
%	\begin{tabular}{????}
%	\toprule
% 	% first line
%	\midrule
%	% tale body	
%	\bottomrule			
%	\end{tabular}
%	\caption{}
%	\label{tab:????}	
%\end{table}

%%%%%%%%%%%%%%%%%%%%%%%%%%%%%%%%%%%%%%%%%%%%%%
% End of  paragraph numbering. Please leave this untouched
\endparano

%%%%%%%%%%%%%%%%%%%%%%%%%%%%%%%%%%%%%%%%%%%%%%

%%%%%%%%%%%%%%%%%%%%%%%%%%%%%%%%%%%%%%%%%%%%%%

% APPENDICES
% Put your appendices here. Please use the normal sectioning command, e.g.,
% \section{Appendix A: <title of the appendix>}
% \section{Appendix B: <title of the appendix>}
% ...

%%%%%%%%%%%%%%%%%%%%%%%%%%%%%%%%%%%%%%%%%%%%%%

% ENDNOTES. Please uncomment the line below in case of notes.
% \theendnotes

%%%%%%%%%%%%%%%%%%%%%%%%%%%%%%%%%%%%%%%%%%%%%%

% REFERENCES.
% The JASSS bibliographic style file (jasss.bst) is included in the bundle. Please use BibTeX, not BibLaTeX.
% Use natbib commands for references (\citep{}, \citet{}, etc.), not standard LaTeX ones (\cite{}).
% Remember to include the doi and url fields in your bib database. The address field should be included for books.
% Please upload the bib file (not just the bbl one) when submitting.
 
\bibliographystyle{jasss}
%\bibliography{} % Please set the right name for your bib file
\bibliography{../../../references/phdReferences.bib}

%%%%%%%%%%%%%%%%%%%%%%%%%%%%%%%%%%%%%%%%%%%%%%

\end{document}