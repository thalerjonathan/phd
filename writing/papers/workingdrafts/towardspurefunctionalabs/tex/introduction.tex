\section{Introduction}
The traditional approach to Agent-Based Simulation (ABS) has so far always been object-oriented techniques, due to the influence of the seminal work of Epstein et al \cite{epstein_growing_1996} in which the authors claim "[..] object-oriented programming to be a particularly natural development environment for Sugarscape specifically and artificial societies generally [..]" (p. 179). This work established the metaphor in the ABS community, that \textit{agents map naturally to objects} \cite{north_managing_2007} which still holds up today.

In this paper we challenge this metaphor and explore ways of approaching ABS using the functional programming paradigm with the language Haskell. By doing this we expect to leverage the benefits of it % \cite{hudak_history_2007}: %TODO: this is already by far too technical here, we are writing for people who have basically no understanding of FP. higher expressivity through declarative code, being polymorph and explicit about side-effects through monads, more robust and less susceptible to bugs due to explicit data flow and lack of implicit side-effects.

As use-case we employ the famous SugarScape model \cite{epstein_growing_1996} because it can be seen as one of the most famous models in ABS which also laid the foundations of using object-oriented programming.

TODO: shortly describe what we are doing e.g. we introduce pure functional programming concepts and more advanced stuff for implementing ABS

The aim of this paper is to show \textit{how} to implement ABS in functional programming as in Haskell and \textit{why} it is of benefit of doing so. By doing this we give the reader a good understanding of what functional programming is, what the challenges are in applying it to ABS and how we solve these in our approach. 

The paper makes the following contributions:

\begin{itemize}
	\item first to systematically introduce functional programming concepts to agent-based simulation
	\item applied property-based testing to ABS
	\item applied STM to ABS
	\item effect-polymor implementation: allows to run concurrently or not
\end{itemize}