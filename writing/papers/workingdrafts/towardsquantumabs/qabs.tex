% Formatted for ICFP 2018: ACM Small template
\documentclass[format=acmsmall, review=false, screen=true]{acmart}

\usepackage{minted}

% TODO: fill in the right metainfo
% Metadata Information
\acmJournal{PACMPL}
\acmVolume{9}
\acmNumber{4}
\acmArticle{39}
\acmYear{2018}
\acmMonth{9}
\copyrightyear{2018}
%\acmArticleSeq{9}

% TODO : select the right one
% Copyright
%\setcopyright{acmcopyright}	% = copyright transfer to ACM
%\setcopyright{acmlicensed} 		% = retaining copyright but granting ACM exclusive publication rights
\setcopyright{rightsretained}  % = open access on payment of a fee
%\setcopyright{usgov}
%\setcopyright{usgovmixed}
%\setcopyright{cagov}
%\setcopyright{cagovmixed}

% TODO : get the data
% DOI
\acmDOI{0000001.0000001}

% TODO: fill in
% Paper history
\received{March 2018}
%\received[revised]{March 2018}
%\received[accepted]{March 2018}

% Document starts
\begin{document}

% set to true to compile haskell code formated with minted package
% requires pygmentize!
\providecommand\haskellMinted{true}
\ifthenelse{\equal{\haskellMinted}{true} }{
\newminted[QCode]{haskell}{fontsize=\footnotesize}
}{
\newenvironment{QCode}
{\begin{comment}} % seems not to work, don't know why the f***
{\end{comment}}
}

% Title portion. Note the short title for running heads
\title[Towards Quantum Computing in Agent-Based Simulation]{Towards Quantum Computing in Agent-Based Simulation}
\subtitle{A Practical Approach}

\author{Jonathan Thaler}
\orcid{TODO}
\email{jonathan.thaler@nottingham.ac.uk}
\author{Thorsten Altenkirch}
\email{thorsten.altenkirch@nottingham.ac.uk}
\author{Peer-Olaf Siebers}
\affiliation{%
  \institution{University of Nottingham}
  \streetaddress{7301 Wollaton Rd}
  \city{Nottingham}
  \postcode{NG8 1BB}
  \country{United Kingdom}}
\email{peer-olaf.siebers@nottingham.ac.uk}

\begin{abstract}
TODO: fill in the right metainfo
TODO: select the right copyright
TODO: get DOI
TODO: fill in paper history
TODO: generate CCSXML

Agent-Based Simulation (ABS) is a methodology in which a system is simulated in a bottom-up approach by modelling the micro interactions of its constituting parts, called agents, out of which the global macro system behaviour emerges. 
\end{abstract}

%
% The code below should be generated by the tool at
% http://dl.acm.org/ccs.cfm
% Please copy and paste the code instead of the example below.
%
% TODO needs to be generated
%\begin{CCSXML}
%<ccs2012>
% <concept>
%  <concept_id>10010520.10010553.10010562</concept_id>
%  <concept_desc>Computer systems organization~Embedded systems</concept_desc>
%  <concept_significance>500</concept_significance>
% </concept>
% <concept>
%  <concept_id>10010520.10010575.10010755</concept_id>
%  <concept_desc>Computer systems organization~Redundancy</concept_desc>
%  <concept_significance>300</concept_significance>
% </concept>
% <concept>
%  <concept_id>10010520.10010553.10010554</concept_id>
%  <concept_desc>Computer systems organization~Robotics</concept_desc>
%  <concept_significance>100</concept_significance>
% </concept>
% <concept>
%  <concept_id>10003033.10003083.10003095</concept_id>
%  <concept_desc>Networks~Network reliability</concept_desc>
%  <concept_significance>100</concept_significance>
% </concept>
%</ccs2012>
%\end{CCSXML}
%
%\ccsdesc[500]{Computer systems organization~Embedded systems}
%\ccsdesc[300]{Computer systems organization~Redundancy}
%\ccsdesc{Computer systems organization~Robotics}
%\ccsdesc[100]{Networks~Network reliability}

%
% End generated code
%

\keywords{Agent-Based Simulation, Quantum Computing, Functional Programming}

\maketitle

%*******************************************************************************
%*********************************** First Chapter *****************************
%*******************************************************************************

\chapter{Introduction}  %Title of the First Chapter
I noticed that it is pretty hard to convince an agent-based economics specialist who is not a computer scientist about a pure functional approach. My conjecture is that the implementation technique and method does not matter much to them because they have very little knowledge about programming and are almost always self-taught - they don't know about software-engineering, nothing about proper software-design and architecture, nothing about software-maintenance, nothing about unit-testing,... In the end they just "hack" the simulation in whatever language they are able to: C++, Visual Basic, Java or toolboxes like Netlogo. For them it is all about to \textit{get things done somehow} and not to get things done the right way or in a beautiful way - the way and the method doesn't matter, its just a necessary evil which needs to be done. Thus if functional programming could make their lives easier, then they will definitely welcome it. But functional programming is, i think, harder to learn and harder to understand - so one needs to provide an abstraction through EDSL. So I REALLY need to come up with convincing arguments why to use pure functional approaches in ACE THEY can understand, otherwise I will be lost and not heard (not published,...). \\

What ACE economists care for:

\begin{itemize}
\item Very: Qualitative modelling with quantitative results
\item Yes: Easy reproducibility
\item Likely: Reasoning about convergence?
\item Likely: EDSL
\end{itemize}

My contributions are: pure functional framework, functional agent-model for market-simulations, EDSL for market-simulations, qualitative / implicit modelling with quanitative results, reasoning in my framework about convergence \\

IDEA: could I develop non-causal modelling (models are expressed in terms of non-directed equations, modelled in signal-relations) to allow for qualitative modelling for the agent-based economists? See hybrid modelling paper of Yampa. \textbf{THIS WOULD BE A HUGE NOVEL CONTRIBUTION TO ACE ESPECIALLY WHEN COMBINED WITH AN EDSL AND PROVIDING FULL REFERENTIAL TRANSPARENCY TO KEEP THE ABILITY TO REASON ABOUT CONVERGENCE}. This should be covered in the "EDSL"-paper.

TODO: maybe i should really focus only on market models? otherwise too much? \\

central novelty of my PhD: model specification = runnable code. possible through EDSL. but only in specific subfield of ACE: market-models. need a functional description of the model, then translate it to model specification in EDSL and then run it to see dynamics. But: model specification moves closer to functional programming languages. \\

another novelty approach: model specification through qualitative instead of quantiative approaches. is this possible? \\

WHY FUNCTIONAL? "because its the ultimate approach to scientific computing": fewer bugs due to mutable state (why? is thos shown obkectively by someone?), shorter (again as above, productivity), more expressive and closer to math, EDSL, EDSL=model=simulation, better parallelising due to referental transparency, reasoning \\

scientific results need to be reproduced, especially when they have high impact. a more formal approach of specifying the model and the simulation (model=simulation) could lead to easier sharing and easier reporduction without ambigouites \\

pure functional agent-model \& theory, EDSL framework in Haskell for ACE

\begin{enumerate}
\item Which kind of problem do we have?
\item What aim is there? Solving the problem? 
\item How the aim is achieved by enumerating VERY CLEAR objectives.
\item What the impact one expects (hypothesis) and what it is (after results).
\end{enumerate}

Note: It is not in the interest of the researcher to develop new economic theories but to research the use of functional methods (programming and specification) in agent-based computational economics (ACE).

NOTE: Get the reader’s attention early in the introduction: motivation, significance, originality and novelty.

\section{Methods}
Methods need to be selected to implement the simulations. Special emphasis will be put on functional ones which will then be compared to established methods in the field of ABM/S and ACE. \\

Claim: non-programming environments are considered to be not powerful enough to capture the complexity of ACE implementations thus a programming approach to ACE will be always required.

\section{Scenarios}
To apply and test functional methods in ACE, four scenarios of ACE are selected and then the methods applied and compared with each other to see how each of them perform in comparison. The 4 selected scenarios represent a selection of the challenges posed in ACE: from very abstract ones to very operational ones.

\section{Comparison}
Each of the selected scenarios is then implemented using the selected methods where each solution is then compared against the following criteria: 

\begin{enumerate}
\item suitability for scientific computation
\item robustness
\item error-sources
\item testability
\item stability
\item extendability
\item size of code
\item maintainability
\item time taken for development
\item verification \& correctness
\item replications \& parallelism
\item EDSL
\end{enumerate}

This will then allow to compare the different methods against each other and to show under which circumstances functional methods shine and when they should not be used.

\section{Agent-Based Modelling and Simulation (ABM/S)}
ABM/S is a method of modelling and simulating a system where the global behaviour may be unknown but the behaviour and interactions of the parts making up the system is of knowledge (Wooldrige, M. (2009). An Introduction to MultiAgent Systems. John Wiley & Sons). Those parts, called agents, are modelled and simulated out of which then the aggregate global behaviour of the whole system emerges. Thus the central aspect of ABM/S is the concept of an Agent which can be understood as a metaphor for a pro-active unit, able to spawn new Agents, and interacting with other Agents in a network of neighbours by exchange of messages. The implementation of Agents can vary and strongly depends on the programming language and the kind of domain the simulation and model is situated in.

\section{Agent-Based Economics (ACE)}
According to Leigh Tesfatsion (Tesfatsion, L. (2006). Agent-based computational economics: A constructive approach to economic theory. In Tesfatsion, L. and Judd, K. L., editors, Handbook of Computational Economics, volume 2, chapter 16, pages 831–880. Elsevier, 1 edition.), one of the leading figures, ACE is "[...] computational modelling of economic processes (including whole economies) as open-ended dynamic systems of interacting agents." - thus lending perfectly to the use of ABM/S as already the name suggests. Whereas classical economic models fall short by only looking at the average, pure rational, individual interacting in anonymous markets, the ACE approach looks at heterogeneous, non-rational individuals interacting with each other in networks (Kirman, A. (2010). Complex Economics: Individual and Collective Rationality. Routledge, London ; New York, NY.). Thus ACE can be understood as a combination of computer-science, cognitive/social science and evolutionary economics.

\section{Functional programming}
TODO: read \cite{Backus1978}

The state-of-the-art approach to implementing Agents are object-oriented methods and programming as the metaphor of an Agent as presented above lends itself very naturally to object-orientation (OO). The author of this thesis claims that OO in the hands of inexperienced or ignorant programmers is dangerous, leading to bugs and hardly maintainable and extensible code. The reason for this is that OO provides very powerful techniques of organising and structuring programs through Classes, Type Hierarchies and Objects, which, when misused, lead to the above mentioned problems. Also major problems, which experts face as well as beginners are 1. state is highly scattered across the program which disguises the flow of data in complex simulations and 2. objects don’t compose as well as functions. The reason for this is that objects always carry around some internal state which makes it obviously much more complicated as complex dependencies can be introduced according to the internal state.
All this is tackled by (pure) functional programming which abandons the concept of global state, Objects and Classes and makes data-flow explicit. This then allows to reason about correctness, termination and other properties of the program e.g. if a given function exhibits side-effects or not. Other benefits are fewer lines of code, easier maintainability and ultimately fewer bugs thus making functional programming the ideal choice for scientific computing and simulation and thus also for ACE. A very powerful feature of functional programming is Lazy evaluation. It allows to describe infinite data-structures and functions producing an infinite stream of output but which are only computed as currently needed. Thus the decision of how many is decoupled from how to (Hughes, J. (1989). Why functional programming matters. Comput. J., 32(2):98–107.).
The most powerful aspect using pure functional programming however is that it allows the design of embedded domain specific languages (EDSL). In this case one develops and programs primitives e.g. types and functions in a host language (embed) in a way that they can be combined. The combination of these primitives then looks like a language specific to a given domain, in the case of this thesis ACE. The ease of development of EDSLs in pure functional programming is also a proof of the superior extensibility and composability of pure functional languages over OO (Henderson P. (1982). Functional Geometry. Proceedings of the 1982 ACM Symposium on LISP and Functional Programming.).
One of the most compelling example to utilize pure functional programming is the reporting of Hudak (Hudak P., Jones M. (1994). Haskell vs. Ada vs. C++ vs. Awk vs. ... An Experiment in Software Prototyping Productivity. Department of Computer Science, Yale University.)  where in a prototyping contest of DARPA the Haskell prototype was by far the shortest with 85 lines of code. Also the Jury mistook the code as specification because the prototype did actually implement a small EDSL which is a perfect proof how close EDSL can get to and look like a specification.

Functional languages can best be characterized by their way computation works: instead of \textit{how} something is computed, \textit{what} is computed is described. Thus functional programming follows a declarative instead of an imperative style of programming. The key points are:
\begin{itemize}
\item No assignment statements - variables values can never change once given a value.
\item Function calls have no side-effect and will only compute the results - this makes order of execution irrelevant, as due to the lack of side-effects the logical point in \textit{time} when the function is calculated within the program-execution does not matter.
\item higher-order functions
\item lazy evaluation
\item Looping is achieved using recursion, mostly through the use of the general fold or the more specific map.
\item Pattern-matching
\end{itemize}

This alone does not really explain the \textit{real} advantages of functional programming and one must look for better motivations using functional programming languages. One motivation is given in \cite{Hughes1989} which is a great paper explaining to non-functional programmers what the significance of functional programming is and helping functional programmers putting functional languages to maximum use by showing the real power and advantages of functional languages. The main conclusion is that \textit{modularity}, which is the key to successful programming, can be achieved best using higher-order functions and lazy evaluation provided in functional languages like Haskell. \cite{Hughes1989} argues that the ability to divide problems into sub-problems depends on the ability to glue the sub-problems together which depends strongly on the programming-language and \cite{Hughes1989} argues that in this ability functional languages are superior to structured programming.

TODO: comparison of functional and object-oriented programming. My points are:
\begin{itemize}
\item The way state can be changed and treated - distributed over multiple objects - is often very difficult to understand.
\item Inheritance is a dangerous thing if not used with care because inheritance introduces very strong dependencies which cannot be changed during runtime anymore.
\item Objects don't compose very well: \url{http://zeroturnaround.com/rebellabs/why-the-debate-on-object-oriented-vs-functional-programming-is-all-about-composition/}
\item (Nearly) impossible to reason about programs
\end{itemize}

In conclusion the upsides of functional programming as opposed to OO are:
\begin{itemize}
\item Much more explicit flow of data \& control
\item Much better compose-able
\item Much better parallelism
\end{itemize}

\section{Related Research}
Tim Sweeney, CTO of Epic Games gave an invited talk about how "future programming languages could help us write better code" by "supplying stronger typing, reduce run-time failures;  and the need for pervasive concurrency support, both implicit and explicit, to effectively exploit the several forms of parallelism present in games and graphics." \cite{Sweeney2006}. Although the fields of games and agent-based simulations seem to be very different in the end, they have also very important similarities: both are simulations which perform numerical computations and update objects - in games they are called "game-objects" and in abm they are called agents but they are in fact the same thing - in a loop either concurrently or sequential. His key-points were:

\begin{itemize}
\item Dependent types as the remedy of most of the run-time failures.
\item Parallelism for numerical computation: these are pure functional algorithms, operate locally on mutable state. Haskell ST, STRef solution enables encapsulating local heaps and mutability within referentially transparent code.
\item Updating game-objects (agents) concurrently using STM: update all objects concurrently in arbitrary order, with each update wrapped in atomic block - depends on collisions if performance goes up.
\end{itemize}

\section{Related Research}

\cite{schneider_towards_2012} and \cite{vendrov_frabjous:_2014} present a domain-specific language for developing functional reactive agent-based simulations. This language called FRABJOUS is very human readable and easily understandable by domain-experts. It is not directly implemented in FRP/Haskell/Yampa but is compiled to Haskell/Yampa code which they claim is also readable. This is the direction we want to head but we don't want this intermediate step but look for how a most simple domain-specific language embedded in Haskell would look like. In this paper we explicitly dive deep into FRP And Yampa and see how we can combine the best of both.

\section{Background}

\subsection{Schelling Segregation}
We follow in our implementation the original paper of Schelling as in \cite{schelling_dynamic_1971} where we focus on the \textit{Area Distribution} section (Schelling starts with movement in a linear, 1-dimensional world where agents are able to move to the nearest point which meets the agents satisfaction but this is not what we follow here). One assumes a discrete 2-dimensional lattice-world with NxM fields. Each field is either occupied by an agent of a given color (e.g. Red or Green) or is free. Each field has 8 neighbours, which denotes a Moore-Neighbourhood. In Schellings original work the lattice-world is limited at its borders but we assume a torus world which is wrapped around in both the x- and y-dimensions resulting in 8 neighbours also for fields at the border. The occupation density was set by Schelling to be about 70\%-75\% which he identifies as being a setting which allows the agents to move around freely without making the lattice-world too sparse.
Now the agents make their move sequentially one after another. In each move an agent calculates the number of neighbours which are of equal color. If the number satisfies the agents needs about the neighbourhood then the agent is regarded as being 'happy' and will stay on this field. On the other hand the agent moves to the nearest unoccupied field which satisfies its needs. An agent which moves selects an unoccupied place randomly relative from its current place within a rectangle of side-length 2r where its current place is at the center. The interpretation for that behaviour is that agents won't move too far as it could be costly. Also children might attend a school in this area or the family has friends in this area, so they don't want to break that.



Agents just move depending on their movement-strategy to another place if they are not happy on the current one - they don't care how the target place is in the present or in the future, they will decide again in the next time-step. The interpretation for that behaviour is: agents want to 'just get out' at any cost, not caring what the future place will look like - it might be better or worse but they will see then.

\subsubsection{Optimizing behaviour}
TODO: define utility

The original schelling model didn't have a move-optimizing behaviour, meaning agents are just binary: if it is happy it will not move, if it is unhappy it will move but they won't care where they move. We introduce local move-optimizing behaviours which can be interpreted as being realistic in the real-world. It is important to note that we focus on \textit{local} instead of \textit{global} move-optimization: the agents are limited in their reasoning-capabilities and have limited information available: they cannot check out \textit{every} place and pick the globally best one.\\

\subsubsection{Anticipating behaviour}
Schelling explicitly mentions in \cite{schelling_dynamic_1971} that nobody anticipates moves of others. This is what we introduce using the recursive simulation.

TODO: is this optimizing behaviour in the spirit of schellings original work? 

\paragraph{Optimizing future} Agents pick an unoccupied random place and move to it if it increases their utility in the future. The interpretation for that behaviour is: agents heard about a place which will be cool in the future.

\paragraph{Optimizing present \& future} Agents pick an unoccupied random place and move to it if it increases their utility in the now and in the future. The interpretation for that behaviour is: agents heard about a cool spot in town, check it out and move to it if they like it but they also anticipate the coolness of the place in the future and if it seems that the place is going down then they won't move there.

\subsection{Related Research}
TODO: \cite{kirman_complex_2010} mention kirman complex economics where he investigates the model more in depth


\section{A quantum computing approach to ABS}
\label{sec:quantum_approach}



\section{Conclusions}
\label{sec:conclusions}

Our approach is radically different from traditional approaches in the ABS community. First it builds on the already quite powerful FRP paradigm. Second, due to our continuous time approach, it forces one to think properly of time-semantics of the model and how small $\Delta t$ should be. Third it requires to think about agent interactions in a new way instead of being just method-calls.

Because no part of the simulation runs in the IO Monad and we do not use unsafePerformIO we can rule out a serious class of bugs caused by implicit data-dependencies and side-effects which can occur in traditional imperative implementations.

Also we can statically guarantee the reproducibility of the simulation, which means that repeated runs with the same initial conditions are guaranteed to result in the same dynamics. Although we allow side-effects within agents, we restrict them to only the Random and State Monad in a controlled, deterministic way and never use the IO Monad which guarantees the absence of non-deterministic side effects within the agents and other parts of the simulation.

Determinism is also ensured by fixing the $\Delta t$ and not making it dependent on the performance of e.g. a rendering-loop or other system-dependent sources of non-determinism as described by \cite{perez_testing_2017}. Also by using FRP we gain all the benefits from it and can use research on testing, debugging and exploring FRP systems \cite{perez_testing_2017, perez_back_2017}.

\subsection*{Issues}
Currently, the performance of the system is not comparable to imperative implementations but our research was not focusing on this aspect. We leave the investigation and optimization of the performance aspect of our approach for further research.

Despite the strengths and benefits we get by leveraging on FRP, there are errors that are not raised at compile time, e.g. we can still have infinite loops and run-time errors. This was for example investigated in \cite{sculthorpe_safe_2009} where the authors use dependent types to avoid some run-time errors in FRP. We suggest that one could go further and develop a domain specific type system for FRP that makes the FRP based ABS more predictable and that would support further mathematical analysis of its properties. Furthermore, moving to dependent types would pose a unique benefit over the traditional object-oriented approach and should allow us to express and guarantee even more properties at compile time. We leave this for further research.

In our pure functional approach, agent identity is not as clear as in traditional object-oriented programming, where an agent can be hidden behind a polymorphic interface which is much more abstract than in our approach. Also the identity of an agent is much clearer in object-oriented programming due to the concept of object-identity and the encapsulation of data and methods.

We can conclude that the main difficulty of a pure functional approach evolves around the communication and interaction between agents, which is a direct consequence of the issue with agent identity. Agent interaction is straight-forward in object-oriented programming, where it is achieved using method-calls mutating the internal state of the agent, but that comes at the cost of a new class of bugs due to implicit data flow. In pure functional programming these data flows are explicit but our current approach of feeding back the states of all agents as inputs is not very general and we have added further mechanisms of agent interaction which we had to omit due to lack of space.

\section{Further Research}
\label{sec:further_research}

We see this paper as an intermediary and necessary step towards dependent types for which we first needed to understand the potentials and limitations of a non-dependently typed pure functional approach in Haskell. Dependent types are extremely promising in functional programming as they allow us to express stronger guarantees about the correctness of programs and go as far as allowing to formulate programs and types as constructive proofs \cite{wadler_propositions_2015} which must be total by definition \cite{thompson_type_1991}, \cite{altenkirch_why_2005}, \cite{altenkirch_pi_2010}, \cite{program_homotopy_2013}. So far no research using dependent types in agent-based simulation exists at all and it is not clear whether dependent types make sense in this context. In our next paper we want to explore this for the first time and ask more specifically how we can add dependent types to our pure functional approach, which conceptual implications this has for ABS and what we gain from doing so. We plan on using Idris \cite{brady_idris_2013}, \cite{brady_type-driven_2017} as the language of choice as it is very close to Haskell with focus on real-world application and running programs as opposed to other languages with dependent types e.g. Agda and Coq which serve primarily as proof assistants.
It would be of immense interest whether we could apply dependent types to the model meta-level or not - this boils down to the question if we can encode our model specification in a dependent type way. This would allow the ABS community for the first time to reason about a proper formalisation of a model. We plan to implement a total and terminating implementation of our approach which would be a formal proof-by-construction that the agent-based approach of the SIR model terminates after a finite number of steps.

\begin{acks}
The authors would like to thank TODO for constructive comments and valuable discussions.
\end{acks}

% Bibliography
\bibliographystyle{../../templates/acmart/ACM-Reference-Format}
\bibliography{../../../references/phdReferences.bib}

%\begin{appendix}
%\section{Emulating System Dynamics}
The introduction of data-flows in section \ref{sec:step3_dataflow} allows us to emulate the system dynamics (SD) approach because we can now express a system with parallel continuous-time flows between the stocks and flows. Each stock $S(t)$, $I(t)$, $R(t)$ and each flow $infectionRate$, $recoveryRate$ is implemented as an agent with a fixed agent id. The connections between them are implemented using the previously introduced data-flow mechanism. We start by refining the types for our SIR implementation:

\begin{minted}[fontsize=\footnotesize]{haskell}
type SDMsg      = Double
type SDAgentIn  = AgentIn SDMsg
type SDObs      = Maybe Double
type SDEntity   = Agent SDObs SDMsg
type SDEntityId = AgentId

totalPopulation :: Double
totalPopulation = 1000

infectedCount :: Double
infectedCount = 1
\end{minted}

The message-data is now a plain Double and the observable data has been changed to a \textit{Maybe} Double: instead of discrete agent-states we are dealing now with stocks and flows which are aggregates represented by continuous values. Note that we use a Maybe type as flows only connect stocks and transform their values but don't have any observable state themselves. Note also that the population size and number of infected is specified now as Double as we are dealing with continuous aggregates.

We give hard-coded agent ids to our stocks and flows. This allows then for setting up hard-coded connections between them at compile time.
\begin{minted}[fontsize=\footnotesize]{haskell}
susceptibleStockId :: SDEntityId
susceptibleStockId = 0

infectiousStockId :: SDEntityId
infectiousStockId = 1

recoveredStockId :: SDEntityId
recoveredStockId = 2

infectionRateFlowId :: SDEntityId
infectionRateFlowId = 3

recoveryRateFlowId :: SDEntityId
recoveryRateFlowId = 4
\end{minted}

Next we give the implementation of the infectious stock (the implementations of the susceptible and recovered stock work in a similar way and are left as an easy exercise to the reader):

\begin{minted}[fontsize=\footnotesize]{haskell}
infectiousStock :: Double -> SDEntity
infectiousStock initValue = proc ain -> do
  let infectionRate = flowInFrom infectionRateFlowId ain
      recoveryRate  = flowInFrom recoveryRateFlowId ain

  stockValue <- (initValue+) ^<< integral -< (infectionRate - recoveryRate)
  
  let ao   = agentOut (Just stockValue)
      ao'  = dataFlow (infectionRateFlowId, stockValue) ao
      ao'' = dataFlow (recoveryRateFlowId, stockValue) ao'
      
  returnA -< ao''
\end{minted}

The stock receives flows from both the infection-rate and recovery-rate flow using the function \textit{flowInFrom} (see below). Then the current stock value is calculated using the \textit{integral} function of Yampa with an initial value added which are the initially infected people. The integral primitive of Yampa integrates the fed in data over time using the rectangle rule which means it simply multiplies the input values by the current $\Delta t$ and accumulates them. Note that we can directly express the SD equation using Yampas DSL for continuous-time systems. The current stock value is then set as the observable value of the stock and sent to the infection- and recovery-rate flows. For convenience we implemented an additional function \textit{flowInFrom} which returns the first value sent from the corresponding agent id or 0.0 if none was sent.

\begin{minted}[fontsize=\footnotesize]{haskell}
flowInFrom :: SDEntityId -> SDAgentIn -> Double
flowInFrom senderId ain = firstValue dsFiltered
  where 
    dsFiltered = filter ((==senderId) . fst) (aiData ain)

    firstValue :: [AgentData SDMsg] -> Double
    firstValue [] = 0.0
    firstValue ((_, v) : _) = v
\end{minted}
	
The \textit{infectionRate} flow is implemented as follows (the implementations of the recovery-rate flow works in a similar way and is left as an easy exercise to the reader):

\begin{minted}[fontsize=\footnotesize]{haskell}
infectionRateFlow :: SDEntity
infectionRateFlow = proc ain -> do
  let susceptible = flowInFrom susceptibleStockId ain 
      infectious  = flowInFrom infectiousStockId ain

      flowValue   = (infectious * contactRate * susceptible * infectivity) / totalPopulation
  
      ao          = agentOut Nothing
      ao'         = dataFlow (susceptibleStockId, flowValue) ao
      ao''        = dataFlow (infectiousStockId, flowValue) ao'
      
  returnA -< ao''
\end{minted}

Instead of integrating a value over time a stock just transforms incoming values from the connected stocks - in this case the susceptible and infectious stocks. Note again how directly we can express the formula for the infection rate.

When running the simulation one must make sure to use a small enough $\Delta t$ as \textit{integral} of Yampa is implemented using the rectangle rule which leads to considerable numerical errors with large $\Delta t$. Figure \ref{fig:sir_sd_dynamics} was created with this SD emulation for which we used $\Delta t = 0.01$.
%
%\subsection{Step VI: Adding agent transactions}
Imagine two agents A and B want to engage in a bartering process where agent A, is the seller who wants to sell an asset to agent B who is the buyer. Agent A sends Agent B a sell offer depending on how much agent A values this asset. Agent B receives this sell offer, checks if the price satisfies its utility, if it has enough wealth to buy the asset and replies with either a refusal or its own price offer. Agent A then considers agent Bs offer and if it is happy it replies to agent B with an acceptance of the offer, removes the asset from its inventory and increases its wealth. Agent B receives this acceptance offer, puts the asset in its inventory and decreases its wealth (note that this process could involve a potentially arbitrary number of steps without loss of generality).
We can see this behaviour as a kind of multi-step transactional behaviour because agents have to respect their budget constraints which means that they cannot spend more wealth or assets than they have. This implies that they have to 'lock' the asset and the amount of cash they are bartering about during the bartering process. If both come to an agreement they will swap the asset and the cash and if they refuse their offers they have to 'unlock' them.
In classic OO implementations it is quite easy to implement this as normally only one agent is active at a time due to sequential (discrete event scheduling approach) scheduling of the simulation. This allows then agent A which is active, to directly interact with agent B through method calls. The sequential updating ensures that no other agent will touch the asset or cash and the direct method calls ensure a synchronous updating of the mutable state of both objects with no time passing between these updates.

\subsubsection{Implementation}
We start with the implementation of step 4 with the Random Monad and remove the data-flows from AgentIn and AgentOut. We then add a field in AgentOut which allows the agent to indicate that it wants to start a transaction with another agent with an initial data-package. Also we add a field in AgentIn which indicates an incoming transaction request from another agent with the given data-package. In addition we need another field in AgentOut which allows the agent to indicate that it accepts the incoming request:

\begin{minted}[fontsize=\footnotesize]{haskell}
data AgentIn d = AgentIn
  {
    aiId        :: !AgentId
  , aiRequestTx :: !(Event (AgentData d))
  } deriving (Show)

data AgentOut m o d = AgentOut
  {
    aoObservable :: !o
  , aoRequestTx  :: !(Event (AgentData d, AgentTX m o d))
  , aoAcceptTx   :: !(Event (d, AgentTX m o d))
  }
\end{minted}

We run the transactions in the specialised agent-transaction signal-functions \textit{AgentTX} with different input and output types. This allows us to restrict the possible actions of an agent within a transaction:

\begin{minted}[fontsize=\footnotesize]{haskell}
type AgentTX m o d = SF m (AgentTXIn d) (AgentTXOut m o d)
\end{minted}

The input \textit{AgentTXIn} to an agent-transaction holds optional data and flags which indicate that the other agent has either committed or aborted the transaction.

\begin{minted}[fontsize=\footnotesize]{haskell}
data AgentTXIn d = AgentTXIn
  { aiTxData   :: Maybe d
  , aiTxCommit :: Bool
  , aiTxAbort  :: Bool
  }
\end{minted}

The output \textit{AgentTXOut} of an agent-transaction hold optional data a flag to abort the transaction and optional commit data which is Just in case the agent wants to commit. When committing the agent has to provide a potentially changed AgentOut and optionally a new agent behaviour signal-function. If the agent provides a signal-function when committing, the behaviour of the agent after the transaction will be this signal-function. If no signal-function is provided then the original one will be used.

\begin{minted}[fontsize=\footnotesize]{haskell}
data AgentTXOut m o d = AgentTXOut
  { aoTxData   :: Maybe d
  , aoTxCommit :: Maybe (AgentOut m o d, Maybe (Agent m o d))
  , aoTxAbort  :: Bool
  }
\end{minted}

We also provide type aliases for our SIR implementation:
\begin{minted}[fontsize=\footnotesize]{haskell}
type SIRMonad g    = Rand g
data SIRMsg        = Contact SIRState deriving (Show, Eq)
type SIRAgentIn    = AgentIn SIRMsg
type SIRAgentOut g = AgentOut (SIRMonad g) SIRState SIRMsg
type SIRAgent g    = Agent (SIRMonad g) SIRState SIRMsg
type SIRAgentTX g  = AgentTX (SIRMonad g) SIRState SIRMsg
\end{minted}

Stepping the simulation is now slightly more complex as in every step we need to run the transactions. Fortunately it is easy to provide customised implementations of MSFs in dunai, which is a bit more tricky in Yampa and requires to expose internals.

\begin{minted}[fontsize=\footnotesize]{haskell}
stepSimulation :: RandomGen g
               => [SIRAgent g]
               -> [SIRAgentIn] 
               -> SF (SIRMonad g) () [SIRAgentOut g]
stepSimulation sfs ains = MSF TODO DOLLAR \_ -> do
  res <- mapM (\ (ai, sf) -> unMSF sf ai) (zip ains sfs)
  let aos  = fmap fst res
      sfs' = fmap snd res

      ais  = map aiId ains
      aios = zip ais aos

  -- this works only because runTransactions is stateless
  -- and runs the SFs with dt = 0
  ((aios', sfs''), _) <- unMSF runTransactions (aios, sfs')

  let aos'  = map snd aios'
      ains' = map agentIn ais
      ct    = stepSimulation sfs'' ains'

  return (aos', ct)
\end{minted}

The implementation of \textit{runTransactions} is quite involved and omitted here because it would require too much space \footnote{The full code to all steps is freely available under: TODO provide link to a stable subfolder of my git repo.}, but we will give a short informal description.
All agents are iterated in an unspecified sequence and if an agent requests a transaction the other agent is looked up and the transaction-pair is run. This is done recursively until there are no transaction requests any-more (note that through the AgentOut of a committed transaction, an agent can request a new transaction within the same time-step). Running a transaction-pair works as follows:
The target agents signal-function is run again (resulting in a second, or third,... execution, depending on how many transactions have this agent as target) but now with a $\Delta t = 0$. The target agent can then accept the incoming transaction or simply ignore it. If it is ignored the transaction will never start. The fact that the target agent signal-function is run more than once within a simulation step but with a $\Delta t = 0$ requires agents to make their actions time-dependent \textit{but} they must listen to incoming transactions independent of time. The implementation of the infected agent below will make this more clear.
When the transaction is accepted the system switches to running the transaction signal-functions after another with passing the data forward and backward between the two agents. It is most important to note that again the signal functions are run with $\Delta t = 0$ because conceptionally transactions happen \textit{instantaneously} without time advancing. This has important implications, and means that we cannot use any time-accumulating function e.g. integral or after within a transaction - simply because it makes no sense as no time passes. If \textit{both} agents commit the transaction their new AgentOuts will replace the ones for the current simulation-step. If either one agent aborts the transaction the current AgentOuts of the current simulation-step will be used.

We provide a sequence diagram of data-flow in a multi-step negotiation as described in the introduction for a visual explanation of the complex protocol which is going on in a transaction.

Now it is time to look at the new agent implementations which use now the agent-transaction mechanism. The recovered agent is exactly the same but the susceptible and infected agent behaviour are very different now. Lets first look at the susceptible agent:

\begin{minted}[fontsize=\footnotesize]{haskell}
susceptibleAgent :: RandomGen g => [AgentId] -> SIRAgent g
susceptibleAgent ais = proc _ -> do
    makeContact <- occasionallyM (1 / contactRate) () -< ()

    if not TODO DLLAR isEvent makeContact
      then returnA -< agentOut Susceptible
      else (do
        contactId <- drawRandomElemS -< ais
        returnA -< requestTx 
                    (contactId, Contact Susceptible) 
                    susceptibleTx
                    (agentOut Susceptible))
  where
    susceptibleTx :: RandomGen g => SIRAgentTX g
    susceptibleTx = proc txIn -> do
      -- should have always tx data
      if hasTxDataIn txIn 
          then (do
            let (Contact s) = txDataIn txIn 
            -- only infected agents reply, but make it explicit
            if Infected /= s
              -- don't commit with continuation, no change in behaviour
              then returnA -< commitTx (agentOut Susceptible) agentTXOut
              else (do
                infected <- arrM (\_ -> lift TODO DOLLAR randomBoolM infectivity) -< ()
                if infected
                  -- commit with continuation as we switch into infected behaviour
                  then returnA -< commitTxWithCont 
                                    (agentOut Infected) 
                                    infectedAgent
                                    agentTXOut
                  -- don't commit with continuation, no change in behaviour
                  else returnA -< commitTx 
                                    (agentOut Susceptible) agentTXOut))
          else returnA -< abortTx agentTXOut
\end{minted}

Instead of using a switch the susceptible agent behaves completely time-dependent and occasionally starts a new agent-transaction with a random agent. The function \textit{susceptibleTx} handles the reply of the other agent. Note that we only commit with a continuation in case the agent becomes infected.

The infected agent is slightly less complex and still uses the switch mechanism:
\begin{minted}[fontsize=\footnotesize]{haskell}
infectedAgent :: RandomGen g => SIRAgent g
infectedAgent = 
    switch
    infected 
      (const recoveredAgent)
  where
    infected :: RandomGen g => SF (SIRMonad g) SIRAgentIn (SIRAgentOut g, Event ())
    infected = proc ain -> do
      recEvt <- occasionallyM illnessDuration () -< ()
      let a = event Infected (const Recovered) recEvt
      -- note that at the moment of recovery the agent can still infect others
      -- because it will still reply with Infected
      let ao = agentOut a

      if isRequestTx ain 
        then (do
          returnA -< (acceptTX 
                      (Contact Infected)
                      (infectedTx ao)
                      ao, recEvt))
        else returnA -< (ao, recEvt)

    infectedTx :: RandomGen g => SIRAgentOut g -> SIRAgentTX g
    infectedTx ao = proc _ -> do
      -- it is important not to commit with continuation as it
      -- would reset the time of the SF to 0. Still occasionally
      -- would work as it does not accumulate time but functions
      -- like after or integral would fail
      returnA -< commitTx ao agentTXOut
\end{minted}

The agent acts time-dependent which in this case is the transition from infected to recovered - if occasionallyM is run with a dt of 0 then no Event can happen (todo: is this really true??). The agent checks on every function call of infected for incoming transactions and accepts them all, independent of the state - only susceptible agents request transactions anyway. The agent simply replies with a Contact Infected and immediately commits the transaction in the transaction signal-function but does not switch into a new continuation.

\subsubsection{Reflection}
Note that the transactions run in the same monad as the normal agent behaviour signal-function which allows to add an environment as in step 5. In this case care must be taken when one has changed the environment but aborts the transaction as a roll back of the environment won't happen automatically. A different approach would allow to run the TX in a different monad and bring in e.g. the  transactional state monad Control.Monad.Tx which supports rolling back of changes to the state.

The concept of agent-transactions is not explicitly known in the agent-based community and a novel development of this paper. The reason for this is that agent-transactions are already implicitly available in traditional OO implementations in which agents can call each others methods and change their state. By implementing this necessary and important concept in a pure functional approach we arrived at agent-transactions which make these synchronous, instantaneous, one-to-one interactions explicit.
%\end{appendix}

\end{document}
