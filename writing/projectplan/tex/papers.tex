\section{Papers}
This is the list of papers I have in my mind and includes both work mandatory for the PhD and optional ones. The latter ones are just fun/philosophical-papers, not directly related to my PhD but somehow tangential with the very basic direction - they are intended to be worked on in my free-time and to free my head when wrestling too hard with my PhDs main work.

\subsection{The Art of Iterating: Update-Strategies in Agent-Based Simulations}
\textbf{Type:} Groundwork \\
\textbf{Target:} Conference \\
\textbf{Requirement:} Mandatory \\

When developing a model for an Agent-Based Simulation (ABS) it is of very importance to select the right update-strategy for the agents to produce the desired results. In this paper we develop a systematic treatment of all general properties, derive the possible update-strategies in ABS and discuss their interpretation and semantics something which is still lacking in the literature on ABS. Further we investigate the suitability of the three very different programming languages Java, Haskell and  Scala with Actors to implement each of the update-strategies. Thus this papers contribution is the development of a new, general terminology of update-strategies and their implementation comparison in various kinds of programming languages.

\subsection{Specification equals Code: An EDSL for pure functional Agent-Based Modelling \& Simulation}
\textbf{Type:} Main PhD Work \\
\textbf{Target:} Conference/Journal \\
\textbf{Requirement:} Mandatory \\

In our previous work on update-strategies in Agent-Based Modelling \& Simulation (ABM/S) we showed that Haskell is a very attractive alternative to existing object-oriented approaches but our presented approach was too limited and we hypothesized that embedding it within a functional reactive framework like Yampa would leverage it to be able to build much more complex models. In this paper we investigate whether this hypothesis is true by testing if our approach can easily be transferred to Yampa, what we really gain from it and if more complex models can become reality. Also we ask in this paper if the declarative power of the pure functional language can be utilized to write specifications for ABS models which can be directly translated to our Haskell implementation. As a proof-of-concept we implement the famous SugarScape model to proof that FrABS is able to implement also very complicated agent-based models and to see if we can formalize the rules of the model into our EDSL. 

\subsection{MetaABS: Recursive Agent-Based Simulation}
\textbf{Type:} Groundwork / New Idea \\
\textbf{Target:} Conference \\
\textbf{Requirement:} Mandatory \\

\subsection{Reasoning about dynamics and emergent properties in Agent-Based Simulations}
\textbf{Type:} Main PhD Work \\
\textbf{Target:} Journal \\
\textbf{Requirement:} Mandatory \\

Is the main work of the PhD and targeted at publication in a Journal.

%
%\subsection{Time in Games: a Tron Light-Cycle Game in Dunai}
%\textbf{Type:} Fun \\
%\textbf{Target:} Conference \\
%\textbf{Requirement:} Optional \\
%
%This paper describes the 2D light-cycle game inspired by the movie Tron implemented in Dunai. It allows to turn back time.
%
%\subsection{Pure Functional Islamic Design}
%\textbf{Type:} Fun \\
%\textbf{Target:} Conference \\
%\textbf{Requirement:} Optional \\
%
%Inspired by the paper "Functional Geometry" by Peter Henderson I had the idea to come up with a  EDSL for declaratively describing pictures of islamic design which are then rendered using the gloss-library. From its focus totally unrelated to the PhD topic but still a great opportunity to learn Haskell, to learn to think functional, to learn to design my own EDSL - thus it may be a great paper to pursue even if I won't finish or produce something publishable.

\subsection{The Genesis According to Computer-Science: Reality as Simulation of Free Will}
\textbf{Type:} Philosophy \\
\textbf{Target:} ? \\
\textbf{Requirement:} Optional \\

All sciences have their genesis-model which are basically explanations of how the world did come into existence, what the reason for existence is and who or what God is. Unfortunately computer science has none so far, so the aim of this paper is to set out to develop such genesis-model from a computer-science perspective. The model is motivated from the perspective that the world and humankind is a simulation to see free will in action in a sand-boxed environment as opposed to the afterlife / after-world or just the Beyond, which is an outer level of simulation and itself again a simulation as will be shown in subsequent sections.
This paper addresses important fundamental questions of belief and religion and tries to explain them using this model - which it does surprisingly well. Thus this paper has a keen aim: it wants to amalgamate religious concepts with concepts of theoretical computer science.
We build on the discussion of the \textit{simulation argument} proposed by \cite{bostrom_are_2003} which states that we \textit{may} live in a simulation, created by transhumans (TODO: explain). Although we neither can proof or disproof the \textit{simulation argument}, we find the idea of existence being a simulation highly intriguing as it finally allows us to \textit{investigate how existence can be understood from the perspective of computer science in a scientific way}. This was so far not possible due to the lack of a scientific context, which is now given through the \textit{simulation argument}. The obvious question which raises, is then \textit{why are we simulated?}. Bostrom does not give any reason for it in his paper as this obviously touches on ideological and religious ground. Our main hypothesis is that if we are living in a simulation, then it is to simulate Free Will. Further we hypothesize that the emergent properties of this simulation are the emergence of ideologies, brought forward by Free Will.