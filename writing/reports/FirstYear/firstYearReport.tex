\documentclass[oneside]{book}

\setcounter{tocdepth}{3}
\setcounter{secnumdepth}{3}

\usepackage[toc,page]{appendix}
\usepackage{hyperref}
\usepackage[utf8]{inputenc}
\usepackage{graphicx} % Required for the inclusion of images
\usepackage{amsmath} % Required for some math elements 
\usepackage[utf8]{inputenc}
\usepackage[english]{babel}
\newtheorem{theorem}{Theorem}
\newtheorem{corollary}{Corollary}[theorem]
\newtheorem{lemma}[theorem]{Lemma}
\usepackage{listings}
\usepackage{pdfpages}
\usepackage{amssymb}
\usepackage{pdflscape}

%fa
%TODO follow the instructions on http://workspace.nottingham.ac.uk/display/CompSci/Procedure+for+PhD+Annual+Reviews

% A written report that contains the following sections. There is no specific requirement on formatting but it should be well-presented and proof-read to ensure it is a piece of high-quality academic writing. As guidance, the whole report including the appendix should normally be around 8000-10000 words in length.

%TODO Current title of PhD project, Whether it is a first, second or third annual review report, word count, student name, student ID, names of supervisors, initial registration date, expected date for entering thesis pending stage, expected date for thesis submission.
\title{1st Year Report on \\ \textit{Pure Functional \\ Agent-Based Modelling \& Simulation}} % Title

\author{Jonathan \textsc{Thaler} \\ jonathan.thaler@nottingham.ac.uk} % Author name

\date{\today} % Date for the report


%----------------------------------------------------------------------------------------
%	BEGIN OF DOCUMENT
%----------------------------------------------------------------------------------------

\begin{document}

\maketitle % Insert the title, author and date

%\thispagestyle{fancy} % All pages have headers and footers

\cleardoublepage
\section*{Abstract}
%A succinct and concise summary (250 words maximum) of the report contents and presented on a single page.

So far specifying Agent-Based Models and implementing them as an Agent-Based Simulation (ABS) is done using object-oriented methods like UML and object-oriented  programming languages like Java. The reason for this is that until now the concept of an agent was always understood to be very close to, even equals to - which it is not - the concept of an object. Therefore, the reasoning goes, object-oriented methods and languages should apply naturally to specify and implement agent-based simulations. In this thesis we fundamentally challenge this assumption by investigating how Agent-Based Modelling and Simulation (ABS) can be done using pure functional methods. 
We show how ABS can be implemented in the pure functional language Haskell and develop a basic underlying theoretical framework in category-theory which allows to view and specify ABS from a new point-of-view. Having these tools at hand the major benefit using them is the potential for an unprecedented approach to validation \& verification of an ABS. First: due to the declarative nature of pure functional programming in Haskell it is possible to implement an EDSL for ABS which is both specification- and implementation-language thus closing the gap between model-specification and implementation. Second: by extracting a category-theoretical view on a real-world example and showing that the pure functional code is actually an implementation of this category-theoretical view allows a new, formal approach to validation.

\clearpage
\tableofcontents
\clearpage

%*******************************************************************************
%*********************************** First Chapter *****************************
%*******************************************************************************

\chapter{Introduction}  %Title of the First Chapter
I noticed that it is pretty hard to convince an agent-based economics specialist who is not a computer scientist about a pure functional approach. My conjecture is that the implementation technique and method does not matter much to them because they have very little knowledge about programming and are almost always self-taught - they don't know about software-engineering, nothing about proper software-design and architecture, nothing about software-maintenance, nothing about unit-testing,... In the end they just "hack" the simulation in whatever language they are able to: C++, Visual Basic, Java or toolboxes like Netlogo. For them it is all about to \textit{get things done somehow} and not to get things done the right way or in a beautiful way - the way and the method doesn't matter, its just a necessary evil which needs to be done. Thus if functional programming could make their lives easier, then they will definitely welcome it. But functional programming is, i think, harder to learn and harder to understand - so one needs to provide an abstraction through EDSL. So I REALLY need to come up with convincing arguments why to use pure functional approaches in ACE THEY can understand, otherwise I will be lost and not heard (not published,...). \\

What ACE economists care for:

\begin{itemize}
\item Very: Qualitative modelling with quantitative results
\item Yes: Easy reproducibility
\item Likely: Reasoning about convergence?
\item Likely: EDSL
\end{itemize}

My contributions are: pure functional framework, functional agent-model for market-simulations, EDSL for market-simulations, qualitative / implicit modelling with quanitative results, reasoning in my framework about convergence \\

IDEA: could I develop non-causal modelling (models are expressed in terms of non-directed equations, modelled in signal-relations) to allow for qualitative modelling for the agent-based economists? See hybrid modelling paper of Yampa. \textbf{THIS WOULD BE A HUGE NOVEL CONTRIBUTION TO ACE ESPECIALLY WHEN COMBINED WITH AN EDSL AND PROVIDING FULL REFERENTIAL TRANSPARENCY TO KEEP THE ABILITY TO REASON ABOUT CONVERGENCE}. This should be covered in the "EDSL"-paper.

TODO: maybe i should really focus only on market models? otherwise too much? \\

central novelty of my PhD: model specification = runnable code. possible through EDSL. but only in specific subfield of ACE: market-models. need a functional description of the model, then translate it to model specification in EDSL and then run it to see dynamics. But: model specification moves closer to functional programming languages. \\

another novelty approach: model specification through qualitative instead of quantiative approaches. is this possible? \\

WHY FUNCTIONAL? "because its the ultimate approach to scientific computing": fewer bugs due to mutable state (why? is thos shown obkectively by someone?), shorter (again as above, productivity), more expressive and closer to math, EDSL, EDSL=model=simulation, better parallelising due to referental transparency, reasoning \\

scientific results need to be reproduced, especially when they have high impact. a more formal approach of specifying the model and the simulation (model=simulation) could lead to easier sharing and easier reporduction without ambigouites \\

pure functional agent-model \& theory, EDSL framework in Haskell for ACE

\begin{enumerate}
\item Which kind of problem do we have?
\item What aim is there? Solving the problem? 
\item How the aim is achieved by enumerating VERY CLEAR objectives.
\item What the impact one expects (hypothesis) and what it is (after results).
\end{enumerate}

Note: It is not in the interest of the researcher to develop new economic theories but to research the use of functional methods (programming and specification) in agent-based computational economics (ACE).

NOTE: Get the reader’s attention early in the introduction: motivation, significance, originality and novelty.

\section{Methods}
Methods need to be selected to implement the simulations. Special emphasis will be put on functional ones which will then be compared to established methods in the field of ABM/S and ACE. \\

Claim: non-programming environments are considered to be not powerful enough to capture the complexity of ACE implementations thus a programming approach to ACE will be always required.

\section{Scenarios}
To apply and test functional methods in ACE, four scenarios of ACE are selected and then the methods applied and compared with each other to see how each of them perform in comparison. The 4 selected scenarios represent a selection of the challenges posed in ACE: from very abstract ones to very operational ones.

\section{Comparison}
Each of the selected scenarios is then implemented using the selected methods where each solution is then compared against the following criteria: 

\begin{enumerate}
\item suitability for scientific computation
\item robustness
\item error-sources
\item testability
\item stability
\item extendability
\item size of code
\item maintainability
\item time taken for development
\item verification \& correctness
\item replications \& parallelism
\item EDSL
\end{enumerate}

This will then allow to compare the different methods against each other and to show under which circumstances functional methods shine and when they should not be used.

\section{Agent-Based Modelling and Simulation (ABM/S)}
ABM/S is a method of modelling and simulating a system where the global behaviour may be unknown but the behaviour and interactions of the parts making up the system is of knowledge (Wooldrige, M. (2009). An Introduction to MultiAgent Systems. John Wiley & Sons). Those parts, called agents, are modelled and simulated out of which then the aggregate global behaviour of the whole system emerges. Thus the central aspect of ABM/S is the concept of an Agent which can be understood as a metaphor for a pro-active unit, able to spawn new Agents, and interacting with other Agents in a network of neighbours by exchange of messages. The implementation of Agents can vary and strongly depends on the programming language and the kind of domain the simulation and model is situated in.

\section{Agent-Based Economics (ACE)}
According to Leigh Tesfatsion (Tesfatsion, L. (2006). Agent-based computational economics: A constructive approach to economic theory. In Tesfatsion, L. and Judd, K. L., editors, Handbook of Computational Economics, volume 2, chapter 16, pages 831–880. Elsevier, 1 edition.), one of the leading figures, ACE is "[...] computational modelling of economic processes (including whole economies) as open-ended dynamic systems of interacting agents." - thus lending perfectly to the use of ABM/S as already the name suggests. Whereas classical economic models fall short by only looking at the average, pure rational, individual interacting in anonymous markets, the ACE approach looks at heterogeneous, non-rational individuals interacting with each other in networks (Kirman, A. (2010). Complex Economics: Individual and Collective Rationality. Routledge, London ; New York, NY.). Thus ACE can be understood as a combination of computer-science, cognitive/social science and evolutionary economics.

\section{Functional programming}
TODO: read \cite{Backus1978}

The state-of-the-art approach to implementing Agents are object-oriented methods and programming as the metaphor of an Agent as presented above lends itself very naturally to object-orientation (OO). The author of this thesis claims that OO in the hands of inexperienced or ignorant programmers is dangerous, leading to bugs and hardly maintainable and extensible code. The reason for this is that OO provides very powerful techniques of organising and structuring programs through Classes, Type Hierarchies and Objects, which, when misused, lead to the above mentioned problems. Also major problems, which experts face as well as beginners are 1. state is highly scattered across the program which disguises the flow of data in complex simulations and 2. objects don’t compose as well as functions. The reason for this is that objects always carry around some internal state which makes it obviously much more complicated as complex dependencies can be introduced according to the internal state.
All this is tackled by (pure) functional programming which abandons the concept of global state, Objects and Classes and makes data-flow explicit. This then allows to reason about correctness, termination and other properties of the program e.g. if a given function exhibits side-effects or not. Other benefits are fewer lines of code, easier maintainability and ultimately fewer bugs thus making functional programming the ideal choice for scientific computing and simulation and thus also for ACE. A very powerful feature of functional programming is Lazy evaluation. It allows to describe infinite data-structures and functions producing an infinite stream of output but which are only computed as currently needed. Thus the decision of how many is decoupled from how to (Hughes, J. (1989). Why functional programming matters. Comput. J., 32(2):98–107.).
The most powerful aspect using pure functional programming however is that it allows the design of embedded domain specific languages (EDSL). In this case one develops and programs primitives e.g. types and functions in a host language (embed) in a way that they can be combined. The combination of these primitives then looks like a language specific to a given domain, in the case of this thesis ACE. The ease of development of EDSLs in pure functional programming is also a proof of the superior extensibility and composability of pure functional languages over OO (Henderson P. (1982). Functional Geometry. Proceedings of the 1982 ACM Symposium on LISP and Functional Programming.).
One of the most compelling example to utilize pure functional programming is the reporting of Hudak (Hudak P., Jones M. (1994). Haskell vs. Ada vs. C++ vs. Awk vs. ... An Experiment in Software Prototyping Productivity. Department of Computer Science, Yale University.)  where in a prototyping contest of DARPA the Haskell prototype was by far the shortest with 85 lines of code. Also the Jury mistook the code as specification because the prototype did actually implement a small EDSL which is a perfect proof how close EDSL can get to and look like a specification.

Functional languages can best be characterized by their way computation works: instead of \textit{how} something is computed, \textit{what} is computed is described. Thus functional programming follows a declarative instead of an imperative style of programming. The key points are:
\begin{itemize}
\item No assignment statements - variables values can never change once given a value.
\item Function calls have no side-effect and will only compute the results - this makes order of execution irrelevant, as due to the lack of side-effects the logical point in \textit{time} when the function is calculated within the program-execution does not matter.
\item higher-order functions
\item lazy evaluation
\item Looping is achieved using recursion, mostly through the use of the general fold or the more specific map.
\item Pattern-matching
\end{itemize}

This alone does not really explain the \textit{real} advantages of functional programming and one must look for better motivations using functional programming languages. One motivation is given in \cite{Hughes1989} which is a great paper explaining to non-functional programmers what the significance of functional programming is and helping functional programmers putting functional languages to maximum use by showing the real power and advantages of functional languages. The main conclusion is that \textit{modularity}, which is the key to successful programming, can be achieved best using higher-order functions and lazy evaluation provided in functional languages like Haskell. \cite{Hughes1989} argues that the ability to divide problems into sub-problems depends on the ability to glue the sub-problems together which depends strongly on the programming-language and \cite{Hughes1989} argues that in this ability functional languages are superior to structured programming.

TODO: comparison of functional and object-oriented programming. My points are:
\begin{itemize}
\item The way state can be changed and treated - distributed over multiple objects - is often very difficult to understand.
\item Inheritance is a dangerous thing if not used with care because inheritance introduces very strong dependencies which cannot be changed during runtime anymore.
\item Objects don't compose very well: \url{http://zeroturnaround.com/rebellabs/why-the-debate-on-object-oriented-vs-functional-programming-is-all-about-composition/}
\item (Nearly) impossible to reason about programs
\end{itemize}

In conclusion the upsides of functional programming as opposed to OO are:
\begin{itemize}
\item Much more explicit flow of data \& control
\item Much better compose-able
\item Much better parallelism
\end{itemize}

\section{Related Research}
Tim Sweeney, CTO of Epic Games gave an invited talk about how "future programming languages could help us write better code" by "supplying stronger typing, reduce run-time failures;  and the need for pervasive concurrency support, both implicit and explicit, to effectively exploit the several forms of parallelism present in games and graphics." \cite{Sweeney2006}. Although the fields of games and agent-based simulations seem to be very different in the end, they have also very important similarities: both are simulations which perform numerical computations and update objects - in games they are called "game-objects" and in abm they are called agents but they are in fact the same thing - in a loop either concurrently or sequential. His key-points were:

\begin{itemize}
\item Dependent types as the remedy of most of the run-time failures.
\item Parallelism for numerical computation: these are pure functional algorithms, operate locally on mutable state. Haskell ST, STRef solution enables encapsulating local heaps and mutability within referentially transparent code.
\item Updating game-objects (agents) concurrently using STM: update all objects concurrently in arbitrary order, with each update wrapped in atomic block - depends on collisions if performance goes up.
\end{itemize}

\chapter{Literature Review}
\label{chap:literature}

In this chapter we present additional literature fundamental for the aims \& objectives of the 2nd year.

TODO: literature on dependent types
TODO: literature on verification \& validation in Simulation and ABS

\chapter{Reflecting the Literature}
\label{chap:refl}

As pointed out by \cite{jennings_agent-based_2000} on agent-based software-engineering, the problems are that patterns of the interactions are inherently unpredictable and that predicting the global system behaviour is extremely difficult. This observation is in unison with the results of my paper on update-strategies where we showed that a truly agent-based solution (actor-strategy) leads to non-deterministic results due to inherent concurrency.

\section{Conclusion}
we combine the good parts
1. Processes have “share nothing” semantics. This is obvious since they are imagined to run on physically separated machines.
2. Message passing is the only way to pass data between processes. Again since nothing is shared this is the only means possible to exchange data.
3. Isolation implies that message passing is asynchronous. If process communication is synchronous then a software error in the receiver of a message could indefinitely block the sender of the message destroying the property of isolation.
4. Since nothing is shared, everything necessary to perform a distributed computation must be copied. Since nothing is shared, and the only way to communicate between processes is by message passing, then we will never know if our messages arrive (remember we said that message passing is inherently unreliable.) The only way to know if a message has been correctly sent is to send a confirmation message back

we take the philosophy of the actor model as implemented in erlang (quoted above from Joe Armstrongs Thesis) and
implement it synchronously with reliable message-passing in a pure functional language. To add parallelism and concurrency
is not straight-forward for ABS because it depends on model-semantics but
is much much easier in a pure functional language


functional programming is our choice because it allows a deterministic, synchronized, actor style  implementation with  immutable messages and no sharing. and it allows algebraic and equational reasoning like process calculi\\

the single strength of todays oop is its use as a modelling language. the problem is what is going on under the hood: the sharing of mutable state. this is the real problem of mixing up the concept of object and agent

TODO: put at the end
Thread of argumentation: actor model is nice but for the kind of simulation we want to do we need too much synchronization otherwise we end up non-deterministic \& non-replicable. also actor-model very difficult to reason about because of non-deterministic specifications of message-transmission. what we need is a combination: deterministic, synchronized traversal of agents which are represented as a pure function: immutable messages and no exchange of aliases through which state can be mutated.

process calculi are nice for algebraic reasoning but are too cumbersome and not feasible for real, complex ABS. You do not program a large system in the lambda calculus, as you would not program a real distributed system in a process calculus \footnote{it was shown by TODO: cite that the pi-calculus can encode the lambda calculus, thus it is conceptually on a very low level: too much raw power leads to chaos.}. they may be of use for verification \& validation later on of small, critical parts of the ABS-communication which can be mapped to e.g. the pi-calculus and then apply algebraic reasoning.
As emphasised in the literature-review, not research was found on using process-calculi in the field of ABS. Although we can reason that if the $\pi$-calculus can be used to specifying and reasoning about MAS then it should be possible to do so for ABS or parts of it. There exists also a connection from the actor-model to process calculi \cite{agha_foundation_1997}, which strengthen our argument. 

the concept of homoiconicity of LISP seems to be very interesting and powerful to apply to ABS but an agent-based model where this extremely powerful technique can be applied is lacking. Also it is clear that when using this technique verification and validation becomes immensely more difficult, if not even impossible. Thus we refrain from LISP and its homoiconicity and look for something more structured, static and not as dynamic.

This leads us to pure functional programming with FRP which combines all the benefits: local memory, messages, switching of behaviour, algebraic reasoning, static typesystem

By the literature-review it seems that all the problems of object-oriented programming (as it is done in Java and C++) mentioned in the introduction, can be solved by (pure) functional programming which abandons the concept of global state, Objects and Classes and makes data-flow explicit. This then allows to reason about correctness, termination and other properties of the program e.g. if a given function exhibits side-effects or not. Other benefits are fewer lines of code, easier maintainability and ultimately fewer bugs thus making functional programming the ideal choice for scientific computing and simulation and thus also for ACE.

\section{Actor Model}
upside: extreme huge number of agnets possible due to distributed and parallel technology 
downside: depends on system \& hardware: scheduler, system time, systime resolution (not very nice for scientific computation), much more complicated, debugging difficult due to concurrency, no global notion of time appart from systemtime, thus always runs in real-time, but there is no global notion of time in the actor model anyway, no EDSL full of technical details, no determinism, no reasoning

Agents more a high-level concept, Actors low level, technical concurrency primitives

This makes simulations very difficult and also due to concurrency implementing a sync conversation among agents is very cumbersome. I have already experience with the Actor Model when implementing a small version of my Master-Thesis Simulation in Erlang which uses the Actor Model as well. For a continuous simulation it was actually not that bad but the problem there was that between a round-trip between 2 agents other messages could have already interfered - this was a problem when agents trade with each other, so one has to implement synchronized trading where only messages from the current agent one trades with are allowed otherwise budget constraints could be violated. Thus I think Erlang/Akka/Actor Model is better suited for distributed high-tolerance concurrent/parallel systems instead for simulations. Note: this is definitely a major point I have to argue in my thesis: why I am rejecting the actor model.


AKKA: thus my prediction is: akka/actor model is very well suited to simulations which 1. dont rely on global time 2. dont have multi-step conversations: interactions among agents which are only question-answer. TODO: find some classical simulation model which satisfies these criterias.

how can we simulate global time? how can we implement multistep conversations (by futures)?

The real problem seems to be concurrency but i feel we can simulate concurrency by synchronizing to continuous time. computations are carried out after another but because time is explicitly modelled they happen logically at the same time. these rules hold: an agent cannot be in two conversations at the same time, the agent can be in only one or none conversation at a given time t.

What if time is of no importance and only the continuous dynamics are of interest?

To put it another way: real concurrency (with threads) makes time implicit by connecting it to the real-time of the real-world, which is what one does NOT want in simulation. Maybe FRP is the way to go because it allows to explicitly model continuous and discrete time, but I have to get into FRP first to make a proper judgement about its suitability.

\cite{bezirgiannis_improving_2013} describes in chapter 3.3 a naive clone of NetLogo in the Erlang programming language where each agent was represented as an Erlang process. The author claims the 1:1 mapping between agent and process to "be inherently wrong" because when recursively sending messages (e.g. A to B to A) it will deadlock as A is already awaiting Bs answer. Of course this is one of the problems when adopting Erlang/Scala with Akka/the Actor Model for implementing agents \textit{but it is inherently short-sighted to discharge the actor-model approach just because recursive messaging leads to a deadlock}. It is not a problem of the actor-model but merely a very problem with the communication protocol which needs to be more sophisticated than \cite{bezirgiannis_improving_2013} described. The hypothesis is that the communication protocol will be in fact \textit{very highly application-specific} thus leading to non-reusable agents (across domains, they should but be re-usable within domains e.g. market-simulations) as they only understand the domain-specific protocol. This is definitely NOT a drawback but can't be solved otherwise as in the end (the content of the) communication can be understand to be the very domain of the simulation and is thus not generalizable. Of course specific patterns will show up like "multi-step handshakes" but they are again then specifically applied to the concrete domain.

\subsection{Alan Kays Object-Oriented Programming}
Alan Kay, the inventor of the OO idea had a system like the actor-model in mind when he conceived oop \url{http://wiki.c2.com/?AlanKaysDefinitionOfObjectOriented}. It follows the actor model in the essence that each object has a mailbox to which other objects can send immutable messages which contain non-shareable state (e.g. no pointers, references,...). Erlang follows this approach strictly but Scala allows one to circumvent this approach by sending mutable messages and references thus violating the locality of state.\\

\section{Pure Functional Programming}
why pure functional programming? what are its strengths? why does it overcome the problems? what are possible problems when doing it pure functional instead of oop?

\subsection{Strengths}
\begin{itemize}
	\item Pure - explicit about effects through monads
	\item Composability through higher-order functions and lazyness
	\item EDSL by its declarative style
	\item Reasoning through EDSL and lack of implicit side-effects 
	\item Static Type System
\end{itemize}

\subsection{Weaknesses}
Here we give an overview of the weaknesses and problems of pure, lazy functional programming in Haskell.

\paragraph{Space-Leaks}
The main issue in a lazy functional programming language is the difficulty of predicting space behaviour, which is very hard even for experienced programmers \cite{hudak_history_2007}. The problem arises from the fact, that Haskell abstracts away from evaluation order and object lifetimes. Programmers have no way to determine which data-structures live for how long - indeed they don't want and should not be bothered to think about these details as this would violate the whole concept behind pure lazyness \cite{hudak_history_2007}.

\paragraph{Debugging}
Due to the lazy evaluation and non-imperative programming style it becomes apparent that debugging needs to be approached completely different than in imperative programming where one can freely set breakpoints to statements and inspect data. This is not possible in Haskell as there are no imperative statements and the data may have not been evaluated yet due to the unpredictable evaluation order as mentioned already in the space-leak problem.

\paragraph{Performance}
Due to the lack of side-effects and aliasing, efficient in-order updates of memory is not as easily possible as in imperative languages like C thus real-time applications like Games which have a big global mutable state run much slower compared to its imperative implementations.
TODO: check if these references really support my argument \cite{mun_hon_functional_2005}, \cite{meisinger_game-engine-architektur_2010} - The works on game-programming in Yampa mention a similar problem (FRP-section).

\paragraph{Monad composition}
Monads do not compose in a nice, modular way and this issue is still open research \cite{hudak_history_2007}. TODO: is this still the case?

To do verification we need a form of formal specification which can be translated easily to the code. Being inspired by the previously mentioned work on a functional framework for agent-based models of exchange in \cite{botta_functional_2011} we opt for a similar direction. Having Haskell as the implementation language instead of an object-oriented one like Java allows us to build on the above proposed EDSL for ABS. Because of the declarative nature of the hypothesized EDSL it can act both as specification- and implementation-language which closes the gap between specification and implementation. This would give us a way of formally specifying the model but still in a more readable way than pure mathematics. This form of formal specification can act easily as a medium for communication between team-memberes and to the scientific audience in papers. Most important the explicit step of verification becomes obsolete as there exists no more difference between specification and implementation. The last point seems to be quite ambitious but this is a hypothesis and we will see in the course of the thesis how far we can close the gap in the end with this approach.

In \cite{claessen_quickcheck:_2000} introduce \textit{QuickCheck}, a testing-framework which allows to specifify properties and invariants of ones functions and then test them using randomly generated test-data. This is an additional tool of model-specification and increases the power and strength of the verification process and more properties of a model can be expressed which are directly formulated in code through the EDSL of QuickCheck AND the EDSL of FrABS. Of course it also serves for testing (e.g. regression) and points out errors in the implementation e.g. wrong assumptions about input-data. The authors claim that the major advante of QuickCheckj is to formulate formal specifications which help in understanding a program.
TODO: the question is whether it can be used for Validation as well.



\section{Identifying the Gap}
- Functional programming in this area exists but only scratches the surface and focus only on implementing agent-behaviour frameworks like BDI. An in-depth treatise of Agent-Based Modelling and implementing an Agent-Based Simulation in a pure functional language has so far never been attempted.

- There basically exists no approach to Agent-Based Modelling \& Simulation in terms of Category-Theory and Type-Theory

- Verification is an issue in ABS as they are very often described in natural language and supplemented with a few formulas. This leads to implementation-errors, e.g. Gintis Bartering-Paper, and results become hard to reproduce. Such errors become a threatening problem when simulation-results are used in decision making e.g. economics, policy-making, ...

- Validation is basically an untouched topic in ABS: models are formulated, a few hypotheses are formulated, the model is implemented and run, then the results are checked against the hypotheses. What the field of ABS needs is an in-depth discussion on how to rigorously validate a model. Validation is of course only as strong as the verification part: if the implementation is wrong anyway then we can not rely on anything (from false comes nothing)


- developing a category- \& type-theoretical view on Agent-Based Modelling \& Simulation which will 
	-> 1. give a deeper insight into the structure of agents, agent-models and agent-based simulation
	-> 2. serves as the basis for the pure functional implementation
	-> 3. serves as a high-level specification tool for agent-models

- implementing a library called FrABS based upon the FRP paradigm which allows to specify Agent-Based Models in an EDSL and run them

- Verification: closing the gap between specification and implementation through the category- \& type-theoretical view and the EDSL

- Validation: formalizing hypotheses and reasoning about dynamics and expected outcomes of the simulation

Define 5 general research questions for each Research-Context
	\begin{itemize}
    \item 2 related to FP
    \item 1 related to integration of FP to ABM/S
    \item 2 related to ABM/S
    \end{itemize}
    
\section{Deriving the research-direction}
TODO: select the direction: Category-Theory and not Actor Model, Haskell with FRP and not Scala\&Actors/Erlang. need a thorough explanation reasoning

emphasize my own research on scala with actors, experiments with erlang and prototyping with haskell.

As becomes evident from the literature-review we advocate pure functional programming in Haskell and its category-theoretic foundations as a solution to the questions posed. The usage of pure functional programming in ABS is also a strong motivation by itself as it hasn't been researched yet and deserves a thorough treatment on its own. Surprisingly there exist hardly any attempts on implementing ABS in pure functional programming as will become clear in the literature-review. Maybe this can also be seen as a hint that ABS lacks a level of deductive formalism which we hope to repair with our thesis. 

So put short the motivation is a twofold direction, referring to each other in a circular way. First, pure functional programming has not been researched for implementing and specifying ABS so far. Second,  the current state-of-the-art seems to be susceptible to flaws and bugs due to the lack of powerful verification. Combining both issues forms the very basic motivation of our thesis: use pure functional programming and its underlying theoretical framework to develop new methods for specifying, implementing, verifying and validating ABS to create simulations which are more reliable, reproducible and communicatable.


\chapter{Aims and Objectives}
\label{chap:aimsObj}

WARNING:
STM is possible in other languages as well!! rework this in the report. only haskell can guarantee specific things already statically at compile time





From the annual review the following things become clear:
- the aim is basically "Explore using Haskell for Agent-Based Simulation with its benefits and drawbacks".
- The 3 major benefits of the approach I claim
	1. code == spec
	2. can rule out serious class of bugs
	3. we can perform reasoning about the simulation in code
	need to be metricated: e.g. this is really only possible in Haskell and not in Java. This needs thorough thinking about which metrics are used, how they can be aquired, how they can be compared,...
- Why ACE and Social Simulation? Did i only pick these fields because they are easily applicable to the problems I want to solve? Which properties do they exhibit which make them interesting for my problem? Just to say that "Sugarscape and bilateral decentralized bartering is interesting and fascinates me" is not enough in a final viva/thesis/paper.
- Reasoning must be very clear. So far I have 2 ideas for formal reasoning in code:
	1. SIR
		-> my emulation of SD using ABS is really an implementation of the SD model and follows it - they are equivalent
		-> my ABS implementation is the same as / equivalent to the SD emulation
			=> thus if i can show that my SD emulation is equlas to the SD model
			=> AND that the ABS implementation is the same as the SD emulation
			=> THEN the ABS implementation is an SD implementation, and we have shown this in code for the first time in ABS

	2. Decentralized Bilateral Bartering
		-> can we reason about the equilibrium prices in an ABS setting? e.g. show formally why equilibrium prices are not reached, under which circumstance they are reached,...
			-> need to combine General Equilibrium Theory
			-> with Bilateral decentralized exchange
			-> with Agent-Based Simulation 
			
- STILL I NEED TO SHOW HOW I CAN MAKE HASKELL RELEVANT IN THE FIELD OF ABS
	-> as far as I know so far no reasoning has been done in the way I intend to do it in the field of ABS. My hypothesis is that it is really only possible in Haskell due to its explicit side-effects, type-system, declarative style,... 
		-> TODO: need to check if this is really unique to haskell
	-> the functional-reactive approach seems to bring a new view to ABS with an embedded language for explicit time-semantics. Together with parallel/sequential updating this allows implementing System-Dynamics and agents which rely on continuous time-semantics e.g. SIR-Agents. Maybe i invented a hybrid between SD and ABS? Also what about time-traveling? The problem is that this is not really clear as i hypothesize that is completely novel approach to ABS - again I need to check this!
		-> TODO: is this really unique to functional reactive? E.g. what about Repast, NetLogo, AnyLogic, other Java-Frameworks? 
	-> maybe i have to admit that its not as unique as thought

In General i need to show that
- Haskells general benefits \& drawbacks over other Languages in the Field of ABS (e.g. Java, NetLogo, Repast) e.g. declarative style, reasoning, explicit about side-effects, performance, difficult to reason about performance, space-leaks difficult. So this focuses on the general comparison between the established technologies of ABS and Haskell but not yet on Haskells suitability in comparison to these other technologies. Here we talk about reasoning, side-effects, performance IN GENERAL TERMS, NOT SPECIFIC TO ABS
- Haskells suitability to implement ABS in comparison to other languages and technologies in the Field of. Here the focus is on general problems in ABS and how they can and are solved using Haskell.
- Why using Haskell in ABS - do the general benefits / drawbacks apply equally well? Are there unique advantages? Can we do things in Haskell which are not possible in other technologies or just very hard? E.g. the hybrid-approach I created with FRP: how unique is it e.g. can other technologies easily implement it as well? Other potential advantages: recursive simulation. Here we DO NOT concentrate on general technicalities but see how they apply when using it for ABS and if they create a unique benefit for Haskell in ABS.

\section{Aims}
The aim of this Ph.D is to explore the benefits and drawbacks using Haskell in  Agent-Based Simulation. First a library for general-purpose ABS in Haskell is built which serves as the primary object to study the benefits and drawbacks. After having investigated the benefits and drawbacks the library will be used to research  verification and reasoning in ABS in the context of decentralized bilateral bartering as specified in the Sugarscape model.

\section{Objectives}
\begin{enumerate}
	\item Implement a library for general-purpose Agent-Based Simulation in Haskell 
	\item Objectively and scientifically compare the usage of Haskell in ABS to the usage of Java in ABS: what are the benefits/drawbacks of Haskell and what are the benefits/drawbacks of Java? Are they orthogonal to each other e.g. are the weaknesses of one language the other languages strength?
	\item Define scientific measures: e.g. Lines Of Code (show relation to Bugs \& Defects, which is an objective measure: http://www.stevemcconnell.com/est.htm, \url{https://softwareengineering.stackexchange.com/questions/185660/is-the-average-number-of-bugs-per-loc-the-same-for-different-programming-languag}, Book: Code Complete, \url{https://www.mayerdan.com/ruby/2012/11/11/bugs-per-line-of-code-ratio}), also experience reports by companies which show that Haskell has huge benefits when applied to the same domain of a previous implementation of a different language, post on stack overflow / research gate / reddit, read experience reports from \url{http://cufp.org/2015/}
	\item Develop reasoning-techniques using Functional Programming in ABS by comparing the implementations of the SD- and ABS-model of the SIR compartment model in epidemiology.
	\item Investigate the usage of STM for concurrent ABS
\end{enumerate}

\section{Research Questions}
\begin{enumerate}
	\item Which is the best / a valid / good working approach of implementing ABS in Haskell?
	\item What are the benefits of using Haskell in ABS?
	\item What are the drawbacks of using Haskell in ABS?
	\item Are there things which are unique when doing Haskell in ABS and cannot be done in a Java approach?
	\item Both the System-Dynamics and Agent-Based implementation of the SIR compartment model in epidemiology lead to the same dynamics or put different: the Agent-Based implementation shows the same dynamics of the SD implementation when using replications. This is shown by plotting the dynamics as graphs. Can we show that they are equivalent through reasoning about the code?
	\item In the Sugarscape model where Agents engage in bilateral decentralized bartering Equilibrium is only reached when neo-classical agents are used which don't die of natural age. The equilibrium is not reached when more realistic assumptions are made. This is shown by plotting the prices over time. Can we show that the equilibrium is reached / not reached when using neo-classical agents / realistic agents through reasoning in the code?
	\item What and to which extent can we reason about an Agent-Based Simulation using my implementation in Haskell?
	\item Can we run Agents concurrently in STM but still retain reproducibility of the simulation?
\end{enumerate}

\section{Hypotheses}
\begin{enumerate}
	\item Yampa is a valid / good approach of implementing ABS in Haskell.
	\item Haskell benefits (which are not possible in java) are: Code == Spec, can statically rule out a major and very relevant class of bugs, can perform reasoning and proofing of properties of the program
	\item Haskell drawbacks over java are: slower, potential for difficult to find space-leaks, much more difficult to reason about performance in general, steeper learning curve, think ABS different 
	\item reproducibility of a system: lack of unpredictable side-effects statically enforced in the type-system
	\item Unique to Haskell is that it enables STM.
\end{enumerate}

\chapter{Work To Date}
\label{chap:work_to_date}

From a very general perspective I am researching a novel implementation approach to ABS. The hypothesis is that due to its underlying foundations of pure functional programming, this approach leads to simulation software which is easier to verify and validate and thus more likely to be correct, less sources of bugs and is conceptually cleaner.
In the first half of my PhD (October 2016 - March 2018) I have learned the underlying foundations of pure functional programming, did lots of prototyping and ultimately developed a way of implementing ABS in this approach. This resulted in a paper, submitted in March 2018 to the Haskell Symposium 2018, which discusses \textit{how} to do agent-based simulation with pure functional programming as foundation and how to solve the fundamental problems of encapsulating agent-state, doing agent-interactions and bringing in environments in this setting.
We found out that we immediately benefit from this approach in various ways, supporting our initial hypothesis but didn't investigate it in scientific details, which we leave for the next 12 months, conducted between April 2018 and April 2019. Thus we will be researching the \textit{why} of our approach, which we claim is an easier and stronger approach to verification and validation (V \& V). We need to clarify the meaning of V \& V in both areas we trying to gap, pure functional programming and agent-based simulation, and how they related to each other and how we can connect them. Further we need to quantify our claims of less sources of bugs through other research and comparing it to imperative OO approaches.
In this time we will investigate the use of dependent types in our pure functional approach to agent-based simulation which we hypothesise should allow an unprecedented level of verification and validation, not possible (even not on a theoretical level) with imperative, traditional object-oriented approaches. There exists literally no research on this topic thus it will form the unique and sufficiently advanced, novel contribution of our PhD to the field. We will also write an additional paper which will investigate how dependent types can be made of use in ABS.
Around December 2018 I will start writing another paper which is targeted for an agent-based simulation journal and is written as a conceptual paper, describing the approach and benefits of purely and dependently typed agent-based simulation. While writing this paper I will start constructing the main argument structure of my thesis so I have structure already when I start writing the thesis in April 2019. The last 6 months of the PhD (April 2019 - September 2019) will be dedicated to writing up the thesis and conducting additional research if still necessary.

So roughly the PhD can be split into 3 phases:
Researching the HOW: September 2016 - March 2018
Researching the WHY: April 2018 - March 2019
Writing the Thesis: April 2019 - September 2019

2016
October - December
	proper learning haskell programming
	experiments with scala \& akka (actor model)
	
2017
January - March
	writing 1st paper: Art Of Iteration
	MGS2017
	deepening haskell programming skills

April - July
	getting into functional reactive programming
	literature research \& 1st year report and review
	prototyping concepts of purely functional ABS
	presentation to FP group at FP lunch

August
	holiday
	reading book and papers on functional programming

September
	preparation for social simulation conference 2017 (SSC2017)

October - December
	working on 2nd paper (pure functional epidemics): first draft
	generalising the functional reactive programming approach to monadic stream functions 

2018
January - February
	prototyping event-scheduling concepts in pure functional ABS
	completely reworking 2nd paper: 2nd draft
	learning Idris language (pure functional, dependently typed programming)

March
	finalising 2nd paper and submission to Haskell Symposium 2018
	deepening Idris knowledge and experience 

April
	bit of work on verification and validation of the purely functional SIR abs implementation using quickcheck. this is not original research but will then be useful for the final thesis as a small separate section
	started 3rd paper on dependent types in purely functional ABS

May
	feedback on 2nd paper on 18 May	
	researching concepts of dependent types in purely functional ABS
	research \& writing 3rd paper

June - July
	2nd year report \& review
	research \& writing 3rd paper


Here we give a concise overview over the activities performed in the 2nd year.

\section{Social Simulation Conference 2017}


\section{Paper Published}
TODO: Art of Iteration was published in SSC2017 proceedings

\section{Papers Submitted}
\subsection{Pure Functional Epidemics}
This paper, which is attached in Appendix \ref{app:pfe}

\section{Reports}
\subsection{Haskell Communities and Activities Report (HCAR) May 2017}
We wrote a new entry for the HCAR May 2017, which tries to compile and publish novel and on-going ideas in the Haskell community. It is freely available under \url{https://www.haskell.org/communities/05-2017/html/report.html}. We hope that our idea and the work of our PhD gets a bit more attention and may start some discussions with people interested in this work.

\subsection{2nd Year Report}
This document.


\section{Talks}
So far only two talks were given. The first one was a presentation of the ideas underlying the update-strategies paper at the IMA - seminar day. The second was presenting my ideas about functional reactive ABS to the FP-Lab Group at the FPLunch.



\section{Future Work Plan}
TODO: A future work plan that is consistent with the progress to date, stating clearly the research question(s) to be addressed during the next year of the PhD.

TODO: gantt chart!

\subsection{TODOs}
out of this i will build the gantt chart for the next 12 months+

\subsubsection{Category Theory}
develop category theory behind FrABS: look into monads, arrows

category theory foundations (monads, arrows)

\subsubsection{Implementation and Software-Engineering}
%how to implement in haskell (sugarscape, agent_zero) 
implement chapter 4 of sugarscape
implement chapter 5 of sugarscape
use monadic or arrowized programming for structuribg the siftware
implement schelling segregation in recursiveABS and report results

FrABS: SugarScape
1st prototype: pure-functional implemented, no category-theory/type-theory applied
2nd prototype: category-theory/type-theory applied: clean monadic / arrowized programming applied

Agent\_Zero
1st prototype: implemented the book, based upon FrABS 

\subsubsection{Verification and Validation}
look into QuickCheck to test and verificate FrABS. start with SIRS
(quickcheck, isabelle, agda?), recursive simulation

\subsubsection{Papers}
paper 2: recursive ABS
paper 3: FrABS - Towards pure functional programming in ABS
paper 4: Towards category theory in ABS
paper 5: verification and validation in ABS with pure functional programming


\subsubsection{Reading}
read "Writing For Computer Science"
read "Agent\_Zero"
read "category theory for the sciences"


\subsection{Concept of an Agent}
an agent is not an object but when implementing ABS in oo then it is tempting to treat an agent like that. when implementing it in a pure functional language like haskell, this temptation cannot arise which creates a different view on agents.



\chapter{Conclusions}
\label{ch:conclusions}

This chapter concludes the whole thesis and outlines future research. Roughly 20\% exists already.

%we now know how to engineer time- and event-driven ABS with complex state both in the agent and environment, main difficulty is direct agent-interaction (see macal classification into 4 types of ABS), compile-time guaranteed reproducibility, explicit handling of complex state (read only, read/write), concurrency explicit and limited to STM, very promising concurrency but direct agent-interactions main problem (erlang as a rescue?), main drawbacks: everything is explicit, performance

\section{Further Research}
clearly outline the ideas for further research

\subsection{A general purpose library}
generalise concepts explored into a pure functional ABS library in Haskell (called chimera)

\subsection{Dependent and linear types}
dependent types and linear types are the next big step, towards a stronger formalisation of agents and ABS,
focus on the equilibrium - totality correspondence

\subsection{Concurrent event-driven ABS}
stm based concurrency for event-driven ABS using parallel DES. challenge is the time-warp implementation using monads. in general it should be easy to roll-back agents actions but with monads we have to be careful - for some monads rolling back is not neccessary e.g. rand and reader, for others it is, and for some it is impossible e.g. IO

\bibliographystyle{acm}
\bibliography{../../references/phdReferences}

\begin{appendices}

TODO: add full code of SIR implementation

\chapter{Validating Sugarscape in Haskell}
\label{app:validating_sugarscape}

In this chapter we look at how property-based testing can be made of use to verify the \textit{exploratory} Sugarscape model \cite{epstein_growing_1996} as introduced in Chapter \ref{sec:sugarscape}. Whereas in the chapters on testing the explanatory SIR model we had an analytical solution, the fundamental difference in the exploratory Sugarscape model is that none such analytical solutions exist. This raises the question, which properties we can actually test in such a model.

The answer lies in the very nature of exploratory models, they exist to explore and understand phenomena of the real world. Researchers come up with a model to explain the phenomena and (hopefully) with a few questions and \textit{hypotheses} about the emergent properties. The actual simulation is then used to test and refine the hypotheses. Indeed, descriptions, assumptions and hypotheses of varying formal degree abound in the Sugarscape model. Examples are: \textit{the carrying capacity becomes stable after 100 steps; when agents trade with each other, after 1000 steps the standard deviation of trading prices is less than 0.05; when there are cultures, after 2700 steps either one culture dominates the other or both are equally present}. 

We show how to use property-based testing to formalise and check such hypotheses. For this purpose we undertook a full \textit{verification} of our \href{https://github.com/thalerjonathan/haskell-sugarscape}{implementation}~\cite{thaler_sugarscape_repository} from Chapter \ref{sec:sugarscape}. We validated it against the book \cite{epstein_growing_1996} and a NetLogo implementation \cite{weaver_replicating_2009}  \footnote{Lending didn't properly work in their NetLogo code and that they didn't implement Combat.}.

\section{Property-based hypothesis testing}
The property we test for is whether \textit{the emergent property / hypothesis under test is stable under replicated runs} or not. To put it more technical, we use QuickCheck to run multiple replications with the same configuration but with different random-number streams and require that all tests pass. During the verification process we have derived and implemented property tests for the following hypotheses:

\begin{enumerate}
	\item Disease dynamics where all agents recover - when disease are turned on, if the number of initial diseases is 10, then the population is  able to rid itself completely from all disease within 100 ticks. 
	
	\item Disease dynamics where a minority recovers - when disease are turned on, if the number of initial diseases is 25, the population is not able to rid itself completely from all diseases within 1,000 ticks.
	
	\item Trading dynamics - when trading is enabled, the trading prices stabilise after 1,000 ticks with the standard deviation of the prices having dropped below 0.05.
	
	\item Cultural dynamics - when having two cultures, red and blue, after 2,700 ticks, either the red or the blue culture dominates or both are equally strong. If they dominate they make up 95\% of all agents, if they are equally strong they are both within 45\% - 55\%.
	
	\item Inheritance Gini coefficient - when agents reproduce and can die of age then inheritance of their wealth leads to an unequal wealth distribution measured using the Gini Coefficient \textit{averaging} at 0.7.
	
	\item Carrying capacity - when agents don't mate nor can die from age, due to the environment, there is an \textit{average} maximum carrying capacity of agents the environment can sustain. The capacity should be reached after 100 ticks and should be stable from then on.
		
	\item Terracing - when resources regrow immediately, after a few steps the simulation becomes static. Agents will stay on their terraces and will not move any more because they have found the best spot due to their behaviour. About 45\% will be on terraces and 95\% - 100\% are static, not moving any more.
\end{enumerate}

The hypotheses and their validation is described more in-depth in the section \ref{sec:hypotheses_testcases} below.

\subsection{Implementation}
To start with, we implement a custom data generator to produce output from a Sugarscape simulation. The generator takes the number of ticks and the scenario with which to run the simulation and returns a list of outputs, one for each tick.

\begin{HaskellCode}
sugarscapeUntil :: Int                -- ^ Number of ticks to run
                -> SugarScapeScenario -- ^ Scenario to run
                -> Gen [SimStepOut]   -- ^ Output of each step
sugarscapeUntil ticks params = do
  -- create a random-number generator
  g <- genStdGen
  -- initialise the simulation state with the given random-number generator
  -- and the scenario
  let simState = initSimulationRng g params
  -- run the simulation with the given state for number of ticks
  return (simulateUntil ticks simState)
\end{HaskellCode}

Using this generator, we can very conveniently produce Sugarscape data within a QuickCheck \texttt{Property}. Depending on the problem, we can generate only a single run or multiple replications, in case the hypothesis is assuming \textit{averages}. To see its use, we show the implementation of the \textit{Disease Dynamics (1)} hypothesis.

\begin{HaskellCode}
prop_disease_allrecover :: Property
prop_disease_allrecover = property (do
  -- after 100 ticks...
  let ticks = 100
  -- ... given Animation V-1 parameter configuration ...
  let params = mkParamsAnimationV_1
  -- ... from 1 sugarscape simulation ...
  aos <- last <*> (sugarscapeUntil ticks params)
  -- ... counting all infected agents ...
  let infected = length (filter (==False)) map (null . sugObsDiseases . snd) aos
  -- ... should result in all agents to be recovered
  return (cover 100 (infected == 0) "Diseases all recover" True)
\end{HaskellCode}

From the implementation it becomes clear, that this hypothesis states that the property has to hold \textit{for all} replications. The \textit{Inheritance Gini Coefficient (5)} hypothesis on the other hand assumes that the Gini Coefficient \textit{averages} at 0.7. We cannot average over replicated runs of the same property thus we generate multiple replications of the Sugarscape data within the property and employ a two-sided t-test with a 95\% confidence to test the hypothesis:

\begin{HaskellCode}
prop_gini :: Int      -- ^ Number of replications
          -> Double   -- ^ Confidence of the t-test
          -> Property
prop_gini repls confidence = property (do
  -- after 1000 ticks...
  let ticks = 1000
  -- ... the gini coefficient should average at 0.7 ...
  let expGini = 0.7
  -- ... given the Figure III-7 parameter configuration ...
  let params = mkParamsFigureIII_7
  -- ... from 100 replications ... 
  gini <- vectorOf repls (genGiniCoeff ticks params)
  -- on a two-tailed t-test with given confidence
  return (tTestSamples TwoTail expGini (1 - confidence) gini)
\end{HaskellCode}

%genGiniCoeff :: Int -> SugarScapeScenario -> Gen Double
%genGiniCoeff ticks params = do
%  -- generate sugarscape data
%  aos <- sugarscapeUntil ticks params
%  -- extract wealth of the agents in the last step
%  let agentWealths = map (sugObsSugLvl . snd) (last aos)
%  -- compute gini coefficient and return it
%  return (giniCoeff agentWealths)

\subsection{Running the tests}
As already pointed out in Part \ref{ch:property}, QuickCheck by default runs up to 100 test cases of a property and if all evaluate to \texttt{True} the property test succeeds. On the other hand, QuickCheck will stop at the first test case which evaluates to \texttt{False} and marks the whole property test as failed, no matter how many test cases got through already. For this reason we have used \texttt{cover} with an expected percentage of 100, meaning that we expect all tests to fall into the coverage class. This allows us to emulate failure with QuickCheck reporting the actual percentage of passed test cases.

Due to the duration even 1,000 ticks can take to compute, to get a first estimate of our hypotheses tests within reasonable time, we reduce the number of maximum successful replications required to 10 and when doing t-tests 10 replications are run there as well. 

\begin{verbatim}
SugarScape Tests
  Disease Dynamics All Recover:      OK (29.25s)
    +++ OK, passed 10 tests (100% Diseases all recover).
    
  Disease Dynamics Minority Recover: OK (536.00s)
    +++ OK, passed 10 tests (100% Diseases no recover).
    
  Trading Dynamics:                  OK (149.33s)
    +++ OK, passed 10 tests (70% Prices std less than 5.0e-2).
    Only 70% Prices std less than 5.0e-2, but expected 100%
    
  Cultural Dynamics:                 OK (996.84s)
    +++ OK, passed 10 tests (50% Cultures dominate or equal).
    Only 50% Cultures dominate or equal, but expected 100%
    
  Carrying Capacity:   OK (988.20s)
    +++ OK, passed 10 tests (90% Carrying capacity averages at 204.0).    
    Only 90% Carrying capacity averages at 204.0, but expected 100%
    
  Terracing:           OK (280.59s)
    +++ OK, passed 10 tests (80% Terracing is happening).
    Only 80% Terracing is happening, but expected 100%
    
  Inheritance Gini:    OK (7232.59s)
    +++ OK, passed 0 tests (0% Gini coefficient averages at 0.7).
    Only 0% Gini coefficient averages at 0.7, but expected 100%
\end{verbatim}

%\begin{enumerate}
%	\item Disease Dynamics all recover: \textit{+++ OK, passed 10 tests.}
%
%	\item Disease Dynamics minority recover: \textit{+++ OK, passed 10 tests.}
%		
%	\item Trading Dynamics: \textit{+++ OK, passed 10 tests; 2 failed (16\%).} (In total 12 tests (replications) were run, out of which 2 failed, which is a 16\% failure rate.)
%	
%	\item Cultural Dynamics: \textit{+++ OK, passed 10 tests; 3 failed (23\%).}
%
%	\item Inheritance Gini Coefficient: \textit{*** Failed! Passed only 0 tests; 10 failed (100\%) tests.}
%
%	\item Carrying Capacity: \textit{+++ OK, passed 10 tests; 2 failed (16\%).}
%
%	\item Terracing: \textit{+++ OK, passed 10 tests; 2 failed (16\%).}
%\end{enumerate}

How to deal with the failure of hypotheses is obviously highly model specific. A first approach is to increase the number of replications to run to 100 to get a more robust estimate of the failure rate. If the failure rate stays within reasonable ranges then one can arguably assume that the hypothesis is valid for sufficiently enough cases. On the other hand, if the failure rate escalates, then it is reasonable to deem the hypothesis invalid and refine it or even abandon it altogether.

With the exception of the Gini coefficient, we accept the failure rate of the hypotheses we presented here and deem them sufficiently valid for the task at hand. In case of the Gini coefficient, none of the replication was successful, which makes it obvious that it does \textit{not} average at 0.7. Thus the hypothesis as stated in the book does not hold and is invalid. One way to deal with it would be to simply delete it. Another, more constructive approach, is to keep it but require all replications to fail by marking it with \texttt{expectFailure} instead of \texttt{property}. In this way an invalid hypothesis is marked explicitly and acts as documentation and also as regression test.

\section{Hypotheses and test cases}
\label{sec:hypotheses_testcases}

In this section we briefly describe the process of validating our Sugarscape implementation against the specification of the Sugarscape book \cite{epstein_growing_1996} and the work of \cite{weaver_replicating_2009}.

\subsection{Terracing}
Our implementation reproduces the terracing phenomenon as described in the book and as can be seen in the NetLogo implementation as well. We implemented a property test in which we measure the closeness of agents to the ridge: counting the number of same-level sugars cells around them and if there is at least one lower then they are at the edge. If a certain percentage is at the edge then we accept terracing. The question is just how much, which we estimated from tests and resulted in 45\%. Also, in the terracing animation the agents actually never move which is because sugar immediately grows back thus there is no incentive for an agent to actually move after it has moved to the nearest largest cite in can see. Therefore we test that the coordinates of the agents after 50 steps are the same for the remaining steps.

\subsection{Carrying capacity}
Our simulation reached a steady state (variance $<$ 4 after 100 steps) with a mean around ~182. Epstein reported a carrying capacity of 224 (page 30) and the NetLogo implementations' \cite{weaver_replicating_2009} carrying capacity fluctuates around 205 which both are significantly higher than ours. Something was definitely wrong - the carrying capacity has to be around 200 (we trust in this case the NetLogo implementation and deem 224 an outlier).

After inspection of the NetLogo model we realised that we implicitly assumed that the metabolism range is \textit{continuously} uniformly randomized between 1 and 4 but this seemed not what the original authors intended: in the NetLogo model there were a few agents surviving on sugar level 1 which was never the case in ours as the probability of drawing a metabolism of exactly 1 is practically zero when drawing from a continuous range. We thus changed our implementation to draw a discrete value as the metabolism. %Note that this actually makes sense as massive floating-point number calculations were quite expensive in the mid 90s (e.g. computer games ran still on CPU only and exploited various  clever tricks to avoid the need of floating point calculations whenever possible) when SugarScape was implemented which might have been a reason for the authors to assume it implicitly.

This partly solved the problem, the carrying capacity was now around 204 which is much better than 182 but still a far cry from 210 or even 224. After adjusting the order in which agents apply the Sugarscape rules, by looking at the code of the NetLogo implementation, we arrived at a comparable carrying capacity of the NetLogo implementation: agents first make their move and harvest sugar and only after this the agents metabolism is applied (and ageing in subsequent experiments).

For regression tests we implemented a property test which tests that the carrying capacity of 100 simulation runs lies within a 95\% confidence interval of a 210 mean. These values are quite reasonable to assume, when looking at the NetLogo implementation - again we deem the reported carrying capacity of 224 in the book to be an outlier / part of other details we don't know.

One lesson learned is that even such seemingly minor things like continuous vs. discrete or order of actions an agent makes, can have substantial impact on the dynamics of a simulation.

\subsection{Wealth distribution}
By visual comparison we validated that the wealth distribution (page 32-37) becomes strongly skewed with a histogram showing a fat tail, power-law distribution where very few agents are very rich and most of the agents are quite poor. We compute the skewness and kurtosis of the distribution which is around a skewness of 1.5, clearly indicating a right skewed distribution and a kurtosis which is around 2.0 which clearly indicates the 1st histogram of Animation II-3 on page 34. Also we compute the Gini coefficient and it varies between 0.47 and 0.5 - this is accordance with Animation II-4 on page 38 which shows a gini-coefficient which stabilises around 0.5 after. 
We implemented a regression-test testing skewness, kurtosis and gini coefficients of 100 runs to be within a 95\% confidence interval of a two-sided t-test using an expected skewness of 1.5, kurtosis of 2.0 and gini coefficient of 0.48.

\subsection{Migration}
With the information provided by \cite{weaver_replicating_2009} we could replicate the waves as visible in the NetLogo implementation as well. Also we propose that a vision of 10 is not enough yet and shall be increased to 15 which makes the waves very prominent and keeps them up for much longer - agent waves are travelling back and forth between both Sugarscape peaks. We have not implemented a regression test for this property as we couldn't come up with a reasonable straightforward approach to implement it.

\subsection{Pollution and diffusion}
With the information provided by \cite{weaver_replicating_2009} we could replicate the pollution behaviour as visible in the NetLogo implementation as well. We have not implemented a regression test for this property as we couldn't come up with a reasonable straightforward approach to implement it.

%Note that we spent quite a lot of time of getting this and the terracing properties right because they form the very basics of the other ones which follow so we had to be sure that those were correct otherwise validating would have been much more difficult.

%\section{Order of Rules}
%order in which rules are applied is not specified and might have an impact on dynamics e.g. when does the agent mate with others: is it after it has harvested but before metabolism kicks in?

\subsection{Mating}
We could not replicate Figure III-1 - our dynamics first raised and then plunged to about 100 agents and go then on to recover and fluctuate around 300. This findings are in accordance with \cite{weaver_replicating_2009}, where they report similar findings - also when running their NetLogo code we find the dynamics to be qualitatively the same.

Also at first we weren't able to reproduce the cycles of population sizes. Then we realised that our agent behaviour was not correct: agents which died from age or metabolism could still engage in mating before actually dying - fixing this to the behaviour, that agents which died from age or metabolism will not engage in mating solved that and produces the same swings as in \cite{weaver_replicating_2009}. Although our bug might be obvious, the lack of specification of the order of the application of the rules is an issue in the SugarScape book.

\subsection{Inheritance}
We couldn't replicate the findings of the Sugarscape book regarding the Gini coefficient with inheritance. The authors report that they reach a gini coefficient of 0.7 and above in Animation III-4. Our Gini coefficient fluctuated around 0.35. Compared to the same configuration but without inheritance (Animation III-1) which reached a Gini coefficient of about 0.21, this is indeed a substantial increase - also with inheritance we reach a larger number of agents of around 1,000 as compared to around 300 without inheritance.
The Sugarscape book compares this to chapter II, Animation II-4 for which they report a Gini coefficient of around 0.5 which we could reproduce as well. The question remains, why it is lower (lower inequality) with inheritance?

The baseline is that this shows that inheritance indeed has an influence on the inequality in a population. Thus we deemed that our results are qualitatively the same as the make the same point. Still there must be some mechanisms going on behind the scenes which are unspecified in the original Sugarscape.

\subsection{Cultural dynamics}
We could replicate the cultural dynamics of AnimationIII-6 / Figure III-8: after 2700 steps either one culture (red / blue) dominates both hills or each hill is dominated by a different ulture. We wrote a test for it in which we run the simulation for 2.700 steps and then check if either culture dominates with a ratio of 95\% or if they are equal dominant with 45\%. Because always a few agents stay stationary on sugarlevel 1 (they have a metabolism of 1 and cant see far enough to move towards the hills, thus stay always on same spot because no improvement and grow back to 1 after 1 step), there are a few agents which never participate in the cultural process and thus no complete convergence can happen. This is accordance with \cite{weaver_replicating_2009}.

\subsection{Combat}
Unfortunately \cite{weaver_replicating_2009} didn't implement combat, so we couldn't compare it to their dynamics. Also, we weren't able to replicate the dynamics found in the Sugarscape book: the two tribes always formed a clear battlefront where some agents engage in combat, for example when one single agent strays too far from its tribe and comes into vision of the other tribe it will be killed almost always immediately. This is because crossing the sugar valley is costly: this agent wont harvest as much as the agents staying on their hill thus will be less wealthy and thus easier killed off. Also retaliation is not possible without any of its own tribe anywhere near.

We didn't see a single run where an agent of an opposite tribe "invaded" the other tribes hill and ran havoc killing off the entire tribe. We don't see how this can happen: the two tribes start in opposite corners and quickly occupy the respective sugar hills. So both tribes are acting on average the same and also because of the number of agents no single agent can gather extreme amounts of wealth - the wealth should rise in both tribes equally on average. Thus it is very unlikely that a super-wealthy agent emerges, which makes the transition to the other side and starts killing off agents at large. First: a super-wealthy agent is unlikely to emerge, second making the transition to the other side is costly and also low probability, third the other tribe is quite wealthy as well having harvested for the same time the sugar hill, thus it might be that the agent might kill a few but the closer it gets to the center of the tribe the less like is a kill due to retaliation avoidance - the agent will simply get killed by others.

Also it is unclear in case of AnimationIII-11 if the R rule also applies to agents which get killed in combat. Nothing in the book makes this clear and we left it untouched so that agents who only die from age (original R rule) are replaced. This will lead to a near extinction of the whole population quite quickly as agents kill each other off until 1 single agent is left which will never get killed in combat because there are no other agents who could kill it - instead it will enter an infinite die and  reborn cycle thanks to the R rule.

\subsection{Spice}
The book specifies for AnimationIV-1 a vision between 1-10 and a metabolism between 1-5. The last one seems to be quite strange because the maximum sugar / spice an agent can find is 4 which means that agents with metabolism of either 5 will die no matter what they do because the can never harvest enough to satisfy their metabolism. When running our implementation with this configuration the number of agents quickly drops from 400 to 105 and continues to slowly degrade below 90 after around 1000 steps.
The implementation of \cite{weaver_replicating_2009} used a slightly different configuration for AnimationIV-1, where they set vision to 1-6 and metabolism to 1-4. Their dynamics stabilise to 97 agents after around 500+ steps. When we use the same configuration as theirs, we produce the same dynamics.
Also it is worth nothing that our visual output is strikingly similar to both the book AnimationIV-1 and \cite{weaver_replicating_2009}.

\subsection{Trading}
For trading we had a look at the NetLogo implementation of \cite{weaver_replicating_2009}: there an agent engages in trading with its neighbours \textit{over multiple rounds} until either MRSs cross over or no trade has happened anymore. Because \cite{weaver_replicating_2009} were able to exactly replicate the dynamics of the trading time series we assume that their implementation is correct. We think that the fact that an agent interact with its neighbours over multiple rounds is made not very clear in the book. The only hint is found on page 102: \textit{"This process is repeated until no further gains from trades are possible."} which is not very clear and does not specify exactly what is going on: does the agent engage with all neighbours again? is the ordering random? Another hint is found on page 105 where trading is to be stopped after MRS crossover to prevent an infinite loop. Unfortunately this is missing in the Agent trade rule T on page 105. Additional information on this is found in footnote 23 on page 107. Further on page 107: \textit{"If exchange of the commodities will not cause the agents' MRSs to cross over then the transaction occurs, the agents recompute their MRSs, and bargaining begins anew."}. This is probably the clearest hint that trading could occur over multiple rounds.

We still managed to exactly replicate the trading dynamics as shown in the book in Figure IV-3, Figure IV-4 and Figure IV-5. The book is also pretty specific on the dynamics of the trading prices standard deviation: on page 109 the authors specify that at t=1000 the standard deviation will have always fallen below 0.05 (Figure IV-5), thus we implemented a property test which tests for exactly that property. Unfortunately we didn't reach the same magnitude of the trading volume where ours is much lower around 50 but it is equally erratic, so we attribute these differences to other missing specifications or different measurements because the price dynamics match that well already so we can safely assume that our trading implementation is correct.

According to the book, Carrying Capacity (Animation II-2) is increased by Trade (page 111/112). To check this it is important to compare it not against AnimationII-2 but a variation of the configuration for it where spice is enabled, otherwise the results are not comparable because carrying capacity changes substantially when spice is on the environment and trade turned off. We could replicate the findings of the book: the carrying capacity increases slightly when trading is turned on. Also does the average vision decrease and the average metabolism increase. This makes perfect sense: trading allows genetically weaker agents to survive which results in a slightly higher carrying capacity but shows a weaker genetic performance of the population.

According to the book, increasing the agent vision leads to a faster convergence towards the (near) equilibrium price (page 117/118/119, Figure IV-8 and Figure IV-9). We could replicate this behaviour as well.

According to the book, when enabling R rule and giving agents a finite life span between 60 and 100 this will lead to price dispersion: the trading prices will not converge around the equilibrium and the standard deviation will fluctuate wildly (page 120, Figure IV-10 and Figure IV-11). We could replicate this behaviour as well.

The Gini coefficient should be higher when trading is enabled (page 122, Figure IV-13) - We could replicate this behaviour.

Finite lives with sexual reproduction lead to prices which don't converge (page 123, Figure IV-14). We could reproduce this as well but it was important to set the parameters to reasonable values: increasing number of agents from 200 to 400, metabolism to 1-4 and vision to 1-6, most important the initial endowments back to 5-25 (both sugar and spice) otherwise hardly any mating would happen because the agents need too much wealth to engage (only fertile when have gathered more than initial endowment). What was kind of interesting is that in this scenario the trading volume of sugar is substantially higher than the spice volume - about 3 times as high. 

From this part, we didn't implement: Effect of Culturally Varying Preferences, page 124 - 126, Externalities and Price Disequilibrium: The effect of Pollution, page 126 - 118, On The Evolution of Foresight page 129 / 130. 

%\section{Lending (Credit)}
%Not really much information to validate was available and the \cite{weaver_replicating_2009} implementation ran into an exception so there was not much to validate against. What was unexpected was that this was the most complex behaviour to implement, with lots of subtle details to take care of (spice on/off, inheritance,...).
%Note that we implemented lending of sugar and spice, although it looks from the book (Animation IV-5) that they only implemented it for sugar.

\subsection{Diseases}
We were able to exactly replicate the behaviour of Animation V-1 and Animation V-2: in the first case the population rids itself of all diseases (maximum 10) which happens pretty quickly, in less than 100 ticks. In the second case the population fails to do so because of the much larger number of diseases (25) in circulation. We used the same parameters as in the book. 
The authors of \cite{weaver_replicating_2009} could only replicate the first animation exactly and the second was only deemed "good". Their implementation differs slightly from ours: In their case a disease can be passed to an agent who is immune to it - this is not possible in ours. In their case if an agent has already the disease, the transmitting agent selects a new disease, the other agent has not yet - this is not the case in our implementation and we think this is unreasonable to follow: it would require too much information and is also unrealistic.
We wrote regression tests which check for animation V-1 that after 100 ticks there are no more infected agents and for animation V-2 that after 1000 ticks there are still infected agents left and they dominate: there are more infected than recovered agents.

\section{Discussion}
In this appendix we showed how to use QuickCheck to formalise and check hypotheses about an \textit{exploratory} agent-based model, in which no ground truth exists. Due to ABS stochastic nature in general it became obvious that to get a good measure of a hypotheses validity we need to emulate failure using the \texttt{cover} function of QuickCheck. This allowed us to show that the hypotheses we have presented are sufficiently valid for the task at hand and can indeed be used for expressing and formalising emergent properties of the model and also as regression tests within a TDD cycle.

%What is particularly powerful is that one has complete control and insight over the changed state before and after e.g. a function was called on an agent: thus it is very easy to check if the function just tested has changed the agent-state itself or the environment: the new environment is returned after running the agent and can be checked for equality of the initial one - if the environments are not the same, one simply lets the test fail. This behaviour is very hard to emulate in OOP because one can not exclude side-effect at compile time, which means that some implicit data-change might slip away unnoticed. In FP we get this for free.

\end{appendices}

\end{document}
