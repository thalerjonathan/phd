\subsection{Actor Model}
The Actor-Model, a model of concurrency, has been around since the paper \cite{hewitt_universal_1973} in 1973. It was a major influence in designing the concept of agents and although there are important differences between actors and agents there are huge similarities thus the idea to use actors to build agent-based simulations comes quite natural. TODO: short explain the similarities / differences

The theory was put on firm semantic grounds first through Irene Greif by defining its operational semantics \cite{greif_semantics_1975} and then Will Clinger by defining denotational semantics \cite{clinger_foundations_1981}. In the seminal work of Agha \cite{agha_actors:_1986} he developed a programming model for the actor-model which he termed \textit{actors}.

TODO \cite{agha_foundation_1997}
TODO \cite{hewitt_what_2007}
TODO \cite{hewitt_actor_2010}
TODO \cite{agha_algebraic_2004}

\subsubsection{Erlang}
The programming-model of actors \cite{agha_actors:_1986} was the inspiration for the Erlang programming language which was created in the 80s by Eriksson for developing distributed high reliability software in telecommunications. There exists very little research in using Erlang for ABS \cite{varela_modelling_2004}, \cite{di_stefano_using_2005}, \cite{di_stefano_exat:_2007}, \cite{sher_agent-based_2013}, \cite{Bezirgiannis2013}. TODO: short explanation of what they are doing

\subsubsection{Scala}
Scala is a multi-paradigm language which also comes with an implementation of the actor-model as a library which enables to do actor-programming in the way of Erlang. It was developed in 2004 and became popular in recent years due to the increased availability of multi-core CPUs which emphasised the distributed, parallel and concurrent programming for which the actor-model is highly suited.
As for Erlang, there exists even less research in using Scala \& Actors for ABS TODO: cite.  TODO: short explanation of what they are doing