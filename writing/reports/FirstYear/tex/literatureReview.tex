\section{Literature Review}
Literature Review
- trichter: mit den 3 themen beginnen und dann runterbrechen und ins detail gehen, bis der gap gefunden wurde

% MAJOR TOPIC
\subsection{Agent-Based Modelling \& Simulation}
 wooldridge,  will clinger, hewitt

TODO: baas: emergence, hierarchies and hyperstructures
TODO: \cite{baas_emergence_1997}

% SUB TOPIC
\subsubsection{Social Simulation}
The SugarScape model \cite{epstein_growing_1996} is one of the most influential models of agent-based simulation in the social sciences. The book heavily promotes object-oriented programming (note that in 1996 oop was still in its infancy and not yet very well understood by the mainstream software-engineering industry). We ask how it can be done using pure functional programming paradigm and what the benefits and limits are. We hypothesize that our solution will be shorter (original reported 20.000 LOC), can make use of EDSL thus making it much more expressive, can utilize QuickCheck for a completely new dimension of model-checking and debugging and allows a very natural implementation of MetaABS (see Part III) due to its recursive and declarative nature.

TODO \cite{huberman_evolutionary_1993} 
TODO \cite{nowak_evolutionary_1992}
% SUB TOPIC
\subsubsection{Computational Economics}
\cite{tesfatsion_agent-based_2006} gives a broad overview of agent-based computational economics (ACE), gives the four primary objectives of it and discusses advantages and disadvantages. She introduces a model called \textit{ACE Trading World} in which she shows how an artificial economy can be implemented without the \textit{Walrasian Auctioneer} but just by agents and their interactions. She gives a detailed mathematical specification in the appendix of the paper which should allow others to implement the simulation.

- Artificial agent-based economies: \cite{tesfatsion_agent-based_2006}, \cite{gintis_emergence_2006}, \cite{gintis_dynamics_2007}, \cite{gaffeo_adaptive_2008}, \cite{botta_functional_2011}
- Artificial agent-based markets: \cite{mackie-mason_chapter_2006}, \cite{darley_nasdaq_2007}
- Agent-Based Market Design: \cite{marks_chapter_2006}, \cite{budish_editors_2015}

market-microstructure: \cite{LehalleLaruelle2013}, \cite{baker_market_2013}

Basics of Economics \cite{bowles_understanding_2005}, \cite{kirman_complex_2010}

% SUB TOPIC
\subsection{Verification \& Validation}
model checking and reasoning by quickcheck: \cite{claessen_quickcheck:_2000}, \cite{hutton_tutorial_1999}

Verification/Reasoning ist einer der größten Pluspunkte von rein funktionaler Programmierung, da durch den deklarativen Stil und das Fehlen von Sideeffects und Globalen Daten equational/algebraic/inductive Reasoning betrieben werden kann. Hier habe ich noch garnichts dazu gemacht, aber sollte mit den oben genannten Ideen sicherlich interessant werden - ein interessantes Paper von Graham Hutton (für den ich übrigens dieses Semester ein Tutor in seiner Haskell-Laborübung bin) gibt interessante Richtungen für Reasoning vor: http://dl.acm.org/citation.cfm?id=968579

[ ] deadlock: when messages need to be exchanged but mutual waiting
[ ] silence: no more message exchange
[ ] protocoll: ensure happens before / sequences (like necessary for 2D prisoner dilemma)

% MAJOR TOPIC
\subsection{Functional Programming}
all FRP, quickcheck, arrows, monads, wadler, hughes

\subsubsection{EDSL}
EDSL steht für Embedded Domain Specific Language d.h. man implementiert in Haskell eine Art von 'Spezifikations-Sprache' für eine spezielle Domain (z.b. ABM/S), die - dank der rein funktionalen, deklarativen Natur von Haskell - auch gleichzeitig Haskell Code ist - der Unterschied zwischen Spezifikation und Implementierung verschwindet dann (idealserweise). In diese Richtung arbeite ich erst seit kurzem, durch die Umsetzung von ABS/M mit Yampa. Yampa ist ebenfalls eine EDSL um funktional-reaktive Systeme zu beschreiben/implementieren, ich werde auf dieser EDSL aufsetzen und sie um ABS/M erweitern - so zumindest der Plan. Dann habe ich die theoretische Grundlage von FRP, auf die ich dann auch theorie von ABS/M (z.b. Actor Semantics) setzen kann und somit zum nächsten Punkt komme:

% SUB TOPIC
\subsubsection{General principles}
% SUB TOPIC
\subsubsection{Structuring}
Monads
Arrows
Continuations
% SUB TOPIC
\subsubsection{Paradigm: FRP}
TODO: why Yampa? There are lots of other FRP-libraries for Haskell. Reason: in-house knowledge (Nilsson, Perez), start with \textit{some} FRP-library to get familiar with the concept and see if FRP is applicable to ABS. TODO: short overview over other FRP-libraries but leave a in-depth evaluation for further-research out of the scope of the PhD as Yampa seems to be suitable. One exception: the extension of Yampa to Dunai to be able to do FRP in Monads, something which will be definitely useful for a better and clearer structuring of the implementation.
TODO: Push vs. Pull

TODO: describe FRP

TODO: 1st year report Ivan: "FPR tries to shift the direction of data-flow, from message passing onto data dependency. This helps reason about what things are over time, as opposed to how changes propagate". QUESTION: Message-passing is an essential concept in ABS, thus is then FRP still the right way to do ABS or DO WE HAVE TO LOOK AT MESSAGE PASSING IN A DIFFERENT WAY IN FRP, TO VIEW AND MODEL IT AS DATA-DEPENDENCY? HOW CAN THIS BE DONE?
BUT: agent-relations in interactions are NEVER FIXED and always completely dynamic, forming a network. The question is: is there a mechanism in which we have explicit data-dependency but which is dynamic like message-passing but does not try to fake method-calls? maybe the conversations come very close


% MAJOR TOPIC
\subsection{Category- \& Type-Theory}
Category-Theory \cite{Pierce1991} \cite{spivak_category_2014}

include paper on arrows my hughes
apply category theory to agent-based simulation: how can a ABS system itself be represented in category theory and can we represent models in this category theory as well?
ADOM: Agent Domain of Monads: https://www.haskell.org/communities/11-2006/html/report.html
develop category theory behind FrABS: look into monads, arrows


\subsection{Identifying the Gap}
- Functional programming in this area exists but only scratches the surface and focus only on implementing agent-behaviour frameworks like BDI. An in-depth treatise of Agent-Based Modelling and implementing an Agent-Based Simulation in a pure functional language has so far never been attempted.

- There basically exists no approach to Agent-Based Modelling \& Simulation in terms of Category-Theory and Type-Theory

- Verification is an issue in ABS as they are very often described in natural language and supplemented with a few formulas. This leads to implementation-errors, e.g. Gintis Bartering-Paper, and results become hard to reproduce. Such errors become a threatening problem when simulation-results are used in decision making e.g. economics, policy-making, ...

- Validation is basically an untouched topic in ABS: models are formulated, a few hypotheses are formulated, the model is implemented and run, then the results are checked against the hypotheses. What the field of ABS needs is an in-depth discussion on how to rigorously validate a model. Validation is of course only as strong as the verification part: if the implementation is wrong anyway then we can not rely on anything (from false comes nothing)