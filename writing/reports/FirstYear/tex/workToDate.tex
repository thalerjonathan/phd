\chapter{Work To Date}
TODO: Describe the research work carried out during this stage of the PhD and the outcomes. A literature review must be included. Then, as appropriate according to the PhD project, this section can also include theoretical and/or experimental methods, presentation and discussion of results, etc. In the case that papers have been submitted or published within the year of the review, this section can be shorter and focused on discussing the outcomes from those papers within the wider context of the PhD programme of study (papers to be included in the appendix).

We began the search for oo alternatives initially with experimenting in Erlang and then ended up in haskell

\subsection{Papers Submitted}
\paragraph{Update-Strategies in ABS}
SSC 2017, Deadline in 31st March, 25-29th September 2017, \url{http://www.sim2017.com/}


TODO: attach as appendix

A foundational paper

\subsection{Paper Drafts}
\paragraph{Programming Paradigms and ABS}
SSC 2018, Deadline in March 2018?

TODO: attach as appendix

In this work I investigated the suitability of three fundamentally different programming pardigms to implement an ABS. The paradigms I looked at was object-oriented using Java, pure functional using Haskell and multi-paradigm functional using Scala with the Actors library. It is important to note that at this point I didn't use FRP as underlying paradigm in Haskell TODO: would this have changed my final conclusion on its suitability?

STM: the really unique thing which is ONLY possible in pure functional programming is composition of concurrency. TODO: cite Tim Sweeny
Actors in Scala

\subsubsection{Papers in Progress}
\paragraph{Recursive ABS}
dont pursue recursive ABS further as separate paper: it could take a whole PhD by itself. proof-of-concept is ok but not groundbreaking. maybe some really groundbreaking idea will come up in a few months, then can work on it again. incorporate it maybe in FrABS paper. definitely incorporate it in final thesis

basic work and prototype within Schelling Segregation is running. The main finding for now is that it does not increase the convergence speed to equilibrium but can lead to extreme volatility of dynamics although the system seems to be near to complete equilibrium. In the case of a 10x10 field it was observed that although the system was nearly in its steady state - all but one agents were satisfied - the move of a single agent caused the system to become completele unstable and depart from its near-equilibrium state to a highly desequilibrium state.

Give each  Agent the ability to run the simulation locally from its point of view do anticipate its actions and change them in the future thus introducing a meta-level in the simulation, from which the method derives its name.

- TODO:  i have only the idea but am lacking a theory or hypothesis for its use

- meta need a kind of decision error measure to distinguish between various meta-simulations. also we need a mechanism to sample the decision space => it can be considered to be an optimization technique.

Problems
\begin{itemize}
	\item Definition of a recursive, declarative description of the Model.
	\item Perfect information about other agents is not realistic and runs counter to agent-based simulation (especially in social sciences) thus an Agent needs to be able to have local, noisy representations of the other agents.
	\item Local representation of other agents could be captured by Hidden Markov Models: observe what other agents do but have hidden interpretation of their internal state - these internal state-representations can be different between the local and the global version whereas the agent learns to represent the global version as best as possible locally.
	\item Infinite regress is theoretically possible but not on computers, we need to terminate at some point
\end{itemize}

Interpretation: It can be regarded as a Model of Free Will in ABS, which allows learning in an ABS environment in a new way - look on the section of interpretation.
Application: hypothesis: allows to model social and psychological phenomena like free will. Mostly in social sciences, maybe also in economics. Investigate SugarScape, PrisonersDilemma and ACE Trading World

TODO: question: what is the meaning of an entity running simulations? it strongly depends on the context: in ACE it may be search for optimization behaviour, in Social Simulation it may be interpreted as a kind of free will

Research Questions
\begin{enumerate}
	\item How does deep regression influence the dynamics of a system? Hypothesis: TODO
	\item How do the dynamics of a system change when using perfect information or learning local information? Hypothesis: TODO
	\item Is a hidden markov model suitable for the local learning? Hypothesis: TODO
	\item How can MetaABS best be implemented? Hypothesis: implementing a MetaABS EDSL in a pure functional language like Haskell, should be best suited due to its inherent recursive, declarative nature, which should allow a direct mapping of features of this paradigm to the specification of the meta-model
\end{enumerate}

- functional programming perfect. standard toolkits (anylogic, netlogo, repast) are not capable of doing this
- extend my existing EDSL for functional reactive agent-based simulation \& modelling (FrABS/M) with recursive functionality
 
Related Research:
TODO: \cite{gilmer_recursive_2000} cite paper of recursive simulation: [ ] military simulation, [ ] not explicitly abs, [ ] implemented in c++, [ ] deterministic models seem to benefit significantly from using recursions of the simulation for the decision making process. when using stochastic models this benefit seems to be lost


\paragraph{Functional Reactive ABS}
TODO: put into separate chapter as this is the fundamental approach put forward in this thesis
TFP 2017, Deadline 5th May 2017, 19-21th June 2017, \url{https://www.cs.kent.ac.uk/events/tfp17/index.html}


\subsubsection{Software}
TODO: make a better, more detailed list of what prototyping i did

Lots of prototyping:
Heroes \& Cowards, SIRS \& Schelling Segregation in Java, Haskell and Scala
Parallelism and Concurrency in Haskell

Lots of learning \& prototyping for Haskell:
IO, STM, Pure Functional, FRP

FrABS: SugarScape Model as use-case no.1. TODO: available on github

\subsection{Talks}
presenting the ideas of my Update-Strategies paper at the IMA - seminar day
presenting my FrABS ideas to the FP-Lab Group at the FPLunch
