\chapter{Agent Updating}

\section{OO}
After creating the Agents one ends up with a collection of Agents, represented either as a List, a Vector or a Map. In OO updating is pretty trivial: one iterates over the collection and calls some update-method of the Agent objects. This implies that if one uses inheritance and has a general Agent-Class, this class needs to provide an update-method which feeds a time-delta.
When implementing the parallel-strategy things become complicated in OO though. Changes must only be visible in the next iteration. This can only be achieved by either messaging instead of method-calls or creating new Agent-objects after every iteration.

\section{FP}
In FP after the construction phase one also ends up with a collection of Agents either a list or a Map. Updating in FP is more subtle because it lacks references and mutable data. In case of the sequential strategy more work needs to be done and we can see the problem in general as a fold over the list of agents. In the case of the parallel strategy we can directly make use of FPs immutability.