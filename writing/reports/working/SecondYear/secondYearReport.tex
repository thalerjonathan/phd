\documentclass[oneside]{book}

\setcounter{tocdepth}{1}
\setcounter{secnumdepth}{3}

\usepackage[toc,page]{appendix}
\usepackage{minted}
\usepackage[english]{babel}
\usepackage{graphicx}
\usepackage{hyperref}
\usepackage{amsmath} % Required for some math elements 
\usepackage{pdflscape}
\usepackage{pdfpages}

\begin{document}

\begin{titlepage}
	\centering
	\includegraphics[width=0.60\textwidth]{./logo/UoN_Primary_Logo_RGB.png}\par\vspace{1cm}
	{\scshape\Large 2nd Year Report\par}
	\vspace{1.5cm}
	{\huge\bfseries Pure Functional \\ Agent-Based Simulation\par}
	\vspace{2cm}
	{\Large\itshape Jonathan Thaler (4276122) \\ jonathan.thaler@nottingham.ac.uk \par}
	\vfill
	supervised by\par
	Dr. Peer-Olaf \textsc{Siebers} \\
	Dr. Thorsten \textsc{Altenkirch}

	\vfill

	{\large \today\par}
\end{titlepage}

\cleardoublepage

\section*{Abstract}
TODO: we need to distinguish between various meanings of 'correct'
1. correctness of software: the implementation is correct up to a model specification
2. correctness of a simulation: the implementation is correct up to a model specification AND it generates the same dynamics

This Ph.D. investigates how Agent-Based Simulations (ABS) can be implemented using the functional programming paradigm and what the benefits and drawbacks are when doing so. Due to the nature of the functional paradigm we hypothesize that by using this approach we can increase the correctness and gain insights into dynamics of the simulation to on an unprecedented level not possible with the established object-oriented approaches in the field. The correctness of a simulation and its results is of paramount interest in scientific computing thus making our research high impact.

So far we researched \textit{how} to do ABS using the functional paradigm, where we implemented a highly promising approach by building on Functional Reactive Programming using the library Yampa and generalising it to Monadic Stream Functions. By this we could show that ABS is indeed very possible in functional programming as it allowed us to implement a number of different agent-based models, incorporating discrete time-semantics similar to Discrete Event Simulation and continuous time-flows like System Dynamics.

During the research conducted so far, it became apparent that this approach exhibits a few unique properties which indeed supports our initial hypothesis. We have started to systematically explore these properties and will commit our research of the remaining year to develop this fully.

\clearpage
\tableofcontents
\clearpage

%*******************************************************************************
%*********************************** First Chapter *****************************
%*******************************************************************************

\chapter{Introduction}  %Title of the First Chapter
I noticed that it is pretty hard to convince an agent-based economics specialist who is not a computer scientist about a pure functional approach. My conjecture is that the implementation technique and method does not matter much to them because they have very little knowledge about programming and are almost always self-taught - they don't know about software-engineering, nothing about proper software-design and architecture, nothing about software-maintenance, nothing about unit-testing,... In the end they just "hack" the simulation in whatever language they are able to: C++, Visual Basic, Java or toolboxes like Netlogo. For them it is all about to \textit{get things done somehow} and not to get things done the right way or in a beautiful way - the way and the method doesn't matter, its just a necessary evil which needs to be done. Thus if functional programming could make their lives easier, then they will definitely welcome it. But functional programming is, i think, harder to learn and harder to understand - so one needs to provide an abstraction through EDSL. So I REALLY need to come up with convincing arguments why to use pure functional approaches in ACE THEY can understand, otherwise I will be lost and not heard (not published,...). \\

What ACE economists care for:

\begin{itemize}
\item Very: Qualitative modelling with quantitative results
\item Yes: Easy reproducibility
\item Likely: Reasoning about convergence?
\item Likely: EDSL
\end{itemize}

My contributions are: pure functional framework, functional agent-model for market-simulations, EDSL for market-simulations, qualitative / implicit modelling with quanitative results, reasoning in my framework about convergence \\

IDEA: could I develop non-causal modelling (models are expressed in terms of non-directed equations, modelled in signal-relations) to allow for qualitative modelling for the agent-based economists? See hybrid modelling paper of Yampa. \textbf{THIS WOULD BE A HUGE NOVEL CONTRIBUTION TO ACE ESPECIALLY WHEN COMBINED WITH AN EDSL AND PROVIDING FULL REFERENTIAL TRANSPARENCY TO KEEP THE ABILITY TO REASON ABOUT CONVERGENCE}. This should be covered in the "EDSL"-paper.

TODO: maybe i should really focus only on market models? otherwise too much? \\

central novelty of my PhD: model specification = runnable code. possible through EDSL. but only in specific subfield of ACE: market-models. need a functional description of the model, then translate it to model specification in EDSL and then run it to see dynamics. But: model specification moves closer to functional programming languages. \\

another novelty approach: model specification through qualitative instead of quantiative approaches. is this possible? \\

WHY FUNCTIONAL? "because its the ultimate approach to scientific computing": fewer bugs due to mutable state (why? is thos shown obkectively by someone?), shorter (again as above, productivity), more expressive and closer to math, EDSL, EDSL=model=simulation, better parallelising due to referental transparency, reasoning \\

scientific results need to be reproduced, especially when they have high impact. a more formal approach of specifying the model and the simulation (model=simulation) could lead to easier sharing and easier reporduction without ambigouites \\

pure functional agent-model \& theory, EDSL framework in Haskell for ACE

\begin{enumerate}
\item Which kind of problem do we have?
\item What aim is there? Solving the problem? 
\item How the aim is achieved by enumerating VERY CLEAR objectives.
\item What the impact one expects (hypothesis) and what it is (after results).
\end{enumerate}

Note: It is not in the interest of the researcher to develop new economic theories but to research the use of functional methods (programming and specification) in agent-based computational economics (ACE).

NOTE: Get the reader’s attention early in the introduction: motivation, significance, originality and novelty.

\section{Methods}
Methods need to be selected to implement the simulations. Special emphasis will be put on functional ones which will then be compared to established methods in the field of ABM/S and ACE. \\

Claim: non-programming environments are considered to be not powerful enough to capture the complexity of ACE implementations thus a programming approach to ACE will be always required.

\section{Scenarios}
To apply and test functional methods in ACE, four scenarios of ACE are selected and then the methods applied and compared with each other to see how each of them perform in comparison. The 4 selected scenarios represent a selection of the challenges posed in ACE: from very abstract ones to very operational ones.

\section{Comparison}
Each of the selected scenarios is then implemented using the selected methods where each solution is then compared against the following criteria: 

\begin{enumerate}
\item suitability for scientific computation
\item robustness
\item error-sources
\item testability
\item stability
\item extendability
\item size of code
\item maintainability
\item time taken for development
\item verification \& correctness
\item replications \& parallelism
\item EDSL
\end{enumerate}

This will then allow to compare the different methods against each other and to show under which circumstances functional methods shine and when they should not be used.

\section{Agent-Based Modelling and Simulation (ABM/S)}
ABM/S is a method of modelling and simulating a system where the global behaviour may be unknown but the behaviour and interactions of the parts making up the system is of knowledge (Wooldrige, M. (2009). An Introduction to MultiAgent Systems. John Wiley & Sons). Those parts, called agents, are modelled and simulated out of which then the aggregate global behaviour of the whole system emerges. Thus the central aspect of ABM/S is the concept of an Agent which can be understood as a metaphor for a pro-active unit, able to spawn new Agents, and interacting with other Agents in a network of neighbours by exchange of messages. The implementation of Agents can vary and strongly depends on the programming language and the kind of domain the simulation and model is situated in.

\section{Agent-Based Economics (ACE)}
According to Leigh Tesfatsion (Tesfatsion, L. (2006). Agent-based computational economics: A constructive approach to economic theory. In Tesfatsion, L. and Judd, K. L., editors, Handbook of Computational Economics, volume 2, chapter 16, pages 831–880. Elsevier, 1 edition.), one of the leading figures, ACE is "[...] computational modelling of economic processes (including whole economies) as open-ended dynamic systems of interacting agents." - thus lending perfectly to the use of ABM/S as already the name suggests. Whereas classical economic models fall short by only looking at the average, pure rational, individual interacting in anonymous markets, the ACE approach looks at heterogeneous, non-rational individuals interacting with each other in networks (Kirman, A. (2010). Complex Economics: Individual and Collective Rationality. Routledge, London ; New York, NY.). Thus ACE can be understood as a combination of computer-science, cognitive/social science and evolutionary economics.

\section{Functional programming}
TODO: read \cite{Backus1978}

The state-of-the-art approach to implementing Agents are object-oriented methods and programming as the metaphor of an Agent as presented above lends itself very naturally to object-orientation (OO). The author of this thesis claims that OO in the hands of inexperienced or ignorant programmers is dangerous, leading to bugs and hardly maintainable and extensible code. The reason for this is that OO provides very powerful techniques of organising and structuring programs through Classes, Type Hierarchies and Objects, which, when misused, lead to the above mentioned problems. Also major problems, which experts face as well as beginners are 1. state is highly scattered across the program which disguises the flow of data in complex simulations and 2. objects don’t compose as well as functions. The reason for this is that objects always carry around some internal state which makes it obviously much more complicated as complex dependencies can be introduced according to the internal state.
All this is tackled by (pure) functional programming which abandons the concept of global state, Objects and Classes and makes data-flow explicit. This then allows to reason about correctness, termination and other properties of the program e.g. if a given function exhibits side-effects or not. Other benefits are fewer lines of code, easier maintainability and ultimately fewer bugs thus making functional programming the ideal choice for scientific computing and simulation and thus also for ACE. A very powerful feature of functional programming is Lazy evaluation. It allows to describe infinite data-structures and functions producing an infinite stream of output but which are only computed as currently needed. Thus the decision of how many is decoupled from how to (Hughes, J. (1989). Why functional programming matters. Comput. J., 32(2):98–107.).
The most powerful aspect using pure functional programming however is that it allows the design of embedded domain specific languages (EDSL). In this case one develops and programs primitives e.g. types and functions in a host language (embed) in a way that they can be combined. The combination of these primitives then looks like a language specific to a given domain, in the case of this thesis ACE. The ease of development of EDSLs in pure functional programming is also a proof of the superior extensibility and composability of pure functional languages over OO (Henderson P. (1982). Functional Geometry. Proceedings of the 1982 ACM Symposium on LISP and Functional Programming.).
One of the most compelling example to utilize pure functional programming is the reporting of Hudak (Hudak P., Jones M. (1994). Haskell vs. Ada vs. C++ vs. Awk vs. ... An Experiment in Software Prototyping Productivity. Department of Computer Science, Yale University.)  where in a prototyping contest of DARPA the Haskell prototype was by far the shortest with 85 lines of code. Also the Jury mistook the code as specification because the prototype did actually implement a small EDSL which is a perfect proof how close EDSL can get to and look like a specification.

Functional languages can best be characterized by their way computation works: instead of \textit{how} something is computed, \textit{what} is computed is described. Thus functional programming follows a declarative instead of an imperative style of programming. The key points are:
\begin{itemize}
\item No assignment statements - variables values can never change once given a value.
\item Function calls have no side-effect and will only compute the results - this makes order of execution irrelevant, as due to the lack of side-effects the logical point in \textit{time} when the function is calculated within the program-execution does not matter.
\item higher-order functions
\item lazy evaluation
\item Looping is achieved using recursion, mostly through the use of the general fold or the more specific map.
\item Pattern-matching
\end{itemize}

This alone does not really explain the \textit{real} advantages of functional programming and one must look for better motivations using functional programming languages. One motivation is given in \cite{Hughes1989} which is a great paper explaining to non-functional programmers what the significance of functional programming is and helping functional programmers putting functional languages to maximum use by showing the real power and advantages of functional languages. The main conclusion is that \textit{modularity}, which is the key to successful programming, can be achieved best using higher-order functions and lazy evaluation provided in functional languages like Haskell. \cite{Hughes1989} argues that the ability to divide problems into sub-problems depends on the ability to glue the sub-problems together which depends strongly on the programming-language and \cite{Hughes1989} argues that in this ability functional languages are superior to structured programming.

TODO: comparison of functional and object-oriented programming. My points are:
\begin{itemize}
\item The way state can be changed and treated - distributed over multiple objects - is often very difficult to understand.
\item Inheritance is a dangerous thing if not used with care because inheritance introduces very strong dependencies which cannot be changed during runtime anymore.
\item Objects don't compose very well: \url{http://zeroturnaround.com/rebellabs/why-the-debate-on-object-oriented-vs-functional-programming-is-all-about-composition/}
\item (Nearly) impossible to reason about programs
\end{itemize}

In conclusion the upsides of functional programming as opposed to OO are:
\begin{itemize}
\item Much more explicit flow of data \& control
\item Much better compose-able
\item Much better parallelism
\end{itemize}

\section{Related Research}
Tim Sweeney, CTO of Epic Games gave an invited talk about how "future programming languages could help us write better code" by "supplying stronger typing, reduce run-time failures;  and the need for pervasive concurrency support, both implicit and explicit, to effectively exploit the several forms of parallelism present in games and graphics." \cite{Sweeney2006}. Although the fields of games and agent-based simulations seem to be very different in the end, they have also very important similarities: both are simulations which perform numerical computations and update objects - in games they are called "game-objects" and in abm they are called agents but they are in fact the same thing - in a loop either concurrently or sequential. His key-points were:

\begin{itemize}
\item Dependent types as the remedy of most of the run-time failures.
\item Parallelism for numerical computation: these are pure functional algorithms, operate locally on mutable state. Haskell ST, STRef solution enables encapsulating local heaps and mutability within referentially transparent code.
\item Updating game-objects (agents) concurrently using STM: update all objects concurrently in arbitrary order, with each update wrapped in atomic block - depends on collisions if performance goes up.
\end{itemize}

\chapter{Verification \& Validation}
\label{chap:v_and_v}

In this chapter we will have a closer look on the topic of Verification \& Validation. It is of most importance to understand the ideas and concepts behind it, when we are referring to the \textit{correctness of a simulation}, as in our initial hypothesis. 

We first will give a short and concise introduction \ref{sec:vav_introduction} on the topic \footnote{In this short introduction we closely follow the book \cite{robinson_simulation:_2014}} and then briefly discuss verification \& validation in the context of agent-based simulation \ref{sec:vav_abs}.

%TODO: we need to distinguish between various meanings of 'correct'
%1. correctness of software: the implementation is correct up to a model specification
%2. correctness of a simulation: the implementation is correct up to a model specification AND it generates the same dynamics

\section{Introduction}
\label{sec:vav_introduction}
%According to \cite{robinson_simulation:_2014}, when implementing a model one should have the following aims in mind:
%\begin{itemize}
%	\item Speed Of Coding - how quickly can we implement a model.
%	\item Transparency - how easily can the code be understood.
%	\item Flexibility - how easily can the code be changed to model changes.
%	\item Run-Speed - how quickly will the code execute.
%\end{itemize}
%
%Further the author distinguishes three activities 
%\begin{itemize}
%	\item Coding - implementing the model.
%	\item Testing - verifying and white-box validating the model.
%	\item Documenting - recording details of the model.
%\end{itemize}
%
%Also it is important to distinguish between different natures of a simulation: terminating or non-terminating, and its output transient or steady-state (cycle or shifting).

%steward robinson simulation book on implementation
%- meaning of implementation
%	-> 1 implementing the findings: conduct a study which defines and gathers all findings about the model and document them
%	-> 2 implementing the model
%	-> 3 implementing the learning

\textbf{\textit{Validation}} is the process of ensuring that the model is sufficiently accurate for the purpose at hand. \textbf{\textit{Verification}} is the process of ensuring that the model design has been transformed into a computer model with sufficient accuracy. \cite{balci_verification_1998} define verification as "are we building the model right?" and validation as "are we building the right model?".

In this research we will primarily focus on verification, because its there where one ensures that the model is programmed correctly, the algorithms have been implemented properly, and the model does not contain errors, oversights, or bugs. Note that verification has a narrow definition and can be seen as a subset of the wider issue of validation. 

In verification and validation the aim is to ensure, that the model is sufficiently accurate, which always implies its purpose. Therefore the purpose and objectives must be known before it is validated. One distinguishes between:

\begin{itemize}
	\item White-box Validation: detailed, micro check if each part of the model represent the real world with sufficient accuracy. It is therefore intrinsic to model coding. The ways to do it is checking the code, visual checks, inspecting output reports. 
	\item Black-box Validation: overall, macro check whether the model provides a sufficiently accurate representation of the real world system. It can only be performed once model code is complete. Ways to do this is comparison with the real system or with other (simpler) models.

	\item White-box Verification: compares the content of the model to the \textit{conceptual} model. This is different to white-box validation which compares the content of the model to the \textit{real world}
	\item Black-box Verification: treating the functionality to test as a black box with inputs and outputs and comparing controlled inputs to expected outputs.

	%\item Data Validation: determining that the contextual data and the data required for model realisation and validation are sufficiently accurate for the purpose at hand.
\end{itemize}

%good paper \url{http://www2.econ.iastate.edu/tesfatsi/VVAccreditationSimModels.OBalci1998.pdf} : very nice 15 guidelines and life cycles, VERY valuable for background and introduction

So in general one can see verification as a test of the fidelity with which the conceptual model is converted into the computer model. Verification (and validation) is a continuous process and if it is already there in the programming language / supported by then this is much easier to do. This is the fundamental basis of our hypothesis where we claim that by choosing a programming language which supports this continuous verification and validation process, then the result is an implementation of a model which is more likely to be correct.

Unfortunately, there is no such thing as general validity: a model should be built for one purpose as simple as possible and not be too general, otherwise it becomes too bloated and too difficult / impossible to analyse. Also, there may be no real world to compare against: simulations are developed for proposed systems, new production / facilities which don't exist yet. Further, it is questionable which real world one is speaking of: the real world can be interpreted in different ways, therefore a model valid to one person might not be valid to another. Sometimes validation struggles because the real world data are inaccurate or there is not enough time to verify and validate everything.

In general this implies that we can only \textit{raise the confidence} in the correctness of the simulation, not validity: it is not possible to prove that a model is valid, instead one should think of confidence in its validity. Therefore, the process of verification and validation is not the proof that a model is correct but trying to prove that the model is incorrect! The more tests/checks one carries out which show that it is not incorrect, the more confidence we can place on the models validity.

%	- methods of verification and validation
%	-> conceptual model validation: judment based on the documentation
%	-> data validation: analysing data for inconsistencies
%	-> verification and white-box validation
%		-> both conceptually different but often treated together because both occur continuously through model coding
%		-> what should be checked: timings (cycle times, arrival times,...), control of elements (breakdown frequency, shift patterns), control flows (e.g. routing), control logic (e.g. scheduling, stock replenishment), distribution sampling (samples obtained from an empirial distribution)
%	-> verification and whilte-box validation methods
%		-> checking code: reading through code and ensure right data and logic is there. explain to others/discuss together/others should look at your code. 
%		-> Visual checks
%		-> inspecting output reports
%		
%	-> black-box testing: consider overall behaviour of the model without looking into its parts, basically two ways
%		-> comparison with the real system: statistical tests
%		-> comparison with another model (e.g. mathematical equations): could compare exactly or also through statistical tests
%		-> 
%		

%peers slides (inspired by steward robinson book):
%    - Experimentation Validation: Determining that the experimental procedures adopted are providing results that are sufficiently accurate for the purpose at hand.
%    	-> How can we do this?
%    		- Graphical or statistical methods for determining warm-up period, run length and replications (to obtain accurate results)
%			- Sensitivity analysis (to improve the understanding of the model)
%	
%	- Solution Validation: Determining that the results obtained from the model of the proposed solution are sufficiently accurate for the purpose at hand
%		-> How does this differ from Black Box Validation? Solution validation compares the model of the proposed solution to the implemented solution while black-box validation compares the base model to the real world
%		-> How can we do this? Once implemented it should be possible to validate the implemented solution against the model results
%		
%	- Verification: Testing the fidelity with which the conceptual model is converted into the computer model. Verification is done to ensure that the model is programmed correctly, the algorithms have been implemented properly, and the model does not contain errors, oversights, or bugs.
%
%		-> How can we do this? Same methods as for white-box validation (checking the code, visual checks, inspecting output reports) but ... Verification compares the content of the model to the conceptual model while white-box validation compares the content of the model to the real world
%		
%	- Difficulties of verification and validation
%		-> There is no such thing as general validity: a model is only valid with respect to its purpose
%		-> There may be no real world to compare against
%		-> Which real world? Different people have different interpretations of the real world
%		->  Often real world data are inaccurate: If the data are not accurate it is difficult to determine if the model's results are correct. Even if the data is accurate, the real worl  data are only a sample, which in itself creates inaccuracy
%		-> There is not enough time to verify and validate every aspect of a model
%		
%	- Some final remarks:
%		-> V\&V is a continuous and iterative process that is performed throughout the life cycle of a simulation study.
%			Example: If the conceptual model is revised as the project progresses it needs to be re-validated
%		-> V\&V work together by removing barriers and objections to model use and hence establishing credibility.
%		
%- Conclusion: Although, in theory, a model is either valid or not, proving this in practice is a very different matter. It is better to think in terms of confidence that can be placed in a model!

\section{Verification \& Validation in Agent-Based Simulation}
\label{sec:vav_abs}
In our research we focus primarily on the \textit{Verification} aspect of agent-based simulation: ensuring that the implementation reflects the specifications of the \textit{conceptual} model - have we built the model right? Thus we are not interested in our research into making connections to the real world and always see the model specifications as our "last resort", our ground truth beyond nothing else exists. When there are hypotheses formulated, we always treat and interpret them in respect of the conceptual model.
In \cite{ormerod_validation_2006} the authors clarify on verification in ABS. Verification "... is essentially the question: does the model do what we think it is supposed to do? Whenever a model has an analytical solution, a condition which embraces almost all conventional economic theory, verification is a matter of checking the mathematics.". They say about validation that "In an important sense, the current process of building ABMs is a discovery process, of discovering the types of behavioural rules for agents which appear to be consistent with phenomena we observe.". Further they claim that "Because such models are based on simulation, the lack of an analytical solution (in general) means that verification is harder, since there is no single result the model must match. Moreover, testing the range of model outcomes provides a test only in respect to a prior judgment on the plausibility of the potential range of outcomes. In this sense, verification blends into validation."
So the baseline is that either one has an analytical model as the basis of an agent-based model or one does not. In the former case, e.g. the SIR model, one can very easily validate the dynamics generated by the simulation to the one generated by the analytical solution (e.g. through System Dynamics). In the latter case one has basically no idea or description of the emergent behaviour of the system prior to its execution e.g. SugarScape. It is important to have some hypothesis about the emergent property / dynamics. The question is how verification / validation works in this setting as there is no formal description of the expected behaviour: we don't have a ground-truth against which we can compare our simulation dynamics. %(eventuell hilft hier hans vollbrecht weiter: Simulation hat hier den Sinn, die Controller anhand der Roboteraufgabe zu validieren, Bei solchen Simulationen ist man interessiert an allen möglichen Sequenzen, und da das meist zu viele sind, an einer möglichst gut verteilten Stichprobenmenge. Hier geht es weniger um richtige Zeitmodellierung, sondern um den Test aller möglichen Ereignissequenzen.)

General there are the following simulation requirements to ABS \cite{robinson_simulation:_2014}, which all can be addressed in our \textit{pure} functional approach as described in the paper in Appendix \ref{app:pfe}:

\begin{itemize}
	\item Modelling progress of time - achieved using functional reactive programming (FRP)
	\item Modelling variability - achieved using FRP
	\item Fixing random number streams to allow simulations to be repeated under same conditions - ensured by \textit{pure} functional programming and Random Monads
	\item Rely only on past - guaranteed with \textit{Arrowized} FRP
	\item Bugs due to implicitly mutable state - reduced using pure functional programming
	\item Ruling out external sources of non-determinism / randomness - ensured by \textit{pure} functional programming
	\item Deterministic time-delta - ensured by \textit{pure} functional programming
	\item Repeated runs lead to same dynamics - ensured by \textit{pure} functional programming
\end{itemize}

\cite{kleijnen_verification_1995} suggests good programming practice which is extremely important for high quality code and reduces bugs but real world practice and experience shows that this alone is not enough, even the best programmers make mistakes which often can be prevented through a strong static or a dependent type system already at compile time. What we can guarantee already at compile time, doesn't need to be checked at run-time which saves substantial amount of time as at run-time there may be a large number of execution paths through the simulation which is almost always simply not feasible to check (note that we also need to check all combinations). This paper also cites modularity as very important for verification: divide and conquer and test all modules separately. We claim that this is especially easy in functional programming as code composes better than in traditional object-oriented programming due to the lack of interdependence between data and code as in objects and the lack of global mutable state (e.g. class variables or global variables) - this makes code extremely convenient to test. The paper also discusses statistical tests (the t test) to check if the outcome of a simulation is sufficiently close to real-world dynamics - we explicitly omit this as it part of validation and not the focus of this research. % Also the paper suggests using animations to visualise the processes within the simulation for verification purposes (of course they note that animation may be misleading when one focuses on too short simulation runs).

%So we ask whether we can encode phenomena we observe in the types? can we use types for the discovery process as well? can dependent types guide our exploratory approach to ABS?

\cite{polhill_ghost_2005}: "For some time now, Agent Based Modelling has been used to simulate and explore complex systems, which have proved intractable to other modelling approaches such as mathematical modelling. More generally, computer modelling offers a greater flexibility and scope to represent phenomena that do not naturally translate into an analytical framework. Agent Based Models however, by their very nature, require more rigorous programming standards than other computer simulations. This is because researchers are cued to expect the unexpected in the output of their simulations: they are looking for the 'surprise' that shows an interesting emergent effect in the complex system. It is important, then, to be absolutely clear that the model running in the computer is behaving exactly as specified in the design. It is very easy, in the several thousand lines of code that are involved in programming an Agent Based Model, for bugs to creep in. Unlike mathematical models, where the derivations are open to scrutiny in the publication of the work, the code used for an Agent Based Model is not checked as part of the peer-review process, and there may even be Intellectual Property Rights issues with providing the source code in an accompanying web page."

\cite{galan_errors_2009}: "a prerequisite to understanding a simulation is to make sure that there is no significant disparity between what we think the computer code is doing and what is actually doing. One could be tempted to think that, given that the code has been programmed by someone, surely there is always at least one person - the programmer - who knows precisely what the code does. Unfortunately, the truth tends to be quite different, as the leading figures in the field report, including the following: You should assume that, no matter how carefully you have designed and built your simulation, it will contain bugs (code that does something different to what you wanted and expected), "Achieving internal validity is harder than it might seem. The problem is knowing whether an unexpected result is a reflection of a mistake in the programming, or a surprising consequence of the model itself. […] As is often the case, confirming that the model was correctly programmed was substantially more work than programming the model in the first place. This problem is particularly acute in the case of agent-based simulation. The complex and exploratory nature of most agent-based models implies that, before running a model, there is some uncertainty about what the model will produce. Not knowing a priori what to expect makes it difficult to discern whether an unexpected outcome has been generated as a legitimate result of the assumptions embedded in the model or, on the contrary, it is due to an error or an artefact created in the model design, its implementation, or its execution."

\subsection{Testing}
Although (pure) functional programming allows us to have stronger guarantees about the behaviour and absence of bugs of the simulation already at compile-time, we still need to test all the properties of our simulation which we cannot guarantee at compile-time.

We found property-based testing particularly well suited for ABS. Although it is now available in a wide range of programming languages and paradigms, propert-based testing has its origins in Haskell \cite{claessen_quickcheck:_2000, claessen_testing_2002} and we argue that for that reason it really shines in pure functional programming.
Property-based testing allows to formulate \textit{functional specifications} in code which then the property-testing library (e.g. QuickCheck \cite{claessen_quickcheck:_2000}) tries to falsify by automatically generating random test-data covering as much cases as possible. When an input is found for which the property fails, the library then reduces it to the most simple one. It is clear to see that this kind of testing is especially suited to ABS, because we can formulate specifications, meaning we describe \textit{what} to test instead of \textit{how} to test (again the declarative nature of functional programming shines through).

Generally we need to distinguish between two types of testing/verification: 1. testing/verification of models for which we have real-world data or an analytical solution which can act as a ground-truth. examples for such models are the SIR model, stock-market simulations, social simulations of all kind and 2. testing/verification of models which are just exploratory and which are only be inspired by real-world phenomena. examples for such models are Epsteins Sugarscape and Agent\_Zero.

In both cases this leaves us with Black-Box and White-Box Verification:

%\subsection{Comparison of dynamics against existing data}
%- utilise a statistical test with H0 "ABS and comparison is not the same" and H1 "ABS and comparison is the same"
%- how many replications and how do we average?
%- which statistical test do we implement? (steward robinson simulation book, chapter 12.4.4)
%	-> Normalizsed Mean Squared Error (NMSE)
%	-> TODO: implement confidence interval 
%	-> TODO: what about chi-squared?
%	-> TODO: what about paired-t confidence interval
%
%IMPORTANT: this is not what we are after here in this paper, statistical tests are a science on their own and there actually exists quite a large amount of literature for conducting statistical tests on ABS dynamics: Robinson Book (TODO: find additional literature)	

\subsubsection{Black Box Verification}
In Black Box Verification one generally feeds input and compares it to expected output. In the case of ABS we have the following examples of black-box test:
\begin{enumerate}
	\item Isolated Agent Behaviour - test isolated agent behaviour under given inputs using unit- and property-based testing
	\item Interacting Agent Behaviour - test if interaction between agents are correct 
	\item Simulation Dynamics - compare emergent dynamics of the ABS as a whole under given inputs to an analytical solution / real-world dynamics in case there exists some using statistical tests
	\item Hypotheses- test whether hypotheses are valid / invalid using unit- and property-based testing. % TODO: how can we formulate hypotheses in unit- and/or property-based tests?
\end{enumerate}

%- testing of the final dynamics: how close do they match the analytical solution
%- can we express model properties in tests e.g. quickcheck?
%- property-testing shines here
%- isolated tests: how easy can we test parts of an agent / simulation?

Using black-box verification and property-based testing we can apply for the following use cases for testing ABS in FRP:

\paragraph{Finding optimal $\Delta t$}
The selection of the right $\Delta t$ can be quite difficult in FRP because we have to make assumptions about the system a priori. One could just play it safe with a very conservatively selected small $\Delta t < 0.1$ but the smaller $\Delta t$, the lower the performance as it quickly multiplies the number of steps to calculate. Obviously one wants to select the \textit{optimal} $\Delta t$, which in the case of ABS is the largest possible $\Delta t$ for which we still get the correct simulation dynamics.
To find out the \textit{optimal} $\Delta t$ one can make direct use of the Black Box tests: start with a large $\Delta t = 1.0$ and reduce it by half every time the tests fail until no more tests fail - if for $\Delta t = 1.0$ tests already pass, increasing it may be an option. It is important to note that although isolated agent behaviour tests might result in larger $\Delta t$, in the end when they are run in the aggregate system, one needs to sample the whole system with the smallest $\Delta t$ found amongst all tests. Another option would be to apply super-sampling to just the parts which need a very small $\Delta t$ but this is out of scope of this paper.

\paragraph{Agents as signals}
Agents \textit{might} behave as signals in FRP which means that their behaviour is completely determined by the passing of time: they only change when time changes thus if they are a signal they should stay constant if time stays constant. This means that they should not change in case one is sampling the system with $\Delta t = 0$. Of course to prove whether this will \textit{always} be the case is strictly speaking impossible with a Black Box verification but we can gain a good level of confidence with them also because we are staying pure. It is only through white box verification that we can really guarantee and prove this property.

\subsubsection{White-Box Verification}
White-Box verification is necessary when we need to reason about properties like \textit{forever}, \textit{never}, which cannot be guaranteed from black-box tests. Additional help can be coverage tests with which we can show that all code paths have been covered in our tests.

\subsection{Example: Property-Based Testing of SIR}
As an example we discuss the black-box testing for the SIR model using property-testing. We test if the \textit{isolated} behaviour of an agent in all three states Susceptible, Infected and Recovered, corresponds to model specifications. The crucial thing though is that we are dealing with a stochastic system where the agents act \textit{on averages}, which means we need to average our tests as well. We conducted the tests on the implementation found in the paper of Appendix \ref{app:pfe}.

\subsubsection{Black-Box Verification}
The interface of the agent behaviours are defined below. When running the SF with a given $\Delta t$ one has to feed in the state of all the other agents as input and the agent outputs its state it is after this $\Delta t$.

\begin{HaskellCode}
data SIRState 
  = Susceptible 
  | Infected 
  | Recovered
  
type SIRAgent = SF [SIRState] SIRState

susceptibleAgent :: RandomGen g => g -> SIRAgent
infectedAgent :: RandomGen g => g -> SIRAgent
recoveredAgent :: SIRAgent
\end{HaskellCode}

\paragraph{Susceptible Behaviour}
A susceptible agent \textit{may} become infected, depending on the number of infected agents in relation to non-infected the susceptible agent has contact to. To make this property testable we run a susceptible agent for 1.0 time-unit (note that we are sampling the system with a smaller $\Delta t = 0.1$) and then check if it is infected - that is it returns infected as its current state.

Obviously we need to pay attention to the fact that we are dealing with a stochastic system thus we can only talk about averages and thus it does not suffice to only run a single agent but we are repeating this for e.g. $N = 10.000$ agents (all with different RNGs). We then need a formula for the required fraction of the N agents which should have become infected on average. Per 1.0 time-unit, a susceptible agent makes \textit{on average} contact with $\beta$ other agents where in the case of a contact with an infected agent the susceptible agent becomes infected with a given probability $\gamma$. In this description there is another probability hidden, which is the probability of making contact with an infected agent which is simply the ratio of number of infected agents to number not infected agents. The formula for the target fraction of agents which become infected is then: $\beta * \gamma * \frac{number of infected}{number of non-infected}$. To check whether this test has passed we compare the required amount of agents which on average should become infected to the one from our tests (simply count the agents which got infected and divide by N) and if the value lies within some small $\epsilon$ then we accept the test as passed.

Obviously the input to the susceptible agents which we can vary is the set of agents with which the susceptible agents make contact with. To save us from constructing all possible edge-cases and combinations and testing them with unit-tests we use property-testing with QuickCheck which creates them randomly for us and reduces them also to all relevant edge-cases. This is an example for how to use property-based testing in ABS where QuickCheck can be of immense help generating random test-data to cover all cases.

%TODO: derive the target-fraction formula from the differential equations
%TODO: can we encode this somehow on a type level using dependent types? then we don't need to test this property any more

\paragraph{Infected Behaviour}
An infected agent \textit{will always} recover after a finite time, which is \textit{on average} after $\delta$ time-units. Note that this property involves stochastics too, so to test this property we run a large number of infected agents e.g. $N = 10.000$ (all with different RNGs) until they recover, record the time of each agents recovery and then average over all recovery times. To check whether this test has passed we compare the average recovery times to $\delta$ and if they lie within some small $\epsilon$ then we accept the test as passed.

We use property-testing with QuickCheck in this case as well to generate the set of other agents as input for the infected agents. Strictly speaking this would not be necessary as an infected agent never makes contact with other agents and simply ignores them - we could as well just feed in an empty list. We opted for using QuickCheck for the following reasons:

\begin{itemize}
	\item We wanted to stick to the interface specification of the agent-implementation as close as possible which asks to pass the states of all agents as input.
	\item We shouldn't make any assumptions about the actual implementation and if it REALLY ignores the other agents, so we strictly stick to the interface which requires us to input the states of all the other agents.
	\item The set of other agents is ignored when determining whether the test has failed or not which indicates by construction that the behaviour of an infected agent does not depend on other agents.
	\item We are not just running a single replication over 10.000 agents but 100 of them which should give black-box verification more strength.
\end{itemize}

%TODO: derive the average formula from the differential equations
%TODO: can we encode this somehow on a type level using dependent types? then we don't need to test this property any more

\paragraph{Recovered Behaviour}
A recovered agent will stay in the recovered state \textit{forever}. Obviously we cannot write a black-box test that truly verifies that because it had to run in fact forever. In this case we need to resort to White-Box Verification (see below).

Because we use multiple replications in combination with QuickCheck obviously results in longer test-runs (about 5 minutes on my machine)
In our implementation we utilized the FRP paradigm. It seems that functional programming and FRP allow extremely easy testing of individual agent behaviour because FP and FRP compose extremely well which in turn means that there are no global dependencies as e.g. in OOP where we have to be very careful to clean up the system after each test - this is not an issue at all in our \textit{pure} approach to ABS.

\paragraph{Simulation Dynamics}
We won't go into the details of comparing the dynamics of an ABS to an analytical solution, that has been done already by \cite{macal_agent-based_2010}. What is important is to note that population-size matters: different population-size results in slightly different dynamics in SD => need same population size in ABS (probably...?). Note that it is utterly difficult to compare the dynamics of an ABS to the one of a SD approach as ABS dynamics are stochastic which explore a much wider spectrum of dynamics e.g. it could be the case, that the infected agent recovers without having infected any other agent, which would lead to an extreme mismatch to the SD approach but is absolutely a valid dynamic in the case of an ABS. The question is then rather if and how far those two are \textit{really} comparable as it seems that the ABS is a more powerful system which presents many more paths through the dynamics.
%TODO: i really want to solve this for the SIR approach
%	-> confidence intervals?
%	-> NMSE?
%	-> does it even make sense?

\paragraph{Finding optimal $\Delta t$}
Obviously the \textit{optimal} $\Delta t$ of the SIR model depends heavily on the model parameters: contact rate $\beta$ and illness duration $\delta$. We fixed them in our tests to be $\beta = 5$ and $\delta = 15$. By using the isolated behaviour tests we found an optimal $\Delta t = 0.125$ for the susceptible behaviour and $\Delta t = 0.25$ for the infected behaviour. %TODO: dynamics comparison?

\paragraph{Agents as signals}
Our SIR agents \textit{are} signals due to the underlying continuous nature of the analytical SIR model and to some extent we can guarantee this through black box testing. For this we write tests for each individual behaviour as previously but instead of checking whether agents got infected or have recovered we assume that they stay constant: they will output always the same state when sampling the system with $\Delta t = 0$. The tests are conceptual the complementary tests of the previous behaviour tests so in conjunction with them we can assume to some extent that agents are signals. To prove it, we need to look into white box verification as we cannot make guarantees about properties which should hold \textit{forever} in a computational setting.

\subsubsection{White-Box Verification}
%TODO: the implementation below has a SEVERE bug, all stochastic functions are correlated because they use the same RNG. this leads to different distributions of the dynamics, which can be shown using the test-code which generates the dynamics. The random monad version seems to perform much better where the mean is very close to the SD solution.

In the case of the SIR model we have the following invariants: 
\begin{itemize}
	\item A susceptible agent will \textit{never} make the transition to recovered.
	\item An infected agent will \textit{never} make the transition to susceptible.
	\item A recovered agent will \textit{forever} stay recovered.
\end{itemize}

All these invariants can be guaranteed when reasoning about the code. An additional help will be then coverage testing with which we can show that an infected agent never returns susceptible, and a susceptible agent never returned infected given all of their functionality was covered which has to imply that it can never occur!

%Lets start with looking at the recovered behaviour as it is the simplest one. We then continue with the infected behaviour and end with the susceptible behaviour as it is the most complex one.

We will only look at the recovered behaviour as it is the simplest one. We leave the susceptible and infected behaviours for further research / the final thesis because the conceptual idea becomes clear from looking at the recovered agent.

\paragraph{Recovered Behaviour}
The implementation of the recovered behaviour is as follows:

\begin{HaskellCode}
recoveredAgent :: SIRAgent
recoveredAgent = arr (const Recovered)
\end{HaskellCode}

Just by looking at the type we can guarantee the following:
\begin{itemize}
	\item it is pure, no side-effects of any kind can occur
	\item no stochasticity possible because no RNG is fed in / we don't run in the random monad
\end{itemize}

The implementation is as concise as it can get and we can reason that it is indeed a correct implementation of the recovered specification: we lift the constant function which returns the Recovered state into an arrow. Per definition and by looking at the implementation, the constant function ignores its input and returns always the same value. This is exactly the behaviour which we need for the recovered agent. Thus we can reason that the recovered agent will return Recovered \textit{forever} which means our implementation is indeed correct.

\chapter{Dependent Types}
\label{chap:dependent_types}
Dependent types are a very powerful addition to functional programming as they allow us to express even stronger guarantees about the correctness of programs \textit{already at compile-time}. They go as far as allowing to formulate programs and types as constructive proofs which must be \textit{total} by definition \cite{thompson_type_1991, mckinna_why_2006, altenkirch_pi_2010}. 

We hypothesise, that  dependent types will allow us to push the correctness of agent-based simulations to a new, unprecedented level. The investigation of dependent types in ABS will be the main unique contribution to knowledge of my Ph.D.

So far no research using dependent types in agent-based simulation exists at all. We have already started to explore this for the first time and ask more specifically how we can add dependent types to our functional approach, which conceptual implications this has for ABS and what we gain from doing so. We plan on using Idris \cite{brady_idris_2013} as the language of choice as it is very close to Haskell with focus on real-world application and running programs as opposed to other languages with dependent types e.g. Agda and Coq which serve primarily as proof assistants.

We hypothesize that dependent types could help ruling out even more classes of bugs at compile time and even encode invariants and model specifications on the type level which would allow the ABS community to reason about a model directly in code. 

Dependent types could be made of use in ABS in the following ways:

\begin{itemize}
	\item Accessing e.g. discrete 2D environments involves (almost always) indexed array access which is always potentially dangerous as the indices have to be checked at run-time.
	
	Using dependent types it should be possible to encode the environment dimensions into the types. In combination with suitable data types (finite sets) for coordinates one should be able to ensure already at compile time that access happens only within the bounds of the environment.
	
	\item Often, Agent-Based Models define their agents in terms of state-machines. It is easy to make wrong state-transitions e.g. in the SIR model when an infected agent should recover, nothing prevents one from making the transition back to susceptible. 
	
	Using dependent types it might be possible to encode invariants and state-machines on the type level which can prevent such invalid transitions already at compile time. This would be a huge benefit for ABS because of the popularity of state-machines in agent-based models.
	
	\item State-Machines often have timed transitions e.g. in the SIR model, an infected agent recovers after a given time. Nothing prevents us from introducing a bug and \textit{never} doing the transition at all.
	
	With dependent types we might be able to encode the passing of time in the types and guarantee on a type level that an infected agent has to recover after a finite number of time steps.
	
	\item In more sophisticated models agents interact in more complex ways with each other e.g. through message exchange using agent IDs to identify target agents. The existence of an agent is not guaranteed and depends on the simulation time because agents can be created or terminated at any point during simulation. 
	
	Dependent types could be used to implement agent IDs as a proof that an agent with the given id exists \textit{at the current time-step}. This also implies that such a proof cannot be used in the future, which is prevented by the type system as it is not safe to assume that the agent will still exist in the next step.
	
	\item Using dependent types we might be able to encode a protocol for agent-agent interactions which e.g. ensures on the type-level that an agent has to reply to a request or that a more specific protocol has to be followed e.g. in auction- or trading-simulations.
	
	%\item Randomness is of central importance in agent-based simulation but nothing enforces from which distribution to draw. 
	
	%With dependent types we might to implement probabilistic types which can encode probability distributions in types already about which we can then reason and guarantee at compile-time that we draw from the correct distribution.
	
	\item For some agent-based simulations there exists equilibria, which means that from that point the dynamics won't change any more e.g. when a given type of agents vanishes from the simulation or resources are consumed. This means that at that point the dynamics won't change any more, thus one can safely terminate the simulation. But still such simulations are stepped for a fixed number of time-steps or events or the termination criterion is checked at run-time in the feedback-loop. 
	
	Using dependent types it might be possible to encode equilibria properties in the types in a way that the simulation automatically terminates when they are reached. This results then in a \textit{total} simulation, creating a correspondence between the equilibrium of a simulation and the totality of its implementation. Of course this is only possible for models in which we know about their equilibria a priori or in which we can reason somehow that an equilibrium exists.
\end{itemize}

\section{Related Work}
In \cite{botta_functional_2011} the authors are using functional programming as a specification for an agent-based model of exchange markets but leave the implementation for further research where they claim that it requires dependent types. This paper is the closest usage of dependent types in agent-based simulation we could find in the existing literature and to our best knowledge there exists no work on general concepts of implementing pure functional agent-based simulations with dependent types. As a remedy to having no related work to build on, we looked into works which apply dependent types to solve real world problems from which we then can draw inspiration from. 

The paper \cite{brady_correct-by-construction_2010} discusses depend types to implement correct-by-construction concurrency in the Idris language \cite{brady_idris_2013}. The authors introduce the concept of a Embedded Domain Specific Language (EDSL) for concurrently locking/unlocking and reading/writing of resources and show that an implementation and formalisation are the same thing when using dependent types. We can draw inspiration from it by taking into consideration that we might develop a EDSL in a similar fashion for specifying general commands which agents can execute. The interpreter of such a EDSL can be pure itself and doesn't have to run in the IO Monad as our previous research (TODO: cite my PFE paper) has shown that ABS can be implemented pure.

In \cite{brady_idris_2011} the authors discuss systems programming with focus on network packet parsing with full dependent types in the Idris language \cite{brady_idris_2013}. Although they use an older version of it where a few features are now deprecated, they follow the same approach as in the previous paper of constructing an EDSL and and writing an interpreter for the EDSL. In a longer introduction of Idris the authors discus its ability for termination checking in case that recursive calls have an argument which is structurally smaller than the input argument in the same position and that these arguments belong to a strictly positive data type. We are particularly interested in whether we can implement an agent-based simulation which termination can be checked at compile time - it is total.

In \cite{brady_programming_2013} the author discusses programming and reasoning with algebraic effects and dependent types in the Idris language \cite{brady_idris_2013}. They claim that monads do not compose very well as monad transformer can quickly become unwieldy when there are lots of effects to manage. As a remedy they propose algebraic effects and implement them in Idris and show how dependent types can be used to reason about states in effectful programs. In our previous research (TODO: cite my PFE paper) we relied heavily on Monads and transformer stacks and we indeed also experienced the difficulty when using them. Algebraic effects might be a promising alternative for handling state as the global environment in which the agents live or threading of random-numbers through the simulation which is of fundamental importance in ABS. Unfortunately algebraic effects cannot express continuations (according to the authors of the paper) which is but of fundamental importance for pure functional ABS as agents are on the lowest level built on continuations - synchronous agent interactions and time-stepping builds directly on continuations. Thus we need to find a different representation of agents - GADTs seem to be a natural choice as all examples build heavily on them and they are very flexible.

In \cite{fowler_dependent_2014} the authors apply dependent types to achieve safe and secure web programming. This paper shows how to implement dependent effects, which we might draw inspiration from of how to implement agent-interactions which, depending on their kind, are effectful e.g. agent-transactions or events.

In \cite{brady_state_2016} the author introduces the ST library in Idris, which allows a new way of implementing dependently typed state machines and compose them vertically (implementing a state machine in terms of others) and horizontally (using multiple state machines within a function). In addition this approach allows to manage stateful resources e.g. create new ones, delete existing ones. We can draw further inspiration from that approach on how to implement dependently typed state machines, especially composing them hierarchically, which is a common use case in agent-based models where agents behaviour is modelled through hierarchical state-machines. As with the Algebraic Effects, this approach doesn't support continuations (TODO: is this so?), so it is not really an option to build our architecture for our agents on it, but it may be used internally to implement agents or other parts of the system. What we definitely can draw inspiration from is the implementation of the indexed Monad \textit{STrans} which is the main building block for the ST library.

The book \cite{brady_type-driven_2017} is a great source to learn pure functional dependently typed programming and in the advanced chapters introduces the fundamental concepts of dependent state machine and dependently typed concurrent programming on a simpler level than the papers above. One chapter discusses on how to implement a messaging protocol for concurrent programming, something we can draw inspiration from for implementing our synchronous agent interaction protocols.

In \cite{sculthorpe_safe_2009} the authors apply dependent types to FRP to avoid some run-time errors and implement a dependently typed version of the Yampa library in Agda. FRP was the underlying concept of implementing agent-based model we took on in our previous approach (TODO: cite). We could have taken the same route and lift FRP into dependent types but we chose explicitly to not go into this direction and look into complementing approaches on how to implement agent-based models.

The fundamental difference to all these real-world examples is that in our approach, the system evolves over time and agents act over time. A fundamental question will be how we encode the monotonous increasing flow of time in types and how we can reflect in the types that agents act over time.

%An agent can be seen as a potentially infinite stream of continuations which at some point could return information to stop evaluating the next item of the stream which allows an agent to terminate.
%correspondence between temporal logics and FRP due to jeffery: is abs just another temporal logic?

%The authors of \cite{ionescu_dependently-typed_2012} discuss how to use dependent types to specify fundamental theorems of economics, unfortunately they are not computable and thus not constructive, thus leaving it more to a theoretical, specification side.
%Ionesus talk on dependently typed programming in scientific computing
%https://www.pik-potsdam.de/members/ionescu/cezar-ifl2012-slides.pdf
%Ionescus talk on Increasingly Correct Scientific Computing
%%https://www.cicm-conference.org/2012/slides/CezarIonescu.pdf
%Ionescus talk on Economic Equilibria in Type Theory
%https://www.pik-potsdam.de/members/ionescu/cezar-types11-slides.pdf
%Ionescus talk on Dependently-Typed Programming in Economic Modelling
%https://www.pik-potsdam.de/members/ionescu/ee-tt.pdf

\section{Background}
\label{sec:dep_background}

In this section we give an overview of the concepts behind dependent types and what they can do. Generally dependent types add the following concepts to existing pure functional programming:

\begin{enumerate}
	\item Types are first-class citizen - In dependently types languages, types can depend on any \textit{values}, and can be \textit{computed} at compile time which makes them first-class citizen.

	\item Totality and termination - A total function is defined in \cite{brady_type-driven_2017}: it terminates with a well-typed result or produces a non-empty finite prefix of a well-typed infinite result in finite time. Idris is turing-complete but is able to check the totality of a function under some circumstances but not in general as it would imply that it can solve the halting problem. Other dependently typed languages like Agda or Coq restrict recursion to ensure totality of all their functions - this makes them non turing-complete.

	\item Types as proofs - Because types can depend on any values and can be computed at compile time, they can be used as constructive proofs (see \ref{sub:dep_foundations}) which must terminate, this means a well-typed program (which is itself a proof) is always terminating which in turn means that it must consist out of total functions. Note that Idris does not restrict us to total functions but we can enforce it through compiler flags.
\end{enumerate}

\subsection{An example: Vector}
To give a concrete example of dependent types and their concepts, we introduce the canonical example used in all tutorials on dependent types: the Vector.

In Haskell (or in Java) there exists the List data-structure which holds a finite number of homogeneous elements, where the type of the elements can be fixed at compile-time. Using dependent types we can implement the same but adding the length of the list to the type - we call this data-structure a vector.

We define the vector as a Generalised Algebraic Data Type (GADT). A vector has a \textit{Nil} element which marks the end of a vector and a \textit{(::)} which is a recursive (inductive) definition of a linked List. We defined some vectors and we see that the length of the vector is directly encoded in its first type-variable of type Nat, natural numbers. Note that the compiler will refuse to accept \textit{testVectFail} because the type specifies that it holds 2 elements but the constructed vector only has 1 element.

\begin{HaskellCode}
data Vect : Nat -> Type -> Type where
	Nil  : Vect Z e
	(::) : (elem : e) -> (xs : Vect n e) -> Vect (S n) e
	
testVect : Vect 3 String
testVect = "Jonathan" :: "Andreas" :: "Thaler" :: Nil

testVectFail : Vect 2 Nat
testVectFail = 42 :: Nil
\end{HaskellCode}

We can now go on and implement a function \textit{append} which simply appends two vectors. Here we directly see \textit{type-level computations} as we compute the length of the resulting vector. Also this function is \textit{total}, as it covers all input cases and the recursion happens on a \textit{structurally smaller argument}:

\begin{HaskellCode}
append : Vect n e -> Vect m e -> Vect (n + m) e
append Nil ys = ys
append (x :: xs) ys = x :: append xs ys

append testVect testVect
["Jonathan", "Andreas", "Thaler", "Jonathan", "Andreas", "Thaler"] : Vect 8 String
\end{HaskellCode}

What if we want to implement a \textit{filter} function, which, depending on a given predicate, returns a new vector which holds only the elements for which the predicates returns true? How can we compute the length of the vector at compile-time? In short: we can't, but we can make us of \textit{dependent pairs} where the \textit{type} of the second element depends on the \textit{value} of the first (dependent pairs are also known as $\Sigma$ types, see \ref{sub:dep_foundations} below).

The function is total as well and works very similar to \textit{append} but uses dependent types as return, which are indicated by \textit{**}:

\begin{HaskellCode}
filter : Vect n e -> (e -> Bool) -> (k ** Vect k e)
filter [] f = (Z ** Nil)
filter (elem :: xs) f =
  case f elem of
    False => filter xs f
    True  => let (_ ** xs') = filter xs f
             in  (_ ** elem :: xs')
             
filter testVect (=="Jonathan")
(1 ** ["Jonathan"]) : (k : Nat ** Vect k String)
\end{HaskellCode}

It might seem that writing a \textit{reverse} function for a Vector is very easy, and we might give it a go by writing:
\begin{HaskellCode}
reverse : Vect n e -> Vect n e
reverse [] = []
reverse (elem :: xs) = append (reverse xs) [elem]
\end{HaskellCode}

Unfortunately the compiler complains because it cannot unify 'Vect (n + 1) e' and 'Vect (S n) e'. In the end, the compiler tells us that it cannot determine that (n + 1) is the same as (1 + n). The compiler does not know anything about the commutativity of addition which is due to how natural numbers and their addition are defined.

Lets take a detour. The natural numbers can be inductively defined by their initial element zero Z and the successor. The number 3 is then defined as the successor of successor of successor of zero:

\begin{HaskellCode}
data Nat = Z | S Nat

three : Nat 
three = S (S (S Z))
\end{HaskellCode}

Defining addition over the natural numbers is quite easy by pattern-matching over the first argument: 

\begin{HaskellCode}
plus : (n, m : Nat) -> Nat
plus Z right        = right
plus (S left) right = S (plus left right)
\end{HaskellCode}

Now we can see why the compiler cannot infer that (n + 1) is the same as (1 + n). The expression (n + 1) is translated to (plus n 1), where we pattern-match over the first argument, so we cannot reach a case in which (plus n 1) = S n. To do that we would need to define a different plus function which pattern-matches over the second argument - which is clearly the wrong way to go.

To solve this problem we can exploit the fact that dependent types allow us to perform type-level computations. This should allow us to express commutativity of addition over the natural numbers as a type. For that we define a function which takes in two natural numbers and returns a proof that addition commutes. 

\begin{HaskellCode}
plusCommutative : (left : Nat) -> (right : Nat) -> left + right = right + left
\end{HaskellCode}

We now begin to understand what it means when we speak of \textit{types as proofs}: we can actually express e.g. laws of the natural numbers in types and proof them by implementing a program which inhibits the type - we speak then of a constructive proof (see more on that below \ref{sub:dep_foundations}). Note that \textit{plusCommutative} is already implemented in Idris and we omit the actual implementation as it is beyond the scope of this introduction

Having our proof of commutativity of natural numbers, we can now implement a working (speak: correct) version of \textit{reverse}. The function \textit{rewrite} is provided by Idris: if we have a proof for x = y, the 'rewrite expr in' syntax will search for x in the required type of expr and replace it with y:

\begin{HaskellCode}
reverse : Vect n e -> Vect n e
reverse [] = []
reverse (elem :: xs) = append (reverse xs) [elem]
  where
    reverseProof : Vect (k + 1) a -> Vect (S k) a
    reverseProof {k} result = rewrite plusCommutative 1 k in result
\end{HaskellCode}

On of the most powerful aspects of dependent types is that they allow us to express equality on an unprecedented level. Non-dependently typed languages have only very basic ways of expressing the equality of two elements of same type. Either we use a boolean or another data-structure which can indicate equality or not. Idris supports this type of equality as well through \textit{(==) : Eq ty $\Rightarrow$ ty $\rightarrow$ ty $\rightarrow$ Bool}. The drawback of using a boolean is that in the end we don't have a real evidence of equality: even though the elements might be equal, the compiler has no means of inferring this and we can still make programming mistakes after the equality check because of this lack of compiler support.

This is different in dependent types which allow us to define \textit{decidable} equality through a type (see more on decidable / non-decidable equality below \ref{sub:dep_foundations}). Idris defines a decidable property as the following:

\begin{HaskellCode}
-- Decidability. A decidable property either holds or is a contradiction.
data Dec : Type -> Type where
  -- The case where the property holds
  -- @ prf the proof
  Yes : (prf : prop) -> Dec prop

  -- The case where the property holding would be a contradiction
  -- @ contra a demonstration that prop would be a contradiction
  No  : (contra : prop -> Void) -> Dec prop
\end{HaskellCode}

With that we can implement a function which constructs a proof that two natural numbers are equal, or not. We do this simply by pattern matching over both numbers with corresponding base cases and inductions. In case they are not equal we need to construct a proof that they are actually not equal which is done by showing that given some property results in a contradiction - indicated by the type \textit{Void}. In case of \textit{zeroNotSuc} the first number is zero (Z) whereas the other one is non-zero (a successor of some k), which can never be equal, thus we return a \textit{No} instance of the decidable property for which we need to provide the contradiction. In case of \textit{sucNotZero} its just the other way around. \textit{noRec} works very similar but here we are in the induction case which says that if k equals j leads to a contradiction, (k + 1) and (j + 1) can't be equal as well (induction hypothesis).

\begin{HaskellCode}
checkEqNat : (num1 : Nat) -> (num2 : Nat) -> Dec (num1 = num2)
checkEqNat Z Z         = Yes Refl
checkEqNat Z (S k)     = No zeroNotSuc
checkEqNat (S k) Z     = No sucNotZero
checkEqNat (S k) (S j) = case checkEqNat k j of
                              Yes prf   => Yes (cong prf)
                              No contra => No (noRec contra)
                              
zeroNotSuc : (0 = S k) -> Void
zeroNotSuc Refl impossible

sucNotZero : (S k = 0) -> Void
sucNotZero Refl impossible

noRec : (contra : (k = j) -> Void) -> (S k = S j) -> Void
noRec contra Refl = contra Refl
\end{HaskellCode}                              

The important thing to understand here is that our Dec property holds much more information than just a boolean flag which indicates whether Yes/No that two elements of a type are equal: in case of Yes we have a type which says that num1 is equal to num2, which can be directly used by the compiler, both elements are treated as the same.

\subsection{Foundations}
\label{sub:dep_foundations}


- dependently typed functions (pi types)
- dependent pairs (sigma types)
- decidable equality

\subsection{Constructivism}
TODO: ABS is constructive: "if you can't grow it, you can't explain it" (epstein)
TODO: Dependent Types are constructive
=> there are no excluded middle in both approaches
=> are there deeper, philosophical connections going on? does it have even deeper implications?
TODO: shortly discuss Propositions as types from HOTT 1.11. In the end a dependently typed ABS is then a constructive proof of WHAT? the model? if we have a total SIR implementation its a constructive proof that the agent-based implementation is total / will reach an equilibrium after a finite number of steps. Still it is not entirely clear WHAT WE ARE PROVING when we are constructing dependently typed agent-based simulations. I need to think about this more carefully
TODO: checkout my notes in 1st annual review on constructivism / popper 

Law of excluded middle does not hold anymore because it would require us to be able to effectively compute / decide whether a proposition is true or false - which amounts to solving the halting problem, which is not possible in the general case.

An important concept of this constructive approach is that the (proposition of) equality between two elements of the same type are is itself a type, called equality or identity types. This is much more expressive than a boolean proposition which evaluates to True in case they are the same and False if not as an equality type encodes much richer information which can be used by the type system. With the boolean approach, also known as boolean blindness, although one has compare two elements on equality and this check has returned true, the compiler has still no way of knowing \textit{after} the check that both elements are indeed the same - with equality types we can provide this information which can be used by the compiler (TODO: discuss further how this can be of use).
If we have an element of this type (speak a witness / the type is inhibited) then we know the two elements are equal.

\cite{thompson_type_1991} discusses constructive vs. classic mathematics in chapter 3. In general there are two conflicting philosophical views of the foundations of mathematics: the constructive and the classic one. The constructive view has been identified with realism, empirical computational content where the classical one with idealism and pragmatic. TODO: work through chapter 3

dependent types as a perfect match and correspondence to the constructive nature of ABS, which is a 3rd way after induction and deduction

TODO: shortly discuss that dependent types are based on martin-löf intuitions type theory.

\subsection{Intensionality vs. Extensionality}
HOTT book, NOTES on chapter 1: "Extensional theory makes no distinction between judgmental and propositional equality, the intensional theory regards judgmental equality as purely definitional, and admits a much broader proof-relevant interpretation of the identity type that is central to the homotopy interpretation."

Propositional equality allows to assume that a variable x of type p is equal to y: p : x = y.

Judgemental equality (or definitional equality) means "equal by definition" e.g. if we have a function $f : N -> N by f(x) = x^2$ then f(3) is equal to $3^2$ by definition. Whether or not two expressions are equal by definition is just a matter of expanding out the definitions, in particula it is algorithmically decidable.

Fact: Idris, Agda and Coq are intensional


\section{Concepts of Dependent Types in Agent-Based Simulation}
\label{sec:dep_absconcepts}

dependent types: model- vs. agent-centric. model-centric means one looks at the model and its specifications as a whole and encodes them e.g. totality of SIR. agent-centric means one looks only at the agent level and encodes that as dependently typed as possible and hopes that model guarantees emerge: emergence on a metalevel - put otherwise: does the totality of SIR emerge when we follow an agent-centric approach?

If we can construct a dependently typed program of the SIR ABM which is total, then we have a proof-by-construction that the SIR model reaches a steady-state after finite time

dependent-types:
-> encode dynamics (what? feedbacks? positive/negative) on a meta-level
-> probabilistic types can encode probability distributions in types already about which we can then reason
-> agents as dependently typed continuations?: need a dependently typed concept of a process over time

\subsection{General Agent Interface}
using dependent types to specify the general commands available for an agent. here we can follow the approach of an DSEL as described in \cite{brady_correct-by-construction_2010} and write then an interpreter for it. It is of importance that the interpreter shall be pure itself and does not make use of any fancy IO stuff.

\subsection{Dependent State Machines}
dependent state machines in abs for internal state because that is very Common in ABS. Here we can draw inspiration from the paper \cite{brady_state_2016} and book \cite{brady_type-driven_2017}.

\subsection{Environment}
One of the main advantages of Agent-Based Simulation over other simulation methods e.g. System Dynamics is that agents can live within an environment. Many agent-based models place their agents within a 2D discrete NxM environment where agents either stay always on the same cell or can move freely within the environment where a cell has 0, 1 or many occupants. Ultimately this boils down to accessing a NxM matrix represented by arrays or a similar data structure. In imperative languages accessing memory always implies the danger of out-of-bounds exceptions \textit{at run-time}. With dependent types we can represent such a 2d environment using vectors which carry their length in the type (TODO: discuss them in background) thus fixing the dimensions of such a 2D discrete environment in the types. This means that there is no need to drag those bounds around explicitly as data. Also by using dependent types like Fin which depend on the dimensions we can enforce at compile time that we can only access the data structure within bounds. If we want to we can also enforce in the types that the environment will never be an empty one where N, M > 0.

\begin{HaskellCode}
Disc2dEnv : (w : Nat) -> (h : Nat) -> (e : Type) -> Type
Disc2dEnv w h e = Vect (S w) (Vect (S h) e)

data Disc2dCoords : (w : Nat) -> (h : Nat) -> Type where
  MkDisc2dCoords : Fin (S w) -> Fin (S h) -> Disc2dCoords w h
  
centreCoords : Disc2dEnv w h e -> Disc2dCoords w h
centreCoords {w} {h} _ =
    let x = halfNatToFin w
        y = halfNatToFin h
    in  mkDisc2dCoords x y
  where
    halfNatToFin : (x : Nat) -> Fin (S x)
    halfNatToFin x = 
      let xh   = divNatNZ x 2 SIsNotZ 
          mfin = natToFin xh (S x)
      in  fromMaybe FZ mfin
      
setCell :  Disc2dCoords w h
        -> (elem : e)
        -> Disc2dEnv w h e
        -> Disc2dEnv w h e
setCell (MkDisc2dCoords colIdx rowIdx) elem env 
    = updateAt colIdx (\col => updateAt rowIdx (const elem) col) env
 
getCell :  Disc2dCoords w h
        -> Disc2dEnv w h e
        -> e
getCell (MkDisc2dCoords colIdx rowIdx) env
    = index rowIdx (index colIdx env)
    
neumann : Vect 4 (Integer, Integer)
neumann = [         (0,  1), 
           (-1,  0),         (1,  0),
                    (0, -1)]

moore : Vect 8 (Integer, Integer)
moore = [(-1,  1), (0,  1), (1,  1),
         (-1,  0),          (1,  0),
         (-1, -1), (0, -1), (1, -1)]

-- TODO: can we express that n <= len?
filterNeighbourhood :  Disc2dCoords w h
                    -> Vect len (Integer, Integer)
                    -> Disc2dEnv w h e 
                    -> (n ** Vect n (Disc2dCoords w h, e))
filterNeighbourhood {w} {h} (MkDisc2dCoords x y) ns env =
    let xi = finToInteger x
        yi = finToInteger y
    in  filterNeighbourhood' xi yi ns env
  where
    filterNeighbourhood' :  (xi : Integer)
                         -> (yi : Integer)
                         -> Vect len (Integer, Integer)
                         -> Disc2dEnv w h e 
                         -> (n ** Vect n (Disc2dCoords w h, e))
    filterNeighbourhood' _ _ [] env = (0 ** [])
    filterNeighbourhood' xi yi ((xDelta, yDelta) :: cs) env 
      = let xd = xi - xDelta
            yd = yi - yDelta
            mx = integerToFin xd (S w)
            my = integerToFin yd (S h)
        in case mx of
            Nothing => filterNeighbourhood' xi yi cs env 
            Just x  => (case my of 
                        Nothing => filterNeighbourhood' xi yi cs env 
                        Just y  => let coord      = MkDisc2dCoords x y
                                       c          = getCell coord env
                                       (_ ** ret) = filterNeighbourhood' xi yi cs env
                                   in  (_ ** ((coord, c) :: ret)))
\end{HaskellCode}

\subsection{Dependent Agent Interactions}
\paragraph{Agent Transactions}
dependently typed message protocols in ABS because its very common, and easily done thorugh methods in OOP: sugarscape mating and trading protocol
using a DSEL \cite{brady_correct-by-construction_2010} to restrict the available primitives in the message protocol?

\paragraph{Data Flow}
TODO: can dependent types be used in the Data Flow Mechanism?
\paragraph{Event Scheduling}
TODO: can dependent types be used in the event-scheduling mechanism?

\paragraph{Flow Of Time}
TODO: can dependent types be used to express the flow of time and its strongly monotonic increasing?

\subsection{Totality}
totality of parts or the whole simulation e.g. in case of the SIR model we can informally reason that the simulation MUST reach an equilibrium (a steady state from which there is no escape: the dynamics wont't change anymore, derivations are 0) after a finite number of steps. if we can construct a total program which expresses this, we have a formal proof of that which is 1) a specification of the model 2) generates the dynamics 3) is a proof that it reaches equilibrium

\subsection{Constructive Proofs}
- An agent-based model and the simulated dynamics of it is itself a constructive proof which explain a real-world phenomenon sufficiently good
- proof of the existence of an agent: holds always only for the current time-step or for all time, depending on the model. e.g. in the SIR model no agents are removed from / added to the system thus a proof holds for all time. In sugarscape agents are removed / added dynamically so a proof might become invalid after a time or one can construct a proof only from a given time on e.g. when one wants to prove that agent X exists but agent X is only created at time t then before time t the prove cannot be constructed and is uninhabited and only inhabited from time t on.

\section{Dependently Typed SIR}
Intuitively, based upon our model and the equations we can argue that the SIR model enters a steady state as soon as there are no more infected agents. Thus we can informally argue that a SIR model must always terminate as:
\begin{enumerate}
	\item Only infected agents can infect susceptible agents.
	\item Eventually after a finite time every infected agent will recover.
	\item There is no way to move from the consuming \textit{recovered} state back into the \textit{infected} or \textit{susceptible} state \footnote{There exists an extended SIR model, called SIRS which adds a cycle to the state-machine by introducing a transition from recovered to susceptible but we don't consider that here.}.
\end{enumerate}

Thus a SIR model must enter a steady state after finite steps / in finite time. 

This result gives us the confidence, that the agent-based approach will terminate, given it is really a correct implementation of the SD model. Still this does not proof that the agent-based approach itself will terminate and so far no proof of the totality of it was given. Dependent Types and Idris ability for totality and termination checking should theoretically allow us to proof that an agent-based SIR implementation terminates after finite time: if an implementation of the agent-based SIR model in Idris is total it is a proof by construction. Note that such an implementation should not run for a limited virtual time but run unrestricted of the time and the simulation should terminate as soon as there are no more infected agents. We hypothesize that it should be possible due to the nature of the state transitions where there are no cycles and that all infected agents will eventually reach the recovered state. 
Abandoning the FRP approach and starting fresh, the question is how we implement a \textit{total} agent-based SIR model in Idris. Note that in the SIR model an agent is in the end just a state-machine thus the model consists of communicating / interacting state-machines. In the book \cite{brady_type-driven_2017} the author discusses using dependent types for implementing type-safe state-machines, so we investigate if and how we can apply this to our model. We face the following questions: how can we be total? can we even be total when drawing random-numbers? Also a fundamental question we need to solve then is how we represent time: can we get both the time-semantics of the FRP approach of Haskell AND the type-dependent expressivity or will there be a trade-off between the two?

-- TODO: express in the types
-- SUSCEPTIBLE: MAY become infected when making contact with another agent
-- INFECTED:    WILL recover after a finite number of time-steps
-- RECOVERED:   STAYS recovered all the time

-- SIMULATION:  advanced in steps, time represented as Nat, as real numbers are not constructive and we want to be total
--              terminates when there are no more INFECTED agents


show formally that abs does resemble the sd approach: need an idea of a proof and then implement it in dependent types: look at 3 agent system: 2 susceptible, 1 infected. or maybe 2 agents only

%A susceptible agent can only become infected when it comes into contact with an infected agent. The probability of a susceptible agent making contact with an infected one is naturally (number of infected agents) / (total number of agents). For the infection to occur we multiply the contact with the infectivity parameter \Gamma. A susceptible agent makes on average \Beta contacts per time-unit. This results in the following formula:
%
%\begin{align}
%prob &= \frac{I \beta \gamma}{N} \\
%\end{align}
%
%This is for a single agent, which we then need to multiply by the number of susceptible agents because all of them make contact.
%
%TODO: implement sir with state-machine approach from Idris. an idea would be to let infected agents generate infection- actions: the more infected agents the more infection-actions => zero infected agents mean zero infection actions. this list can then be reduced?
%
%can we also emulate SD in Idris and formulate positive/negative feedback loops in types?

\subsection{A constructive proof of totality}
The idea is to implement a total agent-based SIR simulation, where the termination does NOT depend on time (is not terminated after a finite number of time-steps, which would be trivial). The dynamics of the system-dynamics SIR model are in equilibrium (won't change anymore) when the infected stock is 0. This can (probably) be shown formally but intuitionistic it is clear because only infected agents can lead to infections of susceptible agents which then make the transition to recovered after having gone through the infection phase. Thus an agent-based implementation of the SIR simulation has to terminate if it is implemented correctly because all infected agents will recover after a finite number of steps after then the dynamics will be in equilibrium.
Thus we need to 'tell' the type-checker the following:
1) no more infected agents is the termination criterion
2) all infected agents will recover after a finite number of time => the simulation will eventually run out of infected agents But when we look at the SIR+S model we have the same termination criterion, but we cannot guarantee that it will run out of infected => we need additional criteria
3) infected agents are 'generated' by susceptible agents
4) susceptible agents are NOT INCREASING (e.g. recovered agents do NOT turn back into susceptibles)
Interesting: can we adopt our solution (if we find it), into a SIRS	implementation? this should then break totality. also how difficult is it?

The HOTT book states that lists, trees,... are inductive types/inductively defined structures where each of them is characterized by a corresponding "induction principle". For a proof of totality of SIR we need to find the "induction principle" of the SIR model and implement it. What is the inductive, defining structure of the SIR model? is it a tree where a path through the tree is one simulation dynamics? or is it something else? it seems that such a tree would grow and then shrink again e.g. infected agents. Can we then apply this further to (agent-based) simulation in general?

TODO: \url{https://stackoverflow.com/questions/19642921/assisting-agdas-termination-checker/39591118}


\chapter{Aims and Objectives}
\label{chap:aimsObj}

WARNING:
STM is possible in other languages as well!! rework this in the report. only haskell can guarantee specific things already statically at compile time





From the annual review the following things become clear:
- the aim is basically "Explore using Haskell for Agent-Based Simulation with its benefits and drawbacks".
- The 3 major benefits of the approach I claim
	1. code == spec
	2. can rule out serious class of bugs
	3. we can perform reasoning about the simulation in code
	need to be metricated: e.g. this is really only possible in Haskell and not in Java. This needs thorough thinking about which metrics are used, how they can be aquired, how they can be compared,...
- Why ACE and Social Simulation? Did i only pick these fields because they are easily applicable to the problems I want to solve? Which properties do they exhibit which make them interesting for my problem? Just to say that "Sugarscape and bilateral decentralized bartering is interesting and fascinates me" is not enough in a final viva/thesis/paper.
- Reasoning must be very clear. So far I have 2 ideas for formal reasoning in code:
	1. SIR
		-> my emulation of SD using ABS is really an implementation of the SD model and follows it - they are equivalent
		-> my ABS implementation is the same as / equivalent to the SD emulation
			=> thus if i can show that my SD emulation is equlas to the SD model
			=> AND that the ABS implementation is the same as the SD emulation
			=> THEN the ABS implementation is an SD implementation, and we have shown this in code for the first time in ABS

	2. Decentralized Bilateral Bartering
		-> can we reason about the equilibrium prices in an ABS setting? e.g. show formally why equilibrium prices are not reached, under which circumstance they are reached,...
			-> need to combine General Equilibrium Theory
			-> with Bilateral decentralized exchange
			-> with Agent-Based Simulation 
			
- STILL I NEED TO SHOW HOW I CAN MAKE HASKELL RELEVANT IN THE FIELD OF ABS
	-> as far as I know so far no reasoning has been done in the way I intend to do it in the field of ABS. My hypothesis is that it is really only possible in Haskell due to its explicit side-effects, type-system, declarative style,... 
		-> TODO: need to check if this is really unique to haskell
	-> the functional-reactive approach seems to bring a new view to ABS with an embedded language for explicit time-semantics. Together with parallel/sequential updating this allows implementing System-Dynamics and agents which rely on continuous time-semantics e.g. SIR-Agents. Maybe i invented a hybrid between SD and ABS? Also what about time-traveling? The problem is that this is not really clear as i hypothesize that is completely novel approach to ABS - again I need to check this!
		-> TODO: is this really unique to functional reactive? E.g. what about Repast, NetLogo, AnyLogic, other Java-Frameworks? 
	-> maybe i have to admit that its not as unique as thought

In General i need to show that
- Haskells general benefits \& drawbacks over other Languages in the Field of ABS (e.g. Java, NetLogo, Repast) e.g. declarative style, reasoning, explicit about side-effects, performance, difficult to reason about performance, space-leaks difficult. So this focuses on the general comparison between the established technologies of ABS and Haskell but not yet on Haskells suitability in comparison to these other technologies. Here we talk about reasoning, side-effects, performance IN GENERAL TERMS, NOT SPECIFIC TO ABS
- Haskells suitability to implement ABS in comparison to other languages and technologies in the Field of. Here the focus is on general problems in ABS and how they can and are solved using Haskell.
- Why using Haskell in ABS - do the general benefits / drawbacks apply equally well? Are there unique advantages? Can we do things in Haskell which are not possible in other technologies or just very hard? E.g. the hybrid-approach I created with FRP: how unique is it e.g. can other technologies easily implement it as well? Other potential advantages: recursive simulation. Here we DO NOT concentrate on general technicalities but see how they apply when using it for ABS and if they create a unique benefit for Haskell in ABS.

\section{Aims}
The aim of this Ph.D is to explore the benefits and drawbacks using Haskell in  Agent-Based Simulation. First a library for general-purpose ABS in Haskell is built which serves as the primary object to study the benefits and drawbacks. After having investigated the benefits and drawbacks the library will be used to research  verification and reasoning in ABS in the context of decentralized bilateral bartering as specified in the Sugarscape model.

\section{Objectives}
\begin{enumerate}
	\item Implement a library for general-purpose Agent-Based Simulation in Haskell 
	\item Objectively and scientifically compare the usage of Haskell in ABS to the usage of Java in ABS: what are the benefits/drawbacks of Haskell and what are the benefits/drawbacks of Java? Are they orthogonal to each other e.g. are the weaknesses of one language the other languages strength?
	\item Define scientific measures: e.g. Lines Of Code (show relation to Bugs \& Defects, which is an objective measure: http://www.stevemcconnell.com/est.htm, \url{https://softwareengineering.stackexchange.com/questions/185660/is-the-average-number-of-bugs-per-loc-the-same-for-different-programming-languag}, Book: Code Complete, \url{https://www.mayerdan.com/ruby/2012/11/11/bugs-per-line-of-code-ratio}), also experience reports by companies which show that Haskell has huge benefits when applied to the same domain of a previous implementation of a different language, post on stack overflow / research gate / reddit, read experience reports from \url{http://cufp.org/2015/}
	\item Develop reasoning-techniques using Functional Programming in ABS by comparing the implementations of the SD- and ABS-model of the SIR compartment model in epidemiology.
	\item Investigate the usage of STM for concurrent ABS
\end{enumerate}

\section{Research Questions}
\begin{enumerate}
	\item Which is the best / a valid / good working approach of implementing ABS in Haskell?
	\item What are the benefits of using Haskell in ABS?
	\item What are the drawbacks of using Haskell in ABS?
	\item Are there things which are unique when doing Haskell in ABS and cannot be done in a Java approach?
	\item Both the System-Dynamics and Agent-Based implementation of the SIR compartment model in epidemiology lead to the same dynamics or put different: the Agent-Based implementation shows the same dynamics of the SD implementation when using replications. This is shown by plotting the dynamics as graphs. Can we show that they are equivalent through reasoning about the code?
	\item In the Sugarscape model where Agents engage in bilateral decentralized bartering Equilibrium is only reached when neo-classical agents are used which don't die of natural age. The equilibrium is not reached when more realistic assumptions are made. This is shown by plotting the prices over time. Can we show that the equilibrium is reached / not reached when using neo-classical agents / realistic agents through reasoning in the code?
	\item What and to which extent can we reason about an Agent-Based Simulation using my implementation in Haskell?
	\item Can we run Agents concurrently in STM but still retain reproducibility of the simulation?
\end{enumerate}

\section{Hypotheses}
\begin{enumerate}
	\item Yampa is a valid / good approach of implementing ABS in Haskell.
	\item Haskell benefits (which are not possible in java) are: Code == Spec, can statically rule out a major and very relevant class of bugs, can perform reasoning and proofing of properties of the program
	\item Haskell drawbacks over java are: slower, potential for difficult to find space-leaks, much more difficult to reason about performance in general, steeper learning curve, think ABS different 
	\item reproducibility of a system: lack of unpredictable side-effects statically enforced in the type-system
	\item Unique to Haskell is that it enables STM.
\end{enumerate}

\chapter{Work To Date}
\label{chap:work_to_date}

From a very general perspective I am researching a novel implementation approach to ABS. The hypothesis is that due to its underlying foundations of pure functional programming, this approach leads to simulation software which is easier to verify and validate and thus more likely to be correct, less sources of bugs and is conceptually cleaner.
In the first half of my PhD (October 2016 - March 2018) I have learned the underlying foundations of pure functional programming, did lots of prototyping and ultimately developed a way of implementing ABS in this approach. This resulted in a paper, submitted in March 2018 to the Haskell Symposium 2018, which discusses \textit{how} to do agent-based simulation with pure functional programming as foundation and how to solve the fundamental problems of encapsulating agent-state, doing agent-interactions and bringing in environments in this setting.
We found out that we immediately benefit from this approach in various ways, supporting our initial hypothesis but didn't investigate it in scientific details, which we leave for the next 12 months, conducted between April 2018 and April 2019. Thus we will be researching the \textit{why} of our approach, which we claim is an easier and stronger approach to verification and validation (V \& V). We need to clarify the meaning of V \& V in both areas we trying to gap, pure functional programming and agent-based simulation, and how they related to each other and how we can connect them. Further we need to quantify our claims of less sources of bugs through other research and comparing it to imperative OO approaches.
In this time we will investigate the use of dependent types in our pure functional approach to agent-based simulation which we hypothesise should allow an unprecedented level of verification and validation, not possible (even not on a theoretical level) with imperative, traditional object-oriented approaches. There exists literally no research on this topic thus it will form the unique and sufficiently advanced, novel contribution of our PhD to the field. We will also write an additional paper which will investigate how dependent types can be made of use in ABS.
Around December 2018 I will start writing another paper which is targeted for an agent-based simulation journal and is written as a conceptual paper, describing the approach and benefits of purely and dependently typed agent-based simulation. While writing this paper I will start constructing the main argument structure of my thesis so I have structure already when I start writing the thesis in April 2019. The last 6 months of the PhD (April 2019 - September 2019) will be dedicated to writing up the thesis and conducting additional research if still necessary.

So roughly the PhD can be split into 3 phases:
Researching the HOW: September 2016 - March 2018
Researching the WHY: April 2018 - March 2019
Writing the Thesis: April 2019 - September 2019

2016
October - December
	proper learning haskell programming
	experiments with scala \& akka (actor model)
	
2017
January - March
	writing 1st paper: Art Of Iteration
	MGS2017
	deepening haskell programming skills

April - July
	getting into functional reactive programming
	literature research \& 1st year report and review
	prototyping concepts of purely functional ABS
	presentation to FP group at FP lunch

August
	holiday
	reading book and papers on functional programming

September
	preparation for social simulation conference 2017 (SSC2017)

October - December
	working on 2nd paper (pure functional epidemics): first draft
	generalising the functional reactive programming approach to monadic stream functions 

2018
January - February
	prototyping event-scheduling concepts in pure functional ABS
	completely reworking 2nd paper: 2nd draft
	learning Idris language (pure functional, dependently typed programming)

March
	finalising 2nd paper and submission to Haskell Symposium 2018
	deepening Idris knowledge and experience 

April
	bit of work on verification and validation of the purely functional SIR abs implementation using quickcheck. this is not original research but will then be useful for the final thesis as a small separate section
	started 3rd paper on dependent types in purely functional ABS

May
	feedback on 2nd paper on 18 May	
	researching concepts of dependent types in purely functional ABS
	research \& writing 3rd paper

June - July
	2nd year report \& review
	research \& writing 3rd paper


Here we give a concise overview over the activities performed in the 2nd year.

\section{Social Simulation Conference 2017}


\section{Paper Published}
TODO: Art of Iteration was published in SSC2017 proceedings

\section{Papers Submitted}
\subsection{Pure Functional Epidemics}
This paper, which is attached in Appendix \ref{app:pfe}

\section{Reports}
\subsection{Haskell Communities and Activities Report (HCAR) May 2017}
We wrote a new entry for the HCAR May 2017, which tries to compile and publish novel and on-going ideas in the Haskell community. It is freely available under \url{https://www.haskell.org/communities/05-2017/html/report.html}. We hope that our idea and the work of our PhD gets a bit more attention and may start some discussions with people interested in this work.

\subsection{2nd Year Report}
This document.


\section{Talks}
So far only two talks were given. The first one was a presentation of the ideas underlying the update-strategies paper at the IMA - seminar day. The second was presenting my ideas about functional reactive ABS to the FP-Lab Group at the FPLunch.



\section{Future Work Plan}
TODO: A future work plan that is consistent with the progress to date, stating clearly the research question(s) to be addressed during the next year of the PhD.

TODO: gantt chart!

\subsection{TODOs}
out of this i will build the gantt chart for the next 12 months+

\subsubsection{Category Theory}
develop category theory behind FrABS: look into monads, arrows

category theory foundations (monads, arrows)

\subsubsection{Implementation and Software-Engineering}
%how to implement in haskell (sugarscape, agent_zero) 
implement chapter 4 of sugarscape
implement chapter 5 of sugarscape
use monadic or arrowized programming for structuribg the siftware
implement schelling segregation in recursiveABS and report results

FrABS: SugarScape
1st prototype: pure-functional implemented, no category-theory/type-theory applied
2nd prototype: category-theory/type-theory applied: clean monadic / arrowized programming applied

Agent\_Zero
1st prototype: implemented the book, based upon FrABS 

\subsubsection{Verification and Validation}
look into QuickCheck to test and verificate FrABS. start with SIRS
(quickcheck, isabelle, agda?), recursive simulation

\subsubsection{Papers}
paper 2: recursive ABS
paper 3: FrABS - Towards pure functional programming in ABS
paper 4: Towards category theory in ABS
paper 5: verification and validation in ABS with pure functional programming


\subsubsection{Reading}
read "Writing For Computer Science"
read "Agent\_Zero"
read "category theory for the sciences"


\subsection{Concept of an Agent}
an agent is not an object but when implementing ABS in oo then it is tempting to treat an agent like that. when implementing it in a pure functional language like haskell, this temptation cannot arise which creates a different view on agents.



\chapter{Conclusions}
\label{ch:conclusions}

This chapter concludes the whole thesis and outlines future research. Roughly 20\% exists already.

%we now know how to engineer time- and event-driven ABS with complex state both in the agent and environment, main difficulty is direct agent-interaction (see macal classification into 4 types of ABS), compile-time guaranteed reproducibility, explicit handling of complex state (read only, read/write), concurrency explicit and limited to STM, very promising concurrency but direct agent-interactions main problem (erlang as a rescue?), main drawbacks: everything is explicit, performance

\section{Further Research}
clearly outline the ideas for further research

\subsection{A general purpose library}
generalise concepts explored into a pure functional ABS library in Haskell (called chimera)

\subsection{Dependent and linear types}
dependent types and linear types are the next big step, towards a stronger formalisation of agents and ABS,
focus on the equilibrium - totality correspondence

\subsection{Concurrent event-driven ABS}
stm based concurrency for event-driven ABS using parallel DES. challenge is the time-warp implementation using monads. in general it should be easy to roll-back agents actions but with monads we have to be careful - for some monads rolling back is not neccessary e.g. rand and reader, for others it is, and for some it is impossible e.g. IO

\renewcommand\bibname{References}

\bibliographystyle{acm}
\bibliography{../../../references/phdReferences}

\begin{appendices}

TODO: add full code of SIR implementation

\chapter{Validating Sugarscape in Haskell}
\label{app:validating_sugarscape}

In this chapter we look at how property-based testing can be made of use to verify the \textit{exploratory} Sugarscape model \cite{epstein_growing_1996} as introduced in Chapter \ref{sec:sugarscape}. Whereas in the chapters on testing the explanatory SIR model we had an analytical solution, the fundamental difference in the exploratory Sugarscape model is that none such analytical solutions exist. This raises the question, which properties we can actually test in such a model.

The answer lies in the very nature of exploratory models, they exist to explore and understand phenomena of the real world. Researchers come up with a model to explain the phenomena and (hopefully) with a few questions and \textit{hypotheses} about the emergent properties. The actual simulation is then used to test and refine the hypotheses. Indeed, descriptions, assumptions and hypotheses of varying formal degree abound in the Sugarscape model. Examples are: \textit{the carrying capacity becomes stable after 100 steps; when agents trade with each other, after 1000 steps the standard deviation of trading prices is less than 0.05; when there are cultures, after 2700 steps either one culture dominates the other or both are equally present}. 

We show how to use property-based testing to formalise and check such hypotheses. For this purpose we undertook a full \textit{verification} of our \href{https://github.com/thalerjonathan/haskell-sugarscape}{implementation}~\cite{thaler_sugarscape_repository} from Chapter \ref{sec:sugarscape}. We validated it against the book \cite{epstein_growing_1996} and a NetLogo implementation \cite{weaver_replicating_2009}  \footnote{Lending didn't properly work in their NetLogo code and that they didn't implement Combat.}.

\section{Property-based hypothesis testing}
The property we test for is whether \textit{the emergent property / hypothesis under test is stable under replicated runs} or not. To put it more technical, we use QuickCheck to run multiple replications with the same configuration but with different random-number streams and require that all tests pass. During the verification process we have derived and implemented property tests for the following hypotheses:

\begin{enumerate}
	\item Disease dynamics where all agents recover - when disease are turned on, if the number of initial diseases is 10, then the population is  able to rid itself completely from all disease within 100 ticks. 
	
	\item Disease dynamics where a minority recovers - when disease are turned on, if the number of initial diseases is 25, the population is not able to rid itself completely from all diseases within 1,000 ticks.
	
	\item Trading dynamics - when trading is enabled, the trading prices stabilise after 1,000 ticks with the standard deviation of the prices having dropped below 0.05.
	
	\item Cultural dynamics - when having two cultures, red and blue, after 2,700 ticks, either the red or the blue culture dominates or both are equally strong. If they dominate they make up 95\% of all agents, if they are equally strong they are both within 45\% - 55\%.
	
	\item Inheritance Gini coefficient - when agents reproduce and can die of age then inheritance of their wealth leads to an unequal wealth distribution measured using the Gini Coefficient \textit{averaging} at 0.7.
	
	\item Carrying capacity - when agents don't mate nor can die from age, due to the environment, there is an \textit{average} maximum carrying capacity of agents the environment can sustain. The capacity should be reached after 100 ticks and should be stable from then on.
		
	\item Terracing - when resources regrow immediately, after a few steps the simulation becomes static. Agents will stay on their terraces and will not move any more because they have found the best spot due to their behaviour. About 45\% will be on terraces and 95\% - 100\% are static, not moving any more.
\end{enumerate}

The hypotheses and their validation is described more in-depth in the section \ref{sec:hypotheses_testcases} below.

\subsection{Implementation}
To start with, we implement a custom data generator to produce output from a Sugarscape simulation. The generator takes the number of ticks and the scenario with which to run the simulation and returns a list of outputs, one for each tick.

\begin{HaskellCode}
sugarscapeUntil :: Int                -- ^ Number of ticks to run
                -> SugarScapeScenario -- ^ Scenario to run
                -> Gen [SimStepOut]   -- ^ Output of each step
sugarscapeUntil ticks params = do
  -- create a random-number generator
  g <- genStdGen
  -- initialise the simulation state with the given random-number generator
  -- and the scenario
  let simState = initSimulationRng g params
  -- run the simulation with the given state for number of ticks
  return (simulateUntil ticks simState)
\end{HaskellCode}

Using this generator, we can very conveniently produce Sugarscape data within a QuickCheck \texttt{Property}. Depending on the problem, we can generate only a single run or multiple replications, in case the hypothesis is assuming \textit{averages}. To see its use, we show the implementation of the \textit{Disease Dynamics (1)} hypothesis.

\begin{HaskellCode}
prop_disease_allrecover :: Property
prop_disease_allrecover = property (do
  -- after 100 ticks...
  let ticks = 100
  -- ... given Animation V-1 parameter configuration ...
  let params = mkParamsAnimationV_1
  -- ... from 1 sugarscape simulation ...
  aos <- last <*> (sugarscapeUntil ticks params)
  -- ... counting all infected agents ...
  let infected = length (filter (==False)) map (null . sugObsDiseases . snd) aos
  -- ... should result in all agents to be recovered
  return (cover 100 (infected == 0) "Diseases all recover" True)
\end{HaskellCode}

From the implementation it becomes clear, that this hypothesis states that the property has to hold \textit{for all} replications. The \textit{Inheritance Gini Coefficient (5)} hypothesis on the other hand assumes that the Gini Coefficient \textit{averages} at 0.7. We cannot average over replicated runs of the same property thus we generate multiple replications of the Sugarscape data within the property and employ a two-sided t-test with a 95\% confidence to test the hypothesis:

\begin{HaskellCode}
prop_gini :: Int      -- ^ Number of replications
          -> Double   -- ^ Confidence of the t-test
          -> Property
prop_gini repls confidence = property (do
  -- after 1000 ticks...
  let ticks = 1000
  -- ... the gini coefficient should average at 0.7 ...
  let expGini = 0.7
  -- ... given the Figure III-7 parameter configuration ...
  let params = mkParamsFigureIII_7
  -- ... from 100 replications ... 
  gini <- vectorOf repls (genGiniCoeff ticks params)
  -- on a two-tailed t-test with given confidence
  return (tTestSamples TwoTail expGini (1 - confidence) gini)
\end{HaskellCode}

%genGiniCoeff :: Int -> SugarScapeScenario -> Gen Double
%genGiniCoeff ticks params = do
%  -- generate sugarscape data
%  aos <- sugarscapeUntil ticks params
%  -- extract wealth of the agents in the last step
%  let agentWealths = map (sugObsSugLvl . snd) (last aos)
%  -- compute gini coefficient and return it
%  return (giniCoeff agentWealths)

\subsection{Running the tests}
As already pointed out in Part \ref{ch:property}, QuickCheck by default runs up to 100 test cases of a property and if all evaluate to \texttt{True} the property test succeeds. On the other hand, QuickCheck will stop at the first test case which evaluates to \texttt{False} and marks the whole property test as failed, no matter how many test cases got through already. For this reason we have used \texttt{cover} with an expected percentage of 100, meaning that we expect all tests to fall into the coverage class. This allows us to emulate failure with QuickCheck reporting the actual percentage of passed test cases.

Due to the duration even 1,000 ticks can take to compute, to get a first estimate of our hypotheses tests within reasonable time, we reduce the number of maximum successful replications required to 10 and when doing t-tests 10 replications are run there as well. 

\begin{verbatim}
SugarScape Tests
  Disease Dynamics All Recover:      OK (29.25s)
    +++ OK, passed 10 tests (100% Diseases all recover).
    
  Disease Dynamics Minority Recover: OK (536.00s)
    +++ OK, passed 10 tests (100% Diseases no recover).
    
  Trading Dynamics:                  OK (149.33s)
    +++ OK, passed 10 tests (70% Prices std less than 5.0e-2).
    Only 70% Prices std less than 5.0e-2, but expected 100%
    
  Cultural Dynamics:                 OK (996.84s)
    +++ OK, passed 10 tests (50% Cultures dominate or equal).
    Only 50% Cultures dominate or equal, but expected 100%
    
  Carrying Capacity:   OK (988.20s)
    +++ OK, passed 10 tests (90% Carrying capacity averages at 204.0).    
    Only 90% Carrying capacity averages at 204.0, but expected 100%
    
  Terracing:           OK (280.59s)
    +++ OK, passed 10 tests (80% Terracing is happening).
    Only 80% Terracing is happening, but expected 100%
    
  Inheritance Gini:    OK (7232.59s)
    +++ OK, passed 0 tests (0% Gini coefficient averages at 0.7).
    Only 0% Gini coefficient averages at 0.7, but expected 100%
\end{verbatim}

%\begin{enumerate}
%	\item Disease Dynamics all recover: \textit{+++ OK, passed 10 tests.}
%
%	\item Disease Dynamics minority recover: \textit{+++ OK, passed 10 tests.}
%		
%	\item Trading Dynamics: \textit{+++ OK, passed 10 tests; 2 failed (16\%).} (In total 12 tests (replications) were run, out of which 2 failed, which is a 16\% failure rate.)
%	
%	\item Cultural Dynamics: \textit{+++ OK, passed 10 tests; 3 failed (23\%).}
%
%	\item Inheritance Gini Coefficient: \textit{*** Failed! Passed only 0 tests; 10 failed (100\%) tests.}
%
%	\item Carrying Capacity: \textit{+++ OK, passed 10 tests; 2 failed (16\%).}
%
%	\item Terracing: \textit{+++ OK, passed 10 tests; 2 failed (16\%).}
%\end{enumerate}

How to deal with the failure of hypotheses is obviously highly model specific. A first approach is to increase the number of replications to run to 100 to get a more robust estimate of the failure rate. If the failure rate stays within reasonable ranges then one can arguably assume that the hypothesis is valid for sufficiently enough cases. On the other hand, if the failure rate escalates, then it is reasonable to deem the hypothesis invalid and refine it or even abandon it altogether.

With the exception of the Gini coefficient, we accept the failure rate of the hypotheses we presented here and deem them sufficiently valid for the task at hand. In case of the Gini coefficient, none of the replication was successful, which makes it obvious that it does \textit{not} average at 0.7. Thus the hypothesis as stated in the book does not hold and is invalid. One way to deal with it would be to simply delete it. Another, more constructive approach, is to keep it but require all replications to fail by marking it with \texttt{expectFailure} instead of \texttt{property}. In this way an invalid hypothesis is marked explicitly and acts as documentation and also as regression test.

\section{Hypotheses and test cases}
\label{sec:hypotheses_testcases}

In this section we briefly describe the process of validating our Sugarscape implementation against the specification of the Sugarscape book \cite{epstein_growing_1996} and the work of \cite{weaver_replicating_2009}.

\subsection{Terracing}
Our implementation reproduces the terracing phenomenon as described in the book and as can be seen in the NetLogo implementation as well. We implemented a property test in which we measure the closeness of agents to the ridge: counting the number of same-level sugars cells around them and if there is at least one lower then they are at the edge. If a certain percentage is at the edge then we accept terracing. The question is just how much, which we estimated from tests and resulted in 45\%. Also, in the terracing animation the agents actually never move which is because sugar immediately grows back thus there is no incentive for an agent to actually move after it has moved to the nearest largest cite in can see. Therefore we test that the coordinates of the agents after 50 steps are the same for the remaining steps.

\subsection{Carrying capacity}
Our simulation reached a steady state (variance $<$ 4 after 100 steps) with a mean around ~182. Epstein reported a carrying capacity of 224 (page 30) and the NetLogo implementations' \cite{weaver_replicating_2009} carrying capacity fluctuates around 205 which both are significantly higher than ours. Something was definitely wrong - the carrying capacity has to be around 200 (we trust in this case the NetLogo implementation and deem 224 an outlier).

After inspection of the NetLogo model we realised that we implicitly assumed that the metabolism range is \textit{continuously} uniformly randomized between 1 and 4 but this seemed not what the original authors intended: in the NetLogo model there were a few agents surviving on sugar level 1 which was never the case in ours as the probability of drawing a metabolism of exactly 1 is practically zero when drawing from a continuous range. We thus changed our implementation to draw a discrete value as the metabolism. %Note that this actually makes sense as massive floating-point number calculations were quite expensive in the mid 90s (e.g. computer games ran still on CPU only and exploited various  clever tricks to avoid the need of floating point calculations whenever possible) when SugarScape was implemented which might have been a reason for the authors to assume it implicitly.

This partly solved the problem, the carrying capacity was now around 204 which is much better than 182 but still a far cry from 210 or even 224. After adjusting the order in which agents apply the Sugarscape rules, by looking at the code of the NetLogo implementation, we arrived at a comparable carrying capacity of the NetLogo implementation: agents first make their move and harvest sugar and only after this the agents metabolism is applied (and ageing in subsequent experiments).

For regression tests we implemented a property test which tests that the carrying capacity of 100 simulation runs lies within a 95\% confidence interval of a 210 mean. These values are quite reasonable to assume, when looking at the NetLogo implementation - again we deem the reported carrying capacity of 224 in the book to be an outlier / part of other details we don't know.

One lesson learned is that even such seemingly minor things like continuous vs. discrete or order of actions an agent makes, can have substantial impact on the dynamics of a simulation.

\subsection{Wealth distribution}
By visual comparison we validated that the wealth distribution (page 32-37) becomes strongly skewed with a histogram showing a fat tail, power-law distribution where very few agents are very rich and most of the agents are quite poor. We compute the skewness and kurtosis of the distribution which is around a skewness of 1.5, clearly indicating a right skewed distribution and a kurtosis which is around 2.0 which clearly indicates the 1st histogram of Animation II-3 on page 34. Also we compute the Gini coefficient and it varies between 0.47 and 0.5 - this is accordance with Animation II-4 on page 38 which shows a gini-coefficient which stabilises around 0.5 after. 
We implemented a regression-test testing skewness, kurtosis and gini coefficients of 100 runs to be within a 95\% confidence interval of a two-sided t-test using an expected skewness of 1.5, kurtosis of 2.0 and gini coefficient of 0.48.

\subsection{Migration}
With the information provided by \cite{weaver_replicating_2009} we could replicate the waves as visible in the NetLogo implementation as well. Also we propose that a vision of 10 is not enough yet and shall be increased to 15 which makes the waves very prominent and keeps them up for much longer - agent waves are travelling back and forth between both Sugarscape peaks. We have not implemented a regression test for this property as we couldn't come up with a reasonable straightforward approach to implement it.

\subsection{Pollution and diffusion}
With the information provided by \cite{weaver_replicating_2009} we could replicate the pollution behaviour as visible in the NetLogo implementation as well. We have not implemented a regression test for this property as we couldn't come up with a reasonable straightforward approach to implement it.

%Note that we spent quite a lot of time of getting this and the terracing properties right because they form the very basics of the other ones which follow so we had to be sure that those were correct otherwise validating would have been much more difficult.

%\section{Order of Rules}
%order in which rules are applied is not specified and might have an impact on dynamics e.g. when does the agent mate with others: is it after it has harvested but before metabolism kicks in?

\subsection{Mating}
We could not replicate Figure III-1 - our dynamics first raised and then plunged to about 100 agents and go then on to recover and fluctuate around 300. This findings are in accordance with \cite{weaver_replicating_2009}, where they report similar findings - also when running their NetLogo code we find the dynamics to be qualitatively the same.

Also at first we weren't able to reproduce the cycles of population sizes. Then we realised that our agent behaviour was not correct: agents which died from age or metabolism could still engage in mating before actually dying - fixing this to the behaviour, that agents which died from age or metabolism will not engage in mating solved that and produces the same swings as in \cite{weaver_replicating_2009}. Although our bug might be obvious, the lack of specification of the order of the application of the rules is an issue in the SugarScape book.

\subsection{Inheritance}
We couldn't replicate the findings of the Sugarscape book regarding the Gini coefficient with inheritance. The authors report that they reach a gini coefficient of 0.7 and above in Animation III-4. Our Gini coefficient fluctuated around 0.35. Compared to the same configuration but without inheritance (Animation III-1) which reached a Gini coefficient of about 0.21, this is indeed a substantial increase - also with inheritance we reach a larger number of agents of around 1,000 as compared to around 300 without inheritance.
The Sugarscape book compares this to chapter II, Animation II-4 for which they report a Gini coefficient of around 0.5 which we could reproduce as well. The question remains, why it is lower (lower inequality) with inheritance?

The baseline is that this shows that inheritance indeed has an influence on the inequality in a population. Thus we deemed that our results are qualitatively the same as the make the same point. Still there must be some mechanisms going on behind the scenes which are unspecified in the original Sugarscape.

\subsection{Cultural dynamics}
We could replicate the cultural dynamics of AnimationIII-6 / Figure III-8: after 2700 steps either one culture (red / blue) dominates both hills or each hill is dominated by a different ulture. We wrote a test for it in which we run the simulation for 2.700 steps and then check if either culture dominates with a ratio of 95\% or if they are equal dominant with 45\%. Because always a few agents stay stationary on sugarlevel 1 (they have a metabolism of 1 and cant see far enough to move towards the hills, thus stay always on same spot because no improvement and grow back to 1 after 1 step), there are a few agents which never participate in the cultural process and thus no complete convergence can happen. This is accordance with \cite{weaver_replicating_2009}.

\subsection{Combat}
Unfortunately \cite{weaver_replicating_2009} didn't implement combat, so we couldn't compare it to their dynamics. Also, we weren't able to replicate the dynamics found in the Sugarscape book: the two tribes always formed a clear battlefront where some agents engage in combat, for example when one single agent strays too far from its tribe and comes into vision of the other tribe it will be killed almost always immediately. This is because crossing the sugar valley is costly: this agent wont harvest as much as the agents staying on their hill thus will be less wealthy and thus easier killed off. Also retaliation is not possible without any of its own tribe anywhere near.

We didn't see a single run where an agent of an opposite tribe "invaded" the other tribes hill and ran havoc killing off the entire tribe. We don't see how this can happen: the two tribes start in opposite corners and quickly occupy the respective sugar hills. So both tribes are acting on average the same and also because of the number of agents no single agent can gather extreme amounts of wealth - the wealth should rise in both tribes equally on average. Thus it is very unlikely that a super-wealthy agent emerges, which makes the transition to the other side and starts killing off agents at large. First: a super-wealthy agent is unlikely to emerge, second making the transition to the other side is costly and also low probability, third the other tribe is quite wealthy as well having harvested for the same time the sugar hill, thus it might be that the agent might kill a few but the closer it gets to the center of the tribe the less like is a kill due to retaliation avoidance - the agent will simply get killed by others.

Also it is unclear in case of AnimationIII-11 if the R rule also applies to agents which get killed in combat. Nothing in the book makes this clear and we left it untouched so that agents who only die from age (original R rule) are replaced. This will lead to a near extinction of the whole population quite quickly as agents kill each other off until 1 single agent is left which will never get killed in combat because there are no other agents who could kill it - instead it will enter an infinite die and  reborn cycle thanks to the R rule.

\subsection{Spice}
The book specifies for AnimationIV-1 a vision between 1-10 and a metabolism between 1-5. The last one seems to be quite strange because the maximum sugar / spice an agent can find is 4 which means that agents with metabolism of either 5 will die no matter what they do because the can never harvest enough to satisfy their metabolism. When running our implementation with this configuration the number of agents quickly drops from 400 to 105 and continues to slowly degrade below 90 after around 1000 steps.
The implementation of \cite{weaver_replicating_2009} used a slightly different configuration for AnimationIV-1, where they set vision to 1-6 and metabolism to 1-4. Their dynamics stabilise to 97 agents after around 500+ steps. When we use the same configuration as theirs, we produce the same dynamics.
Also it is worth nothing that our visual output is strikingly similar to both the book AnimationIV-1 and \cite{weaver_replicating_2009}.

\subsection{Trading}
For trading we had a look at the NetLogo implementation of \cite{weaver_replicating_2009}: there an agent engages in trading with its neighbours \textit{over multiple rounds} until either MRSs cross over or no trade has happened anymore. Because \cite{weaver_replicating_2009} were able to exactly replicate the dynamics of the trading time series we assume that their implementation is correct. We think that the fact that an agent interact with its neighbours over multiple rounds is made not very clear in the book. The only hint is found on page 102: \textit{"This process is repeated until no further gains from trades are possible."} which is not very clear and does not specify exactly what is going on: does the agent engage with all neighbours again? is the ordering random? Another hint is found on page 105 where trading is to be stopped after MRS crossover to prevent an infinite loop. Unfortunately this is missing in the Agent trade rule T on page 105. Additional information on this is found in footnote 23 on page 107. Further on page 107: \textit{"If exchange of the commodities will not cause the agents' MRSs to cross over then the transaction occurs, the agents recompute their MRSs, and bargaining begins anew."}. This is probably the clearest hint that trading could occur over multiple rounds.

We still managed to exactly replicate the trading dynamics as shown in the book in Figure IV-3, Figure IV-4 and Figure IV-5. The book is also pretty specific on the dynamics of the trading prices standard deviation: on page 109 the authors specify that at t=1000 the standard deviation will have always fallen below 0.05 (Figure IV-5), thus we implemented a property test which tests for exactly that property. Unfortunately we didn't reach the same magnitude of the trading volume where ours is much lower around 50 but it is equally erratic, so we attribute these differences to other missing specifications or different measurements because the price dynamics match that well already so we can safely assume that our trading implementation is correct.

According to the book, Carrying Capacity (Animation II-2) is increased by Trade (page 111/112). To check this it is important to compare it not against AnimationII-2 but a variation of the configuration for it where spice is enabled, otherwise the results are not comparable because carrying capacity changes substantially when spice is on the environment and trade turned off. We could replicate the findings of the book: the carrying capacity increases slightly when trading is turned on. Also does the average vision decrease and the average metabolism increase. This makes perfect sense: trading allows genetically weaker agents to survive which results in a slightly higher carrying capacity but shows a weaker genetic performance of the population.

According to the book, increasing the agent vision leads to a faster convergence towards the (near) equilibrium price (page 117/118/119, Figure IV-8 and Figure IV-9). We could replicate this behaviour as well.

According to the book, when enabling R rule and giving agents a finite life span between 60 and 100 this will lead to price dispersion: the trading prices will not converge around the equilibrium and the standard deviation will fluctuate wildly (page 120, Figure IV-10 and Figure IV-11). We could replicate this behaviour as well.

The Gini coefficient should be higher when trading is enabled (page 122, Figure IV-13) - We could replicate this behaviour.

Finite lives with sexual reproduction lead to prices which don't converge (page 123, Figure IV-14). We could reproduce this as well but it was important to set the parameters to reasonable values: increasing number of agents from 200 to 400, metabolism to 1-4 and vision to 1-6, most important the initial endowments back to 5-25 (both sugar and spice) otherwise hardly any mating would happen because the agents need too much wealth to engage (only fertile when have gathered more than initial endowment). What was kind of interesting is that in this scenario the trading volume of sugar is substantially higher than the spice volume - about 3 times as high. 

From this part, we didn't implement: Effect of Culturally Varying Preferences, page 124 - 126, Externalities and Price Disequilibrium: The effect of Pollution, page 126 - 118, On The Evolution of Foresight page 129 / 130. 

%\section{Lending (Credit)}
%Not really much information to validate was available and the \cite{weaver_replicating_2009} implementation ran into an exception so there was not much to validate against. What was unexpected was that this was the most complex behaviour to implement, with lots of subtle details to take care of (spice on/off, inheritance,...).
%Note that we implemented lending of sugar and spice, although it looks from the book (Animation IV-5) that they only implemented it for sugar.

\subsection{Diseases}
We were able to exactly replicate the behaviour of Animation V-1 and Animation V-2: in the first case the population rids itself of all diseases (maximum 10) which happens pretty quickly, in less than 100 ticks. In the second case the population fails to do so because of the much larger number of diseases (25) in circulation. We used the same parameters as in the book. 
The authors of \cite{weaver_replicating_2009} could only replicate the first animation exactly and the second was only deemed "good". Their implementation differs slightly from ours: In their case a disease can be passed to an agent who is immune to it - this is not possible in ours. In their case if an agent has already the disease, the transmitting agent selects a new disease, the other agent has not yet - this is not the case in our implementation and we think this is unreasonable to follow: it would require too much information and is also unrealistic.
We wrote regression tests which check for animation V-1 that after 100 ticks there are no more infected agents and for animation V-2 that after 1000 ticks there are still infected agents left and they dominate: there are more infected than recovered agents.

\section{Discussion}
In this appendix we showed how to use QuickCheck to formalise and check hypotheses about an \textit{exploratory} agent-based model, in which no ground truth exists. Due to ABS stochastic nature in general it became obvious that to get a good measure of a hypotheses validity we need to emulate failure using the \texttt{cover} function of QuickCheck. This allowed us to show that the hypotheses we have presented are sufficiently valid for the task at hand and can indeed be used for expressing and formalising emergent properties of the model and also as regression tests within a TDD cycle.

%What is particularly powerful is that one has complete control and insight over the changed state before and after e.g. a function was called on an agent: thus it is very easy to check if the function just tested has changed the agent-state itself or the environment: the new environment is returned after running the agent and can be checked for equality of the initial one - if the environments are not the same, one simply lets the test fail. This behaviour is very hard to emulate in OOP because one can not exclude side-effect at compile time, which means that some implicit data-change might slip away unnoticed. In FP we get this for free.

\end{appendices}

\end{document}
