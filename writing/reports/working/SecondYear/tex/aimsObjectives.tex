\chapter{Aim and Objectives}
\label{chap:aimsObj}

This chapter gives a compact and concise overview of the aims and objectives. Chapter \ref{chap:future} gives a more in-depth plan and details in how we will approach the aim in general and the objectives in particular.

\section{Aim}
The aim of this Ph.D. is to investigate how the pure functional programming paradigm without dependent types, as in Haskell, and with dependent types, as in Idris, can be used to implement Agent-Based Simulation and what the benefits are in doing so.

\section{Hypotheses}
\subsection{Hypothesis 1}
As shown partly in the Chapter \ref{chap:v_and_v} and the paper of Appendix \ref{app:pfe}, the benefits of functional programming in Haskell directly translate to our functional Agent-Based Simulation approach. We emphasise this in a hypothesis for the final thesis:

\paragraph{Using pure functional programming in Haskell to implement Agent-Based Simulations makes it harder to introduce bugs and leads to simulation software which is easier to verify and more likely to be correct.}

\subsection{Hypothesis 2}
As shown in Chapter \ref{chap:stm}, Software Transactional Memory is a highly promising approach to implement massively large-scale Agent-Based Simulations. We therefore formulate our findings into a hypothesis for the final thesis:

\paragraph{The concepts of Software Transactional Memory map naturally to functional Agent-Based Simulation and allow to implement massively large-scale simulations without the drawbacks of low level concurrent programming.}

\subsection{Hypothesis 3}
As shown in Chapter \ref{chap:dependent_types}, dependent types allow an unprecedented level of compile-time guarantees and we have outlined a number of conceptual ideas how to use them for functional Agent-Based Simulation. This is the main focus of research until the writing of the thesis, thus we formulate the following hypothesis:

\paragraph{Using pure functional programming with dependent types to implement Agent-Based Simulations allows to narrow the gap between model-specification and its implementation substantially up to a correct-by-construction level. By definition, this leads to a simulation which is even more likely to be correct.}

\section{Objectives}
\begin{enumerate}
	% NOTE: we have done that in our 2nd paper: Pure Functional Epidemics
	\item Implement the SIR and SugarScape model in pure functional Haskell. This will allow us to develop the fundamental concepts of pure functional programming in Agent-Based Simulation in general.

	% NOTE: we have done that in our 2nd paper: Pure Functional Epidemics
	\item Explore the benefits and drawbacks of using \textit{pure} functional programming (without dependent types) in Agent-Based Simulations.

	\item Explore the use of Software Transactional Memory for scaling up \textit{pure} functional Agent-Based Simulations.

	\item Implement a dependently typed agent-based simulation of the SIR and SugarScape model. This will allow us to develop the fundamental concepts of dependent types in Agent-Based Simulation in general.

	\item Explore how far we can narrow the gap between model-specification and implementation using dependent types for implementing a total Agent-Based Simulation of the SIR model.
\end{enumerate}