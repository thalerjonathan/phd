\chapter{Dependent Types}
\label{chap:dependent_types}
Dependent types are a very powerful addition to functional programming as they allow us to express even stronger guarantees about the correctness of programs \textit{already at compile-time}. They go as far as allowing to formulate programs and types as constructive proofs which must be \textit{total} by definition \cite{thompson_type_1991, mckinna_why_2006, altenkirch_pi_2010}. 

So far no research using dependent types in agent-based simulation exists at all. We have already started to explore this for the first time and ask more specifically how we can add dependent types to our functional approach, which conceptual implications this has for ABS and what we gain from doing so. We are using Idris \cite{brady_idris_2013} as the language of choice as it is very close to Haskell with focus on real-world application and running programs as opposed to other languages with dependent types e.g. Agda and Coq which serve primarily as proof assistants.

We hypothesise, that  dependent types will allow us to push the correctness of agent-based simulations to a new, unprecedented level by narrowing the gap between model specification and implementation. The investigation of dependent types in ABS will be the main unique contribution to knowledge of my Ph.D.

In the following section \ref{sec:dep_background}, we give an introduction of the concepts behind dependent types and what they can do. Further we give a very brief overview of the foundational and philosophical concepts behind dependent types. In Section \ref{sec:dep_absconcepts} we briefly discuss ideas of how the concepts of dependent types could be applied to agent-based simulation and in Section \ref{sec:dep_vav_deptypes} we very shortly discuss the connection between Verification \& Validation and dependent types.

\section{Background}
\label{sec:dep_background}

In this section we give an overview of the concepts behind dependent types and what they can do. Generally dependent types add the following concepts to existing pure functional programming:

\begin{enumerate}
	\item Types are first-class citizen - In dependently types languages, types can depend on any \textit{values}, and can be \textit{computed} at compile time which makes them first-class citizen.

	\item Totality and termination - A total function is defined in \cite{brady_type-driven_2017}: it terminates with a well-typed result or produces a non-empty finite prefix of a well-typed infinite result in finite time. Idris is turing-complete but is able to check the totality of a function under some circumstances but not in general as it would imply that it can solve the halting problem. Other dependently typed languages like Agda or Coq restrict recursion to ensure totality of all their functions - this makes them non turing-complete.

	\item Types as proofs - Because types can depend on any values and can be computed at compile time, they can be used as constructive proofs (see \ref{sub:dep_foundations}) which must terminate, this means a well-typed program (which is itself a proof) is always terminating which in turn means that it must consist out of total functions. Note that Idris does not restrict us to total functions but we can enforce it through compiler flags.
\end{enumerate}

\subsection{An example: Vector}
To give a concrete example of dependent types and their concepts, we introduce the canonical example used in all tutorials on dependent types: the Vector.

In Haskell (or in Java) there exists the List data-structure which holds a finite number of homogeneous elements, where the type of the elements can be fixed at compile-time. Using dependent types we can implement the same but adding the length of the list to the type - we call this data-structure a vector.

We define the vector as a Generalised Algebraic Data Type (GADT). A vector has a \textit{Nil} element which marks the end of a vector and a \textit{(::)} which is a recursive (inductive) definition of a linked List. We defined some vectors and we see that the length of the vector is directly encoded in its first type-variable of type Nat, natural numbers. Note that the compiler will refuse to accept \textit{testVectFail} because the type specifies that it holds 2 elements but the constructed vector only has 1 element.

\begin{HaskellCode}
data Vect : Nat -> Type -> Type where
	Nil  : Vect Z e
	(::) : (elem : e) -> (xs : Vect n e) -> Vect (S n) e
	
testVect : Vect 3 String
testVect = "Jonathan" :: "Andreas" :: "Thaler" :: Nil

testVectFail : Vect 2 Nat
testVectFail = 42 :: Nil
\end{HaskellCode}

We can now go on and implement a function \textit{append} which simply appends two vectors. Here we directly see \textit{type-level computations} as we compute the length of the resulting vector. Also this function is \textit{total}, as it covers all input cases and the recursion happens on a \textit{structurally smaller argument}:

\begin{HaskellCode}
append : Vect n e -> Vect m e -> Vect (n + m) e
append Nil ys = ys
append (x :: xs) ys = x :: append xs ys

append testVect testVect
["Jonathan", "Andreas", "Thaler", "Jonathan", "Andreas", "Thaler"] : Vect 8 String
\end{HaskellCode}

What if we want to implement a \textit{filter} function, which, depending on a given predicate, returns a new vector which holds only the elements for which the predicates returns true? How can we compute the length of the vector at compile-time? In short: we can't, but we can make us of \textit{dependent pairs} where the \textit{type} of the second element depends on the \textit{value} of the first (dependent pairs are also known as $\Sigma$ types, see \ref{sub:dep_foundations} below).

The function is total as well and works very similar to \textit{append} but uses dependent types as return, which are indicated by \textit{**}:

\begin{HaskellCode}
filter : Vect n e -> (e -> Bool) -> (k ** Vect k e)
filter [] f = (Z ** Nil)
filter (elem :: xs) f =
  case f elem of
    False => filter xs f
    True  => let (_ ** xs') = filter xs f
             in  (_ ** elem :: xs')
             
filter testVect (=="Jonathan")
(1 ** ["Jonathan"]) : (k : Nat ** Vect k String)
\end{HaskellCode}

It might seem that writing a \textit{reverse} function for a Vector is very easy, and we might give it a go by writing:
\begin{HaskellCode}
reverse : Vect n e -> Vect n e
reverse [] = []
reverse (elem :: xs) = append (reverse xs) [elem]
\end{HaskellCode}

Unfortunately the compiler complains because it cannot unify 'Vect (n + 1) e' and 'Vect (S n) e'. In the end, the compiler tells us that it cannot determine that (n + 1) is the same as (1 + n). The compiler does not know anything about the commutativity of addition which is due to how natural numbers and their addition are defined.

Lets take a detour. The natural numbers can be inductively defined by their initial element zero Z and the successor. The number 3 is then defined as the successor of successor of successor of zero:

\begin{HaskellCode}
data Nat = Z | S Nat

three : Nat 
three = S (S (S Z))
\end{HaskellCode}

Defining addition over the natural numbers is quite easy by pattern-matching over the first argument: 

\begin{HaskellCode}
plus : (n, m : Nat) -> Nat
plus Z right        = right
plus (S left) right = S (plus left right)
\end{HaskellCode}

Now we can see why the compiler cannot infer that (n + 1) is the same as (1 + n). The expression (n + 1) is translated to (plus n 1), where we pattern-match over the first argument, so we cannot reach a case in which (plus n 1) = S n. To do that we would need to define a different plus function which pattern-matches over the second argument - which is clearly the wrong way to go.

To solve this problem we can exploit the fact that dependent types allow us to perform type-level computations. This should allow us to express commutativity of addition over the natural numbers as a type. For that we define a function which takes in two natural numbers and returns a proof that addition commutes. 

\begin{HaskellCode}
plusCommutative : (left : Nat) -> (right : Nat) -> left + right = right + left
\end{HaskellCode}

We now begin to understand what it means when we speak of \textit{types as proofs}: we can actually express e.g. laws of the natural numbers in types and proof them by implementing a program which inhibits the type - we speak then of a constructive proof (see more on that below \ref{sub:dep_foundations}). Note that \textit{plusCommutative} is already implemented in Idris and we omit the actual implementation as it is beyond the scope of this introduction

Having our proof of commutativity of natural numbers, we can now implement a working (speak: correct) version of \textit{reverse}. The function \textit{rewrite} is provided by Idris: if we have a proof for x = y, the 'rewrite expr in' syntax will search for x in the required type of expr and replace it with y:

\begin{HaskellCode}
reverse : Vect n e -> Vect n e
reverse [] = []
reverse (elem :: xs) = append (reverse xs) [elem]
  where
    reverseProof : Vect (k + 1) a -> Vect (S k) a
    reverseProof {k} result = rewrite plusCommutative 1 k in result
\end{HaskellCode}

On of the most powerful aspects of dependent types is that they allow us to express equality on an unprecedented level. Non-dependently typed languages have only very basic ways of expressing the equality of two elements of same type. Either we use a boolean or another data-structure which can indicate equality or not. Idris supports this type of equality as well through \textit{(==) : Eq ty $\Rightarrow$ ty $\rightarrow$ ty $\rightarrow$ Bool}. The drawback of using a boolean is that in the end we don't have a real evidence of equality: even though the elements might be equal, the compiler has no means of inferring this and we can still make programming mistakes after the equality check because of this lack of compiler support.

This is different in dependent types which allow us to define \textit{decidable} equality through a type (see more on decidable / non-decidable equality below \ref{sub:dep_foundations}). Idris defines a decidable property as the following:

\begin{HaskellCode}
-- Decidability. A decidable property either holds or is a contradiction.
data Dec : Type -> Type where
  -- The case where the property holds
  -- @ prf the proof
  Yes : (prf : prop) -> Dec prop

  -- The case where the property holding would be a contradiction
  -- @ contra a demonstration that prop would be a contradiction
  No  : (contra : prop -> Void) -> Dec prop
\end{HaskellCode}

With that we can implement a function which constructs a proof that two natural numbers are equal, or not. We do this simply by pattern matching over both numbers with corresponding base cases and inductions. In case they are not equal we need to construct a proof that they are actually not equal which is done by showing that given some property results in a contradiction - indicated by the type \textit{Void}. In case of \textit{zeroNotSuc} the first number is zero (Z) whereas the other one is non-zero (a successor of some k), which can never be equal, thus we return a \textit{No} instance of the decidable property for which we need to provide the contradiction. In case of \textit{sucNotZero} its just the other way around. \textit{noRec} works very similar but here we are in the induction case which says that if k equals j leads to a contradiction, (k + 1) and (j + 1) can't be equal as well (induction hypothesis).

\begin{HaskellCode}
checkEqNat : (num1 : Nat) -> (num2 : Nat) -> Dec (num1 = num2)
checkEqNat Z Z         = Yes Refl
checkEqNat Z (S k)     = No zeroNotSuc
checkEqNat (S k) Z     = No sucNotZero
checkEqNat (S k) (S j) = case checkEqNat k j of
                              Yes prf   => Yes (cong prf)
                              No contra => No (noRec contra)
                              
zeroNotSuc : (0 = S k) -> Void
zeroNotSuc Refl impossible

sucNotZero : (S k = 0) -> Void
sucNotZero Refl impossible

noRec : (contra : (k = j) -> Void) -> (S k = S j) -> Void
noRec contra Refl = contra Refl
\end{HaskellCode}                              

The important thing to understand here is that our Dec property holds much more information than just a boolean flag which indicates whether Yes/No that two elements of a type are equal: in case of Yes we have a type which says that num1 is equal to num2, which can be directly used by the compiler, both elements are treated as the same.

\subsection{Foundations}
\label{sub:dep_foundations}


- dependently typed functions (pi types)
- dependent pairs (sigma types)
- decidable equality

\subsection{Constructivism}
TODO: ABS is constructive: "if you can't grow it, you can't explain it" (epstein)
TODO: Dependent Types are constructive
=> there are no excluded middle in both approaches
=> are there deeper, philosophical connections going on? does it have even deeper implications?
TODO: shortly discuss Propositions as types from HOTT 1.11. In the end a dependently typed ABS is then a constructive proof of WHAT? the model? if we have a total SIR implementation its a constructive proof that the agent-based implementation is total / will reach an equilibrium after a finite number of steps. Still it is not entirely clear WHAT WE ARE PROVING when we are constructing dependently typed agent-based simulations. I need to think about this more carefully
TODO: checkout my notes in 1st annual review on constructivism / popper 

Law of excluded middle does not hold anymore because it would require us to be able to effectively compute / decide whether a proposition is true or false - which amounts to solving the halting problem, which is not possible in the general case.

An important concept of this constructive approach is that the (proposition of) equality between two elements of the same type are is itself a type, called equality or identity types. This is much more expressive than a boolean proposition which evaluates to True in case they are the same and False if not as an equality type encodes much richer information which can be used by the type system. With the boolean approach, also known as boolean blindness, although one has compare two elements on equality and this check has returned true, the compiler has still no way of knowing \textit{after} the check that both elements are indeed the same - with equality types we can provide this information which can be used by the compiler (TODO: discuss further how this can be of use).
If we have an element of this type (speak a witness / the type is inhibited) then we know the two elements are equal.

\cite{thompson_type_1991} discusses constructive vs. classic mathematics in chapter 3. In general there are two conflicting philosophical views of the foundations of mathematics: the constructive and the classic one. The constructive view has been identified with realism, empirical computational content where the classical one with idealism and pragmatic. TODO: work through chapter 3

dependent types as a perfect match and correspondence to the constructive nature of ABS, which is a 3rd way after induction and deduction

TODO: shortly discuss that dependent types are based on martin-löf intuitions type theory.

\subsection{Intensionality vs. Extensionality}
HOTT book, NOTES on chapter 1: "Extensional theory makes no distinction between judgmental and propositional equality, the intensional theory regards judgmental equality as purely definitional, and admits a much broader proof-relevant interpretation of the identity type that is central to the homotopy interpretation."

Propositional equality allows to assume that a variable x of type p is equal to y: p : x = y.

Judgemental equality (or definitional equality) means "equal by definition" e.g. if we have a function $f : N -> N by f(x) = x^2$ then f(3) is equal to $3^2$ by definition. Whether or not two expressions are equal by definition is just a matter of expanding out the definitions, in particula it is algorithmically decidable.

Fact: Idris, Agda and Coq are intensional


\section{Concepts of Dependent Types in Agent-Based Simulation}
\label{sec:dep_absconcepts}

dependent types: model- vs. agent-centric. model-centric means one looks at the model and its specifications as a whole and encodes them e.g. totality of SIR. agent-centric means one looks only at the agent level and encodes that as dependently typed as possible and hopes that model guarantees emerge: emergence on a metalevel - put otherwise: does the totality of SIR emerge when we follow an agent-centric approach?

If we can construct a dependently typed program of the SIR ABM which is total, then we have a proof-by-construction that the SIR model reaches a steady-state after finite time

dependent-types:
-> encode dynamics (what? feedbacks? positive/negative) on a meta-level
-> probabilistic types can encode probability distributions in types already about which we can then reason
-> agents as dependently typed continuations?: need a dependently typed concept of a process over time

\subsection{General Agent Interface}
using dependent types to specify the general commands available for an agent. here we can follow the approach of an DSEL as described in \cite{brady_correct-by-construction_2010} and write then an interpreter for it. It is of importance that the interpreter shall be pure itself and does not make use of any fancy IO stuff.

\subsection{Dependent State Machines}
dependent state machines in abs for internal state because that is very Common in ABS. Here we can draw inspiration from the paper \cite{brady_state_2016} and book \cite{brady_type-driven_2017}.

\subsection{Environment}
One of the main advantages of Agent-Based Simulation over other simulation methods e.g. System Dynamics is that agents can live within an environment. Many agent-based models place their agents within a 2D discrete NxM environment where agents either stay always on the same cell or can move freely within the environment where a cell has 0, 1 or many occupants. Ultimately this boils down to accessing a NxM matrix represented by arrays or a similar data structure. In imperative languages accessing memory always implies the danger of out-of-bounds exceptions \textit{at run-time}. With dependent types we can represent such a 2d environment using vectors which carry their length in the type (TODO: discuss them in background) thus fixing the dimensions of such a 2D discrete environment in the types. This means that there is no need to drag those bounds around explicitly as data. Also by using dependent types like Fin which depend on the dimensions we can enforce at compile time that we can only access the data structure within bounds. If we want to we can also enforce in the types that the environment will never be an empty one where N, M > 0.

\begin{HaskellCode}
Disc2dEnv : (w : Nat) -> (h : Nat) -> (e : Type) -> Type
Disc2dEnv w h e = Vect (S w) (Vect (S h) e)

data Disc2dCoords : (w : Nat) -> (h : Nat) -> Type where
  MkDisc2dCoords : Fin (S w) -> Fin (S h) -> Disc2dCoords w h
  
centreCoords : Disc2dEnv w h e -> Disc2dCoords w h
centreCoords {w} {h} _ =
    let x = halfNatToFin w
        y = halfNatToFin h
    in  mkDisc2dCoords x y
  where
    halfNatToFin : (x : Nat) -> Fin (S x)
    halfNatToFin x = 
      let xh   = divNatNZ x 2 SIsNotZ 
          mfin = natToFin xh (S x)
      in  fromMaybe FZ mfin
      
setCell :  Disc2dCoords w h
        -> (elem : e)
        -> Disc2dEnv w h e
        -> Disc2dEnv w h e
setCell (MkDisc2dCoords colIdx rowIdx) elem env 
    = updateAt colIdx (\col => updateAt rowIdx (const elem) col) env
 
getCell :  Disc2dCoords w h
        -> Disc2dEnv w h e
        -> e
getCell (MkDisc2dCoords colIdx rowIdx) env
    = index rowIdx (index colIdx env)
    
neumann : Vect 4 (Integer, Integer)
neumann = [         (0,  1), 
           (-1,  0),         (1,  0),
                    (0, -1)]

moore : Vect 8 (Integer, Integer)
moore = [(-1,  1), (0,  1), (1,  1),
         (-1,  0),          (1,  0),
         (-1, -1), (0, -1), (1, -1)]

-- TODO: can we express that n <= len?
filterNeighbourhood :  Disc2dCoords w h
                    -> Vect len (Integer, Integer)
                    -> Disc2dEnv w h e 
                    -> (n ** Vect n (Disc2dCoords w h, e))
filterNeighbourhood {w} {h} (MkDisc2dCoords x y) ns env =
    let xi = finToInteger x
        yi = finToInteger y
    in  filterNeighbourhood' xi yi ns env
  where
    filterNeighbourhood' :  (xi : Integer)
                         -> (yi : Integer)
                         -> Vect len (Integer, Integer)
                         -> Disc2dEnv w h e 
                         -> (n ** Vect n (Disc2dCoords w h, e))
    filterNeighbourhood' _ _ [] env = (0 ** [])
    filterNeighbourhood' xi yi ((xDelta, yDelta) :: cs) env 
      = let xd = xi - xDelta
            yd = yi - yDelta
            mx = integerToFin xd (S w)
            my = integerToFin yd (S h)
        in case mx of
            Nothing => filterNeighbourhood' xi yi cs env 
            Just x  => (case my of 
                        Nothing => filterNeighbourhood' xi yi cs env 
                        Just y  => let coord      = MkDisc2dCoords x y
                                       c          = getCell coord env
                                       (_ ** ret) = filterNeighbourhood' xi yi cs env
                                   in  (_ ** ((coord, c) :: ret)))
\end{HaskellCode}

\subsection{Dependent Agent Interactions}
\paragraph{Agent Transactions}
dependently typed message protocols in ABS because its very common, and easily done thorugh methods in OOP: sugarscape mating and trading protocol
using a DSEL \cite{brady_correct-by-construction_2010} to restrict the available primitives in the message protocol?

\paragraph{Data Flow}
TODO: can dependent types be used in the Data Flow Mechanism?
\paragraph{Event Scheduling}
TODO: can dependent types be used in the event-scheduling mechanism?

\paragraph{Flow Of Time}
TODO: can dependent types be used to express the flow of time and its strongly monotonic increasing?

\subsection{Totality}
totality of parts or the whole simulation e.g. in case of the SIR model we can informally reason that the simulation MUST reach an equilibrium (a steady state from which there is no escape: the dynamics wont't change anymore, derivations are 0) after a finite number of steps. if we can construct a total program which expresses this, we have a formal proof of that which is 1) a specification of the model 2) generates the dynamics 3) is a proof that it reaches equilibrium

\subsection{Constructive Proofs}
- An agent-based model and the simulated dynamics of it is itself a constructive proof which explain a real-world phenomenon sufficiently good
- proof of the existence of an agent: holds always only for the current time-step or for all time, depending on the model. e.g. in the SIR model no agents are removed from / added to the system thus a proof holds for all time. In sugarscape agents are removed / added dynamically so a proof might become invalid after a time or one can construct a proof only from a given time on e.g. when one wants to prove that agent X exists but agent X is only created at time t then before time t the prove cannot be constructed and is uninhabited and only inhabited from time t on.

\section{Dependently Typed SIR}
Intuitively, based upon our model and the equations we can argue that the SIR model enters a steady state as soon as there are no more infected agents. Thus we can informally argue that a SIR model must always terminate as:
\begin{enumerate}
	\item Only infected agents can infect susceptible agents.
	\item Eventually after a finite time every infected agent will recover.
	\item There is no way to move from the consuming \textit{recovered} state back into the \textit{infected} or \textit{susceptible} state \footnote{There exists an extended SIR model, called SIRS which adds a cycle to the state-machine by introducing a transition from recovered to susceptible but we don't consider that here.}.
\end{enumerate}

Thus a SIR model must enter a steady state after finite steps / in finite time. 

This result gives us the confidence, that the agent-based approach will terminate, given it is really a correct implementation of the SD model. Still this does not proof that the agent-based approach itself will terminate and so far no proof of the totality of it was given. Dependent Types and Idris ability for totality and termination checking should theoretically allow us to proof that an agent-based SIR implementation terminates after finite time: if an implementation of the agent-based SIR model in Idris is total it is a proof by construction. Note that such an implementation should not run for a limited virtual time but run unrestricted of the time and the simulation should terminate as soon as there are no more infected agents. We hypothesize that it should be possible due to the nature of the state transitions where there are no cycles and that all infected agents will eventually reach the recovered state. 
Abandoning the FRP approach and starting fresh, the question is how we implement a \textit{total} agent-based SIR model in Idris. Note that in the SIR model an agent is in the end just a state-machine thus the model consists of communicating / interacting state-machines. In the book \cite{brady_type-driven_2017} the author discusses using dependent types for implementing type-safe state-machines, so we investigate if and how we can apply this to our model. We face the following questions: how can we be total? can we even be total when drawing random-numbers? Also a fundamental question we need to solve then is how we represent time: can we get both the time-semantics of the FRP approach of Haskell AND the type-dependent expressivity or will there be a trade-off between the two?

-- TODO: express in the types
-- SUSCEPTIBLE: MAY become infected when making contact with another agent
-- INFECTED:    WILL recover after a finite number of time-steps
-- RECOVERED:   STAYS recovered all the time

-- SIMULATION:  advanced in steps, time represented as Nat, as real numbers are not constructive and we want to be total
--              terminates when there are no more INFECTED agents


show formally that abs does resemble the sd approach: need an idea of a proof and then implement it in dependent types: look at 3 agent system: 2 susceptible, 1 infected. or maybe 2 agents only

%A susceptible agent can only become infected when it comes into contact with an infected agent. The probability of a susceptible agent making contact with an infected one is naturally (number of infected agents) / (total number of agents). For the infection to occur we multiply the contact with the infectivity parameter \Gamma. A susceptible agent makes on average \Beta contacts per time-unit. This results in the following formula:
%
%\begin{align}
%prob &= \frac{I \beta \gamma}{N} \\
%\end{align}
%
%This is for a single agent, which we then need to multiply by the number of susceptible agents because all of them make contact.
%
%TODO: implement sir with state-machine approach from Idris. an idea would be to let infected agents generate infection- actions: the more infected agents the more infection-actions => zero infected agents mean zero infection actions. this list can then be reduced?
%
%can we also emulate SD in Idris and formulate positive/negative feedback loops in types?

\subsection{A constructive proof of totality}
The idea is to implement a total agent-based SIR simulation, where the termination does NOT depend on time (is not terminated after a finite number of time-steps, which would be trivial). The dynamics of the system-dynamics SIR model are in equilibrium (won't change anymore) when the infected stock is 0. This can (probably) be shown formally but intuitionistic it is clear because only infected agents can lead to infections of susceptible agents which then make the transition to recovered after having gone through the infection phase. Thus an agent-based implementation of the SIR simulation has to terminate if it is implemented correctly because all infected agents will recover after a finite number of steps after then the dynamics will be in equilibrium.
Thus we need to 'tell' the type-checker the following:
1) no more infected agents is the termination criterion
2) all infected agents will recover after a finite number of time => the simulation will eventually run out of infected agents But when we look at the SIR+S model we have the same termination criterion, but we cannot guarantee that it will run out of infected => we need additional criteria
3) infected agents are 'generated' by susceptible agents
4) susceptible agents are NOT INCREASING (e.g. recovered agents do NOT turn back into susceptibles)
Interesting: can we adopt our solution (if we find it), into a SIRS	implementation? this should then break totality. also how difficult is it?

The HOTT book states that lists, trees,... are inductive types/inductively defined structures where each of them is characterized by a corresponding "induction principle". For a proof of totality of SIR we need to find the "induction principle" of the SIR model and implement it. What is the inductive, defining structure of the SIR model? is it a tree where a path through the tree is one simulation dynamics? or is it something else? it seems that such a tree would grow and then shrink again e.g. infected agents. Can we then apply this further to (agent-based) simulation in general?

TODO: \url{https://stackoverflow.com/questions/19642921/assisting-agdas-termination-checker/39591118}


\section{Verification, Validation and Dependent Types}
\label{sec:dep_vav_deptypes}
Dependent types allow to encode specifications on an unprecedented level, narrowing the gap between specification and implementation - ideally the code becomes the specification, making it correct-by-construction. The question is ultimately how far we can formulate model specifications in types - how far we can close the gap in the domain of ABS. Unless we cannot close that gap completely, to arrive at a sufficiently confidence in correctness, we still need to test all properties at run-time which we cannot encode at compile-time in types.

Nonetheless, dependent types should allow to substantially reduce the amount of testing which is of immense benefit when testing is costly. Especially in simulations, testing and validating a simulation can often take many hours - thus guaranteeing properties and correctness already at compile time can reduce that bottleneck substantially by reducing the number of test-runs to make.

Ultimately this leads to a very different development process than in the established object-oriented approaches, which follow a test-driven process. There one defines the necessary interface of an object with empty implementations for a given use-case first, then writes tests which cover all possible cases for the given use-case. Obviously all tests should fail because the functionality behind it was not implemented yet. Then one starts to implement the functionality behind it  step-by-step until no test-case fails. This means that one runs all tests repeatedly to both check if the test-case one is working on is not failing anymore and to make sure that old test-cases are not broken by new code. The resulting software is then trusted to be correct because no counter examples through test hypotheses, could be found. The problem is: we could forget / not think of cases, which is the easier the more complex the software becomes (and simulations are quite complex beasts). Thus in the end this is a deductive approach.

With pure functional programming and dependent types the process is now mostly constructive, type-driven (see \cite{brady_type-driven_2017}). In that approach one defines types first and is then guided by these types and the compiler in an interactive fashion towards a correct implementation, ensured at compile-time. As already noted, the ABS methodology is constructive in nature but the established object-oriented test-driven implementation approach not as much, creating an impedance mismatch. We expect that a type-driven approach using dependent types reduces that mismatch by a substantial amount.

Note that \textit{validation} is a different matter here: independent of our implementation approach we still need to validate the simulation against the real-world / ground-truth. This obviously requires to run the full simulation which could take up hours in either programming paradigm, making them absolutely equal in this respect. Also the comparison of the output to the real-world / ground-truth is completely independent to the paradigm. The fundamental difference happens in case of changes made to the code during validation: in case of the established test-driven object-oriented approach for every minor change one (should) re-run all tests, which could take up a substantial amount of additional time. Using a constructive, type-driven approach this is dramatically reduced and can often be completely omitted because the correctness of the change can be either guaranteed in the type or by informally reasoning about the code.

%todo: connection between black-box verification and dependent types
%todo: connection between white-box verification and dependent types