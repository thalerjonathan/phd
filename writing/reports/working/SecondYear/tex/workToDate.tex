\chapter{Work To Date}
\label{chap:work_to_date}

%From a very general perspective I am researching a novel implementation approach to ABS. The hypothesis is that due to its underlying foundations of pure functional programming, this approach leads to simulation software which is easier to verify and validate and thus more likely to be correct, less sources of bugs and is conceptually cleaner.

In the first half of my PhD (October 2016 - March 2018) I have learned the underlying foundations of pure functional programming, did lots of prototyping and ultimately developed a way of implementing ABS in this approach. This resulted in a paper (see Appendix \ref{app:pfe}) which discusses \textit{how} to do agent-based simulation with pure functional programming as foundation and how to solve the fundamental problems of encapsulating agent-state, doing agent-interactions and bringing in environments in this setting.

%2016
%October - December
%	proper learning haskell programming
%	experiments with scala \& akka (actor model)
%	
%2017
%January - March
%	writing 1st paper: Art Of Iteration
%	MGS2017
%	deepening haskell programming skills
%
%April - July
%	getting into functional reactive programming
%	literature research \& 1st year report and review
%	prototyping concepts of purely functional ABS
%	presentation to FP group at FP lunch
%
%August
%	holiday
%	reading book and papers on functional programming
%
%September
%	preparation for social simulation conference 2017 (SSC2017)
%
%October - December
%	working on 2nd paper (pure functional epidemics): first draft
%	generalising the functional reactive programming approach to monadic stream functions 
%
%2018
%January - February
%	prototyping event-scheduling concepts in pure functional ABS
%	completely reworking 2nd paper: 2nd draft
%	learning Idris language (pure functional, dependently typed programming)
%
%March
%	finalising 2nd paper and submission to Haskell Symposium 2018
%	deepening Idris knowledge and experience 
%
%April
%	bit of work on verification and validation of the purely functional SIR abs implementation using quickcheck. this is not original research but will then be useful for the final thesis as a small separate section
%	started 3rd paper on dependent types in purely functional ABS
%
%May
%	feedback on 2nd paper on 18 May	
%	researching concepts of dependent types in purely functional ABS
%	research \& writing 3rd paper
%
%June - July
%	2nd year report \& review
%	research \& writing 3rd paper

Here we give a concise overview over the activities performed in the 2nd year.

\section{Social Simulation Conference 2017}
I participated in the Social Simulation Conference 2017 (SSC2017) in Dublin from 24th - 29th September. I gave a 30 minutes talk on our submitted paper \cite{thaler_art_2017} which was very well received and discussed. The paper got selected to be published in the Conference Proceedings.

\section{Papers submitted: Pure Functional Epidemics}
The paper as attached in Appendix \ref{app:pfe}.

\section{Haskell Communities and Activities Report (HCAR) May 2018}
We wrote a new entry for the HCAR May 2018, which tries to compile and publish novel and on-going ideas in the Haskell community. It is freely available under \url{https://www.haskell.org/communities/05-2018/html/report.html}. We hope that our idea and the work of our PhD gets a bit more attention and may start some discussions with people interested in this work.

\section{2nd Year Report}
This document.

\section{Talks}
So far three talks were given:

\begin{enumerate}
	\item Presenting our paper \cite{thaler_art_2017} at the SSC2017 .
	\item Presentation of pure functional programming concepts in ABS to Master Students of my 1st Supervisor.
	\item Presenting the ideas on dependent types in ABS as outlined in section \ref{sec:dep_absconcepts} to the Functional Programming Lab at the FP Away Day 28/29th June 2018.
\end{enumerate} 

TODO: need to sell myself better here, list research activities
- worked on paper with peer
- stm research
- learning dependent types
- 