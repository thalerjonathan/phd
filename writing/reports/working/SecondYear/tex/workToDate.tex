\chapter{Work To Date}
\label{chap:work_to_date}

From a very general perspective I am researching a novel implementation approach to ABS. The hypothesis is that due to its underlying foundations of pure functional programming, this approach leads to simulation software which is easier to verify and validate and thus more likely to be correct, less sources of bugs and is conceptually cleaner.
In the first half of my PhD (October 2016 - March 2018) I have learned the underlying foundations of pure functional programming, did lots of prototyping and ultimately developed a way of implementing ABS in this approach. This resulted in a paper, submitted in March 2018 to the Haskell Symposium 2018, which discusses \textit{how} to do agent-based simulation with pure functional programming as foundation and how to solve the fundamental problems of encapsulating agent-state, doing agent-interactions and bringing in environments in this setting.
We found out that we immediately benefit from this approach in various ways, supporting our initial hypothesis but didn't investigate it in scientific details, which we leave for the next 12 months, conducted between April 2018 and April 2019. Thus we will be researching the \textit{why} of our approach, which we claim is an easier and stronger approach to verification and validation (V \& V). We need to clarify the meaning of V \& V in both areas we trying to gap, pure functional programming and agent-based simulation, and how they related to each other and how we can connect them. Further we need to quantify our claims of less sources of bugs through other research and comparing it to imperative OO approaches.
In this time we will investigate the use of dependent types in our pure functional approach to agent-based simulation which we hypothesise should allow an unprecedented level of verification and validation, not possible (even not on a theoretical level) with imperative, traditional object-oriented approaches. There exists literally no research on this topic thus it will form the unique and sufficiently advanced, novel contribution of our PhD to the field. We will also write an additional paper which will investigate how dependent types can be made of use in ABS.
Around December 2018 I will start writing another paper which is targeted for an agent-based simulation journal and is written as a conceptual paper, describing the approach and benefits of purely and dependently typed agent-based simulation. While writing this paper I will start constructing the main argument structure of my thesis so I have structure already when I start writing the thesis in April 2019. The last 6 months of the PhD (April 2019 - September 2019) will be dedicated to writing up the thesis and conducting additional research if still necessary.

So roughly the PhD can be split into 3 phases:
Researching the HOW: September 2016 - March 2018
Researching the WHY: April 2018 - March 2019
Writing the Thesis: April 2019 - September 2019

2016
October - December
	proper learning haskell programming
	experiments with scala \& akka (actor model)
	
2017
January - March
	writing 1st paper: Art Of Iteration
	MGS2017
	deepening haskell programming skills

April - July
	getting into functional reactive programming
	literature research \& 1st year report and review
	prototyping concepts of purely functional ABS
	presentation to FP group at FP lunch

August
	holiday
	reading book and papers on functional programming

September
	preparation for social simulation conference 2017 (SSC2017)

October - December
	working on 2nd paper (pure functional epidemics): first draft
	generalising the functional reactive programming approach to monadic stream functions 

2018
January - February
	prototyping event-scheduling concepts in pure functional ABS
	completely reworking 2nd paper: 2nd draft
	learning Idris language (pure functional, dependently typed programming)

March
	finalising 2nd paper and submission to Haskell Symposium 2018
	deepening Idris knowledge and experience 

April
	bit of work on verification and validation of the purely functional SIR abs implementation using quickcheck. this is not original research but will then be useful for the final thesis as a small separate section
	started 3rd paper on dependent types in purely functional ABS

May
	feedback on 2nd paper on 18 May	
	researching concepts of dependent types in purely functional ABS
	research \& writing 3rd paper

June - July
	2nd year report \& review
	research \& writing 3rd paper


Here we give a concise overview over the activities performed in the 2nd year.

\section{Social Simulation Conference 2017}


\section{Paper Published}
TODO: Art of Iteration was published in SSC2017 proceedings

\section{Papers Submitted}
\subsection{Pure Functional Epidemics}
This paper, which is attached in Appendix \ref{app:pfe}

\section{Reports}
\subsection{Haskell Communities and Activities Report (HCAR) May 2017}
We wrote a new entry for the HCAR May 2017, which tries to compile and publish novel and on-going ideas in the Haskell community. It is freely available under \url{https://www.haskell.org/communities/05-2017/html/report.html}. We hope that our idea and the work of our PhD gets a bit more attention and may start some discussions with people interested in this work.

\subsection{2nd Year Report}
This document.


\section{Talks}
So far only two talks were given. The first one was a presentation of the ideas underlying the update-strategies paper at the IMA - seminar day. The second was presenting my ideas about functional reactive ABS to the FP-Lab Group at the FPLunch.

