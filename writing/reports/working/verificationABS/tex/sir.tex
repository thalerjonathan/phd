\section{Dependently Typed SIR}
Intuitively, based upon our model and the equations we can argue that the SIR model enters a steady state as soon as there are no more infected agents. Thus we can informally argue that a SIR model must always terminate as:
\begin{enumerate}
	\item Only infected agents can infect susceptible agents.
	\item Eventually after a finite time every infected agent will recover.
	\item There is no way to move from the consuming \textit{recovered} state back into the \textit{infected} state \footnote{There exists an extended SIR model, called SIRS which adds a cycle to the state-machine by introducing a transition from recovered to susceptible but we don't consider that here.}.
\end{enumerate}

Thus a SIR model must enter a steady state after finite steps / in finite time. 

This result gives us the confidence, that the agent-based approach will terminate, given it is really a correct implementation of the SD model. Still this does not proof that the agent-based approach itself will terminate and so far no proof of the totality of it was given. Dependent Types and Idris ability for totality and termination checking should theoretically allow us to proof that an agent-based SIR implementation terminates after finite time: if an implementation of the agent-based SIR model in Idris is total it is a proof by construction. Note that such an implementation should not run for a limited virtual time but run unrestricted of the time and the simulation should terminate as soon as there are no more infected agents. We hypothesize that it should be possible due to the nature of the state transitions where there are no cycles and that all infected agents will eventually reach the recovered state. 
Abandoning the FRP approach and starting fresh, the question is how we implement a \textit{total} agent-based SIR model in Idris. Note that in the SIR model an agent is in the end just a state-machine thus the model consists of communicating / interacting state-machines. In the book \cite{brady_type-driven_2017} the author discusses using dependent types for implementing type-safe state-machines, so we investigate if and how we can apply this to our model. We face the following questions: how can we be total? can we even be total when drawing random-numbers? Also a fundamental question we need to solve then is how we represent time: can we get both the time-semantics of the FRP approach of Haskell AND the type-dependent expressivity or will there be a trade-off between the two?

-- TODO: express in the types
-- SUSCEPTIBLE: MAY become infected when making contact with another agent
-- INFECTED:    WILL recover after a finite number of time-steps
-- RECOVERED:   STAYS recovered all the time

-- SIMULATION:  advanced in steps, time represented as Nat, as real numbers are not constructive and we want to be total
--              terminates when there are no more INFECTED agents
