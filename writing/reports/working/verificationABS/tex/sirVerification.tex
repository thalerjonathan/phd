\section{Verification of SIR}
In this section we verify our agent-based implementation of the SIR model. Verification of our implementation should be fairly straight-forward and easy as the model is given in differential equations which gives us a formal specification of the model which we can use directly in our verification process. We will also conduct white-box verification and for one property it will be the only way of ensuring that our model is correct as we cannot guarantee it through black-box verification.

\subsection{Black Box Verification}
\subsubsection{Agent Behaviour}
When conducting black-box testing for the SIR model, we test if the \textit{isolated} behaviour of an agent in all three states Susceptible, Infected and Recovered, corresponds to model specifications. The crucial thing though is that we are dealing with a stochastic system where the agents act \textit{on averages}, which means we need to average our tests as well.

\paragraph{Susceptible Behaviour}
A susceptible agent \textit{may} become infected, depending on the number of infected agents in relation to non-infected the susceptible agent has contact to. To make this property testable we run a susceptible agent for 1.0 time-unit (note that we are sampling the system with small $\Delta t$ e.g. 0.1) and then check if it is infected - that is it returns infected as its current state. Obviously we need to pay attention to the fact that we are dealing with a stochastic system thus we can only talk about averages and thus it does not suffice to only run a single agent but we are repeating this for e.g. $N = 10.000$ agents (all with different RNGs). We then need a formula for the required fraction of the N agents which should have become infected on average. Per 1.0 time-unit, a susceptible agent makes \textit{on average} contact with $\beta$ other agents where in the case of a contact with an infected agent the susceptible agent becomes infected with a given probability $\gamma$. In this description there is another probability hidden, which is the probability of making contact with an infected agent which is simply the ratio of number of infected agents to number not infected agents. The formula for the target fraction of agents which become infected is then: $\beta * \gamma * \frac{number of infected}{number of non-infected}$. To check whether this test has passed we compare the required amount of agents which on average should become infected to the one from our tests (simply count the agents which got infected and divide by N) and if the value lies within some small $\epsilon$ then we accept the test as passed.

\paragraph{Inflected Behaviour}


\paragraph{Recovered Behaviour}
A recovered agent will stay in the recovered state \textit{forever}. Obviously we cannot write a black-box test that truly verifies that because it had to run forever. In this case we need to resort to White Box Verification (see below).

In our implementation we utilized the FRP paradigm. It seems that functional programming and FRP allow extremely easy testing of individual agent behaviour because FP and FRP compose extremely well which in turn means that there are no global dependencies as e.g. in OOP where we have to be very careful to clean up the system after each test - this is not an issue at all in our \textit{pure} approach to ABS.

\subsubsection{Simulation Dynamics}
We won't go into the details of comparing the dynamics of an ABS to an analytical solution, that has been done already by \cite{macal_agent-based_2010}. What is important is to note that population-size matters: different population-size results in slightly different dynamics in SD => need same population size in ABS (probably...?). Note that it is utterly difficult to compare the dynamics of an ABS to the one of a SD approach as ABS dynamics are stochastic which explore a much wider spectrum of dynamics e.g. it could be the case, that the infected agent recovers without having infected any other agent, which would lead to an extreme mismatch to the SD approach but is absolutely a valid dynamic in the case of an ABS. The question is then rather if and how far those two are \textit{really} comparable as it seems that the ABS is a more powerful system which presents many more paths through the dynamics.
TODO: i really want to solve this for the SIR approac
	-> confidence intervals?
	-> NMSE?
	-> does it even make sense?

\subsection{White Box Verification}
Defined as directly looking at the code and reasoning that the code does really what it has to implement.

- coverage testing