\documentclass[oneside]{book}

\setcounter{tocdepth}{1}
\setcounter{secnumdepth}{3}

\usepackage[toc,page]{appendix}
\usepackage{hyperref}
\usepackage[utf8]{inputenc}
\usepackage{graphicx} % Required for the inclusion of images
\usepackage{amsmath} % Required for some math elements 
\usepackage[utf8]{inputenc}
\usepackage[english]{babel}
\newtheorem{theorem}{Theorem}
\newtheorem{corollary}{Corollary}[theorem]
\newtheorem{lemma}[theorem]{Lemma}
\usepackage{listings}
\usepackage{pdfpages}
\usepackage{amssymb}
\usepackage{pdflscape}

\begin{document}

\begin{titlepage}
	\centering
	\includegraphics[width=0.60\textwidth]{../logo/UoN_Primary_Logo_RGB.png}\par\vspace{1cm}
	{\scshape\Large PhD Thesis\par}
	\vspace{1.5cm}
	{\huge\bfseries Foundations of Pure Functional Agent-Based Simulation\par}
	\vspace{2cm}
	{\Large\itshape Jonathan Thaler (4276122) \\ jonathan.thaler@nottingham.ac.uk \par}
	\vfill
	supervised by\par
	Dr. Peer-Olaf \textsc{Siebers} \\
	Dr. Thorsten \textsc{Altenkirch}

	\vfill

	{\large \today\par}
\end{titlepage}

\cleardoublepage

\section*{Abstract}
TODO

\clearpage
\tableofcontents
\clearpage

\section{Introduction}
first year report,
FpOOPAbs report,
snd year report
contributions

\chapter*{Part I: General Concepts}
FpOOPAbs report,
abs defined,
functional programming defined
effects and purity defined
short overview of impure functional programming in IO like avika: in the end the same as imperative programming, not what we want here
artiterating paper,
towards pure functional paper

\chapter{Part II: Pure functional ABS}
concepts of time- and event-driven approach in haskell (see robinson book and pidd book)
FrABS report,
pure functional epidemics paper
additional research on event-driven approach in haskell: unscheduling events in functional style easy: rollback to previous state is easy but memory costly. look into that in the thesis. clarify easy rolling back of system: can capture the whole state at a given point which allows reverting the system to a state in case of cancelling of an event
applicability of UML and peers Framework to pure functional ABS 
add section on recursive ABS
parallelise using cloud haskell?

agent-interactions
this is the central problem of the FP approach: basically the agent-interactions define the level of abstractions over the agents. unfortunately we have to say that this is easier and more elegant in OOP. Still by using a strong type system we can have advantages which the OOP doesnt have.
1) data-flow: for continuous ABS systems where data takes 1 dt to appear at the target agent and does not persist e.g. the target agent can check it or not but the data received in the current step will be gone in the next. Use-case: implementing continuous time-dependent ABS e.g. SIR model
2) events: an agent can send to an arbitrary other agent an event which happens at a given time in the future: when the event happens this means the target agent is executed with the information about the receiving event. Use-Case: implementing discrete event simulation ABS e.g. a bank, very useful when the model is specified in terms of events and not in a continuous fashion
3) transactions: method-call emulation, which takes no time at all and can involve an arbitrary but finite number of steps between agents. Use-case: trading between agents where they must to come to terms within the same time-step but where the negotiation process takes multiple steps between the agents.

Verification:
have written about it already in the report on SIR verification: incorporate QuickCheck. need to do a lot more work there. maybe can get it done in context of the 4th paper (towards pure functional ABS)

\chapter{Part III: Dependent types in ABS}
3rd paper

\chapter{Part IV: Verification}

\chapter{Conclusions}

\bibliographystyle{acm}
\bibliography{../../references/phdReferences}

\end{document}
